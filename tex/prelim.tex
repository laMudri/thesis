\chapter{Mathematical preliminaries}

In this section, I recall some standard categorical and algebraic notions that
are used in the rest of this thesis.
I present most of these notions in terms of \emph{multicategories}, whose main
purpose I see as giving universal properties to monoidal products --- for
example, vector spaces and multilinear maps form a multicategory, and in this
multicategory we can state the universal property of the tensor product of
vector spaces.
However, a reader already familiar with monoidal categories presented some other
way may wish to ignore the parts about multicategories and just skim the other
definitions to make sure they seem familiar.

Categories, (co)products, monoidal categories, string diagrams (in Rel),
monoid objects.
Examples: Set, Rel, partial functions, commutative monoids.

\section{Categories and multicategories}

\begin{definition}
  A \emph{category} is a collection $\obj$ of objects, a family
  $\hom : \obj \times \obj \to \Set$, for each object $A$ an element
  $\id : \hom(A,A)$, and for each triple of objects $A, B, C$ a function
  ${\circ} : \hom(B,C) \times \hom(A,B) \to \hom(A,C)$ such that the following
  laws hold for any $f : \hom(A,B), g : \hom(B,C), h : \hom(C,D)$.
  \begin{itemize}
    \item $\id \circ f = f$
    \item $f \circ \id = f$
    \item $(h \circ g) \circ f = h \circ (g \circ f)$
  \end{itemize}
  I sometimes write $f; g$ for $g \circ f$.
\end{definition}

I give a somewhat unusual definition of \emph{monoidal categories} via
\emph{multicategories}.
I use this approach to capture the purpose of monoidal categories in many
examples, particularly in algebra and programming languages: to provide a way to
encode morphisms of arbitrary arity as unary morphisms by representing $n$-ary
domains as internal $n$-ary products.

I choose to present multicategories in a style where composition models
simultaneous substitution rather than single substitution.
This choice is consistent with the rest of this thesis, which concerns
simultaneous substitution.

\begin{definition}
  A \emph{multicategory} is a collection $\obj$ of objects, a family
  $\hom : \List \obj \times \obj \to \Set$ (with $\hom([A_1, \ldots, A_n], B)$
  written $\hom(A_1, \ldots, A_n; B)$), for each object $A$ an element
  $\id : \hom(A; A)$, and for each list of lists $\Psi$, each list $\Gamma$, and
  each object $A$, a function
  ${\circ} : \hom(\Gamma; A) \times \plr{\prod_i \hom(\Psi_i; \Gamma_i)} \to
  \hom\plr{\sum_i \Psi_i; A}$ such that the following laws hold.
  \begin{itemize}
    \item $\id \circ f = f$
    \item $f \circ \plr{\lambda i.~\id} = f$
    \item $h \circ \plr{\lambda i.~g_i \circ \lambda j.~f_{ij}} =
      (h \circ \lambda i.~g_i) \circ \lambda ij.~f_{ij}$
  \end{itemize}
\end{definition}

\begin{example}
  The following give rise to multicategories:
  \begin{itemize}
    \item Sets and multi-ary functions (functions of the form
      $A_1 \times \cdots \times A_n \to B$)
    \item $K$-vector spaces and multilinear maps, for any field $K$
    \item For any simple type theory with sequents of the form
      $\Gamma \vdash A$, where $\Gamma$ is a list of types and $A$ is a type,
      types and open terms (up to $\beta\eta$-equivalence) form a multicategory.
  \end{itemize}
\end{example}

Related to the multicategory of $K$-vector spaces and multilinear maps is the
multicategory of commutative monoids and multilinear maps.
I use the latter later in this chapter\todo{forward reference}, so I will
develop it fully as an example.

\begin{example}
  The multicategory of \emph{commutative monoids and multilinear maps} has as
  objects commutative monoids.
  A morphism of type $A_1, \ldots, A_n \to B$ is a function
  $f : A_1 \times \cdots \times A_n \to B$ such that the following laws hold:
  \begin{itemize}
    \item $f(x_1, \ldots, 0_{A_i}, \ldots, x_n) = 0_B$
    \item $f(x_1, \ldots, y +_{A_i} z, \ldots, x_n) =
      f(x_1, \ldots, y, \ldots, x_n) +_B f(x_1, \ldots, z, \ldots, x_n)$
  \end{itemize}
  The identity function of type $A \to A$ serves as the identity
  (multi)morphism.
  Composition is given by multi-place composition of functions, which satisfies
  the above laws because in a composite $g \circ [f_1, \ldots, f_n]$, a
  commutative monoid operation in the domain of the composite bubbles through
  one of the $f_i$ to become one operation in position $i$ of the domain of $g$,
  and then bubbles through $g$.
  That this identity and composition follow the equational laws follows
  straightforwardly by considering the underlying functions.
\end{example}

\begin{definition}
  A multicategory $\C$ gives rise to its \emph{category of contexts} $\C^*$ as
  follows:
  \begin{itemize}
    \item $\obj(\C^*) \coloneqq \List(\obj\C)$.
    \item $\C^*\plr{\sum_i \Psi_i, \Delta} \coloneqq \prod_i \C(\Psi_i; \Delta_i)$
      for any $\Psi : \List(\List(\obj\C))$.
    \item $\id^* : \Gamma \to^* \Gamma \coloneqq \overline\id$.
    \item $(\sigma : \Delta \to^* \Theta) \circ^* (\rho : \Gamma \to^* \Delta) :
      (\Gamma \to^* \Theta) \coloneqq \lambda i.~\sigma_i \circ \rho|_{\sigma,i}$,
      where $\rho|_{\sigma,i}$ is the restriction of $\rho$ to those objects
      in the domain of $\sigma_i$.
  \end{itemize}
\end{definition}

\begin{definition}
  Each multicategory $\C$ \emph{restricts} to a category $\C^1$, where
  $\C^1(A, B) \coloneqq \C(A; B)$ and identity and composition are as in $\C$.
\end{definition}

\begin{definition}\label{def:tensor-product}
  A \emph{tensor product} in a multicategory $\C$ of objects $A$ and $B$ is an
  object $A \otimes B$ and a map ${\otimes} : A, B \to A \otimes B$ such that
  any morphism $f : \Gamma, A, B, \Delta \to C$ uniquely factors as:

  \begin{tikzcd}
    \Gamma, A, B, \Delta \arrow[d,"f"']
    \arrow[r,"{\overline\id, \otimes, \overline\id}"]
    & \Gamma, A \otimes B, \Delta
    \arrow[dl,dashed,"f^{\Gamma, A \otimes B, \Delta}"] \\
    C
  \end{tikzcd}

  When it is clear from context which two objects are being multiplied, I will
  write just $f^{\otimes}$ for the unique morphism.
\end{definition}

\begin{definition}\label{def:tensor-unit}
  A \emph{tensor unit} in a multicategory $\C$ is an object $I$ and a map
  $I : {\cdot} \to I$ such that any morphism
  $f : \Gamma, \Delta \to C$ uniquely factors as:

  \begin{tikzcd}
    \Gamma, \Delta \arrow[d,"f"']
    \arrow[r,"{\overline\id, I, \overline\id}"]
    & \Gamma, I, \Delta \arrow[dl,dashed,"f^{\Gamma, I, \Delta}"] \\
    C
  \end{tikzcd}

  When it is clear from context where the $I$ is being introduced, I will
  write just $f^I$ for the unique morphism.
\end{definition}

\begin{lemma}\label{thm:tensor-product-terms}
  For each morphism $f : \Gamma \to A$ and $g : \Delta \to B$, we have a
  morphism $(f, g) : \Gamma, \Delta \to A \otimes B$ such that we have, for any
  $h : \Gamma', A, B, \Delta' \to C$, that
  $h^{\otimes} \circ [\overline\id, (f, g), \overline\id] =
  h \circ [\overline\id, f, g, \overline\id]$, and that
  $\id = (\id, \id)^{\otimes} : A \otimes B \to A \otimes B$.
\end{lemma}
\begin{proof}
  The composite $(\otimes) \circ [f, g]$ has the required type.

  To prove the first equation, note that
  \begin{align*}
    h^{\otimes} \circ [\overline\id, (f, g), \overline\id]
    &= h^{\otimes} \circ [\overline\id, {\otimes} \circ [f, g], \overline\id] \\
    &= h^{\otimes} \circ [\overline\id, {\otimes}, \overline\id]
    \circ [\overline\id, f, g, \overline\id].
  \end{align*}
  Then, we have that
  $h^{\otimes} \circ [\overline\id, {\otimes}, \overline\id] = h$
  by the defining diagram of the tensor product (\cref{def:tensor-product}).

  To prove the second equation, note that the diagram below commutes.
  Because this is an instance of the diagram in \cref{def:tensor-product},
  uniqueness gives us the desired equation.

  \begin{tikzcd}
    A, B \arrow[d,"{\otimes}"'] \arrow[r,"{\otimes}"]
    & A \otimes B \arrow[dl,"\id"] \\
    A \otimes B
  \end{tikzcd}
\end{proof}

\begin{lemma}\label{thm:tensor-unit-terms}
  For any $h : \Gamma', \Delta' \to C$, we have that
  $h^I \circ [\overline\id, I, \overline\id] =
  h \circ [\overline\id, \overline\id]$
  and $\id = I^I : I \to I$.
\end{lemma}
\begin{proof}
  The first equation is just the commuting diagram of \cref{def:tensor-unit},
  modulo composition with identity morphisms.
  For the second equation, note that the diagram below commutes.
  Because this is an instance of the diagram in \cref{def:tensor-unit},
  uniqueness gives us the desired equation.

  \begin{tikzcd}
    {\cdot} \arrow[d,"I"'] \arrow[r,"I"]
    & I \arrow[dl,"\id"] \\
    I
  \end{tikzcd}
\end{proof}

\begin{lemma}
  In a multicategory $\C$ with all tensor products, $\otimes$ induces a
  bifunctor ${\otimes} : \C, \C \to \C$.
\end{lemma}
\begin{proof}
  The action on objects is given by $\otimes$.
  The action on morphisms $f : A \to A'$ and $g : B \to B'$ is given by
  $(f, g)^{\otimes}$.
  Preservation of identity is already proved in \cref{thm:tensor-product-terms},
  and preservation of composition is left as an exercise.
\end{proof}

\begin{lemma}
  For any tensor product and tensor unit, we have the following natural
  isomorphisms:
  \begin{itemize}
    \item $I \otimes A \cong A$
    \item $A \otimes I \cong A$
    \item $(A \otimes B) \otimes C \cong A \otimes (B \otimes C)$
  \end{itemize}
\end{lemma}

\begin{definition}\label{def:monoidal-category}
  A \emph{monoidal category} is a multicategory equipped with a tensor product
  and tensor unit.
  %A \emph{monoidal category} is a category $\C = (\obj, \hom, \id, \circ)$ with
  %functors $I : 1 \to \C$ and ${\otimes} : \C \times \C \to \C$ together with
  %the following natural isomorphisms:
  %\begin{itemize}
  %  \item $\lambda : (I \otimes {-}) \simeq {-}$
  %  \item $\rho : ({-} \otimes I) \simeq {-}$
  %  \item $\alpha :
  %    (({-} \otimes {-}) \otimes {-}) \simeq ({-} \otimes ({-} \otimes {-}))$
  %\end{itemize}
  %such that the following coherence conditions hold:
  %\begin{itemize}
  %  \item
  %    \begin{tikzcd}
  %      (A \otimes I) \otimes B \arrow[dr,"\rho_A \otimes B"'] \arrow[rr,"\alpha_{A,I,B}"]
  %      && A \otimes (I \otimes B) \arrow[dl,"A \otimes \lambda_B"] \\
  %      & A \otimes B
  %    \end{tikzcd}
  %  \item
  %    \begin{tikzcd}
  %      & (A \otimes B) \otimes (C \otimes D)
  %      \arrow[dr,"\alpha_{A,B,C \otimes D}"] &
  %      \\
  %      ((A \otimes B) \otimes C) \otimes D
  %      \arrow[ur,"\alpha_{A \otimes B,C,D}"]
  %      \arrow[d,"\alpha_{A,B,C} \otimes D"]
  %      &&
  %      A \otimes (B \otimes (C \otimes D))
  %      \\
  %      (A \otimes (B \otimes C)) \otimes D
  %      \arrow[rr,"\alpha_{A,B \otimes C,D}"]
  %      &&
  %      A \otimes ((B \otimes C) \otimes D)
  %      \arrow[u,"A \otimes \alpha_{B,C,D}"]
  %    \end{tikzcd}
  %\end{itemize}
\end{definition}

\begin{remark}
  We can talk about monoidal products on a category $\C$ by finding a
  multicategory which restricts to $\C$ and constructing tensor products in it.
\end{remark}

\begin{definition}
  A \emph{monoid} in a multicategory is an object $M$ and multimorphisms
  $\eta : {\cdot} \to M$ and $\mu : M, M \to M$ such that the following
  equations hold:
  \begin{itemize}
    \item $[\eta, \id]; \mu = \id : M \to M$
    \item $[\id, \eta]; \mu = \id : M \to M$
    \item $[\mu, \id]; \mu = [\id, \mu]; \mu : M, M, M \to M$
  \end{itemize}
\end{definition}

\begin{definition}
  A \emph{symmetric multicategory} is a multicategory $\C$ together with, for
  each permutation $\sigma : \Gamma \tilde\to \Delta$, an action
  $\sigma(-) : \C(\Delta; A) \to \C(\Gamma; A)$ respecting identity and
  composition of permutations.
\end{definition}

\begin{remark}
  It is tempting to define symmetric multicategories by replacing the lists of
  objects in the definition of multicategories by finite multisets.
  However, doing so entirely forgets the order of inputs to an operation, for
  example forcing every morphism with a type of the form $A, A \to B$ to be
  commutative.
  There is a way to refine this idea to get a correct definition, at the cost of
  requiring more sophisticated higher categorical notions.
\end{remark}

\begin{definition}
  A \emph{symmetric monoidal category} is a representable symmetric
  multicategory, analogously to how I defined a monoidal category.
\end{definition}

\begin{definition}
  A \emph{commutative monoid} in a symmetric multicategory is a monoid in the
  underlying multicategory such that $\mathrm{swap}(\mu) = \mu$, where
  $\mathrm{swap} : M, M \tilde\to M, M$ is the permutation swapping two
  elements.
\end{definition}

\section{Algebra}

\begin{definition}
  A \emph{semiring} is a monoid in the multicategory of commutative monoids and
  multilinear maps.
  Unpacked, this means that we have a set $\mathscr R$ together with elements
  $0$ and $1$ and binary operators $+$ and $\cdot$ (with $\cdot$ usually written
  as juxtaposition) such that the following hold for all
  $x, y, z \in \mathscr R$.
  \begin{itemize}
    \item $0 + x = x$; $x + 0 = x$; $(x + y) + z = x + (y + z)$; $x + y = y + x$
    \item $1x = x$; $x1 = x$; $(xy)z = x(yz)$
    \item $0x = 0$; $x0 = 0$; $(x + y)z = xz + yz$; $x(y + z) = xy + xz$
  \end{itemize}
\end{definition}
