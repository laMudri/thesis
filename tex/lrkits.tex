In an effort to reuse the syntactic kits and traversals approach of
\cref{sec:syntactic-kits}, I will derive the types of simultaneous renaming
and simultaneous substitution from a generic type of \emph{environments}.
To get a type of environments suitable for the usage-aware setting, I first
analyse intuitionistic environments (as introduced in \cref{sec:syntactic-kits}
definition \AgdaFunction{Env}), distilling the easy-to-use functional definition
(\cref{def:simple-env}) into a more basic recursive definition
(\cref{def:simple-rec-env}).
This recursive definition is easy to make usage-aware (\cref{def:lr-rec-env}),
which gives a basis from which to derive the function-based definition I will
take as primary (\cref{def:lr-env}).
The resulting definition makes explicit the role of algebraic linearity in the
metatheory of semiring-annotated calculi.

Recalling from \cref{sec:kits}, we have the following definition of
environments for simple types.

\begin{definition}[Simple environment]\label{def:simple-env}
  For $\V : \mathrm{Ctx} \to \mathrm{Ty} \to \mathrm{Set}$,
  a $\V$-\emph{environment} between simply typed contexts $\Gamma$ and $\Delta$
  is a function, polymorphic in type $A$, from variables of type $A$ in
  $\Delta$ to inhabitants of $\V\,\Gamma\,A$.
  We write the type of such environments as $\Gamma \env\V \Delta$.
\end{definition}

This definition is inadequate for \name{}.
For example, suppose we have a term
$\plr{M{_\otimes}N} : \grR\gamma \vdash A \otimes B$ and a substitution
$\sigma : \grR\gamma \env\vdash \grR\gr'\delta$.
From the $\otimes$-I rule, we have $M : \grP\gamma \vdash A$ and
$N : \grQ\gamma \vdash B$ for some $\grP$ and $\grQ$ such that
$\grR \leq \grP + \grQ$.
We want to apply $\sigma$ to the subterms $M$ and $N$, but this is impossible
because their contexts are not $\grR\gamma$, and we have no way to adapt
$\sigma$ to these new contexts.
Another instructive failure is the general non-existence of identity
environments, like a renaming of type $\gr1A, \gr1B \env\sqni \gr1A, \gr1B$.
We do not have a variable of type $\gr1A, \gr1B \sqni A$ or, symmetrically,
$\gr1A, \gr1B \sqni B$, because, in each case, there is one variable with
annotation $\gr1$ which we have not actually used.
This example suggests that the values of a usage-aware environment should be
derived in \emph{different} usage contexts, such as in $\gr1A, \gr0B \sqni A$.

To see why this definition of environment works for simply typed
$\lambda$-calculus but not \name{}, let us look at an equivalent definition by
recursion on the target context.
This recursive definition (\cref{def:simple-rec-env}), and particularly the
case where $\Delta$ is a concatenation, makes it clear how $\Gamma$ is being
copied for use in each $\V$-value.
I take the equivalence of \cref{def:simple-env} and \cref{def:simple-rec-env}
as obvious, because any function from variables in $\Delta$ can be
defunctionalised as a data structure with the same shape as $\Delta$.

\begin{definition}[Simple recursive environment]\label{def:simple-rec-env}
  A \emph{recursive $\V$-environment} between simply typed contexts $\Gamma$ and
  $\Delta$ is defined by cases on the shape of $\Delta$ (where
  $\Gamma \env\V_R \Delta$ is the notation for the type of recursive
  environments for given $\V$, $\Gamma$, and $\Delta$):
  \begin{itemize}
    \item There is an environment $\alr{} : \Gamma \env\V_R {\cdot}$.
    \item For $\rho_l : \Gamma \env\V_R \Delta_l$ and
      $\rho_r : \Gamma \env\V_R \Delta_r$, we have an environment
      $\alr{\rho_l, \rho_r} : \Gamma \env\V_R \Delta_l, \Delta_r$.
    \item For any value $v : \V\,\Gamma\,A$, we have an environment
      $\alr{v} : \Gamma \env\V_R A$.
  \end{itemize}
\end{definition}

I picture the sharing of $\Gamma$ in \cref{def:simple-rec-env} in the diagram
below.
The converging arrows from $\Gamma$ to each $\Delta_i$ represent the indices of
values appearing in a simple environment.

\begin{displaymath}
  \begin{tikzpicture}[baseline]
    \path
    (-1,1) node (Gtop) {}
    (-1,0) node (G) {$\Gamma$}
    (-1,-1) node (Gbot) {}
    ;
    \node[draw,dotted,fit=(Gtop) (G) (Gbot)] (GG) {};

    \path
    (1,1) node (Dtop) {}
    (1,0) node (D) {$\Delta$}
    (1,-1) node (Dbot) {}
    ;
    \node[draw,dotted,fit=(Dtop) (D) (Dbot)] (DD) {};

    \draw[->,double] (GG) -- (DD);
  \end{tikzpicture}
  \coloneqq
  \begin{tikzpicture}[baseline]
    \path
    (-1,1) node (Gtop) {}
    (-1,0) node (G) {$\Gamma$}
    (-1,-1) node (Gbot) {}
    ;
    \node[draw,dotted,fit=(Gtop) (G) (Gbot)] (GG) {};

    \path
    (1,1) node[draw] (Dtop) {$\Delta_1$}
    (1,0) node (D) {$\vdots$}
    (1,-1) node[draw] (Dbot) {$\Delta_n$}
    ;

    \fill[green!20!white,opacity=1] (GG.north east)
    parabola[bend at end] (Dtop.west)
    parabola[bend at start] (GG.south east)
    -- cycle;
    \fill[blue!40!white,opacity=.5] (GG.north east)
    parabola[bend at end] (Dbot.west)
    parabola[bend at start] (GG.south east)
    -- cycle;

    \draw[->] (GG.north east) parabola[bend at end] (Dtop.west);
    \draw (GG.south east) parabola[bend at end] (Dtop.west);
    \draw[->] (GG.north east) parabola[bend at end] (D.west);
    \draw (GG.south east) parabola[bend at end] (D.west);
    \draw[->] (GG.north east) parabola[bend at end] (Dbot.west);
    \draw (GG.south east) parabola[bend at end] (Dbot.west);
  \end{tikzpicture}
\end{displaymath}

To account for usage, we must replace the simple repetition of $\Gamma$ by
repetition of just the types $\gamma$ and \emph{redistribution} of the usage
annotations $\grP$.
Fortunately, our three basic ways of sharing up usage vectors --- zero,
addition, and scaling --- apply directly to the three possible shapes of the
target context --- empty, concatenation, and a usage-annotated singleton.

\begin{definition}[Usage-annotated recursive environment]\label{def:lr-rec-env}
  A \emph{recursive $\V$-environment} between annotated contexts $\Gamma$ and
  $\Delta$ is defined by cases on the shape of $\Delta$ (where
  $\Gamma \env\V_R \Delta$ is the notation for the
  type of recursive environments for given $\V$, $\Gamma$, and $\Delta$):
  \begin{itemize}
    \item There is one environment $\alr{} : \grP\gamma \env\V_R {\cdot}$
      whenever $\grP \leq \gr0$.
    \item For $\rho_l : \grPl\gamma \env\V_R \Delta_l$ and
      $\rho_r : \grPr\gamma \env\V_R \Delta_r$, we have an environment
      $\alr{\rho_l, \rho_r} : \grP\gamma \env\V_R \Delta_l, \Delta_r$ whenever
      $\grP \leq \grPl + \grPr$.
    \item For any value $v : \V\,\grPprime\gamma\,A$, we have an environment
      $\alr{v} : \grP\gamma \env\V_R \gr rA$ whenever
      $\grP \leq \gr r\grPprime$.
  \end{itemize}
\end{definition}

\begin{example}
  Take $\Ann = \plr{\mathbb N, =, 0, +, 1, \times}$, with the equality order
  chosen to avoid any concerns around subsumption of annotations.
  Then, there is an intuitionistic recursive environment as follows, where
  $y\,z$ is the application of $y$ to $z$.
  \[
    \alr{\alr{z},\alr{y\,z}} :
    \plr{x : A, y : B \to C, z : B} \env\vdash_R \plr{B, C}.
  \]
  There is also a usage-aware recursive environment
  \[
    \alr{\alr{z},\alr{y\,z}} :
    \plr{\gr0x : A, \gr2y : B \multimap C, \gr3z : B} \env\vdash_R
    \plr{\gr1B, \gr2C}.
  \]
  The latter relies on the observations that
  $\begin{pmatrix} \gr0 & \gr2 & \gr3 \end{pmatrix} =
  \begin{pmatrix} \gr0 & \gr0 & \gr1 \end{pmatrix}
  + \begin{pmatrix} \gr0 & \gr2 & \gr2 \end{pmatrix}$ and, on the right, that
  $\begin{pmatrix} \gr0 & \gr2 & \gr2 \end{pmatrix} =
  \gr2\begin{pmatrix} \gr0 & \gr1 & \gr1 \end{pmatrix}$.
  Then, we have $\gr0x : A, \gr0y : B \multimap C, \gr1z : B \vdash z : B$ and
  $\gr0x : A, \gr1y : B \multimap C, \gr1z : B \vdash y\,z : C$.
\end{example}

From the example, we can see that the important usage vectors are the initial
one $\begin{pmatrix} \gr0 & \gr2 & \gr3 \end{pmatrix}$ and the usage vectors
at which terms are derived: $\begin{pmatrix} \gr0 & \gr0 & \gr1 \end{pmatrix}$
and $\begin{pmatrix} \gr0 & \gr1 & \gr1 \end{pmatrix}$.
I will call the latter the \emph{leaf vectors}.
The intermediate vector $\begin{pmatrix} \gr0 & \gr2 & \gr2 \end{pmatrix}$ can
be worked out from the leaf vector
$\begin{pmatrix} \gr0 & \gr1 & \gr1 \end{pmatrix}$ and the scaling factor
$\gr2$ found in the codomain context $\gr1B, \gr2C$.
Even when the ordering on annotations is given by a non-equivalence relation
$\leq$, there is a canonical least choice for all of the intermediate vectors,
together with a constraint that the entire linear combination of all the leaf
vectors is less than or equal to the initial usage vector.
In symbols, we may let $\gr\Psi$ be the collection of leaf vectors indexed by
items in $\Delta$, and state
the constraint as $\grP \leq \sum_{\plr{x : \gr rA} \in \Delta} \gr r\gr\Psi_x$.
Seeing $\gr\Psi$ instead as a $\size\Delta \times \size\Gamma$ matrix, this
constraint is $\grP \leq \grQ\gr\Psi$, using vector-matrix multiplication.
The resulting picture is below, showing $\grP$ being split up into $\gr\Psi$,
and then each $\V$-value being constructed in a separate $\gr\Psi_i\gamma$.

\begin{displaymath}
  \begin{tikzpicture}[baseline]
    \path
    (-1,1) node (Gtop) {}
    (-1,0) node (G) {$\grP\gamma$}
    (-1,-1) node (Gbot) {}
    ;
    \node[draw,dotted,fit=(Gtop) (G) (Gbot)] (GG) {};

    \path
    (1,1) node (Dtop) {}
    (1,0) node (D) {$\grQ\delta$}
    (1,-1) node (Dbot) {}
    ;
    \node[draw,dotted,fit=(Dtop) (D) (Dbot)] (DD) {};

    \draw[->,double] (GG) -- (DD);
  \end{tikzpicture}
  \coloneqq
  \begin{tikzpicture}[baseline]
    \path
    (-1,1) node (Gtop) {}
    (-1,0) node (G) {$\grP\gamma$}
    (-1,-1) node (Gbot) {}
    ;
    \node[draw,dotted,fit=(Gtop) (G) (Gbot)] (GG) {};

    \path
    (1,3) node (G1top) {}
    (1,2) node (G1) {$\gr\Psi_1\gamma$}
    (1,1) node (G1bot) {}
    ;
    \node[draw,dotted,fit=(G1top) (G1) (G1bot)] (GG1) {};
    \draw[->] (GG) -- (GG1);

    \path (1,0) node {$\vdots$};

    \path
    (1,-1) node (Gntop) {}
    (1,-2) node (Gn) {$\gr\Psi_n\gamma$}
    (1,-3) node (Gnbot) {}
    ;
    \node[draw,dotted,fit=(Gntop) (Gn) (Gnbot)] (GGn) {};
    \draw[->] (GG) -- (GGn);

    \path
    (3,1) node[draw] (Dtop) {$\delta_1$}
    (3,0) node (D) {$\vdots$}
    (3,-1) node[draw] (Dbot) {$\delta_n$}
    ;

    \fill[green!20!white] (GG1.north east)
    parabola[bend at end] (Dtop.west)
    parabola[bend at start] (GG1.south east)
    -- cycle;
    \draw[->] (GG1.north east) parabola[bend at end] (Dtop.west);
    \draw (GG1.south east) parabola[bend at end] (Dtop.west);

    \fill[blue!20!white] (GGn.north east)
    parabola[bend at end] (Dbot.west)
    parabola[bend at start] (GGn.south east)
    -- cycle;
    \draw[->] (GGn.north east) parabola[bend at end] (Dbot.west);
    \draw (GGn.south east) parabola[bend at end] (Dbot.west);
  \end{tikzpicture}
  \quad\textrm{where }\grP \leq \grQ\gr\Psi
\end{displaymath}

From this point, we can recover a functional-style definition of usage-aware
environments.
We choose our leaf vectors $\gr\Psi$ up-front, check the inequality, and then
produce a value at each leaf vector.
%From this definition, we can recover a functional-style definition by
%separating choices of usage vectors from the provision of $\V$-values.
%In particular, the only choices of usage vectors that are essential are the
%$\grPprime$s in the singleton case, with the choices in the concatenation case
%being determined as scalings and sums of these $\grPprime$s.
%I let $\gr\Psi$ collect up these $\size\Delta$-many choices of
%$\size\Gamma$-length usage vectors and note that the constraint on $\gr\Psi$
%generated by all the scaling and summing is
%$\grP = \sum_{\plr{x : \gr rA} \in \Delta} \gr r\gr\Psi_x$.

\begin{definition}[Usage-annotated environment (tentative)]
  A \emph{$\V$-environment} between annotated contexts $\Gamma$ and $\Delta$
  (written $\grP\gamma$ and $\grQ\delta$, respectively, when convenient)
  is a matrix $\gr\Psi : \Ann^{\size\Delta \times \size\Gamma}$ such that
  $\grP \leq \grQ\gr\Psi$ and for each
  $\plr{x : A} \in \delta$ we have a value of type $\V\,\gr\Psi_x\gamma\,A$.
\end{definition}

I find this definition somewhat fiddly because of its reliance on low-level
concepts like non-usage-checked variables and rows of a matrix.
We note that $\gr\Psi_x = \bra x\gr\Psi$, from which point, requiring not
just $\V\,\gr\Psi_x\gamma\,A$ but rather $\V\,\plr{\grQprime\gr\Psi}\gamma\,A$
for any $\grQprime \leq \bra x$ is a minor change (and equivalent if $\V$
respects subusaging, which is practically always the case).
``An $x$ such that $(x : A) \in \delta$ and $\grQprime \leq \bra x\gr\Psi$''
is exactly the definition of $\grQprime\delta \sqni A$.
I further regularise this clause by asking for a
$\grPprime \leq \grQprime\gr\Psi$ rather than $\grQprime\gr\Psi$ exactly,
leaving us needing, for each $\grPprime$ and $\grQprime$ related in the same
way ($\gr\Psi$) as $\grP$ and $\grQ$, a function from $\grQprime\delta \sqni A$
to $\V\,\grPprime\gamma\,A$.
Finally, I choose to switch from matrices and matrix multiplication to
linear maps and their actions, which are easier to work with.
All of these changes yield my primary definition of an environment for
usage-annotated calculi, which will be used for the rest of this chapter and in
\cref{sec:framework}.

\begin{definition}[Usage-annotated environment]\label{def:lr-env}
  A \emph{$\V$-environment} between annotated contexts $\Gamma$ and $\Delta$
  (written $\grP\gamma$ and $\grQ\delta$, respectively, when convenient)
  is a linear map $\gr\Psi : \Ann^{\size\Delta} \to \Ann^{\size\Gamma}$ (written
  postfix) such that $\grP \leq \grQ\gr\Psi$ and for each $A$, $\grPprime$, and
  $\grQprime$ such that $\grPprime \leq \grQprime\gr\Psi$, a function from
  $\grQprime\delta \sqni A$ to $\V\,\grPprime\gamma\,A$.

  When there are multiple environments in question and $\rho$ is such an
  environment, I use the notation $\rho.\gr\Psi$ to refer to $\gr\Psi$.
  For example, $\grP \leq \grQ\plr{\rho.\gr\Psi}$.
  For the action on variables, I write $\rho(x)$, where
  $x : \grQprime\delta \sqni A$.
  The expression ``$\rho(x)$'' alone is ambiguous because of the slack in the
  usage context $\grPprime$ of the resulting value.
  Therefore, I will always make sure $\grPprime$ and $\grQprime$ clear when
  using this notation.
\end{definition}

The following simple lemma shows that usage-annotated environments are, in a
sense, as good as simple environments on usage-checked variables.
What usage-annotated environments give us beyond simple environments is the
ability to accommodate linear decompositions, in a way I will make precise in
the next section.

\begin{lemma}
  We can use an environment $\rho : \Gamma \env\V \Delta$ to map a
  usage-checked variable $x : \Delta \sqni A$ to a value of type
  $\V\,\Gamma\,A$.
\end{lemma}
\begin{proof}
  Let $\Gamma = \grP\gamma$ and $\Delta = \grQ\delta$.
  Set $\grPprime \coloneqq \grP$ and $\grQprime \coloneqq \grQ$, then
  $\grP \leq \grQ\gr\Psi$ by the constraint in $\rho$, so we can take
  the $\V$-value $\rho(x)$.
\end{proof}
