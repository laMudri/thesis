As we discussed in \autoref{sec:kits}, simultaneous substitution gives a
notion of derivability of one context from another, while simultaneous renaming
gives a similar notion of derivability restricted to structural rules.
To adapt these notions from an intuitionistic setting to our substructural
setting, we must examine what it means to derive one context from another
substructually.

\begin{displaymath}
  \begin{tikzpicture}[baseline]
    \path
    (-1,1) node (Gtop) {}
    (-1,0) node (G) {$\Gamma$}
    (-1,-1) node (Gbot) {}
    ;
    \node[draw,dotted,fit=(Gtop) (G) (Gbot)] (GG) {};

    \path
    (1,1) node (Dtop) {}
    (1,0) node (D) {$\Delta$}
    (1,-1) node (Dbot) {}
    ;
    \node[draw,dotted,fit=(Dtop) (D) (Dbot)] (DD) {};

    \draw[->,double] (GG) -- (DD);
  \end{tikzpicture}
  \coloneqq
  \begin{tikzpicture}[baseline]
    \path
    (-1,1) node (Gtop) {}
    (-1,0) node (G) {$\Gamma$}
    (-1,-1) node (Gbot) {}
    ;
    \node[draw,dotted,fit=(Gtop) (G) (Gbot)] (GG) {};

    \path
    (1,1) node[draw] (Dtop) {$\Delta_1$}
    (1,0) node (D) {$\vdots$}
    (1,-1) node[draw] (Dbot) {$\Delta_n$}
    ;

    \fill[green!20!white,opacity=1] (GG.north east)
    parabola[bend at end] (Dtop.west)
    parabola[bend at start] (GG.south east)
    -- cycle;
    \fill[blue!40!white,opacity=.5] (GG.north east)
    parabola[bend at end] (Dbot.west)
    parabola[bend at start] (GG.south east)
    -- cycle;

    \draw[->] (GG.north east) parabola[bend at end] (Dtop.west);
    \draw (GG.south east) parabola[bend at end] (Dtop.west);
    \draw[->] (GG.north east) parabola[bend at end] (D.west);
    \draw (GG.south east) parabola[bend at end] (D.west);
    \draw[->] (GG.north east) parabola[bend at end] (Dbot.west);
    \draw (GG.south east) parabola[bend at end] (Dbot.west);
  \end{tikzpicture}
\end{displaymath}

Semantically, we interpret a simple substitution from $\Gamma$ to $\Delta$ as
a map of type $\sem\Gamma \to \sem\Delta$.
With semantics of contexts given by a Cartesian product of their elements,
this type is
$\left(\prod_{(x : A) \in \Gamma}\sem A\right) \to
\left(\prod_{(y : B) \in \Delta}\sem B\right)$.
We can then do the following rearrangement.
\todo{Internal vs external homs are not clear.}

\begin{align*}
  &\phantom{{}={}} \left(\prod_{(x : A) \in \Gamma}\sem A\right) \to
    \left(\prod_{(y : B) \in \Delta}\sem B\right) \\
  &\cong \prod_{(y : B) \in \Delta}\left(
         \left(\prod_{(x : A) \in \Gamma}\sem A\right) \to \sem B
         \right) \\
  &= \prod_{(y : B) \in \Delta}\sem{\Gamma \vdash B} \\
  &\cong \prod_B\left(\Delta \ni B \to \sem{\Gamma \vdash B}\right)
\end{align*}

Taking the semantics to be the term model, we get the standard definition of a
simultaneous substitution.
To see how this definition works, let us construct an example substitution:
$A, B \to C, B \Longrightarrow B, C$.
Intuitionistically, it suffices to give two separate substitutions,
$A, B \to C, B \Longrightarrow B$ and $A, B \to C, B \Longrightarrow C$,
which are both just terms.
We indeed have terms $x : A, y : B \to C, z : B \vdash y : B$ and
$x : A, y : B \to C, z : B \vdash y\,z : C$.

When working with our semiring-annotated calculus \name{}, instead of Cartesian
products we have (weighted) monoidal products, giving the type
$\left(\bigotimes_{(\gr rx : A) \in \Gamma}\oc \gr r\sem A\right) \to
\left(\bigotimes_{(\gr sy : B) \in \Delta}\oc \gr s\sem B\right)$.
As the monoidal product is not a limit, we cannot do the rearrangement we did
above.
Instead, we must divide the domain up into pieces, with each piece supporting
one factor from the codomain.

To see this division, let us adapt the previous example to have type
$\gr0A, \gr2(B \multimap C), \gr3B \Longrightarrow \gr1B, \gr2C$.
To divide the goal up, we note that $
\begin{pmatrix} \gr0 & \gr2 & \gr3 \end{pmatrix} =
\begin{pmatrix} \gr0 & \gr0 & \gr1 \end{pmatrix} +
\begin{pmatrix} \gr0 & \gr2 & \gr2 \end{pmatrix}
$, so it suffices to give substitutions of types
$\gr0A, \gr0(B \multimap C), \gr1B \Longrightarrow \gr1B$ and
$\gr0A, \gr2(B \multimap C), \gr2B \Longrightarrow \gr2C$.
Furthermore, our term calculus only supports $\gr1$-annotated conclusions,
so we divide the second substitution type through by $\gr2$.
Finally, we give the terms largely as before:
$\gr0x : A, \gr0y : B \to C, \gr1z : B \vdash y : B$ and
$\gr0x : A, \gr1y : B \to C, \gr1z : B \vdash z\,y : C$.

\begin{displaymath}
  \begin{tikzpicture}[baseline]
    \path
    (-1,1) node (Gtop) {}
    (-1,0) node (G) {$\grP\gamma$}
    (-1,-1) node (Gbot) {}
    ;
    \node[draw,dotted,fit=(Gtop) (G) (Gbot)] (GG) {};

    \path
    (1,1) node (Dtop) {}
    (1,0) node (D) {$\grQ\delta$}
    (1,-1) node (Dbot) {}
    ;
    \node[draw,dotted,fit=(Dtop) (D) (Dbot)] (DD) {};

    \draw[->,double] (GG) -- (DD);
  \end{tikzpicture}
  \coloneqq
  \begin{tikzpicture}[baseline]
    \path
    (-1,1) node (Gtop) {}
    (-1,0) node (G) {$\grP\gamma$}
    (-1,-1) node (Gbot) {}
    ;
    \node[draw,dotted,fit=(Gtop) (G) (Gbot)] (GG) {};

    \path
    (1,3) node (G1top) {}
    (1,2) node (G1) {$\gr\Psi_1\gamma$}
    (1,1) node (G1bot) {}
    ;
    \node[draw,dotted,fit=(G1top) (G1) (G1bot)] (GG1) {};
    \draw[->] (GG) -- (GG1);

    \path (1,0) node {$\vdots$};

    \path
    (1,-1) node (Gntop) {}
    (1,-2) node (Gn) {$\gr\Psi_n\gamma$}
    (1,-3) node (Gnbot) {}
    ;
    \node[draw,dotted,fit=(Gntop) (Gn) (Gnbot)] (GGn) {};
    \draw[->] (GG) -- (GGn);

    \path
    (3,1) node[draw] (Dtop) {$\delta_1$}
    (3,0) node (D) {$\vdots$}
    (3,-1) node[draw] (Dbot) {$\delta_n$}
    ;

    \fill[green!20!white] (GG1.north east)
    parabola[bend at end] (Dtop.west)
    parabola[bend at start] (GG1.south east)
    -- cycle;
    \draw[->] (GG1.north east) parabola[bend at end] (Dtop.west);
    \draw (GG1.south east) parabola[bend at end] (Dtop.west);

    \fill[blue!20!white] (GGn.north east)
    parabola[bend at end] (Dbot.west)
    parabola[bend at start] (GGn.south east)
    -- cycle;
    \draw[->] (GGn.north east) parabola[bend at end] (Dbot.west);
    \draw (GGn.south east) parabola[bend at end] (Dbot.west);
  \end{tikzpicture}
  \quad\textrm{where }\grP = \grQ\gr\Psi
\end{displaymath}

Generalising this division process, we derive the following (letting
$\Gamma = \grP\gamma$ and $\Delta = \grQ\delta$).
The various fragments of the original usage vector are collected up in the
existentially quantified $\gr\Psi$.

\begin{align*}
  &\phantom{{}={}}
    \left(\bigotimes_{(\gr rx : A) \in \Gamma}\oc \gr r\sem A\right) \to
    \left(\bigotimes_{(\gr sy : B) \in \Delta}\oc \gr s\sem B\right) \\
  &\cong \sum_{\gr\Psi : \size\Delta \to \size\Gamma \to \Ann}
    \left(\grP = \sum\gr\Psi\right) \times
    \prod_{(\gr sy : B) \in \Delta}\left(
    \left(\bigotimes_{(\gr rx : A) \in \gr\Psi_y\gamma}\oc \gr r\sem A\right)
    \to \oc \gr s\sem B
    \right) \\
  &\cong \sum_{\gr\Psi : \size\Delta \to \size\Gamma \to \Ann}
    \left(\grP = \sum_{(\gr sy : B) \in \Delta}\gr s\gr\Psi_y\right) \times
    \prod_{(\gr sy : B) \in \Delta}\left(
    \left(\bigotimes_{(\gr rx : A) \in \gr\Psi_y\gamma}\oc \gr r\sem A\right)
    \to \sem B
    \right) \\
  &= \sum_{\gr\Psi : \size\Delta \to \size\Gamma \to \Ann}
    \left(\grP = \grQ\gr\Psi\right) \times
    \prod_{(\gr sy : B) \in \Delta} \sem{\gr\Psi_y\gamma \vdash B}
\end{align*}

In other words, we interpret a linear substitution as a \emph{matrix}
$\gr\Psi$ relating the two usage contexts, together with a term for each
variable from $\Delta$ typed in a fragmented usage context dictated by
$\gr\Psi$.

While we naturally derive a matrix as a fragmentation of a usage vector, we can
get a slightly cleaner presentation by instead using an abstract linear map.
Let $\gr\Psi$ now be a linear map of type
$\Ann^{\size\Delta} \to \Ann^{\size\Gamma}$, with application written postfix.
The equation $\grP = \grQ\gr\Psi$ remains unchanged.
As for $\prod_{(\gr sy : B) \in \Delta}\sem{\gr\Psi_y\gamma \vdash B}$, the most
direct fix is to replace $\gr\Psi_y$ by $\langle y \rvert\gr\Psi$.
But then we notice that $\langle y \rvert$ is exactly the $\grQprime$ satisfying
$\grQprime\delta \sqni y : B$.
Thus, let us derive the following.

\begin{align*}
  &\phantom{{}={}} \prod_{(\gr sy : B) \in \Delta}
    \sem{\langle y \rvert \gr\Psi\gamma \vdash B} \\
  &\cong \prod_{B,\grQprime} \left(\grQprime\delta \sqni B \to
    \sem{\grQprime\gr\Psi\gamma \vdash B}\right) \\
  &\cong \prod_{B,\grQprime,\grPprime} \left(
    \grPprime = \grQprime\gr\Psi \to \grQprime\delta \sqni B \to
    \sem{\grPprime\gamma \vdash B}\right)
\end{align*}

Putting everything back together gives us the final expression.

\begin{align*}
  &\sum_{\gr\Psi : \Ann^{\size\Delta} \to \Ann^{\size\Gamma}}
    \left(\grP = \grQ\gr\Psi\right) \times
    \prod_{B,\grQprime,\grPprime} \left(
    \grPprime = \grQprime\gr\Psi \to \grQprime\delta \sqni B \to
    \sem{\grPprime\gamma \vdash B}\right)
\end{align*}

We now have a new reading for the interpretation of a linear substitution:
a linear map $\gr\Psi$ relating the two usage vectors $\grP$ and $\grQ$, and
for any two similarly related usage vectors $\grPprime$ and $\grQprime$, we
have a type-preserving function from variables in $\grQprime\delta$ to terms in
$\grPprime\gamma$.
Even though we don't use $\gr\Psi$ as a matrix containing fragmented usage
vectors, we can still justify why it should be a \emph{linear} map.
We need $\gr\Phi$ to respect all fragmentation of the usage context in a typing
rule, and we know that all such fragmentation is done by linear operations
zero, addition, and scaling by a constant.
\todo{Expand. Substitutions need to preserve everything done to the context,
and linear things are all we do to the context.}
