We have seen in previous sections a method for defining well typed terms,
providing them with the basic operations of renaming and substitution, and
defining type-preserving semantic traversals over those terms.
However, the Agda code we have seen only deals with one specific kind of terms
--- simply typed $\lambda$-calculus with a base type and function types.
The aim of this section is to write some code to which we can pass a
\emph{description} or \emph{signature} of a syntax and have it produce all of
the same machinery.

The description of a syntax will closely resemble the logical rules Gentzen gave
for natural deduction systems NJ and NK, but we give them a revised
interpretation.
Where Gentzen intended his rules to be applied schematically, and hypothetical
proofs to be handled via \emph{discharge} of hypotheses, we will take the rules
formally to produce a system with explicit contexts and a variable rule.
However, knowing that this resulting system came from such a description means
that we can derive variable-handling features, such as substitution, in a
generic way.

I will present a scheme based on the work of \citet{AACMM21} such that
\cref{fig:app-lam} is interpreted as the type
system we studied in the previous sections (simply typed $\lambda$-calculus with
a base type and function types).
Remember that, while these look like inference rules, I am treating them
entirely formally, collected together into a \emph{syntax description}.
The information presented in \cref{fig:app-lam} is essentially all of the
information needed for the type system sans any details about variables.
In particular, notice:
\begin{itemize}
  \item Contexts, in particular the context of a rule's conclusion, which is
    shared in all premises in the resulting type system, are elided.
    The only part of any context I record is the newly bound variables in
    premises, such as the variable bound by a $\lambda$-abstraction.
  \item There is no explicit variable rule.
    It is understood that any $x : A$ in the context of the resulting type
    system can be used to yield a term with type $A$.
\end{itemize}

\begin{figure}
  \begin{mathpar}
    \ebrule{%
      \hypo{\vdash A \to B}
      \hypo{\vdash A}
      \infer2[app]{\vdash B}
    }
    \and
    \ebrule{%
      \hypo{A \vdash B}
      \infer1[lam]{\vdash A \to B}
    }
  \end{mathpar}
  \caption{An example syntax description}
  \label{fig:app-lam}
\end{figure}

Such a scheme commits us to a certain approach to variable binding and context
management, but does not commit us to anything about the meaning of types.
For example, we do not declare that \TirName{app} and \TirName{lam} are
``elimination'' and ``introduction'' forms for the function type former.
This limits our generic results to matters of syntax and variables, but provides
a platform upon which a future semantic scheme could rest.

\begin{figure}
  \begin{align*}
    \text{Premises} && \mathit{ps}, \mathit{qs} &\Coloneqq
      \Delta \vdash A \mid {} \mid \mathit{ps} \quad \mathit{qs}
    \\
    \text{Rule} && r, s &\Coloneqq {\mathit{ps} \over \vdash A}
  \end{align*}
  \caption{The grammar of typing rules}
  \label{fig:simple-syntax}
\end{figure}

A syntax description is a set of fully instantiated rules.
In our running example, this set contains a \TirName{app}-rule and a
\TirName{lam}-rule for each pair of types $A$ and $B$.

To construct the syntax given by a description, we keep
\AgdaInductiveConstructor{var} as before, and have another constructor
\AgdaInductiveConstructor{con} for all of the logical rules.
\AgdaInductiveConstructor{con} takes a rule \AgdaBound{r} with premises
$\Delta_1 \vdash A_1; \ldots; \Delta_n \vdash A_n$ and conclusion $A$, and the
remainder of its type is as follows.
\[
  \AgdaInductiveConstructor{con}~\AgdaBound{r}~\AgdaSymbol{:}~%
  \forall\Gamma.~(\Gamma, \Delta_1 \vdash A_1) \times \cdots
  \times (\Gamma, \Delta_n \vdash A_n) \to \Gamma \vdash A
\]
Note that, in this type, $\vdash$ is the type family of terms we are
inductively constructing, as opposed to the description syntax found in the
premises.

In our generic version of \AgdaRecord{Semantics}, we keep the
\AgdaField{ren\textasciicircum$\V$} and \AgdaField{$\llbracket$var$\rrbracket$}
fields as before, and replace \AgdaField{$\llbracket$app$\rrbracket$} and
\AgdaField{$\llbracket$lam$\rrbracket$} by a
\AgdaField{$\llbracket$con$\rrbracket$} field as follows.
\[
  \AgdaField{$\llbracket$con$\rrbracket$}~\AgdaBound{r}~\AgdaSymbol{:}~%
  \forallb{%
    \Box\plr{{} \env\V \Delta_1 \dotto {} \sdtstile{}\C A_1} \dottimes
    \cdots \dottimes
    \Box\plr{{} \env\V \Delta_n \dotto {} \sdtstile{}\C A_n} \dotto
    {} \sdtstile{}\C A}
\]
I use ${} \sdtstile{}\C A$ for the Agda notation
\AgdaFunction{\_$\C\vDash$}\AgdaSpace{}\AgdaBound{A}, while ${} \env\V \Delta$
stands for the type family of environments
\AgdaSymbol{$\lambda$}\AgdaSpace{}\AgdaBound{$\Gamma$}\AgdaSpace{}%
\AgdaSymbol{$\to$}\AgdaSpace{}\AgdaFunction{Env}\AgdaSpace{}\AgdaBound{$\V$}%
\AgdaSpace{}\AgdaBound{$\Gamma$}\AgdaSpace{}\AgdaBound{$\Delta$}.
Environments appear in this definition simply as a way to write a product of
$\V$-values --- one value for each element of $\Delta$.
We could make a special case of premises which do not bind any variables, as did
\citet{AACMM21}, eliding the $\Box$ and empty environment, but I choose not to
for uniformity and simplicity of presentation.

To generate the expressions involving ellipses, I give an interpretation of the
formal rule descriptions.
The interpretation is parametrised on some
\ExecuteMetaData[\SimpleSyntaxtex]{WithScope}, where, in
\AgdaBound{,}\AgdaSpace{}\AgdaBound{$\Delta$}\AgdaSpace{}%
\AgdaBound{$\llbracket$}\AgdaSpace{}\AgdaBound{$\Gamma$}\AgdaSpace{}%
\AgdaBound{$\vdash$}\AgdaSpace{}\AgdaBound{A}\AgdaSpace{}%
\AgdaBound{$\rrbracket$}, the context \AgdaBound{$\Delta$} stands for the newly
bound variables of a premise, \AgdaBound{$\Gamma$} is the context as it was
below the rule's horizontal line, and \AgdaBound{A} is the type of the premise.
In the \AgdaInductiveConstructor{con} constructor for terms, the parameter is
$\Gamma, \Delta \vdash A$, and in the \AgdaField{$\llbracket$con$\rrbracket$}
field for semantics, the parameter is
$\Box\plr{{} \env\V \Delta \dotto {} \sdtstile{}\C A}\,\Gamma$.

A single premise with newly bound variables is interpreted by shuffling the
parts into the right place, while multiple premises are interpreted as pointwise
products of the individual premises (giving the ellipses above).

\ExecuteMetaData[\SimpleSyntaxtex]{semp}

A rule, with all its parameters instantiated, targets a specific type
\AgdaBound{A$'$}, which we check to match the desired type \AgdaBound{A}.
Finally, a whole \AgdaRecord{System} comprises a set \AgdaBound{L} of rule
labels, and \AgdaBound{rs}\AgdaSpace{}\AgdaSymbol{:}\AgdaSpace{}\AgdaBound{L}%
\AgdaSpace{}\AgdaSymbol{$\to$}\AgdaSpace{}\AgdaRecord{Rule}.
The interpretation of these data is to pick a rule label \AgdaBound{l}, and then
take the interpretation of the rule \AgdaBound{rs}\AgdaSpace{}\AgdaBound{l}.
For the sake of defining terms as a least fixed point, it is important to note
that the interpretation of a syntax description is strictly positive in the
parameter \AgdaBound{,\_$\llbracket$\_$\vdash$\_$\rrbracket$}.

\ExecuteMetaData[\SimpleSyntaxtex]{semr}
\ExecuteMetaData[\SimpleSyntaxtex]{sems}

The interpretation of a system description as a single layer of syntax is
functorial, supporting the \AgdaFunction{map-s} function when the parameter
\AgdaBound{,\_$\llbracket$\_$\vdash$\_$\rrbracket$} is given as an extra
argument named \AgdaBound{X} or \AgdaBound{Y} (which are both fixed as
parameters of \AgdaFunction{map-s}, together with \AgdaBound{$\Gamma$} and
\AgdaBound{$\Delta$}).

\ExecuteMetaData[\SimpleSyntaxtex]{map-s-type}

The implementation of \AgdaFunction{map-s} is straightforward, so I do not list
it here.
We preserve
the shape of the syntactic layer, applying the function to each \AgdaBound{X}
we find (wherever the description contains \AgdaInductiveConstructor{$\langle$}%
\AgdaSpace{}\AgdaBound{$\Delta$}\AgdaSpace{}%
\AgdaInductiveConstructor{`$\vdash$}\AgdaSpace{}\AgdaBound{A}\AgdaSpace{}%
\AgdaInductiveConstructor{$\rangle$}).

This \AgdaFunction{map-s} will be used in the generic syntax version of
\AgdaFunction{sem} to recursively apply \AgdaFunction{sem} to all subterms.
However, a major distinction between generic syntax and the specific syntax of
previous sections is that the subterms found by \AgdaFunction{map-s} are not
recognised by Agda's termination checker as \emph{structurally smaller} than
the original term.
Therefore, a na\"{i}vely written \AgdaFunction{sem} will fail Agda's termination
check.

To make \AgdaFunction{sem} pass the termination check, we have four major
options:
\begin{enumerate}
  \item Assert \AgdaFunction{sem} to be terminating, bypassing the termination
    check.
  \item Use Agda's \emph{sized types}~\citep{Abel10} to remember that the
    subterms are smaller.
  \item Avoid sized types, and index terms over some user-defined type (for
    example, natural numbers or ordinal notations) which is structurally smaller
    at subterms.
  \item Inline a new, instantiated version of \AgdaFunction{map-s} wherever it
    is used.
\end{enumerate}

Each of these approaches has drawbacks.
Approach 1 is clearly unsafe, in the sense that the fundamental lemma
\AgdaFunction{sem} is not being completely checked for type-correctness.
Approach 2 is also unsafe, because Agda's sized type implementation is known to
make the system inconsistent~\citep{AgdaIssue1201}.
Meanwhile, approach 3 is safe, but entails a lot of manually extracted and
supplied extra arguments, which I think would distract from the presentation and
make the resulting code harder to use.
Finally, approach 4 is safe, but limits code reuse (both of the function
\AgdaFunction{map-s} itself and any lemmas we may prove about it).
I choose to follow \citet{AACMM21} in using sized types, justified by the idea
that Agda may eventually have a sound implementation of sized types, at which
point I would want my code to be as easy to update for that new version of Agda
as possible.
\Citet{FS22} use approach 4, and in fact have only one use of (their equivalent
of) \AgdaFunction{map-s} in their development.

Using sized types, my type family of \AgdaRecord{System}-generic terms is as
below.
\AgdaFunction{Scope} is a name for the transformation of an
\AgdaFunction{OpenFam} into a \AgdaRecord{Ctx}\AgdaSpace{}\AgdaSymbol{$\to$}%
\AgdaSpace{}\AgdaFunction{OpenFam} which appends the extra context to the
existing context before applying the original \AgdaFunction{OpenFam}
(in this case, producing something like $\Gamma, \Delta \vdash A$ from
$\vdash$).
The Agda builtin \AgdaPostulate{$\uparrow$} produces a bigger size from an
existing size \AgdaBound{sz}, giving us here that the size of a term is 1 bigger
than the size of all of its immediate subterms.
Agda's elaborator and termination checker have special support for sizes, so we
do not have to worry much about them from this point on.

\ExecuteMetaData[\SimpleTermtex]{Tm}

Corresponding to the generic \AgdaInductiveConstructor{`con} constructor for
terms, we have a generic field \AgdaField{$\llbracket$con$\rrbracket$} in the
updated \AgdaRecord{Semantics} record.
In place of where one might expect
\AgdaFunction{Scope}\AgdaSpace{}\AgdaBound{$\C$}, we instead have
\AgdaFunction{Kripke}\AgdaSpace{}\AgdaBound{$\V$}\AgdaSpace{}\AgdaBound{$\C$},
with \AgdaFunction{Kripke} defined below.
The form of \AgdaFunction{Kripke} follows from the shape we saw in the type of
the \AgdaField{$\llbracket$lam$\rrbracket$} field we saw in \cref{sec:gen-sem},
where I use an environment targeting \AgdaBound{$\Delta$} as a way to say
``a value for each type in \AgdaBound{$\Delta$}''.

\ExecuteMetaData[\SimpleSemanticstex]{Kripke}
\ExecuteMetaData[\SimpleSemanticstex]{Semantics}

Finally, we get a generic semantic traversal as follows.
The function \AgdaFunction{bindEnv} is unchanged from \cref{sec:gen-sem}, as it
never mentions the syntax.
The type of the traversal \AgdaFunction{sem} is also basically unchanged --- we
just need to account for arbitrary term sizes (\AgdaBound{sz}), which will get
smaller when recursing on subterms.
I have chosen, as did \citet{AACMM21}, to define \AgdaFunction{sem} mutually
with a function \AgdaFunction{body}, which is like a counterpart to
\AgdaFunction{sem} dealing with newly bound variables.
Note that the mutual recursion is not essential --- for example,
\AgdaFunction{body} could simply be inlined.
The \AgdaInductiveConstructor{`var} case of \AgdaFunction{sem} is as before.
The \AgdaInductiveConstructor{`con} case, if viewed appropriately, is a direct
generalisation of the \AgdaInductiveConstructor{lam} case from earlier.
We recursively apply \AgdaFunction{sem} to all immediate subterms contained in
\AgdaBound{M} (as found by \AgdaFunction{map-s}), with an environment updated to
reflect the newly bound variables in each premise of the rule that was applied.

\ExecuteMetaData[\SimpleSemanticstex]{sem}
