We have seen in previous sections a method for defining well typed terms,
providing them with the basic operations of renaming and substitution, and
defining type-preserving semantic traversals over those terms.
However, the Agda code we have seen only deals with one specific kind of terms
--- simply typed $\lambda$-calculus with a base type and function types.
The aim of this section is to write some code to which we can pass a
\emph{description} or \emph{signature} of a syntax and have it produce all of
the same machinery.

The description of a syntax will closely resemble the logical rules Gentzen gave
for natural deduction systems NJ and NK, but we give them a revised
interpretation.
Where Gentzen intended his rules to be applied schematically, and hypothetical
proofs to be handled via \emph{discharge} of hypotheses, we will take the rules
formally to produce a system with explicit contexts and a variable rule.
However, knowing that this resulting system came from such a description means
that we can derive variable-handling features, such as substitution, in a
generic way.

I will present a scheme such that \cref{fig:app-lam} is interpreted as the type
system we studied in the previous sections (simply typed $\lambda$-caluclus with
a base type and function types).
Remember that, while these look like inference rules, I am treating them
entirely formally, collected together into a \emph{syntax description}.
The information presented in \cref{fig:app-lam} is essentially all of the
information needed for the type system sans any details about variables.
In particular, notice:
\begin{itemize}
  \item Contexts, in particular the context of a rule's conclusion, which is
    shared in all premises in the resulting type system, are elided.
    The only part of any context I record is the newly bound variables in
    premises, such as the variable bound by a $\lambda$-abstraction.
  \item There is no explicit variable rule.
    It is understood that any $x : A$ in the context of the resulting type
    system can be used to yield a term with type $A$.
\end{itemize}

\begin{figure}
  \begin{mathpar}
    \ebrule{%
      \hypo{\vdash A \to B}
      \hypo{\vdash A}
      \infer2[app]{\vdash B}
    }
    \and
    \ebrule{%
      \hypo{A \vdash B}
      \infer1[lam]{\vdash A \to B}
    }
  \end{mathpar}
  \caption{An example syntax description}
  \label{fig:app-lam}
\end{figure}

Such a scheme commits us to a certain approach to variable binding and context
management, but does not commit us to anything about the meaning of types.
For example, we do not declare that \TirName{app} and \TirName{lam} are
``elimination'' and ``introduction'' forms for the function type former.
This limits our generic results to matters of syntax and variables, but provides
a platform upon which a future semantic scheme could rest.

\begin{figure}
  \begin{align*}
    \text{Premises} && \mathit{ps}, \mathit{qs} &\Coloneqq
      \Delta \vdash A \mid {} \mid \mathit{ps} \quad \mathit{qs}
    \\
    \text{Rule} && r, s &\Coloneqq {\mathit{ps} \over \vdash A}
  \end{align*}
  \caption{The grammar of typing rules}
  \label{fig:simple-syntax}
\end{figure}

A syntax description is a set of fully instantiated rules.
In our running example, this set contains a \TirName{app}-rule and a
\TirName{lam}-rule for each pair of types $A$ and $B$.

To construct the syntax given by a description, we keep
\AgdaInductiveConstructor{var} as before, and have another constructor
\AgdaInductiveConstructor{con} for all of the logical rules.
\AgdaInductiveConstructor{con} takes a rule \AgdaBound{r} with premises
$\Delta_1 \vdash A_1; \ldots; \Delta_n \vdash A_n$ and conclusion $A$, and the
remainder of its type is as follows.
\[
  \AgdaInductiveConstructor{con}~\AgdaBound{r}~\AgdaSymbol{:}~%
  \forall\Gamma.~(\Gamma, \Delta_1 \vdash A_1) \times \cdots
  \times (\Gamma, \Delta_n \vdash A_n) \to \Gamma \vdash A
\]
Note that, in this type, $\vdash$ is the type family of terms we are
inductively constructing, as opposed to the description syntax found in the
premises.

In our generic version of \AgdaRecord{Semantics}, we keep the
\AgdaField{ren\textasciicaron$\V$} and \AgdaField{$\llbracket$var$\rrbracket$}
fields as before, and replace \AgdaField{$\llbracket$app$\rrbracket$} and
\AgdaField{$\llbracket$lam$\rrbracket$} by a
\AgdaField{$\llbracket$con$\rrbracket$} field as follows.
\[
  \AgdaField{$\llbracket$con$\rrbracket$}~\AgdaBound{r}~\AgdaSymbol{:}~%
  \forallb{%
    \Box\plr{{} \env\V \Delta_1 \dotto {} \sdtstile{}\C A_1} \dottimes
    \cdots \dottimes
    \Box\plr{{} \env\V \Delta_n \dotto {} \sdtstile{}\C A_n} \dotto
    {} \sdtstile{}\C A}
\]
I use ${} \sdtstile{}\C A$ for the Agda notation
\AgdaFunction{\_$\C\vDash$}\AgdaSpace{}\AgdaBound{A}, while ${} \env\V \Delta$
stands for the type family of environments
\AgdaSymbol{$\lambda$}\AgdaSpace{}\AgdaBound{$\Gamma$}\AgdaSpace{}%
\AgdaSymbol{$\to$}\AgdaSpace{}\AgdaFunction{Env}\AgdaSpace{}\AgdaBound{$\V$}%
\AgdaSpace{}\AgdaBound{$\Gamma$}\AgdaSpace{}\AgdaBound{$\Delta$}.
Environments appear in this definition simply as a way to write a product of
$\V$-values --- one value for each element of $\Delta$.
We could make a special case of premises which do not bind any variables, as did
\citet{AACMM21}, eliding the $\Box$ and empty environment, but I choose not to
for uniformity and simplicity of presentation.
