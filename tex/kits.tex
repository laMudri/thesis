\def\prefix{../agda/latex}
\CatchFileBetweenTags{\Var}{\prefix/SimpleKits.tex}{Var}
\CatchFileBetweenTags{\Term}{\prefix/SimpleKits.tex}{Term}
\CatchFileBetweenTags{\Ren}{\prefix/SimpleKits.tex}{Ren}
\CatchFileBetweenTags{\bindRen}{\prefix/SimpleKits.tex}{bindRen}
\CatchFileBetweenTags{\rename}{\prefix/SimpleKits.tex}{rename}
\CatchFileBetweenTags{\leftTerm}{\prefix/SimpleKits.tex}{leftTerm}
\CatchFileBetweenTags{\Sub}{\prefix/SimpleKits.tex}{Sub}
\CatchFileBetweenTags{\bindSub}{\prefix/SimpleKits.tex}{bindSub}
\CatchFileBetweenTags{\substitute}{\prefix/SimpleKits.tex}{substitute}
\CatchFileBetweenTags{\Env}{\prefix/SimpleKits.tex}{Env}
\CatchFileBetweenTags{\RenSub}{\prefix/SimpleKits.tex}{RenSub}
\CatchFileBetweenTags{\Kit}{\prefix/SimpleKits.tex}{Kit}
\CatchFileBetweenTags{\trav}{\prefix/SimpleKits.tex}{trav}
\CatchFileBetweenTags{\renKit}{\prefix/SimpleKits.tex}{renKit}
\CatchFileBetweenTags{\subKit}{\prefix/SimpleKits.tex}{subKit}

\CatchFileBetweenTags{\RenGD}{\prefix/SimpleKits.tex}{RenGD}
\CatchFileBetweenTags{\RenGADA}{\prefix/SimpleKits.tex}{RenGADA}
\CatchFileBetweenTags{\GTh}{\prefix/SimpleKits.tex}{GTh}
\CatchFileBetweenTags{\DTh}{\prefix/SimpleKits.tex}{DTh}
\CatchFileBetweenTags{\GD}{\prefix/SimpleKits.tex}{GD}

\subsection{Simultaneous renaming and simultaneous substitution}

A simultaneous renaming from $\Gamma$ to $\Delta$ is a type-preserving map from
variables in $\Delta$ to \emph{variables} in $\Gamma$, while a simultaneous
substitution is a map into \emph{terms} in $\Gamma$.
While simultaneous substitution gives us a notion of one context being
\emph{derivable} from another, simultaneous renaming gives a similar notion
of derivability restricted to structural rules.

\begin{mathpar}
  \inferrule*[right=Subst]
  {%
    \inferrule*[right=$\to$-E]
    {%
      \inferrule*[right=Var]{~}{A \to B, A \vdash A \to B}
      \\
      \inferrule*[Right=Var]{~}{A \to B, A \vdash A}
    }
    {A \to B, A \vdash B}
    \\
    B \vdash C
  }
  {A \to B, A \vdash C}
\end{mathpar}

\begin{displaymath}
  \begin{prooftree}
    \infer0[Var]{A \to B, A \vdash A \to B}
    \infer0[Var]{A \to B, A \vdash A}
    \infer2[$\to$-E]{A \to B, A \vdash B}
    \hypo{B \vdash C}
    \infer2[Subst]{A \to B, A \vdash C}
  \end{prooftree}
\end{displaymath}

\subsection{Proofs of renaming and substitution}

We start with a data type \AgdaDatatype{\_⊢\_} of intrinsically simply typed
terms.
Beside base types, the only type former we have is the function type constructor
\AgdaInductiveConstructor{\_`→\_}.
Contexts (of type \AgdaRecord{Ctx}) are implemented as the free magma on types
(\AgdaDatatype{Ty}).
Context concatenation is \AgdaFunction{\_++ᶜ\_}, and \AgdaFunction{[\_]ᶜ}
embeds types into contexts.
Typed variables in a context are given by \AgdaRecord{\_∋\_}.
A variable in
\AgdaBound{$\Gamma$}\AgdaSpace{}\AgdaRecord{$\ni$}\AgdaSpace{}\AgdaBound{A}
is given by a path \AgdaField{idx} to a type in \AgdaBound{$\Gamma$}, together
with a proof \AgdaField{tyq} that this type is equal to \AgdaBound{A}.
Variables embed into terms via the \AgdaInductiveConstructor{var} constructor.

\Var{}
\Term{}

For this syntax, a renaming from \AgdaBound{$\Gamma$} to \AgdaBound{$\Delta$}
is a map from variables in \AgdaBound{$\Delta$} to variables in
\AgdaBound{$\Gamma$}.
Substitutions instead map into terms in \AgdaBound{$\Gamma$}.

\Ren{}
\Sub{}

In the following, \AgdaFunction{ren} gives the action of a renaming on terms.
We replace variables by new variables given by the renaming \AgdaBound{$\rho$}.
For applications, we rename each subterm using the same renaming.
For $\lambda$-abstractions, we want to rename the body, but the renaming we
have has type \RenGD{}, whereas the renaming we need is of type \RenGADA{}.
To make up the difference, we introduce the \AgdaFunction{bindRen} lemma,
saying that any renaming can be extended to the right by a context
\AgdaBound{$\Theta$}~\footnote{%
  We only require extension to the right because our syntax only has binding
  on the right.
  We could also extend on the left.
}.
To produce such an extended renaming, we receive a variable in \DTh{}, with the
two cases being that this variable is either in \AgdaBound{$\Delta$} or
\AgdaBound{$\Theta$}.
In the first case, we can use the original renaming \AgdaBound{$\rho$} to get
a variable in \AgdaBound{$\Gamma$}, which can be extended in the obvious way
to a variable in \GTh{}.
In the second case, we have a variable in \AgdaBound{$\Theta$}, and just need
to extend it to be a variable in \GTh{}.

\bindRen{}
\rename{}

We use renaming to show that, like variables, we can extend the context of terms
to the right.
The operation below renames a term by replacing all variables with variables
pointing to the left of the term's new context \GD{}.

\leftTerm{}

The action of a substitution on a term is given below.
The structure is very similar to that of renaming.
The differences are the following:

\begin{itemize}
  \item A substitution gives us terms, rather than variables.
        This means that when we use the substitution on a variable, we already
        have a term, and don't need to do any embedding into terms.
  \item When dealing with newly bound variables in \AgdaFunction{bindSub}, we
        need to produce terms, so we embed the newly bound variables into terms
        using \AgdaInductiveConstructor{var}.
  \item We use \AgdaFunction{$↙ᵗ$} instead of \AgdaFunction{$↙ᵛ$}, because we
        are dealing with terms rather than variables.
\end{itemize}

\bindSub{}
\substitute{}

\subsection{Kits}

To abstract over the similarities between renaming and substitution, we can use
\emph{kits} as introduced by McBride~\cite{McBride05,BHKM12}.
Each of the three differences above is turned into a parameter of
\AgdaFunction{trav} (generalising \AgdaFunction{ren} and \AgdaFunction{sub})
and \AgdaFunction{bindEnv} (generalising \AgdaFunction{bindRen} and
\AgdaFunction{bindSub}).
In the types, \AgdaFunction{Ren} and \AgdaFunction{Sub} are generalised by
\AgdaFunction{Env}\AgdaSpace{}\AgdaBound{K} --- a function from variables in
\AgdaBound{$\Delta$} to \AgdaBound{K}-things in \AgdaBound{$\Gamma$}.

\Env{}

The function \AgdaFunction{trav} produces a term-to-term mapping based on an
environment \AgdaBound{$\rho$}.
Like renaming and substitution, the traversal \AgdaFunction{trav} replaces
variables according to \AgdaBound{$\rho$}, while keeping the rest of the
syntactic forms intact.
The three differences between renaming and substitution present themselves as
requirements of the notion of \emph{kit} we can choose:

\begin{itemize}
  \item In the \AgdaInductiveConstructor{var} case of \AgdaFunction{trav}, we
        apply the environment \AgdaBound{$\rho$} to get a \AgdaBound{K}-thing,
        and need something to turn this \AgdaBound{K}-thing into a term.
        We let \AgdaField{tm} be this context- and type-preserving map from
        \AgdaBound{K}-things to terms.
  \item When dealing with newly bound variables in \AgdaFunction{bindEnv}, we
        need to convert the new variable into a \AgdaBound{K}-thing in order to
        put it into the environment.
        We let \AgdaField{vr} be this context- and type-preserving map from
        variables to \AgdaBound{K}-things.
  \item When working out where to map old variables in an extended context, we
        need \AgdaBound{K}-things to be stable under context extensions.
        We let \AgdaField{↙ᵏ} be the function embedding a \AgdaBound{K}-thing
        into an extended context.
\end{itemize}

\trav{}

The three parameters can be given to these functions by filling the fields of
the \AgdaRecord{Kit} record.

\Kit{}

We may now redefine the types \AgdaFunction{Ren} and \AgdaFunction{Sub} via
\AgdaFunction{Env}, and derive the actions of renaming and substitution from
\AgdaFunction{trav}.
Notice that \AgdaFunction{↙ᵗ} (written out as
\AgdaFunction{ren}\AgdaSpace{}\AgdaFunction{↙ᵛ}) still relies on renaming, but
because \AgdaFunction{ren} is only being used to fill a parameter of
\AgdaFunction{trav}, \AgdaFunction{trav} itself can be used to define
\AgdaFunction{ren} in a non-circular way.
Thus, we have succeeded in avoiding the duplication of code between
\AgdaFunction{ren} and \AgdaFunction{sub}.
%\todo{Distribute code horizontally}

\begin{multicols}{3}
  \noindent\RenSub{} \columnbreak

  \noindent\renKit{} \columnbreak

  \noindent\subKit{}
\end{multicols}

%\noindent
%\begin{tabular}{l|l|l}
%  \begin{minipage}{.25\textwidth}
%    \RenSub{}
%  \end{minipage}
%  &
%  \begin{minipage}{.25\textwidth}
%    \renKit{}
%  \end{minipage}
%  &
%  \begin{minipage}{.25\textwidth}
%    \subKit{}
%  \end{minipage}
%\end{tabular}
