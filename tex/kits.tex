\def\SimpleKits{../agda/processed-latex/SimpleKits.tex}

%Explain:
%
%\begin{itemize}
%  \item Specific uses of renaming/substitution in $\lambda$-calculus semantics.
%  \item General role of renaming/substitution in abstract algebra/syntax with
%    binding.
%\end{itemize}

A basic operation on any syntax with variables is \emph{substitution} --- the
replacement of variables in a term by terms with the same type as the variables.
In a sense, this is the defining operation of variables --- a variable is a
placeholder for a term, or equivalently in logic, a hypothesis is a placeholder
for an arbitrary proof.
In a type theory or logic, terms can bind variables, and we will typically have
operational semantics rules combining a term binding a variable with a term that
is to be substituted into the place of that variable, like the $\beta$-rule for
$\lambda$-calculus functions.

While substitution has this extra role in a lot of the syntaxes with binding we
care about, variable-binding also significantly complicates the substitution
operation.
Substitution acts on the free variables of a term, replacing them by terms, but
binders mean that some subterms have \emph{more} free variables than our
original term.
This causes different challenges for different representations of terms.
For example, with named variables and shadowing, na\"{i}vely defined
substitution could fall foul of variable capture.
In our approach, based on De Bruijn indices, the difficulty is that an index $i$
outside a binder of $n$ variables corresponds to an index $n + i$ inside the
binder.
Therefore, when substituting under a binder, we must first increment any free
variables contained in terms we are substituting in, which is a form of
\emph{renaming}.
Renaming replaces each free variable by another free variable, and is a special
case of substitution.
We must, however, define renaming before substitution, so as to avoid the
definition of substitution being circular.
Renaming avoids a similar circularity because when renaming goes under a binder,
we only have to increment each variable being renamed in, rather than each
variable \emph{in each term} being substituted in.

In this section, I formally implement simultaneous renaming and substitution for
the terms defined in the previous section.
Simultaneous substitution turns out to have a simple definition, which
generalises into other algorithms over terms with binders.
The section concludes with a unified implementation of renaming and
substitution, leaving further generalisation to the next section.

\subsection{Simultaneous renaming and simultaneous substitution}

A simultaneous renaming from $\Gamma$ to $\Delta$ is a type-preserving map from
variables in $\Delta$ to \emph{variables} in $\Gamma$, while a simultaneous
substitution is a map into \emph{terms} in $\Gamma$.
While simultaneous substitution gives us a notion of one context being
\emph{derivable} from another, simultaneous renaming gives a similar notion
of derivability restricted to structural rules.

In the derivation below, we assume the existence of a derivation of
$B, C \to C \vdash C$, and by the admissibility of substitution we thus have a
derivation of $A \to B, A \vdash C$.
Intuitively, the context $A \to B, A$ derives the context $B, C \to C$, so
anything derived from $B, C \to C$ can also be derived from $A \to B, A$.
We see formally that $A \to B, A$ derives $B, C \to C$ by deriving each element
of the latter from the former --- hence the first two premises of the \TirName{Subst}
rule below, deriving $B$ and $C \to C$ from $A \to B, A$.

\begin{align*}
  &\begin{prooftree}
    \hypo{\Pi}
    \infer[no rule]1{A \to B, A \vdash B}
    \infer0[Var]{A \to B, A, C \vdash C}
    \infer1[$\to$-I]{A \to B, A \vdash C \to C}
    \hypo{B, C \to C \vdash C}
    \infer3[Subst]{A \to B, A \vdash C}
  \end{prooftree}
  \\\\
  &\textrm{where }\Pi \coloneqq
  \begin{prooftree}
    \infer0[Var]{A \to B, A \vdash A \to B}
    \infer0[Var]{A \to B, A \vdash A}
    \infer2[$\to$-E]{A \to B, A \vdash B}
  \end{prooftree}
\end{align*}

\subsection{Proofs of admissibility of renaming and substitution}

A renaming from \AgdaBound{$\Gamma$} to \AgdaBound{$\Delta$}
is a map from variables in \AgdaBound{$\Delta$} to variables in
\AgdaBound{$\Gamma$}, represented in Agda as follows.

\Ren{}

The action of a renaming \AgdaBound{$\rho$} on terms is given by
\AgdaFunction{ren}\AgdaSpace{}\AgdaBound{$\rho$}, with \AgdaFunction{ren}
defined below.
The idea of simultaneous renaming is to preserve the structure of the term, but
replace all of the variables from \AgdaBound{$\Delta$} by variables from
\AgdaBound{$\Gamma$}, with the mapping given by the renaming \AgdaBound{$\rho$}.

\rename{}

The \AgdaInductiveConstructor{var} case is where the action of the renaming
happens: the variable \AgdaBound{x} from \AgdaBound{$\Delta$} is mapped to the
variable \AgdaBound{$\rho$}\AgdaSpace{}\AgdaBound{x} from \AgdaBound{$\Gamma$}.
In the \AgdaInductiveConstructor{app} case, we have terms \AgdaBound{M}
\AgdaSymbol{:} \AgdaBound{$\Delta$} \AgdaDatatype{$\vdash$}
\AgdaBound{A} \AgdaInductiveConstructor{`$\to$} \AgdaBound{B} and \AgdaBound{N}
\AgdaSymbol{:} \AgdaBound{$\Delta$} \AgdaDatatype{$\vdash$} \AgdaBound{A}.
We may apply \AgdaFunction{ren} \AgdaBound{$\rho$} recursively to both
\AgdaBound{M} and \AgdaBound{N} to change their contexts from
\AgdaBound{$\Delta$} to \AgdaBound{$\Gamma$}, and the
\AgdaInductiveConstructor{app} constructor then produces the desired term in
\AgdaBound{$\Gamma$}.
Finally, in the \AgdaInductiveConstructor{lam} case, we get a term
\AgdaBound{M} \AgdaSymbol{:} \DA{} \AgdaDatatype{$\vdash$} \AgdaBound{B} and,
after introducing a \AgdaInductiveConstructor{lam} on the right, are in need
of a term of type \GA{} \AgdaDatatype{$\vdash$} \AgdaBound{B}.
To recursively apply \AgdaFunction{ren} to \AgdaBound{M}, we must thus extend
the renaming \AgdaBound{$\rho$} \AgdaSymbol{:} \RenGD{} with the newly bound
variable.
For this, we need an auxiliary function \AgdaFunction{bindRen} such that
\AgdaFunction{bindRen} \AgdaBound{$\rho$} \AgdaSymbol{:} \RenGADA{}.
This new renaming will act like \AgdaBound{$\rho$} for variables in
\AgdaBound{$\Delta$}, and map the new variable of type \AgdaBound{A} to the
corresponding new variable in \GA{}.

\bindRen{}

The \AgdaFunction{bindRen} given here has a slightly generalised type, where
instead of binding just a single variable of type \AgdaBound{A}, we could
bind a whole context \AgdaBound{$\Theta$} of new variables.
The first case of \AgdaFunction{bindRen} is for old variables from
\AgdaBound{$\Delta$}, where we apply \AgdaBound{$\rho$} to get a variable in
\AgdaBound{$\Gamma$}, and then use \AgdaFunction{$↙ᵛ$} to embed that variable
into \GTh{}.
The second case is for new variables from \AgdaBound{$\Theta$}, which embed
straight into \GTh{}.

Meanwhile, a substitution from \AgdaBound{$\Gamma$} to \AgdaBound{$\Delta$} is
an inhabitant of \AgdaFunction{Sub}\AgdaSpace{}\AgdaBound{$\Gamma$}\AgdaSpace{}%
\AgdaBound{$\Delta$}, as defined below.
This definition is identical to the definition of \AgdaFunction{Ren}, except
that it gives us \emph{terms} in \AgdaBound{$\Gamma$} rather than variables.

\Sub{}

The \AgdaFunction{sub} function below gives the action of a substitution.
Similarly to renaming, we want to preserve the structure of the term, except
now variables in the original term are replaced by \emph{terms} in the new
context.

\substitute{}

Given that this time, \AgdaBound{$\rho$} is a substitution rather than a
renaming, \AgdaBound{$\rho$} \AgdaBound{x} is a term, and is sufficient in the
\AgdaInductiveConstructor{var} case.
The \AgdaInductiveConstructor{app} case again deals with the subterms
recursively and then recombines them with \AgdaInductiveConstructor{app}.
In the \AgdaInductiveConstructor{lam} case, we again have a mismatch if we
want to apply \AgdaFunction{sub} recursively to the subterm \AgdaBound{M} with
an extra free variable.
We have \AgdaBound{$\rho$} \AgdaSymbol{:} \SubGD{} but need a substitution of
type \SubGADA{}, so we introduce the auxiliary definition
\AgdaFunction{bindSub}.

\bindSub{}

For the old variables in the first case, we have \AgdaBound{$\rho$} to turn
them into terms in \AgdaBound{$\Gamma$}.
Turning a term in \AgdaBound{$\Gamma$} into a term in \GTh{} requires a form
of weakening we have not yet proved, so I write \AgdaFunction{$↙ᵗ$} in analogy
with \AgdaFunction{$↙ᵛ$}, and prove it below.
In the second case, we want to substitute the new variable by the \emph{term}
referring to this new variable in \GTh{}.

The final piece to define substitution is to define the function that weakens
a term by some newly bound variables \AgdaBound{$\Delta$}.
For this, we use the action of renaming, which we have fully defined already,
and in particular rename each variable in the term from a variable in
\AgdaBound{$\Gamma$} to a variable in \GD{}.

\leftTerm{}

With this, the action of substitution is defined, and depends on the action
of renaming.

\subsection{Syntactic kits}\label{sec:syntactic-kits}

As observed by \citet{McBride05,BHKM12},
the statements of simultaneous renaming and simultaneous substitution are
very similar, with substitution being the generalisation that allows
replacement of variables by terms rather than just other variables.
Following \citet{McBride05},
I will introduce a type family \AgdaFunction{Env} of \emph{environments}, and
redefine \AgdaFunction{Ren} and \AgdaFunction{Sub} as environments of
variables and terms, respectively.

\Env{}
\RenSub{}

The processes I described for constructing proofs of the admissibility of
renaming and substitution were also similar.
Indeed, when we line up the resulting functions, \AgdaFunction{ren} and
\AgdaFunction{sub}, and their auxiliaries, \AgdaFunction{bindRen} and
\AgdaFunction{bindSub}, we notice only three key
differences:

\begin{itemize}
  \item In the first cases of \AgdaFunction{bindRen} and \AgdaFunction{bindSub},
    we do \AgdaFunction{$↙ᵛ$} and \AgdaFunction{$↙ᵗ$}, respectively, based on
    whether we are weakening a variable or a term.
  \item In the second case of \AgdaFunction{bindSub}, we do an extra wrapping of
    the new variable by \AgdaInductiveConstructor{var}, so as to make it a term
    to go in the substitution.
  \item In the \AgdaInductiveConstructor{var} case of \AgdaFunction{ren}, we
    have \AgdaInductiveConstructor{var} \AgdaSymbol(\AgdaBound{$\rho$}
    \AgdaBound{x}\AgdaSymbol) rather than just \AgdaBound{$\rho$} \AgdaBound{x},
    because the renaming \AgdaBound{$\rho$} gives us a variable rather than a
    term.
\end{itemize}

We may abstract over these three differences using the record \AgdaRecord{Kit}.
As in \AgdaFunction{Env}, we think of \AgdaBound{K} as being either
\AgdaDatatype{\_$\ni$\_} or \AgdaDatatype{\_$\vdash$\_}.
The fields of \AgdaRecord{Kit} are given in the same order as the points of
difference above.
Wherever the difference was presence or absence of
\AgdaInductiveConstructor{var}, we will be able to fill that field with either
\AgdaInductiveConstructor{var} or the identity function \AgdaFunction{id}.

\Kit{}

The field \AgdaField{$\swarrow^k$} can be seen as a property of the
judgement form \AgdaBound{K}, saying that it supports a form of weakening.
We use \AgdaField{vr} when adding a newly bound variable to an environment, and
use \AgdaField{tm} when we do a lookup from an environment and want to get a
term out.
Given a \AgdaRecord{Kit} \AgdaBound{K}, we can write the syntactic traversal
function \AgdaFunction{trav} and its auxiliary \AgdaFunction{bindEnv}, in the
model of \AgdaFunction{ren}, \AgdaFunction{sub}, and their auxiliaries.

\trav{}
\bindEnv{}

Concrete kits can be given for variables and terms either by inspecting
\AgdaFunction{ren} and \AgdaFunction{sub} or by following the types.
Notice that the kit for terms requires the admissibility of renaming so as to
achieve weakening of a substitution by newly bound variables.
Fortunately, this can be the \AgdaFunction{ren} defined below in terms of
\AgdaFunction{trav}, so we can keep \AgdaFunction{trav} as the only syntactic
traversal we have to write.

\begin{multicols}{2}
  \noindent\renKit{} \columnbreak

  \noindent\subKit{}
\end{multicols}
