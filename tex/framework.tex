\chapter{A framework for usage-restricted calculi}\label{sec:framework}

In \cref{sec:semirings}, we saw how to use parametrisation over a partially
ordered semiring to recreate a range of usage-aware calculi.
However, $\name$, with its fixed set of type formers and syntactic forms, is a
long way from capturing the full range of linear-like programming languages
studied in the literature and required in practice.

In this chapter, I take the framework for typed syntaxes with binding developed
by \citet{AACMM21} and apply the principles we discovered in
\cref{sec:semirings} to yield a framework allowing for semiring-based usage
restrictions on variables.
Syntactically, I claim that this framework ranges over all finitiary
variable-based simply typed semiring-annotated calculi, with justification by
comparison to the framework of \citet{AACMM21} and some novel examples in
\cref{sec:other-syntaxes}.
I also derive analogues to some of the semantic results of \citet{AACMM21},
strengthening them to take advantage of usage restrictions.

The work in this chapter is fully mechanised in Agda, which allows me to be
precise about the various levels of domain-specific languages which appear.
I assume that the reader is familiar with the bunched connectives introduced in
\cref{fig:bunched} and the usage-aware environments of \cref{def:lr-env}.

\section{Syntax}

We take the insights of the previous section and use them to build a
generic framework for posemiring-annotated substructural systems in
Agda. We will first show \emph{descriptions} of systems, which are
comprised of rules that have premises combined using the bunched
combinators. We then show how to construct the Agda data type of
intrinsically well scoped, typed, and resourced terms for any given
system of our framework. We use the prototypical system from
\figref{fig:comb-lr} as a running example. \secref{sec:other-syntaxes}
presents further examples that our framework can express.

We now start to use Agda notation for record and data type
declarations, to emphasise that our framework has been implemented.

\subsection{Descriptions of Systems}

% We capture the form of rules exemplified previously\todo{Previously?} via
% \emph{descriptions} of rules.
% The key to making these descriptions work is that they only allow syntactic
% forms that preserve environments.
% These forms are: absence and multiplicity of subterms with the same usage
% annotations, absence and multiplicity of subterms with summed usage annotations,
% scaling of a subterm, and variable binding.\todo{Switching to Agda}

\paragraph{\AgdaDatatype{Premises}, \AgdaRecord{Rule}s, and \AgdaRecord{System}s.}

A type \AgdaRecord{System} is made up of multiple \AgdaRecord{Rule}s.
Each \AgdaRecord{Rule} comprises a tree of \AgdaDatatype{Premises} and
a type of conclusion. We assume that there is a
$\AgdaBound{Ty} : \AgdaPrimitiveType{Set}$ of types for the system in
scope.

The \AgdaDatatype{Premise} data type describes premises of rules,
using the bunched combinators from the previous section. A single
premise is introduced by the
\AgdaInductiveConstructor{$\langle$\_`$\vdash$\_$\rangle$}
constructor.  This allows binding of additional variables
\AgdaBound{$\Delta$} (with specified types and usage annotations) and
the specification of a conclusion type \AgdaBound{A} for this premise.
The remaining constructors are descriptions for the first-order
bunched connectives, and will be interpreted directly as such, below.

\ExecuteMetaData[\Syntaxtex]{Premises}

A \AgdaRecord{Rule} is a pair of some \AgdaDatatype{Premises} and a
conclusion. We use an infix arrow as a suggestive notation for rules.

\ExecuteMetaData[\Syntaxtex]{Rule}

Finally, a \AgdaRecord{System} consists of a set of rule labels (i.e.,
constructor names), and for each label a description of the
corresponding rule. We use $\rhd$ as infix notation for systems to
associate the label set with the rules.

\ExecuteMetaData[\Syntaxtex]{System}

\paragraph{\figref{fig:lr-comb} as a \AgdaRecord{System}.}

As an example, we transcribe the system defined in
\figref{fig:lr-comb} into a description.  We give the set of types of
this system as a data type \AgdaDatatype{Ty} (together with a base
type \AgdaInductiveConstructor{$\iota$}). We assume that there is a
posemiring \AgdaInductiveConstructor{Ann} in scope for the
annotations.There is one label for each instantiation of a logical
rule, but the labels contain no further information about subterms or
restrictions on the context. This will be provided when we associate
labels with \AgdaRecord{Rule}s in a \AgdaRecord{System}.

\noindent
\begin{minipage}[t]{0.5\textwidth}
  \ExecuteMetaData[\PaperExamplestex]{Ty}
  \ExecuteMetaData[\PaperExamplestex]{Side}
\end{minipage}
\begin{minipage}[t]{0.5\textwidth}
  \ExecuteMetaData[\PaperExamplestex]{qlR}
\end{minipage}

To build a system, we associate with each label a rule:

\ExecuteMetaData[\PaperExamplestex]{lR}

Compared to \figref{fig:lr-comb}, modulo the Agda notation, we can see
that the fundamental structure has been preserved: the rules match
one-to-one, and the bunched premises are the same. A major difference
is that we do not include a counterpart to the
\AgdaInductiveConstructor{var} rule in a
\AgdaRecord{System}. Variables are common to all the systems
representable in our framework.

\paragraph{Terms of a \AgdaRecord{System}.}

The next thing we want to do is to build terms in the described type system.
The following definitions are useful for talking about types indexed over
contexts, judgement forms, and judgement forms admitting newly bound variables,
respectively.

\ExecuteMetaData[\Syntaxtex]{OpenFam}

To specify the meaning of descriptions, we assume some \AgdaBound{X} : \AgdaFunction{ExtOpenFam},
% \ExecuteMetaData[\Interpretationtex]{X},
over which we form one layer of syntax, using the function
\AgdaFunction{$\llbracket$\_$\rrbracket$p} that interprets
\AgdaDatatype{Premises} defined below.  The first argument to
\AgdaBound{X} is the new variables bound by this layer of syntax, as
exemplified in the first clause of
\AgdaFunction{$\llbracket$\_$\rrbracket$p}.  The second argument is
the context containing the variables being carried over from the
previous layer.  Notice that this is not, in general, the same as the
context from the previous layer, because the usage annotations may
have been changed by connectives like
\AgdaInductiveConstructor{\_`$*$\_} and
\AgdaInductiveConstructor{\_`$\cdot$\_}.  The third argument is the
type of subterm required.

With the first clause of \AgdaFunction{$\llbracket$\_$\rrbracket$p} explained,
the rest are simply interpretations of premises into bunched combinators.
The superscript \AgdaFunction{$^c$} on the bunched connectives denotes that
they have been lifted from predicates on usage vectors to predicates on
contexts, with the type component of the context shared throughout.
Additive connectives \AgdaFunction{$\dot1$} and \AgdaFunction{$\dottimes$} are
already polymorphic (not relying on anything specific about usage vectors), so
do not need a \AgdaFunction{$^c$} variant.

\ExecuteMetaData[\Interpretationtex]{semp}

The interpretation of a \AgdaRecord{Rule} checks that the rule targets
the desired type and then interprets the rule's premises \AgdaBound{ps}.
Notice that the interpretation of the premises is independent of the conclusion
of the rule, which accounts for the difference in type between
\AgdaFunction{$\llbracket$\_$\rrbracket$p} and
\AgdaFunction{$\llbracket$\_$\rrbracket$r}.

\ExecuteMetaData[\Interpretationtex]{semr}

The interpretation of a \AgdaRecord{System} is to choose a rule label
\AgdaBound{l} from \AgdaBound{L} and interpret the corresponding rule
\AgdaBound{rs}\AgdaSpace{}\AgdaBound{l} in the same context and for the same
conclusion.

\ExecuteMetaData[\Interpretationtex]{sems}

The most obvious way to make such an \AgdaBound{X} is to use some existing
\AgdaFunction{OpenFam} on an extended context.
We defined \AgdaFunction{Scope} to do this: take the new variables
\AgdaBound{$\Delta$}, concatenate them onto the existing context
\AgdaBound{$\Gamma$}, and pass the extended context onto the judgement
\AgdaBound{T}.

\ExecuteMetaData[\Syntaxtex]{Scope}

%{\color{red}(Forward ref: for now, we could have inlined \texttt{Scope}.)}

We use \AgdaFunction{Scope} to deal with new variables in syntax.
Terms resemble the free monad over a layer-of-syntax functor, though
that picture is complicated by variable binding.  A term is either a
variable or a use of a logical rule together with terms for each of
the required subterms. The \AgdaFunction{Size} argument is where we
use sized types to record that subterms are smaller than the
surrounding term.

\ExecuteMetaData[\Termtex]{Term}

This definition uses \AgdaFunction{$\dotto$}, which, analogously to
\AgdaFunction{$\dottimes$}, is an index-preserving version of the function
space.
We take \AgdaFunction{$\dotto$} to handle $n$ many indices --- in this
case two (the context and the type).
Informally,
\AgdaFunction{$\forall[$}\AgdaSpace{}\AgdaBound{T}\AgdaSpace{}\AgdaFunction{]}
stands for
\AgdaSymbol{$\forall$}\AgdaSpace{}\AgdaSymbol{\{}%
\AgdaBound{x$_1$}\AgdaSpace{}\AgdaSymbol{$\ldots$}\AgdaSpace{}\AgdaBound{x$_n$}%
\AgdaSymbol{\}}\AgdaSpace{}\AgdaSymbol{$\to$}\AgdaSpace{}\AgdaBound{T}%
\AgdaSpace{}%
\AgdaBound{x$_1$}\AgdaSpace{}\AgdaSymbol{$\ldots$}\AgdaSpace{}\AgdaBound{x$_n$},
where \AgdaBound{T} is a type family with $n$ many indices.

Terms defined like this are still quite difficult to write, mainly because of
frequently changing usage contexts and the need for proofs that they all match
up.
We will see a way of automating these proofs in \cref{sec:usage-elaborator}.

%Here is an example term, using the \AgdaFunction{$\lambda$R} system.
%First, for ease of writing, we introduce pattern synonyms for each of the
%typing rules we use.

%\ExecuteMetaData[\PaperExamplestex]{patterns}

%Our example term is a function that flips a tagged union wrapped in an
%arbitrarily annotated \emph{bang}.
%Much of the effort in writing such a term goes into writing the various
%relatedness proofs between usage contexts --- observing, for example, that two
%usage contexts sum together to make a third, or that a usage context used for
%a variable is a basis vector.
%We give a method of automating these proofs in \cref{sec:usage-elaborator}.
%\todo{To be clear, we don't actually write this.}

%\ExecuteMetaData[\HeavyItex]{lR-term}

% A layer of syntax supports the following functorial action.

% \ExecuteMetaData[\Maptex]{map-s-type}

\subsection{Other syntaxes and syntactic forms}\label{sec:other-syntaxes}

\paragraph{The system $\mu\tilde\mu$.}
We can encode a usage-annotated version of System $L$/the
$\mu\tilde\mu$-calculus~\cite{CH00} --- a syntax for classical logic --- in
such a way that contexts capture the undistinguished parts of the sequent.
As such, the generic substitution lemma we get in \cref{sec:kits} is the form
of substitution required in standard $\mu\tilde\mu$-calculus metatheory.
Though the $\mu\tilde\mu$-calculus is originally described as a sequent
calculus~\cite{CH00}, we use the techniques of
\citet[p.~12]{herbelin-hab} and \citet{LC06} to present it as a natural
deduction system, thus giving a notion of \emph{variable} to the system.

Unlike the single judgement form of \name{} and standard simply typed
$\lambda$-calculi, the $\mu\tilde\mu$-calculus has three judgement forms:
terms, coterms, and commands.
Read logically, terms and coterms are seen to, respectively, prove and refute
propositions (types), while commands exhibit contradictions.
This means that the abstract \AgdaBound{Ty} in the generic framework is
instantiated to \AgdaDatatype{Conc} (for \emph{conclusion}) as below, with
\AgdaDatatype{Ty} not being exposed directly to the generic framework.
For now, we just consider multiplicative disjunction $\parr$ (\emph{par}) and
negation/duality, beside an uninterpreted base type.
These are enough to exhibit classical behaviour.

\noindent
\begin{minipage}[t]{0.5\textwidth}
  \ExecuteMetaData[\MuMuTildetex]{Ty}
\end{minipage}
\begin{minipage}[t]{0.5\textwidth}
  \ExecuteMetaData[\MuMuTildetex]{Conc}
\end{minipage}

With \AgdaBound{Ty} instantiated as \AgdaDatatype{Conc}, all terms are assigned
\AgdaDatatype{Conc} type, as are all the variables.
No variables are given \AgdaInductiveConstructor{com} type, similar to how in
the bidirectional typing syntax of \citet[p.~25]{AACMM21}, no variables are
given \AgdaInductiveConstructor{Check} type.
How to observe this invariant is covered in the latter paper, so we will not
repeat it here (having not yet seen how to write traverals on terms).

The syntax comprises a \emph{cut} between a term and a coterm of the same type,
the eponymous $\mu$ and $\tilde\mu$ constructs for proof by contradiction, and
then term and coterm (introduction and elimination) forms for negation and
\emph{par}.

\ExecuteMetaData[\MuMuTildetex]{MMT}

%With a collection of pattern synonyms and the machinery from
%\cref{sec:usage-elaborator}, we can write an example term: a function which
%flips the disjuncts of a \emph{par}.

%\ExecuteMetaData[\MuMuTildeTermtex]{patterns}
%\ExecuteMetaData[\MuMuTildeTermtex]{myComm}

\paragraph{Duplicability}
There is one more bunched combinator we have experimented with adding to the
framework:

\[
  \plr{\Box T}\,\grR \coloneqq \Sigma\grRprime.~\plr{\grRprime \leq \grR}
  \times \plr{\grRprime \leq \gr0}
  \times \plr{\grRprime \leq \grRprime + \grRprime}
  \times T\,\grRprime
\]

The idea of $\plr{\Box T}\,\grR$ is to assert that $\grR$, or some refinement
of it, can be both discarded and duplicated indefinitely, and in the
refinement we have a $T$.
We use this combinator to introduce subterms that are used an unknown number of
times, for example the continuations of the eliminator of an inductive type,
or other fixed points.
We can also use it in linear/non-linear style systems~\cite{Benton94} to make
sure linear variables are not available in the intuitionistic fragment.

Adding the $\Box$ combinator is the only thing we have found that requires our
linear maps be functional rather than merely relational.


\section{Semantics}

Having fixed a universe of syntaxes, in which we can build terms, the next thing
to do is to write recursive functions on terms.
With terms being given by an inductive \AgdaSymbol{data} type definition, they
already come with a recursion and an induction principle.
However, these principles do not handle variable-binding, which we have seen
with the fact that we had to write the \AgdaFunction{bind} helper
function for renaming and substitution in \cref{sec:lrsub}.

In this chapter, the central construct is a function \AgdaFunction{semantics}
which, for a $\V$-environment $\rho : \Gamma \env\V \Delta$,
maps a term $M : \Delta \vdash A$ to some semantic value in the type
$\C\,\Gamma\,A$.
This is a direct adaptation of the \AgdaFunction{semantics} function of
\cref{sec:gen-sem}, which has the same kind of action on intuitionistic terms,
given similar operations on $\V$ and $\C$ as what we had earlier.
The \AgdaFunction{semantics} function recurses on the term $M$, updating $\rho$
whenever new variables are bound.
In our usage-aware case, $\rho$ is also updated whenever we come across linear
combinations induced by premise combinators $I^{\sep}$, $\sep$, and
$\gr r \cdot {}$.

This chapter is structured as follows.
I start by giving a quick introduction to linear relations --- a generalisation
of linear maps --- with reference to their use in mechanised algebraic
reasoning, in \cref{sec:lin-rel}.
Using linear relations, I give a functorial \emph{map} operation to a single
layer of syntax in \cref{sec:functorial}.
I then adapt the \AgdaFunction{Kripke} function space to the usage-aware
setting in \cref{sec:kripke}.
Then I apply the \AgdaFunction{Kripke} function space, along with much of the
machinery I have introduced in previous chapters, to give the
\AgdaFunction{semantics} function in \cref{sec:traversal}.
Finally, I give the syntax-generic simultaneous renaming and substitution
operations in \cref{sec:kit-to-sem}.

% Our goal in this section is to define \AgdaFunction{semantics}, a
% recursor that turns a term into a \AgdaBound{$\C$}-value using a
% \AgdaBound{$\V$}-environment, in a type preserving way:\bob{Get rid of
%   ``body'' here}

% \ExecuteMetaData[\Semanticstex]{semantics-type}

% The \AgdaBound{$\V$} and \AgdaBound{$\C$} are \AgdaFunction{OpenFam}s,
% representing the interpretations of variables and terms
% respectively. In \cref{sec:traversal} we will see the data that must
% be provided to make a \AgdaFunction{semantics} for a given
% system. Before that, we must see how to deal with the two complicated
% features of our syntax: the usage annotations (\cref{sec:functorial})
% and variable binding (\cref{sec:kripke}). \todo{fwd ref to where these are used}

\section{Linear relations in Agda}\label{sec:lin-rel}

In \cref{sec:lrkits}, I defined \emph{usage-annotated environments}
(\cref{def:lr-env}).
One component of a usage-annotated environment is a linear map $\gr\Psi$ which,
when applied to the target usage vector, gives a vector compatible with the
source usage vector.

When it comes to mechanisation, I prefer to replace an assertion
``$\grP \leq \grQ\gr\Psi$'', involving a linear map $\gr\Psi$, by an assertion
``$\grP\gr\Psi\grQ$'', where $\gr\Psi$ is now a linear \emph{relation} said to
relate $\grP$ and $\grQ$.
I define linear relations as follows, where the reader may wish to check that
a linear map gives rise to a linear relation via the expression
$\grP \leq \grQ\gr\Psi$.

\begin{definition}\label{def:linear-relation}
  Given a posemiring $\Ann$ and modules $\mathscr M$ and $\mathscr N$ over
  $\Ann$, a \emph{linear relation} between $\mathscr M$ and $\mathscr N$ is
  a relation $\gr\Psi$ between the underlying sets of $\mathscr M$ and
  $\mathscr N$ such that the following properties hold of all
  $\grP, \grPprime, \grPl, \grPr \in \mathscr M$ and all
  $\grQ, \grQprime, \grQl, \grQr \in \mathscr N$.
  \begin{align*}
    \grPprime \leq \grP \land \grP\gr\Psi\grQ \land \grQ \leq \grQprime
    &\implies \grPprime\gr\Psi\grQprime
    \\
    \plr{\exists\grQ.~\grP\gr\Psi\grQ \land \grQ \leq \gr0}
    &\implies \grP \leq \gr0
    \\
    \plr{\exists\grQ.~\grP\gr\Psi\grQ \land \grQ \leq \grQl + \grQr}
    &\implies \plr{\exists\grPl,\grPr.~\grP \leq \grPl + \grPr
      \land \grPl\gr\Psi\grQl \land \grPr\gr\Psi\grQr}
    \\
    \plr{\exists\grQ.~\grP\gr\Psi\grQ \land \grQ \leq \gr r\grQprime}
    &\implies \plr{\exists\grPprime.~\grP \leq \gr r\grPprime
      \land \grPprime\gr\Psi\grQprime}
  \end{align*}
  I write $\mathscr M \rel \mathscr N$ as the type of linear relations between
  $\mathscr M$ and $\mathscr N$.
\end{definition}

Relations have several advantages over functions when doing mechanised algebra
in type theory.
For one, what are compound expressions in functional style --- for example
$x \leq f(y) + g(z)$ --- become collections of simple relationships in
relational style --- for example $\exists v,w.~vfy \land wgz \land
\mathrm{Add}\,x\,v\,w$.
The advantage of this is that we have immediate access to all of the expressions
and subexpressions, and the proofs of the relationships between them.
This means that there is no need for congruence or monotonicity lemmas, and
correspondingly no need to explicitly describe the syntactic context in which
some algebraic manipulation is being applied and we rely less on the unifier.
Another advantage is that one can design relations so that pattern-matching
suggestively decomposes complex relationships.
For example, given $F : \mathscr M \rel \mathscr M'$ and
$G : \mathscr N \rel \mathscr N'$, we can define a relation
$F \oplus G : \mathscr M \oplus \mathscr N \rel \mathscr M' \oplus \mathscr N'$
pointwise, so that a proof of $(x, x')(F \oplus G)(y, y')$ is a proof of $xFy$
together with a proof of $x'Gy'$.
Pattern-matching on such a proof immediately gives us these constituent parts,
whereas proofs of the corresponding statement involving functions would require
using a lemma to get the parts.
There is a dual advantage when producing one of these proofs, where we can
introduce the canonical constructor (for pairs, in this example) rather than
having to find the appropriate lemma.

Relations also have several disadvantages, though I have found that for my use
case, these are outweighed by the advantages.
For example, automated algebraic solvers are better developed for function-based
algebraic expressions, and sometimes the fact that functions satisfy unitality
and associativity up to decidable judgemental equality means that some proofs
can be avoided.
The handling of compound expressions can also be a disadvantage in that it
necessitates lots of new variable names and obscures goal and context displays.
Finally, in predicative systems such as Agda, relations typically live in a
larger universe than the corresponding functions.
In practice, this means quantifying over an extra level variable for each
relation involved in general lemmas.

There are more relations than there are functions, so statements involving
relations are more general than the corresponding statements involving
functions.
However, one part of the development requires functions rather than relations,
so I impose functionality on relations after the fact.
The appropriate notion of functional relation I use is slightly different to the
standard one, in that I take account of the order on the codomain, and thus ask
for the \emph{largest} solution rather than the \emph{unique} solution.

\begin{definition}\label{def:functional-linear-relation}
  A linear relation $\gr\Psi$ between $\mathscr M$ and $\mathscr N$ is
  \emph{(right-to-left) functional} if, for every $\grQ \in \mathscr N$, there
  exists universally a $\grP \in \mathscr M$ such that $\grP\gr\Psi\grQ$.
  Universality means that, for all $\grPprime$ such that $\grPprime\gr\Psi\grQ$,
  we have $\grPprime \leq \grP$ (i.e.\ $\grP$ is the largest solution).
\end{definition}

In Agda code, $\gr\Psi$ becomes \AgdaBound{$\Psi$} and the fact that $\gr\Psi$
relates $\grP$ and $\grQ$ (in this section written $\grP\gr\Psi\grQ$) is
rendered as
\PsiDot{rel}\AgdaSpace{}\AgdaBound{P}\AgdaSpace{}\AgdaBound{Q}.
That $\gr\Psi$ respects the orders on its arguments is given by
\PsiDot{rel-$\leq_m$}, and the various linearity properties are given by
\PsiDot{rel-0$_m$}, \PsiDot{rel-+$_m$}, and \PsiDot{rel-*$_m$}.

\section{A layer of syntax is functorial}\label{sec:functorial}

A basic property of the universe of syntaxes
is that every syntax supports a functorial action on subterms, realised by a
function \AgdaFunction{map-s}.
Its type says that to map a function \AgdaBound{f}
over a layer of syntax, there must be a linear map \AgdaBound{$\Psi$} relating the
domain and codomain usage contexts, and \AgdaBound{f} should be usable
wherever the domain and codomain usage contexts are similarly related by
\AgdaBound{$\Psi$}.

\ExecuteMetaData[\Maptex]{map-s-type}

This generality is needed because usage contexts change between
a term and its immediate subterms---they are decomposed according to the bunched connectives used in the rules.
\AgdaBound{X} and \AgdaBound{Y} are \AgdaFunction{ExtOpenFam}s, with
\AgdaBound{$\Theta$} being the context extension for a subterm (i.e., the
variables newly bound in that subterm).
Unlike usage annotations, types in the contexts \AgdaBound{$\gamma$} and \AgdaBound{$\delta$}, and the conclusion types implicit here, are preserved throughout.
This is the essence of the usage annotation based approach---we use traditional techniques for variable binding, with the usage annotations layered on top.

The heart of \AgdaFunction{map-s} is \AgdaFunction{map-p}, which recursively
works through the structure \AgdaBound{ps} of premises of the rule applied,
acting on each subterm it finds.
Here, particularly in the clauses for \AgdaInductiveConstructor{`$\sep$} and
\AgdaInductiveConstructor{`$\cdot$}, we see why it is not enough for the
function on subterms to apply at usage contexts \AgdaBound{P} and \AgdaBound{Q}
--- rather, it also needs to apply at any similarly related \AgdaBound{P$'$}
and \AgdaBound{Q$'$}.
In the case of \AgdaInductiveConstructor{`$\sep$}, we have that
$\grP \leq \grP_M + \grP_N$, with \AgdaBound{M} and \AgdaBound{N} being
collections of subterms in usage contexts $\grP_M$ and $\grP_N$, respectively.
Linearity of \AgdaBound{$\Psi$} yields $\grQ_M$ and $\grQ_N$ such that
$\grQ \leq \grQ_M + \grQ_N$ and we use \AgdaFunction{map-p} recursively at
$(\grP_M, \grQ_M)$ and $(\grP_N, \grQ_N)$ on \AgdaBound{M} and \AgdaBound{N}.
The cases for \AgdaInductiveConstructor{`$\cdot$} and
\AgdaInductiveConstructor{`$I^*$} are similar, each using a different aspect
of linearity.
In contrast, the cases for \AgdaInductiveConstructor{`$\dot1$} and
\AgdaInductiveConstructor{`$\dot\times$}, which are the only constructors used in fully structural
systems, do not involve any changes in the usage contexts.

The linearity of relation \AgdaBound{$\Psi$} is given by fields
\AgdaField{rel-0$_m$}, \AgdaField{rel-+$_m$}, and \AgdaField{rel-*$_m$} (with
the subscript-m being a mnemonic for \emph{module}, as opposed to scalar).

\ExecuteMetaData[\Maptex]{map-p}

I have also extended \AgdaFunction{map-p} to handle the various
$\Box$-modalities described in \cref{sec:dup-lnl}.
The Agda code for this extension is not particularly readable, so I do not
include it in this document.
However, this extension is notable as the only part of the framework requiring
that the linear relation \AgdaBound{$\Psi$} be functional (i.e., total and
deterministic).

\section{The Kripke function space}\label{sec:kripke}

At this point I introduce a minor generalisation to
\AgdaFunction{OpenFam} and \AgdaFunction{ExtOpenFam} (as defined in
\cref{sec:terms-of-system}):
\AgdaBound{I}\AgdaSpace{}\AgdaFunction{---OpenFam} and
\AgdaBound{I}\AgdaSpace{}\AgdaFunction{---ExtOpenFam}.  We obtain the
definition of \AgdaBound{I}\AgdaSpace{}\AgdaFunction{---OpenFam} by
replacing the textual occurrence of \AgdaBound{Ty} by the parameter
\AgdaBound{I}, though there is still reference to the ambient \AgdaBound{Ty}
via \AgdaRecord{Ctx}.
The main value I am interested in \AgdaBound{I} taking, other than
\AgdaBound{Ty}, is \AgdaRecord{Ctx} --- for example, the type family of
$\V$-environments, for a given $\V$, is a
\AgdaRecord{Ctx}\AgdaSpace{}\AgdaFunction{---OpenFam}%
\AgdaSpace{}\AgdaSymbol{\_}.
I use this generality in the type of \AgdaFunction{extend} in the next section.

\ExecuteMetaData[\Syntaxtex]{dashOpenFam}

The definition
\AgdaFunction{Kripke}\AgdaSpace{}\AgdaBound{$\V$}\AgdaSpace{}\AgdaBound{$\C$}%
\AgdaSpace{}\AgdaBound{$\Delta$} is a kind
of function space that describes a \AgdaBound{$\C$}-value parametrised by
\AgdaBound{$\Delta$}-many additional \AgdaBound{$\V$}-values (all correctly
typed and usage-annotated).
It is used to describe how to go under binders in a
Higher-Order Abstract Syntax style: To go under a binder we must
provide semantic interpretations for all the additional variables.

% When going under binders during a recursion, the context will be extended by some $\Theta$. This means that the current environment must be extended with $\Theta$s-worth of $\V$s

% we need the ability to say that

% Kripke V C is given the extension \Theta

% In \cref{sec:terms}, we defined \AgdaFunction{Scope} to let a
% judgement-indexed family admit context extensions. However, a key
% component of our generic semantic traversal is to make use of the open
% family \AgdaBound{$\V$} of \emph{values}, which are the sort of thing
% we store in an environment.  The definition \AgdaFunction{Kripke}
% gives an alternative to \AgdaFunction{Scope} which interprets the
% newly bound variables via a requirement of $\V$-values rather than
% extra assumptions for the $\C$-computation.

\ExecuteMetaData[\Semanticstex]{Kripke}

\AgdaFunction{Wrap}
is a device that turns any type family into an equivalent type family
that is judgementally injective in its indices, which helps with
Agda's type inference.
It turns the type family into a parametrised
record with a single field \AgdaField{get} whose type is the type in
the body of the $\lambda$-abstraction.
For understanding the meaning of
\AgdaFunction{Kripke}, \AgdaFunction{Wrap} can be ignored.

If $\Delta$ is of the form $\gr{s_1}B_1, \ldots, \gr{s_n}B_n$, then
\ExecuteMetaData[\Snippetstex]{KripkeVCDGA}\ is equivalent to
\ExecuteMetaData[\Snippetstex]{KripkeExpanded}\ by Currying.  That is
to say, the Kripke function is expecting a value for each newly bound
variable, at the multiplicity of its annotation, together with the
resources supporting each of those values. We use the ``magic wand''
function space here to enforce the invariant that the freshly bound
variables have usage annotations that are added to the existing
variables, not shared with them. The use of the
\AgdaFunction{$\Box^r$} modality ensures that we can still use it in
the presence of additional variables introduced by weakening.

\AgdaFunction{Kripke} is functorial in the \AgdaBound{$\C$} argument,
as witnessed by the \AgdaFunction{mapK$\C$} function, which is essentially
post-composition:

\ExecuteMetaData[\Semanticstex]{mapKC}

% is exemplified by the following construct
% \AgdaFunction{reify}, where we weaken \AgdaBound{$\Gamma$} by a $\gr0$ed-out
% version of \AgdaBound{$\Delta$}.
% The \AgdaBound{$\Delta$} then gets filled in by the $\V$-values.

% \bob{Move this para}
% This means that \AgdaBound{A} in the definition of \AgdaFunction{Kripke} has
% type \AgdaBound{I}, rather than specifically \AgdaBound{Ty}.
% We use this generality later in \AgdaFunction{extend}, setting \AgdaBound{I}
% to \AgdaDatatype{Ctx}.

\section{Semantic traversal}\label{sec:traversal}

We can now state the data required to implement a traversal assigning
semantics to terms. For open families $\V$ and $\C$, interpreting
variables and terms respectively, we assume that $\V$ is renameable
(i.e., that $\sdtstile{}\V A \rightarrowtriangle \Box\plr{\sdtstile{}\V A}$ for
all $A$),
that $\V$ is embeddable in $\C$, and that we have an algebra for a
layer of syntax, where bound variables are handled using the Kripke
function space:

% The aim of this subsection is to give an alternative recursion principle for
% terms that incorporates some of the environment-handling seen in the
% implementations of renaming and substitution.
% The rest of this section assumes the following: a renameable open family
% \AgdaBound{$\V$} that embeds into the open family \AgdaBound{$\C$}, and an
% algebra for a layer of syntax at \AgdaBound{$\C$}.

\ExecuteMetaData[\Semanticstex]{Semantics}

%\ExecuteMetaData[\Semanticstex]{Comp}

We mutually define the action \AgdaFunction{semantics} and its lemma
\AgdaFunction{body}.
The purpose of \AgdaFunction{semantics} is to turn a term into a
\AgdaBound{$\C$}-value using a \AgdaBound{$\V$}-environment and the fields of
\AgdaRecord{Semantics}.
Meanwhile, \AgdaFunction{body} does a similar job, but also deals with
newly bound variables.
In particular, \AgdaFunction{body} takes a term in a context extended by
\AgdaBound{$\Theta$}, and produces a Kripke function from
\AgdaBound{$\V$}-values for \AgdaBound{$\Theta$} to \AgdaBound{$\C$}-values.

\ExecuteMetaData[\Semanticstex]{semantics-type}

To implement the new recursor \AgdaFunction{semantics}, we use the standard
recursor, which in one case gives us a variable \AgdaBound{v}, and in the other
gives us a structure of subterms \AgdaBound{M}, each of which is in an extended
context.
To deal with a variable \AgdaBound{v}, we look it
up in the environment \AgdaBound{$\rho$}, then use the
\AgdaField{$\sem{\text{var}}$} field to map the resulting
\AgdaBound{$\V$}-value to a \AgdaBound{$\C$}-value.
To deal with a structure of subterms \AgdaBound{M}, we use the functoriality of
the syntactic structure to consider each subterm separately.
On a subterm, we apply \AgdaFunction{body}, which amounts to a recursive call
to \AgdaFunction{semantics} with an extended environment.
Recall that \AgdaFunction{relocate} (\cref{thm:env-resize}) adjusts the
environment \AgdaBound{$\rho$} to work in the usage contexts of the subterms.

\ExecuteMetaData[\Semanticstex]{semantics}

For \AgdaFunction{body}, we are given a subterm \AgdaBound{M}, to
which we want to apply \AgdaFunction{semantics}.  To do so, we need an
extended version of the initial environment \AgdaBound{$\rho$}. We
express this as the generation of a Kripke function that produces the
extended environment given interpretations of the fresh variables. We
take \AgdaBound{$\rho$}, which is an environment covering
\AgdaBound{$\Delta$}, and \AgdaBound{$\sigma$}, which is an
environment covering \AgdaBound{$\Theta$}, and glue them together
using the inductive rules for generating environments, after having
renamed \AgdaBound{$\rho$} via \cref{thm:env-ren} to make it fit the
new context \AgdaBound{$\Gamma^+$} (intended to be
\ExecuteMetaData[\Snippetstex]{GT}):

\ExecuteMetaData[\Semanticstex]{extend}

% The best we can achieve without identity environments for \AgdaBound{$\V$} is
% a Kripke function returning an extended environment.
To define \AgdaFunction{body}, we use \AgdaFunction{mapK$\C$} to
post-compose the environment extension by the
\AgdaSymbol{$\lambda$}-function taking an extended environment and
acting with it on \AgdaBound{M}.

\ExecuteMetaData[\Semanticstex]{body}

% \todo{FIX} Under the assumption that \AgdaBound{$\V$} is renameable, we can decompose
% \cref{thm:lr-bind} as
% \AgdaFunction{reify}\AgdaSpace{}\AgdaOperator{\AgdaFunction{$\circ$}}%
% \AgdaSpace{}\AgdaFunction{extend}, with \AgdaFunction{extend} defined below.
% We can think of \AgdaFunction{extend} as our best effort to extend an
% environment by \AgdaBound{$\Theta$} without access to an identity environment
% at \AgdaBound{$\Theta$}.

\AgdaFunction{semantics} is the fundamental lemma of the framework.
With it proven, I move onto corollaries and specific applications.

\section{Reifying the Kripke function space}\label{sec:reify}

A result I will use throughout the rest of this thesis is \emph{reification}.
When we have an index-preserving mapping from usage-checked variables to
$\V$-environments, we can construct environments of the form
$\Gamma \env\V \Gamma$ (identity environments) for all $\Gamma$.
This lets us write the \AgdaFunction{reify} function, which  simplifies our
obligations in giving a \AgdaRecord{Semantics} by coercing
\AgdaFunction{Kripke} functions into just
\AgdaBound{$\C$}-values in an extended context.

\begin{lemma}[\AgdaFunction{reify}]\label{thm:reify}
  If $\V$ is an open family such that there is a function
  $v : {\sqni} \rightarrowtriangle \V$, we get a function of type
  $\mathrm{Kripke}\,\V\,\C \rightarrowtriangle \mathrm{Scope}\,\C$ for any $\C$.
\end{lemma}
\begin{proof}
  Let $b : \mathrm{Kripke}\,\V\,\C\,\Delta\,\Gamma\,A$.
  That is, $b$ is a Kripke function yielding $\C$-computations
  We want to apply $b$ so as to get a $\C\,\plr{\Gamma, \Delta}\,A$.
  Let $\grP\gamma = \Gamma$ and $\grQ\delta = \Delta$.
  The $\Box^r$ in the type of $b$ allows us to reverse-rename $\Gamma$ to
  $\Gamma, \gr0\delta$.
  Then we give the $\wand$-function an argument in context
  $\gr0\gamma, \Delta$, noting that
  $\plr{\Gamma, \gr0\delta} + \plr{\gr0\gamma, \Delta} = \plr{\Gamma, \Delta}$,
  as we wanted from the result.
  The argument needs type $\gr0\gamma, \Delta \env\V \Delta$.
  We produce this via \cref{thm:env-postren} from an environment
  $\rho : \gr0\gamma, \Delta \env\V \gr0\gamma, \Delta$ created using $v$
  and a renaming which is the complement to that used on $\Box^r$.
\end{proof}

All of the $\V$s used in examples in this paper support identity environments.
However, \citet[p.~27]{AACMM21} give some important examples that do not
support identity environments, and thus cannot use \AgdaFunction{reify}
(\cref{thm:reify}).
The feature that causes the lack of support for identity environments is that
a semantics can make use of the fact that only variables of particular kinds
are bound by the syntax.
In the examples of \citeauthor{AACMM21}, a bidirectionally typed language only
binds variables that synthesise their type, as opposed to those whose type is
checked.
The semantics of type-checking and elaboration rely on variables synthesising
their type, so \AgdaBound{$\V$} is chosen to cover only those variables.
Instead of using \AgdaFunction{reify}, we must observe that each syntactic form
only binds such synthesising variables.
Similar phenomena would appear in, say, a call-by-value language where
variables are values (not computations), or a polarised language where
variables always have a polarity matching their type.

\section{Renaming and substitution}\label{sec:kit-to-sem}

The final completely syntax-generic result I present is simultaneous
substitution.
I derive this as I did in the simply typed case in \cref{sec:gen-sem}:
I first show that a syntactic kit can be turned into a semantics, and then by
instantiating the notion of kit for, in turn, renaming and substitution, the
general semantic traversal gives the result we want.

The notion of \AgdaRecord{Kit} is essentially the same as in the simply typed
case, once we allow for changes to the basic definitions of variables, terms,
and environments (in particular, renamings).

\ExecuteMetaData[\Syntactictex]{Kit}

The first two fields of \AgdaRecord{Semantics} derive directly from fields of
\AgdaRecord{Kit}.
Meanwhile, to handle term constructors, we first \AgdaFunction{reify} to get a
collection of traversed subterms, and then use \AgdaInductiveConstructor{`con}
to assemble these subterms into a similarly shaped syntactic form as we started
with.
The \AgdaField{vr} field is used implicitly in \AgdaFunction{reify}, as it is
used to show that $\V$-identity environments exist.

\ExecuteMetaData[\Syntactictex]{kit-to-sem}

The action of a syntactic traversal on logical rules is basically fixed: we
preserve the logical rule and extend the environment with any newly bound
variables according to \AgdaField{vr}.
Meanwhile, the action on variables is relatively unconstrained: we look up the
variable in the environment to get a $\V$-value, then transform that $\V$-value
into a term using \AgdaField{tm}.

The idea of simultaneous renaming is that variables replace variables, whereas
with simultaneous substitution, terms replace variables.
This translates to environments for renaming containing $\sqni$-values
(variables), and environments for substitution containing $\vdash$-values
(terms).

%To implement renaming and substitution for terms, we now just implement
%syntactic kits for variables and terms, respectively.

\ExecuteMetaData[\Syntactictex]{Ren-Kit}

Notice that \AgdaFunction{ren\textasciicircum$\vdash$}, witnessing the fact
that terms are renameable, is a corollary of \AgdaFunction{Ren-Kit}.

\ExecuteMetaData[\Syntactictex]{Sub-Kit}


\section{Example semantics}\label{sec:example-semantics}

TODO: introductory paragraph

\bob{Reification dumped here for now}
When \AgdaBound{$\V$} supports identity environments (as per \cref{thm:env-id}),
we can simplify our obligations in giving a \AgdaRecord{Semanitics} by coercing
\AgdaFunction{Kripke} functions into just
\AgdaBound{$\C$}-values in an extended context.

%To show this, we instantiate the Kripke function with the renaming
%$\swarrow^r : \Gamma, \gr0\delta \env\sqni \Gamma$ to extend the scope, and
%pass it an argument environment
%$\plr{\id^\V; \searrow^r} : \gr0\gamma, \Delta \env\V \Delta$ to fill in the
%extended part.
%Post-composition of a renaming onto an environment comes from
%\cref{thm:env-postren}.

%\ExecuteMetaData[\Syntactictex]{reify}

\begin{lemma}[\AgdaFunction{reify}]\label{thm:reify}
  If $\V$ is an open family such that there is a function
  $v : \forallb{{\sqni} \dotto \V}$, we get a function of type
  $\forallb{\mathrm{Kripke}\,\V\,\C \dotto \mathrm{Scope}\,\C}$ for any $\C$.
\end{lemma}
\begin{proof}
  Let $b : \mathrm{Kripke}\,\V\,\C\,\Delta\,\Gamma\,A$.
  That is, $b$ is a Kripke function yielding $\C$-computations
  We want to apply $b$ so as to get a $\C\,\plr{\Gamma, \Delta}\,A$.
  Let $\grP\gamma = \Gamma$ and $\grQ\delta = \Delta$.
  The $\Box^r$ in the type of $b$ allows us to reverse-rename $\Gamma$ to
  $\Gamma, \gr0\delta$.
  Then we give the $\wand$-function an argument in context
  $\gr0\gamma, \Delta$, noting that
  $\plr{\Gamma, \gr0\delta} + \plr{\gr0\gamma, \Delta} = \plr{\Gamma, \Delta}$,
  as we wanted from the result.
  The argument needs type $\gr0\gamma, \Delta \env\V \Delta$.
  We produce this through the composition
  $\gr0\gamma, \Delta \env\V \gr0\gamma, \Delta \env{\sqni} \Delta$,
  using the instance of composition mentioned in \cref{thm:env-postren}.
  The $\V$-environment follows from $v$ and \cref{thm:env-id}, while the
  renaming is the complement to that used on $\Box^r$.
\end{proof}

All of the $\V$s used in examples in this paper support identity environments.
However, \citet[p.~27]{AACMM21} give some important examples that do not
support identity environments, and thus cannot use \AgdaFunction{reify}
(\cref{thm:reify}).
The feature that causes the lack of support for identity environments is that
a semantics can make use of the fact that only variables of particular kinds
are bound by the syntax.
In the examples of \citeauthor{AACMM21}, a bidirectionally typed language only
binds variables that synthesise their type, as opposed to those whose type is
checked.
The semantics of type-checking and elaboration rely on variables synthesising
their type, so \AgdaBound{$\V$} is chosen to cover only those variables.
Instead of using \AgdaFunction{reify}, we must observe that each syntactic form
only binds such synthesising variables.
Similar phenomena would appear in, say, a call-by-value language where
variables are values (not computations), or a polarised language where
variables always have a polarity matching their type.


\subsection{Renaming and substitution}\label{sec:kits}

In an unpublished note, \citet{McBride05} gives a parametrised traversal
yielding homomorphisms of syntax, the canonical examples of which are
simultaneous renaming and simultaneous substitution.
The parameters are collected in the record \AgdaRecord{Kit}.
We make a minor change to the original presentation, where instead of our
\AgdaField{ren\textasciicircum{}$\V$} field, \citeauthor{McBride05} has the
field \AgdaField{wk} allowing only context extensions.
As for the other two fields, \AgdaField{vr} allows us to map variables to
$\V$-values, so as to put newly bound variables in environments; and
\AgdaField{tm} allows us to extract terms from $\V$-values, as required when
we use the environment to handle a free variable.

\ExecuteMetaData[\Syntactictex]{Kit}

Where \citeauthor{McBride05} gave the traversal explicitly, we go via our
generic semantic traversal.
The first two fields of \AgdaRecord{Semantics} derive directly from fields of
\AgdaRecord{Kit}.
Meanwhile, to handle term constructors, we first \AgdaFunction{reify} to get a
collection of traversed subterms, and then use \AgdaInductiveConstructor{`con}
to assemble these subterms into a similarly shaped syntactic form as we started
with.
The \AgdaField{vr} field is used implicitly in \AgdaFunction{reify}, as it is
used to show that $\V$-identity environments exist.

\ExecuteMetaData[\Syntactictex]{kit-to-sem}

The action of a syntactic traversal on logical rules is basically fixed: we
preserve the logical rule and extend the environment with any newly bound
variables according to \AgdaField{vr}.
Meanwhile, the action on variables is relatively unconstrained: we look up the
variable in the environment to get a $\V$-value, then transform that $\V$-value
into a term using \AgdaField{tm}.

The idea of simultaneous renaming is that variables replace variables, whereas
with simultaneous substitution, terms replace variables.
This translates to environments for renaming containing $\sqni$-values
(variables), and environments for substitution containing $\vdash$-values
(terms).

%To implement renaming and substitution for terms, we now just implement
%syntactic kits for variables and terms, respectively.

\ExecuteMetaData[\Syntactictex]{Ren-Kit}

Notice that \AgdaFunction{ren\textasciicircum$\vdash$}, witnessing the fact
that terms are renameable, is a corollary of \AgdaFunction{Ren-Kit}.

\ExecuteMetaData[\Syntactictex]{Sub-Kit}

\subsection{A usage elaborator}\label{sec:usage-elaborator}

Using the constructs we have seen so far, producing example terms soon becomes
extremely tedious.
We can achieve some abbreviation by using pattern synonyms to wrap around
\AgdaInductiveConstructor{`con} expressions, but we still have to
produce essentially bespoke proofs whenever we use a usage-sensitive part of the
syntax.
The size of each of these proofs is roughly proportional to the number of free
variables, so the amount of proof we have to write grows roughly quadratically
with the size of terms.
An additional factor, which we can't see on paper, is that type checking time
for these proofs soon becomes prohibitive to interactive development.

Our aim in this subsection is to automate usage constraint proofs, making terms
both easier to write and more performant to check.
We invoke the automation by writing terms in a syntax where usage constraints
have been trivialised, and then use a semantic traversal over the simplified
syntax to try to produce a fully elaborated term in the original syntax.
We write the automation in a way that is generic in the syntax description, thus
avoiding repetition and facilitating the prototyping of new type systems.

The type of syntax descriptions depends on the type of usage annotations because
of variable binding.
For example, in the $\oc{\gr r}$-E rule of \cref{fig:lr-comb}, the right
premise binds a new variable with annotation $\gr r$, where $\gr r$ is drawn
from the ambient posemiring.
The scaling combinator also makes direct reference to the posemiring.
To produce a simplified syntax description, where usage constraints are
trivialised, we set the ambient posemiring to the 1-element $\mathbf0$
posemiring.
In contrast to syntax descriptions, even though types can contain usage
annotations, the type of types does not depend on the type of usage annotations.
This means that, in our simplified syntax, terms have types from the original
system even though variables have trivial usage annotations.
We define the $\mathbf0$ posemiring as follows, being careful to use the
0-field record type \AgdaRecord{$\top$} so that everything algebraic gets
solved by Agda's $\eta$-laws.
Indeed, in this very definition, all of the semiring operations and laws are
canonically inferred.

\ExecuteMetaData[\UsageChecktex]{0-poSemiring}

The elaboration process is monadic.
In particular, we use the \AgdaDatatype{List}/non-determinism monad to give
\emph{all} of the possible annotation choices on the free variables of a term.
We believe the commitment to multiple solutions is inherent when the syntax
contains \AgdaInductiveConstructor{`$\dot1$}.
For example, in the intermediate stages of elaborating
$\plr{\vdash \lambda x.~\plr{*,*}} : A \multimap \top \otimes \top$ with a
usage counting posemiring (assuming reasonable rules for $\top$ and $\otimes$),
it is unclear whether to use the variable $x$ in the left $*$ or the right $*$.
This uncertainty should be reflected in the final result.

The non-deterministic choices we make during elaboration are enumerated by
the fields of \AgdaRecord{NonDetInverses}.
These choices are driven by the typing rules and a candidate usage vector for
the conclusion.
For example, \AgdaField{+$^{-1}$}\AgdaSpace{}\AgdaBound{r} is needed when we
encounter a \AgdaInductiveConstructor{`$\sep$} in the syntax and the candidate
usage annotation we are considering is \AgdaBound{r}.
Then, \AgdaField{+$^{-1}$}\AgdaSpace{}\AgdaBound{r} is a list of pairs of
annotations \AgdaBound{p} and \AgdaBound{q} that \AgdaBound{r} can split into,
together with a proof of the splitting.
For \AgdaField{0\#$^{-1}$} and \AgdaField{1\#$^{-1}$}, inverses to constants,
we are given the candidate \AgdaBound{r} and typically return an empty list if
the constraint cannot be satisfied, or a singleton list containing a proof.
\AgdaField{*$^{-1}$} is used when we encounter scaling, in which case we know
both the scaling factor \AgdaBound{r} (from the syntax description) and the
candidate \AgdaBound{q}.
These inverse operations combine monadically (in fact, applicatively) to give
inverses to the vector operations of zero, addition, scaling, and basis.

\ExecuteMetaData[\UsageChecktex]{NonDetInverses}

We choose the \AgdaBound{$\V$} of our semantics to be (unannotated) variables.
For the \AgdaBound{$\C$}, we consider \emph{functions} from candidate usage
vectors \AgdaBound{R} to the list of elaborated derivations with usage
annotations given by \AgdaBound{R}.
The protocol this encodes is that the user will provide an unannotated term
together with a candidate usage context \AgdaBound{R}, and usage elaboration
returns a list of possible ways the term could be annotated such that the
conclusion has usage context \AgdaBound{R}.
The module name \AgdaModule{U} refers to the fact that we are taking the
ambient posemiring to be $\mathbf0$ in \AgdaFunction{OpenFam}.
The effect on \AgdaFunction{OpenFam} is that the usage annotations of any
contexts we consider are uninformative (hence the \AgdaSymbol{\_} on the left).

\ExecuteMetaData[\UsageChecktex]{C}

To traverse the unannotated terms, we produce a \AgdaRecord{Semantics} over the
unannotated system \AgdaFunction{uSystem}\AgdaSpace{}\AgdaBound{sys}.
To write it, we make use of idiom brackets
\AgdaSymbol{(|}\AgdaSpace{}\AgdaSymbol{$\ldots$}\AgdaSpace{}\AgdaSymbol{|)},
which have the effect of replacing top-level spines of applications by
(\AgdaDatatype{List}-)applicative applications.
Field by field, we already know that variables are renameable.
To interpret a variable, we consider all the possible proofs that such a
variable could be well annotated, and package them up as a variable term via
the applicative machinery.
Finally, for compound terms, we first reify the unannotated subterms, and then
combine the subterms via a \AgdaFunction{lemma}.

\ExecuteMetaData[\UsageChecktex]{elab-sem}

The \AgdaFunction{lemma} essentially goes through the shape of the premises,
combining the collections of subterms in the natural way.
For example, at each
\AgdaInductiveConstructor{\AgdaUnderscore{}$\dottimes$\AgdaUnderscore{}},
we take the Cartesian product of the possibilities of each half, and at each
\AgdaInductiveConstructor{\AgdaUnderscore{}$\sep$\AgdaUnderscore{}},
we non-deterministically split the usage annotations coming in, and then take
the Cartesian product.
When it comes to newly bound variables, the \emph{syntax description} tells us
their annotations, so there is no further non-determinism introduced here.

\ExecuteMetaData[\UsageChecktex]{lemma-type}

To actually use \AgdaFunction{elab-sem} on terms, we take the associated
\AgdaFunction{semantics} and pass it the identity environment (an identity
\emph{renaming} in this case, because $\V$ is a family of variables).
We use \AgdaFunction{elab-unique}, which further
checks statically that exactly one derivation is returned.
The candidate usage vector \AgdaBound{R} will be \AgdaFunction{[]} for closed
terms, and otherwise we have to supply the intended usage annotations.

The example below is of a term needed to show that, in the
$\{\gr0, \gr1, \gr\omega\}$ (linearity) posemiring of \cref{thm:linearity},
$\oc\gr\omega$ forms a comonad.
We have instantiated the usage elaborator so that:
\AgdaField{0\#$^{-1}$} is a singleton on $\gr0$ and $\gr\omega$, and empty on
$\gr1$;
\AgdaField{1\#$^{-1}$} is a singleton on $\gr1$ and $\gr\omega$, and empty on
$\gr0$;
\AgdaField{+$^{-1}$} gives $\gr0 \mapsto [(\gr0,\gr0)]$,
$\gr1 \mapsto [(\gr0,\gr1),(\gr1,\gr0)]$, and
$\gr\omega \mapsto [(\gr\omega,\gr\omega)]$; and
\AgdaField{*$^{-1}$} gives $(\gr\omega, \gr0) \mapsto [\gr0]$,
$(\gr\omega, \gr1) \mapsto []$, and
$(\gr\omega, \gr\omega) \mapsto [\gr\omega]$
(omitting $(\gr0, \_)$ and $(\gr1, \_)$ cases for brevity).
Note that we do not consider splitting $\gr\omega$ up as, say,
$\gr1 + \gr\omega$, because this splitting would introduce more
non-determinism but not allow any more terms to be typed.
As such, the only non-determinism comes when we have variables annotated
$\gr1$ and need to do an additive split, like when we apply the
\AgdaInductiveConstructor{!E} rule below.
At this point, the variable can become either $\gr0$-annotated in the left
subterm and $\gr1$-annotated on the right, or vice-versa.
We will find that, because the left subterm wants to use that variable, the
former choice will be rejected.
The function \AgdaFunction{var\#} is a convenience for converting statically
known natural numbers, representing de Bruijn \emph{levels}, into variable
terms.

\ExecuteMetaData[\PaperExamplestex]{cojoin}

\subsection{A denotational semantics}

To justify the name \emph{semantics}, we give an example traversal that is a
denotational semantics in the usual sense.
The semantics we take is a refinement of that of \citet{AbelBernardy2020},
which gives a way to extract parametricity theorems from substructurally typed
programs.
Example theorems are that all linear terms act as permutations on some fixed
set of resources, and all monotonically typed terms are really monotonic in the
way the typing suggests they are.

To abbreviate this section, we use a simplified syntax compared to \name{}.
We allow for an arbitrary family of base types \AgdaBound{BaseTy}, and a single
type former \mbox{\ExecuteMetaData[\WReltex]{rAToB}}, equivalent to
\mbox{\ExecuteMetaData[\PaperExamplestex]{BangrAToB}} from the earlier system.

\ExecuteMetaData[\WReltex]{Ty}

In the term syntax, $\lambda$-abstraction now binds a variable with annotation
\AgdaBound{r}, while application needs to scale its argument by \AgdaBound{r}
(both in accordance with the function type they are acting on).

\ExecuteMetaData[\WReltex]{AnnArr}

As a running example, we take the usage annotations to be the 4-element
variance posemiring (\cref{thm:variance}).
We establish the property that all terms are monotonic in their free variables.
This monotonicity can be covariant or contravariant (or neither or both)
depending on the annotation of each free variable.
This provides an additional example to those of \citeauthor{AbelBernardy2020}.

We will take semantics of this system into
\emph{world-indexed relations}~\cite{AbelBernardy2020,context-constrained}.
A world-indexed relation over a poset of worlds \AgdaBound{W} is a set over
which
we have a \AgdaBound{W}-indexed binary relation satisfying a presheaf-like
property with respect to the order on \AgdaBound{W}.

\ExecuteMetaData[\WReltex]{WRel}

\begin{example}
  When \AgdaBound{W} is the 1-element set, a world-indexed relation is just a
  set equipped with a binary relation.
\end{example}

Morphisms between world-indexed relations \AgdaBound{R} and \AgdaBound{S}
consist of a mapping between the underlying sets such that that mapping
preserves relatedness from \AgdaBound{R} to \AgdaBound{S}.

\ExecuteMetaData[\WReltex]{WRelMor}

\todo{Define big intersection.}
When the poset of worlds forms a (relational) commutative monoid, such
world-indexed relations support a symmetric monoidal closed structure, with
objects denoted \AgdaFunction{I$^R$},
\AgdaFunction{\AgdaUnderscore{}$\otimes^R$\AgdaUnderscore{}}, and
\AgdaFunction{\AgdaUnderscore{}$\multimap^R$\AgdaUnderscore{}},.
These reuse the bunched connectives \AgdaRecord{$I^*$}, \AgdaRecord{$\sep$}, and
\AgdaRecord{$\wand$}, now over worlds rather than contexts.

%\ExecuteMetaData[\WReltex]{IR}
%\ExecuteMetaData[\WReltex]{tensorR}
%\ExecuteMetaData[\WReltex]{lollyR}

The final piece of sematics we need is a \emph{bang} operator.
We allow the
semantic \emph{bang} to be an arbitrary annotation-indexed functor on
world-indexed relations.
This functor must respect all of the structure on the indices, making it a
graded comonad over multiplication, as well as being lax monoidal at any
particular index \AgdaBound{r}.

\ExecuteMetaData[\WReltex]{Bang}

\begin{example}
  With \AgdaBound{W} being the 1-element set and annotations coming from the
  variance semiring, we can define the following \emph{bang}.
  It is always the identity on the set component, while the relation component
  consists of flipping the relation for contravariance and taking conjunctions
  to achieve both covariance and contravariance.
  When we want neither covariance nor contravariance, we use the always true
  predicate on worlds \AgdaFunction{$\dot1$}.

  \ExecuteMetaData[\Monotonicitytex]{BangR}
\end{example}

To associate semantics to syntax, we start as standard by associating
world-indexed relations to types.
We also extend the semantics of types to contexts, using \AgdaFunction{I$^R$},
\AgdaFunction{$\otimes^R$}, and \AgdaField{!$^R$} to interpret the empty
context, concatenation, and usage annotations on singletons, respectively.

\ExecuteMetaData[\WReltex]{sem}

The semantics of a term is then to be a morphism from the interpretation of the
context to the interpretation of the term's type.

\ExecuteMetaData[\WReltex]{sem-vdash}

Variables are given semantics by \AgdaFunction{lookup$^R$} (definition omitted).

\ExecuteMetaData[\WReltex]{lookupR-type}

Now, we give a \AgdaRecord{Semantics}.
The choice of \AgdaBound{$\V$} as
\AgdaRecord{\AgdaUnderscore{}$\sqni$\AgdaUnderscore{}} is somewhat arbitrary,
given that a standard denotational semantics would not use intermediate
environments in the same sense as renaming and substitution, but allows us to
reuse the standard facts that variables support renaming and identity
environments.
With this choice of \AgdaBound{$\V$} and \AgdaBound{$\C$}, we interpret
environment entries by \AgdaFunction{lookup$^R$}.
Meanwhile, for the logical rules, we ignore environments by using
\AgdaFunction{reify} to just deal with morphisms in an extended context.
As such, $\lambda$-abstractions are easy to interpret, while applications
require some massaging to remove the extension by an empty context, followed by
some plumbing to split the interpretation of the context according to the usage
constraints and feed the interpretation of the argument \AgdaBound{n$'$} into
the interpretation of the function \AgdaBound{m$'$}.

\ExecuteMetaData[\WReltex]{Wrel}

In order to map open terms to interpretations, we take the action of the
semantics and give the identity renaming as the starting environment.

\ExecuteMetaData[\WReltex]{wrel}

\begin{example}\label{thm:minus}
  We can make a subtraction function from primitive addition and negation on
  integers.
  Subtraction is covariant in its first argument and contravariant in its
  second argument.
  We give the definition in pseudocode, though it is also amenable to the
  usage elaborator of \cref{sec:usage-elaborator}, suitably instantiated.

  \begin{align*}
    &{\sim\sim}p :
      {\uparrow\uparrow}\mathbb Z \multimap
      {\uparrow\uparrow}\mathbb Z \multimap \mathbb Z,
      {\sim\sim}n : {\downarrow\downarrow}\mathbb Z \multimap \mathbb Z
      \vdash \mathnormal{minus} :
      {\uparrow\uparrow}\mathbb Z \multimap
      {\downarrow\downarrow}\mathbb Z \multimap
      \mathbb Z
    \\
    &\mathnormal{minus} \coloneqq \lambda x.~\lambda y.~p\,x\,(n\,y)
  \end{align*}

  After feeding in Agda's addition and negation functions as the
  interpretations of the free variables (noting that they are both monotonic
  in the required way), we get the following free theorem.

  \ExecuteMetaData[\Monotonicitytex]{thm-type}

  %We observe that the set component of this term's semantics is just the
  %expected Agda function when the two free variables are given appropriate
  %interpretations.

  %\ExecuteMetaData[\Monotonicitytex]{minus-set}

  %Furthermore, the relational component of the semantics yields the free
  %theorem that the Agda subtraction so defined is monotonic in the expected way.
  %This relies on library proofs that addition and negation are suitably
  %monotonic.

  %\ExecuteMetaData[\Monotonicitytex]{thm}
\end{example}

