\documentclass[fleqn]{beamer}

\usetheme{Rochester}
\setbeamertemplate{navigation symbols}{}

\usepackage[style=alphabetic,maxalphanames=6]{biblatex}
\bibliography{quantitative}

\usepackage{amsmath}
\usepackage{amssymb}
\usepackage{cmll}
\usepackage{ebproof}
\usepackage{ebproof-rules}
\usepackage{mathpartir}
\usepackage{mathrsfs}
\usepackage{mathtools}
%\usepackage{natbib}
\usepackage{todonotes}
\usepackage{turnstile}

\usepackage{tikz}
\usetikzlibrary{tikzmark,fit}

\definecolor{use}{HTML}{008000}
\newcommand\gr[1]{{\color{use}#1}}
\newcommand\grctx[1]{\gr{\mathcal{#1}}}
\newcommand\grP{\grctx P}
\newcommand\grQ{\grctx Q}
\newcommand\grR{\grctx R}
\newcommand\grPprime{\grP\gr'}
\newcommand\grQprime{\grQ\gr'}
\newcommand\grRprime{\grR\gr'}
\newcommand\name{\ensuremath{\lambda\grR}}
\newcommand\grctxsub[2]{\grctx{#1}_{\gr{#2}}}
\newcommand\grPe{\grctxsub P e}
\newcommand\grPf{\grctxsub P f}
\newcommand\grPl{\grctxsub P l}
\newcommand\grPr{\grctxsub P r}
\newcommand\grQe{\grctxsub Q e}
\newcommand\grQf{\grctxsub Q f}
\newcommand\grQl{\grctxsub Q l}
\newcommand\grQr{\grctxsub Q r}
\newcommand\grRl{\grctxsub R l}
\newcommand\grRr{\grctxsub R r}
\newcommand\ps{\mathit{ps}}
\newcommand\qs{\mathit{qs}}
\newcommand\rs{\mathit{rs}}
\newcommand\dotto{\mathrel{\dot\to}}
\newcommand\dotlr{\mathrel{\dot\leftrightarrow}}
\newcommand\dottimes{\mathbin{\dot\times}}
%\newcommand\dotplus{\mathbin{\dot+}}
\newcommand\wand{\mathrel{\mathord{-}\hspace{-0.75ex}*}}
\newcommand\sep{\mathbin{*}}
\newcommand\Boxzp{\Box^{0{+}}}
\newcommand\Boxzpt{\Box^{0{+}{*}}}
\providecommand\U{}
\renewcommand\U{\mathcal U}
\newcommand\V{\mathcal V}
\newcommand\W{\mathcal W}
\providecommand\C{}
\renewcommand\C{\mathcal C}
\newcommand\sqin{\mathrel{\mathrlap{\sqsubset}{\mathord{-}}}}
\newcommand\sqni{\mathrel{\mathrlap{\sqsupset}{\mathord{-}}}}
\newcommand\Ann{\mathscr R}

\renewcommand\land{~\wedge~}
\renewcommand\lor{~\vee~}
\newcommand\rel{\mathrel{\mathord{\to}\hspace{-2.25ex}+}}

\DeclareMathOperator\Set{Set}
\DeclareMathOperator\obj{Obj}
\DeclareMathOperator\List{List}
\let\hom\relax
\DeclareMathOperator\hom{Hom}
\DeclareMathOperator\id{id}
\DeclareMathOperator\sub{Sub}

\newcommand\env[1]{\stackrel{#1}\Longrightarrow}

\usepackage{mleftright}
\newcommand\lr[3]{\mleft#1{#2}\mright#3}
\newcommand\sem[1]{\lr\llbracket{#1}\rrbracket}
\newcommand\size[1]{\lr\lvert{#1}\rvert}
\newcommand\plr[1]{\lr({#1})}
\newcommand\blr[1]{\lr[{#1}]}
\newcommand\forallb[1]{\forall\blr{~#1~}}
\newcommand\alr[1]{\lr\langle{#1}\rangle}
\newcommand\bra[1]{\lr\langle{#1}\rvert}
\newcommand\ket[1]{\lr\lvert{#1}\rangle}

\newcommand\leO{\;{\leq}0}
\newcommand\leI{\;{\leq}1}

\DeclareMathOperator{\lin}{lin}
\DeclareMathOperator{\intu}{int}
\newcommand\lnl{\ensuremath{\mathrm{L/nL}}}
\newcommand\vdashL{\mathrel{\vdash_{\mathcal L}}}
\newcommand\vdashC{\mathrel{\vdash_{\mathcal C}}}

\ebproofnewstyle{comb}{separation=0.75em}
\ebproofset{right label template=\TirName{\inserttext}}

\newenvironment{eqns}{\begin{array}{r@{\hspace{0.3em}}c@{\hspace{0.3em}}l}}{\end{array}}

\newcommand\PsiDot[1]{%
  \AgdaBound{$\Psi$}\AgdaSpace{}\AgdaSymbol{.}\AgdaField{#1}}

\newcommand\qto{\mathbin{`\!\to}}

\newcommand\Lock{\text{\faLock}}

\newcommand\oiw{\mathrm{\gr{01\upomega}}}

\newcommand\colour{%
{\color{AgdaField}co}{\color{AgdaFunction}lo}{\color{AgdaDatatype}ur}%
}


\newcommand\Rel{\mathrm{Rel}}

\title{Beyond semirings?}
\author{James Wood}
\institute{University of Strathclyde \and Huawei Technologies R\&D UK}
\date{Meeting on Graded Types, 17th June 2022}

\begin{document}

\frame{\titlepage}

\begin{frame}{Introduction}
\end{frame}

\begin{frame}{Monoidal categories}
\[
  \begin{tikzpicture}[baseline]
    \path
    (-1,1) node(0) {0}
    (1,2) node(x) {}
    (0,0) node(*) {*}
    (0,-1) node(res) {}
    ;

    \draw (0) -- (*);
    \draw (x) to[out=270,in=45] (*);
    \draw (*) -- (res);
  \end{tikzpicture}
  =\quad
  \begin{tikzpicture}[baseline]
    \path
    (0,0) node(0) {0}
    (0,2) node(x) {}
    (0,-1) node(res) {}
    (0,1) node[circle,draw](del) {}
    ;

    \draw (0) -- (res);
    \draw (x) -- (del);
  \end{tikzpicture}
  \quad=
  \begin{tikzpicture}[baseline]
    \path
    (1,1) node(0) {0}
    (-1,2) node(x) {}
    (0,0) node(*) {*}
    (0,-1) node(res) {}
    ;

    \draw (0) -- (*);
    \draw (x) to[out=270,in=135] (*);
    \draw (*) -- (res);
  \end{tikzpicture}
\]
\end{frame}

\begin{frame}{Examples of monoidal categories}
  Examples of $(\mathcal C, I, \otimes)$:
  \begin{itemize}
    \item $(\Set, 1, \times)$ --- Cartesian structure of $Set$
    \item $(\Rel, 1, \times)$ --- \emph{Not} the Cartesian structure of $\Rel$
    \item $(\mathrm{CMon}, \mathbb N, \otimes_{\mathbb N})$ ---
      commutative monoids under their tensor product
    \item $(\Set_{\mathrm{part}}, 1, \times)$
  \end{itemize}
\end{frame}

\begin{frame}{Monoids in a monoidal category}
\end{frame}

%\begin{frame}{Monoids are determined by their multiplication}
%\end{frame}

\begin{frame}{There's no such thing as a partial semiring}
  Consider a comonoid
  $(M, \eta : M \rightharpoonup 1, \mu : M \rightharpoonup M \times M)$ in
  $\Set_{\mathrm{part}}$.

  We have:
  \missingfigure{Unit laws}

  By determinism of the comultiplication, we must have $y = z = x$.

  Furthermore, we see that every $x$ is in the counit.
\end{frame}

\begin{frame}{Monoid with actions}
  \begin{block}{Definition: monoid with actions}
    A commutative monoid $M$ together with a set of monoid endomorphisms
    (\emph{actions}) $A$ closed under identity and composition.
  \end{block}
  \only<1>{%
    Expanding that, we have, for any action $\rho$ and elements $x$ and $y$:
    \begin{itemize}
      \item $\rho(0) = 0$
      \item $\rho(x + y) = \rho(x) + \rho(y)$
    \end{itemize}
    Note that this is not the same as a module.
  }
  \begin{block}{Example: linearity in $\Rel$}<2>
    Let $M = (\{0,1,\omega\}, \{0,\omega\}, +)$, where:
    \begin{itemize}
      \item $0 + 0 = 0$
      \item $0 + 1 = 1$
      \item $1 + 0 = 1$
      \item $\omega + \omega = \omega$
    \end{itemize}
    Let $A = \{1, \omega\}$, where:
    \begin{itemize}
      \item $1(x) = x$
      \item $\omega(x) = x$ when $x \neq 1$
    \end{itemize}
  \end{block}
\end{frame}

\begin{frame}{Conclusions}
\end{frame}

\end{document}
