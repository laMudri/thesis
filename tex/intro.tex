\chapter{Introduction}

This thesis advances the frontier of programming languages work that can be done
in a proof assistant.
I will elaborate on both of these aspects in the following paragraphs.

The main aim of programming language research is to find common patterns in
computer programs and represent these patterns via programming language
features.
As a basic example, consider functions in, say, a low level language like C, in
contrast to what these functions are compiled to in assembly language/machine
code.
In order to support function calls, the C runtime system manages a
\emph{call stack}.
A function call is compiled to a sequence of instructions which store the return
address and the values of the arguments on the stack, move the stack pointer,
and jump to the compiled code of the function.
A return from a function then places the return value into the appropriate
register, moves the stack pointer back, and jumps back to the return address.
Advances in programming language techniques are evaluated by the usefulness of
behaviours captured and goodness of the mathematical properties of the
abstractions produced.

Within programming language research, type theory is a methodology for producing
abstractions.
A type system lets us enforce in our language complex invariants which maintain
abstraction boundaries while also ideally being machine-checkable.
For example, we give functions function type to abstract away their
implementation via call stacks and jumps and so on.

In this thesis, the invariants of interest revolve around restricting the usage
of variables.
I introduce this topic thoroughly in \cref{sec:linear}, but in brief\ldots

The work of this thesis relies upon type theory in two distinct ways.
Firstly, the main objects of study are programming languages with interesting
type systems.
Secondly, type theory provides the basis of the proof assistant Agda I use to
implement the aforementioned programming languages and operations upon them.

As for the other topic of this thesis,
a \emph{proof assistant}, also known as an \emph{interactive theorem prover}, is
a piece of software that allows for the encoding of mathematical definitions,
theorems, constructions, and proofs, and furthermore check that such encoded
proofs are correct and that such encoded constructions are well formed.
To truly by interactive, i.e.\ to actually assist, a proof assistant will
usually have a user interface which can read partial proofs, display
information about what more proof needs to be given, and provide actions that
will help complete the proof.

Proof assistants have seen increasing use in programming language research in
recent years.
The most obvious reason why working in a proof assistant is seen as beneficial
is that it ensures correctness.
If the proof assistant accurately implements a suitable mathematical foundation,
then any theorem proved in the proof assistant is guaranteed to be a true
theorem of that foundation.
These guarantees of correctness are particularly important when working with
combinatorially complex mathematical objects, proofs about which often require
the consideration of a large number of cases.
Programming language syntaxes are often such complex objects, motivating the use
of proof assistants when studying programming languages.

A second reason to use proof assistants is for the assistance they provide when
exploring a mathematical theory.
When we make a new definition, we may want to test how it works in a special
case, or what constructions it allows us to perform.
In a proof assistant, the assistance tools give us immediate feedback as to what
moves are and aren't allowed.
For example, if we define a complex type system, a proof assistant will let us
interactively build typing derivations, making clear any side conditions and
types of subderivations as we go.
Also, as I do later in this thesis, a proof assistant allows us to build a very
general theory, and practically use that theory directly in more specific cases
without losing rigour.

Thirdly, analogously to how a strong static type system can give us more
confidence when refactoring a program, the constant checking of proofs in a
proof assistant gives us the confidence to change definitions and lemmas knowing
that we will be guided towards the parts of our theory that need to be
correspondingly changed.
This can help if we are developing a new programming language with a changing
specification.

Finally, many proof assistants --- including Agda~\citep{Agda}, which I use in
this thesis --- double as programming languages themselves.
This means that we can write progams and prove properties of them using the same
tool.
Also, many theorems proven in such a proof assistant have computational content.
For example, if we prove a normalisation theorem for a programming language,
this will typically yield a (verified) normalisation algorithm for it, which we
can really run on a computer.
As such, a development in a proof assistant can provide a reference
implementation of a programming language, or even --- as with Idris
2~\citep{Brady21}, Lean 4~\citep{deMU21}, and Cedille~\citep{GRS16} --- the
actual implementation of a programming language.

FIXME: the start of this chapter is directly from the ESOP22 paper.

In this paper, we treat the metatheory of a class of substructural type
systems related to linear logic~\cite{girard87linear}.
This class is variously known as
\emph{coeffectful}~\cite{PetricekOM14,Granule18},
\emph{quantitative}~\cite{BrunelGMZ14,Atkey18}, or
\emph{resource-aware}~\cite{GhicaS14},
or is given no particular name~\cite{reed10distance,abadi99core},
and generalises bounded linear logic to track variable usage with semiring
annotations.
In all of these systems, we have some ambient semiring $\Ann$, and in the
judgements of the type system, variables are annotated by elements of $\Ann$
describing \emph{how} that variable can be used.
The additive structure of $\Ann$ gives the ability to count, or otherwise
accumulate, usages of variables in multiple subterms.
The multiplicative structure gives rise to a form of modality, for example
allowing multiple or unlimited reuse, or movement between security levels, in
the type system.

The aspect of such systems we tackle here is their basic metatheory and
mechanisation thereof.

We build upon both the general structural framework of
\citet{AACMM21} and the substructural techniques of \citet{WA21}.
The way \citeauthor{AACMM21} consolidate and codify mechanisation techniques for
propositional natural deduction systems based on intrinsically typed syntax and
de Bruijn indices, we aim to replicate for linear-like systems based on
semiring usage annotations.
By picking a trivial semiring, our work can subsume that of
\citeauthor{AACMM21}, except for the many pieces of machinery we have not yet
ported to this new framework.

Our work complements that of \citet{Granule18} on the Granule programming
language.
Where Granule focuses on writing programs \emph{in} the language and running
them, we focus on metatheoretic reasoning \emph{about} type systems.
%As such, whereas Granule has a convenient syntax, performant type-checker,
%interpreter

Our work is similar in scope to that of \citet{LicataSR17}, though we work in
a natural deduction style rather than a sequent calculus style.
Where \citeauthor{LicataSR17} are much more agnostic in terms of
substructurality --- allowing for non-commutative and bunched logics ---
we are much more agnostic in terms of syntax.
The system of \citeauthor{LicataSR17} is essentially a single calculus,
supporting ``product'' ($\mathrm F$) types and ``function'' ($\mathrm U$)
types, parametrised on a \emph{mode theory} describing its structural rules.
For this system, they derive the strong result of cut elimination.
Meanwhile, we leave syntax design to the user, and consequently can only
guarantee substitution (which we can only get because of our commitment to
natural deduction).

\section{Outline of the thesis}

This thesis proceeds as follows.
The next two chapters, \cref{sec:simple,sec:linearity}, are introductory in
nature, and cover two largely independent strands of prior work.
In \cref{sec:simple}, I introduce existing methods of representing and reasoning
about type systems in proof assistants based on dependent type theory.
I start from well established representations of well scoped and well typed
terms, and develop these towards a recent approach to environment-based
semantics given by \citet{AACMM21}.
In \cref{sec:linearity}, I discuss the challenges faced when one extends a
treatment of a simple type system, such as that given in \cref{sec:simple}, to
modal and linear type systems.
We see that modal and linear type systems apparently violate some of the nice
properties of the simply typed $\lambda$-calculus we required in
\cref{sec:simple}.
I present a solution for intuitionistic S4 modal logic, but leave a solution for
linear logic to the following chapters.

In the following two chapters, \cref{sec:semirings,sec:ren-sub-lr}, I present a
calculus $\name$ parametrised by a partially ordered semiring of \emph{usage
annotations}.
In \cref{sec:semirings}, I define the calculus, give some possible extensions,
and show that it subsumes intuitionistic S4 modal logic and Intuitionistic
Linear Logic.
In \cref{sec:ren-sub-lr}, I show that $\name$ enjoys generalised versions of the
nice properties required in \cref{sec:simple}, and I proceed to give novel
definitions of simultaneous substitutions and their action on $\name$ terms.
These two chapters are adapted from the work of \citet{WA21}.

The remaining three main chapters,
\cref{sec:framework,sec:semantics,sec:example-semantics}, adapt the syntactic
and semantic framework of \citet{AACMM21}, as presented at the end of
\cref{sec:simple}, to semiring-annotated calculi.
\Cref{sec:framework,sec:semantics} generalise the work on $\name$ presented in
\cref{sec:semirings,sec:ren-sub-lr}, respectively.
\Cref{sec:framework} shows how to formally describe the syntax of an arbitrary
semiring-annotated calculus, following the constructions used in
\cref{sec:semirings}.
\Cref{sec:semantics} then provides the generic environment-based semantic
traversal on such syntaxes, providing renaming and substitution as per
\cref{sec:ren-sub-lr} for all syntaxes as special cases of the generic
traversal.
\Cref{sec:example-semantics} then gives further example uses of the generic
traversal.

Finally, I conclude with \cref{sec:conc}, which discusses the achievements of
this thesis and openings for future work.

\section{Naming and notation conventions}

I assume familiarity with the Curry-Howard correspondence throughout this
thesis.
I make no distinction between logics and type theories, and use terminology from
each interchangeably.
Each following bullet point lists a collection of synonyms.

\begin{itemize}
  \item assumption, hypothesis, variable
  \item derivation, proof, term
  \item proposition, formula, type
  \item connective, type former
\end{itemize}

I carry out mechanised constructions and proofs in the proof assistant and
programming language Agda~\citep{Agda}.
Agda is based on Martin-L\"{o}f's intensional dependent type theory, so I
similarly present non-mechanised constructions and proofs assuming a foundation
given by dependent type theory, in a style inspired by the HoTT
Book~\citep{hottbook}.

\section{Agda primer}

I use the proof assistant and programming language Agda throughout this thesis,
with Agda code being used particularly in \cref{sec:simple} and the later
chapters.
As such, it is important for the reader to be able to read basic Agda syntax in
order to benefit from the parts of the exposition that reside in code listings.
The syntax of Agda is broadly similar to that of Haskell, and relatively close
to that of Standard ML, OCaml, and Coq's Gallina sublanguage.
I will assume that the reader is able to read basic Haskell code, and spend most
time explaining differences thereof.

\subsection{Lexing and parsing}

Agda is extremely liberal in its set of allowed names.
There is just a single lexical class (unlike in Haskell, where, for example,
constructors start with a capital letter and definitions start with a lowercase
letter), and names can be any string of Unicode characters except whitespace and
special characters \verb|.;{}()@"|, apart from those strings reserved as
keywords or literals.
Therefore, we can introduce names like
\AgdaFunction{0x-+-$\uplambda\rightarrow$}
to stand for any kind of identifiable thing.
With such free-form names, ample spacing is required between identifiers.
For example, while \AgdaNumber{0}\AgdaSpace{}\AgdaDatatype{$\leq$}\AgdaSpace{}%
\AgdaNumber{1} is a possible expression containing three
identifiers, \AgdaInductiveConstructor{0$\leq$1} is a single
valid identifier.
Only the special characters may appear next to names without being separated by
whitespace.

A character with unique behaviour in Agda's syntax is the underscore
(\AgdaSymbol{\_}).
Within a name, an underscore signifies that the name will function as a mixfix
operator, allowing for an argument in the position of the underscore.
For example, the full name of the \AgdaDatatype{$\leq$} operator used in the
previous paragraph is \AgdaDatatype{\_$\leq$\_}, signifying that it can take an
argument to its left and its right.
We can also introduce closed operators, like \AgdaDatatype{[\_]}, which can take
an argument between the square brackets (e.g.\ \AgdaDatatype{[}\AgdaSpace{}%
\AgdaNumber{1}\AgdaSpace{}\AgdaDatatype{]}, with spaces still being important).
Mixfix operators can be partially applied by leaving underscores in the name in
the application.
For example, \AgdaDatatype{\_$\leq$}\AgdaSpace{}\AgdaNumber{1} could be the
predicate asserting that a number is less than or equal to 1.

On its own, an underscore has a completely different meaning, which can depend
on context.
In patterns, an underscore has the same meaning as it has in Haskell and ML ---
it holds the place of a pattern variable, but does not name that variable.
In expressions, an underscore stands for an unspecified subterm which will be
solved by unification.
The solving of unspecified terms is canonical and respects $\beta\eta$-equality,
unlike in Coq.

Spacing is important particularly important when dealing with underscores.
For example,
\AgdaDatatype{\_$\leq$}\AgdaSpace{}\AgdaSymbol{\_} (with a space after the
\AgdaDatatype{$\leq$} but not before) standing for the predicate asserting that
a number is less than or equal to some unspecified number.

Like Haskell, Agda's syntax is indentation-sensitive.
The distinctions conveyed by indentation are largely obvious or intuitive to
human readers (for example, allowing for line-continuation or delineating nested
modules), so I will not discuss them explicitly here.

\subsection{Functions, $\Pi$-types}\label{sec:pi-types}

Simple function types take the form
\AgdaArgument{A}\AgdaSpace{}\AgdaSymbol{$\to$}\AgdaSpace{}\AgdaArgument{B},
coinciding with Haskell's syntax.
Also as in Haskell and ML, the function arrow nests to the right.
However, Agda has a termination checker ensuring that all defineable functions
are total, so many Haskell functions do not have a corresponding Agda function.

The key feature distinguishing Agda from Haskell is the presence of arbitrary
dependent types, including dependent function types ($\Pi$-types).
The basic syntax for $\Pi$-types is
\AgdaSymbol{(}\AgdaBound{x}\AgdaSpace{}\AgdaSymbol{:}\AgdaSpace{}%
\AgdaArgument{A}\AgdaSymbol{)}\AgdaSpace{}\AgdaSymbol{$\to$}\AgdaSpace{}%
\AgdaArgument{B},
where variable \AgdaBound{x} can occur free in expression \AgdaArgument{B}.
However, there are several syntactic conveniences I use throughout the code
listings.
For one, iterated $\Pi$-types can be abbreviated so that
\AgdaSymbol{(}\AgdaBound{x}\AgdaSpace{}\AgdaSymbol{:}\AgdaSpace{}%
\AgdaArgument{A}\AgdaSymbol{)}\AgdaSpace{}\AgdaSymbol{$\to$}\AgdaSpace{}%
\AgdaSymbol{(}\AgdaBound{y}\AgdaSpace{}\AgdaSymbol{:}\AgdaSpace{}%
\AgdaArgument{B}\AgdaSymbol{)}\AgdaSpace{}\AgdaSymbol{$\to$}\AgdaSpace{}%
\AgdaArgument{C}
is written just
\AgdaSymbol{(}\AgdaBound{x}\AgdaSpace{}\AgdaSymbol{:}\AgdaSpace{}%
\AgdaArgument{A}\AgdaSymbol{)}\AgdaSpace{}%
\AgdaSymbol{(}\AgdaBound{y}\AgdaSpace{}\AgdaSymbol{:}\AgdaSpace{}%
\AgdaArgument{B}\AgdaSymbol{)}\AgdaSpace{}\AgdaSymbol{$\to$}\AgdaSpace{}%
\AgdaArgument{C},
omitting the first arrow.
For another, prefixing an arrow with the \AgdaSymbol{$\forall$} symbol allows us
to omit domain types.
For example,
\AgdaSymbol{$\forall$}\AgdaSpace{}\AgdaBound{x}\AgdaSpace{}\AgdaSymbol{$\to$}%
\AgdaSpace{}\AgdaArgument{B}
is equivalent to
\AgdaSymbol{(}\AgdaBound{x}\AgdaSpace{}\AgdaSymbol{:}\AgdaSpace{}%
\AgdaSymbol{\_}\AgdaSymbol{)}\AgdaSpace{}\AgdaSymbol{$\to$}%
\AgdaSpace{}\AgdaArgument{B}.
Notice that this is a very different type to
\AgdaBound{x}\AgdaSpace{}\AgdaSymbol{$\to$}\AgdaSpace{}\AgdaArgument{B},
which is a non-dependent function type equivalent to
\AgdaSymbol{(}\AgdaSymbol{\_}\AgdaSpace{}\AgdaSymbol{:}\AgdaSpace{}%
\AgdaBound{x}\AgdaSymbol{)}\AgdaSpace{}\AgdaSymbol{$\to$}\AgdaSpace{}%
\AgdaArgument{B}.
When writing
\AgdaSymbol{$\forall$}\AgdaSpace{}\AgdaBound{x}\AgdaSpace{}\AgdaSymbol{$\to$}%
\AgdaSpace{}\AgdaArgument{B},
we assume that the occurrence of \AgdaBound{x} in \AgdaArgument{B} tells us what
type \AgdaBound{x} should have (i.e.\ there is enough information to solve the
underscore in
\AgdaSymbol{(}\AgdaBound{x}\AgdaSpace{}\AgdaSymbol{:}\AgdaSpace{}%
\AgdaSymbol{\_}\AgdaSymbol{)}\AgdaSpace{}\AgdaSymbol{$\to$}%
\AgdaSpace{}\AgdaArgument{B}).

Just like in Haskell, functions in Agda can be introduced via
\AgdaSymbol{$\uplambda$}-abstractions and clausal definitions, and are applied
by juxtaposition.
Agda also includes \emph{extended} \AgdaSymbol{$\uplambda$}-abstractions,
introduced via equivalent syntaxes
\AgdaSymbol{$\uplambda$}\AgdaSpace{}\AgdaKeyword{where}\AgdaSpace{}%
\AgdaBound{x}\AgdaSpace{}\AgdaSymbol{$\to$}\AgdaSpace{}\AgdaArgument{M} and
\AgdaSymbol{$\uplambda$}\AgdaSpace{}\AgdaSymbol{\{}\AgdaSpace{}%
\AgdaBound{x}\AgdaSpace{}\AgdaSymbol{$\to$}\AgdaSpace{}\AgdaArgument{M}%
\AgdaSpace{}\AgdaSymbol{\}},
which allow for pattern-matching on the variable \AgdaBound{x} (or all of the
variables, if there are multiple variables).

Agda allows for arbitrary function arguments to be marked as \emph{implicit} by
replacing the round brackets in the type by curly braces.
For example, if we have
\AgdaBound{f}\AgdaSpace{}\AgdaSymbol{:}\AgdaSpace{}%
\AgdaSymbol{\{}\AgdaBound{x}\AgdaSpace{}\AgdaSymbol{:}\AgdaSpace{}%
\AgdaArgument{A}\AgdaSymbol{\}}\AgdaSpace{}\AgdaSymbol{$\to$}\AgdaSpace{}%
\AgdaArgument{B},
then the argument to \AgdaBound{f} is implicit.
Being implicit means that an occurrence of \AgdaBound{f} is treated as if it has
been applied to an underscore, giving the expression \AgdaBound{f} the type
$\AgdaArgument{B}[\AgdaSymbol{\_}/\AgdaBound{x}]$ (i.e.\ \AgdaArgument{B} with
\AgdaSymbol{\_} substituted in for \AgdaBound{x}; the substitution syntax is not
part of Agda syntax).
An implicit argument can also be given explicitly in two ways.
The first of a series of implicit arguments can be given by surrounding the
argument in curly braces, and any other implicit arguments in the series can be
given by including the name of the argument.
For example,
\AgdaBound{f}\AgdaSpace{}\AgdaSymbol{\{}\AgdaSymbol{\_}\AgdaSymbol{\}},
\AgdaBound{f}\AgdaSpace{}\AgdaSymbol{\{}\AgdaArgument{x}\AgdaSpace{}%
\AgdaSymbol{=}\AgdaSpace{}\AgdaSymbol{\_}\AgdaSymbol{\}}, and just
\AgdaBound{f} are all equivalent expressions, and the underscore can be filled
in in either of the first two expressions to provide an actual value for the
implicit argument.
Implicit arguments are usually left out of \AgdaSymbol{$\uplambda$}-abstractions
and clausal definitions, but can be bound to names and pattern-matched on using
the same syntax as in expressions.

There are a few other places in the syntax using single curly braces, all of
which have meanings related to implicit arguments.
I also make a small amount of use of double curly braces
(\AgdaSymbol{\{\{} and \AgdaSymbol{\}\}}), which denote arguments which are to
be solved by instance resolution.
Instance resolution is very similar to Haskell's typeclass resolution ---
finding non-canonical solutions based on the instances in scope.

Agda uses $\Pi$-types where in Haskell we would use polymorphism.
For example, we can define an identity function as below.
The definition relies on quantifying over terms of type \AgdaPrimitive{Set},
i.e.\ (small) types.
This definition also gives an example of defining a function with an implicit
argument (\AgdaBound{X}), which can typically be inferred from either the
kargument type or the return type, so can be omitted.

\ExecuteMetaData[\SnippetsTwotex]{id0}

An unfortunate feature of the definition \AgdaFunction{id$_0$} is that we cannot
apply it to the expression \AgdaPrimitive{Set}, because \AgdaPrimitive{Set}
contains only small types, and itself is a large type.
We can work around these size issues using \emph{universe level polymorphism},
as in the following definition.

\ExecuteMetaData[\SnippetsTwotex]{id}

Universe levels start at \AgdaPrimitive{0$\ell$}, with \AgdaPrimitive{Set} being
an alias for \AgdaPrimitive{Set}\AgdaSpace{}\AgdaPrimitive{0$\ell$} (and also
\AgdaPrimitive{Set$_0$}).
Larger levels can be produced with the successor operator \AgdaPrimitive{suc},
and we can take the least upper bound of two levels using the operator
\AgdaPrimitive{\_$\sqcup$\_}.

\subsection{Data types}

Agda's \AgdaKeyword{data}-declarations are similar in scope to Haskell's, with
the addition of indexing by terms of arbitrary type.
\AgdaKeyword{data}-declarations give us indexed inductive sum-of-product types.

All \AgdaKeyword{data}-declarations use GADT syntax.
The body of a declaration comprises a list of constructor names paired with
their types.
Where two constructors have the same type, they may be written on the same line
with their names separated by whitespace, as I do with the two constructors of
\AgdaDatatype{Bool} below.
\AgdaDatatype{Bool} has two constructors --- \AgdaInductiveConstructor{true} and
\AgdaInductiveConstructor{false} --- both of which have type
\AgdaDatatype{Bool}.
\AgdaDatatype{$\mathbb N$} also has two constructors, where
\AgdaInductiveConstructor{zero} has type \AgdaDatatype{$\mathbb N$} and
\AgdaInductiveConstructor{suc} is inductive, with type
\AgdaDatatype{$\mathbb N$}\AgdaSpace{}\AgdaKeyword{$\to$}\AgdaSpace{}%
\AgdaDatatype{$\mathbb N$}.

\noindent
\begin{minipage}[t]{0.5\textwidth}
  \ExecuteMetaData[\SnippetsThreetex]{Bool}
\end{minipage}
\begin{minipage}[t]{0.5\textwidth}
  \ExecuteMetaData[\SnippetsThreetex]{Nat}
\end{minipage}

\AgdaDatatype{Bool} and \AgdaDatatype{$\mathbb N$} are both types, and indeed
small types, as we can see by the fact that they are annotated to have type
\AgdaPrimitive{Set}.
We can also use \AgdaKeyword{data}-declarations to define type \emph{families}
in various ways.
The simplest is to add \emph{parameters}, as in the type family
\AgdaDatatype{List} below.
Parameters always appear to the left of the colon of the first line of the
\AgdaKeyword{data}-declaration, and are constant throughout the
\AgdaKeyword{data}-declaration.
Variables to the left of the colon can appear in the body of the
\AgdaKeyword{data}-declaration without further quantification.

\ExecuteMetaData[\SnippetsThreetex]{List}

Slightly more flexible than parameters are \emph{Protestant indices}%
\footnote{The terminology of Protestant/Catholic indices comes orally from Conor
McBride. I am not aware of a written reference.}.
Protestant indices also appear to the left of the colon, and also must appear
unmodified in the \emph{return type} of all of the constructors.
However, they may take different values in inductive appearances of the type
family in the argument types of constructors.
Protestant indices give a generalisation of polymorphic
recursion to indices of arbitrary type~\citep{Mycroft84,Henglein93}.

I give two examples of type families with Protestant indices.
The first, \AgdaDatatype{NestedList} is standard from the polymorphic recursion
literature.
It is worth noting at this point that Agda permits overloading of constructors,
which are disambiguated by the type family they are being used to construct.
This overloading allows \AgdaDatatype{List} and \AgdaDatatype{NestedList} to
have constructors with the same names without confusion.
The second example, \AgdaDatatype{ScopedTerm} is a data structure representing
well scoped untyped $\lambda$-calculus terms.
The Protestant index \AgdaBound{s} describes the number of variables in scope,
which increases by 1 when we introduce a $\lambda$-abstraction.
I will introduce \AgdaDatatype{Fin}, a type family with a specified natural
number of inhabitants, in the next set of examples.
As a syntactic note, in the type of the \AgdaInductiveConstructor{app}
constructor, I use the two variable names \AgdaBound{M} and \AgdaBound{N}
separated by whitespace to name two arguments with the same type.

\ExecuteMetaData[\SnippetsThreetex]{NestedList}
\ExecuteMetaData[\SnippetsThreetex]{ScopedTerm}

The most general way to make a type family is to introduce a
\emph{Catholic index}.
The types of Catholic indices are specified to the right of the colon, and can
be instantiated arbitrarily throughout the \AgdaKeyword{data}-declaration.
Catholic indices are not in scope for the body of the
\AgdaKeyword{data}-declaration, so the values filling them may need to be
quantified over in each constructor.
When this quantification is over a large type, like \AgdaPrimitive{Set}, the
type family being defined will itself need to be large, e.g.\ inhabiting
\AgdaPrimitive{Set$_1$}.
This is a major reason for not defining types like \AgdaDatatype{List} and
\AgdaDatatype{NestedList} using Catholic indices.

I give two examples of type families with Catholic indices.
The first is the \AgdaDatatype{Fin} family, as used in \AgdaDatatype{ScopedTerm}
above.
By inspection of the return types of the constructors, there is no way to
produce a canonical inhabitant of
\AgdaDatatype{Fin}\AgdaSpace{}\AgdaInductiveConstructor{zero}.
For
\AgdaDatatype{Fin}\AgdaSpace{}\AgdaSymbol(\AgdaInductiveConstructor{suc}%
\AgdaSpace{}\AgdaBound{n}\AgdaSymbol),
we can potentially use either of the constructors.
Either we use \AgdaInductiveConstructor{zero} to get a canonical inhabitant, or
if we can make a number with a smaller bound (i.e.\ an inhabitant of
\AgdaDatatype{Fin}\AgdaSpace{}\AgdaBound{n}), we can use
\AgdaInductiveConstructor{suc} to produce a larger number.

\ExecuteMetaData[\SnippetsThreetex]{Fin}

The second example of a type family with Catholic indices is more general in
nature.
Below I define \emph{propositional equality}, written
\AgdaDatatype{\_$\equiv$\_}.
It has two parameters and one Catholic index (though the standard library
version of propositional equality I use throughout this thesis has an extra
level parameter for the sake of universe level polymorphism).
The constructor \AgdaInductiveConstructor{refl} constructs an inhabitant of
\AgdaArgument{M}\AgdaSpace{}\AgdaDatatype{\_$\equiv$\_}\AgdaSpace{}%
\AgdaArgument{N} only when \AgdaArgument{N} is definitionally equal to
\AgdaArgument{M} (because terms are considered ``the same'' to the type checker
exactly when they are definitionally equal).
Notice that \AgdaInductiveConstructor{refl} does not quantify over \AgdaBound{x}
because \AgdaBound{x} is already in scope as a parameter.

\ExecuteMetaData[\SnippetsThreetex]{Eq}

It is through type families like \AgdaDatatype{\_$\equiv$\_} that we can state
and prove mathematical theorems in Agda.
In the following subsection, I show how to use such indexed type families.

\subsection{Clausal definitions}

Clausal definitions of functions in Agda look very similar to their equivalents
in Haskell.
However, definitions in Agda regularly make use of
\emph{dependent pattern matching}, which is our primary way of using indexed
data types.
Recursive definitions are also conservatively checked for termination.

I will explain the salient aspects of clausal function definitions via two
examples.
The first, unimaginatively named \AgdaFunction{lemma}, shows a simple case where
pattern matching modifies the context through unification of Catholic indices.
The second, named \AgdaFunction{elim-Fin-zero}, gives an example of proper
dependent pattern matching.

In the following definition \AgdaFunction{lemma}, we want to chase equations in
order to prove that \AgdaBound{x} is propositionally equal to \AgdaBound{z}.
We start with the following incomplete definition, where the expression
\AgdaHole{\{ \}0} marks an interation point, or \emph{hole}, in the program, to
which we can apply interactive commands to complete the program.

\begin{code}
\>[0]\AgdaFunction{lemma}\AgdaSpace{}%
\AgdaSymbol{:}\AgdaSpace{}%
\AgdaSymbol{∀}\AgdaSpace{}%
\AgdaSymbol{\{}\AgdaBound{A}\AgdaSpace{}%
\AgdaSymbol{:}\AgdaSpace{}%
\AgdaPrimitive{Set}\AgdaSymbol{\}}\AgdaSpace{}%
\AgdaSymbol{\{}\AgdaBound{x}\AgdaSpace{}%
\AgdaBound{y}\AgdaSpace{}%
\AgdaBound{z}\AgdaSpace{}%
\AgdaSymbol{:}\AgdaSpace{}%
\AgdaBound{A}\AgdaSymbol{\}}\AgdaSpace{}%
\AgdaSymbol{→}\AgdaSpace{}%
\AgdaBound{x}\AgdaSpace{}%
\AgdaOperator{\AgdaDatatype{≡}}\AgdaSpace{}%
\AgdaBound{y}\AgdaSpace{}%
\AgdaSymbol{→}\AgdaSpace{}%
\AgdaBound{z}\AgdaSpace{}%
\AgdaOperator{\AgdaDatatype{≡}}\AgdaSpace{}%
\AgdaBound{y}\AgdaSpace{}%
\AgdaSymbol{→}\AgdaSpace{}%
\AgdaBound{x}\AgdaSpace{}%
\AgdaOperator{\AgdaDatatype{≡}}\AgdaSpace{}%
\AgdaBound{z}\<%
\\
\>[0]\AgdaFunction{lemma}\AgdaSpace{}%
\AgdaBound{p}\AgdaSpace{}%
\AgdaBound{q}\AgdaSpace{}%
\AgdaSymbol{=}\AgdaSpace{}%
\AgdaHole{\{ \}0}\<%
\end{code}

As a first step, I choose to match on the variable
\AgdaBound{p}\AgdaSpace{}\AgdaSymbol:\AgdaSpace{}%
\AgdaBound{x}\AgdaSpace{}%
\AgdaOperator{\AgdaDatatype{≡}}\AgdaSpace{}%
\AgdaBound{y}.
The only applicable pattern is \AgdaInductiveConstructor{refl}.
Doing this match has the effect of unifying \AgdaBound{y} --- which is taking
the position of the Catholic index of \AgdaDatatype{\_$\equiv$\_} --- with
\AgdaBound{x} --- which is the value of the index specified in the type of
\AgdaInductiveConstructor{refl}.
Local variables act as unification variables, so the unification succeeds with
most general unifier $[x \coloneqq x, y \coloneqq x]$.
Therefore, the type of \AgdaBound{q} becomes
\AgdaBound{z}\AgdaSpace{}%
\AgdaOperator{\AgdaDatatype{≡}}\AgdaSpace{}%
\AgdaBound{x}.

\begin{code}
\>[0]\AgdaFunction{lemma}\AgdaSpace{}%
\AgdaSymbol{:}\AgdaSpace{}%
\AgdaSymbol{∀}\AgdaSpace{}%
\AgdaSymbol{\{}\AgdaBound{A}\AgdaSpace{}%
\AgdaSymbol{:}\AgdaSpace{}%
\AgdaPrimitive{Set}\AgdaSymbol{\}}\AgdaSpace{}%
\AgdaSymbol{\{}\AgdaBound{x}\AgdaSpace{}%
\AgdaBound{y}\AgdaSpace{}%
\AgdaBound{z}\AgdaSpace{}%
\AgdaSymbol{:}\AgdaSpace{}%
\AgdaBound{A}\AgdaSymbol{\}}\AgdaSpace{}%
\AgdaSymbol{→}\AgdaSpace{}%
\AgdaBound{x}\AgdaSpace{}%
\AgdaOperator{\AgdaDatatype{≡}}\AgdaSpace{}%
\AgdaBound{y}\AgdaSpace{}%
\AgdaSymbol{→}\AgdaSpace{}%
\AgdaBound{z}\AgdaSpace{}%
\AgdaOperator{\AgdaDatatype{≡}}\AgdaSpace{}%
\AgdaBound{y}\AgdaSpace{}%
\AgdaSymbol{→}\AgdaSpace{}%
\AgdaBound{x}\AgdaSpace{}%
\AgdaOperator{\AgdaDatatype{≡}}\AgdaSpace{}%
\AgdaBound{z}\<%
\\
\>[0]\AgdaFunction{lemma}\AgdaSpace{}%
\AgdaInductiveConstructor{refl}\AgdaSpace{}%
\AgdaBound{q}\AgdaSpace{}%
\AgdaSymbol{=}\AgdaSpace{}%
\AgdaHole{\{ \}0}\<%
\end{code}

The next step is to match on \AgdaBound{q}.
This similarly unifies \AgdaBound{z} and \AgdaBound{x}, making the conclusion
type
\AgdaBound{z}\AgdaSpace{}%
\AgdaOperator{\AgdaDatatype{≡}}\AgdaSpace{}%
\AgdaBound{z}.
Finally, this conclusion type is in the image of the
\AgdaInductiveConstructor{refl} constructor, so we may fill the hole with
\AgdaInductiveConstructor{refl}.

\ExecuteMetaData[\SnippetsThreetex]{lemma}

Full \emph{dependent} pattern matching, as described by \citet{MM04}, is when
the unification of indices described above takes account of constructors.
In particular, the constructors of a data type satisfy the ``no confusion''
property --- constructors are injective and mutually disjoint.
Where we encounter disjoint constructors during unification, we may dismiss the
corresponding case as impossible.
Consider the following example (\AgdaFunction{elim-Fin-zero}).
We start with an argument
\AgdaBound{i}\AgdaSpace{}\AgdaSymbol:\AgdaSpace{}\AgdaDatatype{Fin}\AgdaSpace{}%
\AgdaInductiveConstructor{zero}, and consider which constructors could possibly
construct such a value.
However, as noted earlier, both constructors of \AgdaDatatype{Fin} target
successor values of the index, from which \AgdaInductiveConstructor{zero} is
disjoint.
Therefore, both cases are impossible.
The notation when all cases are impossible is to place empty round brackets
\AgdaSymbol{()} in the place of the impossible argument, and to not provide a
righthand side to the clause.

\ExecuteMetaData[\SnippetsThreetex]{elim-Fin-zero}

As an example of the injectivity of constructors, the obvious example is to
internalise the proof of injectivity for a given constructor, as I do in
\AgdaFunction{suc-injective}.
We start with an argument
\AgdaBound{p}\AgdaSpace{}\AgdaSymbol:\AgdaSpace{}%
\AgdaInductiveConstructor{suc}\AgdaSpace{}\AgdaBound{m}%
\AgdaSpace{}\AgdaOperator{\AgdaDatatype{$\equiv$}}\AgdaSpace{}%
\AgdaInductiveConstructor{suc}\AgdaSpace{}\AgdaBound{n}
and match on it.
This time, we do have a possible pattern --- \AgdaInductiveConstructor{refl} ---
but working out how to change the context relies on unifying
\AgdaInductiveConstructor{suc}\AgdaSpace{}\AgdaBound{m} with
\AgdaInductiveConstructor{suc}\AgdaSpace{}\AgdaBound{n}.
We are justified in doing this, with most general unifier
$[m \coloneqq m, n \coloneqq m]$, because \AgdaInductiveConstructor{suc} is
injective (with respect to propositional equality).
If the checker for dependent pattern matching did not know that
\AgdaInductiveConstructor{suc} was injective --- for example, if it were instead
a defined function --- then the unification would fail.
This leads to the intuition that constructors and variables are well behaved
with respect to dependent pattern matching, while other expressions are not.

\ExecuteMetaData[\SnippetsThreetex]{suc-injective}

Ordinarily, each clause of a definition gives rise to a \emph{definitional}
equation between its lefthand side and righthand side.
In intensional type theory, as implemented by Agda, definitional and
propositional equality are contrasted to each other.
Definitional equality corresponds to a decidable fragment of the natural
equational theory of the type theory.
As such, definitional equality is an entirely metatheoretic notion, and we can
neither assume nor prove directly definitional equations within the language.
Definitional equality is sometimes also called \emph{judgemental equality},
because it forms a judgement which plays a part in the rules of the type theory.
As well as from the clauses of definitions, we also get definitional equations
from $\beta$-reductions of $\lambda$-abstractions and $\eta$-laws of functions
and records.
Because the type checker treats definitionally equal terms equivalently, we are
able to refactor up to definitional equality without changing any downstream
code.

On the other hand, propositional equality is a notion internal to the language,
as we have seen by defining propositional equality (\AgdaDatatype{\_$\equiv$\_})
and proving things about it (\AgdaFunction{lemma}).
Propositional equality is sometimes known as \emph{typal equality} or
\emph{mathematical equality}.
The latter name comes from the fact that propositional equality is the closest
notion to what mathematicians usually call \emph{equality}, because, for
example, it allows us to prove things like
\AgdaBound{m}\AgdaSpace{}\AgdaFunction{+}\AgdaSpace{}\AgdaBound{n}%
\AgdaSpace{}\AgdaDatatype{$\equiv$}\AgdaSpace{}%
\AgdaBound{n}\AgdaSpace{}\AgdaFunction{+}\AgdaSpace{}\AgdaBound{m} for all
natural numbers \AgdaBound{m} and \AgdaBound{n}.
Propositional equality satisfies Leibniz' law, meaning that an inhabitant of a
type \AgdaBound{A} can be coerced into an inhabitant of any type propositionally
equal to \AgdaBound{A}.
However, this cast requires marking in the code, so is less convenient to use
than definitional equality.

Definitional equality between two terms implies their propositional equality,
because exactly when two terms are definitionally equal, the type checker is
happy to accept \AgdaInductiveConstructor{refl} as a proof.
This relationship between the two is simple, but can still be deceptive.
For example, consider the notion of injectivity with respect to definitional
and propositional equality.
A function $f$ is injective (with respect to some notion of equality $\approx$)
when, for all $x$ and $y$, we have $f\,x \approx f\,y \to x \approx y$.
Because $\approx$ appears both covariantly and contravariantly in this
definition, we have implications in neither direction between definitional
injectivity and propositional injectivity.
Indeed, we can find examples of all four possibilities:
constructors are injective in both senses; type formers, like \AgdaDatatype{Fin}
and \AgdaDatatype{List}, are definitionally injective but not propositionally
injective;
\ExecuteMetaData[\SnippetsThreetex]{double}\ can be proven to be
propositionally injective, but is not definitionally injective because
\AgdaFunction{\_+\_} is not injective;
and nearly everything else is not injective in either sense.

Because the notions of definitional and propositional injectivity are
incomparable, so too are the corresponding unification procedures.
Propositional unification (using only the injectivity of constructors) is used
during dependent pattern matching, while solving of implicit arguments and
underscores in expressions is done by definitional unification.

\subsection{Records, $\Sigma$-types}

While Agda provides built-in basic $\Pi$-types, with special syntax described in
\cref{sec:pi-types}, it does not do the same for $\Sigma$-types.
Instead, the default way to get the functionality of $\Sigma$-types is to
declare record types, similarly to how we get sums via
\AgdaKeyword{data}-declarations.
However, the standard library does provide $\Sigma$-types, via record types,
using the following declaration.

\ExecuteMetaData[\SnippetsThreetex]{Sigma}

As does the standard library, I will begin to use universe level polymorphism in
these example definitions.
Here, \AgdaBound{a} and \AgdaBound{b} are the levels of the two projections.
The level of the record type must be at least the level of the type of each
field, and in this case, the smallest such level is
\AgdaBound{a}\AgdaSpace{}\AgdaOperator{\AgdaPrimitive{$\sqcup$}}\AgdaSpace{}%
\AgdaBound{b}.
As for the main points of interest in this \AgdaKeyword{record}-declaration, it
contains two fields.
The first, \AgdaField{proj$_1$}, simply has type \AgdaBound{A}.
The second, \AgdaField{proj$_2$}, then has a type dependent on the value of the
first field.
Additionally, we give this record type a named constructor
\AgdaInductiveConstructor{\_,\_}.
Any record type can also be constructed using the more verbose syntax
\begin{code}[inline]%
\>[0]\AgdaKeyword{record}\AgdaSpace{}%
\AgdaSymbol{\{}\AgdaSpace{}%
\AgdaField{proj$_1$}\AgdaSpace{}%
\AgdaSymbol{=}\AgdaSpace{}%
\AgdaHole{\{ \}1}\AgdaSpace{}%
\AgdaSymbol{;}\AgdaSpace{}%
\AgdaField{proj$_2$}\AgdaSpace{}%
\AgdaSymbol{=}\AgdaSpace{}%
\AgdaHole{\{ \}2}\AgdaSpace{}%
\AgdaSymbol{\}}\<%
\end{code}.

The standard library provides various notations for
\AgdaRecord{$\Upsigma$}-types, useful in various situations.
In this thesis, I use \ExecuteMetaData[\SnippetsFourtex]{Sigma-syntax}\ and
\ExecuteMetaData[\SnippetsFourtex]{exists}\ as equivalent notations for
\AgdaRecord{$\Upsigma$}\AgdaSpace{}\AgdaBound{A}\AgdaSpace{}\AgdaBound{B}.
Indeed, the $\eta$-contracted form can be used with \AgdaFunction{$\exists$}, as
in \AgdaFunction{$\exists$}\AgdaSpace{}\AgdaBound{B} (\AgdaSymbol{\textbackslash}
is an alternative notation for \AgdaSymbol{$\uplambda$}, as in Haskell).
\AgdaRecord{$\Upsigma$} also specialises to non-dependent products, as given by
the infix operator \AgdaFunction{\_$\times$\_}.
This is achieved by setting the parameter \AgdaBound{B} to be a constant type
family.
The resulting operator \AgdaFunction{\_$\times$\_}, as well as the non-dependent
function type, behave better than their dependent counterparts with respect to
unification because they allow us to remain in the first order fragment of
higher order unification.

There are two main ways of using the fields of a record.
The first is to put the projections into scope using
\AgdaKeyword{open}\AgdaSpace{}\AgdaRecord{$\Upsigma$}, and then to use the field
names to project out of arbitrary terms of \AgdaRecord{$\Upsigma$}-type.
This is what I will always do when using the \AgdaRecord{$\Upsigma$}-type
family.
Within this paradigm, there are two further notational choices.
Either, we can use the field names as functions, so that
\AgdaBound{z}\AgdaSpace{}\AgdaSymbol{=}\AgdaSpace{}\AgdaField{proj$_1$}%
\AgdaSpace{}\AgdaBound{z}\AgdaSpace{}\AgdaOperator{\AgdaInductiveConstructor,}%
\AgdaSpace{}\AgdaField{proj$_2$}\AgdaSpace{}\AgdaBound{z},
or we can use postfix projections via the space-dot notation, as in
\AgdaBound{z}\AgdaSpace{}\AgdaSymbol{=}\AgdaSpace{}\AgdaBound{z}%
\AgdaSpace{}\AgdaSymbol.\AgdaField{proj$_1$}\AgdaSpace{}%
\AgdaOperator{\AgdaInductiveConstructor,}\AgdaSpace{}\AgdaBound{z}%
\AgdaSpace{}\AgdaSymbol.\AgdaField{proj$_2$}.
I tend to prefer the latter, also using it occasionally in ordinary mathematical
notation (without the space).
Both notations can also be used on the lefthand side of a clausal definition as
\emph{copatterns}.
Copatterns let us think of records as being function-like, with the fields of a
record type being the possible arguments we can pass to such a function.

The second way of using the fields of a record requires a motivating example.
Consider the below definition of the type of semigroups at universe level
\AgdaBound{$\ell$}.
A semigroup has a carrier set, a binary operation on that set, and an
associativity law for that binary operation.

\ExecuteMetaData[\SnippetsFourtex]{Semigroup}

In order to use the fields of \AgdaRecord{Semigroup} in the intended way, we
do not open them into global scope.
Doing so would mean that, for example, \AgdaField{\_$\bullet$\_} would take
three arguments: the semigroup and its two intended arguments.
Instead, we get to the point where we have a semigroup \AgdaBound{G} in scope
and use
\AgdaKeyword{open}\AgdaSpace{}\AgdaRecord{Semigroup}\AgdaSpace{}\AgdaBound{G}
to put into scope the components \emph{of \AgdaBound{G}}.
Then, the name \AgdaField{Carrier} in scope will refer to the carrier set of
\AgdaBound{G}, the name \AgdaField{\_$\bullet$\_} will refer to the binary
operator (which really takes two arguments), et cetera.
Doing this gives the impression of working ``inside'' \AgdaBound{G}, which is
the way I typically work with algebraic sets with structure.

By $\eta$-equality, two inhabitants of a record type are definitionally equal
exactly when they agree definitionally on all fields.
This often makes record types much more convenient to work with than the
corresponding single-constructor data types, which do not enjoy any $\eta$-laws.
Notably, all inhabitants of the record type \AgdaRecord{$\top$} with no fields
are definitionally equal.

Along with \AgdaRecord{$\Upsigma$} and \AgdaRecord{$\top$}, there are two more
general-purpose record types I need to cover which take advantage of two special
features of \AgdaKeyword{record}-declarations (and also
\AgdaKeyword{data}-delcarations, but I use \AgdaKeyword{record}-declarations for
the convenience reason given in the previous paragraph).
The first feature is that the universe level of a record type has a lower bound
(the level of each field) but no upper bound.
Therefore, we can introduce the following declaration \AgdaRecord{Lift}, which
takes a type \AgdaBound{A} at level \AgdaBound{a} and produces an equivalent
type at a potentially higher level
\AgdaBound{a}\AgdaSpace{}\AgdaOperator{\AgdaPrimitive{$\sqcup$}}\AgdaSpace{}%
\AgdaBound{$\ell$}.
This type former is useful in situations which require a type at a specific
level, such as when constructing a type using a function.

\ExecuteMetaData[\SnippetsFourtex]{Lift}

The other interesting property we get from \AgdaKeyword{record}-declarations is
that the resulting type family is definitionally injective in its parameters.
Therefore, record types behave well in the form of unification that solves
implicit arguments.
We can use this property to take any type family \AgdaBound{F} and produce an
equivalent family \AgdaRecord{Wrap}\AgdaSpace{}\AgdaBound{F} which is
definitionally injective.

\ExecuteMetaData[\SnippetsFourtex]{Wrap}

As an example, if we have a variable
\AgdaBound{f}\AgdaSpace{}\AgdaSymbol{:}\AgdaSpace{}\AgdaRecord{Wrap}%
\AgdaSpace{}\AgdaFunction{F}\AgdaSpace{}\AgdaBound{y}
and pass it to a function with a type of the form
\AgdaSymbol{$\forall$}\AgdaSpace{}\AgdaSymbol\{\AgdaBound{x}\AgdaSymbol\}%
\AgdaSpace{}\AgdaSymbol{$\to$}\AgdaSpace{}\AgdaRecord{Wrap}\AgdaSpace{}%
\AgdaFunction{F}\AgdaSpace{}\AgdaBound{x}\AgdaSpace{}\AgdaSymbol{$\to$}%
\AgdaSpace{}\AgdaSymbol{\_},
Agda will successfully unify the type of \AgdaBound{f} with the expected type of
the argument, setting $[x \coloneqq y]$.
However, without the \AgdaRecord{Wrap}, we would need to unify
\AgdaFunction{F}\AgdaSpace{}\AgdaBound{y} with
\AgdaFunction{F}\AgdaSpace{}\AgdaBound{x}, which would fail if \AgdaFunction{F}
were not injective, because there may be multiple acceptable values of
\AgdaBound{x} up to definitional equality.

The version of \AgdaRecord{Wrap} found in Agda's standard library is
significantly more complicated to allow for type families with arbitrarily many
arguments in a convenient syntax, using the $n$-ary functions of
\citet{Allais19}.
The version in the standard library is the one I use in this thesis.
In fact, both versions of the \AgdaRecord{Wrap} type family are the first pieces
of novel work to be presented in this thesis.

\subsection{Colours}

I use the ``Conor colours'' option for Agda syntax highlighting.
This set of colours is inspired by Conor McBride's syntax highlighting for
Epigram 2.
The colour given to a name is determined by the type of declaration that name
is bound to.
The main colours are \AgdaDatatype{blue} for types and type families,
\AgdaField{red} for constructors of data types and fields of records,
\AgdaFunction{green} for definitions which may unfold/compute, and
\AgdaBound{purple} for local variables.

Separately, I use \gr{green} in many places for usage annotations in traditional
typeset mathematical notation.
This usage of green contrasts only with ordinary black text.
