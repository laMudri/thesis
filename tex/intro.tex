\chapter{Introduction}

This thesis advances the frontier of programming languages work that can be done
in a proof assistant.
I will elaborate on both of these aspects in the following paragraphs.

The main aim of programming language research is to find common patterns in
computer programs and represent these patterns via programming language
features.
As a basic example, consider functions in, say, a low level language like C, in
contrast to what these functions are compiled to in assembly language/machine
code.
In order to support function calls, the C runtime system manages a
\emph{call stack}.
A function call is compiled to a sequence of instructions which store the return
address and the values of the arguments on the stack, move the stack pointer,
and jump to the compiled code of the function.
A return from a function then places the return value into the appropriate
register, moves the stack pointer back, and jumps back to the return address.
Advances in programming language techniques are evaluated by the usefulness of
behaviours captured and goodness of the mathematical properties of the
abstractions produced.

Within programming language research, type theory is a methodology for enforcing
abstractions by classifying program behaviour.
The simplest and most common type systems classify programs by what values they
may return.
For example, the body of a C function with return type \texttt{char} is a
program that, if it returns a value, will return a character.
If that function has a parameter of type \texttt{int}, then we can compose it
with a function with return type \texttt{int} to build up a larger program.
We expect type systems to stop us from running programs with arguments of the
wrong types.
This holds of both static and dynamic type systems --- with a static type
system, the compiler will refuse to compile our code if we pass a \texttt{char}
value to a function expecting an \texttt{int}, while with a dynamic type system,
our program will do a check before running the function to make sure that we
have passed it an \texttt{int}.
The abstraction produced in such simple type systems is the idea that a function
will take arguments in accordance with its parameter types and produce a result
in accordance with its result type, and furthermore, as readers, we do not have
to inspect the function's implementation to see that these properties are true.

A more recent trend in type theory is to use types to describe other parts of a
program's behaviour than what range of values it may return.
For example, Java's \emph{checked exceptions} can be seen as part of the static
type system, in which programs are classified by which exceptions they may
throw.
More generally, \emph{effect systems} classify programs based on all of the
effects they may have --- such as reading input and writing to files, and also
internally defined effects, such as non-determinism and using an accumulator.
In a type system which tracks effects (an \emph{effect system}), programs by
default are pure (i.e.\ have no effects), with effects being opt-in.
Pure programs generally enjoy good properties, making them easy to reason about
and easy for an optimising compiler to optimise.

In this thesis, I consider the dual of effect systems ---
\emph{coeffect systems} --- as introduced by \citet{POM14}.
In a coeffect system, we are interested not in what extra behaviour a program
may exhibit (as with effects), but rather what extra abilities the context of a
program may provide.
Analogously to the case of effect systems, we typically restrict our
coeffect-free programs to be ``more pure'' than usual.
A standard example is to restrict to \emph{linear} programs, in which each
variable in the context is used exactly once.
The ability to duplicate and discard variables is then seen as a coeffect, which
can be tracked by a coeffect system.
Restricting to linear programs may seem like an arbitrary restriction at first,
but the expectation of linearity arises naturally in applications such as
file-handling, session-typed communication, and approaches to mutable memory.
I introduce linearity and its applications more fully in \cref{sec:linearity}.

%A type system lets us enforce in our language complex invariants which maintain
%abstraction boundaries while also ideally being machine-checkable.
%For example, we give functions function type to abstract away their
%implementation via call stacks and jumps and so on.

%In this thesis, the invariants of interest revolve around restricting the usage
%of variables.
%I introduce this topic thoroughly in \cref{sec:linearity}, but in brief\ldots

The work of this thesis relies upon type theory in two distinct ways.
Firstly, as I have introduced above, the main objects of study in this thesis
are programming languages with interesting type systems.
Secondly, type theory provides the basis of the proof assistant Agda I
use to implement the aforementioned programming languages and operations upon
them.
I will now introduce the idea of proof assistants.

A \emph{proof assistant}, also known as an \emph{interactive theorem prover}, is
a piece of software that allows for the encoding of mathematical definitions,
theorems, constructions, and proofs, and furthermore check that such encoded
proofs are correct and that such encoded constructions are well formed.
To truly by interactive, i.e.\ to actually assist, a proof assistant will
usually have a user interface which can read partial proofs, display
information about what more proof needs to be given, and provide actions that
will help complete the proof.

Proof assistants have seen increasing use in programming language research in
recent years.
The most obvious reason why working in a proof assistant is seen as beneficial
is that it ensures correctness.
If the proof assistant accurately implements a suitable mathematical foundation,
then any theorem proved in the proof assistant is guaranteed to be a true
theorem of that foundation.
These guarantees of correctness are particularly important when working with
combinatorially complex mathematical objects, proofs about which often require
the consideration of a large number of cases.
Programming language syntaxes are often such complex objects, motivating the use
of proof assistants when studying programming languages.

A second reason to use proof assistants is for the assistance they provide when
exploring a mathematical theory.
When we make a new definition, we may want to test how it works in a special
case, or what constructions it allows us to perform.
In a proof assistant, the assistance tools give us immediate feedback as to what
moves are and aren't allowed.
For example, if we define a complex type system, a proof assistant will let us
interactively build typing derivations, making clear any side conditions and
types of subderivations as we go.
Also, as I do later in this thesis, a proof assistant allows us to build a very
general theory, and practically use that theory directly in more specific cases
without losing rigour.

Thirdly, analogously to how a strong static type system can give us more
confidence when refactoring a program, the constant checking of proofs in a
proof assistant gives us the confidence to change definitions and lemmas knowing
that we will be guided towards the parts of our theory that need to be
correspondingly changed.
This can help if we are developing a new programming language with a changing
specification.

Finally, many proof assistants --- including Agda~\citep{Agda}, which I use in
this thesis --- double as programming languages themselves.
This means that we can write programs and prove properties of them using the same
tool.
Also, many theorems proven in such a proof assistant have computational content.
For example, if we prove a normalisation theorem for a programming language,
this will typically yield a (verified) normalisation algorithm for it, which we
can really run on a computer.
As such, a development in a proof assistant can provide a reference
implementation of a programming language, or even --- as with Idris
2~\citep{Brady21}, Lean 4~\citep{deMU21}, and Cedille~\citep{GRS16} --- the
actual implementation of a programming language.

\section{Outline of the thesis}

This thesis proceeds as follows.
The next two chapters, \cref{sec:simple,sec:linearity}, are introductory in
nature, and cover two largely independent strands of prior work.
In \cref{sec:simple}, I introduce existing methods of representing and reasoning
about type systems in proof assistants based on dependent type theory.
I start from well established representations of well scoped and well typed
terms, and develop these towards a recent approach to environment-based
semantics given by \citet{AACMM21}.
In \cref{sec:linearity}, I discuss the challenges faced when one extends a
treatment of a simple type system, such as that given in \cref{sec:simple}, to
modal and linear type systems.
We see that modal and linear type systems apparently violate some of the nice
properties of the simply typed $\lambda$-calculus we required in
\cref{sec:simple}.
I present a solution for intuitionistic S4 modal logic, but leave a solution for
linear logic to the following chapters.

In the following two chapters, \cref{sec:semirings,sec:ren-sub-lr}, I present a
calculus $\name$ parametrised by a partially ordered semiring of \emph{usage
annotations}.
In \cref{sec:semirings}, I define the calculus, give some possible extensions,
and show that it subsumes intuitionistic S4 modal logic and Intuitionistic
Linear Logic.
In \cref{sec:ren-sub-lr}, I show that $\name$ enjoys generalised versions of the
nice properties required in \cref{sec:simple}, and I proceed to give novel
definitions of simultaneous substitutions and their action on $\name$ terms.
These two chapters are adapted from the work of \citet{WA21}.

The remaining three main chapters,
\cref{sec:framework,sec:semantics,sec:example-semantics}, adapt the syntactic
and semantic framework of \citet{AACMM21}, as presented at the end of
\cref{sec:simple}, to semiring-annotated calculi.
\Cref{sec:framework,sec:semantics} generalise the work on $\name$ presented in
\cref{sec:semirings,sec:ren-sub-lr}, respectively.
\Cref{sec:framework} shows how to formally describe the syntax of an arbitrary
semiring-annotated calculus, following the constructions used in
\cref{sec:semirings}.
\Cref{sec:semantics} then provides the generic environment-based semantic
traversal on such syntaxes, providing renaming and substitution as per
\cref{sec:ren-sub-lr} for all syntaxes as special cases of the generic
traversal.
\Cref{sec:example-semantics} then gives further example uses of the generic
traversal.

Finally, I conclude with \cref{sec:conc}, which discusses the achievements of
this thesis and openings for future work.

\section{Naming and notation conventions}

I assume familiarity with the Curry-Howard correspondence~\citep{Howard80}
throughout this thesis.
I make no distinction between logics and type theories, and use terminology from
each interchangeably.
Each following bullet point lists a collection of synonyms.

\begin{itemize}
  \item assumption, hypothesis, variable
  \item proposition, formula, type
  \item connective, type former
  \item derivation, proof, term
  \item derivable (formula), inhabited (type)
\end{itemize}

I carry out mechanised constructions and proofs in the proof assistant and
programming language Agda~\citep{Agda}.
Agda is based on Martin-L\"{o}f's intensional dependent type theory, so I
similarly present non-mechanised constructions and proofs assuming a foundation
given by dependent type theory, in a style inspired by the HoTT
Book~\citep{hottbook}.
I give a fuller introduction to Agda in \cref{sec:agda-primer}.

This thesis is written in \colour{}, but should be readable without.
Agda code has syntax highlighting, and various pieces of notation related to
usage annotations are coloured \gr{green} for emphasis.
