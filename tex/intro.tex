\chapter{Introduction}

FIXME: the start of this chapter is directly from the ESOP22 paper.

In this paper, we treat the metatheory of a class of substructural type
systems related to linear logic~\cite{girard87linear}.
This class is variously known as
\emph{coeffectful}~\cite{PetricekOM14,Granule18},
\emph{quantitative}~\cite{BrunelGMZ14,Atkey18}, or
\emph{resource-aware}~\cite{GhicaS14},
or is given no particular name~\cite{reed10distance,abadi99core},
and generalises bounded linear logic to track variable usage with semiring
annotations.
In all of these systems, we have some ambient semiring $\Ann$, and in the
judgements of the type system, variables are annotated by elements of $\Ann$
describing \emph{how} that variable can be used.
The additive structure of $\Ann$ gives the ability to count, or otherwise
accumulate, usages of variables in multiple subterms.
The multiplicative structure gives rise to a form of modality, for example
allowing multiple or unlimited reuse, or movement between security levels, in
the type system.

The aspect of such systems we tackle here is their basic metatheory and
mechanisation thereof.

We build upon both the general structural framework of
\citet{AACMM21} and the substructural techniques of \citet{WA21}.
The way \citeauthor{AACMM21} consolidate and codify mechanisation techniques for
propositional natural deduction systems based on intrinsically typed syntax and
de Bruijn indices, we aim to replicate for linear-like systems based on
semiring usage annotations.
By picking a trivial semiring, our work can subsume that of
\citeauthor{AACMM21}, except for the many pieces of machinery we have not yet
ported to this new framework.

Our work complements that of \citet{Granule18} on the Granule programming
language.
Where Granule focuses on writing programs \emph{in} the language and running
them, we focus on metatheoretic reasoning \emph{about} type systems.
%As such, whereas Granule has a convenient syntax, performant type-checker,
%interpreter

Our work is similar in scope to that of \citet{LicataSR17}, though we work in
a natural deduction style rather than a sequent calculus style.
Where \citeauthor{LicataSR17} are much more agnostic in terms of
substructurality --- allowing for non-commutative and bunched logics ---
we are much more agnostic in terms of syntax.
The system of \citeauthor{LicataSR17} is essentially a single calculus,
supporting ``product'' ($\mathrm F$) types and ``function'' ($\mathrm U$)
types, parametrised on a \emph{mode theory} describing its structural rules.
For this system, they derive the strong result of cut elimination.
Meanwhile, we leave syntax design to the user, and consequently can only
guarantee substitution (which we can only get because of our commitment to
natural deduction).

\section{Naming and notation conventions}

I assume familiarity with the Curry-Howard correspondence throughout this
thesis.
I make no distinction between logics and type theories, and use terminology from
each interchangeably.
Each following bullet point lists a collection of synonyms.

\begin{itemize}
  \item assumption, hypothesis, variable
  \item derivation, proof, term
  \item proposition, formula, type
  \item connective, type former
\end{itemize}

I carry out mechanised constructions and proofs in the proof assistant and
programming language Agda~\citep{Agda}.
Agda is based on Martin-L\"{o}f's intensional dependent type theory, so I
similarly present non-mechanised constructions and proofs assuming a foundation
given by dependent type theory, in a style inspired by the HoTT
Book~\citep{hottbook}.

\section{Outline of the thesis}

This thesis proceeds as follows.
The next two chapters, \cref{sec:simple,sec:linearity}, are introductory in
nature, and cover two largely independent strands of prior work.
In \cref{sec:simple}, I introduce existing methods of representing and reasoning
about type systems in proof assistants based on dependent type theory.
I start from well established representations of well scoped and well typed
terms, and develop these towards a recent approach to environment-based
semantics given by \citet{AACMM21}.
In \cref{sec:linearity}, I discuss the challenges faced when one extends a
treatment of a simple type system, such as that given in \cref{sec:simple}, to
modal and linear type systems.
We see that modal and linear type systems apparently violate some of the nice
properties of the simply typed $\lambda$-calculus we required in
\cref{sec:simple}.
I present a solution for intuitionistic S4 modal logic, but leave a solution for
linear logic to the following chapters.

In the following two chapters, \cref{sec:semirings,sec:ren-sub-lr}, I present a
calculus $\name$ parametrised by a partially ordered semiring of \emph{usage
annotations}.
In \cref{sec:semirings}, I define the calculus, give some possible extensions,
and show that it subsumes intuitionistic S4 modal logic and Intuitionistic
Linear Logic.
In \cref{sec:ren-sub-lr}, I show that $\name$ enjoys generalised versions of the
nice properties required in \cref{sec:simple}, and I proceed to give novel
definitions of simultaneous substitutions and their action on $\name$ terms.
These two chapters are adapted from the work of \citet{WA21}.

The remaining three main chapters,
\cref{sec:framework,sec:semantics,sec:example-semantics}, adapt the syntactic
and semantic framework of \citet{AACMM21}, as presented at the end of
\cref{sec:simple}, to semiring-annotated calculi.
\Cref{sec:framework,sec:semantics} generalise the work on $\name$ presented in
\cref{sec:semirings,sec:ren-sub-lr}, respectively.
\Cref{sec:framework} shows how to formally describe the syntax of an arbitrary
semiring-annotated calculus, following the constructions used in
\cref{sec:semirings}.
\Cref{sec:semantics} then provides the generic environment-based semantic
traversal on such syntaxes, providing renaming and substitution as per
\cref{sec:ren-sub-lr} for all syntaxes as special cases of the generic
traversal.
\Cref{sec:examples} then gives further example uses of the generic traversal.

Finally, I conclude with \cref{sec:conc}, which discusses the achievements of
this thesis and openings for future work.
