\chapter{Introduction}

FIXME: this chapter is directly from the ESOP22 paper.

In this paper, we treat the metatheory of a class of substructural type
systems related to linear logic~\cite{girard87linear}.
This class is variously known as
\emph{coeffectful}~\cite{PetricekOM14,Granule18},
\emph{quantitative}~\cite{BrunelGMZ14,Atkey18}, or
\emph{resource-aware}~\cite{GhicaS14},
or is given no particular name~\cite{reed10distance,abadi99core},
and generalises bounded linear logic to track variable usage with semiring
annotations.
In all of these systems, we have some ambient semiring $\Ann$, and in the
judgements of the type system, variables are annotated by elements of $\Ann$
describing \emph{how} that variable can be used.
The additive structure of $\Ann$ gives the ability to count, or otherwise
accumulate, usages of variables in multiple subterms.
The multiplicative structure gives rise to a form of modality, for example
allowing multiple or unlimited reuse, or movement between security levels, in
the type system.

The aspect of such systems we tackle here is their basic metatheory and
mechanisation thereof.

We build upon both the general structural framework of
\citet{AACMM21} and the substructural techniques of \citet{WA21}.
The way \citeauthor{AACMM21} consolidate and codify mechanisation techniques for
propositional natural deduction systems based on intrinsically typed syntax and
de Bruijn indices, we aim to replicate for linear-like systems based on
semiring usage annotations.
By picking a trivial semiring, our work can subsume that of
\citeauthor{AACMM21}, except for the many pieces of machinery we have not yet
ported to this new framework.

Our work complements that of \citet{Granule18} on the Granule programming
language.
Where Granule focuses on writing programs \emph{in} the language and running
them, we focus on metatheoretic reasoning \emph{about} type systems.
%As such, whereas Granule has a convenient syntax, performant type-checker,
%interpreter

Our work is similar in scope to that of \citet{LicataSR17}, though we work in
a natural deduction style rather than a sequent calculus style.
Where \citeauthor{LicataSR17} are much more agnostic in terms of
substructurality --- allowing for non-commutative and bunched logics ---
we are much more agnostic in terms of syntax.
The system of \citeauthor{LicataSR17} is essentially a single calculus,
supporting ``product'' ($\mathrm F$) types and ``function'' ($\mathrm U$)
types, parametrised on a \emph{mode theory} describing its structural rules.
For this system, they derive the strong result of cut elimination.
Meanwhile, we leave syntax design to the user, and consequently can only
guarantee substitution (which we can only get because of our commitment to
natural deduction).

This paper proceeds as follows.
In \cref{sec:algebra}, we review and fix conventions pertaining to partially
ordered semirings and vectors over them.
In \cref{sec:bunched-rules}, we introduce an informal meta-syntax allowing us
to state substructural typing rules succinctly and without explicit reference
to contexts.
In \cref{sec:syntax}, we mechanise that meta-syntax, giving a type of
\emph{descriptions} of type systems, and interpreting those descriptions as
types of intrinsically typed terms.
In \cref{sec:env}, we discuss usage-aware environments: a generalisation of
the structures used in simultaneous renaming and substitution proofs.
We use environments in \cref{sec:semantics} to state an alternative
elimination principle for terms, and give examples of such eliminations in
\cref{sec:examples}.
The examples are syntax-generic renaming and substitution, a specific
denotational semantics, and a syntax-generic usage elaborator.
Finally, we conclude and discuss future work in \cref{sec:conc}

The work presented in this paper has been mechanised in Agda,
with the code available for building upon~\cite{generic-lr}.
