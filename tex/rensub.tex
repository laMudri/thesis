\chapter{Renaming and substitution for $\name$}\label{sec:ren-sub-lr}

In \cref{sec:semirings}, I defined my calculus of interest $\name$.
In this chapter, I develop the necessary syntactic metatheory for
specifying and implementing the simultaneous substitution operation.
My aim in this chapter is to replay the development in \cref{sec:kits}, making
minimal changes to lemmas and theorems so as to support semiring usage
annotations.
The big change I make in this chapter is to the definition of
\emph{environment}.
In terms of ideas, I believe that the definition of environment I give in
\cref{sec:lrkits} is the main contribution of this thesis.
Its primary novel feature is the use of a linear map --- in the sense found in
linear algebra --- to distribute usage annotations across the values in the
environment.

The resulting substitution operation bears some similarity to a multi-variable
substitution operation given by \citet{petricek-thesis}.
This is the only other presentation I have found in the literature of an
operation similar to simultaneous substitution for a linear or linear-like
calculus.
I discuss Petricek's substitution lemma in more detail in \cref{sec:petricek}.
In brief, relative to Petricek's work, I extract the environment from the
substitution lemma, tidy up the definition, and present usage-annotated
environments as a reusable tool and object of study in their own right.
I reuse the resulting definition of environment in the remainder of this thesis,
taking advantage of the properties and generality of definition established in
this chapter, particularly in \cref{sec:semantics,sec:example-semantics}.

The body of this chapter starts with a motivation for and a definition of
environment in the usage-annotated setting in \cref{sec:lrkits}.
I then give some key properties of and operations on these environments in
\cref{sec:lenv}.
I apply the previous definitions and properties to the problems of simultaneous
renaming and substitution for $\name$ in \cref{sec:lrsub}, which also connects
the previous sections' mathematical developments with Agda code.
I then compare the yielded substitution operation with the one given by Petricek
in \cref{sec:petricek}, before a concluding discussion in
\cref{sec:ren-sub-lr-conc}.

\section{What are linear renaming and substitution?}\label{sec:lrkits}
\paragraph{New explanation}
Recalling from \cref{sec:kits}, we have the following definition of
\emph{environments} for simple types.

\begin{definition}[Simple environment]
  A $\V$-\emph{environment} between simply typed contexts $\Gamma$ and $\Delta$
  is a function, polymorphic in type $A$, from variables of type $A$ in
  $\Delta$ to inhabitants of $\V\,\Gamma\,A$.
  We write the type of such environments as $\Gamma \env\V \Delta$.
\end{definition}

\begin{definition}[Simple recursive environment]
  A \emph{recursive $\V$-environment} between simply typed contexts $\Gamma$ and
  $\Delta$ is defined by cases on the shape of $\Delta$ (where
  $\Gamma \env\V_R \Delta$ is the notation for the type of recursive
  environments for given $\V$, $\Gamma$, and $\Delta$):
  \begin{itemize}
    \item If $\Delta$ is empty, then there is one environment.
    \item If $\Delta$ is a concatenation $\Delta_l, \Delta_r$, then an
      environment is a pair of environments of types
      $\Gamma \env\V_R \Delta_l$ and $\Gamma \env\V_R \Delta_r$.
    \item If $\Delta$ is a singleton $A$, then an environment is a value of
      type $\V\,\Gamma\,A$.
  \end{itemize}
\end{definition}

\begin{definition}[Usage-annotated recursive environment]
  A \emph{recursive $\V$-environment} between annotated contexts $\Gamma$ and
  $\Delta$ is defined by cases on the shape of $\Delta$ (where
  $\Gamma = \grP\gamma$ and $\Gamma \env\V_R \Delta$ is the notation for the
  type of recursive environments for given $\V$, $\Gamma$, and $\Delta$):
  \begin{itemize}
    \item If $\Delta$ is empty, then an environment exists if $\grP = \gr0$.
    \item If $\Delta$ is a concatenation $\Delta_l, \Delta_r$, then an
      environment is a choice of usage vectors $\grPl$ and $\grPr$ such that
      $\grP = \grPl + \grPr$ and we have a pair of environments of types
      $\grPl\gamma \env\V_R \Delta_l$ and $\grPr\gamma \env\V_R \Delta_r$.
    \item If $\Delta$ is a singleton $\gr rA$, then an environment is a choice
      of a usage vector $\grPprime$ such that $\grP = \gr r\grPprime$ and we
      have a value of type $\V\,\grPprime\gamma\,A$.
  \end{itemize}
\end{definition}

From this definition, we can recover a functional-style definition by
separating choices of usage vectors from the provision of $\V$-values.
In particular, the only choices of usage vectors that are essential are the
$\grPprime$s in the singleton case, with the choices in the concatenation case
being determined as scalings and sums of these $\grPprime$s.
I let $\gr\Psi$ collect up these $\size\Delta$-many choices of
$\size\Gamma$-length usage vectors and note that the constraint on $\gr\Psi$
generated by all the scaling and summing is
$\grP = \sum_{\plr{x : \gr rA} \in \Delta} \gr r\gr\Psi_x$.

\begin{definition}[Usage-annotated environment (tentative)]
  A \emph{$\V$-environment} between annotated contexts $\Gamma$ and $\Delta$
  (written $\grP\gamma$ and $\grQ\delta$, respectively, when convenient)
  is a matrix $\gr\Psi : \Ann^{\size\Delta \times \size\Gamma}$ such that
  $\grP = \sum_{\plr{x : \gr rA} \in \Delta} \gr r\gr\Psi_x$ and for each
  $\plr{x : A} \in \delta$ we have a value of type $\V\,\gr\Psi_x\gamma\,A$.
\end{definition}

We may note, further, that the constraint
$\grP = \sum_{\plr{x : \gr rA} \in \Delta} \gr r\gr\Psi_x$ can be stated as the
vector-matrix multiplication $\grP = \grQ\gr\Psi$.
Using the same operation, we have that $\gr\Psi_x = \langle x \rvert\gr\Psi$.
Because $\langle x \rvert$ is exactly the $\grQprime$ such that
$\plr{x : A} \sqin \grQprime\delta$, we can rephrase the function producing
$\V$-values as: for each $A$, $\grPprime$, and $\grQprime$ such that
$\grPprime = \grQprime\gr\Psi$, a function from $\grQprime\delta \sqni A$ to
$\V\,\grPprime\gamma\,A$.
Finally, I choose to switch from matrices and matrix multiplication to
linear maps and their actions, which are easier to work with.
All of these changes yield my primary definition of an environment for
usage-annotated calculi.

\begin{definition}[Usage-annotated environment]
  A \emph{$\V$-environment} between annotated contexts $\Gamma$ and $\Delta$
  (written $\grP\gamma$ and $\grQ\delta$, respectively, when convenient)
  is a linear map $\gr\Psi : \Ann^{\size\Delta} \to \Ann^{\size\Gamma}$ (written
  postfix) such that $\grP = \grQ\gr\Psi$ and for each $A$, $\grPprime$, and
  $\grQprime$ such that $\grPprime = \grQprime\gr\Psi$, a function from
  $\grQprime\delta \sqni A$ to $\V\,\grPprime\gamma\,A$.
\end{definition}

\paragraph{Old explanation}
As we discussed in \cref{sec:kits}, simultaneous substitution gives a
notion of derivability of one context from another, while simultaneous renaming
gives a similar notion of derivability restricted to structural rules.
To adapt these notions from an intuitionistic setting to our substructural
setting, we must examine what it means to derive one context from another
substructually.

In the intuitionistic case, we say that to derive a context $\Delta$ from a
context $\Gamma$ is to derive each element $\Delta_i$ from $\Gamma$.
We may justify this by an intermediate step --- noting that contexts are
understood to be Cartesian products of their elements, and giving a map into
a Cartesian product is the same as giving a map into each factor.
I picture this definition as the diagram below.

\begin{displaymath}
  \begin{tikzpicture}[baseline]
    \path
    (-1,1) node (Gtop) {}
    (-1,0) node (G) {$\Gamma$}
    (-1,-1) node (Gbot) {}
    ;
    \node[draw,dotted,fit=(Gtop) (G) (Gbot)] (GG) {};

    \path
    (1,1) node (Dtop) {}
    (1,0) node (D) {$\Delta$}
    (1,-1) node (Dbot) {}
    ;
    \node[draw,dotted,fit=(Dtop) (D) (Dbot)] (DD) {};

    \draw[->,double] (GG) -- (DD);
  \end{tikzpicture}
  \coloneqq
  \begin{tikzpicture}[baseline]
    \path
    (-1,1) node (Gtop) {}
    (-1,0) node (G) {$\Gamma$}
    (-1,-1) node (Gbot) {}
    ;
    \node[draw,dotted,fit=(Gtop) (G) (Gbot)] (GG) {};

    \path
    (1,1) node[draw] (Dtop) {$\Delta_1$}
    (1,0) node (D) {$\vdots$}
    (1,-1) node[draw] (Dbot) {$\Delta_n$}
    ;

    \fill[green!20!white,opacity=1] (GG.north east)
    parabola[bend at end] (Dtop.west)
    parabola[bend at start] (GG.south east)
    -- cycle;
    \fill[blue!40!white,opacity=.5] (GG.north east)
    parabola[bend at end] (Dbot.west)
    parabola[bend at start] (GG.south east)
    -- cycle;

    \draw[->] (GG.north east) parabola[bend at end] (Dtop.west);
    \draw (GG.south east) parabola[bend at end] (Dtop.west);
    \draw[->] (GG.north east) parabola[bend at end] (D.west);
    \draw (GG.south east) parabola[bend at end] (D.west);
    \draw[->] (GG.north east) parabola[bend at end] (Dbot.west);
    \draw (GG.south east) parabola[bend at end] (Dbot.west);
  \end{tikzpicture}
\end{displaymath}

To see how this definition works, let us construct an example substitution:
$A, B \to C, B \Longrightarrow B, C$.
Because the codomain is a Cartesian product, it suffices to give two separate
substitutions, $A, B \to C, B \Longrightarrow B$ and
$A, B \to C, B \Longrightarrow C$, with a substitution into a singleton
context being just a term.
We indeed have terms $x : A, y : B \to C, z : B \vdash y : B$ and
$x : A, y : B \to C, z : B \vdash y\,z : C$.
It is also instructive to look at an identity substitution (which is also a
renaming), $A, B \Longrightarrow A, B$, witnessed by terms
$x : A, y : B \vdash x : A$ and $x : A, y : B \vdash y : B$.

When working with our semiring-annotated calculus \name{}, contexts are no
longer understood as Cartesian products.
This means that substitutions of type $\Gamma \Longrightarrow \Delta$ are no
longer equivalent to collections of substitutions
$\Gamma \Longrightarrow \Delta_i$.
Indeed, notice that we should still have an identity substitution of type
$\gr1A, \gr1B \Longrightarrow \gr1A, \gr1B$, but we do not have terms proving
either $\gr1A, \gr1B \vdash A$ or $\gr1A, \gr1B \vdash B$.
What we do have are terms $x : \gr1A, y : \gr0B \vdash x : A$ and
$x : \gr0A, y : \gr1B \vdash y : B$, and if we pointwise add together the
annotations of the two terms, we get back the original context
$x : \gr1A, y : \gr1B$.
Furthermore, adding up the annotations is not just a random operation;
linear contexts are understood to be tensor products of their elements, and
introduction rule for the tensor product involves summing the annotations of
the two sides.

For any annotated context $\Delta$, we have
$\Delta \vdash \bigotimes_{(\gr rx : A) \in \Delta}\oc\gr rA$ by iterated
application of $\otimes$-I with $\oc$-I and Var at the leaves.
Let $\Gamma = \grP\gamma$ and $\Delta = \grQ\delta$.
If we are to produce substitutions from $\Gamma$ to $\Delta$ in this
pattern, we simulate the applications of $\otimes$-I by producing, for each
element in $\Delta$, a usage context for $\gamma$ such that the whole collection
sums to $\grP$, then simulate the applications of $\oc$-I by dividing each of
the new usage contexts by the corresponding annotation in $\Delta$, calling
the divided usage contexts $\gr\Psi_x$, and finally, instead of a variable
from $\delta$, we give a term of type $\gr\Psi_x\gamma \vdash \delta_x$.
In summary, the constraint on the collection of usage contexts $\gr\Psi$ is
that $\grP = \sum_{(\gr rx : A) \in \Delta}\gr r\gr\Psi_x$.
Moreover, if we take $\grP$ and $\grQ$ to be row vectors and $\gr\Psi$ to be a
matrix, the latter expression is equal to the vector-matrix multiplication
$\grQ\gr\Psi$.
The resulting definition of simultaneous substitution is depicted below.

\begin{displaymath}
  \begin{tikzpicture}[baseline]
    \path
    (-1,1) node (Gtop) {}
    (-1,0) node (G) {$\grP\gamma$}
    (-1,-1) node (Gbot) {}
    ;
    \node[draw,dotted,fit=(Gtop) (G) (Gbot)] (GG) {};

    \path
    (1,1) node (Dtop) {}
    (1,0) node (D) {$\grQ\delta$}
    (1,-1) node (Dbot) {}
    ;
    \node[draw,dotted,fit=(Dtop) (D) (Dbot)] (DD) {};

    \draw[->,double] (GG) -- (DD);
  \end{tikzpicture}
  \coloneqq
  \begin{tikzpicture}[baseline]
    \path
    (-1,1) node (Gtop) {}
    (-1,0) node (G) {$\grP\gamma$}
    (-1,-1) node (Gbot) {}
    ;
    \node[draw,dotted,fit=(Gtop) (G) (Gbot)] (GG) {};

    \path
    (1,3) node (G1top) {}
    (1,2) node (G1) {$\gr\Psi_1\gamma$}
    (1,1) node (G1bot) {}
    ;
    \node[draw,dotted,fit=(G1top) (G1) (G1bot)] (GG1) {};
    \draw[->] (GG) -- (GG1);

    \path (1,0) node {$\vdots$};

    \path
    (1,-1) node (Gntop) {}
    (1,-2) node (Gn) {$\gr\Psi_n\gamma$}
    (1,-3) node (Gnbot) {}
    ;
    \node[draw,dotted,fit=(Gntop) (Gn) (Gnbot)] (GGn) {};
    \draw[->] (GG) -- (GGn);

    \path
    (3,1) node[draw] (Dtop) {$\delta_1$}
    (3,0) node (D) {$\vdots$}
    (3,-1) node[draw] (Dbot) {$\delta_n$}
    ;

    \fill[green!20!white] (GG1.north east)
    parabola[bend at end] (Dtop.west)
    parabola[bend at start] (GG1.south east)
    -- cycle;
    \draw[->] (GG1.north east) parabola[bend at end] (Dtop.west);
    \draw (GG1.south east) parabola[bend at end] (Dtop.west);

    \fill[blue!20!white] (GGn.north east)
    parabola[bend at end] (Dbot.west)
    parabola[bend at start] (GGn.south east)
    -- cycle;
    \draw[->] (GGn.north east) parabola[bend at end] (Dbot.west);
    \draw (GGn.south east) parabola[bend at end] (Dbot.west);
  \end{tikzpicture}
  \quad\textrm{where }\grP = \grQ\gr\Psi
\end{displaymath}

In type theory, we write out the definition as follows.

\begin{displaymath}
  \sum_{\gr\Psi : \size\Delta \to \size\Gamma \to \Ann}
    \left(\grP = \grQ\gr\Psi\right) \times
    \prod_{(x : A) \in \delta}\gr\Psi_x\gamma \vdash A
\end{displaymath}

We can see the step-by-step construction of a substitution play out by adapting
the previous example to have type
$\gr0A, \gr2(B \multimap C), \gr3B \Longrightarrow \gr1B, \gr2C$.
To split the goal up, we note that $
\begin{pmatrix} \gr0 & \gr2 & \gr3 \end{pmatrix} =
\begin{pmatrix} \gr0 & \gr0 & \gr1 \end{pmatrix} +
\begin{pmatrix} \gr0 & \gr2 & \gr2 \end{pmatrix}
$, so it suffices to give substitutions of types
$\gr0A, \gr0(B \multimap C), \gr1B \Longrightarrow \gr1B$ and
$\gr0A, \gr2(B \multimap C), \gr2B \Longrightarrow \gr2C$.
Furthermore, our term calculus only supports $\gr1$-annotated conclusions,
so we divide the second substitution type through by $\gr2$.
Finally, we give the terms largely as before:
$\gr0x : A, \gr0y : B \to C, \gr1z : B \vdash y : B$ and
$\gr0x : A, \gr1y : B \to C, \gr1z : B \vdash z\,y : C$.

While we naturally derive a matrix as a fragmentation of a usage vector, we can
get a slightly cleaner presentation by instead using an abstract linear map.
Let $\gr\Psi$ now be a linear map of type
$\Ann^{\size\Delta} \to \Ann^{\size\Gamma}$, with application written postfix.
The equation $\grP = \grQ\gr\Psi$ remains unchanged.
Where we previously wrote $\gr\Psi_x$, the most direct replacement would be
$\langle x \rvert\gr\Psi$, with $\langle x \rvert$ being the $x$th basis row
vector.
But then we notice that $\langle x \rvert$ is exactly the $\grQprime$ satisfying
$\grQprime\delta \sqni x : B$.
This gives us the following definition, which can be verified by equationally
substituting $\grPprime$ and expanding the definition of $\sqni$.

\begin{displaymath}
  \sum_{\gr\Psi : \Ann^{\size\Delta} \to \Ann^{\size\Gamma}}
    \left(\grP = \grQ\gr\Psi\right) \times
    \prod_{A,\grQprime,\grPprime} \left(
    \grPprime = \grQprime\gr\Psi \to \grQprime\delta \sqni A \to
    \grPprime\gamma \vdash A\right)
\end{displaymath}

We now have a new reading for the interpretation of a linear substitution:
a linear map $\gr\Psi$ relating the two usage vectors $\grP$ and $\grQ$, and
for any two similarly related usage vectors $\grPprime$ and $\grQprime$, we
have a type-preserving function from variables in $\grQprime\delta$ to terms in
$\grPprime\gamma$.
Even though we don't use $\gr\Psi$ as a matrix containing fragmented usage
vectors, we can still justify why it should be a \emph{linear} map.
We need $\gr\Phi$ to respect all fragmentation of the usage context in a typing
rule, and we know that all such fragmentation is done by linear operations
zero, addition, and scaling by a constant.
\todo{Expand. Substitutions need to preserve everything done to the context,
and linear things are all we do to the context.}

Taking a lead from \cref{sec:kits}, we deduce a definition of
\emph{environment} by replacing the $\vdash$ in the definition of simultaneous
substitution by an arbitrary type family $\mathcal V$.
Letting $\mathcal V$ be $\sqni$ gives us a notion of simultaneous renaming,
allowing for renamings with types such as
$\gr6A, \gr0B, \gr1C \stackrel\sqni\Longrightarrow \gr1C, \gr2A, \gr4A$.

It is worth noting that, when contexts are Cartesian products, passing from
``a map into $\Delta$'' to ``for each $A \in \Delta$, a map into $A$'' does not
lose any generality because the universal property of Cartesian products
states that every map into a Cartesian product can be given factor-wise.
Hence, $\Delta$ and the one-element context $\prod_{A \in \Delta}A$ are
isomorphic in the category of contexts and simultaneous substitution.
However, tensor products are not limits, so don't have the same universal
property.
Indeed, many annotated contexts are not isomorphic to the weighted product of
their elements.
For example, we do not have a substitution of type
$\gr1(A \otimes B) \Longrightarrow \gr1A, \gr1B$ because we would first need
to pattern-match on the tensor product \emph{before} trying to derive the
target context.
This loss of generality is however justified when we consider the action of a
substitution.
Substitutions should only be replacing variables by terms, whereas if
substitutions were allowed to pattern-match before introducing the target, then
the substitution would have to replace the original term by a term that first
pattern-matches and then continues like the original term.

\section{Properties of linear environments}\label{sec:lenv}
\begin{remark}
  Given an environment $\rho : \grP\gamma \env\V \grQ\delta$ and a $\grPprime$
  and a $\grQprime$ such that $\grPprime = \grQprime\plr{\rho.\gr\Psi}$,
  there is also an environment of type $\grPprime\gamma \env\V \grQprime\delta$
  with the same linear map and action on variables.
\end{remark}
\begin{proof}
  The only part of the definition of an environment dependent on $\grP$ or
  $\grQ$ is the constraint $\grP = \grQ\gr\Psi$, which we are able to replace
  for $\grPprime$ and $\grQprime$.
\end{proof}

When constructing an environment, we can do so by cases on the shape of the
target context.
We can create an environment into the empty context when all usage annotations
on the source context are $\gr0$.
We can create an environment into a concatenated context when we can additively
split up the annotations of the source context and produce environments into
both halves from the split sources.
We can create an environment into a singleton context when there is a context
$\gr r$ times smaller than the source context in which we can produce a value
of the appropriate type.

\begin{lemma}\label{thm:construct-env}
  We can define all of the following equivalences for any values of the free
  variables.
  \begin{itemize}
    \item $\forallb{I \dotlr \plr{{-} \env\V {\cdot}}}$
    \item $\forallb{\plr{{-} \env\V \Gamma} \sep \plr{{-} \env\V \Delta}
      \dotlr \plr{{-} \env\V \Gamma, \Delta}}$
    \item
      $\forallb{\gr r \cdot \plr{\V\,(-)\,A} \dotlr \plr{{-} \env\V \gr rA}}$
  \end{itemize}
\end{lemma}
\begin{proof}
  There are 6 cases to check.
  Throughout, we write $\Gamma$ as $\grP\gamma$ and $\Delta$ as $\grQ\delta$
  when convenient.
  \begin{description}
    \item[$I(\to)$]
      Let $\gr\Psi$ be the unique linear map out of the zero space.
      By definition, $\gr0 = \grQ\gr\Psi$.
      There are no variables to act upon.
    \item[$I(\gets)$]
      $\grQ\gr\Psi$ is an empty sum, so if $\grP = \grQ\gr\Psi$ then
      $\grP = \gr0$.
    \item[$\sep(\to)$]
      Let the given environments be $\rho : \grRl\theta \env\V \Gamma$ and
      $\sigma : \grRr\theta \env\V \Delta$, with $\grR = \grRl + \grRr$.
      Define $\gr\Psi \coloneqq [\rho.\gr\Psi, \sigma.\gr\Psi]$, using the
      coproduct structure of the concatenated vector space.
      We have $\grR = \grRl + \grRr =
      \grP\plr{\rho.\gr\Psi} + \grQ\plr{\sigma.\gr\Psi} =
      \plr{\grP, \grQ}\gr\Psi$.
      To act on variables, we are given
      $\grRprime = \plr{\grPprime, \grQprime}\gr\Psi$ and
      $\grPprime\gamma, \grQprime\delta \sqni A$.
      Without loss of generality, let us have $\grPprime\gamma \sqni A$ and
      $\grQprime = \gr0$.
      Thus, $\grRprime = \grPprime\plr{\rho.\gr\Psi}$, and we can act on the
      variable using $\rho$.
    \item[$\sep(\gets)$]
      Let the unnamed context be $\Theta$, also written $\grR\theta$.
      The linear map
      $\gr\Psi : \Ann^{\size\Gamma + \size\Delta} \to \Ann^{\size\Theta}$ splits
      into
      $\gr\Psi_{\gr l} : \Ann^{\size\Gamma} \to \Ann^{\size\Theta}
      \coloneqq \langle \id, 0 \rangle; \gr\Psi$ and
      $\gr\Psi_{\gr r} : \Ann^{\size\Delta} \to \Ann^{\size\Theta}
      \coloneqq \langle 0, \id \rangle; \gr\Psi$, using the product structure of
      the concatenated vector space.
      Let $\grRl \coloneqq \grP\gr\Psi_{\gr l}$ and
      $\grRr \coloneqq \grQ\gr\Psi_{\gr r}$, by definition satisfying the
      required equations.
      For the action on variables, let us consider the left environment.
      We are given $\grRprime = \grPprime\gr\Psi_{\gr l}$ and
      $\grPprime\gamma \sqni A$.
      From these, we get
      $\grRprime = \grPprime\gr\Psi_{\gr l} = \plr{\grPprime, \gr0}\gr\Psi$ and
      $\grPprime\gamma, \gr0\delta \sqni A$.
      We can therefore act using the original environment.
    \item[$\cdot(\to)$]
      Let $\grP$ and $\grPprime$ be such that $\grP = \gr r\grPprime$ and let
      $v : \V\,\grPprime\gamma\,A$.
      Let $\gr\Psi : \Ann \to \Ann^{\size\gamma}
      \coloneqq \gr r\gr' \mapsto \gr r\gr'\grPprime$.
      By definition and the previous assumption, we have $\grP = \gr r\gr\Psi$.
      When acting on a variable, we have $\grP\gr{''} = \gr r\gr'\gr\Psi$
      and $\gr r\gr'A \sqni A'$.
      The latter tells us that $A = A'$ and $\gr r\gr' = \gr1$.
      Thus, $\grP\gr{''} = \grPprime$.
      We therefore need a value of type $\V\,\grPprime\gamma\,A$, which we can
      take to be $v$.
    \item[$\cdot(\gets)$]
      Let us have an environment of type $\grP\gamma \env\V \gr rA$.
      We want to use its action on variables to yield a value.
      To do this, we let $\grPprime \coloneqq \gr1\gr\Psi$, and use this
      equation, together with the fact that we have a variable of type
      $\gr1A \sqni A$, to get a value of type $\V\,\grPprime\gamma\,A$.
      Furthermore, we derive $\grP = \gr r\gr\Psi = \gr r\grPprime$, as
      required.
  \end{description}
\end{proof}

We could, indeed, use these three clauses to define what an environment is.
However, I find them difficult to work with, as it is often easier to do
linear algebraic proofs separately from the rest of an environment.
For identity and composition, as we are about to see, the original definition
is easier to use because we can rely on the identity and composition of linear
maps.
Concretely, an inductive proof of identity would, for example, involve
constructing an environment of type
$\grP\gamma, \grQ\delta \env\V \grP\gamma, \grQ\delta$ by constructing
environments of types $\grP\gamma, \gr0\delta \env\V \grP\gamma$ and
$\gr0\gamma, \grQ\delta \env\V \grQ\delta$.
These are not identity environments, so we would have to strengthen the
induction hypothesis.

One of the primary test cases for environments is simultaneous substitution,
which will look like the following rule.
The admissibility of substitution will be by induction on the derivation of
$\Delta \vdash A$, so we will need to be able to adapt any environment we are
given to work with any possible context of new premises.
In the simply typed case, the only change to the context we encountered was the
binding of new variables.
Now, with usage annotations, we furthermore have linear decompositions of the
context, necessitating changes to the environment whenever usage annotations
change.
I will deal first with linear decompositions.

\begin{displaymath}
  \begin{prooftree}
    \hypo{\Gamma \env{\vdash} \Delta}
    \hypo{\Delta \vdash A}
    \infer2[sub]{\Gamma \vdash A}
  \end{prooftree}
\end{displaymath}

There are three kinds of linear decompositions we have to deal with: zero,
addition, and scaling; corresponding to bunched connectives $I^*$, $\sep$, and
$\gr r \cdot {}$, respectively.
In each case, we have a simple preservation lemma, transforming an environment
of type $\Gamma \env\V \Delta$ and a decomposition of $\Delta$ into a
decomposition of $\Gamma$ and environments for all of the decomposed fragments
of $\Gamma$ and $\Delta$.

\begin{lemma}[environments preserve zero]\label{thm:lr-env-zero}
  Given an environment of type $\grP\gamma \env\V \grQ\delta$ such that
  $\grQ \leq \gr 0$, we also have that $\grP \leq \gr 0$.
\end{lemma}
\begin{proof}
  $\grP \leq \grQ\gr\Psi \leq \gr0\gr\Psi = \gr0$, by environment
  compatibility and monotonicity and linearity of $\gr\Psi$.
\end{proof}

\begin{lemma}[environments preserve addition]\label{thm:lr-env-add}
  Given an environment of type $\grP\gamma \env\V \grQ\delta$ such that
  $\grQ \leq \grQl + \grQr$ for some $\grQl$ and $\grQr$, we also have $\grPl$
  and $\grPr$ such that $\grP \leq \grPl + \grPr$ and there are environments
  of types $\grPl\gamma \env\V \grQl\delta$ and
  $\grPr\gamma \env\V \grQr\delta$.
\end{lemma}
\begin{proof}
  Let $\grPl \coloneqq \grQl\gr\Psi$ and $\grPr \coloneqq \grQr\gr\Psi$.
  Then, $\grP \leq \grQ\gr\Psi \leq \plr{\grQl + \grQr}\gr\Psi =
  \grQl\gr\Psi + \grQr\gr\Psi = \grPl + \grPr$, satisfying the first condition.
  Because clearly $\grPl \leq \grQl\gr\Psi$ and $\grPr \leq \grQr\gr\Psi$,
  \cref{thm:env-resize} on the original environment gives us the required
  pair of new environments.
\end{proof}

\begin{lemma}[environments preserve scaling]\label{thm:lr-env-scale}
  Given an environment of type $\grP\gamma \env\V \grQ\delta$ such that
  $\grQ \leq \gr r\grQprime$ for some $\grQprime$, we also have a $\grPprime$
  such that $\grP \leq \gr r\grPprime$ and there is an environment of type
  $\grPprime\gamma \env\V \grQprime\delta$.
\end{lemma}
\begin{proof}
  Let $\grPprime \coloneqq \grQprime\gr\Psi$.
  Then, $\grP \leq \grQ\gr\Psi \leq \plr{\gr r\grQprime}\gr\Psi =
  \gr r\plr{\grQprime\gr\Psi} = \gr r\grPprime$, satisfying the first condition.
  Because clearly $\grPprime \leq \grQprime\gr\Psi$,
  \cref{thm:env-resize} on the original environment gives us the required
  new environment.
\end{proof}

Finally, I will also take the opportunity to give the bind lemma, allowing
environments to incorporate newly bound variables.
In the intuitionistic case, the bind lemma had two requirements on $\V$: $\V$
admits weakening and we can map variables into $\V$-values.
With usage annotations, the former is unreasonable, but it turns out that we
only need weakening by variables whose usage annotation is less than or equal
to $\gr0$.
The latter stays as-is, with the note that ``variable'' now means a
usage-checked variable.

\begin{lemma}[bind]\label{thm:lr-bind}
  Given functions
  ${\swarrow^k} : \forall \Gamma, \grR, \theta.~\grR \leq \gr0 \to
  \forallb{\V\,\Gamma \dotto \V\,\plr{\Gamma, \grR\theta}}$ and
  $\mathrm{vr} : \forallb{{\sqni} \dotto \V}$, we can turn an environment of
  type $\Gamma \env\V \Delta$ into an environment of type
  $\Gamma, \Theta \env\V \Delta, \Theta$ for any context $\Theta$.
\end{lemma}
\begin{proof}
  Let $\grP\gamma \coloneqq \Gamma$, $\grQ\delta \coloneqq \Delta$, and
  $\grR\theta \coloneqq \Theta$.
  Let the new linear map $\gr\Psi\gr' : \Ann^{\size\Delta + \size\Theta} \to
  \Ann^{\size\Gamma + \size\Theta}$ be $\gr\Psi \oplus \gr I$.
  That is, in block matrix notation,
  $\begin{pmatrix} \gr\Psi & \gr0 \\ \gr0 & \gr I \end{pmatrix}$.
  Checking that this linear map fits, we have
  $\begin{pmatrix}\grP & \grR\end{pmatrix}
  \leq \begin{pmatrix}\grQ\gr\Psi & \grR\gr I\end{pmatrix}
  = \begin{pmatrix}\grQ & \grR\end{pmatrix}\plr{\gr\Psi \oplus \gr I}$.
  For the action on variables, we are given vectors $\grPprime$,
  $\grR\gr'_\grP$, $\grQprime$, and $\grR\gr'_\grQ$ such that
  $\begin{pmatrix} \grPprime & \grR\gr'_\grP \end{pmatrix} \leq
  \begin{pmatrix} \grQprime & \grR\gr'_\grQ \end{pmatrix}
  \plr{\gr\Psi \oplus \gr I}$ and we have a variable of type
  $\grQprime\delta, \grR\gr'_\grQ\theta \sqni A$ for some type $A$.
  The constraint on the new vectors reduces to $\grPprime \leq \grQprime\gr\Psi$
  and $\grR\gr'_\grP \leq \grR\gr'_\grQ$.
  From the variable we either have a variable $x$ in $\delta$ with
  $\grQprime \leq \langle x \rvert$ and $\grR\gr'_\grQ \leq \gr0$, or a
  variable $y$ in $\theta$ with $\grQprime \leq \gr0$ and
  $\grR\gr'_\grQ \leq \langle y \rvert$.
  In the former case, the action of the original environment on $x$ gives us a
  $\V$-value in $\grPprime\gamma$, and the $\gr0$-weakening principle
  $\swarrow^k$, noting that $\grR\gr'_\grP \leq \grR\gr'_\grQ \leq \gr0$, gives
  us a $\V$-value in $\grPprime\gamma, \grR\gr'_\grP\theta$.
  In the latter case, we have that
  $\begin{pmatrix} \grPprime & \grR\gr'_\grP \end{pmatrix}
  \leq \begin{pmatrix} \grQprime\gr\Psi & \grR\gr'_\grQ \end{pmatrix}
  \leq \begin{pmatrix} \gr0\gr\Psi & \langle y \rvert \end{pmatrix}
  = \begin{pmatrix} \gr0 & \langle y \rvert \end{pmatrix}
  = \left\langle {\searrow}y \right\rvert$, so $y$ also serves as a
  usage-checked variable in $\grPprime\gamma, \grR\gr'_\grP\theta$.
  From this usage-checked variable, we get a $\V$-value in the same context
  using $\mathrm{vr}$.
\end{proof}

The requirements for identity and composition of environments look a bit like
the unit and lift of a Kleisli triple.

\begin{lemma}[Identity environment]
  Given a function $\mathrm{vr} : \forallb{{\sqni} \dotto \V}$, for any
  $\Gamma$ we have an environment of type $\Gamma \env\V \Gamma$.
\end{lemma}
\begin{proof}
  Let $\gr\Psi$ be the identity map, which clearly satisfies
  $\grP = \grP\gr\Psi$.
  When acting on a variable, the equation $\grPprime = \grQprime\gr\Psi$ means
  that $\grPprime = \grQprime$, so we want, from a variable of type
  $\grPprime\gamma \sqni A$, a value of type $\V\,\grPprime\gamma\,A$, which
  we can get from $\mathrm{vr}$.
\end{proof}

\begin{lemma}\label{thm:env-comp-lemma}
  Given an environment $\rho : \Gamma \env\U \Delta$ for which we have, for any
  $\grPprime$ and $\grQprime$ such that
  $\grPprime = \grQprime\plr{\rho.\gr\Psi}$, we have a function
  $\mathrm{lift}_\rho :
  \forallb{\V\,\grQprime\delta \dotto \W\,\grPprime\gamma}$,
  we can map environments of type $\Delta \env\V \Theta$ into environments of
  type $\Gamma \env\W \Theta$.
\end{lemma}
\begin{proof}
  Let $\rho$ be as in the statement, and let $\sigma : \Delta \env\V \Theta$.
  For the environment we are constructing, let
  $\gr\Psi \coloneqq \sigma.\gr\Psi; \rho.\gr\Psi$, noting that
  $\grP = \grQ\plr{\rho.\gr\Psi} =
  \plr{\grR\plr{\sigma.\gr\Psi}}\plr{\rho.\gr\Psi}$.
  For the action on variables, we are given $\grPprime = \grRprime\gr\Psi$ with
  $\grRprime\theta \sqni A$.
  We can immediately apply the action of $\sigma$, giving us a value of type
  $\V\,\plr{\grRprime\plr{\sigma.\gr\Psi}}\,A$.
  We note that
  $\grPprime = \plr{\grRprime\plr{\sigma.\gr\Psi}}\plr{\rho.\gr\Psi}$, and
  apply $\mathrm{lift}_\rho$ to get the desired value.
\end{proof}

\begin{corollary}[Composition of environments]
  Given a function
  $\mathrm{lift} : \plr{\rho : \grP\gamma \env\U \grQ\delta} \to
  \forall \grPprime, \grQprime.~\grPprime = \grQprime\plr{\rho.\gr\Psi} \to
  \forallb{\V\,\grQprime\delta \dotto \W\,\grPprime\gamma}$, then we can
  compose environments of types $\Gamma \env\U \Delta$ and
  $\Delta \env\V \Theta$ into an environment of type $\Gamma \env\W \Theta$.
\end{corollary}

\begin{example}
  We can derive the following instances of environment composition.
  \begin{itemize}
    \item If $\U = \V = \W = {\sqni}$, then $\mathrm{lift}$ is given by the
      action of the renaming $\rho$ on variables.
      This allows us to derive composition of renamings.
    \item More generally, if $\V = {\sqni}$ and $\U = \W$, we can still use
      the action of the environment $\rho$.
      This means that renamings post-compose with any other sort of environment.
    \item If $\V = \W = {\vdash}$, then $\mathrm{lift}$ is given by a
      syntactic traversal.
      For example, if $\U = {\sqni}$, we need the action of renaming on terms
      to show that a renaming followed by a substitution composes to a
      substitution.
      If $\U = {\vdash}$, then the action of substitution on terms gives us that
      substitutions compose.
    \item More generally, if $\V = {\vdash}$ and we have a semantics from
      $\U$ to $\W$, then $\mathrm{lift}$ can be given by the semantic traversal
      of terms.
  \end{itemize}
\end{example}

% Concatenation is difficult; save to after I've talked about renamings.

% Finally for this section, we give the conditions under which the
% context-forming operations (empty, concatenation, and singleton) have a
% functorial action with respect to $\V$-environments.
%
% \begin{lemma}
%   For any $\V$, there is an environment ${\cdot} \env\V {\cdot}$.
% \end{lemma}
% \begin{proof}
%   By \autoref{thm:construct-env}, it suffices to show $I\,{\cdot}$, which is
%   trivially true.
% \end{proof}

\section{Substitution is admissible in \name{}}\label{sec:lrsub}
\def\LRKits{../agda/processed-latex/LRKits.tex}

I can now show that, using the notion of \emph{environment} derived in
\cref{sec:lrkits}, we can replicate the Agda proofs from
\cref{sec:syntactic-kits} in the usage-aware setting of $\name$.
From \cref{sec:lenv}, we know that environments are preserved under all
syntax-forming operations: zero, addition, scaling, and binding.
What is left is to show how these properties are deployed, and also how to
go on and prove the admissibility of simultaneous renaming, simultaneous
substitution, and then single substitution.

There are a few notational changes necessary in the Agda code, compared to the
typeset mathematics above.
Usage vectors, elsewhere called $\grP$, $\grQ$, and $\grR$ are rendered as
\AgdaBound{P}, \AgdaBound{Q}, and \AgdaBound{R}, respectively.
Usage contexts and typing contexts are tied together with the
\AgdaInductiveConstructor{ctx} constructor, rather than simple juxtaposition.
Environments, elsewhere notated $\Gamma \env\V \Delta$, are rendered as
\AgdaRecord{[}\AgdaSpace{}\AgdaBound{$\V$}\AgdaSpace{}\AgdaRecord{]}%
\AgdaSpace{}\AgdaBound{$\Gamma$}\AgdaSpace{}\AgdaRecord{$\Rightarrow^e$}%
\AgdaSpace{}\AgdaBound{$\Delta$}.

We start with a slightly modified definition of \AgdaRecord{Kit}.
We saw in \cref{thm:lr-bind} that in the usage-annotated context, we restrict
weakening of $\V$-values to just $\gr0$-use variables.
Meanwhile, the function $\mathrm{vr}$, also seen in \cref{thm:lr-bind}, maps
usage-checked variables to $\V$-values, and the function $\mathrm{tm}$, used
to coerce $V$-values yielded by the environment into terms, stays the same.
I state weakening in a slightly different way than previously, so as to help
unification against a known result type (avoiding the problem described by
\citet{McBride12} as \emph{green slime}).
The type \AgdaFunction{Weakening}\AgdaSpace{}\AgdaBound{$\V$} can be read as
saying that, for any context $\grP\gamma$ of shape $s + t$, if the right of
$\grP$ is below $\gr0$, then a value in the left part of $\grP\gamma$ weakens
to a value in the whole of $\grP\gamma$.

\ExecuteMetaData[\LRKits]{Kit}

To demonstrate the important points succinctly, I cut \name{} down to just the
$\oc\gr r$-fragment.
The introduction rule and pattern-matching eliminator feature scaling, addition,
and variable binding, missing out only on sharing (which is trivial) and zero
(which is simpler than, and analogous to, addition).
The resulting type of well typed terms is below.

\ExecuteMetaData[\LRKits]{Tm}

Given a \AgdaRecord{Kit}, \cref{thm:lr-bind} looks like the following.
The \AgdaField{lookup} clauses still contain essentially the same structure as
in the intuitionistic case: discriminating on whether the variable is old or
new, using the given environment \AgdaBound{$\rho$} and weakening on the old
variables, and using \AgdaField{vr} to repackage new variables.
I will not explain any of the algebraic manipulations here; see
\cref{thm:lr-bind}.

\ExecuteMetaData[\LRKits]{bindEnv}

Given \AgdaFunction{bindEnv} (\cref{thm:lr-bind}), \AgdaFunction{env-+}
(\cref{thm:lr-env-add}), and \AgdaFunction{env-*} (\cref{thm:lr-env-scale}),
we can reproduce the syntactic traversal \AgdaFunction{trav}.
With all these lemmas in place, writing \AgdaFunction{trav}
becomes routine.
When processing a rule, we work our way up through the
premise connectives, applying \AgdaFunction{env-*} wherever we see a
\AgdaFunction{$\cdot^c$}, \AgdaFunction{env-+} wherever we see a
\AgdaFunction{$*^c$}, and \AgdaFunction{bindEnv} wherever we see a
\AgdaFunction{Bind}.
We then use whatever environments (with names beginning with
\AgdaBound{$\rho$}) and whatever usage vector splitting facts (with names
beginning with \AgdaBound{sp}) come out of this process to recursively
traverse the subterms and recombine the results.

\ExecuteMetaData[\LRKits]{trav}

Instantiating the generic syntactic traversal \AgdaFunction{trav} to renaming
looks just like it did in the intuitionistic case.
I have consistently replaced intuitionistic variables by linear variables, so
\AgdaFunction{id} and \AgdaInductiveConstructor{var} still work to embed
variables into variables and terms, respectively.
Weakening for variables \AgdaFunction{$\swarrow^v$} (not pictured) has been
updated to note that, for $\grP \leq \bra x$ and $\grR \leq \gr0$, we also have
$\begin{pmatrix} \grP & \grR \end{pmatrix} \leq \bra{{\swarrow}x}$.

\ExecuteMetaData[\LRKits]{var-kit}

In the intuitionistic case, environments were just functions, so we passed the
variable weakening function \AgdaFunction{$\swarrow^v$} to the function
\AgdaFunction{ren} to yield a term weakening function.
However, a usage-aware environment is a function packed together with usage
distribution data.
As such, we must make an environment version of \AgdaFunction{$\swarrow^v$}.
I start with a general lemma \AgdaFunction{$\swarrow$\^{}Env}, stating that if
$\V$ supports weakening, then so do $\V$-environments (in their domain
context).
This lemma then specialises to variables, with the identity renaming
\AgdaFunction{id\^{}Env} on the left part of the context and the proof
\AgdaBound{R0} that the right part of the context is below $\gr0$ combining
to give the desired weakening environment.

\ExecuteMetaData[\LRKits]{dlv-env}

This is what we need to instantiate \AgdaFunction{trav} for substitution.
As a reminder, I also give the type of \AgdaFunction{sub} in rule form.

\ExecuteMetaData[\LRKits]{sub}
\[
  \ebrule{%
    \hypo{\Gamma \env\vdash \Delta}
    \hypo{\Delta \vdash B}
    \infer2[sub]{\Gamma \vdash B}
  }
\]

Finally, the simultaneous substitution \AgdaFunction{sub} specialises to
single substitution \AgdaFunction{sub[-]}.
Single substitution is stated as an admissible rule below.
To substitute in for $\gr r$-many $A$ in the second term, we need to derive
one $A$ with usages $\grP$, and then assert that the result can handle the
usages of the original term $\grQ$, plus $\gr r$-many copies of $\grP$.

\[
  \ebrule{%
    \hypo{\grR \leq \grQ + \gr r\grP}
    \hypo{\grP\gamma \vdash A}
    \hypo{\grQ\gamma, \gr rA \vdash B}
    \infer3[singleSub]{\grR\gamma \vdash B}
  }
\]

The proof strategy for producing the substitution \AgdaFunction{$\sigma$} is
to proceed structurally on the codomain context $\grQ\gamma, \gr rA$ using
\cref{thm:construct-env}, applying the identity substitution
\AgdaFunction{id\^{}Env} on the $\gamma$ half, and dropping the term
\AgdaBound{t} in place of the variable we are substituting for.

\ExecuteMetaData[\LRKits]{subSingle}

\section{Comparison with Petricek's substitution lemma}\label{sec:petricek}
A similar substitution lemma to the one presented in this chapter appears in the
PhD thesis of \citet[p.\ 138]{petricek-thesis} under the name
\emph{multi-nary substitution}.
In my notation, \citeauthor{petricek-thesis}'s substitution rule looks like the
following, up to permutation of the contexts containing $\Gamma$.
Note that if $\Delta = \grQ\delta$, then $\gr r\Delta$ denotes the context
$\plr{\gr r\grQ}\delta$.
This rule is essentially an iterated version of the standard linear single
substitution principle, and is used by \citeauthor{petricek-thesis} as a
strengthened induction hypothesis required to derive single substitution.

\[
  \ebrule{%
    \hypo{\Delta_1 \vdash A_1}
    \hypo{\cdots}
    \hypo{\Delta_n \vdash A_n}
    \hypo{\Gamma, \gr{r_1}A_1, \ldots, \gr{r_n}A_n \vdash B}
    \infer4{\Gamma, \gr{r_1}\Delta_1, \ldots, \gr{r_n}\Delta_n \vdash B}
  }
\]

We can derive \citeauthor{petricek-thesis}-style multi-nary substitution as a
corollary of my simultaneous substitution, using reasoning similar to that of
\cref{thm:single-sub}.

\begin{corollary}\label{thm:petricek-sub}
  \Citeauthor{petricek-thesis}'s multi-nary substitution, as stated above, is
  admissible in $\name$.
\end{corollary}
\begin{proof}
  It is enough to provide a substitution of type
  \[
    \Gamma, \gr{r_1}\Delta_1, \ldots, \gr{r_n}\Delta_n
    \env\vdash \Gamma, \gr{r_1}A_1, \ldots, \gr{r_n}A_n.
  \]
  To do this, we use \cref{thm:construct-env} repeatedly, leaving us needing a
  substitution of type
  $\Gamma, \gr0\Delta_1, \ldots, \gr0\Delta_n \env\vdash \Gamma$ and terms of
  types
  \begin{align*}
    \gr0\gamma, \Delta_1, \gr0\delta_2, &\ldots, \gr0\delta_{n-1}, \gr0\delta_n
    \vdash A_1 \\
    &\vdots \\
    \gr0\gamma, \gr0\delta_1, \gr0\delta_2, &\ldots, \gr0\delta_{n-1}, \Delta_n
    \vdash A_n.
  \end{align*}
  The identity substitution and weakening by $\gr0$-annotated variables is
  enough to make these requirements line up with the given hypotheses.
\end{proof}

My substitution principle is stronger than \citeauthor{petricek-thesis}'s.
Where \citeauthor{petricek-thesis} requires that distinct variables be
available for each hypothesis, I allow for separation of uses via addition of
contexts.
Below is a prototypical example.

\begin{example}
  Let $\Ann \coloneqq \plr{\mathbb N, =, 0, +, 1, \times}$, the exact
  usage-counting posemiring.
  Then, we can construct a substitution $\rho : \gr2A \env\vdash \gr1A, \gr1A$,
  yielding a transformation of terms of the following form:
  \[
    \ebrule{%
      \hypo{\gr1A, \gr1A \vdash B}
      \infer1{\gr2A \vdash B}
    }.
  \]
  To construct $\rho$, we use \cref{thm:construct-env} case
  $\sep(\rightarrowtriangle)$, using the fact that $\gr2 \leq \gr1 + \gr1$.
  From there, two identity substitutions suffice.
  The action of $\rho$ on terms is to merge the two variables into one.
  Note that a renaming, rather than a substitution, would also suffice.
\end{example}

Most notably, my (single) substitution principle more naturally fits the
requirement we would have for the reduct of the $\beta$-rule for functions in
$\name$, whereas \citeauthor{petricek-thesis}'s substitution principle would
need some additional transformation for it to fit properly.
This comes from the fact that the $\name$ function application rule introduces
an algebraic ($+$) separation between its premises, whereas
\citeauthor{petricek-thesis}'s substitution principle separates premises only
via concatenation.


\section{Conclusion}\label{sec:ren-sub-lr-conc}

In this and the preceding chapter, I have developed a discipline for specifying
the syntax of linear and modal type systems, and furthermore developing the
syntactic metatheory of those type systems.
All of these are based on semirings, and the linear algebra arising from
considering a usage context full of semiring elements as a vector.

These developments can be seen in retrospect as a generalisation of the methods
explained in \cref{sec:simple}.
In terms of premise connectives in the syntactic rules, we have generalised from
just $\{\dot1, \dottimes\}$ to
$\{\dot1, \dottimes, I^*, \sep, \gr r\cdot{}, \Box^{0{+}}\}$, maintaining our
ability to keep the context implicit.
Similarly to how rule premises can require separation of usage annotations, our
new environments can require such a separation between their entries thanks to
the linear map they now contain.
I have generalised the key property of a kit from arbitrary weakening to
weakening by $\gr0$-annotated variables, and using that have produced a
substitution operation based on the same principles as that from
\cref{sec:kits}.

Having generalised all of the components --- namely the contexts, the syntax,
and the notion of environment --- the type of the substitution operation looks
the same as it did for intuitionistic ST$\lambda$C\@.
Being able to maintain this uniformity is a key step towards generalising the
rest of \cref{sec:simple} (i.e., \cref{sec:gen-sem,sec:gen-syn}), as I do in
\cref{sec:framework}.

Future work may want to extend the work of this chapter, in which case
there are some unanswered questions.
Principal among these, in my mind, is dealing with equivalence/equality of
environments.
We want to talk about equality of environments for two related purposes.
The most immediate is that we want to develop the equational theory of renaming
and substitution --- for example, when we use \cref{thm:env-comp} to compose
substitutions, we expect that composition to be associative and unital (with
respect to \cref{thm:env-id}).
These equations of substitutions should yield equations on terms in which such
substitutions have been applied.
Slightly more abstractly, I would like to develop a theory of
\emph{quantitative multicategories}, in which multimorphisms have in their
domain a list of objects paired with usage annotations.
I would hope for $\name$ types and terms to give an example quantitative
multicategory, analogously to how Lambek calculus gives an example of an
ordinary multicategory and the simply typed $\lambda$-calculus gives an example
of a Cartesian multicategory.

Intuitionistic environments $\rho, \sigma : \Gamma \env\V \Delta$ are equal
if and only if, for each type $A$ and each variable $x : \Delta \ni A$, we have
$\rho\,x = \sigma\,x$.
This follows from what we expect of equality of functions (function
extensionality).
Usage-aware environments, on the other hand, are $\Sigma$-types --- a way of
dividing up the usage annotations of $\Gamma$, and then a function producing
$\V$-values whose usage annotations come from that division.
Equality of $\Sigma$-types is tricky --- we need to equate the first components,
then rewrite the types of the second components by this equation before equating
them.
In practice, the equations rewriting other equations build up so much that
I have given up on a first effort to give a treatment of the equational theory
of substitutions.
Note that recursively defined environments (\cref{def:lr-rec-env}) are also
$\Sigma$-types in cases where $\Delta$ is a concatenation of contexts, so that
definition does not clearly help.

I hope that people working on substructural type systems in the future can take
inspiration from the process laid out in \cref{sec:lrkits} when working out the
appropriate notion of environment for their discipline.
Particularly, \cref{def:lr-rec-env} (the recursive definition) should serve as a
specification, if not the actual implementation, when coming across a new
substructural discipline.
As for the progression to \cref{def:lr-env}, this appears to arise from the fact
that the quantitative usage information is a refinement of the intuitionistic De
Bruijn index-based syntax.
%Being just a refinement means that the action of an environment on a variable
%is largely as it was in the intuitionistic case,
