\chapter{Renaming and substitution for $\name$}\label{sec:ren-sub-lr}

In \cref{sec:semirings}, I defined my calculus of interest $\name$.
In this chapter, I develop the necessary syntactic metatheory for
specifying and implementing the substitution operation.
I follow the approach of \cref{sec:kits} using syntactic kits, but have to make
significant changes to the underlying notion of \emph{environment} before doing
so.
I give and informally motivate these changes to environments in
\cref{sec:lrkits}, and prove some properties of the new definition in
\cref{sec:lenv}.
Finally, I apply these new environments to the syntax of $\name$ in
\cref{sec:lrsub} to derive renaming and substitution operators.

\section{What are linear renaming and substitution?}\label{sec:lrkits}
In an effort to reuse the syntactic kits and traversals approach of
\cref{sec:syntactic-kits}, I will derive the types of simultaneous renaming
and simultaneous substitution from a generic type of \emph{environments}.
To get a type of environments suitable for the usage-aware setting, I first
analyse intuitionistic environments (as introduced in \cref{sec:syntactic-kits}
definition \AgdaFunction{Env}), distilling the easy-to-use functional definition
(\cref{def:simple-env}) into a more basic recursive definition
(\cref{def:simple-rec-env}).
This recursive definition is easy to make usage-aware (\cref{def:lr-rec-env}),
which gives a basis from which to derive the function-based definition I will
take as primary (\cref{def:lr-env}).
The resulting definition makes explicit the role of algebraic linearity in the
metatheory of semiring-annotated calculi.

Recalling from \cref{sec:kits}, we have the following definition of
environments for simple types.

\begin{definition}[Simple environment]\label{def:simple-env}
  For $\V : \mathrm{Ctx} \to \mathrm{Ty} \to \mathrm{Set}$,
  a $\V$-\emph{environment} between simply typed contexts $\Gamma$ and $\Delta$
  is a function, polymorphic in type $A$, from variables of type $A$ in
  $\Delta$ to inhabitants of $\V\,\Gamma\,A$.
  We write the type of such environments as $\Gamma \env\V \Delta$.
\end{definition}

This definition is inadequate for \name{}.
For example, suppose we have a term
$\plr{M{_\otimes}N} : \grR\gamma \vdash A \otimes B$ and a substitution
$\sigma : \grR\gamma \env\vdash \grR\gr'\delta$.
From the $\otimes$-I rule, we have $M : \grP\gamma \vdash A$ and
$N : \grQ\gamma \vdash B$ for some $\grP$ and $\grQ$ such that
$\grR \leq \grP + \grQ$.
We want to apply $\sigma$ to the subterms $M$ and $N$, but this is impossible
because their contexts are not $\grR\gamma$, and we have no way to adapt
$\sigma$ to these new contexts.
Another instructive failure is the general non-existence of identity
environments, like a renaming of type $\gr1A, \gr1B \env\sqni \gr1A, \gr1B$.
We do not have a variable of type $\gr1A, \gr1B \sqni A$ or, symmetrically,
$\gr1A, \gr1B \sqni B$, because, in each case, there is one variable with
annotation $\gr1$ which we have not actually used.
This example suggests that the values of a usage-aware environment should be
derived in \emph{different} usage contexts, such as in $\gr1A, \gr0B \sqni A$.

To see why this definition of environment works for simply typed
$\lambda$-calculus but not \name{}, let us look at an equivalent definition by
recursion on the target context.
This recursive definition (\cref{def:simple-rec-env}), and particularly the
case where $\Delta$ is a concatenation, makes it clear how $\Gamma$ is being
copied for use in each $\V$-value.
I take the equivalence of \cref{def:simple-env} and \cref{def:simple-rec-env}
as obvious, because any function from variables in $\Delta$ can be
defunctionalised as a data structure with the same shape as $\Delta$.

\begin{definition}[Simple recursive environment]\label{def:simple-rec-env}
  A \emph{recursive $\V$-environment} between simply typed contexts $\Gamma$ and
  $\Delta$ is defined by cases on the shape of $\Delta$ (where
  $\Gamma \env\V_R \Delta$ is the notation for the type of recursive
  environments for given $\V$, $\Gamma$, and $\Delta$):
  \begin{itemize}
    \item There is an environment $\alr{} : \Gamma \env\V_R {\cdot}$.
    \item For $\rho_l : \Gamma \env\V_R \Delta_l$ and
      $\rho_r : \Gamma \env\V_R \Delta_r$, we have an environment
      $\alr{\rho_l, \rho_r} : \Gamma \env\V_R \Delta_l, \Delta_r$.
    \item For any value $v : \V\,\Gamma\,A$, we have an environment
      $\alr{v} : \Gamma \env\V_R A$.
  \end{itemize}
\end{definition}

I picture the sharing of $\Gamma$ in \cref{def:simple-rec-env} in the diagram
below.
The converging arrows from $\Gamma$ to each $\Delta_i$ represent the indices of
values appearing in a simple environment.

\begin{displaymath}
  \begin{tikzpicture}[baseline]
    \path
    (-1,1) node (Gtop) {}
    (-1,0) node (G) {$\Gamma$}
    (-1,-1) node (Gbot) {}
    ;
    \node[draw,dotted,fit=(Gtop) (G) (Gbot)] (GG) {};

    \path
    (1,1) node (Dtop) {}
    (1,0) node (D) {$\Delta$}
    (1,-1) node (Dbot) {}
    ;
    \node[draw,dotted,fit=(Dtop) (D) (Dbot)] (DD) {};

    \draw[->,double] (GG) -- (DD);
  \end{tikzpicture}
  \coloneqq
  \begin{tikzpicture}[baseline]
    \path
    (-1,1) node (Gtop) {}
    (-1,0) node (G) {$\Gamma$}
    (-1,-1) node (Gbot) {}
    ;
    \node[draw,dotted,fit=(Gtop) (G) (Gbot)] (GG) {};

    \path
    (1,1) node[draw] (Dtop) {$\Delta_1$}
    (1,0) node (D) {$\vdots$}
    (1,-1) node[draw] (Dbot) {$\Delta_n$}
    ;

    \fill[green!20!white,opacity=1] (GG.north east)
    parabola[bend at end] (Dtop.west)
    parabola[bend at start] (GG.south east)
    -- cycle;
    \fill[blue!40!white,opacity=.5] (GG.north east)
    parabola[bend at end] (Dbot.west)
    parabola[bend at start] (GG.south east)
    -- cycle;

    \draw[->] (GG.north east) parabola[bend at end] (Dtop.west);
    \draw (GG.south east) parabola[bend at end] (Dtop.west);
    \draw[->] (GG.north east) parabola[bend at end] (D.west);
    \draw (GG.south east) parabola[bend at end] (D.west);
    \draw[->] (GG.north east) parabola[bend at end] (Dbot.west);
    \draw (GG.south east) parabola[bend at end] (Dbot.west);
  \end{tikzpicture}
\end{displaymath}

To account for usage, we must replace the simple repetition of $\Gamma$ by
repetition of just the types $\gamma$ and \emph{redistribution} of the usage
annotations $\grP$.
Fortunately, our three basic ways of sharing up usage vectors --- zero,
addition, and scaling --- apply directly to the three possible shapes of the
target context --- empty, concatenation, and a usage-annotated singleton.

\begin{definition}[Usage-annotated recursive environment]\label{def:lr-rec-env}
  A \emph{recursive $\V$-environment} between annotated contexts $\Gamma$ and
  $\Delta$ is defined by cases on the shape of $\Delta$ (where
  $\Gamma \env\V_R \Delta$ is the notation for the
  type of recursive environments for given $\V$, $\Gamma$, and $\Delta$):
  \begin{itemize}
    \item There is one environment $\alr{} : \grP\gamma \env\V_R {\cdot}$
      whenever $\grP \leq \gr0$.
    \item For $\rho_l : \grPl\gamma \env\V_R \Delta_l$ and
      $\rho_r : \grPr\gamma \env\V_R \Delta_r$, we have an environment
      $\alr{\rho_l, \rho_r} : \grP\gamma \env\V_R \Delta_l, \Delta_r$ whenever
      $\grP \leq \grPl + \grPr$.
    \item For any value $v : \V\,\grPprime\gamma\,A$, we have an environment
      $\alr{v} : \grP\gamma \env\V_R \gr rA$ whenever
      $\grP \leq \gr r\grPprime$.
  \end{itemize}
\end{definition}

\begin{example}
  Take $\Ann = \plr{\mathbb N, =, 0, +, 1, \times}$, with the equality order
  chosen to avoid any concerns around subsumption of annotations.
  Then, there is an intuitionistic recursive environment as follows, where
  $y\,z$ is the application of $y$ to $z$.
  \[
    \alr{\alr{z},\alr{y\,z}} :
    \plr{x : A, y : B \to C, z : B} \env\vdash_R \plr{B, C}.
  \]
  There is also a usage-aware recursive environment
  \[
    \alr{\alr{z},\alr{y\,z}} :
    \plr{\gr0x : A, \gr2y : B \multimap C, \gr3z : B} \env\vdash_R
    \plr{\gr1B, \gr2C}.
  \]
  The latter relies on the observations that
  $\begin{pmatrix} \gr0 & \gr2 & \gr3 \end{pmatrix} =
  \begin{pmatrix} \gr0 & \gr0 & \gr1 \end{pmatrix}
  + \begin{pmatrix} \gr0 & \gr2 & \gr2 \end{pmatrix}$ and, on the right, that
  $\begin{pmatrix} \gr0 & \gr2 & \gr2 \end{pmatrix} =
  \gr2\begin{pmatrix} \gr0 & \gr1 & \gr1 \end{pmatrix}$.
  Then, we have $\gr0x : A, \gr0y : B \multimap C, \gr1z : B \vdash z : B$ and
  $\gr0x : A, \gr1y : B \multimap C, \gr1z : B \vdash y\,z : C$.
\end{example}

From the example, we can see that the important usage vectors are the initial
one $\begin{pmatrix} \gr0 & \gr2 & \gr3 \end{pmatrix}$ and the usage vectors
at which terms are derived: $\begin{pmatrix} \gr0 & \gr0 & \gr1 \end{pmatrix}$
and $\begin{pmatrix} \gr0 & \gr1 & \gr1 \end{pmatrix}$.
I will call the latter the \emph{leaf vectors}.
The intermediate vector $\begin{pmatrix} \gr0 & \gr2 & \gr2 \end{pmatrix}$ can
be worked out from the leaf vector
$\begin{pmatrix} \gr0 & \gr1 & \gr1 \end{pmatrix}$ and the scaling factor
$\gr2$ found in the codomain context $\gr1B, \gr2C$.
Even when the ordering on annotations is given by a non-equivalence relation
$\leq$, there is a canonical least choice for all of the intermediate vectors,
together with a constraint that the entire linear combination of all the leaf
vectors is less than or equal to the initial usage vector.
In symbols, we may let $\gr\Psi$ be the collection of leaf vectors indexed by
items in $\Delta$, and state
the constraint as $\grP \leq \sum_{\plr{x : \gr rA} \in \Delta} \gr r\gr\Psi_x$.
Seeing $\gr\Psi$ instead as a $\size\Delta \times \size\Gamma$ matrix, this
constraint is $\grP \leq \grQ\gr\Psi$, using vector-matrix multiplication.
The resulting picture is below, showing $\grP$ being split up into $\gr\Psi$,
and then each $\V$-value being constructed in a separate $\gr\Psi_i\gamma$.

\begin{displaymath}
  \begin{tikzpicture}[baseline]
    \path
    (-1,1) node (Gtop) {}
    (-1,0) node (G) {$\grP\gamma$}
    (-1,-1) node (Gbot) {}
    ;
    \node[draw,dotted,fit=(Gtop) (G) (Gbot)] (GG) {};

    \path
    (1,1) node (Dtop) {}
    (1,0) node (D) {$\grQ\delta$}
    (1,-1) node (Dbot) {}
    ;
    \node[draw,dotted,fit=(Dtop) (D) (Dbot)] (DD) {};

    \draw[->,double] (GG) -- (DD);
  \end{tikzpicture}
  \coloneqq
  \begin{tikzpicture}[baseline]
    \path
    (-1,1) node (Gtop) {}
    (-1,0) node (G) {$\grP\gamma$}
    (-1,-1) node (Gbot) {}
    ;
    \node[draw,dotted,fit=(Gtop) (G) (Gbot)] (GG) {};

    \path
    (1,3) node (G1top) {}
    (1,2) node (G1) {$\gr\Psi_1\gamma$}
    (1,1) node (G1bot) {}
    ;
    \node[draw,dotted,fit=(G1top) (G1) (G1bot)] (GG1) {};
    \draw[->] (GG) -- (GG1);

    \path (1,0) node {$\vdots$};

    \path
    (1,-1) node (Gntop) {}
    (1,-2) node (Gn) {$\gr\Psi_n\gamma$}
    (1,-3) node (Gnbot) {}
    ;
    \node[draw,dotted,fit=(Gntop) (Gn) (Gnbot)] (GGn) {};
    \draw[->] (GG) -- (GGn);

    \path
    (3,1) node[draw] (Dtop) {$\delta_1$}
    (3,0) node (D) {$\vdots$}
    (3,-1) node[draw] (Dbot) {$\delta_n$}
    ;

    \fill[green!20!white] (GG1.north east)
    parabola[bend at end] (Dtop.west)
    parabola[bend at start] (GG1.south east)
    -- cycle;
    \draw[->] (GG1.north east) parabola[bend at end] (Dtop.west);
    \draw (GG1.south east) parabola[bend at end] (Dtop.west);

    \fill[blue!20!white] (GGn.north east)
    parabola[bend at end] (Dbot.west)
    parabola[bend at start] (GGn.south east)
    -- cycle;
    \draw[->] (GGn.north east) parabola[bend at end] (Dbot.west);
    \draw (GGn.south east) parabola[bend at end] (Dbot.west);
  \end{tikzpicture}
  \quad\textrm{where }\grP \leq \grQ\gr\Psi
\end{displaymath}

From this point, we can recover a functional-style definition of usage-aware
environments.
We choose our leaf vectors $\gr\Psi$ up-front, check the inequality, and then
produce a value at each leaf vector.
%From this definition, we can recover a functional-style definition by
%separating choices of usage vectors from the provision of $\V$-values.
%In particular, the only choices of usage vectors that are essential are the
%$\grPprime$s in the singleton case, with the choices in the concatenation case
%being determined as scalings and sums of these $\grPprime$s.
%I let $\gr\Psi$ collect up these $\size\Delta$-many choices of
%$\size\Gamma$-length usage vectors and note that the constraint on $\gr\Psi$
%generated by all the scaling and summing is
%$\grP = \sum_{\plr{x : \gr rA} \in \Delta} \gr r\gr\Psi_x$.

\begin{definition}[Usage-annotated environment (tentative)]
  A \emph{$\V$-environment} between annotated contexts $\Gamma$ and $\Delta$
  (written $\grP\gamma$ and $\grQ\delta$, respectively, when convenient)
  is a matrix $\gr\Psi : \Ann^{\size\Delta \times \size\Gamma}$ such that
  $\grP \leq \grQ\gr\Psi$ and for each
  $\plr{x : A} \in \delta$ we have a value of type $\V\,\gr\Psi_x\gamma\,A$.
\end{definition}

I find this definition somewhat fiddly because of its reliance on low-level
concepts like non-usage-checked variables and rows of a matrix.
We note that $\gr\Psi_x = \bra x\gr\Psi$, from which point, requiring not
just $\V\,\gr\Psi_x\gamma\,A$ but rather $\V\,\plr{\grQprime\gr\Psi}\gamma\,A$
for any $\grQprime \leq \bra x$ is a minor change (and equivalent if $\V$
respects subusaging, which is practically always the case).
``An $x$ such that $(x : A) \in \delta$ and $\grQprime \leq \bra x\gr\Psi$''
is exactly the definition of $\grQprime\delta \sqni A$.
I further regularise this clause by asking for a
$\grPprime \leq \grQprime\gr\Psi$ rather than $\grQprime\gr\Psi$ exactly,
leaving us needing, for each $\grPprime$ and $\grQprime$ related in the same
way ($\gr\Psi$) as $\grP$ and $\grQ$, a function from $\grQprime\delta \sqni A$
to $\V\,\grPprime\gamma\,A$.
Finally, I choose to switch from matrices and matrix multiplication to
linear maps and their actions, which are easier to work with.
All of these changes yield my primary definition of an environment for
usage-annotated calculi, which will be used for the rest of this chapter and in
\cref{sec:framework}.

\begin{definition}[Usage-annotated environment]\label{def:lr-env}
  A \emph{$\V$-environment} between annotated contexts $\Gamma$ and $\Delta$
  (written $\grP\gamma$ and $\grQ\delta$, respectively, when convenient)
  is a linear map $\gr\Psi : \Ann^{\size\Delta} \to \Ann^{\size\Gamma}$ (written
  postfix) such that $\grP \leq \grQ\gr\Psi$ and for each $A$, $\grPprime$, and
  $\grQprime$ such that $\grPprime \leq \grQprime\gr\Psi$, a function from
  $\grQprime\delta \sqni A$ to $\V\,\grPprime\gamma\,A$.
\end{definition}
\begin{notation}
  When there are multiple environments in question and $\rho$ is such an
  environment, I use the notation $\rho.\gr\Psi$ to refer to $\gr\Psi$.
  For example, $\grP \leq \grQ\plr{\rho.\gr\Psi}$.
  For the action on variables, I write $\rho(x)$, where
  $x : \grQprime\delta \sqni A$.
  The expression ``$\rho(x)$'' alone is ambiguous because of the slack in the
  usage context $\grPprime$ of the resulting value.
  Therefore, I will always make sure $\grPprime$ and $\grQprime$ clear when
  using this notation.
\end{notation}

The following simple lemma shows that usage-annotated environments are, in a
sense, as good as simple environments on usage-checked variables.
What usage-annotated environments give us beyond simple environments is the
ability to accommodate linear decompositions, in a way I will make precise in
the next section.

\begin{lemma}
  We can use an environment $\rho : \Gamma \env\V \Delta$ to map a
  usage-checked variable $x : \Delta \sqni A$ to a value of type
  $\V\,\Gamma\,A$.
\end{lemma}
\begin{proof}
  Let $\Gamma = \grP\gamma$ and $\Delta = \grQ\delta$.
  Set $\grPprime \coloneqq \grP$ and $\grQprime \coloneqq \grQ$, then
  $\grP \leq \grQ\gr\Psi$ by the constraint in $\rho$, so we can take
  the $\V$-value $\rho(x)$.
\end{proof}

\section{Properties of linear environments}\label{sec:lenv}
I settle on \cref{def:lr-env}, and prove various properties about it.

\begin{lemma}\label{thm:env-resize}
  Given an environment $\rho : \grP\gamma \env\V \grQ\delta$ and a $\grPprime$
  and a $\grQprime$ such that $\grPprime \leq \grQprime\plr{\rho.\gr\Psi}$,
  there is also an environment of type $\grPprime\gamma \env\V \grQprime\delta$
  with the same linear map and action on variables.
\end{lemma}
\begin{proof}
  The only part of the definition of an environment dependent on $\grP$ or
  $\grQ$ is the constraint $\grP \leq \grQ\gr\Psi$, which we are able to
  replace for $\grPprime$ and $\grQprime$.
\end{proof}

When constructing an environment, we can do so by cases on the shape of the
target context.
We can create an environment into the empty context when all usage annotations
on the source context are $\gr0$.
We can create an environment into a concatenated context when we can additively
split up the annotations of the source context and produce environments into
both halves from the split sources.
We can create an environment into a singleton context when there is a context
$\gr r$ times smaller than the source context in which we can produce a value
of the appropriate type.

\begin{lemma}\label{thm:construct-env}
  We can define all of the following equivalences for any values of the free
  variables, assuming that $\V$ respects subusaging (i.e.,
  $\grPprime \leq \grP \to
  \V\,\grP\gamma \rightarrowtriangle \V\,\grPprime\gamma$).
  \begin{itemize}
    \item $I^{\sep} \leftrightarrowtriangle \plr{{-} \env\V {\cdot}}$
    \item $\plr{{-} \env\V \Delta_l} \sep \plr{{-} \env\V \Delta_r}
      \leftrightarrowtriangle \plr{{-} \env\V \Delta_l, \Delta_r}$
    \item $\gr r \cdot \plr{\V\,(-)\,A}
      \leftrightarrowtriangle \plr{{-} \env\V \gr rA}$
  \end{itemize}
\end{lemma}
\begin{proof}
  There are 6 cases to check.
  Throughout, we write $\Gamma$ as $\grP\gamma$ and $\Delta$ as $\grQ\delta$
  when convenient.
  \begin{description}
    \item[$I^{\sep}(\rightarrowtriangle)$]
      Let $\gr\Psi$ be the unique linear map out of the zero space.
      By assumption and definition, $\grP \leq \gr0 = \grQ\gr\Psi$.
      There are no variables to act upon.
    \item[$I^{\sep}(\leftarrowtriangle)$]
      $\grQ\gr\Psi$ is an empty sum, so if $\grP \leq \grQ\gr\Psi$ then
      $\grP \leq \gr0$.
    \item[$\sep(\rightarrowtriangle)$]
      Let the given environments be $\rho_l : \grPl\gamma \env\V \grQl\delta$
      and $\rho_r : \grPr\gamma \env\V \grQr\delta$, with
      $\grP \leq \grPl + \grPr$.
      Define $\gr\Psi \coloneqq [\rho_l.\gr\Psi, \rho_r.\gr\Psi]$, using the
      coproduct structure of the concatenated vector space.
      We have $\grP \leq \grPl + \grPr \leq
      \grQl\plr{\rho_l.\gr\Psi} + \grQr\plr{\rho_r.\gr\Psi} =
      \begin{pmatrix} \grQl & \grQr \end{pmatrix}\gr\Psi$.
      To act on variables, we are given $\grPprime \leq
      \begin{pmatrix} \gr{\grQ'_l} & \gr{\grQ'_r} \end{pmatrix}\gr\Psi$ and
      $\gr{\grQ'_l}\delta_l, \gr{\grQ'_r}\delta_r \sqni A$.
      Without loss of generality, let us have $\gr{\grQ'_l}\delta_l \sqni A$
      and $\gr{\grQ'_r} \leq \gr0$.
      Thus, $\grPprime \leq
      \gr{\grQ'_l}\plr{\rho_l.\gr\Psi} + \gr{\grQ'_r}\plr{\rho_r.\gr\Psi} \leq
      \gr{\grQ'_l}\plr{\rho_l.\gr\Psi}$,
      and we can act on the variable using $\rho_l$.
    \item[$\sep(\leftarrowtriangle)$]
      Let the unnamed context be $\Gamma$, also written $\grP\gamma$.
      The linear map
      $\gr\Psi : \Ann^{\size{\Delta_l} + \size{\Delta_r}} \to \Ann^{\size\Gamma}$
      splits into
      $\gr\Psi_{\gr l} : \Ann^{\size{\Delta_l}} \to \Ann^{\size\Gamma}
      \coloneqq \alr{\id, 0}; \gr\Psi$ and
      $\gr\Psi_{\gr r} : \Ann^{\size{\Delta_r}} \to \Ann^{\size\Gamma}
      \coloneqq \alr{0, \id}; \gr\Psi$, using the product structure of
      the concatenated vector space.
      Let $\grPl \coloneqq \grQl\gr\Psi_{\gr l}$ and
      $\grPr \coloneqq \grQr\gr\Psi_{\gr r}$, by definition satisfying the
      required constraints.
      For the action on variables, let us consider the left environment (with
      the right environment following symmetrically).
      We are given $\gr{\grP'_l} \leq \gr{\grQ'_l}\gr\Psi_{\gr l}$ and
      $\gr{\grQ'_l}\delta_l \sqni A$.
      From these, we get
      $\gr{\grP'_l} \leq \gr{\grQ'_l}\gr\Psi_{\gr l} =
      \begin{pmatrix} \gr{\grQ'_l} & \gr0 \end{pmatrix}\gr\Psi$ and
      $\gr{\grQ'_l}\delta_l, \gr0\delta_r \sqni A$.
      We can therefore act using the original environment.
    \item[$\cdot(\rightarrowtriangle)$]
      Let $\grP$ and $\grPprime$ be such that $\grP \leq \gr r\grPprime$ and let
      $v : \V\,\grPprime\gamma\,A$.
      Let $\gr\Psi : \Ann \to \Ann^{\size\gamma}
      \coloneqq \gr r\gr' \mapsto \gr r\gr'\grPprime$.
      By definition and the previous assumption, we have
      $\grP \leq \gr r\gr\Psi$.
      When acting on a variable, we have $\grP\gr{''} \leq \gr r\gr'\gr\Psi$
      and $\gr r\gr'A \sqni A'$.
      The latter tells us that $A = A'$ and $\gr r\gr' \leq \gr1$.
      Thus, $\grP\gr{''} \leq \grPprime$.
      Therefore, by subusaging, we may produce a value of type
      $\V\,\grPprime\gamma\,A$, which we can take to be $v$.
    \item[$\cdot(\leftarrowtriangle)$]
      Let us have an environment of type $\grP\gamma \env\V \gr rA$.
      We want to use its action on variables to yield a value.
      To do this, we let $\grPprime \coloneqq \gr1\gr\Psi$, and use this
      equation, together with the fact that we have a variable of type
      $\gr1A \sqni A$, to get a value of type $\V\,\grPprime\gamma\,A$.
      Furthermore, we derive $\grP \leq \gr r\gr\Psi = \gr r\grPprime$, as
      required.
  \end{description}
\end{proof}

We could, as in \cref{def:lr-rec-env}, use these three clauses to define what an
environment is.
However, such a definition appears to require creative induction hypotheses in
the proving of simple lemmas, in contrast to the more direct proofs I achieve
below using \cref{def:lr-env}.
To take a concrete example, consider how we may construct an ``identity''
environment of type $\Gamma \env\V \Gamma$, as in \cref{thm:env-id} below.
If we try to directly proceed by induction on $\Gamma$, we get to the case where
we are aiming to construct an environment of type
$\grP\gamma, \grQ\delta \env\V \grP\gamma, \grQ\delta$ by constructing
environments of types $\grP\gamma, \gr0\delta \env\V \grP\gamma$ and
$\gr0\gamma, \grQ\delta \env\V \grQ\delta$.
These are not identity environments, and thus do not come from the hypotheses of
a simple induction.
In contrast, using \cref{def:lr-env}, in \cref{thm:env-id} we are able to use
the standard fact that there are identity linear maps, and on top of such a map
worry only about the value assigned to each variable.

One of the primary test cases for environments is simultaneous substitution,
which will look like the \TirName{sub} rule below.
Note that we have taken $\V \coloneqq {\vdash}$ --- i.e.\ that the values
yielded by the environment are terms, namely the terms to be substituted in for
the free variables of the derivation of $\Delta \vdash A$.

\begin{displaymath}
  \begin{prooftree}
    \hypo{\Gamma \env{\vdash} \Delta}
    \hypo{\Delta \vdash A}
    \infer2[sub]{\Gamma \vdash A}
  \end{prooftree}
\end{displaymath}

The admissibility of substitution will be by induction on the derivation of
$\Delta \vdash A$, so we will need to be able to adapt any environment we are
given to work with any possible context of new premises yielded by the rules of
\cref{fig:lr-bunched}.
In the simply typed case, the only change to the context we encountered was the
binding of new variables.
With usage annotations, we furthermore have linear decompositions of the
context, necessitating changes to the environment whenever usage annotations
change.

There are three kinds of linear decompositions we have to deal with: zero,
addition, and scaling; corresponding to bunched connectives $I^*$, $\sep$, and
$\gr r \cdot {}$, respectively.
In each of these three cases, we have a simple preservation lemma, transforming
an environment
of type $\Gamma \env\V \Delta$ and a decomposition of $\Delta$ into a
decomposition of $\Gamma$ and environments for all of the decomposed fragments
of $\Gamma$ and $\Delta$.

\begin{lemma}[environments preserve zero]\label{thm:lr-env-zero}
  Given an environment $\rho : \grP\gamma \env\V \grQ\delta$ such that
  $\grQ \leq \gr 0$, we also have that $\grP \leq \gr 0$.
\end{lemma}
\begin{proof}
  $\grP \leq \grQ\gr\Psi \leq \gr0\gr\Psi = \gr0$, by environment
  compatibility from $\rho$ and monotonicity and linearity of $\gr\Psi$.
\end{proof}

\begin{lemma}[environments preserve addition]\label{thm:lr-env-add}
  Given an environment $\rho : \grP\gamma \env\V \grQ\delta$ such that
  $\grQ \leq \grQl + \grQr$ for some $\grQl$ and $\grQr$, we also have $\grPl$
  and $\grPr$ such that $\grP \leq \grPl + \grPr$ and there are environments
  $\rho_l : \grPl\gamma \env\V \grQl\delta$ and
  $\rho_r : \grPr\gamma \env\V \grQr\delta$.
\end{lemma}
\begin{proof}
  Let $\grPl \coloneqq \grQl\gr\Psi$ and $\grPr \coloneqq \grQr\gr\Psi$.
  Then, $\grP \leq \grQ\gr\Psi \leq \plr{\grQl + \grQr}\gr\Psi =
  \grQl\gr\Psi + \grQr\gr\Psi = \grPl + \grPr$, satisfying the first condition.
  Because clearly $\grPl \leq \grQl\gr\Psi$ and $\grPr \leq \grQr\gr\Psi$,
  applying \cref{thm:env-resize} to $\rho$ gives us the required
  new environments $\rho_l$ and $\rho_r$.
\end{proof}

\begin{lemma}[environments preserve scaling]\label{thm:lr-env-scale}
  Given an environment $\rho : \grP\gamma \env\V \grQ\delta$ such that
  $\grQ \leq \gr r\grQprime$ for some $\grQprime$, we also have a $\grPprime$
  such that $\grP \leq \gr r\grPprime$ and there is an environment
  $\rho' : \grPprime\gamma \env\V \grQprime\delta$.
\end{lemma}
\begin{proof}
  Let $\grPprime \coloneqq \grQprime\gr\Psi$.
  Then, $\grP \leq \grQ\gr\Psi \leq \plr{\gr r\grQprime}\gr\Psi =
  \gr r\plr{\grQprime\gr\Psi} = \gr r\grPprime$, satisfying the first condition.
  Because clearly $\grPprime \leq \grQprime\gr\Psi$,
  applying \cref{thm:env-resize} to $\rho$ gives us the required
  new environment $\rho'$.
\end{proof}

The final change environments need to preserve is the binding of new free
variables.
In \cref{sec:syntactic-kits}, we had the operation \AgdaFunction{bindEnv} for
this purpose in the intuitionistic setting.
There, we relied on $\V$ supporting a map from $\ni$-variables and admitting
weakening.
In the usage-annotated setting, the former requirement is updated to having a
map from usage-checked $\sqni$-variables.
As for the latter requirement, it turns out that we only need $\V$ to admit
weakening by $\gr0$-annotated variables, which is much more reasonable than
general weakening.
\Cref{thm:lr-bind} adapts \AgdaFunction{bindEnv} for the usage-annotated
setting.

\begin{lemma}[\AgdaFunction{bindEnv}]\label{thm:lr-bind}
  Given functions
  ${\swarrow^k} : \forall \Gamma, \grR, \theta.~\grR \leq \gr0 \to
  \V\,\Gamma \rightarrowtriangle \V\,\plr{\Gamma, \grR\theta}$ and
  $\mathrm{vr} : {\sqni} \rightarrowtriangle \V$, we can turn an environment of
  type $\Gamma \env\V \Delta$ into an environment of type
  $\Gamma, \Theta \env\V \Delta, \Theta$ for any context $\Theta$.
\end{lemma}
\begin{proof}
  Let $\grP\gamma \coloneqq \Gamma$, $\grQ\delta \coloneqq \Delta$, and
  $\grR\theta \coloneqq \Theta$.
  Let the new linear map $\gr\Psi\gr' : \Ann^{\size\Delta + \size\Theta} \to
  \Ann^{\size\Gamma + \size\Theta}$ be $\gr\Psi \oplus \gr I$.
  That is, in block matrix notation,
  $\begin{pmatrix} \gr\Psi & \gr0 \\ \gr0 & \gr I \end{pmatrix}$.
  Checking that this linear map fits, we have
  $\begin{pmatrix}\grP & \grR\end{pmatrix}
  \leq \begin{pmatrix}\grQ\gr\Psi & \grR\gr I\end{pmatrix}
  = \begin{pmatrix}\grQ & \grR\end{pmatrix}\plr{\gr\Psi \oplus \gr I}$.
  For the action on variables, we are given vectors $\grPprime$,
  $\grR\gr'_\grP$, $\grQprime$, and $\grR\gr'_\grQ$ such that
  $\begin{pmatrix} \grPprime & \grR\gr'_\grP \end{pmatrix} \leq
  \begin{pmatrix} \grQprime & \grR\gr'_\grQ \end{pmatrix}
  \plr{\gr\Psi \oplus \gr I}$ and we have a variable of type
  $\grQprime\delta, \grR\gr'_\grQ\theta \sqni A$ for some type $A$.
  The constraint on the new vectors reduces to $\grPprime \leq \grQprime\gr\Psi$
  and $\grR\gr'_\grP \leq \grR\gr'_\grQ$.
  From the variable we either have a variable $x$ in $\delta$ with
  $\grQprime \leq \langle x \rvert$ and $\grR\gr'_\grQ \leq \gr0$, or a
  variable $y$ in $\theta$ with $\grQprime \leq \gr0$ and
  $\grR\gr'_\grQ \leq \langle y \rvert$.
  In the former case, the action of the original environment on $x$ gives us a
  $\V$-value in $\grPprime\gamma$, and the $\gr0$-weakening principle
  $\swarrow^k$, noting that $\grR\gr'_\grP \leq \grR\gr'_\grQ \leq \gr0$, gives
  us a $\V$-value in $\grPprime\gamma, \grR\gr'_\grP\theta$.
  In the latter case, we have that
  $\begin{pmatrix} \grPprime & \grR\gr'_\grP \end{pmatrix}
  \leq \begin{pmatrix} \grQprime\gr\Psi & \grR\gr'_\grQ \end{pmatrix}
  \leq \begin{pmatrix} \gr0\gr\Psi & \langle y \rvert \end{pmatrix}
  = \begin{pmatrix} \gr0 & \langle y \rvert \end{pmatrix}
  = \left\langle {\searrow}y \right\rvert$, so $y$ also serves as a
  usage-checked variable in $\grPprime\gamma, \grR\gr'_\grP\theta$.
  From this usage-checked variable, we get a $\V$-value in the same context
  using $\mathrm{vr}$.
\end{proof}

I put together the preceding pieces to give a syntactic traversal operation over
$\name$ in the following section.
For the rest of this section, I observe some more constructions purely on
environments --- in particular, composition of environments given certain
assumptions about the families of values.

Following \citet{ACU15}, we expect (intuitionistic) ST$\lambda$C syntax to form
a relative monad over $\ni$ seen as a functor from the category of contexts
under renaming to the functor category $\blr{\mathrm{Ty}, \Set}$, where
$\mathrm{Ty}$ is the discrete category of ST$\lambda$C types.
Notice that, given $F,G : \blr{\mathrm{Ty}, \Set}$, a morphism from $F$ to $G$
is a function of type $F \rightarrowtriangle G$ (with naturality being trivial).
Therefore, we expect a relative monad, given as a Kleisli triple, to have a unit
$\eta_\Gamma : \Gamma \ni \plr{-} \rightarrowtriangle \Gamma \vdash \plr{-}$
given by the variable rule, and a Kleisli extension operator
${^*}_{\Gamma,\Delta} :
\plr{\Gamma \ni \plr{-} \rightarrowtriangle \Delta \vdash \plr{-}} \to
\plr{\Gamma \vdash \plr{-} \rightarrowtriangle \Delta \vdash \plr{-}}$
given by substitution.
Composition of substitutions falls out of this framework as Kleisli composition.
However, in the usage-aware case, substitution needs not just a mapping of
variables $f : \Gamma \sqni \plr{-} \rightarrowtriangle \Delta \vdash \plr{-}$,
but rather an environment $\rho : \Delta \env\vdash \Gamma$, as we have already
discussed.
It therefore makes sense for our replacement for the Kleisli extension operator
to similarly take an environment rather than a simple variable mapping.

\Cref{thm:env-comp} below amounts to deriving a modified notion of Kleisli
composition from a modified Kleisli extension.
Additionally, \cref{thm:env-id} is required to turn a monadic unit into an
identity environment.
Both lemmas are stated in terms of general $\U$/$\V$/$\W$-environments, with
some specific examples (e.g.\ for renaming and substitution) below them.

\begin{lemma}[Identity environment]\label{thm:env-id}
  Given a function
  \[
    \mathrm{vr}_{\Gamma'} :
    \Gamma' \sqni \plr{-} \rightarrowtriangle \V\,\Gamma'
  \]
  for any
  $\Gamma$ we have an environment $\mathrm{id} : \Gamma \env\V \Gamma$.
\end{lemma}
\begin{proof}
  Let $\Gamma = \grP\gamma$.
  Let $\gr\Psi$ be the identity map, which clearly satisfies
  $\grP \leq \grP\gr\Psi$.
  When acting on a variable, the inequality $\grPprime \leq \grQprime\gr\Psi$
  means that $\grPprime \leq \grQprime$.
  We are given a variable of type $\grQprime\gamma \sqni A$, which we can
  coerce to a variable of type $\grPprime\gamma \sqni A$, upon which we apply
  $\mathrm{vr}$ to get the required value of type $\V\,\grPprime\gamma\,A$.
\end{proof}

\begin{lemma}[Composition of environments]\label{thm:env-comp}
  Given a function
  \[
    \mathrm{lift}_{\Gamma', \Delta'} :
    \Gamma' \env\U \Delta' \to \V\,\Delta' \rightarrowtriangle \W\,\Gamma'
  \]
  we can compose environments $\rho : \Gamma \env\U \Delta$ and
  $\sigma : \Delta \env\V \Theta$ into an environment
  $\rho \gg \sigma : \Gamma \env\W \Theta$.
\end{lemma}
\begin{proof}
  Let $\Gamma = \grP\gamma$, $\Delta = \grQ\delta$, and $\Theta = \grR\theta$.
  Take $\gr\Psi$ to be the composition $\plr{\sigma.\gr\Psi}\plr{\rho.\gr\Psi}$,
  noting that
  $\grP \leq \grQ\plr{\rho.\gr\Psi} \leq
  (\grR\plr{\sigma.\gr\Psi})\plr{\rho.\gr\Psi} = \grR\gr\Psi$
  thanks to the inequalities yielded by $\sigma$ and $\rho$.
  When acting on a variable, we are given $\grPprime \leq \grRprime\gr\Psi$ and
  a variable $v : \grRprime\theta \sqni A$, and want a value of type
  $\W\,\grPprime\gamma\,A$.
  Let $\grQprime \coloneqq \grRprime\plr{\sigma.\gr\Psi}$, with inequality
  $\grQprime \leq \grRprime\plr{\sigma.\gr\Psi}$ giving us a value
  $\sigma(v) : \V\,\grQprime\delta\,A$.
  We wish to apply $\mathrm{lift}$ to $\sigma(v)$ with
  $\Gamma' \coloneqq \grPprime\gamma$ and $\Delta' \coloneqq \grQprime\delta$ to
  complete the construction of the $\W$-value.
  To do this, we need an environment of type
  $\grPprime\gamma \env\U \grQprime\delta$, which we can get from $\rho$ using
  \cref{thm:env-resize}, noting that
  $\grPprime \leq \grRprime\plr{\sigma.\gr\Psi}\plr{\rho.\gr\Psi} =
  \grQprime\plr{\rho.\gr\Psi}$.
\end{proof}

We can derive the following corollaries as instances of environment composition.

\begin{corollary}[Composition of renamings]\label{thm:ren-comp}
  Given renamings $\rho : \Gamma \env\sqni \Delta$ and
  $\sigma : \Delta \env\sqni \Theta$, we can form their composite
  $\rho; \sigma : \Gamma \env\sqni \Theta$.
\end{corollary}
\begin{proof}
  Take $\U = \V = \W = {\sqni}$ in \cref{thm:env-comp}.
  Then let $\mathrm{lift}\,\rho\,x \coloneqq \rho(x)$.
\end{proof}

\begin{corollary}[Post-composition with a renaming]\label{thm:ren-env-comp}
  Given an environment $\rho : \Gamma \env\U \Delta$ and a renaming
  $\sigma : \Delta \env\sqni \Theta$, we can form their composite
  $\rho; \sigma : \Gamma \env\U \Theta$.
\end{corollary}
\begin{proof}
  As in \cref{thm:ren-comp}.
\end{proof}

\begin{corollary}[Pointwise renaming of an environment]\label{thm:env-ren}
  If $\sdtstile{}\V$ respects renaming, then so does $\env\V$ (on the left).
\end{corollary}
\begin{proof}
  Suppose we have $\rho : \Gamma \env\sqni \Delta$ and
  $\sigma : \Delta \env\V \Theta$.
  We want to compose these via \cref{thm:env-comp} with $U = {\sqni}$ and
  $\V = \W$.
  The function $\mathrm{lift}$ is given exactly by the fact that $\V$ respects
  renaming.
\end{proof}

\begin{corollary}[Composition of substitutions]\label{thm:sub-comp}
  Given substitutions $\rho : \Gamma \env\vdash \Delta$ and
  $\sigma : \Delta \env\vdash \Theta$, we can form their composite
  $\rho; \sigma : \Gamma \env\vdash \Theta$.
\end{corollary}
\begin{proof}
  Take $\U = \V = \W = {\vdash}$ in \cref{thm:env-comp}.
  Then, $\mathrm{lift}$ is given by the action of a substitution on a term
  (see \AgdaFunction{sub} in the following section).
\end{proof}

\begin{corollary}[Composing semantics with substitution]
  If we have a semantics (in the sense of \cref{sec:gen-sem} and
  \cref{sec:traversal}) from $\U$ to $\W$, then from an environment
  $\rho : \Gamma \env\U \Delta$ and a substitution
  $\sigma : \Delta \env\vdash \Theta$, we can form the composite
  $\rho; \sigma : \Gamma \env\W \Theta$.
\end{corollary}

% Concatenation is difficult; save to after I've talked about renamings.

% Finally for this section, we give the conditions under which the
% context-forming operations (empty, concatenation, and singleton) have a
% functorial action with respect to $\V$-environments.
%
% \begin{lemma}
%   For any $\V$, there is an environment ${\cdot} \env\V {\cdot}$.
% \end{lemma}
% \begin{proof}
%   By \cref{thm:construct-env}, it suffices to show $I\,{\cdot}$, which is
%   trivially true.
% \end{proof}

\section{Substitution is admissible in \name{}}\label{sec:lrsub}
\def\LRKits{../agda/processed-latex/LRKits.tex}

I now show that, using the notion of \emph{environment} derived in
\cref{sec:lrkits}, we can replicate the Agda proofs from
\cref{sec:syntactic-kits} in the usage-aware setting of $\name$.
From \cref{sec:lenv}, we know that environments are preserved under all
syntax-forming operations: zero, addition, scaling, and binding.
What is left is to show how these properties are deployed, and also how to
go on and prove the admissibility of simultaneous renaming, simultaneous
substitution, and then single substitution.

There are a few notational changes necessary in the Agda code, compared to the
typeset mathematics above.
Usage vectors, elsewhere called $\grP$, $\grQ$, and $\grR$ are rendered as
\AgdaBound{P}, \AgdaBound{Q}, and \AgdaBound{R}, respectively.
Usage contexts and typing contexts are tied together with the
\AgdaInductiveConstructor{ctx} constructor, rather than simple juxtaposition.
Environments, elsewhere notated $\Gamma \env\V \Delta$, are rendered as
\AgdaRecord{[}\AgdaSpace{}\AgdaBound{$\V$}\AgdaSpace{}\AgdaRecord{]}%
\AgdaSpace{}\AgdaBound{$\Gamma$}\AgdaSpace{}\AgdaRecord{$\Rightarrow^e$}%
\AgdaSpace{}\AgdaBound{$\Delta$}.

I start with the definition \AgdaFunction{Weakening}, which says what it means
for a family of values \AgdaBound{$\V$} to respect one step of weakening on the
right by $\gr0$-use variables.
I state weakening in a slightly different way to what appears in the statement
of \cref{thm:lr-bind}, so as to help
unification against a known result type (avoiding the problem described by
\citet{McBride12} as \emph{green slime}).
The type \AgdaFunction{Weakening}\AgdaSpace{}\AgdaBound{$\V$} can be read as
saying that, for any context $\grP\gamma$ of shape $s + t$, if the right of
$\grP$ is below $\gr0$, then a value in the left part of $\grP\gamma$ weakens
to a value in the whole of $\grP\gamma$.

\ExecuteMetaData[\LRKits]{Weakening}

Given this new definition of \AgdaFunction{Weakening}, the record
\AgdaRecord{Kit} remains largely unchanged relative to what we saw in
\cref{sec:syntactic-kits}.
We still want $\V$-values to respect weakening (as used in \cref{thm:lr-bind}),
and to support maps \AgdaField{vr} from variables (also used in
\cref{thm:lr-bind}) and \AgdaField{tm} into terms (used when traversal reaches a
variable, we get a value from the environment, and want to produce a term from
that value).
As well as \AgdaFunction{Weakening}, note that \AgdaRecord{\_$\sqni$\_} and, of
course, \AgdaDatatype{\_$\vdash$\_} have different definitions to the
corresponding intuitionistic notions, but they still represent morally the same
parts of the language and its metatheory.

\ExecuteMetaData[\LRKits]{Kit}

To demonstrate the important points succinctly, I cut \name{} down to just the
$\oc\gr r$-fragment.
The introduction rule and pattern-matching eliminator feature scaling, addition,
and variable binding, missing out only on sharing (which is trivial) and zero
(which is simpler than, and analogous to, addition).
The resulting type of well typed terms is below.

\ExecuteMetaData[\LRKits]{Tm}

Given a \AgdaRecord{Kit}\AgdaSpace{}\AgdaBound{$\V$},
\cref{thm:lr-bind} gives a function with the following type.

\ExecuteMetaData[\LRKits]{bindEnv}

Given \AgdaFunction{bindEnv} (\cref{thm:lr-bind}), \AgdaFunction{env-+}
(\cref{thm:lr-env-add}), and \AgdaFunction{env-*} (\cref{thm:lr-env-scale}),
we can reproduce the syntactic traversal \AgdaFunction{trav}.
Similarly to the unchanged high-level definition of \AgdaRecord{Kit}, we are
aiming for an unchanged traversal principle expressed by the rule below.
When $\V$ has a \AgdaRecord{Kit} structure, and we have a $\V$-environment from
$\Gamma$ to $\Delta$, we can transform a term in $\Delta$ to a term in $\Gamma$
of the same type.

\[
  \begin{prooftree}
    \hypo{\text{\AgdaRecord{Kit}\AgdaSpace{}}\V}
    \hypo{\Gamma \env\V \Delta}
    \hypo{\Delta \vdash A}
    \infer3[Trav]{\Gamma \vdash A}
  \end{prooftree}
\]

With all the lemmas of \cref{sec:lenv} in place, writing \AgdaFunction{trav}
becomes routine.
When processing a rule, we work our way up through the
premise connectives, applying \AgdaFunction{env-*} wherever we see a
\AgdaFunction{$\cdot^c$}, \AgdaFunction{env-+} wherever we see a
\AgdaFunction{$*^c$}, and \AgdaFunction{bindEnv} wherever we see a
\AgdaFunction{Bind}.
We then use whatever environments (with names beginning with
\AgdaBound{$\rho$}) and whatever usage vector splitting facts (with names
beginning with \AgdaBound{sp}) come out of this process to recursively
traverse the subterms and recombine the results.

\ExecuteMetaData[\LRKits]{trav}

Instantiating the generic syntactic traversal \AgdaFunction{trav} to renaming
looks just like it did in the intuitionistic case.
I have consistently replaced intuitionistic variables by linear variables, so
\AgdaFunction{id} and \AgdaInductiveConstructor{var} still work to embed
variables into variables and terms, respectively.
Weakening for variables \AgdaFunction{$\swarrow^v$} (not pictured) has been
updated to note that, for $\grP \leq \bra x$ and $\grR \leq \gr0$, we also have
$\begin{pmatrix} \grP & \grR \end{pmatrix} \leq \bra{{\swarrow}x}$.

\ExecuteMetaData[\LRKits]{var-kit}

In the intuitionistic case, environments were just functions, so we passed the
variable weakening function \AgdaFunction{$\swarrow^v$} to the function
\AgdaFunction{ren} to yield a term weakening function.
However, a usage-aware environment is a function packed together with usage
distribution data.
As such, we must make an environment version of \AgdaFunction{$\swarrow^v$}.
I start with a general lemma \AgdaFunction{$\swarrow$\^{}Env}, stating that if
$\V$ supports weakening, then so do $\V$-environments (in their domain
context).
This lemma then specialises to variables, with the identity renaming
\AgdaFunction{id\^{}Env} on the left part of the context and the proof
\AgdaBound{R0} that the right part of the context is below $\gr0$ combining
to give the desired weakening environment.

\ExecuteMetaData[\LRKits]{dlv-env}

This is what we need to instantiate \AgdaFunction{trav} for substitution.
As a reminder, I also give the type of \AgdaFunction{sub} in rule form.

\ExecuteMetaData[\LRKits]{sub}
\[
  \ebrule{%
    \hypo{\Gamma \env\vdash \Delta}
    \hypo{\Delta \vdash B}
    \infer2[sub]{\Gamma \vdash B}
  }
\]

Finally, the simultaneous substitution \AgdaFunction{sub} specialises to
single substitution.

\begin{corollary}[Single substitution]\label{thm:single-sub}
  The following equivalent rules are admissible.
  \begin{mathpar}
    \ebrule{%
      \hypo{\grR \leq \gr r\grP + \grQ}
      \hypo{\grP\gamma \vdash A}
      \hypo{\grQ\gamma, \gr rA \vdash B}
      \infer3{\grR\gamma \vdash B}
    }
    \and
    \ebrule[comb]{%
      \hypo{\gr r \cdot \plr{{} \vdash A}}
      \hypo{\sep}
      \hypo{\gr rA \vdash B}
      \infer3{{} \vdash B}
    }
  \end{mathpar}
\end{corollary}
\begin{proof}
  It is enough to construct a substitution of type
  $\grR\gamma \env\vdash \grQ\gamma, \gr rA$.
  To do this, we use \cref{thm:construct-env} cases $\sep(\rightarrowtriangle)$
  and $\cdot(\rightarrowtriangle)$ on inequalities
  $\grR \leq \grQ + \gr r\grP$ and $\gr r\grP \leq \gr r\grP$ respectively to
  leave us needing a substitution of type $\grQ\gamma \env\vdash \grQ\gamma$ and
  a term of type $\grP\gamma \vdash A$.
  For the substitution, we give the identity substitution (\cref{thm:env-id}),
  and we have the term as a hypothesis.
\end{proof}

\section{Comparison with Petricek's substitution lemma}\label{sec:petricek}
A similar substitution lemma to the one presented in this chapter appears in the
PhD thesis of \citet[p.\ 138]{petricek-thesis} under the name
\emph{multi-nary substitution}.
In my notation, \citeauthor{petricek-thesis}'s substitution rule looks like the
following, up to permutation of the contexts containing $\Gamma$.
Note that if $\Delta = \grQ\delta$, then $\gr r\Delta$ denotes the context
$\plr{\gr r\grQ}\delta$.
This rule is essentially an iterated version of the standard linear single
substitution principle, and is used by \citeauthor{petricek-thesis} as a
strengthened induction hypothesis required to derive single substitution.

\[
  \ebrule{%
    \hypo{\Delta_1 \vdash A_1}
    \hypo{\cdots}
    \hypo{\Delta_n \vdash A_n}
    \hypo{\Gamma, \gr{r_1}A_1, \ldots, \gr{r_n}A_n \vdash B}
    \infer4{\Gamma, \gr{r_1}\Delta_1, \ldots, \gr{r_n}\Delta_n \vdash B}
  }
\]

We can derive \citeauthor{petricek-thesis}-style multi-nary substitution as a
corollary of my simultaneous substitution, using reasoning similar to that of
\cref{thm:single-sub}.

\begin{corollary}\label{thm:petricek-sub}
  \Citeauthor{petricek-thesis}'s multi-nary substitution, as stated above, is
  admissible in $\name$.
\end{corollary}
\begin{proof}
  It is enough to provide a substitution of type
  \[
    \Gamma, \gr{r_1}\Delta_1, \ldots, \gr{r_n}\Delta_n
    \env\vdash \Gamma, \gr{r_1}A_1, \ldots, \gr{r_n}A_n.
  \]
  To do this, we use \cref{thm:construct-env} repeatedly, leaving us needing a
  substitution of type
  $\Gamma, \gr0\Delta_1, \ldots, \gr0\Delta_n \env\vdash \Gamma$ and terms of
  types
  \begin{align*}
    \gr0\gamma, \Delta_1, \gr0\delta_2, &\ldots, \gr0\delta_{n-1}, \gr0\delta_n
    \vdash A_1 \\
    &\vdots \\
    \gr0\gamma, \gr0\delta_1, \gr0\delta_2, &\ldots, \gr0\delta_{n-1}, \Delta_n
    \vdash A_n.
  \end{align*}
  The identity substitution and weakening by $\gr0$-annotated variables is
  enough to make these requirements line up with the given hypotheses.
\end{proof}

My substitution principle is stronger than \citeauthor{petricek-thesis}'s.
Where \citeauthor{petricek-thesis} requires that distinct variables be
available for each hypothesis, I allow for separation of uses via addition of
contexts.
Below is a prototypical example.

\begin{example}
  Let $\Ann \coloneqq \plr{\mathbb N, =, 0, +, 1, \times}$, the exact
  usage-counting posemiring.
  Then, we can construct a substitution $\rho : \gr2A \env\vdash \gr1A, \gr1A$,
  yielding a transformation of terms of the following form:
  \[
    \ebrule{%
      \hypo{\gr1A, \gr1A \vdash B}
      \infer1{\gr2A \vdash B}
    }.
  \]
  To construct $\rho$, we use \cref{thm:construct-env} case
  $\sep(\rightarrowtriangle)$, using the fact that $\gr2 \leq \gr1 + \gr1$.
  From there, two identity substitutions suffice.
  The action of $\rho$ on terms is to merge the two variables into one.
  Note that a renaming, rather than a substitution, would also suffice.
\end{example}

Most notably, my (single) substitution principle more naturally fits the
requirement we would have for the reduct of the $\beta$-rule for functions in
$\name$, whereas \citeauthor{petricek-thesis}'s substitution principle would
need some additional transformation for it to fit properly.
This comes from the fact that the $\name$ function application rule introduces
an algebraic ($+$) separation between its premises, whereas
\citeauthor{petricek-thesis}'s substitution principle separates premises only
via concatenation.

\section{Conclusion}\label{sec:ren-sub-lr-conc}

In this and the preceding chapter, I have developed a discipline for specifying
the syntax of linear and modal type systems, and furthermore developing the
syntactic metatheory of those type systems.
All of these are based on semirings, and the linear algebra arising from
considering a usage context full of semiring elements as a vector.

These developments can be seen in retrospect as a generalisation of the methods
explained in \cref{sec:simple}.
In terms of premise connectives in the syntactic rules, we have generalised from
just $\{\dot1, \dottimes\}$ to
$\{\dot1, \dottimes, I^*, \sep, \gr r\cdot{}, \Box^{0{+}}\}$, maintaining our
ability to keep the context implicit.
Similarly to how rule premises can require separation of usage annotations, our
new environments can require such a separation between their entries thanks to
the linear map they now contain.
I have generalised the key property of a kit from arbitrary weakening to
weakening by $\gr0$-annotated variables, and using that have produced a
substitution operation based on the same principles as that from
\cref{sec:kits}.

Having generalised all of the components --- namely the contexts, the syntax,
and the notion of environment --- the type of the substitution operation looks
the same as it did for intuitionistic ST$\lambda$C\@.
Being able to maintain this uniformity is a key step towards generalising the
rest of \cref{sec:simple} (i.e., \cref{sec:gen-sem,sec:gen-syn}), as I do in
\cref{sec:framework}.

I have made few unforced design decisions in this chapter, so there are few
aspects I can see future work improving upon.\todo{Which?}
However, future work may want to extend the work of this chapter, in which case
there are some unanswered questions.
Principal among these, in my mind, is dealing with equivalence/equality of
environments.
We want to talk about equality of environments for two related purposes.
The most immediate is that we want to develop the equational theory of renaming
and substitution --- for example, when we use \cref{thm:env-comp} to compose
substitutions, we expect that composition to be associative and unital (with
respect to \cref{thm:env-id}).
These equations of substitutions should yield equations on terms in which such
substitutions have been applied.
Slightly more abstractly, I would like to develop a theory of
\emph{quantitative multicategories}, in which multimorphisms have in their
domain a list of objects paired with usage annotations.
I would hope for $\name$ types and terms to give an example quantitative
multicategory, analogously to how Lambek calculus gives an example of an
ordinary multicategory and the simply typed $\lambda$-calculus gives an example
of a Cartesian multicategory.

Intuitionistic environments $\rho, \sigma : \Gamma \env\V \Delta$ are equal
if and only if, for each type $A$ and each variable $x : \Delta \ni A$, we have
$\rho\,x = \sigma\,x$.
This follows from what we expect of equality of functions (function
extensionality).
Usage-aware environments, on the other hand, are $\Sigma$-types --- a way of
dividing up the usage annotations of $\Gamma$, and then a function producing
$\V$-values whose usage annotations come from that division.
Equality of $\Sigma$-types is tricky --- we need to equate the first components,
then rewrite the types of the second components by this equation before equating
them.
In practice, the equations rewriting other other equations build up so much that
I have given up on a first effort to give a treatment of the equational theory
of substitutions.
Note that recursively defined environments (\cref{def:lr-rec-env}) are also
$\Sigma$-types in cases where $\Delta$ is a concatenation of contexts, so that
definition does not clearly help.

I hope that people working on substructural type systems in the future can take
inspiration from the process laid out in \cref{sec:lrkits} when working out the
appropriate notion of environment for their discipline.
Particularly, \cref{def:lr-rec-env} (the recursive definition) should serve as a
specification, if not the actual implementation, when coming across a new
substructural discipline.
As for the progression to \cref{def:lr-env}, this appears to arise from the fact
that the quantitative usage information is a refinement of the intuitionistic de
Bruijn index-based syntax.
%Being just a refinement means that the action of an environment on a variable
%is largely as it was in the intuitionistic case,
