\subsection{Partiality (or not)} \todo{Appendix?}

As we have seen \todo{Have we?}, the way additive and multiplicative rules are
realised algebraically is related to models of separation logic.
Models of separation logic typically use \emph{partial} commutative monoids to
model a heap, so it is tempting to generalise the commutative monoid of
addition in our semirings to a \emph{partial} commutative monoid.
However, we find that the most natural notion of \emph{partial semiring} is
degenerate, in that all partial semirings are actually (total) semirings.

Recall that a commutative monoid (or commutative monoid object) can be
defined in any symmetric monoidal category.
A partial commutative monoid is exactly a commutative monoid object in the
category of sets and partial functions with the usual monoidal product
\todo{Define this monoidal product somewhere.}.
However, semirings need a Cartesian category in order to state the interaction
equations between addition and multiplication.
While the category of sets and partial functions is not Cartesian, the
standard way to manufacture a Cartesian category out of a symmetric monoidal
category $\mathcal C$ is to take the category of cocommutative comonoids
$\mathrm{CComon}(\mathcal C)$.
Intuitively, the cocommutative comonoid structure equips the underlying
object $M$ with a \emph{delete} map $\eta : M \to I$ and a \emph{duplicate}
map $\delta : M \to M \otimes M$ which are coherent with respect to each other.
All morphisms in $\mathrm{CComon}(\mathcal C)$ must respect $\eta$ and
$\delta$; in particular, both addition and multiplication must separately
form bimonoids in $\mathcal C$ together with the cocommutative comonoid.

\[
  \begin{tikzpicture}[baseline]
    \path
    (-1,1) node(0) {0}
    (1,2) node(x) {}
    (0,0) node(*) {*}
    (0,-1) node(res) {}
    ;

    \draw (0) -- (*);
    \draw (x) to[out=270,in=45] (*);
    \draw (*) -- (res);
  \end{tikzpicture}
  =\quad
  \begin{tikzpicture}[baseline]
    \path
    (0,0) node(0) {0}
    (0,2) node(x) {}
    (0,-1) node(res) {}
    (0,1) node[circle,draw](del) {}
    ;

    \draw (0) -- (res);
    \draw (x) -- (del);
  \end{tikzpicture}
  \quad=
  \begin{tikzpicture}[baseline]
    \path
    (1,1) node(0) {0}
    (-1,2) node(x) {}
    (0,0) node(*) {*}
    (0,-1) node(res) {}
    ;

    \draw (0) -- (*);
    \draw (x) to[out=270,in=135] (*);
    \draw (*) -- (res);
  \end{tikzpicture}
\]
\begin{displaymath}
  \begin{matrix}
    \begin{tikzpicture}[baseline]
      \path
      (-1,2) node(x) {}
      (0,2) node(y) {}
      (-0.5,1) node(+) {+}
      (1,2) node(z) {}
      (0,0) node(*) {*}
      (0,-1) node(res) {}
      ;

      \draw (x) to[out=270,in=135] (+);
      \draw (y) to[out=270,in=45] (+);
      \draw (+) to[out=270,in=135] (*);
      \draw (z) to[out=270,in=45] (*);
      \draw (*) -- (res);
    \end{tikzpicture}
    =
    \begin{tikzpicture}[baseline]
      \path
      (-2,3) node(x) {}
      (-1,3) node(y) {}
      (0,3) node(z) {}
      (0,2) node[circle,draw](dup) {}
      (-1,1) node(x*) {*}
      (0,1) node(y*) {*}
      (-0.5,0) node(+) {+}
      (-0.5,-1) node(res) {}
      ;

      \draw (z) -- (dup);
      \draw (x) to[out=270,in=135] (x*);
      \draw (y) to[out=270,in=135] (y*);
      \draw (dup) to[out=180,in=45] (x*);
      \draw (dup) -- (y*);
      \draw (x*) to[out=270,in=135] (+);
      \draw (y*) to[out=270,in=45] (+);
      \draw (+) -- (res);
    \end{tikzpicture}
    &\phantom{mmmm}&
    \begin{tikzpicture}[baseline]
      \path
      (1,2) node(x) {}
      (0,2) node(y) {}
      (0.5,1) node(+) {+}
      (-1,2) node(z) {}
      (0,0) node(*) {*}
      (0,-1) node(res) {}
      ;

      \draw (x) to[out=270,in=45] (+);
      \draw (y) to[out=270,in=135] (+);
      \draw (+) to[out=270,in=45] (*);
      \draw (z) to[out=270,in=135] (*);
      \draw (*) -- (res);
    \end{tikzpicture}
    =
    \begin{tikzpicture}[baseline]
      \path
      (2,3) node(x) {}
      (1,3) node(y) {}
      (0,3) node(z) {}
      (0,2) node[circle,draw](dup) {}
      (1,1) node(x*) {*}
      (0,1) node(y*) {*}
      (0.5,0) node(+) {+}
      (0.5,-1) node(res) {}
      ;

      \draw (z) -- (dup);
      \draw (x) to[out=270,in=45] (x*);
      \draw (y) to[out=270,in=45] (y*);
      \draw (dup) to[out=0,in=135] (x*);
      \draw (dup) -- (y*);
      \draw (x*) to[out=270,in=45] (+);
      \draw (y*) to[out=270,in=135] (+);
      \draw (+) -- (res);
    \end{tikzpicture}
  \end{matrix}
\end{displaymath}

\begin{lemma}
  For each object $X$ in $(\mathrm{Set}_{\mathrm{part}}, {\otimes})$, there is
  a cocommutative comonoid over $X$.
\end{lemma}
\begin{proof}
  Let $\eta(x) \coloneq ()$ and $\delta(x) \coloneq (x, x)$, with both
  being defined for all $x$.
  Checking that these satisfy the laws is routine.
  Alternatively, we can see that both $\eta$ and $\delta$, being total, are
  morphisms in $\mathrm{Set}$, where it is well known that they form a
  cocommutative comonoid.
  The equations in $\mathrm{Set}$ carry over to $\mathrm{Set}_{\mathrm{part}}$.
\end{proof}

\begin{lemma}
  For each object $X$ in $(\mathrm{Set}_{\mathrm{part}}, {\otimes})$, any
  comonoid over $X$ is the one described in the previous lemma.
\end{lemma}
\begin{proof}
  The left unit law says that, for all $x$ and $x'$, we have
  $\exists y.~\delta(x) = (y, x') \land \eta(y) = ()$ if and only if $x = x'$.
  Letting $x'$ be $x$ and reading from right to left, we get that there is
  some $y$ such that $\delta(x) = (y, x)$ and $\eta(y) = ()$.
  Symmetrically, from the right unit law, there is some $z$ such that
  $\delta(x) = (x, z)$ and $\eta(z) = ()$.
  But because $\delta$, as a partial function, is deterministic, we have
  $(y, x) = (x, z)$, giving us that $y = z = x$, and $\delta(x) = (x, x)$.
  Moreover, $\eta(x) = \eta(y) = ()$.
\end{proof}
