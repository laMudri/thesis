The typing rules of $\name$ presented in \cref{fig:lr} contain a lot of detail
and repeated patterns.
For example, nearly half of the rules include the premise
$\grR \leq \grP + \grQ$.
Also, the presence of usage annotations, which are often different in different
parts of a rule, means that we keep repeating the context.
Explicit contexts go against the style we established in \cref{sec:simple},
which is based
around being parametric in the context, so that substitution is agnostic to the
details of typing rules.

To encapsulate the repeated patterns and facilitate an implicit context style,
I introduce the \emph{bunched connectives} for premises.
These are inspired by bunched logic~\citep{oHP99}, and will not only be used for
stating the syntax, but will be used as an abstraction of common patterns in the
development of the metatheory.
The idea is to generalise the space between premises from Gentzen's natural
deduction to allow for any linear combination of usage annotations.
Among other things, this generalisation will allow us to distinguish between
$\with$-introduction and $\otimes$-introduction by a choice of connective:
either \emph{sharing} or \emph{separating} conjunction.
%Similar connectives, but with different interpretations, could be used to
%define other linear-like type theories, like DILL, but here I will focus on the
%usage annotation style.
These connectives are defined in \cref{fig:bunched} in Agda notation.

\begin{figure}
  %\begin{align*}
  %  \dot1\,\grR &\coloneqq 1 \\
  %  (T \dottimes U)\,\grR &\coloneqq T\,\grR \times U\,\grR \\
  %  (T \dotto U)\,\grR &\coloneqq T\,\grR \to U\,\grR \\
  %  I^*\,\grR &\coloneqq \grR \leq \gr0 \\
  %  (T \sep U)\,\grR &\coloneqq \Sigma \grP,\grQ.~\plr{\grR \leq \grP + \grQ}
  %                     \times T\,\grP \times U\,\grQ \\
  %  (\gr r \cdot T)\,\grR &\coloneqq \Sigma \grP.~\plr{\grR \leq \gr r\grP}
  %                     \times T\,\grP
  %\end{align*}
  Parameters:
  \ExecuteMetaData[\Bunchedtex]{BunchedParams}

  Connectives:
  \vspace{-1em}
  \begin{multicols}{2}
    \noindent\ExecuteMetaData[\Bunchedtex]{PointwiseUnit}
    \noindent\ExecuteMetaData[\Bunchedtex]{PointwiseConjunction}
    \columnbreak
    %\AgdaNoSpaceAroundCode{}
    \noindent\ExecuteMetaData[\Bunchedtex]{Entails}
    %\AgdaSpaceAroundCode{}
  \end{multicols}
  \vspace{-2em}
  \ExecuteMetaData[\Bunchedtex]{BunchedUnit}
  \ExecuteMetaData[\Bunchedtex]{BunchedConjunction}
  \ExecuteMetaData[\Bunchedtex]{BunchedImplication}
  \ExecuteMetaData[\Bunchedtex]{BunchedScaling}
  \caption{The bunched connectives}
  \label{fig:bunched}
\end{figure}

The bunched connectives are parametrised over two sets and three relations.
For syntax, the set \AgdaBound{A} will be \AgdaRecord{Ctx}, the type of
contexts, and \AgdaBound{R} will be \AgdaField{Ann}, the type of usage
annotations (scalars).
For the relations, the notation is meant to be suggestive, with
\AgdaBound{$\Gamma$}\AgdaSpace{}\AgdaBound{$\leq$0} typically stating that all
of the annotations in \AgdaBound{$\Gamma$} are less than or equal to $0$;
\AgdaBound{$\Gamma$}\AgdaSpace{}\AgdaBound{$\leq$[}\AgdaSpace{}%
\AgdaBound{$\Delta$}\AgdaSpace{}\AgdaBound{+}\AgdaSpace{}\AgdaBound{$\Theta$}%
\AgdaSpace{}\AgdaBound{]}
typically stating that \AgdaBound{$\Gamma$}, \AgdaBound{$\Delta$}, and
\AgdaBound{$\Theta$} all agree on their types but the usage context of
\AgdaBound{$\Gamma$} is less than or equal to the sum of the usage contexts of
\AgdaBound{$\Delta$} and \AgdaBound{$\Theta$}; and
\AgdaBound{$\Gamma$}\AgdaSpace{}\AgdaBound{$\leq$[}\AgdaSpace{}%
\AgdaBound{r}\AgdaSpace{}\AgdaBound{*$_l$}\AgdaSpace{}\AgdaBound{$\Delta$}%
\AgdaSpace{}\AgdaBound{]}
typically stating that \AgdaBound{$\Gamma$} and \AgdaBound{$\Delta$} agree on
their types but have the evident scaling relationship with \AgdaBound{r} on
their usage annotations.
I use the same symbols for the connectives both in Agda code and in otherwise
standard mathematical/logical notation.

The first two connectives are those we've already seen for intuitionistic
systems --- $\dot1$ and $\dottimes$.
The absence of premises is encoded by $\dot1$, while the space between premises
sharing a context is encoded by $\dottimes$.
As for implication, I temporarily avoid giving a $\dotto$ connectives, instead
fusing it together with $\forallb{-}$ to produce the \emph{set} of
context-preserving functions $T \rightarrowtriangle U$.
When we interpret a typing rule as a constructor of an inductive definition,
$\rightarrowtriangle$
interprets the horizontal line, reflecting the fact that the usage annotations
we start off with in the premises are those of the conclusion, corresponding to
a general principle of resource conservation.
The prototypical rules that use $\dot1$ and $\dottimes$ are the introduction
rules for $\top$ and $\with$, respectively.

\begin{align*}
  \begin{prooftree}
    \infer0{\grR\Gamma \vdash \top}
  \end{prooftree}
  &\quad\rightsquigarrow\quad
  \begin{prooftree}[comb]
    \hypo{\dot1}
    \infer1{\vdash \top}
  \end{prooftree}
  \\\\
  \begin{prooftree}
    \hypo{\grR\Gamma \vdash A}
    \hypo{\grR\Gamma \vdash B}
    \infer2{\grR\Gamma \vdash A \with B}
  \end{prooftree}
  &\quad\rightsquigarrow\quad
  \begin{prooftree}[comb]
    \hypo{\vdash A}
    \hypo{\dottimes}
    \hypo{\vdash B}
    \infer3{\vdash A \with B}
  \end{prooftree}
\end{align*}

The rest of the bunched connectives --- $I^*$, $\sep$, $\cdot$, and $\wand$ ---
involve linear decompositions of the usage annotations.
The three basic left semimodule operators --- zero, addition, and left-scaling
--- each get a bunched connective --- $I^*$, $\sep$, and $\gr r \cdot {}$,
respectively.
The prototypical typing rules for each of these three connectives are the
introduction rules for $I$, $\otimes$, and $\oc{\gr r}$, respectively.

\begin{align*}
  \begin{prooftree}
    \hypo{\grR \leq \gr0}
    \infer1{\grR\Gamma \vdash I}
  \end{prooftree}
  &\quad\rightsquigarrow\quad
  \begin{prooftree}[comb]
    \hypo{I^*}
    \infer1{\vdash I}
  \end{prooftree}
  \\\\
  \begin{prooftree}
    \hypo{\grP\Gamma \vdash A}
    \hypo{\grQ\Gamma \vdash B}
    \hypo{\grR \leq \grP + \grQ}
    \infer3{\grR\Gamma \vdash A \otimes B}
  \end{prooftree}
  &\quad\rightsquigarrow\quad
  \begin{prooftree}[comb]
    \hypo{\vdash A}
    \hypo{\sep}
    \hypo{\vdash B}
    \infer3{\vdash A \otimes B}
  \end{prooftree}
  \\\\
  \begin{prooftree}
    \hypo{\grP\Gamma \vdash A}
    \hypo{\grR \leq \gr r\grP}
    \infer2{\grR\Gamma \vdash \oc{\gr r} A}
  \end{prooftree}
  &\quad\rightsquigarrow\quad
  \begin{prooftree}[comb]
    \hypo{\gr r \cdot (\vdash A)}
    \infer1{\vdash \oc{\gr r} A}
  \end{prooftree}
\end{align*}

\subsection{$\name$ stated using bunched connectives}\label{sec:lr-bunched}
The full system \name{} is stated in terms of bunched connectives in
\cref{fig:lr-bunched}.
The bunched connectives also yield a reasonably concise definition of the Agda
data type of \name{} derivations, as seen in \cref{fig:lr-bunched-Agda}.

\begin{figure}
  \ebproofset{separation=0.75em}
  \begin{displaymath}
    \begin{prooftree}
      \hypo{\sqni A}
      \infer1[Var]{\vdash A}
    \end{prooftree}
  \end{displaymath}

  \begin{displaymath}
    \begin{matrix}
      \begin{prooftree}
        \hypo{I^*}
        \infer1[$I$-I]{\vdash I}
      \end{prooftree}
      &&
      \begin{prooftree}
        \hypo{\vdash I}
        \hypo{\sep}
        \hypo{\vdash C}
        \infer3[$I$-E]{\vdash C}
      \end{prooftree}
      \\\\
      \begin{prooftree}
        \hypo{\vdash A}
        \hypo{\sep}
        \hypo{\vdash B}
        \infer3[$\otimes$-I]{\vdash A \otimes B}
      \end{prooftree}
      &&
      \begin{prooftree}
        \hypo{\vdash A \otimes B}
        \hypo{\sep}
        \hypo{\gr1A, \gr1B \vdash C}
        \infer3[$\otimes$-E]{\vdash C}
      \end{prooftree}
      \\\\
      \begin{prooftree}
        \hypo{\gr1A \vdash B}
        \infer1[$\multimap$-I]{\vdash A \multimap B}
      \end{prooftree}
      &&
      \begin{prooftree}
        \hypo{\vdash A \multimap B}
        \hypo{\sep}
        \hypo{\vdash A}
        \infer3[$\multimap$-E]{\vdash B}
      \end{prooftree}
      \\\\
      \begin{prooftree}
        \hypo{\dot1}
        \infer1[$\top$-I]{\vdash \top}
      \end{prooftree}
      &&
      \textrm{(no $\top$-E)}
      \\\\
      \begin{prooftree}
        \hypo{\vdash A}
        \hypo{\dottimes}
        \hypo{\vdash B}
        \infer3[$\with$-I]{\vdash A \with B}
      \end{prooftree}
      &&
      \begin{prooftree}
        \hypo{\vdash A_0 \with A_1}
        \infer1[$\with$-E$_i$, for $i \in \{0,1\}$]{\vdash A_i}
      \end{prooftree}
      \\\\
      \textrm{(no $0$-I)}
      &&
      \begin{prooftree}
        \hypo{\vdash 0}
        \hypo{\sep}
        \hypo{\dot1}
        \infer3[$0$-E]{\vdash C}
      \end{prooftree}
      \\\\
      \begin{prooftree}
        \hypo{\vdash A_i}
        \infer1[$\oplus$-I$_i$, for $i \in \{0,1\}$]{\vdash A_0 \oplus A_1}
      \end{prooftree}
      &&
      \begin{prooftree}
        \hypo{\vdash A \oplus B}
        \hypo{\sep}
        \hypo{(\gr1A \vdash C}
        \hypo{\dottimes}
        \hypo{\gr1B \vdash C)}
        \infer5[$\oplus$-E]{\vdash C}
      \end{prooftree}
      \\\\
      \begin{prooftree}
        \hypo{\gr r \cdot (\vdash A)}
        \infer1[$\oc$-I]{\vdash \oc{\gr r} A}
      \end{prooftree}
      &&
      \begin{prooftree}
        \hypo{\vdash \oc{\gr r} A}
        \hypo{\sep}
        \hypo{\gr rA \vdash C}
        \infer3[$\oc$-E]{\vdash C}
      \end{prooftree}
    \end{matrix}
  \end{displaymath}
  \ebproofset{separation=1.5em}
  \caption{\name{} stated using bunched connectives}
  \label{fig:lr-bunched}
\end{figure}

\begin{figure}
  \ExecuteMetaData[\SpecificSyntaxtex]{Ty}
  \ExecuteMetaData[\SpecificSyntaxtex]{Bind}
  \ExecuteMetaData[\SpecificSyntaxtex]{lr}
  \caption{\name{} stated using bunched connectives in Agda}
  \label{fig:lr-bunched-Agda}
\end{figure}

\subsection{Connection with bunched logic}\label{sec:bunched-logic}
While we have seen a connection between the bunched connectives and the
connectives of $\name$, the two should not be confused.
In particular, the bunched connectives obey different laws to what we would
expect from linear logic.
For example, it would make sense to define a bunched connective $\dotplus$,
defined analogously to $\dottimes$.
This $\dotplus$ could be used to rephrase the introduction rules for $\oplus$.
We then have maps both ways between $T \dottimes (U \dotplus V)$ and
$(T \dottimes U) \dotplus (T \dottimes V)$, reminiscent of the distributivity of
additive connectives in bunched logic,
whereas linear logic only has a map from $(A \with B) \oplus (A \with C)$ to
$A \with (B \oplus C)$, and not a map the other way.
Looking at the interpretations, the connection with bunched logic makes sense.
Instead of the partial commutative monoid (often representing heaps) found in
standard semantics of bunched logic, we have a left semimodule of usage
contexts, which we are similarly interested in splitting and sharing between
various subterms.

From bunched logic, we would expect the Cartesian product $\dottimes$ to have an
internal hom.
In the intuitionistic case, $\dotto$ filled this role.
However, with usage contexts, it makes sense for open types to be presheaves
over the partial order of contexts under pointwise $\leq$ of usage annotations.
The family $T \dotto U$ does not satisfy the functoriality condition because of
the contravariance in the domain $T$.
Instead, as found in models of bunched logic, we would want a Kripke function
space, like
$\lambda\Gamma.~\forall \Gamma' \leq \Gamma.~T\,\Gamma' \to U\,\Gamma'$.
However, I do not make use of such a connective.

The separating conjunction $\sep$ can be seen as a decategorified version of Day
convolution~\citep{Day70}.
It also resembles the use of ternary frames in semantics of non-distributive
logics~\citep[chapter 12]{Restall99}.

\subsection{Operations on bunched connectives}\label{sec:bunched-op}
To manipulate terms and other open types defined using bunched connectives, we
need the zero, addition, and multiplication relations to satisfy some laws.
For example, to achieve a symmetry map $T \sep U \rightarrowtriangle U \sep T$,
we need addition to satisfy the commutativity law
$\forall x,y,z : A.~x \leq y + z \to x \leq z + y$.

For all uses of bunched connectives in this thesis, the carrier set $A$ will
form a partial order --- for example, with contexts, the order is given by the
pointwise order on the usage vectors.
We then consider the category whose objects are posets and whose morphisms are
relations $R : A \rel B$ satisfying the contravariant-covariant law
$\forall x,x',y,y'.~x' \leq x \to y \leq y' \to xRy \to x'Ry'$.
This category can be given the usual monoidal product of relations, which is
the pointwise product of posets on objects.
Then, we will always expect zero and addition to together form a cocommutative
comonoid in this category.
With this structure, we can get the following equivalences and functions.

\begin{mathpar}
  I^* \sep T \leftrightarrowtriangle T \and
  T \sep I^* \leftrightarrowtriangle T \and
  (T \sep U) \sep V \leftrightarrowtriangle T \sep (U \sep V) \and
  T \sep U \leftrightarrowtriangle U \sep T \and
  (T \wand U) \sep T \rightarrowtriangle U \and
  I^* \wand T \leftrightarrowtriangle T \and
  (T \sep U) \wand V \leftrightarrowtriangle T \wand (U \wand V)
\end{mathpar}

I do not use algebraic properties of multiplication in conjunction with
manipulation of bunched connectives in this thesis, but we could expect
scalar multiplication
to add a comodule structure over cosemiring $R$ to the cocommutative comonoid
given by zero and addition.
