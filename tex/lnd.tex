%This should stay fairly general, so it could be compiled to a DILL-style
%calculus as well as the one with usage annotations.
%Our framework compiles a ND system down to a sequent calculus.
%It isolates the building blocks of typing rules ($\vdash$, $\wedge$, $*$,
%$\Box$, $\cdot$) so that we can do more generic programming (e.g., usage
%checker).

The typing rules presented in the previous section contain a lot of detail and
repeated patterns.
For example, nearly half of the rules include the premise
$\grR \leq \grP + \grQ$.
Also, the presence of usage annotations, which are often different in different
parts of a rule, means that we keep repeating the context.
Explicit contexts go against the style of natural deduction, which is based
around being parametric in the context, so that substitution is agnostic to the
details of typing rules.

To encapsulate the repeated patterns and facilitate an implicit context style,
I introduce the \emph{bunched connectives}.
These are inspired by bunched logic~\cite{oHP99}, and will not only be used for
stating the syntax, but will reappear in the semantics.
The idea is to generalise the space between premises from Gentzen's natural
deduction so as to allow for any linear combination of usage annotations.
Among other things, this generalisation will allow us to distinguish between
$\with$-introduction and $\otimes$-introduction by a choice of connective:
either \emph{sharing} or \emph{separating} conjunction.
Similar connectives, but with different interpretations, could be used to
define other linear-like type theories, like DILL, but here I will focus on the
usage annotation style.
The interpretations I use are defined in \cref{fig:bunched}.

\begin{figure}
  \begin{align*}
    \dot1\,\grR &\coloneqq 1 \\
    (T \dottimes U)\,\grR &\coloneqq T\,\grR \times U\,\grR \\
    (T \dotto U)\,\grR &\coloneqq T\,\grR \to U\,\grR \\
    I^*\,\grR &\coloneqq \grR \leq \gr0 \\
    (T \sep U)\,\grR &\coloneqq \Sigma \grP,\grQ.~\plr{\grR \leq \grP + \grQ}
                       \times T\,\grP \times U\,\grQ \\
    (\gr r \cdot T)\,\grR &\coloneqq \Sigma \grP.~\plr{\grR \leq \gr r\grP}
                       \times T\,\grP
  \end{align*}
  \caption{The bunched connectives}
  \label{fig:bunched}
\end{figure}

The simplest connectives are those we've already seen for intuitionistic
systems --- $\dot1$, $\dottimes$, and $\dotto$.
The absence of premises is encoded by $\dot1$, while the space between premises
sharing a context is encoded by $\dottimes$.
When we interpret a rule as a constructor of an inductive definition, $\dotto$
interprets the horizontal line, reflecting the fact that the usage annotations
we start off with in the premises are those of the conclusion.
The prototypical rules that use $\dot1$ and $\dottimes$ are the introduction
rules for $\top$ and $\with$, respectively.

\begin{align*}
  \begin{prooftree}
    \infer0{\grR\Gamma \vdash \top}
  \end{prooftree}
  &\quad\rightsquigarrow\quad
  \begin{prooftree}[comb]
    \hypo{\dot1}
    \infer1{\vdash \top}
  \end{prooftree}
  \\\\
  \begin{prooftree}
    \hypo{\grR\Gamma \vdash A}
    \hypo{\grR\Gamma \vdash B}
    \infer2{\grR\Gamma \vdash A \with B}
  \end{prooftree}
  &\quad\rightsquigarrow\quad
  \begin{prooftree}[comb]
    \hypo{\vdash A}
    \hypo{\dottimes}
    \hypo{\vdash B}
    \infer3{\vdash A \with B}
  \end{prooftree}
\end{align*}

The rest of the bunched connectives are for linear decompositions of the usage
annotations.
The three basic left semimodule operators --- zero, addition, and left-scaling
--- each get a bunched connective --- $I^*$, $\sep$, and $\gr r \cdot {}$,
respectively.
The prototypical typing rules for each of these three connectives are the
introduction rules for $I$, $\otimes$, and $\oc_{\gr r}$, respectively.

\begin{align*}
  \begin{prooftree}
    \hypo{\grR \leq \gr0}
    \infer1{\grR\Gamma \vdash I}
  \end{prooftree}
  &\quad\rightsquigarrow\quad
  \begin{prooftree}[comb]
    \hypo{I^*}
    \infer1{\vdash I}
  \end{prooftree}
  \\\\
  \begin{prooftree}
    \hypo{\grP\Gamma \vdash A}
    \hypo{\grQ\Gamma \vdash B}
    \hypo{\grR \leq \grP + \grQ}
    \infer3{\grR\Gamma \vdash A \otimes B}
  \end{prooftree}
  &\quad\rightsquigarrow\quad
  \begin{prooftree}[comb]
    \hypo{\vdash A}
    \hypo{\sep}
    \hypo{\vdash B}
    \infer3{\vdash A \otimes B}
  \end{prooftree}
  \\\\
  \begin{prooftree}
    \hypo{\grP\Gamma \vdash A}
    \hypo{\grR \leq \gr r\grP}
    \infer2{\grR\Gamma \vdash \oc_{\gr r} A}
  \end{prooftree}
  &\quad\rightsquigarrow\quad
  \begin{prooftree}[comb]
    \hypo{\gr r \cdot (\vdash A)}
    \infer1{\vdash \oc_{\gr r} A}
  \end{prooftree}
\end{align*}

The full system \name{} is stated in terms of bunched connectives in
\cref{fig:lr-bunched}.
%I have not included a variable rule because, like with intuitionistic
%natural deduction,

These bunched connectives should not be confused for the linear connectives
they appear in the introduction rules of.
For example, it would make sense to define a bunched connective $\dotplus$,
defined analogously to $\dottimes$.
This $\dotplus$ could be used to rephrase the introduction rules for $\oplus$.
We then have maps both ways between $T \dottimes (U \dotplus V)$ and
$(T \dottimes U) \dotplus (T \dottimes V)$, reminiscent of bunched logic,
whereas linear logic only has a map from $(A \with B) \oplus (A \with C)$ to
$A \with (B \oplus C)$, and not a map the other way.
%For example, linearly we have
%$\oc_{\gr r}(A \with B) \simeq \oc_{\gr r} A \otimes \oc_{\gr r} B$, while the
%expressions $\gr r \cdot (A \dottimes B)$ and $\gr r \cdot A \sep \gr r \cdot B$
%are generally incomparable.
We will also see later\todo{Forward reference} that, while
$(\dottimes, \dotto)$ gives a Cartesian
closed structure, $\sep$ is monoidally closed, with an internal hom $\wand$.
Indeed, the Cartesian closed structure is sufficient to give us the interaction
between $\dotplus$ and $\dottimes$, by the fact that $A \dottimes {-}$ is a
left adjoint, and therefore preserves colimits.
Looking at the interpretations, the connection with bunched logic makes sense.
Instead of the partial commutative monoid (representing heaps) found in
standard semantics of bunched logic, we have a left semimodule of usage
contexts, which we are similarly interested in splitting and sharing between
various subterms.

\begin{figure}
  \ebproofset{separation=0.75em}
  \begin{displaymath}
    \begin{prooftree}
      \hypo{\sqni A}
      \infer1[Var]{\vdash A}
    \end{prooftree}
  \end{displaymath}

  \begin{displaymath}
    \begin{matrix}
      \begin{prooftree}
        \hypo{I^*}
        \infer1[$I$-I]{\vdash I}
      \end{prooftree}
      &&
      \begin{prooftree}
        \hypo{\vdash I}
        \hypo{\sep}
        \hypo{\vdash C}
        \infer3[$I$-E]{\vdash C}
      \end{prooftree}
      \\\\
      \begin{prooftree}
        \hypo{\vdash A}
        \hypo{\sep}
        \hypo{\vdash B}
        \infer3[$\otimes$-I]{\vdash A \otimes B}
      \end{prooftree}
      &&
      \begin{prooftree}
        \hypo{\vdash A \otimes B}
        \hypo{\sep}
        \hypo{\gr1A, \gr1B \vdash C}
        \infer3[$\otimes$-E]{\vdash C}
      \end{prooftree}
      \\\\
      \begin{prooftree}
        \hypo{\gr1A \vdash B}
        \infer1[$\multimap$-I]{\vdash A \multimap B}
      \end{prooftree}
      &&
      \begin{prooftree}
        \hypo{\vdash A \multimap B}
        \hypo{\sep}
        \hypo{\vdash A}
        \infer3[$\multimap$-E]{\vdash B}
      \end{prooftree}
      \\\\
      \begin{prooftree}
        \hypo{\dot1}
        \infer1[$\top$-I]{\vdash \top}
      \end{prooftree}
      &&
      \textrm{(no $\top$-E)}
      \\\\
      \begin{prooftree}
        \hypo{\vdash A}
        \hypo{\dottimes}
        \hypo{\vdash B}
        \infer3[$\with$-I]{\vdash A \with B}
      \end{prooftree}
      &&
      \begin{prooftree}
        \hypo{\vdash A_0 \with A_1}
        \infer1[$\with$-E$_i$, for $i \in \{0,1\}$]{\vdash A_i}
      \end{prooftree}
      \\\\
      \textrm{(no $0$-I)}
      &&
      \begin{prooftree}
        \hypo{\vdash 0}
        \hypo{\sep}
        \hypo{\dot1}
        \infer3[$0$-E]{\vdash C}
      \end{prooftree}
      \\\\
      \begin{prooftree}
        \hypo{\vdash A_i}
        \infer1[$\oplus$-I$_i$, for $i \in \{0,1\}$]{\vdash A_0 \oplus A_1}
      \end{prooftree}
      &&
      \begin{prooftree}
        \hypo{\vdash A \oplus B}
        \hypo{\sep}
        \hypo{(\gr1A \vdash C}
        \hypo{\dottimes}
        \hypo{\gr1B \vdash C)}
        \infer5[$\oplus$-E]{\vdash C}
      \end{prooftree}
      \\\\
      \begin{prooftree}
        \hypo{\gr r \cdot (\vdash A)}
        \infer1[$\oc$-I]{\vdash \oc_{\gr r} A}
      \end{prooftree}
      &&
      \begin{prooftree}
        \hypo{\vdash \oc_{\gr r} A}
        \hypo{\sep}
        \hypo{\gr rA \vdash C}
        \infer3[$\oc$-E]{\vdash C}
      \end{prooftree}
    \end{matrix}
  \end{displaymath}
  \ebproofset{separation=1.5em}
  \caption{\name{} stated using bunched connectives}
  \label{fig:lr-bunched}
\end{figure}
