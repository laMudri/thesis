We can express Benton's linear/non-linear logic~\cite{Benton94} in the
framework.
There are two apparent discrepancies between L/nL and what we have seen so far
to be expressible in the framework: the existence of multiple judgement modes
and the restriction on the kinds of assumptions based on the mode.
We will start by addressing these discrepancies, before presenting the encoding
in full and proving the resulting system logically equivalent to
$\lambda\gr{\mathcal R}$.

\subsection{Encoding L/nL}

In L/nL, we have a \emph{linear} and an \emph{intuitionistic} mode, with
separate types (respectively $A$ and $X$) and typing judgements (respectively
$\vdash_{\mathcal L}$ and $\vdash_{\mathcal C}$) connected by modalities $F$ and
$G$.
The mode of the conclusion type is the same as the mode of the judgement, so
only a distinction in the conclusion mode is necessary.
To achieve this distinction in the types, we have an indexed type family
$\mathrm{Ty} : \mathrm{Frag} \to \mathrm{Set}$, and set the framework types to
be the type $(f : \mathrm{Frag}) \times \mathrm{Ty}\,f$.
% \AgdaDatatype{Ty}\AgdaSymbol{:}\AgdaDatatype{Frag}\AgdaSymbol{$\to$}
% \AgdaDatatype{Set}

In L/nL, while linear judgements can have both linear and intuitionistic
assumptions, intuitionistic judgements can only have intuitionistic assumptions.
We encode this invariant on intuitionistic judgements using the $\Box$
premise combinator on all intuitionistic conclusions and wherever
intuitionistic premises occur for linear conclusions.

\subsection{Translating between L/nL and $\lambda\gr{\mathcal R}$}

The translations largely follow Benton's originals.
