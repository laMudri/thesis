We can express Benton's linear/non-linear logic~\cite{Benton94} in the
framework.
There are two apparent discrepancies between L/nL and what we have seen so far
to be expressible in the framework: the existence of multiple judgement modes
and the restriction on the kinds of assumptions based on the mode.
We will start by addressing these discrepancies, before presenting the encoding
in full and proving the resulting system logically equivalent to
$\lambda\gr{\mathcal R}$.

\subsection{Encoding L/nL}

In L/nL, we have a \emph{linear} and an \emph{intuitionistic} mode, with
separate types (respectively $A$ and $X$) and typing judgements (respectively
$\vdash_{\mathcal L}$ and $\vdash_{\mathcal C}$) connected by modalities $F$ and
$G$.
The mode of the conclusion type is the same as the mode of the judgement, so
only a distinction in the conclusion mode is necessary.
To achieve this distinction in the types, I have an indexed type family
$\mathrm{Ty} : \mathrm{Frag} \to \mathrm{Set}$, and set the framework types to
be the type $(f : \mathrm{Frag}) \times \mathrm{Ty}\,f$.
% \AgdaDatatype{Ty}\AgdaSymbol{:}\AgdaDatatype{Frag}\AgdaSymbol{$\to$}
% \AgdaDatatype{Set}
I present these judgements in a standard paper style in \cref{fig:lnl-types}.

In L/nL, while linear judgements can have both linear and intuitionistic
assumptions, intuitionistic judgements can only have intuitionistic assumptions.
I encode this invariant on intuitionistic judgements using the $\Box$
premise combinator on all intuitionistic conclusions and wherever
intuitionistic premises occur for linear conclusions, as seen in
\cref{fig:lnl-desc}.
This scheme means that the rules contain the maximal number of boxes, which is
convenient when consuming terms.
For ease of \emph{producing} terms, we would want a ``garbage in; garbage out''
style with a minimal number of boxes, which could be achieved in two different
ways:
\begin{itemize}
  \item Ensure that $\Gamma, \gr1\,\lin A \vdash \intu X$ is never derivable.
    This leaves boxes only on rules $1$-I and $G$-I.
  \item Ensure that, when working bottom-up, we never transition from a linear
    to an intuitionistic conclusion with linear assumptions present.
    This leaves boxes only on rules $F$-I and $G$-E.
\end{itemize}
It would be interesting to show the equivalence of these three approaches, but
that is left to future work.

\begin{figure}
  \begin{align*}
    A, B, C &\Coloneqq I \mid A \otimes B \mid A \multimap B \mid FX \\
    X, Y, Z &\Coloneqq 1 \mid X \times Y \mid X \to Y \mid GA \\
    \mathcal J &\Coloneqq \lin A \mid \intu X
  \end{align*}
  \caption{Types and judgement forms of L/nL}
  \label{fig:lnl-types}
\end{figure}

\begin{figure}
  \begin{mathpar}
    % lin rules
    \ebrule[comb]{%
      \hypo{I^*}
      \infer1[$I$-I]{\vdash \lin I}
    }
    \and
    \ebrule[comb]{%
      \hypo{\vdash \lin I}
      \hypo{*}
      \hypo{\vdash \lin C}
      \infer3[$I$-E]{\vdash \lin C}
    }
    \and
    \ebrule[comb]{%
      \hypo{\vdash \lin A}
      \hypo{*}
      \hypo{\vdash \lin B}
      \infer3[$\otimes$-I]{\vdash \lin A \otimes B}
    }
    \and
    \ebrule[comb]{%
      \hypo{\vdash \lin A \otimes B}
      \hypo{*}
      \hypo{\gr1\,\lin A, \gr1\,\lin B \vdash \lin C}
      \infer3[$\otimes$-E]{\vdash \lin C}
    }
    \and
    \ebrule[comb]{%
      \hypo{\gr1\,\lin A \vdash \lin B}
      \infer1[$\multimap$-I]{\vdash \lin A \multimap B}
    }
    \and
    \ebrule[comb]{%
      \hypo{\vdash \lin A \multimap B}
      \hypo{*}
      \hypo{\vdash \lin A}
      \infer3[$\multimap$-E]{\vdash \lin B}
    }
    \and
    \ebrule[comb]{%
      \hypo{\Box\plr{\vdash \intu X}}
      \infer1[$F$-I]{\vdash \lin FX}
    }
    \and
    \ebrule[comb]{%
      \hypo{\vdash \lin FX}
      \hypo{*}
      \hypo{\gr\omega\,\intu X \vdash \lin C}
      \infer3[$F$-E]{\vdash \lin C}
    }
    \and
    % int rules
    \ebrule[comb]{%
      \hypo{\Box\plr{\dot1}}
      \infer1[$1$-I]{\vdash \intu 1}
    }
    \and
    \text{(no $1$-E)}
    \and
    \ebrule[comb]{%
      \hypo{\Box(\vdash \intu X}
      \hypo{\dottimes}
      \hypo{\vdash \intu Y)}
      \infer3[$\times$-I]{\vdash \intu X \times Y}
    }
    \and
    \ebrule[comb]{%
      \hypo{\Box\plr{\vdash \intu X \times Y}}
      \infer1[$\times$-El]{\vdash \intu X}
    }
    \and
    \ebrule[comb]{%
      \hypo{\Box\plr{\vdash \intu X \times Y}}
      \infer1[$\times$-Er]{\vdash \intu Y}
    }
    \and
    \ebrule[comb]{%
      \hypo{\Box\plr{\gr\omega\,\intu A \vdash \intu B}}
      \infer1[$\to$-I]{\vdash \intu A \to B}
    }
    \and
    \ebrule[comb]{%
      \hypo{\Box(\vdash \intu A \to B}
      \hypo{\dottimes}
      \hypo{\vdash \intu A)}
      \infer3[$\to$-E]{\vdash \intu B}
    }
    \and
    \ebrule[comb]{%
      \hypo{\Box\plr{\vdash \lin A}}
      \infer1[$G$-I]{\vdash \intu GA}
    }
    \and
    \ebrule[comb]{%
      \hypo{\Box\plr{\vdash \intu GA}}
      \infer1[$G$-E]{\vdash \lin A}
    }
  \end{mathpar}
  \caption{Description of the logical rules of L/nL}
  \label{fig:lnl-desc}
\end{figure}

\begin{proposition}
  We can construct the following translations.
  \begin{align}
    (\Theta \vdash_{\mathcal C} X) &\to (\gr\omega\Theta \vdash \intu X) \\
    (\Theta; \Gamma \vdash_{\mathcal L} A) &\to
    (\gr\omega\Theta, \gr1\Gamma \vdash \lin A)
  \end{align}
\end{proposition}

\begin{proposition}
  We can construct the following translations, where $\Gamma_{\gr r}$ is the
  list of types in $\Gamma$ that are annotated $\gr r$.%
  \todo{I'm not sure the resultant contexts are well formed.}
  \begin{align}
    (\Gamma \vdash \intu X) &\to (\Gamma_{\gr\omega} \vdash_{\mathcal C} X) \\
    (\Gamma \vdash \lin A) &\to
    (\Gamma_{\gr\omega}; \Gamma_{\gr1} \vdash_{\mathcal L} A)
  \end{align}
\end{proposition}

\subsection{Translating between L/nL and $\lambda\gr{\mathcal R}$}

The translations largely follow Benton's originals.
In this section, I take $\name$ to be the system described in
\cref{fig:lr-bunched} instantiated to the $\{\gr0 > \gr\omega < \gr1\}$
posemiring and restricted to the fragment with only connectives $I$, $\otimes$,
$\multimap$, and $\oc\gr\omega$.
Notably, this excludes $\oc\gr0$ and $\oc\gr1$.
I write $\oc\gr\omega$ as just $\oc$, as in traditional linear logic.
Under these restrictions, Benton's translations of types can be used verbatim
(\cref{fig:lnl-lr-types}).

\begin{figure}
  \centering
  \begin{subfigure}{.49\linewidth}
    \centering
    \begin{align*}
      (-)^\circ : \mathrm{Ty}_{\name} \to \mathrm{Ty}_{\lin} \\
      \begin{aligned}
        I^\circ &= I \\
        \plr{A \otimes B}^\circ &= A^\circ \otimes B^\circ \\
        \plr{A \multimap B}^\circ &= A^\circ \multimap B^\circ \\
        \plr{\oc A}^\circ &= GFA^\circ
      \end{aligned}
    \end{align*}
  \end{subfigure}
  \begin{subfigure}{.49\linewidth}
    \centering
    \begin{align*}
      (-)^* : \mathrm{Ty}_\lnl \to \mathrm{Ty}_{\name} \\
      \begin{aligned}
        I^* &= I \\
        \plr{A \otimes B}^* &= A^* \otimes B^* \\
        \plr{A \multimap B}^* &= A^* \multimap B^* \\
        \plr{FX}^* &= \oc X^* \\
        1^* &= I \\
        \plr{X \times Y}^* &= \oc X^* \otimes \oc Y^* \\
        \plr{X \to Y}^* &= \oc X^* \multimap Y^* \\
        \plr{GA}^* &= A^*
      \end{aligned}
    \end{align*}
  \end{subfigure}
  \caption{Translation of types between L/nL and $\name$}
  \label{fig:lnl-lr-types}
\end{figure}

We extend both $(-)^\circ$ and $(-)^*$ to contexts pointwise on the types.
For $(-)^\circ$, this means that the translation lands outside what is usually
expressible in L/nL whenever the context contains annotation $\gr\omega$.
For example, $\plr{\gr\omega I}^\circ = \plr{\gr\omega\,\lin I}$, and we do not
see ``$\gr\omega\,\lin$'' in translations from Benton's L/nL.
This could be fixed by inserting $G$ in such cases.
As for $(-)^*$, note that we do not insert a $\oc$ on intuitionistic assumptions
as Benton does.
Instead, the annotation $\gr\omega$ achieves the same effect for us.
This discrepancy essentially amounts to the fact that $\name$ is more like DILL
than Girard's ILL, and Benton's translations act upon the latter.

\begin{theorem}
  We can translate from $\name$ to the linear fragment of L/nL.
  \begin{align}
    (\Gamma \vdash_{\name} A) \to (\Gamma^\circ \vdash_\lnl \lin A^\circ)
  \end{align}
\end{theorem}

\begin{theorem}
  We can translate any L/nL term to a $\name$ term as follows.
  \begin{align}
    (\Gamma \vdash_\lnl \intu X) &\to (\Gamma^* \vdash_{\name} X^*) \\
    (\Gamma \vdash_\lnl \lin A) &\to (\Gamma^* \vdash_{\name} A^*)
  \end{align}
\end{theorem}
