\chapter{Usage restriction via semirings}\label{sec:semirings}

The methods described in \cref{sec:simple} for the simply typed
$\lambda$-calculus
make crucial use of \emph{weakening} --- the fact that if we have
$\Gamma \vdash A$, then we also have $\Gamma, \Delta \vdash A$.
We use this property to update environments as we take them under binders.
However, as we saw in \cref{sec:linearity}, there are interesting calculi in
which general weakening does not hold.
As such, one of the aims of this chapter will be to find a form of weakening
applicable to variables of any type, while essentially retaining linearity
(as opposed to affineness).

This chapter proceeds as follows.
In \cref{sec:semiring-motivation}, I give an intuitive introduction to semiring
annotations on variables, based on replicating features of DILL (introduced in
\cref{sec:dill}).
Then, \cref{sec:lr} formalises the ideas of \cref{sec:semiring-motivation} into
a calculus $\name$.
This calculus has
appeared in my previous work~\citep{WA21}, and can be seen as a
simply typed version of Atkey's dependently typed calculus QTT~\citep{Atkey18}.
%The idea of \name{} is to augment the simply typed $\lambda$-calculus with
%annotations on free variables, and give enough types to manipulate these
%annotations.
Given this new calculus $\name$, the first goal is to apply the techniques of
\cref{sec:simple} to it, yielding a simultaneous substitution operation.
To do this, I use \cref{sec:lnd} to introduce notation that allows us to restate
the typing rules of \name{} to not mention contexts explicitly, as was the style
in \cref{sec:simple}.
This new notation --- the \emph{bunched connectives} --- is versatile at
defining simply typed usage-aware syntaxes, and I give further non-$\name$
examples in \cref{sec:variant}.
Finally, I justify connections to linear logic and modal logic in
\cref{sec:translation}, where I translate $\name$ terms to and from
DILL~\citep{Barber1996} and the modal calculus of \citet{judgmental}.

This chapter and the following chapter re-present and expand the work of
\citet{WA21}.
For the thesis version, I have dropped mention of \emph{skew} semirings, which
allows the algebraic components to be more robust and better abstracted.
In particular, in \cref{sec:ren-sub-lr}, I talk about linear maps rather than
matrices, and define environments in terms of usage-checked variables rather
than raw well typed variables.

\section{Motivation for semiring annotations}\label{sec:semiring-motivation}

The question of defining calculi which do not semantically admit weakening and
contraction but also do not rely on variables going out of scope is directly
addressed by \citet{McBride16}.
The first technique suggested is to, instead of
removing variables from the context of certain subterms, add an annotation to
free variables saying whether or not they are to be used.
I use an annotation $\gr0$ on variables that are not to be used, and an
annotation $\gr1$ on variables that are to be used.
This convention lets us transcribe the usual $\otimes$-introduction rule
(below left) as a rule with usage annotations (below right).
In the notation on the right, I let $\Gamma = \grP\gamma$ and
$\Delta = \grQ\delta$, where $\Gamma$ is a whole context comprising a
\emph{usage context} $\grP$ and a \emph{typing context} $\gamma$.
A usage context is a list of usage annotations, so
$\grP = \gr{r_1}, \ldots, \gr{r_m}$ and a typing context is a list of types, so
$\gamma = A_1, \ldots, A_m$.
When combined, the usage context and the typing context will be of the same
length.
Explicit contexts will usually be written with usage annotations and types
interspersed, as $\gr{r_1}A_1, \ldots, \gr{r_m}A_m$.
I use $\gr r\gamma$ to abbreviate $\gr rx_1, \ldots, \gr rx_m$.

\[
  \ebrule{%
    \hypo{\Gamma \vdash A}
    \hypo{\Delta \vdash B}
    \infer2{\Gamma, \Delta \vdash A \otimes B}
  }
  \quad\rightsquigarrow\quad
  \ebrule{%
    \hypo{\gr1\gamma, \gr0\delta \vdash A}
    \hypo{\gr0\gamma, \gr1\delta \vdash B}
    \infer2{\gr1\gamma, \gr1\delta \vdash A \otimes B}
  }
\]

The eventual target of all these $\gr0$-annotated variables is the variable
rule, which I transcribe as follows.
The $\gr1$ shows us that we can use the variable thus annotated, while the
$\gr0$s let us discard all of the other variables in $\gamma$.

\[
  \ebrule{%
    \infer0{A \vdash A}
  }
  \quad\rightsquigarrow\quad
  \ebrule{%
    \infer0{\gr0\gamma, \gr1A \vdash A}
  }
\]

The use of $\gr0$ gives us the property that variables never go out of
scope in subterms; rather, we lose the ability to use certain variables, but
retain the ability to refer to them metatheoretically.
Additionally, we recover a form of weakening: if $\Gamma \vdash A$, then also
$\Gamma, \gr0\delta \vdash A$, because the resulting term indeed uses no
variables from $\delta$.
I prove the admissibility of weakening for terms will come in \cref{sec:lrsub}.

If we follow the DILL style of variable management explained in \cref{sec:dill},
there are not just the two
states \emph{to be used} ($\gr1$) and \emph{not to be used} ($\gr0$), but also
\emph{usable unrestrictedly}.
If we assign unrestricted (or \emph{intuitionistic}) variables an annotation
$\gr\omega$, we can make the following transcription of the DILL
$\otimes$-introduction rule.

\[
  \ebrule{%
    \hypo{\Theta; \Gamma \vdash A}
    \hypo{\Theta; \Delta \vdash B}
    \infer2{\Theta; \Gamma, \Delta \vdash A \otimes B}
  }
  \quad\rightsquigarrow\quad
  \ebrule{%
    \hypo{\gr\omega\theta, \gr1\gamma, \gr0\delta \vdash A}
    \hypo{\gr\omega\theta, \gr0\gamma, \gr1\delta \vdash B}
    \infer2{\gr\omega\theta, \gr1\gamma, \gr1\delta \vdash A \otimes B}
  }
\]

To conceptualise the criteria on the usage annotations involved in this rule,
I introduce an additive structure over usage annotations.
The rule stated above relies on the facts that $\gr1 + \gr0 = \gr1$,
$\gr0 + \gr1 = \gr1$, and $\gr\omega + \gr\omega = \gr\omega$.
Addition lifts pointwise to vectors of usage annotations (the green capital
calligraphic $\grP$, $\grQ$, and $\grR$).
A beneficial side-effect of the fact that $\gr0 + \gr0 = \gr0$ is that the
rule on the right below is actually more general, and accepts $\gr0$-annotated
variables in its
conclusion, which is essential for weakening to be admissible.

\[
  \ebrule{%
    \hypo{\gr\omega\theta, \gr1\gamma, \gr0\delta \vdash A}
    \hypo{\gr\omega\theta, \gr0\gamma, \gr1\delta \vdash B}
    \infer2{\gr\omega\theta, \gr1\gamma, \gr1\delta \vdash A \otimes B}
  }
  \quad\rightsquigarrow\quad
  \ebrule{%
    \hypo{\grR = \grP + \grQ}
    \hypo{\grP\gamma \vdash A}
    \hypo{\grQ\gamma \vdash B}
    \infer3{\grR\gamma \vdash A \otimes B}
  }
\]

Some other transcriptions from DILL to the usage annotation style are as
follows.
I unify the variable rules (the one for linear variables and the one for
intuitionistic variables) by introducing a coercibility ordering $\leq$ on usage
annotations.
We have $\gr\omega \leq \gr1$ because an intuitionistic variable can fill the
demand of a linear variable by dereliction.
We also have $\gr\omega \leq \gr0$, because intuitionistic variables can be
weakened away like $\gr0$-annotated variables.
This ordering information is shown in the diagram
\begin{tikzpicture}[baseline]
  \node(omega) at (0,0) {$\gr\omega$};
  \node(0) [above left of=omega] {$\gr0$};
  \node(1) [above right of=omega] {$\gr1$};

  \draw (omega) -- (0);
  \draw (omega) -- (1);
\end{tikzpicture}.
All together, this means that at the (only) variable rule, the variable being
used must have annotation less than or equal to $\gr1$, and every other variable
must have annotation less than or equal to $\gr0$.
I write this requirement as $\grR \leq \langle x \rvert$, where
$\langle x \rvert$ is the \emph{basis vector} at position $x$.

\[
  \ebrule{%
    \infer0{\Theta; A \vdash A}
  }
  \quad\rightsquigarrow\quad
  \ebrule{%
    \infer0{\gr\omega\theta, \gr1A, \gr0\delta \vdash A}
  }
  \quad\rightsquigarrow\quad
  \ebrule{%
    \hypo{\grR \leq \bra x}
    \hypo{\gamma_x = A}
    \infer2{\grR\gamma \vdash A}
  }
\]

\[
  \ebrule{%
    \infer0{\Theta, A; {\cdot} \vdash A}
  }
  \quad\rightsquigarrow\quad
  \ebrule{%
    \infer0{\gr\omega\theta, \gr\omega A, \gr0\delta \vdash A}
  }
  \quad\rightsquigarrow\quad
  \ebrule{%
    \hypo{\grR \leq \bra x}
    \hypo{\gamma_x = A}
    \infer2{\grR\gamma \vdash A}
  }
\]

The final interesting rule form to cover is found in DILL's
$\oc$-introduction rule.
DILL's $\oc$-introduction can be though of as an $\gr\omega$-ary counterpart to
$\otimes$-introduction, though with the same premise each time rather than
$\gr\omega$-many premises.
This explains why only $\gr\omega$- and
$\gr0$-annotated variables can appear in the conclusion of $\oc$-introduction,
and also justifies the choice below of multiplication (vector scaling) as the
algebraic operation controlling the $\oc$-modality.

\[
  \ebrule{%
    \hypo{\Theta; {\cdot} \vdash A}
    \infer1{\Theta; {\cdot} \vdash \oc A}
  }
  \quad\rightsquigarrow\quad
  \ebrule{%
    \hypo{\gr\omega\theta, \gr0\delta \vdash A}
    \infer1{\gr\omega\theta, \gr0\delta \vdash \oc A}
  }
  \quad\rightsquigarrow\quad
  \ebrule{%
    \hypo{\grR \leq \gr\omega\grP}
    \hypo{\grP\gamma \vdash A}
    \infer2{\grR\gamma \vdash \oc_{\gr\omega} A}
  }
\]

In summary, the structure we have required of the set of usage annotations is
that they have addition (for $\otimes$-introduction and similar rules),
multiplication (for $\oc$-introduction), a $1$ (for a variable being used), a
$0$ (for a variable not being used), and an ordering (allowing for subsumption
of usage restrictions).
Together, these form a \emph{partially ordered semiring} (posemiring), the laws
of which are both supported by examples and necessary for the syntax to be well
behaved.

\begin{definition}
  A \emph{semiring} is a monoid in the multicategory of commutative monoids and
  multilinear maps.
  Unpacked, this means that we have a set $\mathscr R$ together with elements
  $0$ and $1$ and binary operators $+$ and $\cdot$ (with $\cdot$ usually written
  as juxtaposition) such that the following hold for all
  $x, y, z \in \mathscr R$.
  \begin{itemize}
    \item $0 + x = x$; $x + 0 = x$; $(x + y) + z = x + (y + z)$; $x + y = y + x$
    \item $1x = x$; $x1 = x$; $(xy)z = x(yz)$
    \item $0x = 0$; $x0 = 0$; $(x + y)z = xz + yz$; $x(y + z) = xy + xz$
  \end{itemize}
\end{definition}
\begin{definition}
  A \emph{posemiring} is a semiring in the category of partially ordered sets
  (posets).
  Unpacked, this means that we have a semiring (in the category of sets)
  $(\mathscr R, 0, +, 1, \cdot)$ such that $\mathscr R$ has a partial order
  (written $\leq$) and addition and multiplication are monotonic with respect
  to $\leq$ in both arguments.
\end{definition}

For concreteness, I collect together the definition of the
$\{\gr0, \gr1, \gr\omega\}$ posemiring I have been using so far.

\begin{example}\label{def:lin-semiring}
  The \emph{$\{\gr0, \gr1, \gr\omega\}$ posemiring}, also known as the
  \emph{linearity posemiring}, has the operations given as follows, with
  $0 \coloneqq \gr0$ and $1 \coloneqq \gr1$:

  \makebox[\textwidth][s]{
    \begin{tabular}{c|ccc}
      $+$ & $\gr0$ & $\gr1$ & $\gr\omega$ \\ \hline
      $\gr0$ & $\gr0$ & $\gr1$ & $\gr\omega$ \\
      $\gr1$ & $\gr1$ & $\gr\omega$ & $\gr\omega$ \\
      $\gr\omega$ & $\gr\omega$ & $\gr\omega$ & $\gr\omega$ \\
    \end{tabular}
    \begin{tabular}{c|ccc}
      $*$ & $\gr0$ & $\gr1$ & $\gr\omega$ \\ \hline
      $\gr0$ & $\gr0$ & $\gr0$ & $\gr0$ \\
      $\gr1$ & $\gr0$ & $\gr1$ & $\gr\omega$ \\
      $\gr\omega$ & $\gr0$ & $\gr\omega$ & $\gr\omega$ \\
    \end{tabular}
    \begin{tikzpicture}[baseline]
      \node(omega) at (0,0) {$\gr\omega$};
      \node(0) [above left of=omega] {$\gr0$};
      \node(1) [above right of=omega] {$\gr1$};

      \draw (omega) -- (0);
      \draw (omega) -- (1);
    \end{tikzpicture}
  }
\end{example}

%In \cref{sec:simple}, we saw that the logical rules of simply typed
%$\lambda$-calculus can be described in terms of three basic premise combinators:
%$\dot1$, standing for no premises; $\dottimes$, allowing for multiple premises
%in the same context; and $\Theta \vdash A$, requiring a subterm of type $A$
%having bound the extra variables from context $\Theta$.
%However, we remember from \cref{sec:linearity} that in a substructural setting,
%we do not always want to copy assumptions for use in all subterms.
%This motivates me to introduce the additional premise combinators $I^*$, $\sep$,
%and $\cdot$ in \cref{sec:lnd}, allowing for the modes of usage exhibited in
%the introduction rules for $I$, $\otimes$, and $\oc$, respectively.

\section{A usage-annotated calculus $\name$}\label{sec:lr}
In this section, I introduce the syntax of the type theory \name{}, which makes
use of posemiring usage annotations to express the usage restrictions found in
DILL and other calculi.
I use this syntax to write some example programs, which will motivate the
denotational semantics explored in \cref{sec:wrel}.
For the rest of this thesis, particularly
\cref{sec:semirings,sec:ren-sub-lr,sec:wrel},
\name{} will serve as both a prototypical
usage-constrained syntax and a target of semantic analyses.

The calculus \name{} is similar in spirit to intuitionistic linear logic (ILL),
which we saw in \cref{sec:linearity}.
The types of \name{}, listed in \cref{fig:lr-types}, are almost identical
to those of ILL, differing only in the exponential modality $\oc$
(read ``bang'').
In particular, I include distinguished tensor- and with-product types
($\otimes$, $\with$) and their units ($I$, $\top$), function types
($\multimap$), additive sum types and their unit ($\oplus$, $0$), and the
graded modality $\oc_{\gr r}$.
The idea of $\oc_{\gr r}$ is to internalise an annotation of $\gr r$ on a
variable in the context.

\begin{figure}
  \begin{displaymath}
    A, B, C \Coloneqq I \mid A \otimes B \mid A \multimap B \mid \top
    \mid A \with B \mid 0 \mid A \oplus B \mid \oc_{\gr r} A
  \end{displaymath}
  \caption{The types of \name{}}
  \label{fig:lr-types}
\end{figure}

\begin{figure}
  \begin{displaymath}
    \begin{prooftree}
      \hypo{\gamma \ni x : A}
      \hypo{\grP \le \langle x \rvert}
      \infer2[Var]{\grP\gamma \vdash A}
    \end{prooftree}
  \end{displaymath}

  \begin{displaymath}
    \begin{matrix}
      \begin{prooftree}
        \hypo{\grP \le \gr0}
        \infer1[$I$-I]{\grP\gamma \vdash I}
      \end{prooftree}
      &&
      \begin{prooftree}
        \hypo{\grR \le \grP + \grQ}
        \hypo{\grP\gamma \vdash I}
        \hypo{\grQ\gamma \vdash C}
        \infer3[$I$-E]{\grR\gamma \vdash C}
      \end{prooftree}
      \\\\
      \begin{prooftree}
        \hypo{\grR \le \grP + \grQ}
        \hypo{%
          \begin{matrix}
            \grP\gamma \vdash A \\ \grQ\gamma \vdash B
          \end{matrix}%
        }
        \infer2[$\otimes$-I]{\grR\gamma \vdash A \otimes B}
      \end{prooftree}
      &&
      \begin{prooftree}
        \hypo{%
          \begin{matrix}
            \grR \le \grP + \grQ \\ \grP\gamma \vdash A \otimes B
          \end{matrix}%
        }
        \hypo{\grQ\gamma, \gr1A, \gr1B \vdash C}
        \infer2[$\otimes$-E]{\grR\gamma \vdash C}
      \end{prooftree}
      \\\\
      \begin{prooftree}
        \hypo{\grR\gamma, \gr1A \vdash B}
        \infer1[$\multimap$-I]{\grR\gamma \vdash A \multimap B}
      \end{prooftree}
      &&
      \begin{prooftree}
        \hypo{\grR \le \grP + \grQ}
        \hypo{\grP\gamma \vdash A \multimap B}
        \hypo{\grQ\gamma \vdash A}
        \infer3[$\multimap$-E]{\grR\gamma \vdash B}
      \end{prooftree}
      \\\\
      \begin{prooftree}
        \infer0[$\top$-I]{\grR\gamma \vdash \top}
      \end{prooftree}
      &&
      \textrm{(no $\top$-E)}
      \\\\
      \begin{prooftree}
        \hypo{\grR\gamma \vdash A}
        \hypo{\grR\gamma \vdash B}
        \infer2[$\with$-I]{\grR\gamma \vdash A \with B}
      \end{prooftree}
      &&
      \begin{prooftree}
        \hypo{\grR\gamma \vdash A_0 \with A_1}
        \infer1[$\with$-E$_i$, for $i \in \{0,1\}$]{\grR\gamma \vdash A_i}
      \end{prooftree}
      \\\\
      \textrm{(no $0$-I)}
      &&
      \begin{prooftree}
        \hypo{\grR \le \grP + \grQ}
        \hypo{\grP\gamma \vdash 0}
        \infer2[$0$-E]{\grR\gamma \vdash C}
      \end{prooftree}
      \\\\
      \begin{prooftree}
        \hypo{\grR\gamma \vdash A_i}
        \infer1[$\oplus$-I$_i$, for $i \in \{0,1\}$]%
        {\grR\gamma \vdash A_0 \oplus A_1}
      \end{prooftree}
      &&
      \begin{prooftree}
        \hypo{%
          \begin{matrix}
            \grR \le \grP + \grQ \\ \grP\gamma \vdash A \oplus B
          \end{matrix}%
        }
        \hypo{%
          \begin{matrix}
            \grQ\gamma, \gr1A \vdash C \\ \grQ\gamma, \gr1B \vdash C
          \end{matrix}%
        }
        \infer2[$\oplus$-E]{\grR\gamma \vdash C}
      \end{prooftree}
      \\\\
      \begin{prooftree}
        \hypo{\grR \le \gr r\grP}
        \hypo{\grP\gamma \vdash A}
        \infer2[$\oc$-I]{\grR\gamma \vdash \oc\gr rA}
      \end{prooftree}
      &&
      \begin{prooftree}
        \hypo{\grR \le \grP + \grQ}
        \hypo{\grP\gamma \vdash \oc\gr rA}
        \hypo{\grQ\gamma, \gr rA \vdash C}
        \infer3[$\oc$-E]{\grR\gamma \vdash C}
      \end{prooftree}
    \end{matrix}
  \end{displaymath}
  \caption{\name{}}
  \label{fig:lr}
\end{figure}

I will not cover any operational semantics or equational theory of \name{} in
this thesis.
I will discuss a denotational semantics in \cref{sec:wrel}.

The following features are of note.

\paragraph{Subusaging}
Several typing rules contain constraints of the form $\grP \leq \grQ$, for
certain usage vectors $\grP$ and $\grQ$.
We saw subusaging in the introduction to this chapter in the specific case of
$\Ann$ being formed from the poset $\{\gr0 > \gr\omega < \gr1\}$.
This allowed variables annotated $\gr\omega$ (``unrestricted'') to be both
weakened/discarded (because $\gr\omega \leq \gr0$) and derelicted/used
(because $\gr\omega \leq \gr1$).
Subsumption of usage annotations is essential to nearly all interesting choices
of $\Ann$.
However, in the toy example of exact usage counting using the set $\mathbb N$ of
annotations, we set the order to be just equality as a matter of simplicity.

For usage annotations $\gr r$ and $\gr s$, the inequality $\gr r \leq \gr s$
states that an assumption with annotation $\gr r$ can be used wherever an
assumption with annotation $\gr s$ is required.
A mnemonic is that $\gr r$ is less specific than $\gr s$.
The principle is reflected by the admissible subusaging rule, where the order
has been lifted from annotations to usage contexts.
The subusaging rule is a simple corollary of renaming, as given in
\cref{sec:ren-sub-lr}.

\[
  \begin{prooftree}
    \hypo{\grP \leq \grQ}
    \hypo{\grQ\gamma \vdash A}
    \infer2[Subuse]{\grP\gamma \vdash A}
  \end{prooftree}
\]

\paragraph{Tensor- and with-products}
Like intuitionistic linear logic (ILL), \name{} distinguishes tensor-products
($A \otimes B$) from with-products ($A \with B$).
Whereas in ILL, rules like $\otimes$-introduction involve splitting the
assumptions between the two subterms, in \name{}, this splitting is done by
choosing usage annotations for the premises which add up to the usage
annotations of the conclusion.
For example, we can derive $\vdash A \otimes B \multimap B \otimes A$ as
follows.
Notice that the assumption $A \otimes B$ is still present in all subderivations,
even after it has been ``used up''.
The only thing that stops us using the assumption again is that, for a general
choice of $\Ann$, we do not have $\gr0 \leq \gr1$ or $\gr1 \leq \gr1 + \gr1$.

\begin{small}
  \[
    \nabla \coloneqq
    \begin{prooftree}
      \infer0{\plr{\gr0\;\gr1\;\gr1} \leq
        \plr{\gr0\;\gr0\;\gr1} + \plr{\gr0\;\gr1\;\gr0}}
      \infer0{\plr{\gr0\;\gr0\;\gr1} \leq \plr{\gr0\;\gr0\;\gr1}}
      \infer1[Var]{\gr0\plr{A \otimes B}, \gr0A, \gr1B \vdash B}
      \infer0{\plr{\gr0\;\gr1\;\gr0} \leq \plr{\gr0\;\gr1\;\gr0}}
      \infer1[Var]{\gr0\plr{A \otimes B}, \gr1A, \gr0B \vdash A}
      \infer3[$\otimes$-I]%
      {\gr0\plr{A \otimes B}, \gr1A, \gr1B \vdash B \otimes A}
    \end{prooftree}
  \]

  \[
    \begin{prooftree}
      \infer0{\plr{\gr1} \leq \plr{\gr1} + \plr{\gr0}}
      \infer0{\plr{\gr1} \leq \plr{\gr1}}
      \infer1[Var]{\gr1\plr{A \otimes B} \vdash A \otimes B}
      \hypo{\nabla}
      \infer[no rule]1{\gr0\plr{A \otimes B}, \gr1A, \gr1B \vdash B \otimes A}
      \infer3[$\otimes$-E]{\gr1\plr{A \otimes B} \vdash B \otimes A}
      \infer1[$\multimap$-I]{\vdash A \otimes B \multimap B \otimes A}
    \end{prooftree}
  \]
\end{small}

\begin{example}
  Let $A \multimapboth B$ abbreviate
  $\plr{A \multimap B} \with \plr{B \multimap A}$.
  Then the following judgements hold for any partially ordered semiring.
  Derivations are left as an exercise to the reader.
  \begin{itemize}
    \item $\vdash A \oplus A \multimap A$
    \item $\vdash A \multimap A \with A$
    \item $\vdash A \oplus 0 \multimapboth A$
    \item $\vdash A \otimes 0 \multimapboth 0$
    \item $\vdash \oc\gr1A \multimapboth A$
    \item If $\gr r \leq \gr s$, then $\vdash \oc\gr rA \multimap \oc\gr sA$
  \end{itemize}
\end{example}

\begin{example}
  Let $\Ann \coloneqq (\mathbb N, =, 0, +, 1, \times)$, that is, specialise to
  the posemiring made of
  the usual semiring of natural numbers with ordering given by equality.
  Under this discipline, the usage constraints enforce a form of exact usage
  counting.
  The following judgements then hold.
  Derivations are left as an exercise to the reader.
  \begin{itemize}
    \item $\vdash \oc\gr2A \multimap A \otimes A$
    \item $\vdash \oc\gr5A \multimap \oc\gr2A \otimes \oc\gr3A$
  \end{itemize}
\end{example}

\subsection{Other posemirings}\label{sec:example-posemirings}

Now that we have seen the role of usage annotations in $\name$, I will give more
examples of posemirings for tracking interesting usage patterns.

\begin{example}\label{def:trivial-posemiring}
  The singleton set gives rise to a posemiring in a unique way.
  When the usage annotations of $\name$ are taken from this trivial posemiring,
  we recover a version of intuitionistic simply typed $\lambda$-calculus
  featuring redundant connectives $\otimes$ (equivalent to $\with$ in the pure
  setting) and $\oc{\gr*}$ (where $\oc{\gr*} A \simeq A$).
\end{example}

\begin{example}\label{def:monotonicity-posemiring}
  The \emph{monotonicity} posemiring is defined over the set of symbols
  $\{\gr{\wn\wn}, \gr{\uparrow\uparrow}, \gr{\downarrow\downarrow},
  \gr{\sim\sim}\}$.
  The idea is that each symbol represents the possible \emph{variance} of an
  input (free variable) with respect to some partial ordering on a semantic
  domain of elements.
  $\gr{\uparrow\uparrow}$ represents covariance (if that input goes up, the
  output goes up), $\gr{\downarrow\downarrow}$ represents contravariance
  (if that input goes \emph{down}, the output goes up), $\gr{\sim\sim}$ gives no
  guarantees (if that input remains constant, the output (trivially) goes up),
  and $\gr{\wn\wn}$ says that that input is irrelevant (whatever changes are
  made to that input, the output (trivially) goes up).

  I take $0 \coloneqq \gr{\wn\wn}$, $1 \coloneqq \gr{\uparrow\uparrow}$,
  and define the following operations:

  \makebox[\textwidth][s]{
    \begin{tabular}{c|cccc}
      $+$ & $\gr{\wn\wn}$ & $\gr{\uparrow\uparrow}$ & $\gr{\downarrow\downarrow}$ & $\gr{\sim\sim}$ \\ \hline
      $\gr{\wn\wn}$ & $\gr{\wn\wn}$ & $\gr{\uparrow\uparrow}$ & $\gr{\downarrow\downarrow}$  & $\gr{\sim\sim}$ \\
      $\gr{\uparrow\uparrow}$ & $\gr{\uparrow\uparrow}$ & $\gr{\uparrow\uparrow}$ & $\gr{\sim\sim}$ & $\gr{\sim\sim}$ \\
      $\gr{\downarrow\downarrow}$ & $\gr{\downarrow\downarrow}$ & $\gr{\sim\sim}$ & $\gr{\downarrow\downarrow}$ & $\gr{\sim\sim}$ \\
      $\gr{\sim\sim}$ & $\gr{\sim\sim}$  & $\gr{\sim\sim}$ & $\gr{\sim\sim}$ & $\gr{\sim\sim}$ \\
    \end{tabular}
    \begin{tabular}{c|cccc}
      $*$ & $\gr{\wn\wn}$ & $\gr{\uparrow\uparrow}$ & $\gr{\downarrow\downarrow}$ & $\gr{\sim\sim}$ \\ \hline
      $\gr{\wn\wn}$ & $\gr{\wn\wn}$ & $\gr{\wn\wn}$ & $\gr{\wn\wn}$  & $\gr{\wn\wn}$ \\
      $\gr{\uparrow\uparrow}$ & $\gr{\wn\wn}$ & $\gr{\uparrow\uparrow}$ & $\gr{\downarrow\downarrow}$ & $\gr{\sim\sim}$ \\
      $\gr{\downarrow\downarrow}$ & $\gr{\wn\wn}$ & $\gr{\downarrow\downarrow}$ & $\gr{\uparrow\uparrow}$ & $\gr{\sim\sim}$ \\
      $\gr{\sim\sim}$ & $\gr{\wn\wn}$  & $\gr{\sim\sim}$ & $\gr{\sim\sim}$ & $\gr{\sim\sim}$ \\
    \end{tabular}
    \begin{tikzpicture}[baseline]
      \node(omega) at (0,-1) {$\gr{\sim\sim}$};
      \node(0) [above left of=omega] {$\gr{\uparrow\uparrow}$};
      \node(1) [above right of=omega] {$\gr{\downarrow\downarrow}$};
      \node(qq) [above right of=0] {$\gr{\wn\wn}$};

      \draw (omega) -- (0);
      \draw (omega) -- (1);
      \draw (0) -- (qq);
      \draw (1) -- (qq);
    \end{tikzpicture}
  }

  Addition represents an intersection of guarantees.
  For example, if a variable is used covariantly in one subterm and
  contravariantly in another, we can only make the trivial guaratee represented
  by $\gr{\sim\sim}$.
  Multiplication is mainly interesting for multiplication by
  $\gr{\downarrow\downarrow}$, which flips the variance on any other annotation.
  As such, $\oc\gr{\downarrow\downarrow}A$ represents a contravariant $A$.
  The flipping (involutive) behaviour of $\gr{\downarrow\downarrow}$ lets us
  notice that $x$ is covariant in
  a term like $-(-x)$, where $-$ is a constant of type
  $\oc\gr{\downarrow\downarrow}\mathbb Z \multimap \mathbb Z$.

  A similar, but distinct, collection of modalities for monotonicity is given by
  \citet{Arntzenius19}.
\end{example}

\begin{example}\label{def:sensitivity-posemiring}
  The \emph{sensitivity} posemiring~\citep{reed10distance} is given by
  $(\mathbb R^+, \geq, \gr0, +, \gr1, \times)$, where $\mathbb R^+$ is the
  non-negative real numbers extended with $\gr\infty$ (distances), and the rest
  of the structure comes from the standard operations on real numbers (except
  that $\gr0 \times \gr\infty = \gr\infty \times \gr0 = \gr0$).
  Note that the order is reversed, making $\gr\infty$ coercible to any other
  annotation and anything coercible to $\gr0$.

  This posemiring can be used for sensitivity analysis, where we want to bound
  the effect of a perturbation on inputs in terms of some semantic notion of
  distance between values.
  An annotation $\gr r$ says that if that input is perturbed by at most $r$,
  then the output will change by at most $1$.
  An $\gr\infty$-annotated variable gives the very strong guarantee that any
  change in the input will make a minimal change to the output, while a
  $\gr0$-annotated variable provides no guarantee at all.
  Addition forbids general contraction, which would otherwise allow arbitrary
  finite blow-up of the effect of any non-$\gr0$-annotated variable.
  However, the ordering, with $\gr0$ at the top, means that we have general
  weakening, so the resulting sensitivity calculus has an affine flavour.
\end{example}

\section{Bunched connectives}\label{sec:lnd}
The typing rules of $\name$ presented in \cref{fig:lr} contain a lot of detail
and repeated patterns.
For example, nearly half of the rules include the premise
$\grR \leq \grP + \grQ$.
Also, the presence of usage annotations, which are often different in different
parts of a rule, means that we keep repeating the context.
Explicit contexts go against the style we established in \cref{sec:simple},
which is based
around being parametric in the context, so that substitution is agnostic to the
details of typing rules.

To encapsulate the repeated patterns and facilitate an implicit context style,
I introduce the \emph{bunched connectives} for premises.
These are inspired by bunched logic~\citep{oHP99}, and will not only be used for
stating the syntax, but will be used as an abstraction of common patterns in the
development of the metatheory.
The idea is to generalise the space between premises from Gentzen's natural
deduction to allow for any linear combination of usage annotations.
Among other things, this generalisation will allow us to distinguish between
$\with$-introduction and $\otimes$-introduction by a choice of connective:
either \emph{sharing} or \emph{separating} conjunction.
%Similar connectives, but with different interpretations, could be used to
%define other linear-like type theories, like DILL, but here I will focus on the
%usage annotation style.
These connectives are defined in \cref{fig:bunched} in Agda notation.

\begin{figure}
  %\begin{align*}
  %  \dot1\,\grR &\coloneqq 1 \\
  %  (T \dottimes U)\,\grR &\coloneqq T\,\grR \times U\,\grR \\
  %  (T \dotto U)\,\grR &\coloneqq T\,\grR \to U\,\grR \\
  %  I^*\,\grR &\coloneqq \grR \leq \gr0 \\
  %  (T \sep U)\,\grR &\coloneqq \Sigma \grP,\grQ.~\plr{\grR \leq \grP + \grQ}
  %                     \times T\,\grP \times U\,\grQ \\
  %  (\gr r \cdot T)\,\grR &\coloneqq \Sigma \grP.~\plr{\grR \leq \gr r\grP}
  %                     \times T\,\grP
  %\end{align*}
  Parameters:
  \ExecuteMetaData[\Bunchedtex]{BunchedParams}

  Connectives:
  \vspace{-1em}
  \begin{multicols}{2}
    \noindent\ExecuteMetaData[\Bunchedtex]{PointwiseUnit}
    \noindent\ExecuteMetaData[\Bunchedtex]{PointwiseConjunction}
    \columnbreak
    %\AgdaNoSpaceAroundCode{}
    \noindent\ExecuteMetaData[\Bunchedtex]{Entails}
    %\AgdaSpaceAroundCode{}
  \end{multicols}
  \vspace{-2em}
  \ExecuteMetaData[\Bunchedtex]{BunchedUnit}
  \ExecuteMetaData[\Bunchedtex]{BunchedConjunction}
  \ExecuteMetaData[\Bunchedtex]{BunchedImplication}
  \ExecuteMetaData[\Bunchedtex]{BunchedScaling}
  \caption{The bunched connectives}
  \label{fig:bunched}
\end{figure}

The bunched connectives are parametrised over two sets and three relations.
For syntax, the set \AgdaBound{A} will be \AgdaRecord{Ctx}, the type of
contexts, and \AgdaBound{R} will be \AgdaField{Ann}, the type of usage
annotations (scalars).
For the relations, the notation is meant to be suggestive, with
\AgdaBound{$\Gamma$}\AgdaSpace{}\AgdaBound{$\leq$0} typically stating that all
of the annotations in \AgdaBound{$\Gamma$} are less than or equal to $0$;
\AgdaBound{$\Gamma$}\AgdaSpace{}\AgdaBound{$\leq$[}\AgdaSpace{}%
\AgdaBound{$\Delta$}\AgdaSpace{}\AgdaBound{+}\AgdaSpace{}\AgdaBound{$\Theta$}%
\AgdaSpace{}\AgdaBound{]}
typically stating that \AgdaBound{$\Gamma$}, \AgdaBound{$\Delta$}, and
\AgdaBound{$\Theta$} all agree on their types but the usage context of
\AgdaBound{$\Gamma$} is less than or equal to the sum of the usage contexts of
\AgdaBound{$\Delta$} and \AgdaBound{$\Theta$}; and
\AgdaBound{$\Gamma$}\AgdaSpace{}\AgdaBound{$\leq$[}\AgdaSpace{}%
\AgdaBound{r}\AgdaSpace{}\AgdaBound{*$_l$}\AgdaSpace{}\AgdaBound{$\Delta$}%
\AgdaSpace{}\AgdaBound{]}
typically stating that \AgdaBound{$\Gamma$} and \AgdaBound{$\Delta$} agree on
their types but have the evident scaling relationship with \AgdaBound{r} on
their usage annotations.
I use the same symbols for the connectives both in Agda code and in otherwise
standard mathematical/logical notation.

The first two connectives are those we've already seen for intuitionistic
systems --- $\dot1$ and $\dottimes$.
The absence of premises is encoded by $\dot1$, while the space between premises
sharing a context is encoded by $\dottimes$.
As for implication, I temporarily avoid giving a $\dotto$ connectives, instead
fusing it together with $\forallb{-}$ to produce the \emph{set} of
context-preserving functions $T \rightarrowtriangle U$.
When we interpret a typing rule as a constructor of an inductive definition,
$\rightarrowtriangle$
interprets the horizontal line, reflecting the fact that the usage annotations
we start off with in the premises are those of the conclusion, corresponding to
a general principle of resource conservation.
The prototypical rules that use $\dot1$ and $\dottimes$ are the introduction
rules for $\top$ and $\with$, respectively.

\begin{align*}
  \begin{prooftree}
    \infer0{\grR\Gamma \vdash \top}
  \end{prooftree}
  &\quad\rightsquigarrow\quad
  \begin{prooftree}[comb]
    \hypo{\dot1}
    \infer1{\vdash \top}
  \end{prooftree}
  \\\\
  \begin{prooftree}
    \hypo{\grR\Gamma \vdash A}
    \hypo{\grR\Gamma \vdash B}
    \infer2{\grR\Gamma \vdash A \with B}
  \end{prooftree}
  &\quad\rightsquigarrow\quad
  \begin{prooftree}[comb]
    \hypo{\vdash A}
    \hypo{\dottimes}
    \hypo{\vdash B}
    \infer3{\vdash A \with B}
  \end{prooftree}
\end{align*}

The rest of the bunched connectives --- $I^*$, $\sep$, $\cdot$, and $\wand$ ---
involve linear decompositions of the usage annotations.
The three basic left semimodule operators --- zero, addition, and left-scaling
--- each get a bunched connective --- $I^*$, $\sep$, and $\gr r \cdot {}$,
respectively.
The prototypical typing rules for each of these three connectives are the
introduction rules for $I$, $\otimes$, and $\oc{\gr r}$, respectively.

\begin{align*}
  \begin{prooftree}
    \hypo{\grR \leq \gr0}
    \infer1{\grR\Gamma \vdash I}
  \end{prooftree}
  &\quad\rightsquigarrow\quad
  \begin{prooftree}[comb]
    \hypo{I^*}
    \infer1{\vdash I}
  \end{prooftree}
  \\\\
  \begin{prooftree}
    \hypo{\grP\Gamma \vdash A}
    \hypo{\grQ\Gamma \vdash B}
    \hypo{\grR \leq \grP + \grQ}
    \infer3{\grR\Gamma \vdash A \otimes B}
  \end{prooftree}
  &\quad\rightsquigarrow\quad
  \begin{prooftree}[comb]
    \hypo{\vdash A}
    \hypo{\sep}
    \hypo{\vdash B}
    \infer3{\vdash A \otimes B}
  \end{prooftree}
  \\\\
  \begin{prooftree}
    \hypo{\grP\Gamma \vdash A}
    \hypo{\grR \leq \gr r\grP}
    \infer2{\grR\Gamma \vdash \oc{\gr r} A}
  \end{prooftree}
  &\quad\rightsquigarrow\quad
  \begin{prooftree}[comb]
    \hypo{\gr r \cdot (\vdash A)}
    \infer1{\vdash \oc{\gr r} A}
  \end{prooftree}
\end{align*}

\subsection{$\name$ stated using bunched connectives}\label{sec:lr-bunched}
The full system \name{} is stated in terms of bunched connectives in
\cref{fig:lr-bunched}.
The bunched connectives also yield a reasonably concise definition of the Agda
data type of \name{} derivations, as seen in \cref{fig:lr-bunched-Agda}.

\begin{figure}
  \ebproofset{separation=0.75em}
  \begin{displaymath}
    \begin{prooftree}
      \hypo{\sqni A}
      \infer1[Var]{\vdash A}
    \end{prooftree}
  \end{displaymath}

  \begin{displaymath}
    \begin{matrix}
      \begin{prooftree}
        \hypo{I^*}
        \infer1[$I$-I]{\vdash I}
      \end{prooftree}
      &&
      \begin{prooftree}
        \hypo{\vdash I}
        \hypo{\sep}
        \hypo{\vdash C}
        \infer3[$I$-E]{\vdash C}
      \end{prooftree}
      \\\\
      \begin{prooftree}
        \hypo{\vdash A}
        \hypo{\sep}
        \hypo{\vdash B}
        \infer3[$\otimes$-I]{\vdash A \otimes B}
      \end{prooftree}
      &&
      \begin{prooftree}
        \hypo{\vdash A \otimes B}
        \hypo{\sep}
        \hypo{\gr1A, \gr1B \vdash C}
        \infer3[$\otimes$-E]{\vdash C}
      \end{prooftree}
      \\\\
      \begin{prooftree}
        \hypo{\gr1A \vdash B}
        \infer1[$\multimap$-I]{\vdash A \multimap B}
      \end{prooftree}
      &&
      \begin{prooftree}
        \hypo{\vdash A \multimap B}
        \hypo{\sep}
        \hypo{\vdash A}
        \infer3[$\multimap$-E]{\vdash B}
      \end{prooftree}
      \\\\
      \begin{prooftree}
        \hypo{\dot1}
        \infer1[$\top$-I]{\vdash \top}
      \end{prooftree}
      &&
      \textrm{(no $\top$-E)}
      \\\\
      \begin{prooftree}
        \hypo{\vdash A}
        \hypo{\dottimes}
        \hypo{\vdash B}
        \infer3[$\with$-I]{\vdash A \with B}
      \end{prooftree}
      &&
      \begin{prooftree}
        \hypo{\vdash A_0 \with A_1}
        \infer1[$\with$-E$_i$, for $i \in \{0,1\}$]{\vdash A_i}
      \end{prooftree}
      \\\\
      \textrm{(no $0$-I)}
      &&
      \begin{prooftree}
        \hypo{\vdash 0}
        \hypo{\sep}
        \hypo{\dot1}
        \infer3[$0$-E]{\vdash C}
      \end{prooftree}
      \\\\
      \begin{prooftree}
        \hypo{\vdash A_i}
        \infer1[$\oplus$-I$_i$, for $i \in \{0,1\}$]{\vdash A_0 \oplus A_1}
      \end{prooftree}
      &&
      \begin{prooftree}
        \hypo{\vdash A \oplus B}
        \hypo{\sep}
        \hypo{(\gr1A \vdash C}
        \hypo{\dottimes}
        \hypo{\gr1B \vdash C)}
        \infer5[$\oplus$-E]{\vdash C}
      \end{prooftree}
      \\\\
      \begin{prooftree}
        \hypo{\gr r \cdot (\vdash A)}
        \infer1[$\oc$-I]{\vdash \oc{\gr r} A}
      \end{prooftree}
      &&
      \begin{prooftree}
        \hypo{\vdash \oc{\gr r} A}
        \hypo{\sep}
        \hypo{\gr rA \vdash C}
        \infer3[$\oc$-E]{\vdash C}
      \end{prooftree}
    \end{matrix}
  \end{displaymath}
  \ebproofset{separation=1.5em}
  \caption{\name{} stated using bunched connectives}
  \label{fig:lr-bunched}
\end{figure}

\begin{figure}
  \ExecuteMetaData[\SpecificSyntaxtex]{Ty}
  \ExecuteMetaData[\SpecificSyntaxtex]{Bind}
  \ExecuteMetaData[\SpecificSyntaxtex]{lr}
  \caption{\name{} stated using bunched connectives in Agda}
  \label{fig:lr-bunched-Agda}
\end{figure}

\subsection{Connection with bunched logic}\label{sec:bunched-logic}
While we have seen a connection between the bunched connectives and the
connectives of $\name$, the two should not be confused.
In particular, the bunched connectives obey different laws to what we would
expect from linear logic.
For example, it would make sense to define a bunched connective $\dotplus$,
defined analogously to $\dottimes$.
This $\dotplus$ could be used to rephrase the introduction rules for $\oplus$.
We then have maps both ways between $T \dottimes (U \dotplus V)$ and
$(T \dottimes U) \dotplus (T \dottimes V)$, reminiscent of the distributivity of
additive connectives in bunched logic,
whereas linear logic only has a map from $(A \with B) \oplus (A \with C)$ to
$A \with (B \oplus C)$, and not a map the other way.
Looking at the interpretations, the connection with bunched logic makes sense.
Instead of the partial commutative monoid (often representing heaps) found in
standard semantics of bunched logic, we have a left semimodule of usage
contexts, which we are similarly interested in splitting and sharing between
various subterms.

From bunched logic, we would expect the Cartesian product $\dottimes$ to have an
internal hom.
In the intuitionistic case, $\dotto$ filled this role.
However, with usage contexts, it makes sense for open types to be presheaves
over the partial order of contexts under pointwise $\leq$ of usage annotations.
The family $T \dotto U$ does not satisfy the functoriality condition because of
the contravariance in the domain $T$.
Instead, as found in models of bunched logic, we would want a Kripke function
space, like
$\lambda\Gamma.~\forall \Gamma' \leq \Gamma.~T\,\Gamma' \to U\,\Gamma'$.
However, I do not make use of such a connective.

The separating conjunction $\sep$ can be seen as a decategorified version of Day
convolution~\citep{Day70}.
It also resembles the use of ternary frames in semantics of non-distributive
logics~\citep[chapter 12]{Restall1999}.

\subsection{Operations on bunched connectives}\label{sec:bunched-op}
To manipulate terms and other open types defined using bunched connectives, we
need the zero, addition, and multiplication relations to satisfy some laws.
For example, to achieve a symmetry map $T \sep U \rightarrowtriangle U \sep T$,
we need addition to satisfy the commutativity law
$\forall x,y,z : A.~x \leq y + z \to x \leq z + y$.

For all uses of bunched connectives in this thesis, the carrier set $A$ will
form a partial order --- for example, with contexts, the order is given by the
pointwise order on the usage vectors.
We then consider the category whose objects are posets and whose morphisms are
relations $R : A \rel B$ satisfying the contravariant-covariant law
$\forall x,x',y,y'.~x' \leq x \to y \leq y' \to xRy \to x'Ry'$.
This category can be given the usual monoidal product of relations, which is
the pointwise product of posets on objects.
Then, we will always expect zero and addition to together form a cocommutative
comonoid in this category.
With this structure, we can get the following equivalences and functions.

\begin{mathpar}
  I^* \sep T \leftrightarrowtriangle T \and
  T \sep I^* \leftrightarrowtriangle T \and
  (T \sep U) \sep V \leftrightarrowtriangle T \sep (U \sep V) \and
  T \sep U \leftrightarrowtriangle U \sep T \and
  (T \wand U) \sep T \rightarrowtriangle U \and
  I^* \wand T \leftrightarrowtriangle T \and
  (T \sep U) \wand V \leftrightarrowtriangle T \wand (U \wand V)
\end{mathpar}

I do not use algebraic properties of multiplication in conjunction with
manipulation of bunched connectives in this thesis, but we could expect
scalar multiplication
to add a comodule structure over cosemiring $R$ to the cocommutative comonoid
given by zero and addition.

\section{Additions to and variations of \name{}}\label{sec:variant}
Based on an intuitive understanding of ``usage'', recursion introduces a new
phenomenon relative to the forms of programs we have seen so far:
terms can be used an unbounded number of times.
For example, notice the following reduction in Agda.

\missingfigure{\texttt{foldr \_+\_ 0 (1 :: 2 :: 3 :: []) = 1 + (2 + (3 + 0))}}

The function \AgdaFunction{\_+\_} has been copied into 3 different places in
the running of the program.
This copying is despite no type telling us that \AgdaFunction{\_+\_} would be
used 3 times (both \verb|[1,2,3]| and \verb|[2,3]| have type
\AgdaDatatype{List}\AgdaSpace{}\AgdaDatatype{$\mathbb N$}, despite the
corresponding folds using \AgdaFunction{\_+\_} a different number of times).
As such, when checking an application of \AgdaFunction{foldr}, we need check
that we can use its functional argument (\AgdaFunction{\_+\_} in this case) an
arbitrary number of times.
If we were to fix $\Ann$ as the $\{\gr0, \gr1, \gr\omega\}$ posemiring, then
wrapping the type of the functional argument in $\oc\gr\omega$ would suffice.
However, we want to remain generic in the choice of semiring.

I propose the following additions to \name{} to support a broad class of
inductive types.
I define strictly positive functors syntactically, with the only notable
restriction being not being allowed to use the type variable $X$ in the domain
of a function type.
I then add least fixed points of such strictly positive functors to the syntax
of types.

\begin{align*}
  U &\Coloneqq A \multimap (-) \mid \oc\gr r(-) \\
  {\odot} &\Coloneqq {\otimes} \mid {\oplus} \mid {\with} \\
  F[X], G[X] &\Coloneqq X \mid A \mid U(F[X]) \mid F[X] \odot G[X] \\
  A &\Coloneqq \cdots \mid \mu X.~F[X]
\end{align*}

\begin{example}
  We may define $\mathrm{List}_A \coloneqq \mu X.~I \oplus \plr{A \otimes X}$.
\end{example}

In the typing rules, introduction of an inductive type is standard.
For the elimination rule, we follow a similar pattern to other pattern-matching
rules --- $\oplus$-E, $\otimes$-E, and $\oc$-E --- by splitting the context
and typing the eliminand in one half ($\grP$).
We type the continuation in the other half, but because the continuation may
be used multiple times, and in a modal context, we require that $\grQ$ is
preserved by all linear operations.

\begin{displaymath}
  \begin{prooftree}
    \hypo{\grR\gamma \vdash F[\mu X.~F[X]]}
    \infer1[$\mu$-I]{\grR\gamma \vdash \mu X.~F[X]}
  \end{prooftree}
\end{displaymath}
\begin{displaymath}
  \begin{prooftree}
    \hypo{\grR \leq \grP + \grQ}
    \hypo{\grP\gamma \vdash \mu X.~F[X]}
    \hypo{%
      \begin{matrix*}[l]
        \grQ \leq \gr0 \\
        \grQ \leq \grQ + \grQ \\
        \forall \gr r.~\grQ \leq \gr r\grQ
      \end{matrix*}%
    }
    \hypo{\grQ\gamma, \gr1F[C] \vdash C}
    \infer4[$\mu$-E]{\grR\gamma \vdash C}
  \end{prooftree}
\end{displaymath}

\begin{example}\label{thm:list-rules}
  For lists, we can derive the following introduction and elimination rules
  (with usage constraints omitted for brevity when obvious).

  \begin{align*}
    \begin{prooftree}
      \hypo{\grR \leq \gr0}
      \infer1[$I$-I]{I}
      \infer1[$\oplus$-I$_0$]%
      {\grR\gamma \vdash I \oplus \plr{A \otimes \mathrm{List}_A}}
      \infer1[$\mu$-I]{\grR\gamma \vdash \mathrm{List}_A}
    \end{prooftree}
    &&
    \begin{prooftree}
      \hypo{\grR \leq \grP + \grQ}
      \hypo{\grP\gamma \vdash A}
      \hypo{\grQ\gamma \vdash \mathrm{List}_A}
      \infer3[$\otimes$-I]{\grR\gamma \vdash A \otimes \mathrm{List}_A}
      \infer1[$\oplus$-I$_1$]%
      {\grR\gamma \vdash I \oplus \plr{A \otimes \mathrm{List}_A}}
      \infer1[$\mu$-I]{\grR\gamma \vdash \mathrm{List}_A}
    \end{prooftree}
  \end{align*}
  \begin{displaymath}
    \begin{prooftree}
      \hypo{\grP\gamma \vdash \mathrm{List}_A}
      \infer0[Var]{\gr0\gamma, \gr1\plr{I \oplus \plr{A \otimes C}}
        \vdash I \oplus \plr{A \otimes C}}
      \hypo{\nabla^n}
      \hypo{\nabla^c}
      \infer3[$\oplus$-E]{\grQ\gamma, \gr1\plr{I \oplus \plr{A \otimes C}}
        \vdash C}
      \infer2[$\mu$-E]{\grR\gamma \vdash C}
    \end{prooftree}
  \end{displaymath}
  \begin{align*}
    \textrm{where }\nabla^n &\coloneqq
    \begin{prooftree}
      \infer0[Var]{\gr0\gamma, \gr1I \vdash I}
      \hypo{\grQ\gamma \vdash C}
      \infer1[Wk]{\grQ\gamma, \gr0I \vdash C}
      \infer2[$I$-E]{\grQ\gamma, \gr1I \vdash C}
      \infer1[Wk]{\grQ\gamma, \gr0\plr{I \oplus \plr{A \otimes C}}, \gr1I
        \vdash C}
    \end{prooftree}
    \\\\
    \textrm{and }\nabla^c &\coloneqq
    \begin{prooftree}
      \infer0[Var]{\gr0\gamma, \gr1\plr{A \otimes C} \vdash A \otimes C}
      \hypo{\grQ\gamma, \gr1A, \gr1C \vdash C}
      \infer1[Wk]{\grQ\gamma, \gr0\plr{A \otimes C}, \gr1A, \gr1C \vdash C}
      \infer2[$\otimes$-E]{\grQ\gamma, \gr1\plr{A \otimes C} \vdash C}
      \infer1[Wk]%
      {\grQ\gamma, \gr0\plr{I \oplus \plr{A \otimes C}}, \gr1\plr{A \otimes C}
        \vdash C}
    \end{prooftree}
  \end{align*}
\end{example}

Following \cref{sec:lnd}, I want to turn the ad hoc constraints on $\grP$,
$\grQ$, and $\grR$ into the result of some premise combinators.
To do this, I introduce a new combinator $\Box^{0{+}{\times}}$ defined below,
along with the resulting implicit-context typing rules.

\begin{align*}
  \plr{\Box^{0{+}{\times}}\,T}\grR \coloneqq
  \plr{\grR \leq \gr0} \times \plr{\grR \leq \grR + \grR} \times
  \plr{\forall \gr r.~\grR \leq \gr r\grR} \times T\,\grR
\end{align*}

\begin{align*}
  \begin{prooftree}[comb]
    \hypo{\vdash F[\mu X.~F[X]]}
    \infer1[$\mu$-I]{\vdash \mu X.~F[X]}
  \end{prooftree}
  &&
  \begin{prooftree}[comb]
    \hypo{\vdash \mu X.~F[X]}
    \hypo{\sep}
    \hypo{\Box^{0{+}{\times}}\plr{\gr1F[C] \vdash C}}
    \infer3[$\mu$-E]{\vdash C}
  \end{prooftree}
\end{align*}

\begin{example}
  We can state the rules for lists derived in \cref{thm:list-rules} as follows.
  \begin{align*}
    \begin{prooftree}[comb]
      \hypo{I^*}
      \infer1{\vdash \mathrm{List}_A}
    \end{prooftree}
    &&
    \begin{prooftree}[comb]
      \hypo{\vdash A}
      \hypo{\sep}
      \hypo{\vdash \mathrm{List}_A}
      \infer3{\vdash \mathrm{List}_A}
    \end{prooftree}
    &&
    \begin{prooftree}[comb]
      \hypo{\vdash \mathrm{List}_A}
      \hypo{\sep}
      \hypo{\Box^{0{+}{\times}}
        \plr{\vdash C\hskip0.75em\dottimes\hskip0.75em\gr1A, \gr1C \vdash C}}
      \infer3{\vdash C}
    \end{prooftree}
  \end{align*}
\end{example}

\section{Representing existing linear and modal logics}\label{sec:translation}
\newcommand\instDILL{\gr{01\omega}}
\newcommand\instPD{\gr{01\Box}}

\newcommand\unused{\gr0}
\newcommand\true{\gr1}
\newcommand\valid{\gr\Box}

A motivating reason to consider the system $\name$ is that
instances of it correspond to previously studied systems.
In this section, I present translations from \name{} to Dual Intuitionistic
Linear Logic~\citep{Barber1996} and the modal system of \citet{judgmental},
and vice versa.
These translations are not mechanised, as part of the reason for developing
\name{} was to avoid mechanising these systems directly.
We cannot prove that the translations form an equivalence, because I have not
written down an equational theory for \name{}, but I expect this to be easy
enough to do.

\subsection{Dual Intuitionistic Linear Logic}\label{sec:trans-dill}

Dual Intuitionistic Linear Logic is a particular formulation of intuitionistic
linear logic introduced by \citet{Barber1996}.
Its key feature, which simplifies the metatheory of linear logic, is the use of
separate contexts for linear and intuitionistic free variables.
Here I show that DILL is a fragment of the instantiation of \name{} at the
linearity semiring $\{\gr0, \gr1, \gr\omega\}$.

The types of DILL are the same as the types of \name, except for the
restriction of $\oc\gr{r}$ to just $\oc\gr\omega$.
I will write the latter simply as $\oc$ when it appears in DILL\@.
I add sums and with-products to the calculus of \cite{Barber1996}, with the
obvious rules (stated fully in \cref{fig:dill}).
These additive type formers present no additional difficulty to the translation.

\begin{figure}
  \begin{mathpar}
    \inferrule*[right=Int-Ax]{ }
    {\gamma, A; \cdot \vdash A}
    \and
    \inferrule*[right=Lin-Ax]{ }
    {\gamma; A \vdash A}
    \and
    \inferrule*[right=$I$-I]{ }
    {\gamma; \cdot \vdash I}
    \and
    \inferrule*[right=$I$-E]
    {\gamma; \delta_1 \vdash I \\ \gamma; \delta_2 \vdash A}
    {\gamma; \delta_1, \delta_2 \vdash A}
    \and
    \inferrule*[right=$\otimes$-I]
    {\gamma; \delta_1 \vdash A \\ \gamma, \delta_2 \vdash B}
    {\gamma; \delta_1, \delta_2 \vdash A \otimes B}
    \and
    \inferrule*[right=$\otimes$-E]
    {\gamma; \delta_1 \vdash A \otimes B \\ \gamma; \delta_2, A, B \vdash C}
    {\gamma; \delta_1, \delta_2 \vdash C}
    \and
    \inferrule*[right=$\multimap$-I]
    {\gamma; \delta, A \vdash B}
    {\gamma; \delta \vdash A \multimap B}
    \and
    \inferrule*[right=$\multimap$-E]
    {\gamma; \delta_1 \vdash A \multimap B \\ \gamma; \delta_2 \vdash A}
    {\gamma; \delta_1, \delta_2 \vdash B}
    \and
    \inferrule*[right=$\oc$-I]
    {\gamma; \cdot \vdash A}
    {\gamma; \cdot \vdash \oc A}
    \and
    \inferrule*[right=$\oc$-E]
    {\gamma; \delta_1 \vdash \oc A \\ \gamma, A; \delta_2 \vdash B}
    {\gamma; \delta_1, \delta_2 \vdash B}
    \and
    \inferrule*[right=$\top$-I]{ }
    {\gamma; \delta \vdash \top}
    \and
    \inferrule*[right=$\with$-I]
    {\gamma; \delta \vdash A \\ \gamma, \delta \vdash B}
    {\gamma; \delta \vdash A \with B}
    \and
    \inferrule*[right=$\with$-E$_i$]
    {\gamma; \delta \vdash A_0 \with A_1}
    {\gamma; \delta \vdash A_i}
    \and
    \inferrule*[right=$0$-E]
    {\gamma; \delta_1 \vdash 0}
    {\gamma; \delta_1, \delta_2 \vdash A}
    \and
    \inferrule*[right=$\oplus$-I$_i$]
    {\gamma; \delta \vdash A_i}
    {\gamma; \delta \vdash A_0 \oplus A_1}
    \and
    \inferrule*[right=$\oplus$-E]
    {
      \gamma; \delta_1 \vdash A \oplus B \\
      \gamma; \delta_2, A \vdash C \\
      \gamma; \delta_2, B \vdash C
    }
    {\gamma; \delta_1, \delta_2 \vdash C}
  \end{mathpar}
  \label{fig:dill}
  \caption{The rules of DILL, extended with additive connectives}
\end{figure}

\begin{figure}
  \begin{mathpar}
    \begin{eqns}
      \mathrm{DILL} &\hookrightarrow& \name_\instDILL \\
      Y &\mapsto& \iota_Y \\
      I &\mapsto& I \\
      A \otimes B &\mapsto& A \otimes B \\
      A \multimap B &\mapsto& A \multimap B \\
      \oc A &\mapsto& \oc{\gr\omega}{A} \\
      0 &\mapsto& 0 \\
      A \oplus B &\mapsto& A \oplus B \\
      \top &\mapsto& \top \\
      A \with B &\mapsto& A \with B
    \end{eqns}
    \and
    \begin{eqns}
      \mathrm{PD} &\hookrightarrow& \name_\instPD \\
      Y &\mapsto& \iota_Y \\
      \top &\mapsto& I \\
      A \wedge B &\mapsto& A \with B \\
      A \supset B &\mapsto& A \multimap B \\
      \Box A &\mapsto& \oc{\valid}{A} \\
      \bot &\mapsto& 0 \\
      A \vee B &\mapsto& A \oplus B
    \end{eqns}
  \end{mathpar}
  \label{fig:dill-pd}
  \caption{Embedding of DILL and PD types into \name}
\end{figure}

\begin{proposition}[DILL $\to$ \name]
  Given a DILL derivation of $\gamma; \delta \vdash A$, we can produce a
  $\name_{\instDILL}$ derivation of
  $\gr\omega\gamma, \gr1\delta \vdash A$.
\end{proposition}
\begin{proof}
  By induction on the derivation.
  We have $\gr\omega \leq \gr0$, which allows us to discard
  intuitionistic variables at the var rules, and both
  $\gr1 \leq \gr1$ and $\gr\omega \leq \gr1$, which allow
  us to use both linear and intuitionistic variables.

  Weakening is used when splitting linear variables between two premises.
  For example, \TirName{$\otimes$-I} in DILL is as follows.
  \[
    \inferrule*[right=$\otimes$-I]
    {\gamma; \delta_t \vdash t : A \\ \gamma; \delta_u \vdash u : B}
    {\gamma; \delta_t, \delta_u \vdash t \otimes u : A \otimes B}
  \]
  From this, our new derivation is as follows.
  \[
    \inferrule*[right=$\otimes$-I]
    {
      \inferrule*[right=Weak]
      {\mathit{ih}_t \\\\
        \gr\omega\gamma, \gr1\delta_t
        \vdash M_t : A}
      {\gr\omega\gamma, \gr1\delta_t,
        \gr0\delta_u
        \vdash M_t : A}
      \\
      \inferrule*[right=Weak]
      {\mathit{ih}_u \\\\
        \gr\omega\gamma, \gr1\delta_u
        \vdash M_u : A}
      {\gr\omega\gamma, \gr0\delta_t,
        \gr1\delta_u
        \vdash M_u : A}
    }
    {\gr\omega\gamma, \gr1\delta_t,
      \gr1\delta_u
      \vdash \plr{M_t, M_u} : A \otimes B}
  \]
\end{proof}

When translating from \name{} to DILL, we first coerce the \name{} derivation
to be in a form easily amenable to translation into DILL\@.
An example of a \name{} derivation with no direct translation into DILL is the
following.
In DILL terms, the intuitionistic variable of the conclusion becomes a linear
variable in the premises.
Such a move is admissible in DILL, but does not come naturally.

\[
  \inferrule*[right=$\otimes$-I]
  {
    \inferrule*[right=var]{ }
    {\gr1A : x \vdash A}
    \\
    \inferrule*[right=var]{ }
    {\gr1A : x \vdash A}
    \\
    \gr\omega \leq \gr1 + \gr1
  }
  {\gr\omega A : x \vdash A \otimes A}
\]

To avoid such situations, and therefore manipulations on DILL derivations, I
show that all $\name_{\instDILL}$ derivations can be made in \emph{bottom-up}
style.
In bottom-up style, the algebraic facts we make use of are dictated by making
most general choices based on the conclusions of rules.
Bottom-up style corresponds to a (non-deterministic) form of
\emph{usage checking}, and the following lemma can be understood as saying
that that form of usage checking is sufficiently general.

\begin{definition}
  A derivation is said to be \emph{$\instDILL$-bottom-up} if only the following
  facts about addition and multiplication are used, and all proofs of
  inequalities not at leaves are by reflexivity (i.e, not using the facts that
  $\gr\omega \leq \gr0$ and $\gr\omega \leq \gr1$).

  \makebox[\textwidth][s]{
    \begin{tabular}{c|ccc}
      $+$ & $\gr0$ & $\gr1$ & $\gr\omega$ \\ \hline
      $\gr0$ & $\gr0$ & $\gr1$ & - \\
      $\gr1$ & $\gr1$ & - & - \\
      $\gr\omega$ & - & - & $\gr\omega$ \\
    \end{tabular}
    \begin{tabular}{c|ccc}
      $*$ & $\gr0$ & $\gr1$ & $\gr\omega$ \\ \hline
      $\gr0$ & - & - & $\gr0$ \\
      $\gr1$ & $\gr0$ & $\gr1$ & $\gr\omega$ \\
      $\gr\omega$ & $\gr0$ & - & $\gr\omega$ \\
    \end{tabular}
  }
\end{definition}

Bottom-up style enforces that whenever we split a context into two (for
example, in the rule \TirName{$\otimes$-I}) all unused variables in the
conclusion stay unused in the premises, intuitionistic variables stay
intuitionistic, and linear variables go either left or right.
Multiplication is only used in the rule \TirName{$\oc\gr{r}$-I}, at which point
both the result and left argument are available.
Here, the bottom-up style enforces that linear variables never appear in the
premise of \TirName{$\oc{\gr\omega}$-I}.

\begin{lemma}
  Every $\name_{\instDILL}$ derivation can be translated into a bottom-up
  $\name_{\instDILL}$ derivation.
\end{lemma}
\begin{proof}
  By induction on the shape of the derivation.
  When we come across a non-bottom-up use of addition, it must be that the
  corresponding variable in the conclusion has annotation $\gr\omega$.
  By subusaging, we can give this variable annotation $\gr\omega$ in
  the premises too, before translating the subderivations to bottom-up
  style.
  A similar argument applies to uses of multiplication, remembering that both
  the left argument and result are fixed.
\end{proof}

\begin{proposition}[\name{} $\to$ DILL]
  Given a $\name_{\instDILL}$ derivation of
  $\gr\omega\gamma, \gr1\delta,
  \gr0\theta \vdash A$ which contains only types expressible
  in DILL, we can produce a DILL derivation of $\gamma; \delta \vdash A$.
\end{proposition}
\begin{proof}
  By induction on the derivation having been translated to bottom-up form.

  In the case of \TirName{var}, all of the unused variables have annotation
  greater than $\gr0$, i.e., $\gr0$ or $\gr\omega$.
  Those annotated $\gr0$ are absent from the DILL derivation, and those
  annotated $\gr\omega$ are in the intuitionistic context.
  The used variable is annotated either $\gr1$ or $\gr\omega$.
  In the first case, we use \TirName{Lin-Ax}, and in the second case,
  \TirName{Int-Ax}.

  All binding of variables in \name{} maps directly onto DILL\@.

  Because we translated to bottom-up form, additions, as seen in, for example,
  the \TirName{$\otimes$-I} rule, can be handled straightforwardly.
  Any intuitionistic variables in the conclusion correspond to intuitionistic
  variables in both premises.
  Any linear variables in the conclusion correspond to a linear variable in
  exactly one of the premises, and is absent in the other premise.

  The only remaining rule is \TirName{$\oc\gr{r}$-I}, of which we only cover
  \TirName{$\oc{\gr\omega}$-I} (the other two targeting types not found
  in DILL).
  In this case, we know that every variable in the conclusion is annotated
  either $\gr0$ or $\gr\omega$, and every variable in the premise is
  annotated the same way.
  This corresponds exactly to the restrictions of DILL's \TirName{$\oc$-I}.
\end{proof}

\subsection{Pfenning Davies}\label{sec:trans-pd}

The translation to and from the modal system of Pfenning and Davies
\cite{judgmental} (henceforth \emph{PD}) is similar to the translation to and
from DILL\@.
I present my variant of PD, again adding some common connectives, in
\cref{fig:pd}
The main difference is the algebra at which \name{} is instantiated.

\begin{figure}
  \begin{mathpar}
    \inferrule*[right=hyp]{ }
    {\gamma; \delta, A\;\mathit{true} \vdash A\;\mathit{true}}
    \and
    \inferrule*[right=hyp*]{ }
    {\gamma, A\;\mathit{valid}; \delta \vdash A\;\mathit{true}}
    \and
    \inferrule*[right=$\supset$I]
    {\gamma; \delta, A\;\mathit{true} \vdash B\;\mathit{true}}
    {\gamma; \delta \vdash A \supset B\;\mathit{true}}
    \and
    \inferrule*[right=$\supset$E]
    {
      \gamma; \delta \vdash A \supset B\;\mathit{true} \\
      \gamma; \delta \vdash A\;\mathit{true}
    }
    {\gamma; \delta \vdash B\;\mathit{true}}
    \and
    \inferrule*[right=$\Box$I]
    {\gamma; \cdot \vdash A\;\mathit{true}}
    {\gamma; \delta \vdash \Box A\;\mathit{true}}
    \and
    \inferrule*[right=$\Box$-E]
    {
      \gamma; \delta \vdash \Box A\;\mathit{true} \\
      \gamma, A\;\mathit{valid}; \delta \vdash B\;\mathit{true}
    }
    {\gamma; \delta \vdash B\;\mathit{true}}
    \and
    \inferrule*[right=$\top$-I]{ }
    {\gamma; \delta \vdash \top\;\mathit{true}}
    \and
    \inferrule*[right=$\wedge$-I]
    {
      \gamma; \delta \vdash A\;\mathit{true} \\
      \gamma, \delta \vdash B\;\mathit{true}
    }
    {\gamma; \delta \vdash A \wedge B\;\mathit{true}}
    \and
    \inferrule*[right=$\wedge$-E$_i$]
    {\gamma; \delta \vdash A_0 \wedge A_1\;\mathit{true}}
    {\gamma; \delta \vdash A_i\;\mathit{true}}
    \and
    \inferrule*[right=$\bot$-E]
    {\gamma; \delta \vdash \bot\;\mathit{true}}
    {\gamma; \delta \vdash A\;\mathit{true}}
    \and
    \inferrule*[right=$\vee$-I$_i$]
    {\gamma; \delta \vdash A_i\;\mathit{true}}
    {\gamma; \delta \vdash A_0 \vee A_1\;\mathit{true}}
    \and
    \inferrule*[right=$\vee$-E]
    {
      \gamma; \delta \vdash A \vee B\;\mathit{true} \\
      \gamma; \delta, A \vdash C\;\mathit{true} \\
      \gamma; \delta, B \vdash C\;\mathit{true}
    }
    {\gamma; \delta \vdash C\;\mathit{true}}
  \end{mathpar}
  \label{fig:pd}
  \caption{The rules of PD, extended with several standard connectives}
\end{figure}

\begin{definition}
  Let $\instPD$ denote the following semiring on the partially ordered set
  $\{\valid \triangleleft \true \triangleleft \unused\}$.
  \begin{itemize}
    \item $0 := \unused$.
    \item $+$ is the meet ($\wedge$) according to the subusaging order.
    \item $1 := \true$.
    \item
      \begin{tabular}{c|ccc}
        $*$ & $\unused$ & $\true$ & $\valid$ \\ \hline
        $\unused$ & $\unused$ & $\unused$ & $\unused$ \\
        $\true$ & $\unused$ & $\true$ & $\valid$ \\
        $\valid$ & $\unused$ & $\valid$ & $\valid$ \\
      \end{tabular}
  \end{itemize}
\end{definition}

The $\unused$ annotation plays only a formal role in this example.
Meanwhile, $\true$ and $\valid$ correspond to the judgement forms
$\mathit{true}$ and $\mathit{valid}$ from PD\@.
Addition being the meet makes it idempotent.
Furthermore, it gives us that $\true + \valid = \valid$ --- if somewhere we
require an assumption to be true, and elsewhere require it to be valid, then
ultimately it must be valid (from which we can deduce that it is true).
Multiplication is designed to make $\oc{\valid}$ act like PD's $\Box$.
In particular, $\valid * \valid = \valid$ says that the valid assumptions are
available before and after \TirName{$\oc{\valid}$-I}, whereas
$\valid * \true = \valid$ says that valid assumptions in the conclusion can be
weakened to true assumptions in the premise.
The latter fact does not appear in PD, and will be excluded from
\emph{bottom-up} derivations.

To keep my notation consistent with that of DILL, I swap the roles of
$\gamma$ and $\delta$ in PD compared to what they were in the original paper.
Thus, my PD judgements are of the form $\gamma; \delta \vdash A~\mathit{true}$,
where $\gamma$ contains valid assumptions and $\delta$ contains true
assumptions.

\begin{proposition}[PD $\to$ \name]
  Given a PD derivation of $\gamma; \delta \vdash t : A~\mathit{true}$, we can
  produce a $\name_{\instPD}$ derivation of
  $\valid\gamma, \true\delta \vdash A$.
\end{proposition}
\begin{proof}
  By induction on the PD derivation.
  Most PD rules have direct $\name$ counterparts, noting that variables of any
  annotation can be discarded and duplicated because we have both
  $\gr r \leq \gr 0$ and
  $\gr r \leq \gr r + \gr r$ for all
  $\gr r$.

  Care must be taken with the \TirName{$\Box$I} rule.
  We have, from the induction hypothesis, a $\name$ derivation of
  $\gr\Box\gamma \vdash A$.
  By \TirName{$\oc\valid$-I}, we have
  $\gr\Box\gamma \vdash \oc\valid A$.
  To get the desired conclusion, we must use \TirName{Weak} to get
  $\gr\Box\gamma, \gr\unused\delta \vdash \oc\valid A$, and
  then \TirName{Subuse} on the variables we just introduced (noting that
  $\true \leq \unused$) to get
  $\gr\Box\gamma, \gr\true\delta \vdash \oc\valid A$.
\end{proof}

For translating from $\name_{\instPD}$ to PD, I introduce a similar notion of
\emph{bottom-up} derivations as I did for DILL\@.
Every $\name_{\instPD}$ derivation can be translated into bottom-up style, and
then be directly translated into PD\@.

\begin{definition}
  A derivation is said to be \emph{$\instPD$-bottom-up} if only the following
  facts about addition and multiplication are used, and all proofs of
  inequalities not at leaves are by reflexivity.

  \makebox[\textwidth][s]{
    \begin{tabular}{c|ccc}
      $+$ & $\unused$ & $\true$ & $\valid$ \\ \hline
      $\unused$ & $\unused$ & - & - \\
      $\true$ & - & $\true$ & - \\
      $\valid$ & - & - & $\valid$ \\
    \end{tabular}
    \begin{tabular}{c|ccc}
      $*$ & $\unused$ & $\true$ & $\valid$ \\ \hline
      $\unused$ & - & - & $\unused$ \\
      $\true$ & $\unused$ & $\true$ & $\valid$ \\
      $\valid$ & $\unused$ & - & $\valid$ \\
    \end{tabular}
  }
\end{definition}

\begin{lemma}
  Every $\name_{\instPD}$ derivation can be translated into a bottom-up
  $\name_{\instPD}$ derivation.
\end{lemma}
\begin{proof}
  By induction on the shape of the derivation.
  Given that addition is a meet, it is clear that any non-bottom-up uses of
  addition come from one of the arguments being greater than the result.
  Therefore, it is safe to make this argument smaller in the corresponding
  premise (via subusaging), before translating that subderivation.
  For multiplication, again, there is always a lesser value of the right
  argument that will take us from a non-bottom-up fact to a bottom-up fact with
  the same left argument and result.
\end{proof}

\begin{proposition}[\name{} $\to$ PD]
  Given a $\name_{\instPD}$ derivation of
  $\valid\gamma, \true\delta, \unused\theta
  \vdash M : A$ which does not contain types using $\oc{\unused}{}$ or
  $\oc{\true}$, we can produce a PD derivation of
  $\gamma; \delta \vdash A~\mathit{true}$.
\end{proposition}
\begin{proof}
  We translate away tensor products and tensor units using
  \cref{thm:top-meet}, and translate the resulting derivation to bottom-up
  form.
  The proof proceeds by induction on the resulting derivation in the obvious
  way.
\end{proof}

As mentioned in \cref{sec:alt}, the system of \citet{AbelBernardy2020}
is unable to embed PD in this way, as it would prove
$\Box(A \wedge B) \to \Box A \wedge \Box B$, where PD and $\name$ do not.
In fact, this example shows that, even when weakening and contraction are
admissible, with- and tensor-products are distinct in their system in the
presence of modalities.


\section{Conclusion}\label{sec:semirings-conc}

In this chapter, I have presented the calculus $\name$, with particular focus on
the posemiring usage annotations.
Posemirings are sufficient for many use cases, as shown by the examples in this
chapter.
However, there are many more substructural disciplines in the literature which
cannot be expressed using posemirings.

An important point to note is that $\name$ cannot be instantiated to become a
bunched logic, in the sense of \citet{oHP99}.
This is despite the use of the bunched connectives in the \emph{definition} of
$\name$, and despite the formally tree-shaped contexts.
However, the variable rule of $\name$ essentially treats the variables in the
context independently, except for the individual checking of usage annotations
on each variable.
Therefore, we cannot talk about more interesting spatial relationships between
variables, as required by bunched logic.
This property also precludes us from capturing calculi without exchange
(non-commutative logics), such as Lambek calculus~\citep{Lambek58}.
Fitch-style systems~\citep{Borghuis-thesis} probably also are precluded
similarly.

The work I present here has some methodological similarities with the earlier
work of \citet{LicataSR17}.
In that framework, one can provide a very precise \emph{mode theory}, expressing
which structural rules are available, and from it get a sequent calculus with
cut-elimination.
They encode many systems, including a non-associative logic (where context are
trees obeying no structural rules) and a bunched logic, and the framework is
probably expressive enough to encode all posemiring-based usage disciplines.
However, aside from the structural rules, the sequent calculus is fixed ---
there are two connectives: $\mathsf F$ (internalising the left of the sequent,
comparable to my
$\oc_{\gr{r_1}}\plr{-} \otimes \cdots \otimes \oc_{\gr{r_n}}\plr{-}$) and
$\mathsf U$ (internalising the whole sequent, comparable to my
$\oc_{\gr{r_1}}\plr{-} \multimap \cdots \multimap \oc_{\gr{r_n}}\plr{-}
\multimap \plr{-}$), which have left and right rules, and then the only other
rule is the variable rule.
This restriction is what allows the cut elimination theorem to be even
plausible, but means that other connectives, like additives and inductive types,
and more exotic syntaxes, like those I will present in
\cref{sec:other-syntaxes}, would have to be developed from scratch.

The distinction between sharing and separating conjunction of premises naturally
falls out of the posemiring approach to usage restrictions.
However, a similar distinction also appears in other approaches to linearity,
and perhaps in other substructural systems.
Indeed, in the paper that inspired the bunched connectives~\citep{RPKV20},
linearity is enforced in a similar style as in Yalla~\citep{laurent18}, using
lists which can be split.
It would be interesting to see future work on substructural systems work out the
appropriate premise connectives and define their systems from there, rather than
directly manipulating contexts.
It may also be possible to see future work profitably abstracting over the
posemiring annotations, requiring only that contexts form something like a
commutative $\mathrm{Rel}$-monoid supporting variable-binding and
variable access.
