\chapter{Usage restriction via semirings}\label{sec:semirings}

The methods described in \cref{sec:simple} for the simply typed
$\lambda$-calculus
make crucial use of \emph{weakening} --- the fact that if we have
$\Gamma \vdash A$, then we also have $\Gamma, \Delta \vdash A$.
We use this property to update environments as we take them under binders.
However, as we saw in \cref{sec:linearity}, there are interesting calculi in
which general weakening does not hold.
As such, one of the aims of this chapter will be to find a form of weakening
applicable to variables of any type, while essentially retaining linearity
(as opposed to affineness).

As explained by \citet{McBride16}, the first insight is to, instead of
removing variables from the context of certain subterms, add an annotation to
free variables saying whether or not they are to be used.
I use an annotation $\gr0$ on variables that are not to be used, and an
annotation $\gr1$ on variables that are to be used.
This convention lets us transcribe the usual $\otimes$-introduction rule
(below left) as a rule with usage annotations (below right).
In the notation on the right, I let $\Gamma = \grP\gamma$ and
$\Delta = \grQ\delta$, where $\Gamma$ is a whole context comprising a
\emph{usage context} $\grP$ and a \emph{typing context} $\gamma$.
A usage context is a list of usage annotations, so
$\grP = \gr{r_1}, \ldots, \gr{r_m}$ and a typing context is a list of types, so
$\gamma = A_1, \ldots, A_m$.
When combined, the usage context and the typing context will be of the same
length.
Explicit contexts will usually be written with usage annotations and types
interspersed, as $\gr{r_1}A_1, \ldots, \gr{r_m}A_m$.
I use $\gr r\gamma$ to abbreviate $\gr rx_1, \ldots, \gr rx_m$.

\[
  \ebrule{%
    \hypo{\Gamma \vdash A}
    \hypo{\Delta \vdash B}
    \infer2{\Gamma, \Delta \vdash A \otimes B}
  }
  \quad\rightsquigarrow\quad
  \ebrule{%
    \hypo{\gr1\gamma, \gr0\delta \vdash A}
    \hypo{\gr0\gamma, \gr1\delta \vdash B}
    \infer2{\gr1\gamma, \gr1\delta \vdash A \otimes B}
  }
\]

The eventual target of all these $\gr0$-annotated variables is the variable
rule, which I transcribe as follows.
The $\gr1$ shows us that we can use the variable thus annotated, while the
$\gr0$s let us discard all of the other variables in $\gamma$.

\[
  \ebrule{%
    \infer0{A \vdash A}
  }
  \quad\rightsquigarrow\quad
  \ebrule{%
    \infer0{\gr0\gamma, \gr1A \vdash A}
  }
\]

The use of $\gr0$ gives us the property that variables never go out of
scope in subterms; rather, we lose the ability to use certain variables, but
retain the ability to refer to them metatheoretically.
Additionally, we recover a form of weakening: if $\Gamma \vdash A$, then also
$\Gamma, \gr0\delta \vdash A$, because the resulting term indeed uses no
variables from $\delta$.
I prove the admissibility of weakening for terms will come in \cref{sec:lrsub}.

If we follow the DILL style of variable management explained in \cref{sec:dill},
there are not just the two
states \emph{to be used} ($\gr1$) and \emph{not to be used} ($\gr0$), but also
\emph{usable unrestrictedly}.
If we assign unrestricted (or \emph{intuitionistic}) variables an annotation
$\gr\omega$, we can make the following transcription of the DILL
$\otimes$-introduction rule.

\[
  \ebrule{%
    \hypo{\Theta; \Gamma \vdash A}
    \hypo{\Theta; \Delta \vdash B}
    \infer2{\Theta; \Gamma, \Delta \vdash A \otimes B}
  }
  \quad\rightsquigarrow\quad
  \ebrule{%
    \hypo{\gr\omega\theta, \gr1\gamma, \gr0\delta \vdash A}
    \hypo{\gr\omega\theta, \gr0\gamma, \gr1\delta \vdash B}
    \infer2{\gr\omega\theta, \gr1\gamma, \gr1\delta \vdash A \otimes B}
  }
\]

To conceptualise the criteria on the usage annotations involved in this rule,
I introduce an additive structure over usage annotations.
The rule stated above relies on the facts that $\gr1 + \gr0 = \gr1$,
$\gr0 + \gr1 = \gr1$, and $\gr\omega + \gr\omega = \gr\omega$.
Addition lifts pointwise to vectors of usage annotations (the green capital
calligraphic $\grP$, $\grQ$, and $\grR$).
A beneficial side-effect of the fact that $\gr0 + \gr0 = \gr0$ is that the
rule on the right below is actually more general, and accepts $\gr0$-annotated
variables in its
conclusion, which is essential for weakening to be admissible.

\[
  \ebrule{%
    \hypo{\gr\omega\theta, \gr1\gamma, \gr0\delta \vdash A}
    \hypo{\gr\omega\theta, \gr0\gamma, \gr1\delta \vdash B}
    \infer2{\gr\omega\theta, \gr1\gamma, \gr1\delta \vdash A \otimes B}
  }
  \quad\rightsquigarrow\quad
  \ebrule{%
    \hypo{\grR = \grP + \grQ}
    \hypo{\grP\gamma \vdash A}
    \hypo{\grQ\gamma \vdash B}
    \infer3{\grR\gamma \vdash A \otimes B}
  }
\]

Some other transcriptions from DILL to the usage annotation style are as
follows.
I unify the variable rules (the one for linear variables and the one for
intuitionistic variables) by introducing a coercibility ordering $\leq$ on usage
annotations.
We have $\gr\omega \leq \gr1$ because an intuitionistic variable can fill the
demand of a linear variable by dereliction.
We also have $\gr\omega \leq \gr0$, because intuitionistic variables can be
weakened away like $\gr0$-annotated variables.
This ordering information is shown in the diagram
\begin{tikzpicture}[baseline]
  \node(omega) at (0,0) {$\gr\omega$};
  \node(0) [above left of=omega] {$\gr0$};
  \node(1) [above right of=omega] {$\gr1$};

  \draw (omega) -- (0);
  \draw (omega) -- (1);
\end{tikzpicture}.
All together, this means that at the (only) variable rule, the variable being
used must have annotation less than or equal to $\gr1$, and every other variable
must have annotation less than or equal to $\gr0$.
I write this requirement as $\grR \leq \langle x \rvert$, where
$\langle x \rvert$ is the \emph{basis vector} at position $x$.

\[
  \ebrule{%
    \infer0{\Theta; A \vdash A}
  }
  \quad\rightsquigarrow\quad
  \ebrule{%
    \infer0{\gr\omega\theta, \gr1A, \gr0\delta \vdash A}
  }
  \quad\rightsquigarrow\quad
  \ebrule{%
    \hypo{\grR \leq \bra x}
    \hypo{\gamma_x = A}
    \infer2{\grR\gamma \vdash A}
  }
\]

\[
  \ebrule{%
    \infer0{\Theta, A; {\cdot} \vdash A}
  }
  \quad\rightsquigarrow\quad
  \ebrule{%
    \infer0{\gr\omega\theta, \gr\omega A, \gr0\delta \vdash A}
  }
  \quad\rightsquigarrow\quad
  \ebrule{%
    \hypo{\grR \leq \bra x}
    \hypo{\gamma_x = A}
    \infer2{\grR\gamma \vdash A}
  }
\]

The final interesting rule form to cover is found in DILL's
$\oc$-introduction rule.
DILL's $\oc$-introduction can be though of as an $\gr\omega$-ary counterpart to
$\otimes$-introduction, though with the same premise each time rather than
$\gr\omega$-many premises.
This explains why only $\gr\omega$- and
$\gr0$-annotated variables can appear in the conclusion of $\oc$-introduction,
and also justifies the choice below of multiplication (vector scaling) as the
algebraic operation controlling the $\oc$-modality.

\[
  \ebrule{%
    \hypo{\Theta; {\cdot} \vdash A}
    \infer1{\Theta; {\cdot} \vdash \oc A}
  }
  \quad\rightsquigarrow\quad
  \ebrule{%
    \hypo{\gr\omega\theta, \gr0\delta \vdash A}
    \infer1{\gr\omega\theta, \gr0\delta \vdash \oc A}
  }
  \quad\rightsquigarrow\quad
  \ebrule{%
    \hypo{\grR \leq \gr\omega\grP}
    \hypo{\grP\gamma \vdash A}
    \infer2{\grR\gamma \vdash \oc_{\gr\omega} A}
  }
\]

In summary, the structure we have required of the set of usage annotations is
that they have addition (for $\otimes$-introduction and similar rules),
multiplication (for $\oc$-introduction), a $1$ (for a variable being used), a
$0$ (for a variable not being used), and an ordering (allowing for subsumption
of usage restrictions).
Together, these form a \emph{partially ordered semiring} (posemiring), the laws
of which are both supported by examples and necessary for the syntax to be well
behaved.
%\todo{Refer back to preliminaries for definition of posemiring}
For concreteness, I collect together the definition of the
$\{\gr0, \gr1, \gr\omega\}$ posemiring I have been using so far.

\begin{example}\label{def:lin-semiring}
  The \emph{$\{\gr0, \gr1, \gr\omega\}$ posemiring}, also known as the
  \emph{linearity posemiring}, has the operations given as follows, with
  $0 \coloneqq \gr0$ and $1 \coloneqq \gr1$:

  \makebox[\textwidth][s]{
    \begin{tabular}{c|ccc}
      $+$ & $\gr0$ & $\gr1$ & $\gr\omega$ \\ \hline
      $\gr0$ & $\gr0$ & $\gr1$ & $\gr\omega$ \\
      $\gr1$ & $\gr1$ & $\gr\omega$ & $\gr\omega$ \\
      $\gr\omega$ & $\gr\omega$ & $\gr\omega$ & $\gr\omega$ \\
    \end{tabular}
    \begin{tabular}{c|ccc}
      $*$ & $\gr0$ & $\gr1$ & $\gr\omega$ \\ \hline
      $\gr0$ & $\gr0$ & $\gr0$ & $\gr0$ \\
      $\gr1$ & $\gr0$ & $\gr1$ & $\gr\omega$ \\
      $\gr\omega$ & $\gr0$ & $\gr\omega$ & $\gr\omega$ \\
    \end{tabular}
    \begin{tikzpicture}[baseline]
      \node(omega) at (0,0) {$\gr\omega$};
      \node(0) [above left of=omega] {$\gr0$};
      \node(1) [above right of=omega] {$\gr1$};

      \draw (omega) -- (0);
      \draw (omega) -- (1);
    \end{tikzpicture}
  }
\end{example}

%In \cref{sec:simple}, we saw that the logical rules of simply typed
%$\lambda$-calculus can be described in terms of three basic premise combinators:
%$\dot1$, standing for no premises; $\dottimes$, allowing for multiple premises
%in the same context; and $\Theta \vdash A$, requiring a subterm of type $A$
%having bound the extra variables from context $\Theta$.
%However, we remember from \cref{sec:linearity} that in a substructural setting,
%we do not always want to copy assumptions for use in all subterms.
%This motivates me to introduce the additional premise combinators $I^*$, $\sep$,
%and $\cdot$ in \cref{sec:lnd}, allowing for the modes of usage exhibited in
%the introduction rules for $I$, $\otimes$, and $\oc$, respectively.

The rest of this chapter proceeds as follows.
In \cref{sec:lr}, I define the usage-annotated calculus \name{}, which has
appeared in my previous work~\citep{WA21}, and can be seen as a
simply typed version of Atkey's dependently typed calculus QTT~\citep{Atkey18}.
The idea of \name{} is to augment the simply typed $\lambda$-calculus with
annotations on free variables, and give enough types to manipulate these
annotations.
Given this new calculus $\name$, the first goal is to apply the techniques of
\cref{sec:simple} to it, yielding a simultaneous substitution operation.
To do this, I use \cref{sec:lnd} to introduce notation that allows us to restate
the typing rules of \name{} to not mention contexts explicitly, as was the style
in \cref{sec:simple}.
This new notation --- the \emph{bunched connectives} --- is versatile at
defining simply typed usage-aware syntaxes, and I give further non-$\name$
examples in \cref{sec:rec}.
Finally, I justify connections to linear logic and modal logic in
\cref{sec:translation}, where I translate $\name$ terms to and from
DILL~\citep{Barber1996} and the modal calculus of \citet{judgmental}.
%The application of the techniques of \cref{sec:simple} is left to
%\cref{sec:ren-sub-lr}.

\section{A usage-annotated calculus $\name$}\label{sec:lr}
In this section, I introduce the syntax of the type theory \name{}, which makes
use of posemiring usage annotations to express the usage restrictions found in
DILL and other calculi.
I use this syntax to write some example programs, which will motivate the
denotational semantics explored in \cref{sec:wrel}.
For the rest of this thesis, particularly
\cref{sec:semirings,sec:ren-sub-lr,sec:wrel},
\name{} will serve as both a prototypical
usage-constrained syntax and a target of semantic analyses.

The calculus \name{} is similar in spirit to intuitionistic linear logic (ILL),
which we saw in \cref{sec:linearity}.
The types of \name{}, listed in \cref{fig:lr-types}, are almost identical
to those of ILL, differing only in the exponential modality $\oc$
(read ``bang'').
In particular, I include distinguished tensor- and with-product types
($\otimes$, $\with$) and their units ($I$, $\top$), function types
($\multimap$), additive sum types and their unit ($\oplus$, $0$), and the
graded modality $\oc_{\gr r}$.
The idea of $\oc_{\gr r}$ is to internalise an annotation of $\gr r$ on a
variable in the context.

\begin{figure}
  \begin{displaymath}
    A, B, C \Coloneqq I \mid A \otimes B \mid A \multimap B \mid \top
    \mid A \with B \mid 0 \mid A \oplus B \mid \oc_{\gr r} A
  \end{displaymath}
  \caption{The types of \name{}}
  \label{fig:lr-types}
\end{figure}

\begin{figure}
  \begin{displaymath}
    \begin{prooftree}
      \hypo{\gamma \ni x : A}
      \hypo{\grP \le \langle x \rvert}
      \infer2[Var]{\grP\gamma \vdash A}
    \end{prooftree}
  \end{displaymath}

  \begin{displaymath}
    \begin{matrix}
      \begin{prooftree}
        \hypo{\grP \le \gr0}
        \infer1[$I$-I]{\grP\gamma \vdash I}
      \end{prooftree}
      &&
      \begin{prooftree}
        \hypo{\grR \le \grP + \grQ}
        \hypo{\grP\gamma \vdash I}
        \hypo{\grQ\gamma \vdash C}
        \infer3[$I$-E]{\grR\gamma \vdash C}
      \end{prooftree}
      \\\\
      \begin{prooftree}
        \hypo{\grR \le \grP + \grQ}
        \hypo{%
          \begin{matrix}
            \grP\gamma \vdash A \\ \grQ\gamma \vdash B
          \end{matrix}%
        }
        \infer2[$\otimes$-I]{\grR\gamma \vdash A \otimes B}
      \end{prooftree}
      &&
      \begin{prooftree}
        \hypo{%
          \begin{matrix}
            \grR \le \grP + \grQ \\ \grP\gamma \vdash A \otimes B
          \end{matrix}%
        }
        \hypo{\grQ\gamma, \gr1A, \gr1B \vdash C}
        \infer2[$\otimes$-E]{\grR\gamma \vdash C}
      \end{prooftree}
      \\\\
      \begin{prooftree}
        \hypo{\grR\gamma, \gr1A \vdash B}
        \infer1[$\multimap$-I]{\grR\gamma \vdash A \multimap B}
      \end{prooftree}
      &&
      \begin{prooftree}
        \hypo{\grR \le \grP + \grQ}
        \hypo{\grP\gamma \vdash A \multimap B}
        \hypo{\grQ\gamma \vdash A}
        \infer3[$\multimap$-E]{\grR\gamma \vdash B}
      \end{prooftree}
      \\\\
      \begin{prooftree}
        \infer0[$\top$-I]{\grR\gamma \vdash \top}
      \end{prooftree}
      &&
      \textrm{(no $\top$-E)}
      \\\\
      \begin{prooftree}
        \hypo{\grR\gamma \vdash A}
        \hypo{\grR\gamma \vdash B}
        \infer2[$\with$-I]{\grR\gamma \vdash A \with B}
      \end{prooftree}
      &&
      \begin{prooftree}
        \hypo{\grR\gamma \vdash A_0 \with A_1}
        \infer1[$\with$-E$_i$, for $i \in \{0,1\}$]{\grR\gamma \vdash A_i}
      \end{prooftree}
      \\\\
      \textrm{(no $0$-I)}
      &&
      \begin{prooftree}
        \hypo{\grR \le \grP + \grQ}
        \hypo{\grP\gamma \vdash 0}
        \infer2[$0$-E]{\grR\gamma \vdash C}
      \end{prooftree}
      \\\\
      \begin{prooftree}
        \hypo{\grR\gamma \vdash A_i}
        \infer1[$\oplus$-I$_i$, for $i \in \{0,1\}$]%
        {\grR\gamma \vdash A_0 \oplus A_1}
      \end{prooftree}
      &&
      \begin{prooftree}
        \hypo{%
          \begin{matrix}
            \grR \le \grP + \grQ \\ \grP\gamma \vdash A \oplus B
          \end{matrix}%
        }
        \hypo{%
          \begin{matrix}
            \grQ\gamma, \gr1A \vdash C \\ \grQ\gamma, \gr1B \vdash C
          \end{matrix}%
        }
        \infer2[$\oplus$-E]{\grR\gamma \vdash C}
      \end{prooftree}
      \\\\
      \begin{prooftree}
        \hypo{\grR \le \gr r\grP}
        \hypo{\grP\gamma \vdash A}
        \infer2[$\oc$-I]{\grR\gamma \vdash \oc\gr rA}
      \end{prooftree}
      &&
      \begin{prooftree}
        \hypo{\grR \le \grP + \grQ}
        \hypo{\grP\gamma \vdash \oc\gr rA}
        \hypo{\grQ\gamma, \gr rA \vdash C}
        \infer3[$\oc$-E]{\grR\gamma \vdash C}
      \end{prooftree}
    \end{matrix}
  \end{displaymath}
  \caption{\name{}}
  \label{fig:lr}
\end{figure}

I will not cover any operational semantics or equational theory of \name{} in
this thesis.
I will discuss a denotational semantics in \cref{sec:wrel}.

The following features are of note.

\paragraph{Subusaging}
Several typing rules contain constraints of the form $\grP \leq \grQ$, for
certain usage vectors $\grP$ and $\grQ$.
We saw subusaging in the introduction to this chapter in the specific case of
$\Ann$ being formed from the poset $\{\gr0 > \gr\omega < \gr1\}$.
This allowed variables annotated $\gr\omega$ (``unrestricted'') to be both
weakened/discarded (because $\gr\omega \leq \gr0$) and derelicted/used
(because $\gr\omega \leq \gr1$).
Subsumption of usage annotations is essential to nearly all interesting choices
of $\Ann$.
However, in the toy example of exact usage counting using the set $\mathbb N$ of
annotations, we set the order to be just equality as a matter of simplicity.

For usage annotations $\gr r$ and $\gr s$, the inequality $\gr r \leq \gr s$
states that an assumption with annotation $\gr r$ can be used wherever an
assumption with annotation $\gr s$ is required.
A mnemonic is that $\gr r$ is less specific than $\gr s$.
The principle is reflected by the admissible subusaging rule, where the order
has been lifted from annotations to usage contexts.
The subusaging rule is a simple corollary of renaming, as given in
\cref{sec:ren-sub-lr}.

\[
  \begin{prooftree}
    \hypo{\grP \leq \grQ}
    \hypo{\grQ\gamma \vdash A}
    \infer2[Subuse]{\grP\gamma \vdash A}
  \end{prooftree}
\]

\paragraph{Tensor- and with-products}
Like intuitionistic linear logic (ILL), \name{} distinguishes tensor-products
($A \otimes B$) from with-products ($A \with B$).
Whereas in ILL, rules like $\otimes$-introduction involve splitting the
assumptions between the two subterms, in \name{}, this splitting is done by
choosing usage annotations for the premises which add up to the usage
annotations of the conclusion.
For example, we can derive $\vdash A \otimes B \multimap B \otimes A$ as
follows.
Notice that the assumption $A \otimes B$ is still present in all subderivations,
even after it has been ``used up''.
The only thing that stops us using the assumption again is that, for a general
choice of $\Ann$, we do not have $\gr0 \leq \gr1$ or $\gr1 \leq \gr1 + \gr1$.

\begin{small}
  \[
    \nabla \coloneqq
    \begin{prooftree}
      \infer0{\plr{\gr0\;\gr1\;\gr1} \leq
        \plr{\gr0\;\gr0\;\gr1} + \plr{\gr0\;\gr1\;\gr0}}
      \infer0{\plr{\gr0\;\gr0\;\gr1} \leq \plr{\gr0\;\gr0\;\gr1}}
      \infer1[Var]{\gr0\plr{A \otimes B}, \gr0A, \gr1B \vdash B}
      \infer0{\plr{\gr0\;\gr1\;\gr0} \leq \plr{\gr0\;\gr1\;\gr0}}
      \infer1[Var]{\gr0\plr{A \otimes B}, \gr1A, \gr0B \vdash A}
      \infer3[$\otimes$-I]%
      {\gr0\plr{A \otimes B}, \gr1A, \gr1B \vdash B \otimes A}
    \end{prooftree}
  \]

  \[
    \begin{prooftree}
      \infer0{\plr{\gr1} \leq \plr{\gr1} + \plr{\gr0}}
      \infer0{\plr{\gr1} \leq \plr{\gr1}}
      \infer1[Var]{\gr1\plr{A \otimes B} \vdash A \otimes B}
      \hypo{\nabla}
      \infer[no rule]1{\gr0\plr{A \otimes B}, \gr1A, \gr1B \vdash B \otimes A}
      \infer3[$\otimes$-E]{\gr1\plr{A \otimes B} \vdash B \otimes A}
      \infer1[$\multimap$-I]{\vdash A \otimes B \multimap B \otimes A}
    \end{prooftree}
  \]
\end{small}

\begin{example}
  Let $A \multimapboth B$ abbreviate
  $\plr{A \multimap B} \with \plr{B \multimap A}$.
  Then the following judgements hold for any partially ordered semiring.
  Derivations are left as an exercise to the reader.
  \begin{itemize}
    \item $\vdash A \oplus A \multimap A$
    \item $\vdash A \multimap A \with A$
    \item $\vdash A \oplus 0 \multimapboth A$
    \item $\vdash A \otimes 0 \multimapboth 0$
    \item $\vdash \oc\gr1A \multimapboth A$
    \item If $\gr r \leq \gr s$, then $\vdash \oc\gr rA \multimap \oc\gr sA$
  \end{itemize}
\end{example}

\begin{example}
  Let $\Ann \coloneqq (\mathbb N, =, 0, +, 1, \times)$, that is, specialise to
  the posemiring made of
  the usual semiring of natural numbers with ordering given by equality.
  Under this discipline, the usage constraints enforce a form of exact usage
  counting.
  The following judgements then hold.
  Derivations are left as an exercise to the reader.
  \begin{itemize}
    \item $\vdash \oc\gr2A \multimap A \otimes A$
    \item $\vdash \oc\gr5A \multimap \oc\gr2A \otimes \oc\gr3A$
  \end{itemize}
\end{example}

\subsection{Other posemirings}\label{sec:example-posemirings}

Now that we have seen the role of usage annotations in $\name$, I will give more
examples of posemirings for tracking interesting usage patterns.

\begin{example}\label{def:trivial-posemiring}
  The singleton set gives rise to a posemiring in a unique way.
  When the usage annotations of $\name$ are taken from this trivial posemiring,
  we recover a version of intuitionistic simply typed $\lambda$-calculus
  featuring redundant connectives $\otimes$ (equivalent to $\with$ in the pure
  setting) and $\oc{\gr*}$ (where $\oc{\gr*} A \simeq A$).
\end{example}

\begin{example}\label{def:monotonicity-posemiring}
  The \emph{monotonicity} posemiring is defined over the set of symbols
  $\{\gr{\wn\wn}, \gr{\uparrow\uparrow}, \gr{\downarrow\downarrow},
  \gr{\sim\sim}\}$.
  The idea is that each symbol represents the possible \emph{variance} of an
  input (free variable) with respect to some partial ordering on a semantic
  domain of elements.
  $\gr{\uparrow\uparrow}$ represents covariance (if that input goes up, the
  output goes up), $\gr{\downarrow\downarrow}$ represents contravariance
  (if that input goes \emph{down}, the output goes up), $\gr{\sim\sim}$ gives no
  guarantees (if that input remains constant, the output (trivially) goes up),
  and $\gr{\wn\wn}$ says that that input is irrelevant (whatever changes are
  made to that input, the output (trivially) goes up).

  I take $0 \coloneqq \gr{\wn\wn}$, $1 \coloneqq \gr{\uparrow\uparrow}$,
  and define the following operations:

  \makebox[\textwidth][s]{
    \begin{tabular}{c|cccc}
      $+$ & $\gr{\wn\wn}$ & $\gr{\uparrow\uparrow}$ & $\gr{\downarrow\downarrow}$ & $\gr{\sim\sim}$ \\ \hline
      $\gr{\wn\wn}$ & $\gr{\wn\wn}$ & $\gr{\uparrow\uparrow}$ & $\gr{\downarrow\downarrow}$  & $\gr{\sim\sim}$ \\
      $\gr{\uparrow\uparrow}$ & $\gr{\uparrow\uparrow}$ & $\gr{\uparrow\uparrow}$ & $\gr{\sim\sim}$ & $\gr{\sim\sim}$ \\
      $\gr{\downarrow\downarrow}$ & $\gr{\downarrow\downarrow}$ & $\gr{\sim\sim}$ & $\gr{\downarrow\downarrow}$ & $\gr{\sim\sim}$ \\
      $\gr{\sim\sim}$ & $\gr{\sim\sim}$  & $\gr{\sim\sim}$ & $\gr{\sim\sim}$ & $\gr{\sim\sim}$ \\
    \end{tabular}
    \begin{tabular}{c|cccc}
      $*$ & $\gr{\wn\wn}$ & $\gr{\uparrow\uparrow}$ & $\gr{\downarrow\downarrow}$ & $\gr{\sim\sim}$ \\ \hline
      $\gr{\wn\wn}$ & $\gr{\wn\wn}$ & $\gr{\wn\wn}$ & $\gr{\wn\wn}$  & $\gr{\wn\wn}$ \\
      $\gr{\uparrow\uparrow}$ & $\gr{\wn\wn}$ & $\gr{\uparrow\uparrow}$ & $\gr{\downarrow\downarrow}$ & $\gr{\sim\sim}$ \\
      $\gr{\downarrow\downarrow}$ & $\gr{\wn\wn}$ & $\gr{\downarrow\downarrow}$ & $\gr{\uparrow\uparrow}$ & $\gr{\sim\sim}$ \\
      $\gr{\sim\sim}$ & $\gr{\wn\wn}$  & $\gr{\sim\sim}$ & $\gr{\sim\sim}$ & $\gr{\sim\sim}$ \\
    \end{tabular}
    \begin{tikzpicture}[baseline]
      \node(omega) at (0,-1) {$\gr{\sim\sim}$};
      \node(0) [above left of=omega] {$\gr{\uparrow\uparrow}$};
      \node(1) [above right of=omega] {$\gr{\downarrow\downarrow}$};
      \node(qq) [above right of=0] {$\gr{\wn\wn}$};

      \draw (omega) -- (0);
      \draw (omega) -- (1);
      \draw (0) -- (qq);
      \draw (1) -- (qq);
    \end{tikzpicture}
  }

  Addition represents an intersection of guarantees.
  For example, if a variable is used covariantly in one subterm and
  contravariantly in another, we can only make the trivial guaratee represented
  by $\gr{\sim\sim}$.
  Multiplication is mainly interesting for multiplication by
  $\gr{\downarrow\downarrow}$, which flips the variance on any other annotation.
  As such, $\oc\gr{\downarrow\downarrow}A$ represents a contravariant $A$.
  The flipping (involutive) behaviour of $\gr{\downarrow\downarrow}$ lets us
  notice that $x$ is covariant in
  a term like $-(-x)$, where $-$ is a constant of type
  $\oc\gr{\downarrow\downarrow}\mathbb Z \multimap \mathbb Z$.

  A similar, but distinct, collection of modalities for monotonicity is given by
  \citet{Arntzenius19}.
\end{example}

\begin{example}\label{def:sensitivity-posemiring}
  The \emph{sensitivity} posemiring~\citep{reed10distance} is given by
  $(\mathbb R^+, \geq, \gr0, +, \gr1, \times)$, where $\mathbb R^+$ is the
  non-negative real numbers extended with $\gr\infty$ (distances), and the rest
  of the structure comes from the standard operations on real numbers (except
  that $\gr0 \times \gr\infty = \gr\infty \times \gr0 = \gr0$).
  Note that the order is reversed, making $\gr\infty$ coercible to any other
  annotation and anything coercible to $\gr0$.

  This posemiring can be used for sensitivity analysis, where we want to bound
  the effect of a perturbation on inputs in terms of some semantic notion of
  distance between values.
  An annotation $\gr r$ says that if that input is perturbed by at most $r$,
  then the output will change by at most $1$.
  An $\gr\infty$-annotated variable gives the very strong guarantee that any
  change in the input will make a minimal change to the output, while a
  $\gr0$-annotated variable provides no guarantee at all.
  Addition forbids general contraction, which would otherwise allow arbitrary
  finite blow-up of the effect of any non-$\gr0$-annotated variable.
  However, the ordering, with $\gr0$ at the top, means that we have general
  weakening, so the resulting sensitivity calculus has an affine flavour.
\end{example}

\section{Bunched connectives}\label{sec:lnd}
The typing rules of $\name$ presented in \cref{fig:lr} contain a lot of detail
and repeated patterns.
For example, nearly half of the rules include the premise
$\grR \leq \grP + \grQ$.
Also, the presence of usage annotations, which are often different in different
parts of a rule, means that we keep repeating the context.
Explicit contexts go against the style we established in \cref{sec:simple},
which is based
around being parametric in the context, so that substitution is agnostic to the
details of typing rules.

To encapsulate the repeated patterns and facilitate an implicit context style,
I introduce the \emph{bunched connectives} for premises.
These are inspired by bunched logic~\citep{oHP99}, and will not only be used for
stating the syntax, but will be used as an abstraction of common patterns in the
development of the metatheory.
The idea is to generalise the space between premises from Gentzen's natural
deduction to allow for any linear combination of usage annotations.
Among other things, this generalisation will allow us to distinguish between
$\with$-introduction and $\otimes$-introduction by a choice of connective:
either \emph{sharing} or \emph{separating} conjunction.
%Similar connectives, but with different interpretations, could be used to
%define other linear-like type theories, like DILL, but here I will focus on the
%usage annotation style.
These connectives are defined in \cref{fig:bunched} in Agda notation.

\begin{figure}
  %\begin{align*}
  %  \dot1\,\grR &\coloneqq 1 \\
  %  (T \dottimes U)\,\grR &\coloneqq T\,\grR \times U\,\grR \\
  %  (T \dotto U)\,\grR &\coloneqq T\,\grR \to U\,\grR \\
  %  I^*\,\grR &\coloneqq \grR \leq \gr0 \\
  %  (T \sep U)\,\grR &\coloneqq \Sigma \grP,\grQ.~\plr{\grR \leq \grP + \grQ}
  %                     \times T\,\grP \times U\,\grQ \\
  %  (\gr r \cdot T)\,\grR &\coloneqq \Sigma \grP.~\plr{\grR \leq \gr r\grP}
  %                     \times T\,\grP
  %\end{align*}
  Parameters:
  \ExecuteMetaData[\Bunchedtex]{BunchedParams}

  Connectives:
  \vspace{-1em}
  \begin{multicols}{2}
    \noindent\ExecuteMetaData[\Bunchedtex]{PointwiseUnit}
    \noindent\ExecuteMetaData[\Bunchedtex]{PointwiseConjunction}
    \columnbreak
    %\AgdaNoSpaceAroundCode{}
    \noindent\ExecuteMetaData[\Bunchedtex]{Entails}
    %\AgdaSpaceAroundCode{}
  \end{multicols}
  \vspace{-2em}
  \ExecuteMetaData[\Bunchedtex]{BunchedUnit}
  \ExecuteMetaData[\Bunchedtex]{BunchedConjunction}
  \ExecuteMetaData[\Bunchedtex]{BunchedImplication}
  \ExecuteMetaData[\Bunchedtex]{BunchedScaling}
  \caption{The bunched connectives}
  \label{fig:bunched}
\end{figure}

The bunched connectives are parametrised over two sets and three relations.
For syntax, the set \AgdaBound{A} will be \AgdaRecord{Ctx}, the type of
contexts, and \AgdaBound{R} will be \AgdaField{Ann}, the type of usage
annotations (scalars).
For the relations, the notation is meant to be suggestive, with
\AgdaBound{$\Gamma$}\AgdaSpace{}\AgdaBound{$\leq$0} typically stating that all
of the annotations in \AgdaBound{$\Gamma$} are less than or equal to $0$;
\AgdaBound{$\Gamma$}\AgdaSpace{}\AgdaBound{$\leq$[}\AgdaSpace{}%
\AgdaBound{$\Delta$}\AgdaSpace{}\AgdaBound{+}\AgdaSpace{}\AgdaBound{$\Theta$}%
\AgdaSpace{}\AgdaBound{]}
typically stating that \AgdaBound{$\Gamma$}, \AgdaBound{$\Delta$}, and
\AgdaBound{$\Theta$} all agree on their types but the usage context of
\AgdaBound{$\Gamma$} is less than or equal to the sum of the usage contexts of
\AgdaBound{$\Delta$} and \AgdaBound{$\Theta$}; and
\AgdaBound{$\Gamma$}\AgdaSpace{}\AgdaBound{$\leq$[}\AgdaSpace{}%
\AgdaBound{r}\AgdaSpace{}\AgdaBound{*$_l$}\AgdaSpace{}\AgdaBound{$\Delta$}%
\AgdaSpace{}\AgdaBound{]}
typically stating that \AgdaBound{$\Gamma$} and \AgdaBound{$\Delta$} agree on
their types but have the evident scaling relationship with \AgdaBound{r} on
their usage annotations.
I use the same symbols for the connectives both in Agda code and in otherwise
standard mathematical/logical notation.

The first two connectives are those we've already seen for intuitionistic
systems --- $\dot1$ and $\dottimes$.
The absence of premises is encoded by $\dot1$, while the space between premises
sharing a context is encoded by $\dottimes$.
As for implication, I temporarily avoid giving a $\dotto$ connectives, instead
fusing it together with $\forallb{-}$ to produce the \emph{set} of
context-preserving functions $T \rightarrowtriangle U$.
When we interpret a typing rule as a constructor of an inductive definition,
$\rightarrowtriangle$
interprets the horizontal line, reflecting the fact that the usage annotations
we start off with in the premises are those of the conclusion, corresponding to
a general principle of resource conservation.
The prototypical rules that use $\dot1$ and $\dottimes$ are the introduction
rules for $\top$ and $\with$, respectively.

\begin{align*}
  \begin{prooftree}
    \infer0{\grR\Gamma \vdash \top}
  \end{prooftree}
  &\quad\rightsquigarrow\quad
  \begin{prooftree}[comb]
    \hypo{\dot1}
    \infer1{\vdash \top}
  \end{prooftree}
  \\\\
  \begin{prooftree}
    \hypo{\grR\Gamma \vdash A}
    \hypo{\grR\Gamma \vdash B}
    \infer2{\grR\Gamma \vdash A \with B}
  \end{prooftree}
  &\quad\rightsquigarrow\quad
  \begin{prooftree}[comb]
    \hypo{\vdash A}
    \hypo{\dottimes}
    \hypo{\vdash B}
    \infer3{\vdash A \with B}
  \end{prooftree}
\end{align*}

The rest of the bunched connectives --- $I^*$, $\sep$, $\cdot$, and $\wand$ ---
involve linear decompositions of the usage annotations.
The three basic left semimodule operators --- zero, addition, and left-scaling
--- each get a bunched connective --- $I^*$, $\sep$, and $\gr r \cdot {}$,
respectively.
The prototypical typing rules for each of these three connectives are the
introduction rules for $I$, $\otimes$, and $\oc{\gr r}$, respectively.

\begin{align*}
  \begin{prooftree}
    \hypo{\grR \leq \gr0}
    \infer1{\grR\Gamma \vdash I}
  \end{prooftree}
  &\quad\rightsquigarrow\quad
  \begin{prooftree}[comb]
    \hypo{I^*}
    \infer1{\vdash I}
  \end{prooftree}
  \\\\
  \begin{prooftree}
    \hypo{\grP\Gamma \vdash A}
    \hypo{\grQ\Gamma \vdash B}
    \hypo{\grR \leq \grP + \grQ}
    \infer3{\grR\Gamma \vdash A \otimes B}
  \end{prooftree}
  &\quad\rightsquigarrow\quad
  \begin{prooftree}[comb]
    \hypo{\vdash A}
    \hypo{\sep}
    \hypo{\vdash B}
    \infer3{\vdash A \otimes B}
  \end{prooftree}
  \\\\
  \begin{prooftree}
    \hypo{\grP\Gamma \vdash A}
    \hypo{\grR \leq \gr r\grP}
    \infer2{\grR\Gamma \vdash \oc{\gr r} A}
  \end{prooftree}
  &\quad\rightsquigarrow\quad
  \begin{prooftree}[comb]
    \hypo{\gr r \cdot (\vdash A)}
    \infer1{\vdash \oc{\gr r} A}
  \end{prooftree}
\end{align*}

\subsection{$\name$ stated using bunched connectives}\label{sec:lr-bunched}
The full system \name{} is stated in terms of bunched connectives in
\cref{fig:lr-bunched}.
The bunched connectives also yield a reasonably concise definition of the Agda
data type of \name{} derivations, as seen in \cref{fig:lr-bunched-Agda}.

\begin{figure}
  \ebproofset{separation=0.75em}
  \begin{displaymath}
    \begin{prooftree}
      \hypo{\sqni A}
      \infer1[Var]{\vdash A}
    \end{prooftree}
  \end{displaymath}

  \begin{displaymath}
    \begin{matrix}
      \begin{prooftree}
        \hypo{I^*}
        \infer1[$I$-I]{\vdash I}
      \end{prooftree}
      &&
      \begin{prooftree}
        \hypo{\vdash I}
        \hypo{\sep}
        \hypo{\vdash C}
        \infer3[$I$-E]{\vdash C}
      \end{prooftree}
      \\\\
      \begin{prooftree}
        \hypo{\vdash A}
        \hypo{\sep}
        \hypo{\vdash B}
        \infer3[$\otimes$-I]{\vdash A \otimes B}
      \end{prooftree}
      &&
      \begin{prooftree}
        \hypo{\vdash A \otimes B}
        \hypo{\sep}
        \hypo{\gr1A, \gr1B \vdash C}
        \infer3[$\otimes$-E]{\vdash C}
      \end{prooftree}
      \\\\
      \begin{prooftree}
        \hypo{\gr1A \vdash B}
        \infer1[$\multimap$-I]{\vdash A \multimap B}
      \end{prooftree}
      &&
      \begin{prooftree}
        \hypo{\vdash A \multimap B}
        \hypo{\sep}
        \hypo{\vdash A}
        \infer3[$\multimap$-E]{\vdash B}
      \end{prooftree}
      \\\\
      \begin{prooftree}
        \hypo{\dot1}
        \infer1[$\top$-I]{\vdash \top}
      \end{prooftree}
      &&
      \textrm{(no $\top$-E)}
      \\\\
      \begin{prooftree}
        \hypo{\vdash A}
        \hypo{\dottimes}
        \hypo{\vdash B}
        \infer3[$\with$-I]{\vdash A \with B}
      \end{prooftree}
      &&
      \begin{prooftree}
        \hypo{\vdash A_0 \with A_1}
        \infer1[$\with$-E$_i$, for $i \in \{0,1\}$]{\vdash A_i}
      \end{prooftree}
      \\\\
      \textrm{(no $0$-I)}
      &&
      \begin{prooftree}
        \hypo{\vdash 0}
        \hypo{\sep}
        \hypo{\dot1}
        \infer3[$0$-E]{\vdash C}
      \end{prooftree}
      \\\\
      \begin{prooftree}
        \hypo{\vdash A_i}
        \infer1[$\oplus$-I$_i$, for $i \in \{0,1\}$]{\vdash A_0 \oplus A_1}
      \end{prooftree}
      &&
      \begin{prooftree}
        \hypo{\vdash A \oplus B}
        \hypo{\sep}
        \hypo{(\gr1A \vdash C}
        \hypo{\dottimes}
        \hypo{\gr1B \vdash C)}
        \infer5[$\oplus$-E]{\vdash C}
      \end{prooftree}
      \\\\
      \begin{prooftree}
        \hypo{\gr r \cdot (\vdash A)}
        \infer1[$\oc$-I]{\vdash \oc{\gr r} A}
      \end{prooftree}
      &&
      \begin{prooftree}
        \hypo{\vdash \oc{\gr r} A}
        \hypo{\sep}
        \hypo{\gr rA \vdash C}
        \infer3[$\oc$-E]{\vdash C}
      \end{prooftree}
    \end{matrix}
  \end{displaymath}
  \ebproofset{separation=1.5em}
  \caption{\name{} stated using bunched connectives}
  \label{fig:lr-bunched}
\end{figure}

\begin{figure}
  \ExecuteMetaData[\SpecificSyntaxtex]{Ty}
  \ExecuteMetaData[\SpecificSyntaxtex]{Bind}
  \ExecuteMetaData[\SpecificSyntaxtex]{lr}
  \caption{\name{} stated using bunched connectives in Agda}
  \label{fig:lr-bunched-Agda}
\end{figure}

\subsection{Connection with bunched logic}\label{sec:bunched-logic}
While we have seen a connection between the bunched connectives and the
connectives of $\name$, the two should not be confused.
In particular, the bunched connectives obey different laws to what we would
expect from linear logic.
For example, it would make sense to define a bunched connective $\dotplus$,
defined analogously to $\dottimes$.
This $\dotplus$ could be used to rephrase the introduction rules for $\oplus$.
We then have maps both ways between $T \dottimes (U \dotplus V)$ and
$(T \dottimes U) \dotplus (T \dottimes V)$, reminiscent of the distributivity of
additive connectives in bunched logic,
whereas linear logic only has a map from $(A \with B) \oplus (A \with C)$ to
$A \with (B \oplus C)$, and not a map the other way.
Looking at the interpretations, the connection with bunched logic makes sense.
Instead of the partial commutative monoid (often representing heaps) found in
standard semantics of bunched logic, we have a left semimodule of usage
contexts, which we are similarly interested in splitting and sharing between
various subterms.

From bunched logic, we would expect the Cartesian product $\dottimes$ to have an
internal hom.
In the intuitionistic case, $\dotto$ filled this role.
However, with usage contexts, it makes sense for open types to be presheaves
over the partial order of contexts under pointwise $\leq$ of usage annotations.
The family $T \dotto U$ does not satisfy the functoriality condition because of
the contravariance in the domain $T$.
Instead, as found in models of bunched logic, we would want a Kripke function
space, like
$\lambda\Gamma.~\forall \Gamma' \leq \Gamma.~T\,\Gamma' \to U\,\Gamma'$.
However, I do not make use of such a connective.

The separating conjunction $\sep$ can be seen as a decategorified version of Day
convolution~\citep{Day70}.
It also resembles the use of ternary frames in semantics of non-distributive
logics~\citep[chapter 12]{Restall1999}.

\subsection{Operations on bunched connectives}\label{sec:bunched-op}
To manipulate terms and other open types defined using bunched connectives, we
need the zero, addition, and multiplication relations to satisfy some laws.
For example, to achieve a symmetry map $T \sep U \rightarrowtriangle U \sep T$,
we need addition to satisfy the commutativity law
$\forall x,y,z : A.~x \leq y + z \to x \leq z + y$.

For all uses of bunched connectives in this thesis, the carrier set $A$ will
form a partial order --- for example, with contexts, the order is given by the
pointwise order on the usage vectors.
We then consider the category whose objects are posets and whose morphisms are
relations $R : A \rel B$ satisfying the contravariant-covariant law
$\forall x,x',y,y'.~x' \leq x \to y \leq y' \to xRy \to x'Ry'$.
This category can be given the usual monoidal product of relations, which is
the pointwise product of posets on objects.
Then, we will always expect zero and addition to together form a cocommutative
comonoid in this category.
With this structure, we can get the following equivalences and functions.

\begin{mathpar}
  I^* \sep T \leftrightarrowtriangle T \and
  T \sep I^* \leftrightarrowtriangle T \and
  (T \sep U) \sep V \leftrightarrowtriangle T \sep (U \sep V) \and
  T \sep U \leftrightarrowtriangle U \sep T \and
  (T \wand U) \sep T \rightarrowtriangle U \and
  I^* \wand T \leftrightarrowtriangle T \and
  (T \sep U) \wand V \leftrightarrowtriangle T \wand (U \wand V)
\end{mathpar}

I do not use algebraic properties of multiplication in conjunction with
manipulation of bunched connectives in this thesis, but we could expect
scalar multiplication
to add a comodule structure over cosemiring $R$ to the cocommutative comonoid
given by zero and addition.

\section{Additions to and variations of \name{}}\label{sec:variant}
Based on an intuitive understanding of ``usage'', recursion introduces a new
phenomenon relative to the forms of programs we have seen so far:
terms can be used an unbounded number of times.
For example, notice the following reduction in Agda.

\missingfigure{\texttt{foldr \_+\_ 0 (1 :: 2 :: 3 :: []) = 1 + (2 + (3 + 0))}}

The function \AgdaFunction{\_+\_} has been copied into 3 different places in
the running of the program.
This copying is despite no type telling us that \AgdaFunction{\_+\_} would be
used 3 times (both \verb|[1,2,3]| and \verb|[2,3]| have type
\AgdaDatatype{List}\AgdaSpace{}\AgdaDatatype{$\mathbb N$}, despite the
corresponding folds using \AgdaFunction{\_+\_} a different number of times).
As such, when checking an application of \AgdaFunction{foldr}, we need check
that we can use its functional argument (\AgdaFunction{\_+\_} in this case) an
arbitrary number of times.
If we were to fix $\Ann$ as the $\{\gr0, \gr1, \gr\omega\}$ posemiring, then
wrapping the type of the functional argument in $\oc\gr\omega$ would suffice.
However, we want to remain generic in the choice of semiring.

I propose the following additions to \name{} to support a broad class of
inductive types.
I define strictly positive functors syntactically, with the only notable
restriction being not being allowed to use the type variable $X$ in the domain
of a function type.
I then add least fixed points of such strictly positive functors to the syntax
of types.

\begin{align*}
  U &\Coloneqq A \multimap (-) \mid \oc\gr r(-) \\
  {\odot} &\Coloneqq {\otimes} \mid {\oplus} \mid {\with} \\
  F[X], G[X] &\Coloneqq X \mid A \mid U(F[X]) \mid F[X] \odot G[X] \\
  A &\Coloneqq \cdots \mid \mu X.~F[X]
\end{align*}

\begin{example}
  We may define $\mathrm{List}_A \coloneqq \mu X.~I \oplus \plr{A \otimes X}$.
\end{example}

In the typing rules, introduction of an inductive type is standard.
For the elimination rule, we follow a similar pattern to other pattern-matching
rules --- $\oplus$-E, $\otimes$-E, and $\oc$-E --- by splitting the context
and typing the eliminand in one half ($\grP$).
We type the continuation in the other half, but because the continuation may
be used multiple times, and in a modal context, we require that $\grQ$ is
preserved by all linear operations.

\begin{displaymath}
  \begin{prooftree}
    \hypo{\grR\gamma \vdash F[\mu X.~F[X]]}
    \infer1[$\mu$-I]{\grR\gamma \vdash \mu X.~F[X]}
  \end{prooftree}
\end{displaymath}
\begin{displaymath}
  \begin{prooftree}
    \hypo{\grR \leq \grP + \grQ}
    \hypo{\grP\gamma \vdash \mu X.~F[X]}
    \hypo{%
      \begin{matrix*}[l]
        \grQ \leq \gr0 \\
        \grQ \leq \grQ + \grQ \\
        \forall \gr r.~\grQ \leq \gr r\grQ
      \end{matrix*}%
    }
    \hypo{\grQ\gamma, \gr1F[C] \vdash C}
    \infer4[$\mu$-E]{\grR\gamma \vdash C}
  \end{prooftree}
\end{displaymath}

\begin{example}\label{thm:list-rules}
  For lists, we can derive the following introduction and elimination rules
  (with usage constraints omitted for brevity when obvious).

  \begin{align*}
    \begin{prooftree}
      \hypo{\grR \leq \gr0}
      \infer1[$I$-I]{I}
      \infer1[$\oplus$-I$_0$]%
      {\grR\gamma \vdash I \oplus \plr{A \otimes \mathrm{List}_A}}
      \infer1[$\mu$-I]{\grR\gamma \vdash \mathrm{List}_A}
    \end{prooftree}
    &&
    \begin{prooftree}
      \hypo{\grR \leq \grP + \grQ}
      \hypo{\grP\gamma \vdash A}
      \hypo{\grQ\gamma \vdash \mathrm{List}_A}
      \infer3[$\otimes$-I]{\grR\gamma \vdash A \otimes \mathrm{List}_A}
      \infer1[$\oplus$-I$_1$]%
      {\grR\gamma \vdash I \oplus \plr{A \otimes \mathrm{List}_A}}
      \infer1[$\mu$-I]{\grR\gamma \vdash \mathrm{List}_A}
    \end{prooftree}
  \end{align*}
  \begin{displaymath}
    \begin{prooftree}
      \hypo{\grP\gamma \vdash \mathrm{List}_A}
      \infer0[Var]{\gr0\gamma, \gr1\plr{I \oplus \plr{A \otimes C}}
        \vdash I \oplus \plr{A \otimes C}}
      \hypo{\nabla^n}
      \hypo{\nabla^c}
      \infer3[$\oplus$-E]{\grQ\gamma, \gr1\plr{I \oplus \plr{A \otimes C}}
        \vdash C}
      \infer2[$\mu$-E]{\grR\gamma \vdash C}
    \end{prooftree}
  \end{displaymath}
  \begin{align*}
    \textrm{where }\nabla^n &\coloneqq
    \begin{prooftree}
      \infer0[Var]{\gr0\gamma, \gr1I \vdash I}
      \hypo{\grQ\gamma \vdash C}
      \infer1[Wk]{\grQ\gamma, \gr0I \vdash C}
      \infer2[$I$-E]{\grQ\gamma, \gr1I \vdash C}
      \infer1[Wk]{\grQ\gamma, \gr0\plr{I \oplus \plr{A \otimes C}}, \gr1I
        \vdash C}
    \end{prooftree}
    \\\\
    \textrm{and }\nabla^c &\coloneqq
    \begin{prooftree}
      \infer0[Var]{\gr0\gamma, \gr1\plr{A \otimes C} \vdash A \otimes C}
      \hypo{\grQ\gamma, \gr1A, \gr1C \vdash C}
      \infer1[Wk]{\grQ\gamma, \gr0\plr{A \otimes C}, \gr1A, \gr1C \vdash C}
      \infer2[$\otimes$-E]{\grQ\gamma, \gr1\plr{A \otimes C} \vdash C}
      \infer1[Wk]%
      {\grQ\gamma, \gr0\plr{I \oplus \plr{A \otimes C}}, \gr1\plr{A \otimes C}
        \vdash C}
    \end{prooftree}
  \end{align*}
\end{example}

Following \cref{sec:lnd}, I want to turn the ad hoc constraints on $\grP$,
$\grQ$, and $\grR$ into the result of some premise combinators.
To do this, I introduce a new combinator $\Box^{0{+}{\times}}$ defined below,
along with the resulting implicit-context typing rules.

\begin{align*}
  \plr{\Box^{0{+}{\times}}\,T}\grR \coloneqq
  \plr{\grR \leq \gr0} \times \plr{\grR \leq \grR + \grR} \times
  \plr{\forall \gr r.~\grR \leq \gr r\grR} \times T\,\grR
\end{align*}

\begin{align*}
  \begin{prooftree}[comb]
    \hypo{\vdash F[\mu X.~F[X]]}
    \infer1[$\mu$-I]{\vdash \mu X.~F[X]}
  \end{prooftree}
  &&
  \begin{prooftree}[comb]
    \hypo{\vdash \mu X.~F[X]}
    \hypo{\sep}
    \hypo{\Box^{0{+}{\times}}\plr{\gr1F[C] \vdash C}}
    \infer3[$\mu$-E]{\vdash C}
  \end{prooftree}
\end{align*}

\begin{example}
  We can state the rules for lists derived in \cref{thm:list-rules} as follows.
  \begin{align*}
    \begin{prooftree}[comb]
      \hypo{I^*}
      \infer1{\vdash \mathrm{List}_A}
    \end{prooftree}
    &&
    \begin{prooftree}[comb]
      \hypo{\vdash A}
      \hypo{\sep}
      \hypo{\vdash \mathrm{List}_A}
      \infer3{\vdash \mathrm{List}_A}
    \end{prooftree}
    &&
    \begin{prooftree}[comb]
      \hypo{\vdash \mathrm{List}_A}
      \hypo{\sep}
      \hypo{\Box^{0{+}{\times}}
        \plr{\vdash C\hskip0.75em\dottimes\hskip0.75em\gr1A, \gr1C \vdash C}}
      \infer3{\vdash C}
    \end{prooftree}
  \end{align*}
\end{example}

\section{Translation to and from existing systems}\label{sec:translation}
\newcommand\instDILL{\gr{01\omega}}
\newcommand\instPD{\gr{01\Box}}

\newcommand\unused{\gr0}
\newcommand\true{\gr1}
\newcommand\valid{\gr\Box}

A motivating reason to consider the system $\name$ is that
instances of it correspond to previously studied systems.
In this section, I present translations from \name{} to Dual Intuitionistic
Linear Logic~\citep{Barber1996} and the modal system of \citet{judgmental},
and vice versa.
These translations are not mechanised, as part of the reason for developing
\name{} was to avoid mechanising these systems directly.
We cannot prove that the translations form an equivalence, because I have not
written down an equational theory for \name{}, but I expect this to be easy
enough to do.

\subsection{Dual Intuitionistic Linear Logic}\label{sec:trans-dill}

Dual Intuitionistic Linear Logic is a particular formulation of intuitionistic
linear logic introduced by \citet{Barber1996}.
Its key feature, which simplifies the metatheory of linear logic, is the use of
separate contexts for linear and intuitionistic free variables.
Here I show that DILL is a fragment of the instantiation of \name{} at the
linearity semiring $\{\gr0, \gr1, \gr\omega\}$.

The types of DILL are the same as the types of \name, except for the
restriction of $\oc\gr{r}$ to just $\oc\gr\omega$.
I will write the latter simply as $\oc$ when it appears in DILL\@.
I add sums and with-products to the calculus of \cite{Barber1996}, with the
obvious rules (stated fully in \cref{fig:dill}).
These additive type formers present no additional difficulty to the translation.

\begin{figure}
  \begin{mathpar}
    \inferrule*[right=Int-Ax]{ }
    {\gamma, A; \cdot \vdash A}
    \and
    \inferrule*[right=Lin-Ax]{ }
    {\gamma; A \vdash A}
    \and
    \inferrule*[right=$I$-I]{ }
    {\gamma; \cdot \vdash I}
    \and
    \inferrule*[right=$I$-E]
    {\gamma; \delta_1 \vdash I \\ \gamma; \delta_2 \vdash A}
    {\gamma; \delta_1, \delta_2 \vdash A}
    \and
    \inferrule*[right=$\otimes$-I]
    {\gamma; \delta_1 \vdash A \\ \gamma, \delta_2 \vdash B}
    {\gamma; \delta_1, \delta_2 \vdash A \otimes B}
    \and
    \inferrule*[right=$\otimes$-E]
    {\gamma; \delta_1 \vdash A \otimes B \\ \gamma; \delta_2, A, B \vdash C}
    {\gamma; \delta_1, \delta_2 \vdash C}
    \and
    \inferrule*[right=$\multimap$-I]
    {\gamma; \delta, A \vdash B}
    {\gamma; \delta \vdash A \multimap B}
    \and
    \inferrule*[right=$\multimap$-E]
    {\gamma; \delta_1 \vdash A \multimap B \\ \gamma; \delta_2 \vdash A}
    {\gamma; \delta_1, \delta_2 \vdash B}
    \and
    \inferrule*[right=$\oc$-I]
    {\gamma; \cdot \vdash A}
    {\gamma; \cdot \vdash \oc A}
    \and
    \inferrule*[right=$\oc$-E]
    {\gamma; \delta_1 \vdash \oc A \\ \gamma, A; \delta_2 \vdash B}
    {\gamma; \delta_1, \delta_2 \vdash B}
    \and
    \inferrule*[right=$\top$-I]{ }
    {\gamma; \delta \vdash \top}
    \and
    \inferrule*[right=$\with$-I]
    {\gamma; \delta \vdash A \\ \gamma, \delta \vdash B}
    {\gamma; \delta \vdash A \with B}
    \and
    \inferrule*[right=$\with$-E$_i$]
    {\gamma; \delta \vdash A_0 \with A_1}
    {\gamma; \delta \vdash A_i}
    \and
    \inferrule*[right=$0$-E]
    {\gamma; \delta_1 \vdash 0}
    {\gamma; \delta_1, \delta_2 \vdash A}
    \and
    \inferrule*[right=$\oplus$-I$_i$]
    {\gamma; \delta \vdash A_i}
    {\gamma; \delta \vdash A_0 \oplus A_1}
    \and
    \inferrule*[right=$\oplus$-E]
    {
      \gamma; \delta_1 \vdash A \oplus B \\
      \gamma; \delta_2, A \vdash C \\
      \gamma; \delta_2, B \vdash C
    }
    {\gamma; \delta_1, \delta_2 \vdash C}
  \end{mathpar}
  \label{fig:dill}
  \caption{The rules of DILL, extended with additive connectives}
\end{figure}

\begin{figure}
  \begin{mathpar}
    \begin{eqns}
      \mathrm{DILL} &\hookrightarrow& \name_\instDILL \\
      Y &\mapsto& \iota_Y \\
      I &\mapsto& I \\
      A \otimes B &\mapsto& A \otimes B \\
      A \multimap B &\mapsto& A \multimap B \\
      \oc A &\mapsto& \oc{\gr\omega}{A} \\
      0 &\mapsto& 0 \\
      A \oplus B &\mapsto& A \oplus B \\
      \top &\mapsto& \top \\
      A \with B &\mapsto& A \with B
    \end{eqns}
    \and
    \begin{eqns}
      \mathrm{PD} &\hookrightarrow& \name_\instPD \\
      Y &\mapsto& \iota_Y \\
      \top &\mapsto& I \\
      A \wedge B &\mapsto& A \with B \\
      A \supset B &\mapsto& A \multimap B \\
      \Box A &\mapsto& \oc{\valid}{A} \\
      \bot &\mapsto& 0 \\
      A \vee B &\mapsto& A \oplus B
    \end{eqns}
  \end{mathpar}
  \label{fig:dill-pd}
  \caption{Embedding of DILL and PD types into \name}
\end{figure}

\begin{proposition}[DILL $\to$ \name]
  Given a DILL derivation of $\gamma; \delta \vdash A$, we can produce a
  $\name_{\instDILL}$ derivation of
  $\gr\omega\gamma, \gr1\delta \vdash A$.
\end{proposition}
\begin{proof}
  By induction on the derivation.
  We have $\gr\omega \leq \gr0$, which allows us to discard
  intuitionistic variables at the var rules, and both
  $\gr1 \leq \gr1$ and $\gr\omega \leq \gr1$, which allow
  us to use both linear and intuitionistic variables.

  Weakening is used when splitting linear variables between two premises.
  For example, \TirName{$\otimes$-I} in DILL is as follows.
  \[
    \inferrule*[right=$\otimes$-I]
    {\gamma; \delta_t \vdash t : A \\ \gamma; \delta_u \vdash u : B}
    {\gamma; \delta_t, \delta_u \vdash t \otimes u : A \otimes B}
  \]
  From this, our new derivation is as follows.
  \[
    \inferrule*[right=$\otimes$-I]
    {
      \inferrule*[right=Weak]
      {\mathit{ih}_t \\\\
        \gr\omega\gamma, \gr1\delta_t
        \vdash M_t : A}
      {\gr\omega\gamma, \gr1\delta_t,
        \gr0\delta_u
        \vdash M_t : A}
      \\
      \inferrule*[right=Weak]
      {\mathit{ih}_u \\\\
        \gr\omega\gamma, \gr1\delta_u
        \vdash M_u : A}
      {\gr\omega\gamma, \gr0\delta_t,
        \gr1\delta_u
        \vdash M_u : A}
    }
    {\gr\omega\gamma, \gr1\delta_t,
      \gr1\delta_u
      \vdash \plr{M_t, M_u} : A \otimes B}
  \]
\end{proof}

When translating from \name{} to DILL, we first coerce the \name{} derivation
to be in a form easily amenable to translation into DILL\@.
An example of a \name{} derivation with no direct translation into DILL is the
following.
In DILL terms, the intuitionistic variable of the conclusion becomes a linear
variable in the premises.
Such a move is admissible in DILL, but does not come naturally.

\[
  \inferrule*[right=$\otimes$-I]
  {
    \inferrule*[right=var]{ }
    {\gr1A : x \vdash A}
    \\
    \inferrule*[right=var]{ }
    {\gr1A : x \vdash A}
    \\
    \gr\omega \leq \gr1 + \gr1
  }
  {\gr\omega A : x \vdash A \otimes A}
\]

To avoid such situations, and therefore manipulations on DILL derivations, I
show that all $\name_{\instDILL}$ derivations can be made in \emph{bottom-up}
style.
In bottom-up style, the algebraic facts we make use of are dictated by making
most general choices based on the conclusions of rules.
Bottom-up style corresponds to a (non-deterministic) form of
\emph{usage checking}, and the following lemma can be understood as saying
that that form of usage checking is sufficiently general.

\begin{definition}
  A derivation is said to be \emph{$\instDILL$-bottom-up} if only the following
  facts about addition and multiplication are used, and all proofs of
  inequalities not at leaves are by reflexivity (i.e, not using the facts that
  $\gr\omega \leq \gr0$ and $\gr\omega \leq \gr1$).

  \makebox[\textwidth][s]{
    \begin{tabular}{c|ccc}
      $+$ & $\gr0$ & $\gr1$ & $\gr\omega$ \\ \hline
      $\gr0$ & $\gr0$ & $\gr1$ & - \\
      $\gr1$ & $\gr1$ & - & - \\
      $\gr\omega$ & - & - & $\gr\omega$ \\
    \end{tabular}
    \begin{tabular}{c|ccc}
      $*$ & $\gr0$ & $\gr1$ & $\gr\omega$ \\ \hline
      $\gr0$ & - & - & $\gr0$ \\
      $\gr1$ & $\gr0$ & $\gr1$ & $\gr\omega$ \\
      $\gr\omega$ & $\gr0$ & - & $\gr\omega$ \\
    \end{tabular}
  }
\end{definition}

Bottom-up style enforces that whenever we split a context into two (for
example, in the rule \TirName{$\otimes$-I}) all unused variables in the
conclusion stay unused in the premises, intuitionistic variables stay
intuitionistic, and linear variables go either left or right.
Multiplication is only used in the rule \TirName{$\oc\gr{r}$-I}, at which point
both the result and left argument are available.
Here, the bottom-up style enforces that linear variables never appear in the
premise of \TirName{$\oc{\gr\omega}$-I}.

\begin{lemma}
  Every $\name_{\instDILL}$ derivation can be translated into a bottom-up
  $\name_{\instDILL}$ derivation.
\end{lemma}
\begin{proof}
  By induction on the shape of the derivation.
  When we come across a non-bottom-up use of addition, it must be that the
  corresponding variable in the conclusion has annotation $\gr\omega$.
  By subusaging, we can give this variable annotation $\gr\omega$ in
  the premises too, before translating the subderivations to bottom-up
  style.
  A similar argument applies to uses of multiplication, remembering that both
  the left argument and result are fixed.
\end{proof}

\begin{proposition}[\name{} $\to$ DILL]
  Given a $\name_{\instDILL}$ derivation of
  $\gr\omega\gamma, \gr1\delta,
  \gr0\theta \vdash A$ which contains only types expressible
  in DILL, we can produce a DILL derivation of $\gamma; \delta \vdash A$.
\end{proposition}
\begin{proof}
  By induction on the derivation having been translated to bottom-up form.

  In the case of \TirName{var}, all of the unused variables have annotation
  greater than $\gr0$, i.e., $\gr0$ or $\gr\omega$.
  Those annotated $\gr0$ are absent from the DILL derivation, and those
  annotated $\gr\omega$ are in the intuitionistic context.
  The used variable is annotated either $\gr1$ or $\gr\omega$.
  In the first case, we use \TirName{Lin-Ax}, and in the second case,
  \TirName{Int-Ax}.

  All binding of variables in \name{} maps directly onto DILL\@.

  Because we translated to bottom-up form, additions, as seen in, for example,
  the \TirName{$\otimes$-I} rule, can be handled straightforwardly.
  Any intuitionistic variables in the conclusion correspond to intuitionistic
  variables in both premises.
  Any linear variables in the conclusion correspond to a linear variable in
  exactly one of the premises, and is absent in the other premise.

  The only remaining rule is \TirName{$\oc\gr{r}$-I}, of which we only cover
  \TirName{$\oc{\gr\omega}$-I} (the other two targeting types not found
  in DILL).
  In this case, we know that every variable in the conclusion is annotated
  either $\gr0$ or $\gr\omega$, and every variable in the premise is
  annotated the same way.
  This corresponds exactly to the restrictions of DILL's \TirName{$\oc$-I}.
\end{proof}

\subsection{Pfenning Davies}\label{sec:trans-pd}

The translation to and from the modal system of Pfenning and Davies
\cite{judgmental} (henceforth \emph{PD}) is similar to the translation to and
from DILL\@.
I present my variant of PD, again adding some common connectives, in
\cref{fig:pd}
The main difference is the algebra at which \name{} is instantiated.

\begin{figure}
  \begin{mathpar}
    \inferrule*[right=hyp]{ }
    {\gamma; \delta, A\;\mathit{true} \vdash A\;\mathit{true}}
    \and
    \inferrule*[right=hyp*]{ }
    {\gamma, A\;\mathit{valid}; \delta \vdash A\;\mathit{true}}
    \and
    \inferrule*[right=$\supset$I]
    {\gamma; \delta, A\;\mathit{true} \vdash B\;\mathit{true}}
    {\gamma; \delta \vdash A \supset B\;\mathit{true}}
    \and
    \inferrule*[right=$\supset$E]
    {
      \gamma; \delta \vdash A \supset B\;\mathit{true} \\
      \gamma; \delta \vdash A\;\mathit{true}
    }
    {\gamma; \delta \vdash B\;\mathit{true}}
    \and
    \inferrule*[right=$\Box$I]
    {\gamma; \cdot \vdash A\;\mathit{true}}
    {\gamma; \delta \vdash \Box A\;\mathit{true}}
    \and
    \inferrule*[right=$\Box$-E]
    {
      \gamma; \delta \vdash \Box A\;\mathit{true} \\
      \gamma, A\;\mathit{valid}; \delta \vdash B\;\mathit{true}
    }
    {\gamma; \delta \vdash B\;\mathit{true}}
    \and
    \inferrule*[right=$\top$-I]{ }
    {\gamma; \delta \vdash \top\;\mathit{true}}
    \and
    \inferrule*[right=$\wedge$-I]
    {
      \gamma; \delta \vdash A\;\mathit{true} \\
      \gamma, \delta \vdash B\;\mathit{true}
    }
    {\gamma; \delta \vdash A \wedge B\;\mathit{true}}
    \and
    \inferrule*[right=$\wedge$-E$_i$]
    {\gamma; \delta \vdash A_0 \wedge A_1\;\mathit{true}}
    {\gamma; \delta \vdash A_i\;\mathit{true}}
    \and
    \inferrule*[right=$\bot$-E]
    {\gamma; \delta \vdash \bot\;\mathit{true}}
    {\gamma; \delta \vdash A\;\mathit{true}}
    \and
    \inferrule*[right=$\vee$-I$_i$]
    {\gamma; \delta \vdash A_i\;\mathit{true}}
    {\gamma; \delta \vdash A_0 \vee A_1\;\mathit{true}}
    \and
    \inferrule*[right=$\vee$-E]
    {
      \gamma; \delta \vdash A \vee B\;\mathit{true} \\
      \gamma; \delta, A \vdash C\;\mathit{true} \\
      \gamma; \delta, B \vdash C\;\mathit{true}
    }
    {\gamma; \delta \vdash C\;\mathit{true}}
  \end{mathpar}
  \label{fig:pd}
  \caption{The rules of PD, extended with several standard connectives}
\end{figure}

\begin{definition}
  Let $\instPD$ denote the following semiring on the partially ordered set
  $\{\valid \triangleleft \true \triangleleft \unused\}$.
  \begin{itemize}
    \item $0 := \unused$.
    \item $+$ is the meet ($\wedge$) according to the subusaging order.
    \item $1 := \true$.
    \item
      \begin{tabular}{c|ccc}
        $*$ & $\unused$ & $\true$ & $\valid$ \\ \hline
        $\unused$ & $\unused$ & $\unused$ & $\unused$ \\
        $\true$ & $\unused$ & $\true$ & $\valid$ \\
        $\valid$ & $\unused$ & $\valid$ & $\valid$ \\
      \end{tabular}
  \end{itemize}
\end{definition}

The $\unused$ annotation plays only a formal role in this example.
Meanwhile, $\true$ and $\valid$ correspond to the judgement forms
$\mathit{true}$ and $\mathit{valid}$ from PD\@.
Addition being the meet makes it idempotent.
Furthermore, it gives us that $\true + \valid = \valid$ --- if somewhere we
require an assumption to be true, and elsewhere require it to be valid, then
ultimately it must be valid (from which we can deduce that it is true).
Multiplication is designed to make $\oc{\valid}$ act like PD's $\Box$.
In particular, $\valid * \valid = \valid$ says that the valid assumptions are
available before and after \TirName{$\oc{\valid}$-I}, whereas
$\valid * \true = \valid$ says that valid assumptions in the conclusion can be
weakened to true assumptions in the premise.
The latter fact does not appear in PD, and will be excluded from
\emph{bottom-up} derivations.

To keep my notation consistent with that of DILL, I swap the roles of
$\gamma$ and $\delta$ in PD compared to what they were in the original paper.
Thus, my PD judgements are of the form $\gamma; \delta \vdash A~\mathit{true}$,
where $\gamma$ contains valid assumptions and $\delta$ contains true
assumptions.

\begin{proposition}[PD $\to$ \name]
  Given a PD derivation of $\gamma; \delta \vdash t : A~\mathit{true}$, we can
  produce a $\name_{\instPD}$ derivation of
  $\valid\gamma, \true\delta \vdash A$.
\end{proposition}
\begin{proof}
  By induction on the PD derivation.
  Most PD rules have direct $\name$ counterparts, noting that variables of any
  annotation can be discarded and duplicated because we have both
  $\gr r \leq \gr 0$ and
  $\gr r \leq \gr r + \gr r$ for all
  $\gr r$.

  Care must be taken with the \TirName{$\Box$I} rule.
  We have, from the induction hypothesis, a $\name$ derivation of
  $\gr\Box\gamma \vdash A$.
  By \TirName{$\oc\valid$-I}, we have
  $\gr\Box\gamma \vdash \oc\valid A$.
  To get the desired conclusion, we must use \TirName{Weak} to get
  $\gr\Box\gamma, \gr\unused\delta \vdash \oc\valid A$, and
  then \TirName{Subuse} on the variables we just introduced (noting that
  $\true \leq \unused$) to get
  $\gr\Box\gamma, \gr\true\delta \vdash \oc\valid A$.
\end{proof}

For translating from $\name_{\instPD}$ to PD, I introduce a similar notion of
\emph{bottom-up} derivations as I did for DILL\@.
Every $\name_{\instPD}$ derivation can be translated into bottom-up style, and
then be directly translated into PD\@.

\begin{definition}
  A derivation is said to be \emph{$\instPD$-bottom-up} if only the following
  facts about addition and multiplication are used, and all proofs of
  inequalities not at leaves are by reflexivity.

  \makebox[\textwidth][s]{
    \begin{tabular}{c|ccc}
      $+$ & $\unused$ & $\true$ & $\valid$ \\ \hline
      $\unused$ & $\unused$ & - & - \\
      $\true$ & - & $\true$ & - \\
      $\valid$ & - & - & $\valid$ \\
    \end{tabular}
    \begin{tabular}{c|ccc}
      $*$ & $\unused$ & $\true$ & $\valid$ \\ \hline
      $\unused$ & - & - & $\unused$ \\
      $\true$ & $\unused$ & $\true$ & $\valid$ \\
      $\valid$ & $\unused$ & - & $\valid$ \\
    \end{tabular}
  }
\end{definition}

\begin{lemma}
  Every $\name_{\instPD}$ derivation can be translated into a bottom-up
  $\name_{\instPD}$ derivation.
\end{lemma}
\begin{proof}
  By induction on the shape of the derivation.
  Given that addition is a meet, it is clear that any non-bottom-up uses of
  addition come from one of the arguments being greater than the result.
  Therefore, it is safe to make this argument smaller in the corresponding
  premise (via subusaging), before translating that subderivation.
  For multiplication, again, there is always a lesser value of the right
  argument that will take us from a non-bottom-up fact to a bottom-up fact with
  the same left argument and result.
\end{proof}

\begin{proposition}[\name{} $\to$ PD]
  Given a $\name_{\instPD}$ derivation of
  $\valid\gamma, \true\delta, \unused\theta
  \vdash M : A$ which does not contain types using $\oc{\unused}{}$ or
  $\oc{\true}$, we can produce a PD derivation of
  $\gamma; \delta \vdash A~\mathit{true}$.
\end{proposition}
\begin{proof}
  We translate away tensor products and tensor units using
  \cref{thm:top-meet}, and translate the resulting derivation to bottom-up
  form.
  The proof proceeds by induction on the resulting derivation in the obvious
  way.
\end{proof}

As mentioned in \cref{sec:alt}, the system of \citet{AbelBernardy2020}
is unable to embed PD in this way, as it would prove
$\Box(A \wedge B) \to \Box A \wedge \Box B$, where PD and $\name$ do not.
In fact, this example shows that, even when weakening and contraction are
admissible, with- and tensor-products are distinct in their system in the
presence of modalities.

\section{Addendum: (lack of) partiality}\label{sec:part}
As we have seen, the way additive and multiplicative rules are
realised algebraically is related to models of separation logic.
Models of separation logic typically use \emph{partial} commutative monoids to
model a heap, so it is tempting to generalise the commutative monoid of
addition in our semirings to a \emph{partial} commutative monoid.
However, we find that the most natural notion of \emph{partial semiring} is
degenerate, in the sense that all partial semirings are actually (total)
semirings.

Recall that a commutative monoid (or commutative monoid object) can be
defined in any symmetric monoidal category.
A partial commutative monoid is exactly a commutative monoid object in the
category of sets and partial functions with the usual monoidal product given
by pairing of objects and morphisms (like the Cartesian product in $\Set$).
However, semirings need a Cartesian category in order to state the interaction
equations between addition and multiplication.
While the category of sets and partial functions is not Cartesian, the
standard way to manufacture a Cartesian category out of a symmetric monoidal
category $\mathcal C$ is to take the category of cocommutative comonoids
$\mathrm{CComon}(\mathcal C)$.
Intuitively, the cocommutative comonoid structure equips the underlying
object $M$ with a \emph{delete} map $\eta : M \to I$ and a \emph{duplicate}
map $\delta : M \to M \otimes M$ which are coherent with respect to each other.
All morphisms in $\mathrm{CComon}(\mathcal C)$ must respect $\eta$ and
$\delta$; in particular, both addition and multiplication must separately
form bimonoids in $\mathcal C$ together with the cocommutative comonoid.

The distributivity laws of semirings are stated below.
I include these to show that the cocommutative comonoids of a monoidal category
give enough structure to state these laws.
The other laws --- that all morphisms respect $\eta$ and $\delta$, that addition
forms a commutative monoid, and that multiplication forms a monoid --- are
standard in symmetric monoidal category theory.

\[
  \begin{tikzpicture}[baseline]
    \path
    (-1,1) node(0) {0}
    (1,2) node(x) {}
    (0,0) node(*) {*}
    (0,-1) node(res) {}
    ;

    \draw (0) -- (*);
    \draw (x) to[out=270,in=45] (*);
    \draw (*) -- (res);
  \end{tikzpicture}
  =\quad
  \begin{tikzpicture}[baseline]
    \path
    (0,0) node(0) {0}
    (0,2) node(x) {}
    (0,-1) node(res) {}
    (0,1) node(del) {$\eta$}
    ;

    \draw (0) -- (res);
    \draw (x) -- (del);
  \end{tikzpicture}
  \quad=
  \begin{tikzpicture}[baseline]
    \path
    (1,1) node(0) {0}
    (-1,2) node(x) {}
    (0,0) node(*) {*}
    (0,-1) node(res) {}
    ;

    \draw (0) -- (*);
    \draw (x) to[out=270,in=135] (*);
    \draw (*) -- (res);
  \end{tikzpicture}
\]
\begin{displaymath}
  \begin{matrix}
    \begin{tikzpicture}[baseline]
      \path
      (-1,2) node(x) {}
      (0,2) node(y) {}
      (-0.5,1) node(+) {+}
      (1,2) node(z) {}
      (0,0) node(*) {*}
      (0,-1) node(res) {}
      ;

      \draw (x) to[out=270,in=135] (+);
      \draw (y) to[out=270,in=45] (+);
      \draw (+) to[out=270,in=135] (*);
      \draw (z) to[out=270,in=45] (*);
      \draw (*) -- (res);
    \end{tikzpicture}
    =
    \begin{tikzpicture}[baseline]
      \path
      (-2,3) node(x) {}
      (-1,3) node(y) {}
      (0,3) node(z) {}
      (0,2) node(dup) {$\delta$}
      (-1,1) node(x*) {*}
      (0,1) node(y*) {*}
      (-0.5,0) node(+) {+}
      (-0.5,-1) node(res) {}
      ;

      \draw (z) -- (dup);
      \draw (x) to[out=270,in=135] (x*);
      \draw (y) to[out=270,in=135] (y*);
      \draw (dup) to[out=-150,in=45] (x*);
      \draw (dup) -- (y*);
      \draw (x*) to[out=270,in=135] (+);
      \draw (y*) to[out=270,in=45] (+);
      \draw (+) -- (res);
    \end{tikzpicture}
    &\phantom{mmmm}&
    \begin{tikzpicture}[baseline]
      \path
      (1,2) node(x) {}
      (0,2) node(y) {}
      (0.5,1) node(+) {+}
      (-1,2) node(z) {}
      (0,0) node(*) {*}
      (0,-1) node(res) {}
      ;

      \draw (x) to[out=270,in=45] (+);
      \draw (y) to[out=270,in=135] (+);
      \draw (+) to[out=270,in=45] (*);
      \draw (z) to[out=270,in=135] (*);
      \draw (*) -- (res);
    \end{tikzpicture}
    =
    \begin{tikzpicture}[baseline]
      \path
      (2,3) node(x) {}
      (1,3) node(y) {}
      (0,3) node(z) {}
      (0,2) node(dup) {$\delta$}
      (1,1) node(x*) {*}
      (0,1) node(y*) {*}
      (0.5,0) node(+) {+}
      (0.5,-1) node(res) {}
      ;

      \draw (z) -- (dup);
      \draw (x) to[out=270,in=45] (x*);
      \draw (y) to[out=270,in=45] (y*);
      \draw (dup) to[out=-30,in=135] (x*);
      \draw (dup) -- (y*);
      \draw (x*) to[out=270,in=45] (+);
      \draw (y*) to[out=270,in=135] (+);
      \draw (+) -- (res);
    \end{tikzpicture}
  \end{matrix}
\end{displaymath}

It is well known that all commutative comonoids in $(\Set, \times)$, and indeed
any Cartesian monoidal category, are trivial, in the sense that every object of
$\Set$ gives rise to exactly one commutative comonoid.
We find in the following two lemmas that this property also holds of
$\plr{\Set_{\mathrm{part}}, \otimes}$.

\begin{lemma}\label{thm:ccomon-exists}
  For each object $X$ in $\plr{\Set_{\mathrm{part}}, {\otimes}}$, there is
  a cocommutative comonoid over $X$.
\end{lemma}
\begin{proof}
  Let $\eta(x) \coloneq ()$ and $\delta(x) \coloneq (x, x)$, with both
  being defined for all $x$.
  Checking that these satisfy the cocomutative comonoid laws is routine.
  Alternatively, we can see that both $\eta$ and $\delta$, being total, are
  morphisms in $\mathrm{Set}$, where it is well known that they form a
  cocommutative comonoid.
  The equations in $\mathrm{Set}$ carry over to $\mathrm{Set}_{\mathrm{part}}$.
\end{proof}

\begin{lemma}\label{thm:ccomon-unique}
  For each object $X$ in $\plr{\Set_{\mathrm{part}}, {\otimes}}$, any
  comonoid over $X$ is the one described in \cref{thm:ccomon-exists}.
\end{lemma}
\begin{proof}
  The left unit law says that, for all $x$ and $x'$, we have
  $\exists y.~\delta(x) = (y, x') \land \eta(y) = ()$ if and only if $x = x'$.
  Letting $x'$ be $x$ and reading from right to left, we get that there is
  some $y$ such that $\delta(x) = (y, x)$ and $\eta(y) = ()$.
  Symmetrically, from the right unit law, we get some $z$ such that
  $\delta(x) = (x, z)$ and $\eta(z) = ()$.
  But because $\delta$, being a partial function, is deterministic, we have
  $(y, x) = (x, z)$, giving us that $y = z = x$, and $\delta(x) = (x, x)$.
  Moreover, because the chosen $y$ is equal to $x$, we have for all $x$ that
  $\eta(x) = ()$.
\end{proof}

That a morphism $f$ respects the $\eta$ given in \cref{thm:ccomon-exists} is
equivalent to saying that $f$ is total.
Therefore, all possible semiring operators in
$\mathrm{CComon}\plr{\Set_{\mathrm{part}}, \otimes}$ are total, meaning that
there is a corresponding semiring in $\plr{\Set, \times}$.

The above reasoning shows that semirings in the category of sets and partial
functions are not worth studying.
If we want partiality, there appear to be two options.
The first option is to give up on multiplication.
We could imagine replacing the binary multiplication operator by a set of
unary modalities satisfying fewer laws.
In particular, I make little use of addition on the left of a multiplication,
and multiplying by $\gr0$ on the left (as done by $\oc\gr0$) is unwanted in some
cases (such as when encoding DILL and PD, as in \cref{sec:translation}).
With unary modalities, we could expect all of the required laws to be
expressible in a symmetric monoidal category.
The second option is to use a different notion of partiality.
The notion of partiality given by the category of sets and partial functions is
``strict'', in that composing with an everywhere-undefined function yields an
everywhere-undefined function.
With a non-strict notion of partial function, we may be able to have interesting
partial semirings.


%%%%%%%%%%%%%%%%%%%%%%%%%%%%%%%%%%%%%%%%

\chapter{Renaming and substitution for $\name$}\label{sec:ren-sub-lr}

In \cref{sec:semirings}, I defined my calculus of interest $\name$.
In this chapter, I develop the necessary syntactic metatheory for
specifying and implementing the substitution operation.
I follow the approach of \cref{sec:kits} using syntactic kits, but have to make
significant changes to the underlying notion of \emph{environment} before doing
so.
I give and informally motivate these changes to environments in
\cref{sec:lrkits}, and prove some properties of the new definition in
\cref{sec:lenv}.
Finally, I apply these new environments to the syntax of $\name$ in
\cref{sec:lrsub} to derive renaming and substitution operators.

\section{What are linear renaming and substitution?}\label{sec:lrkits}
In an effort to reuse the syntactic kits and traversals approach of
\cref{sec:syntactic-kits}, I will derive the types of simultaneous renaming
and simultaneous substitution from a generic type of \emph{environments}.
To get a type of environments suitable for the usage-aware setting, I first
analyse intuitionistic environments (as introduced in \cref{sec:syntactic-kits}
definition \AgdaFunction{Env}), distilling the easy-to-use functional definition
(\cref{def:simple-env}) into a more basic recursive definition
(\cref{def:simple-rec-env}).
This recursive definition is easy to make usage-aware (\cref{def:lr-rec-env}),
which gives a basis from which to derive the function-based definition I will
take as primary (\cref{def:lr-env}).
The resulting definition makes explicit the role of algebraic linearity in the
metatheory of semiring-annotated calculi.

Recalling from \cref{sec:kits}, we have the following definition of
environments for simple types.

\begin{definition}[Simple environment]\label{def:simple-env}
  For $\V : \mathrm{Ctx} \to \mathrm{Ty} \to \mathrm{Set}$,
  a $\V$-\emph{environment} between simply typed contexts $\Gamma$ and $\Delta$
  is a function, polymorphic in type $A$, from variables of type $A$ in
  $\Delta$ to inhabitants of $\V\,\Gamma\,A$.
  We write the type of such environments as $\Gamma \env\V \Delta$.
\end{definition}

This definition is inadequate for \name{}.
For example, suppose we have a term
$\plr{M{_\otimes}N} : \grR\gamma \vdash A \otimes B$ and a substitution
$\sigma : \grR\gamma \env\vdash \grR\gr'\delta$.
From the $\otimes$-I rule, we have $M : \grP\gamma \vdash A$ and
$N : \grQ\gamma \vdash B$ for some $\grP$ and $\grQ$ such that
$\grR \leq \grP + \grQ$.
We want to apply $\sigma$ to the subterms $M$ and $N$, but this is impossible
because their contexts are not $\grR\gamma$, and we have no way to adapt
$\sigma$ to these new contexts.
Another instructive failure is the general non-existence of identity
environments, like a renaming of type $\gr1A, \gr1B \env\sqni \gr1A, \gr1B$.
We do not have a variable of type $\gr1A, \gr1B \sqni A$ or, symmetrically,
$\gr1A, \gr1B \sqni B$, because, in each case, there is one variable with
annotation $\gr1$ which we have not actually used.
This example suggests that the values of a usage-aware environment should be
derived in \emph{different} usage contexts, such as in $\gr1A, \gr0B \sqni A$.

To see why this definition of environment works for simply typed
$\lambda$-calculus but not \name{}, let us look at an equivalent definition by
recursion on the target context.
This recursive definition (\cref{def:simple-rec-env}), and particularly the
case where $\Delta$ is a concatenation, makes it clear how $\Gamma$ is being
copied for use in each $\V$-value.
I take the equivalence of \cref{def:simple-env} and \cref{def:simple-rec-env}
as obvious, because any function from variables in $\Delta$ can be
defunctionalised as a data structure with the same shape as $\Delta$.

\begin{definition}[Simple recursive environment]\label{def:simple-rec-env}
  A \emph{recursive $\V$-environment} between simply typed contexts $\Gamma$ and
  $\Delta$ is defined by cases on the shape of $\Delta$ (where
  $\Gamma \env\V_R \Delta$ is the notation for the type of recursive
  environments for given $\V$, $\Gamma$, and $\Delta$):
  \begin{itemize}
    \item There is an environment $\alr{} : \Gamma \env\V_R {\cdot}$.
    \item For $\rho_l : \Gamma \env\V_R \Delta_l$ and
      $\rho_r : \Gamma \env\V_R \Delta_r$, we have an environment
      $\alr{\rho_l, \rho_r} : \Gamma \env\V_R \Delta_l, \Delta_r$.
    \item For any value $v : \V\,\Gamma\,A$, we have an environment
      $\alr{v} : \Gamma \env\V_R A$.
  \end{itemize}
\end{definition}

I picture the sharing of $\Gamma$ in \cref{def:simple-rec-env} in the diagram
below.
The converging arrows from $\Gamma$ to each $\Delta_i$ represent the indices of
values appearing in a simple environment.

\begin{displaymath}
  \begin{tikzpicture}[baseline]
    \path
    (-1,1) node (Gtop) {}
    (-1,0) node (G) {$\Gamma$}
    (-1,-1) node (Gbot) {}
    ;
    \node[draw,dotted,fit=(Gtop) (G) (Gbot)] (GG) {};

    \path
    (1,1) node (Dtop) {}
    (1,0) node (D) {$\Delta$}
    (1,-1) node (Dbot) {}
    ;
    \node[draw,dotted,fit=(Dtop) (D) (Dbot)] (DD) {};

    \draw[->,double] (GG) -- (DD);
  \end{tikzpicture}
  \coloneqq
  \begin{tikzpicture}[baseline]
    \path
    (-1,1) node (Gtop) {}
    (-1,0) node (G) {$\Gamma$}
    (-1,-1) node (Gbot) {}
    ;
    \node[draw,dotted,fit=(Gtop) (G) (Gbot)] (GG) {};

    \path
    (1,1) node[draw] (Dtop) {$\Delta_1$}
    (1,0) node (D) {$\vdots$}
    (1,-1) node[draw] (Dbot) {$\Delta_n$}
    ;

    \fill[green!20!white,opacity=1] (GG.north east)
    parabola[bend at end] (Dtop.west)
    parabola[bend at start] (GG.south east)
    -- cycle;
    \fill[blue!40!white,opacity=.5] (GG.north east)
    parabola[bend at end] (Dbot.west)
    parabola[bend at start] (GG.south east)
    -- cycle;

    \draw[->] (GG.north east) parabola[bend at end] (Dtop.west);
    \draw (GG.south east) parabola[bend at end] (Dtop.west);
    \draw[->] (GG.north east) parabola[bend at end] (D.west);
    \draw (GG.south east) parabola[bend at end] (D.west);
    \draw[->] (GG.north east) parabola[bend at end] (Dbot.west);
    \draw (GG.south east) parabola[bend at end] (Dbot.west);
  \end{tikzpicture}
\end{displaymath}

To account for usage, we must replace the simple repetition of $\Gamma$ by
repetition of just the types $\gamma$ and \emph{redistribution} of the usage
annotations $\grP$.
Fortunately, our three basic ways of sharing up usage vectors --- zero,
addition, and scaling --- apply directly to the three possible shapes of the
target context --- empty, concatenation, and a usage-annotated singleton.

\begin{definition}[Usage-annotated recursive environment]\label{def:lr-rec-env}
  A \emph{recursive $\V$-environment} between annotated contexts $\Gamma$ and
  $\Delta$ is defined by cases on the shape of $\Delta$ (where
  $\Gamma \env\V_R \Delta$ is the notation for the
  type of recursive environments for given $\V$, $\Gamma$, and $\Delta$):
  \begin{itemize}
    \item There is one environment $\alr{} : \grP\gamma \env\V_R {\cdot}$
      whenever $\grP \leq \gr0$.
    \item For $\rho_l : \grPl\gamma \env\V_R \Delta_l$ and
      $\rho_r : \grPr\gamma \env\V_R \Delta_r$, we have an environment
      $\alr{\rho_l, \rho_r} : \grP\gamma \env\V_R \Delta_l, \Delta_r$ whenever
      $\grP \leq \grPl + \grPr$.
    \item For any value $v : \V\,\grPprime\gamma\,A$, we have an environment
      $\alr{v} : \grP\gamma \env\V_R \gr rA$ whenever
      $\grP \leq \gr r\grPprime$.
  \end{itemize}
\end{definition}

\begin{example}
  Take $\Ann = \plr{\mathbb N, =, 0, +, 1, \times}$, with the equality order
  chosen to avoid any concerns around subsumption of annotations.
  Then, there is an intuitionistic recursive environment as follows, where
  $y\,z$ is the application of $y$ to $z$.
  \[
    \alr{\alr{z},\alr{y\,z}} :
    \plr{x : A, y : B \to C, z : B} \env\vdash_R \plr{B, C}.
  \]
  There is also a usage-aware recursive environment
  \[
    \alr{\alr{z},\alr{y\,z}} :
    \plr{\gr0x : A, \gr2y : B \multimap C, \gr3z : B} \env\vdash_R
    \plr{\gr1B, \gr2C}.
  \]
  The latter relies on the observations that
  $\begin{pmatrix} \gr0 & \gr2 & \gr3 \end{pmatrix} =
  \begin{pmatrix} \gr0 & \gr0 & \gr1 \end{pmatrix}
  + \begin{pmatrix} \gr0 & \gr2 & \gr2 \end{pmatrix}$ and, on the right, that
  $\begin{pmatrix} \gr0 & \gr2 & \gr2 \end{pmatrix} =
  \gr2\begin{pmatrix} \gr0 & \gr1 & \gr1 \end{pmatrix}$.
  Then, we have $\gr0x : A, \gr0y : B \multimap C, \gr1z : B \vdash z : B$ and
  $\gr0x : A, \gr1y : B \multimap C, \gr1z : B \vdash y\,z : C$.
\end{example}

From the example, we can see that the important usage vectors are the initial
one $\begin{pmatrix} \gr0 & \gr2 & \gr3 \end{pmatrix}$ and the usage vectors
at which terms are derived: $\begin{pmatrix} \gr0 & \gr0 & \gr1 \end{pmatrix}$
and $\begin{pmatrix} \gr0 & \gr1 & \gr1 \end{pmatrix}$.
I will call the latter the \emph{leaf vectors}.
The intermediate vector $\begin{pmatrix} \gr0 & \gr2 & \gr2 \end{pmatrix}$ can
be worked out from the leaf vector
$\begin{pmatrix} \gr0 & \gr1 & \gr1 \end{pmatrix}$ and the scaling factor
$\gr2$ found in the codomain context $\gr1B, \gr2C$.
Even when the ordering on annotations is given by a non-equivalence relation
$\leq$, there is a canonical least choice for all of the intermediate vectors,
together with a constraint that the entire linear combination of all the leaf
vectors is less than or equal to the initial usage vector.
In symbols, we may let $\gr\Psi$ be the collection of leaf vectors indexed by
items in $\Delta$, and state
the constraint as $\grP \leq \sum_{\plr{x : \gr rA} \in \Delta} \gr r\gr\Psi_x$.
Seeing $\gr\Psi$ instead as a $\size\Delta \times \size\Gamma$ matrix, this
constraint is $\grP \leq \grQ\gr\Psi$, using vector-matrix multiplication.
The resulting picture is below, showing $\grP$ being split up into $\gr\Psi$,
and then each $\V$-value being constructed in a separate $\gr\Psi_i\gamma$.

\begin{displaymath}
  \begin{tikzpicture}[baseline]
    \path
    (-1,1) node (Gtop) {}
    (-1,0) node (G) {$\grP\gamma$}
    (-1,-1) node (Gbot) {}
    ;
    \node[draw,dotted,fit=(Gtop) (G) (Gbot)] (GG) {};

    \path
    (1,1) node (Dtop) {}
    (1,0) node (D) {$\grQ\delta$}
    (1,-1) node (Dbot) {}
    ;
    \node[draw,dotted,fit=(Dtop) (D) (Dbot)] (DD) {};

    \draw[->,double] (GG) -- (DD);
  \end{tikzpicture}
  \coloneqq
  \begin{tikzpicture}[baseline]
    \path
    (-1,1) node (Gtop) {}
    (-1,0) node (G) {$\grP\gamma$}
    (-1,-1) node (Gbot) {}
    ;
    \node[draw,dotted,fit=(Gtop) (G) (Gbot)] (GG) {};

    \path
    (1,3) node (G1top) {}
    (1,2) node (G1) {$\gr\Psi_1\gamma$}
    (1,1) node (G1bot) {}
    ;
    \node[draw,dotted,fit=(G1top) (G1) (G1bot)] (GG1) {};
    \draw[->] (GG) -- (GG1);

    \path (1,0) node {$\vdots$};

    \path
    (1,-1) node (Gntop) {}
    (1,-2) node (Gn) {$\gr\Psi_n\gamma$}
    (1,-3) node (Gnbot) {}
    ;
    \node[draw,dotted,fit=(Gntop) (Gn) (Gnbot)] (GGn) {};
    \draw[->] (GG) -- (GGn);

    \path
    (3,1) node[draw] (Dtop) {$\delta_1$}
    (3,0) node (D) {$\vdots$}
    (3,-1) node[draw] (Dbot) {$\delta_n$}
    ;

    \fill[green!20!white] (GG1.north east)
    parabola[bend at end] (Dtop.west)
    parabola[bend at start] (GG1.south east)
    -- cycle;
    \draw[->] (GG1.north east) parabola[bend at end] (Dtop.west);
    \draw (GG1.south east) parabola[bend at end] (Dtop.west);

    \fill[blue!20!white] (GGn.north east)
    parabola[bend at end] (Dbot.west)
    parabola[bend at start] (GGn.south east)
    -- cycle;
    \draw[->] (GGn.north east) parabola[bend at end] (Dbot.west);
    \draw (GGn.south east) parabola[bend at end] (Dbot.west);
  \end{tikzpicture}
  \quad\textrm{where }\grP \leq \grQ\gr\Psi
\end{displaymath}

From this point, we can recover a functional-style definition of usage-aware
environments.
We choose our leaf vectors $\gr\Psi$ up-front, check the inequality, and then
produce a value at each leaf vector.
%From this definition, we can recover a functional-style definition by
%separating choices of usage vectors from the provision of $\V$-values.
%In particular, the only choices of usage vectors that are essential are the
%$\grPprime$s in the singleton case, with the choices in the concatenation case
%being determined as scalings and sums of these $\grPprime$s.
%I let $\gr\Psi$ collect up these $\size\Delta$-many choices of
%$\size\Gamma$-length usage vectors and note that the constraint on $\gr\Psi$
%generated by all the scaling and summing is
%$\grP = \sum_{\plr{x : \gr rA} \in \Delta} \gr r\gr\Psi_x$.

\begin{definition}[Usage-annotated environment (tentative)]
  A \emph{$\V$-environment} between annotated contexts $\Gamma$ and $\Delta$
  (written $\grP\gamma$ and $\grQ\delta$, respectively, when convenient)
  is a matrix $\gr\Psi : \Ann^{\size\Delta \times \size\Gamma}$ such that
  $\grP \leq \grQ\gr\Psi$ and for each
  $\plr{x : A} \in \delta$ we have a value of type $\V\,\gr\Psi_x\gamma\,A$.
\end{definition}

I find this definition somewhat fiddly because of its reliance on low-level
concepts like non-usage-checked variables and rows of a matrix.
We note that $\gr\Psi_x = \bra x\gr\Psi$, from which point, requiring not
just $\V\,\gr\Psi_x\gamma\,A$ but rather $\V\,\plr{\grQprime\gr\Psi}\gamma\,A$
for any $\grQprime \leq \bra x$ is a minor change (and equivalent if $\V$
respects subusaging, which is practically always the case).
``An $x$ such that $(x : A) \in \delta$ and $\grQprime \leq \bra x\gr\Psi$''
is exactly the definition of $\grQprime\delta \sqni A$.
I further regularise this clause by asking for a
$\grPprime \leq \grQprime\gr\Psi$ rather than $\grQprime\gr\Psi$ exactly,
leaving us needing, for each $\grPprime$ and $\grQprime$ related in the same
way ($\gr\Psi$) as $\grP$ and $\grQ$, a function from $\grQprime\delta \sqni A$
to $\V\,\grPprime\gamma\,A$.
Finally, I choose to switch from matrices and matrix multiplication to
linear maps and their actions, which are easier to work with.
All of these changes yield my primary definition of an environment for
usage-annotated calculi, which will be used for the rest of this chapter and in
\cref{sec:framework}.

\begin{definition}[Usage-annotated environment]\label{def:lr-env}
  A \emph{$\V$-environment} between annotated contexts $\Gamma$ and $\Delta$
  (written $\grP\gamma$ and $\grQ\delta$, respectively, when convenient)
  is a linear map $\gr\Psi : \Ann^{\size\Delta} \to \Ann^{\size\Gamma}$ (written
  postfix) such that $\grP \leq \grQ\gr\Psi$ and for each $A$, $\grPprime$, and
  $\grQprime$ such that $\grPprime \leq \grQprime\gr\Psi$, a function from
  $\grQprime\delta \sqni A$ to $\V\,\grPprime\gamma\,A$.
\end{definition}
\begin{notation}
  When there are multiple environments in question and $\rho$ is such an
  environment, I use the notation $\rho.\gr\Psi$ to refer to $\gr\Psi$.
  For example, $\grP \leq \grQ\plr{\rho.\gr\Psi}$.
  For the action on variables, I write $\rho(x)$, where
  $x : \grQprime\delta \sqni A$.
  The expression ``$\rho(x)$'' alone is ambiguous because of the slack in the
  usage context $\grPprime$ of the resulting value.
  Therefore, I will always make sure $\grPprime$ and $\grQprime$ clear when
  using this notation.
\end{notation}

The following simple lemma shows that usage-annotated environments are, in a
sense, as good as simple environments on usage-checked variables.
What usage-annotated environments give us beyond simple environments is the
ability to accommodate linear decompositions, in a way I will make precise in
the next section.

\begin{lemma}
  We can use an environment $\rho : \Gamma \env\V \Delta$ to map a
  usage-checked variable $x : \Delta \sqni A$ to a value of type
  $\V\,\Gamma\,A$.
\end{lemma}
\begin{proof}
  Let $\Gamma = \grP\gamma$ and $\Delta = \grQ\delta$.
  Set $\grPprime \coloneqq \grP$ and $\grQprime \coloneqq \grQ$, then
  $\grP \leq \grQ\gr\Psi$ by the constraint in $\rho$, so we can take
  the $\V$-value $\rho(x)$.
\end{proof}

\section{Properties of linear environments}\label{sec:lenv}
I settle on \cref{def:lr-env}, and prove various properties about it.

\begin{lemma}\label{thm:env-resize}
  Given an environment $\rho : \grP\gamma \env\V \grQ\delta$ and a $\grPprime$
  and a $\grQprime$ such that $\grPprime \leq \grQprime\plr{\rho.\gr\Psi}$,
  there is also an environment of type $\grPprime\gamma \env\V \grQprime\delta$
  with the same linear map and action on variables.
\end{lemma}
\begin{proof}
  The only part of the definition of an environment dependent on $\grP$ or
  $\grQ$ is the constraint $\grP \leq \grQ\gr\Psi$, which we are able to
  replace for $\grPprime$ and $\grQprime$.
\end{proof}

When constructing an environment, we can do so by cases on the shape of the
target context.
We can create an environment into the empty context when all usage annotations
on the source context are $\gr0$.
We can create an environment into a concatenated context when we can additively
split up the annotations of the source context and produce environments into
both halves from the split sources.
We can create an environment into a singleton context when there is a context
$\gr r$ times smaller than the source context in which we can produce a value
of the appropriate type.

\begin{lemma}\label{thm:construct-env}
  We can define all of the following equivalences for any values of the free
  variables, assuming that $\V$ respects subusaging (i.e.,
  $\grPprime \leq \grP \to
  \V\,\grP\gamma \rightarrowtriangle \V\,\grPprime\gamma$).
  \begin{itemize}
    \item $I^{\sep} \leftrightarrowtriangle \plr{{-} \env\V {\cdot}}$
    \item $\plr{{-} \env\V \Delta_l} \sep \plr{{-} \env\V \Delta_r}
      \leftrightarrowtriangle \plr{{-} \env\V \Delta_l, \Delta_r}$
    \item $\gr r \cdot \plr{\V\,(-)\,A}
      \leftrightarrowtriangle \plr{{-} \env\V \gr rA}$
  \end{itemize}
\end{lemma}
\begin{proof}
  There are 6 cases to check.
  Throughout, we write $\Gamma$ as $\grP\gamma$ and $\Delta$ as $\grQ\delta$
  when convenient.
  \begin{description}
    \item[$I^{\sep}(\rightarrowtriangle)$]
      Let $\gr\Psi$ be the unique linear map out of the zero space.
      By assumption and definition, $\grP \leq \gr0 = \grQ\gr\Psi$.
      There are no variables to act upon.
    \item[$I^{\sep}(\leftarrowtriangle)$]
      $\grQ\gr\Psi$ is an empty sum, so if $\grP \leq \grQ\gr\Psi$ then
      $\grP \leq \gr0$.
    \item[$\sep(\rightarrowtriangle)$]
      Let the given environments be $\rho_l : \grPl\gamma \env\V \grQl\delta$
      and $\rho_r : \grPr\gamma \env\V \grQr\delta$, with
      $\grP \leq \grPl + \grPr$.
      Define $\gr\Psi \coloneqq [\rho_l.\gr\Psi, \rho_r.\gr\Psi]$, using the
      coproduct structure of the concatenated vector space.
      We have $\grP \leq \grPl + \grPr \leq
      \grQl\plr{\rho_l.\gr\Psi} + \grQr\plr{\rho_r.\gr\Psi} =
      \begin{pmatrix} \grQl & \grQr \end{pmatrix}\gr\Psi$.
      To act on variables, we are given $\grPprime \leq
      \begin{pmatrix} \gr{\grQ'_l} & \gr{\grQ'_r} \end{pmatrix}\gr\Psi$ and
      $\gr{\grQ'_l}\delta_l, \gr{\grQ'_r}\delta_r \sqni A$.
      Without loss of generality, let us have $\gr{\grQ'_l}\delta_l \sqni A$
      and $\gr{\grQ'_r} \leq \gr0$.
      Thus, $\grPprime \leq
      \gr{\grQ'_l}\plr{\rho_l.\gr\Psi} + \gr{\grQ'_r}\plr{\rho_r.\gr\Psi} \leq
      \gr{\grQ'_l}\plr{\rho_l.\gr\Psi}$,
      and we can act on the variable using $\rho_l$.
    \item[$\sep(\leftarrowtriangle)$]
      Let the unnamed context be $\Gamma$, also written $\grP\gamma$.
      The linear map
      $\gr\Psi : \Ann^{\size{\Delta_l} + \size{\Delta_r}} \to \Ann^{\size\Gamma}$
      splits into
      $\gr\Psi_{\gr l} : \Ann^{\size{\Delta_l}} \to \Ann^{\size\Gamma}
      \coloneqq \alr{\id, 0}; \gr\Psi$ and
      $\gr\Psi_{\gr r} : \Ann^{\size{\Delta_r}} \to \Ann^{\size\Gamma}
      \coloneqq \alr{0, \id}; \gr\Psi$, using the product structure of
      the concatenated vector space.
      Let $\grPl \coloneqq \grQl\gr\Psi_{\gr l}$ and
      $\grPr \coloneqq \grQr\gr\Psi_{\gr r}$, by definition satisfying the
      required constraints.
      For the action on variables, let us consider the left environment (with
      the right environment following symmetrically).
      We are given $\gr{\grP'_l} \leq \gr{\grQ'_l}\gr\Psi_{\gr l}$ and
      $\gr{\grQ'_l}\delta_l \sqni A$.
      From these, we get
      $\gr{\grP'_l} \leq \gr{\grQ'_l}\gr\Psi_{\gr l} =
      \begin{pmatrix} \gr{\grQ'_l} & \gr0 \end{pmatrix}\gr\Psi$ and
      $\gr{\grQ'_l}\delta_l, \gr0\delta_r \sqni A$.
      We can therefore act using the original environment.
    \item[$\cdot(\rightarrowtriangle)$]
      Let $\grP$ and $\grPprime$ be such that $\grP \leq \gr r\grPprime$ and let
      $v : \V\,\grPprime\gamma\,A$.
      Let $\gr\Psi : \Ann \to \Ann^{\size\gamma}
      \coloneqq \gr r\gr' \mapsto \gr r\gr'\grPprime$.
      By definition and the previous assumption, we have
      $\grP \leq \gr r\gr\Psi$.
      When acting on a variable, we have $\grP\gr{''} \leq \gr r\gr'\gr\Psi$
      and $\gr r\gr'A \sqni A'$.
      The latter tells us that $A = A'$ and $\gr r\gr' \leq \gr1$.
      Thus, $\grP\gr{''} \leq \grPprime$.
      Therefore, by subusaging, we may produce a value of type
      $\V\,\grPprime\gamma\,A$, which we can take to be $v$.
    \item[$\cdot(\leftarrowtriangle)$]
      Let us have an environment of type $\grP\gamma \env\V \gr rA$.
      We want to use its action on variables to yield a value.
      To do this, we let $\grPprime \coloneqq \gr1\gr\Psi$, and use this
      equation, together with the fact that we have a variable of type
      $\gr1A \sqni A$, to get a value of type $\V\,\grPprime\gamma\,A$.
      Furthermore, we derive $\grP \leq \gr r\gr\Psi = \gr r\grPprime$, as
      required.
  \end{description}
\end{proof}

We could, as in \cref{def:lr-rec-env}, use these three clauses to define what an
environment is.
However, such a definition appears to require creative induction hypotheses in
the proving of simple lemmas, in contrast to the more direct proofs I achieve
below using \cref{def:lr-env}.
To take a concrete example, consider how we may construct an ``identity''
environment of type $\Gamma \env\V \Gamma$, as in \cref{thm:env-id} below.
If we try to directly proceed by induction on $\Gamma$, we get to the case where
we are aiming to construct an environment of type
$\grP\gamma, \grQ\delta \env\V \grP\gamma, \grQ\delta$ by constructing
environments of types $\grP\gamma, \gr0\delta \env\V \grP\gamma$ and
$\gr0\gamma, \grQ\delta \env\V \grQ\delta$.
These are not identity environments, and thus do not come from the hypotheses of
a simple induction.
In contrast, using \cref{def:lr-env}, in \cref{thm:env-id} we are able to use
the standard fact that there are identity linear maps, and on top of such a map
worry only about the value assigned to each variable.

One of the primary test cases for environments is simultaneous substitution,
which will look like the \TirName{sub} rule below.
Note that we have taken $\V \coloneqq {\vdash}$ --- i.e.\ that the values
yielded by the environment are terms, namely the terms to be substituted in for
the free variables of the derivation of $\Delta \vdash A$.

\begin{displaymath}
  \begin{prooftree}
    \hypo{\Gamma \env{\vdash} \Delta}
    \hypo{\Delta \vdash A}
    \infer2[sub]{\Gamma \vdash A}
  \end{prooftree}
\end{displaymath}

The admissibility of substitution will be by induction on the derivation of
$\Delta \vdash A$, so we will need to be able to adapt any environment we are
given to work with any possible context of new premises yielded by the rules of
\cref{fig:lr-bunched}.
In the simply typed case, the only change to the context we encountered was the
binding of new variables.
With usage annotations, we furthermore have linear decompositions of the
context, necessitating changes to the environment whenever usage annotations
change.

There are three kinds of linear decompositions we have to deal with: zero,
addition, and scaling; corresponding to bunched connectives $I^*$, $\sep$, and
$\gr r \cdot {}$, respectively.
In each of these three cases, we have a simple preservation lemma, transforming
an environment
of type $\Gamma \env\V \Delta$ and a decomposition of $\Delta$ into a
decomposition of $\Gamma$ and environments for all of the decomposed fragments
of $\Gamma$ and $\Delta$.

\begin{lemma}[environments preserve zero]\label{thm:lr-env-zero}
  Given an environment $\rho : \grP\gamma \env\V \grQ\delta$ such that
  $\grQ \leq \gr 0$, we also have that $\grP \leq \gr 0$.
\end{lemma}
\begin{proof}
  $\grP \leq \grQ\gr\Psi \leq \gr0\gr\Psi = \gr0$, by environment
  compatibility from $\rho$ and monotonicity and linearity of $\gr\Psi$.
\end{proof}

\begin{lemma}[environments preserve addition]\label{thm:lr-env-add}
  Given an environment $\rho : \grP\gamma \env\V \grQ\delta$ such that
  $\grQ \leq \grQl + \grQr$ for some $\grQl$ and $\grQr$, we also have $\grPl$
  and $\grPr$ such that $\grP \leq \grPl + \grPr$ and there are environments
  $\rho_l : \grPl\gamma \env\V \grQl\delta$ and
  $\rho_r : \grPr\gamma \env\V \grQr\delta$.
\end{lemma}
\begin{proof}
  Let $\grPl \coloneqq \grQl\gr\Psi$ and $\grPr \coloneqq \grQr\gr\Psi$.
  Then, $\grP \leq \grQ\gr\Psi \leq \plr{\grQl + \grQr}\gr\Psi =
  \grQl\gr\Psi + \grQr\gr\Psi = \grPl + \grPr$, satisfying the first condition.
  Because clearly $\grPl \leq \grQl\gr\Psi$ and $\grPr \leq \grQr\gr\Psi$,
  applying \cref{thm:env-resize} to $\rho$ gives us the required
  new environments $\rho_l$ and $\rho_r$.
\end{proof}

\begin{lemma}[environments preserve scaling]\label{thm:lr-env-scale}
  Given an environment $\rho : \grP\gamma \env\V \grQ\delta$ such that
  $\grQ \leq \gr r\grQprime$ for some $\grQprime$, we also have a $\grPprime$
  such that $\grP \leq \gr r\grPprime$ and there is an environment
  $\rho' : \grPprime\gamma \env\V \grQprime\delta$.
\end{lemma}
\begin{proof}
  Let $\grPprime \coloneqq \grQprime\gr\Psi$.
  Then, $\grP \leq \grQ\gr\Psi \leq \plr{\gr r\grQprime}\gr\Psi =
  \gr r\plr{\grQprime\gr\Psi} = \gr r\grPprime$, satisfying the first condition.
  Because clearly $\grPprime \leq \grQprime\gr\Psi$,
  applying \cref{thm:env-resize} to $\rho$ gives us the required
  new environment $\rho'$.
\end{proof}

The final change environments need to preserve is the binding of new free
variables.
In \cref{sec:syntactic-kits}, we had the operation \AgdaFunction{bindEnv} for
this purpose in the intuitionistic setting.
There, we relied on $\V$ supporting a map from $\ni$-variables and admitting
weakening.
In the usage-annotated setting, the former requirement is updated to having a
map from usage-checked $\sqni$-variables.
As for the latter requirement, it turns out that we only need $\V$ to admit
weakening by $\gr0$-annotated variables, which is much more reasonable than
general weakening.
\Cref{thm:lr-bind} adapts \AgdaFunction{bindEnv} for the usage-annotated
setting.

\begin{lemma}[\AgdaFunction{bindEnv}]\label{thm:lr-bind}
  Given functions
  ${\swarrow^k} : \forall \Gamma, \grR, \theta.~\grR \leq \gr0 \to
  \V\,\Gamma \rightarrowtriangle \V\,\plr{\Gamma, \grR\theta}$ and
  $\mathrm{vr} : {\sqni} \rightarrowtriangle \V$, we can turn an environment of
  type $\Gamma \env\V \Delta$ into an environment of type
  $\Gamma, \Theta \env\V \Delta, \Theta$ for any context $\Theta$.
\end{lemma}
\begin{proof}
  Let $\grP\gamma \coloneqq \Gamma$, $\grQ\delta \coloneqq \Delta$, and
  $\grR\theta \coloneqq \Theta$.
  Let the new linear map $\gr\Psi\gr' : \Ann^{\size\Delta + \size\Theta} \to
  \Ann^{\size\Gamma + \size\Theta}$ be $\gr\Psi \oplus \gr I$.
  That is, in block matrix notation,
  $\begin{pmatrix} \gr\Psi & \gr0 \\ \gr0 & \gr I \end{pmatrix}$.
  Checking that this linear map fits, we have
  $\begin{pmatrix}\grP & \grR\end{pmatrix}
  \leq \begin{pmatrix}\grQ\gr\Psi & \grR\gr I\end{pmatrix}
  = \begin{pmatrix}\grQ & \grR\end{pmatrix}\plr{\gr\Psi \oplus \gr I}$.
  For the action on variables, we are given vectors $\grPprime$,
  $\grR\gr'_\grP$, $\grQprime$, and $\grR\gr'_\grQ$ such that
  $\begin{pmatrix} \grPprime & \grR\gr'_\grP \end{pmatrix} \leq
  \begin{pmatrix} \grQprime & \grR\gr'_\grQ \end{pmatrix}
  \plr{\gr\Psi \oplus \gr I}$ and we have a variable of type
  $\grQprime\delta, \grR\gr'_\grQ\theta \sqni A$ for some type $A$.
  The constraint on the new vectors reduces to $\grPprime \leq \grQprime\gr\Psi$
  and $\grR\gr'_\grP \leq \grR\gr'_\grQ$.
  From the variable we either have a variable $x$ in $\delta$ with
  $\grQprime \leq \langle x \rvert$ and $\grR\gr'_\grQ \leq \gr0$, or a
  variable $y$ in $\theta$ with $\grQprime \leq \gr0$ and
  $\grR\gr'_\grQ \leq \langle y \rvert$.
  In the former case, the action of the original environment on $x$ gives us a
  $\V$-value in $\grPprime\gamma$, and the $\gr0$-weakening principle
  $\swarrow^k$, noting that $\grR\gr'_\grP \leq \grR\gr'_\grQ \leq \gr0$, gives
  us a $\V$-value in $\grPprime\gamma, \grR\gr'_\grP\theta$.
  In the latter case, we have that
  $\begin{pmatrix} \grPprime & \grR\gr'_\grP \end{pmatrix}
  \leq \begin{pmatrix} \grQprime\gr\Psi & \grR\gr'_\grQ \end{pmatrix}
  \leq \begin{pmatrix} \gr0\gr\Psi & \langle y \rvert \end{pmatrix}
  = \begin{pmatrix} \gr0 & \langle y \rvert \end{pmatrix}
  = \left\langle {\searrow}y \right\rvert$, so $y$ also serves as a
  usage-checked variable in $\grPprime\gamma, \grR\gr'_\grP\theta$.
  From this usage-checked variable, we get a $\V$-value in the same context
  using $\mathrm{vr}$.
\end{proof}

I put together the preceding pieces to give a syntactic traversal operation over
$\name$ in the following section.
For the rest of this section, I observe some more constructions purely on
environments --- in particular, composition of environments given certain
assumptions about the families of values.

Following \citet{ACU15}, we expect (intuitionistic) ST$\lambda$C syntax to form
a relative monad over $\ni$ seen as a functor from the category of contexts
under renaming to the functor category $\blr{\mathrm{Ty}, \Set}$, where
$\mathrm{Ty}$ is the discrete category of ST$\lambda$C types.
Notice that, given $F,G : \blr{\mathrm{Ty}, \Set}$, a morphism from $F$ to $G$
is a function of type $F \rightarrowtriangle G$ (with naturality being trivial).
Therefore, we expect a relative monad, given as a Kleisli triple, to have a unit
$\eta_\Gamma : \Gamma \ni \plr{-} \rightarrowtriangle \Gamma \vdash \plr{-}$
given by the variable rule, and a Kleisli extension operator
${^*}_{\Gamma,\Delta} :
\plr{\Gamma \ni \plr{-} \rightarrowtriangle \Delta \vdash \plr{-}} \to
\plr{\Gamma \vdash \plr{-} \rightarrowtriangle \Delta \vdash \plr{-}}$
given by substitution.
Composition of substitutions falls out of this framework as Kleisli composition.
However, in the usage-aware case, substitution needs not just a mapping of
variables $f : \Gamma \sqni \plr{-} \rightarrowtriangle \Delta \vdash \plr{-}$,
but rather an environment $\rho : \Delta \env\vdash \Gamma$, as we have already
discussed.
It therefore makes sense for our replacement for the Kleisli extension operator
to similarly take an environment rather than a simple variable mapping.

\Cref{thm:env-comp} below amounts to deriving a modified notion of Kleisli
composition from a modified Kleisli extension.
Additionally, \cref{thm:env-id} is required to turn a monadic unit into an
identity environment.
Both lemmas are stated in terms of general $\U$/$\V$/$\W$-environments, with
some specific examples (e.g.\ for renaming and substitution) below them.

\begin{lemma}[Identity environment]\label{thm:env-id}
  Given a function
  \[
    \mathrm{vr}_{\Gamma'} :
    \Gamma' \sqni \plr{-} \rightarrowtriangle \V\,\Gamma'
  \]
  for any
  $\Gamma$ we have an environment $\mathrm{id} : \Gamma \env\V \Gamma$.
\end{lemma}
\begin{proof}
  Let $\Gamma = \grP\gamma$.
  Let $\gr\Psi$ be the identity map, which clearly satisfies
  $\grP \leq \grP\gr\Psi$.
  When acting on a variable, the inequality $\grPprime \leq \grQprime\gr\Psi$
  means that $\grPprime \leq \grQprime$.
  We are given a variable of type $\grQprime\gamma \sqni A$, which we can
  coerce to a variable of type $\grPprime\gamma \sqni A$, upon which we apply
  $\mathrm{vr}$ to get the required value of type $\V\,\grPprime\gamma\,A$.
\end{proof}

\begin{lemma}[Composition of environments]\label{thm:env-comp}
  Given a function
  \[
    \mathrm{lift}_{\Gamma', \Delta'} :
    \Gamma' \env\U \Delta' \to \V\,\Delta' \rightarrowtriangle \W\,\Gamma'
  \]
  we can compose environments $\rho : \Gamma \env\U \Delta$ and
  $\sigma : \Delta \env\V \Theta$ into an environment
  $\rho \gg \sigma : \Gamma \env\W \Theta$.
\end{lemma}
\begin{proof}
  Let $\Gamma = \grP\gamma$, $\Delta = \grQ\delta$, and $\Theta = \grR\theta$.
  Take $\gr\Psi$ to be the composition $\plr{\sigma.\gr\Psi}\plr{\rho.\gr\Psi}$,
  noting that
  $\grP \leq \grQ\plr{\rho.\gr\Psi} \leq
  (\grR\plr{\sigma.\gr\Psi})\plr{\rho.\gr\Psi} = \grR\gr\Psi$
  thanks to the inequalities yielded by $\sigma$ and $\rho$.
  When acting on a variable, we are given $\grPprime \leq \grRprime\gr\Psi$ and
  a variable $v : \grRprime\theta \sqni A$, and want a value of type
  $\W\,\grPprime\gamma\,A$.
  Let $\grQprime \coloneqq \grRprime\plr{\sigma.\gr\Psi}$, with inequality
  $\grQprime \leq \grRprime\plr{\sigma.\gr\Psi}$ giving us a value
  $\sigma(v) : \V\,\grQprime\delta\,A$.
  We wish to apply $\mathrm{lift}$ to $\sigma(v)$ with
  $\Gamma' \coloneqq \grPprime\gamma$ and $\Delta' \coloneqq \grQprime\delta$ to
  complete the construction of the $\W$-value.
  To do this, we need an environment of type
  $\grPprime\gamma \env\U \grQprime\delta$, which we can get from $\rho$ using
  \cref{thm:env-resize}, noting that
  $\grPprime \leq \grRprime\plr{\sigma.\gr\Psi}\plr{\rho.\gr\Psi} =
  \grQprime\plr{\rho.\gr\Psi}$.
\end{proof}

We can derive the following corollaries as instances of environment composition.

\begin{corollary}[Composition of renamings]\label{thm:ren-comp}
  Given renamings $\rho : \Gamma \env\sqni \Delta$ and
  $\sigma : \Delta \env\sqni \Theta$, we can form their composite
  $\rho; \sigma : \Gamma \env\sqni \Theta$.
\end{corollary}
\begin{proof}
  Take $\U = \V = \W = {\sqni}$ in \cref{thm:env-comp}.
  Then let $\mathrm{lift}\,\rho\,x \coloneqq \rho(x)$.
\end{proof}

\begin{corollary}[Post-composition with a renaming]\label{thm:ren-env-comp}
  Given an environment $\rho : \Gamma \env\U \Delta$ and a renaming
  $\sigma : \Delta \env\sqni \Theta$, we can form their composite
  $\rho; \sigma : \Gamma \env\U \Theta$.
\end{corollary}
\begin{proof}
  As in \cref{thm:ren-comp}.
\end{proof}

\begin{corollary}[Pointwise renaming of an environment]\label{thm:env-ren}
  If $\sdtstile{}\V$ respects renaming, then so does $\env\V$ (on the left).
\end{corollary}
\begin{proof}
  Suppose we have $\rho : \Gamma \env\sqni \Delta$ and
  $\sigma : \Delta \env\V \Theta$.
  We want to compose these via \cref{thm:env-comp} with $U = {\sqni}$ and
  $\V = \W$.
  The function $\mathrm{lift}$ is given exactly by the fact that $\V$ respects
  renaming.
\end{proof}

\begin{corollary}[Composition of substitutions]\label{thm:sub-comp}
  Given substitutions $\rho : \Gamma \env\vdash \Delta$ and
  $\sigma : \Delta \env\vdash \Theta$, we can form their composite
  $\rho; \sigma : \Gamma \env\vdash \Theta$.
\end{corollary}
\begin{proof}
  Take $\U = \V = \W = {\vdash}$ in \cref{thm:env-comp}.
  Then, $\mathrm{lift}$ is given by the action of a substitution on a term
  (see \AgdaFunction{sub} in the following section).
\end{proof}

\begin{corollary}[Composing semantics with substitution]
  If we have a semantics (in the sense of \cref{sec:gen-sem} and
  \cref{sec:traversal}) from $\U$ to $\W$, then from an environment
  $\rho : \Gamma \env\U \Delta$ and a substitution
  $\sigma : \Delta \env\vdash \Theta$, we can form the composite
  $\rho; \sigma : \Gamma \env\W \Theta$.
\end{corollary}

% Concatenation is difficult; save to after I've talked about renamings.

% Finally for this section, we give the conditions under which the
% context-forming operations (empty, concatenation, and singleton) have a
% functorial action with respect to $\V$-environments.
%
% \begin{lemma}
%   For any $\V$, there is an environment ${\cdot} \env\V {\cdot}$.
% \end{lemma}
% \begin{proof}
%   By \cref{thm:construct-env}, it suffices to show $I\,{\cdot}$, which is
%   trivially true.
% \end{proof}

\section{Substitution is admissible in \name{}}\label{sec:lrsub}
\def\LRKits{../agda/processed-latex/LRKits.tex}

I now show that, using the notion of \emph{environment} derived in
\cref{sec:lrkits}, we can replicate the Agda proofs from
\cref{sec:syntactic-kits} in the usage-aware setting of $\name$.
From \cref{sec:lenv}, we know that environments are preserved under all
syntax-forming operations: zero, addition, scaling, and binding.
What is left is to show how these properties are deployed, and also how to
go on and prove the admissibility of simultaneous renaming, simultaneous
substitution, and then single substitution.

There are a few notational changes necessary in the Agda code, compared to the
typeset mathematics above.
Usage vectors, elsewhere called $\grP$, $\grQ$, and $\grR$ are rendered as
\AgdaBound{P}, \AgdaBound{Q}, and \AgdaBound{R}, respectively.
Usage contexts and typing contexts are tied together with the
\AgdaInductiveConstructor{ctx} constructor, rather than simple juxtaposition.
Environments, elsewhere notated $\Gamma \env\V \Delta$, are rendered as
\AgdaRecord{[}\AgdaSpace{}\AgdaBound{$\V$}\AgdaSpace{}\AgdaRecord{]}%
\AgdaSpace{}\AgdaBound{$\Gamma$}\AgdaSpace{}\AgdaRecord{$\Rightarrow^e$}%
\AgdaSpace{}\AgdaBound{$\Delta$}.

I start with the definition \AgdaFunction{Weakening}, which says what it means
for a family of values \AgdaBound{$\V$} to respect one step of weakening on the
right by $\gr0$-use variables.
I state weakening in a slightly different way to what appears in the statement
of \cref{thm:lr-bind}, so as to help
unification against a known result type (avoiding the problem described by
\citet{McBride12} as \emph{green slime}).
The type \AgdaFunction{Weakening}\AgdaSpace{}\AgdaBound{$\V$} can be read as
saying that, for any context $\grP\gamma$ of shape $s + t$, if the right of
$\grP$ is below $\gr0$, then a value in the left part of $\grP\gamma$ weakens
to a value in the whole of $\grP\gamma$.

\ExecuteMetaData[\LRKits]{Weakening}

Given this new definition of \AgdaFunction{Weakening}, the record
\AgdaRecord{Kit} remains largely unchanged relative to what we saw in
\cref{sec:syntactic-kits}.
We still want $\V$-values to respect weakening (as used in \cref{thm:lr-bind}),
and to support maps \AgdaField{vr} from variables (also used in
\cref{thm:lr-bind}) and \AgdaField{tm} into terms (used when traversal reaches a
variable, we get a value from the environment, and want to produce a term from
that value).
As well as \AgdaFunction{Weakening}, note that \AgdaRecord{\_$\sqni$\_} and, of
course, \AgdaDatatype{\_$\vdash$\_} have different definitions to the
corresponding intuitionistic notions, but they still represent morally the same
parts of the language and its metatheory.

\ExecuteMetaData[\LRKits]{Kit}

To demonstrate the important points succinctly, I cut \name{} down to just the
$\oc\gr r$-fragment.
The introduction rule and pattern-matching eliminator feature scaling, addition,
and variable binding, missing out only on sharing (which is trivial) and zero
(which is simpler than, and analogous to, addition).
The resulting type of well typed terms is below.

\ExecuteMetaData[\LRKits]{Tm}

Given a \AgdaRecord{Kit}\AgdaSpace{}\AgdaBound{$\V$},
\cref{thm:lr-bind} gives a function with the following type.

\ExecuteMetaData[\LRKits]{bindEnv}

Given \AgdaFunction{bindEnv} (\cref{thm:lr-bind}), \AgdaFunction{env-+}
(\cref{thm:lr-env-add}), and \AgdaFunction{env-*} (\cref{thm:lr-env-scale}),
we can reproduce the syntactic traversal \AgdaFunction{trav}.
Similarly to the unchanged high-level definition of \AgdaRecord{Kit}, we are
aiming for an unchanged traversal principle expressed by the rule below.
When $\V$ has a \AgdaRecord{Kit} structure, and we have a $\V$-environment from
$\Gamma$ to $\Delta$, we can transform a term in $\Delta$ to a term in $\Gamma$
of the same type.

\[
  \begin{prooftree}
    \hypo{\text{\AgdaRecord{Kit}\AgdaSpace{}}\V}
    \hypo{\Gamma \env\V \Delta}
    \hypo{\Delta \vdash A}
    \infer3[Trav]{\Gamma \vdash A}
  \end{prooftree}
\]

With all the lemmas of \cref{sec:lenv} in place, writing \AgdaFunction{trav}
becomes routine.
When processing a rule, we work our way up through the
premise connectives, applying \AgdaFunction{env-*} wherever we see a
\AgdaFunction{$\cdot^c$}, \AgdaFunction{env-+} wherever we see a
\AgdaFunction{$*^c$}, and \AgdaFunction{bindEnv} wherever we see a
\AgdaFunction{Bind}.
We then use whatever environments (with names beginning with
\AgdaBound{$\rho$}) and whatever usage vector splitting facts (with names
beginning with \AgdaBound{sp}) come out of this process to recursively
traverse the subterms and recombine the results.

\ExecuteMetaData[\LRKits]{trav}

Instantiating the generic syntactic traversal \AgdaFunction{trav} to renaming
looks just like it did in the intuitionistic case.
I have consistently replaced intuitionistic variables by linear variables, so
\AgdaFunction{id} and \AgdaInductiveConstructor{var} still work to embed
variables into variables and terms, respectively.
Weakening for variables \AgdaFunction{$\swarrow^v$} (not pictured) has been
updated to note that, for $\grP \leq \bra x$ and $\grR \leq \gr0$, we also have
$\begin{pmatrix} \grP & \grR \end{pmatrix} \leq \bra{{\swarrow}x}$.

\ExecuteMetaData[\LRKits]{var-kit}

In the intuitionistic case, environments were just functions, so we passed the
variable weakening function \AgdaFunction{$\swarrow^v$} to the function
\AgdaFunction{ren} to yield a term weakening function.
However, a usage-aware environment is a function packed together with usage
distribution data.
As such, we must make an environment version of \AgdaFunction{$\swarrow^v$}.
I start with a general lemma \AgdaFunction{$\swarrow$\^{}Env}, stating that if
$\V$ supports weakening, then so do $\V$-environments (in their domain
context).
This lemma then specialises to variables, with the identity renaming
\AgdaFunction{id\^{}Env} on the left part of the context and the proof
\AgdaBound{R0} that the right part of the context is below $\gr0$ combining
to give the desired weakening environment.

\ExecuteMetaData[\LRKits]{dlv-env}

This is what we need to instantiate \AgdaFunction{trav} for substitution.
As a reminder, I also give the type of \AgdaFunction{sub} in rule form.

\ExecuteMetaData[\LRKits]{sub}
\[
  \ebrule{%
    \hypo{\Gamma \env\vdash \Delta}
    \hypo{\Delta \vdash B}
    \infer2[sub]{\Gamma \vdash B}
  }
\]

Finally, the simultaneous substitution \AgdaFunction{sub} specialises to
single substitution.

\begin{corollary}[Single substitution]\label{thm:single-sub}
  The following equivalent rules are admissible.
  \begin{mathpar}
    \ebrule{%
      \hypo{\grR \leq \gr r\grP + \grQ}
      \hypo{\grP\gamma \vdash A}
      \hypo{\grQ\gamma, \gr rA \vdash B}
      \infer3{\grR\gamma \vdash B}
    }
    \and
    \ebrule[comb]{%
      \hypo{\gr r \cdot \plr{{} \vdash A}}
      \hypo{\sep}
      \hypo{\gr rA \vdash B}
      \infer3{{} \vdash B}
    }
  \end{mathpar}
\end{corollary}
\begin{proof}
  It is enough to construct a substitution of type
  $\grR\gamma \env\vdash \grQ\gamma, \gr rA$.
  To do this, we use \cref{thm:construct-env} cases $\sep(\rightarrowtriangle)$
  and $\cdot(\rightarrowtriangle)$ on inequalities
  $\grR \leq \grQ + \gr r\grP$ and $\gr r\grP \leq \gr r\grP$ respectively to
  leave us needing a substitution of type $\grQ\gamma \env\vdash \grQ\gamma$ and
  a term of type $\grP\gamma \vdash A$.
  For the substitution, we give the identity substitution (\cref{thm:env-id}),
  and we have the term as a hypothesis.
\end{proof}

\section{Conclusion}\label{sec:ren-sub-lr-conc}

In the preceding two chapters, I have developed a discipline for specifying
the syntax of linear and modal type systems, and furthermore developing the
syntactic metatheory of those type systems.
All of these are based on semirings, and the linear algebra arising from
considering a usage context full of semiring elements as a vector.

These developments can be seen in retrospect as a generalisation of the methods
explained in \cref{sec:simple}.
In terms of premise connectives in the syntactic rules, we have generalised from
just $\{\dot1, \dottimes\}$ to
$\{\dot1, \dottimes, I^*, \sep, \gr r\cdot{}, \Box^{0{+}}\}$, maintaining our
ability to keep the context implicit.
Similarly to how rule premises can require separation of usage annotations, our
new environments can require such a separation between their entries thanks to
the linear map they now contain.
I have generalised the key property of a kit from arbitrary weakening to
weakening by $\gr0$-annotated variables, and using that have produced a
substitution operation based on the same principles as that from
\cref{sec:kits}.

Having generalised all of the components --- namely the contexts, the syntax,
and the notion of environment --- the type of the substitution operation looks
the same as it did for intuitionistic ST$\lambda$C\@.
Being able to maintain this uniformity is a key step towards generalising the
rest of \cref{sec:simple} (i.e., \cref{sec:gen-sem,sec:gen-syn}), as I do in
\cref{sec:framework}.

A similar substitution lemma appears in the PhD thesis of
\citet[p.\ 138]{petricek-thesis} under the name \emph{multi-nary substitution}.
In my notation, \citeauthor{petricek-thesis}'s substitution rule looks like the
following, up to permutation of the contexts containing $\Gamma$.
Note that if $\Delta = \grQ\delta$, then $\gr r\Delta$ denotes the context
$\plr{\gr r\grQ}\delta$.
This rule is essentially an iterated version of the standard linear single
substitution principle, and is used by \citeauthor{petricek-thesis} as a
strengthened induction hypothesis required to derive single substitution.

\[
  \ebrule{%
    \hypo{\Delta_1 \vdash A_1}
    \hypo{\cdots}
    \hypo{\Delta_n \vdash A_n}
    \hypo{\Gamma, \gr{r_1}A_1, \ldots, \gr{r_n}A_n \vdash B}
    \infer4{\Gamma, \gr{r_1}\Delta_1, \ldots, \gr{r_n}\Delta_n \vdash B}
  }
\]

We can derive \citeauthor{petricek-thesis}-style multi-nary substitution as a
corollary of my simultaneous substitution, using reasoning similar to that of
\cref{thm:single-sub}.

\begin{corollary}\label{thm:petricek-sub}
  \Citeauthor{petricek-thesis}'s multi-nary substitution, as stated above, is
  admissible in $\name$.
\end{corollary}
\begin{proof}
  It is enough to provide a substitution of type
  \[
    \Gamma, \gr{r_1}\Delta_1, \ldots, \gr{r_n}\Delta_n
    \env\vdash \Gamma, \gr{r_1}A_1, \ldots, \gr{r_n}A_n.
  \]
  To do this, we use \cref{thm:construct-env} repeatedly, leaving us needing a
  substitution of type
  $\Gamma, \gr0\Delta_1, \ldots, \gr0\Delta_n \env\vdash \Gamma$ and terms of
  types
  \begin{align*}
    \gr0\gamma, \Delta_1, \gr0\delta_2, &\ldots, \gr0\delta_{n-1}, \gr0\delta_n
    \vdash A_1 \\
    &\vdots \\
    \gr0\gamma, \gr0\delta_1, \gr0\delta_2, &\ldots, \gr0\delta_{n-1}, \Delta_n
    \vdash A_n.
  \end{align*}
  The identity substitution and weakening by $\gr0$-annotated variables is
  enough to make these requirements line up with the given hypotheses.
\end{proof}

My substitution principle is stronger than \citeauthor{petricek-thesis}'s.
Where \citeauthor{petricek-thesis} requires that distinct variables be
available for each hypothesis, I allow for separation of uses via addition of
contexts.
Below is a prototypical example.

\begin{example}
  Let $\Ann \coloneqq \plr{\mathbb N, =, 0, +, 1, \times}$, the exact
  usage-counting posemiring.
  Then, we can construct a substitution $\rho : \gr2A \env\vdash \gr1A, \gr1A$,
  yielding a transformation of terms of the following form:
  \[
    \ebrule{%
      \hypo{\gr1A, \gr1A \vdash B}
      \infer1{\gr2A \vdash B}
    }.
  \]
  To construct $\rho$, we use \cref{thm:construct-env} case
  $\sep(\rightarrowtriangle)$, using the fact that $\gr2 \leq \gr1 + \gr1$.
  From there, two identity substitutions suffice.
  The action of $\rho$ on terms is to merge the two variables into one.
  Note that a renaming, rather than a substitution, would also suffice.
\end{example}

Most notably, my (single) substitution principle more naturally fits the
requirement we would have for the reduct of the $\beta$-rule for functions in
$\name$, whereas \citeauthor{petricek-thesis}'s substitution principle would
need some additional transformation for it to fit properly.
This comes from the fact that the $\name$ function application rule introduces
an algebraic ($+$) separation between its premises, whereas
\citeauthor{petricek-thesis}'s substitution principle separates premises only
via concatenation.
