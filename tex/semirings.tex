\chapter{Usage restriction via semirings}\label{sec:semirings}

The methods described in \cref{sec:simple} for simply typed $\lambda$-calculus
make crucial use of \emph{weakening} --- the fact that if we have
$\Gamma \vdash A$, then we also have $\Gamma, \Delta \vdash A$.
We use this property to update environments as they go under binders.
However, as we saw in \cref{sec:linearity}, there are interesting calculi in
which general weakening does not hold.
As such, one of the aims of this chapter will be to find a form of weakening
applicable to variables of any type, while essentially retaining linearity
(as opposed to affinity).

As explained by \citet{McBride16}, the first insight is to, instead of
removing variables from the context of certain subterms, add an annotation to
free variables saying whether they are to be used or not.
We may use an annotation $\gr0$ on variables that are not to be used, and an
annotation $\gr1$ on variables that are to be used.
This convention lets us transcribe the usual $\otimes$-introduction rule
(left) as a rule with usage annotations (right).
I use lowercase $\gamma$ and $\delta$ to emphasise that the whole context now
looks like $\grP\gamma$, where $\grP$ is a list of usage annotations
$\gr{r_1}, \ldots, \gr{r_m}$ and $\gamma$ is a list of types $A_1, \ldots, A_m$
of the same length.
Explicit contexts will usually be written with usage annotations and types
interspersed, as $\gr{r_1}A_1, \ldots, \gr{r_m}A_m$.
I use $\gr r\gamma$ to abbreviate $\gr rx_1, \ldots, \gr rx_m$.

\[
  \ebrule{%
    \hypo{\Gamma \vdash A}
    \hypo{\Delta \vdash B}
    \infer2{\Gamma, \Delta \vdash A \otimes B}
  }
  \quad\rightsquigarrow\quad
  \ebrule{%
    \hypo{\gr1\gamma, \gr0\delta \vdash A}
    \hypo{\gr0\gamma, \gr1\delta \vdash B}
    \infer2{\gr1\gamma, \gr1\delta \vdash A \otimes B}
  }
\]

What the existence of $\gr0$ has bought us is that variables never go out of
scope in subterms; rather, we lose the ability to use certain variables that
remain in scope.
Additionally, we recover a form of weakening: if $\Gamma \vdash A$, then also
$\Gamma, \gr0\delta \vdash A$, because the resulting term indeed uses no
variables from $\delta$.
A rigorous proof of weakening for terms will come in \cref{sec:lrsub}.

If we follow the DILL style of variable management, there are not just the two
states \emph{used} ($\gr1$) and \emph{not used} ($\gr0$), but also
\emph{usable unrestrictedly}.
If we assign unrestricted (or \emph{intuitionistic}) variables an annotation
$\gr\omega$, we can make the following transcription of the DILL
$\otimes$-introduction rule.

\[
  \ebrule{%
    \hypo{\Theta; \Gamma \vdash A}
    \hypo{\Theta; \Delta \vdash B}
    \infer2{\Theta; \Gamma, \Delta \vdash A \otimes B}
  }
  \quad\rightsquigarrow\quad
  \ebrule{%
    \hypo{\gr\omega\theta, \gr1\gamma, \gr0\delta \vdash A}
    \hypo{\gr\omega\theta, \gr0\gamma, \gr1\delta \vdash B}
    \infer2{\gr\omega\theta, \gr1\gamma, \gr1\delta \vdash A \otimes B}
  }
\]

To conceptualise the criteria on the usage annotations involved in this rule,
I introduce a formal $+$ operation on usage annotations.
The rule stated above relies on the facts that $\gr1 + \gr0 = \gr1$,
$\gr0 + \gr1 = \gr1$, and $\gr\omega + \gr\omega = \gr\omega$.
Addition lifts pointwise to vectors of usage annotations (the green capital
calligraphic $\grP$, $\grQ$, and $\grR$).
It is worth noting now that, thanks to the fact that $\gr0 + \gr0 = \gr0$, the
rule on the right below now accepts $\gr0$-annotated variables in its
conclusion, which is essential for weakening to be admissible.

\[
  \ebrule{%
    \hypo{\gr\omega\theta, \gr1\gamma, \gr0\delta \vdash A}
    \hypo{\gr\omega\theta, \gr0\gamma, \gr1\delta \vdash B}
    \infer2{\gr\omega\theta, \gr1\gamma, \gr1\delta \vdash A \otimes B}
  }
  \quad\rightsquigarrow\quad
  \ebrule{%
    \hypo{\grR = \grP + \grQ}
    \hypo{\grP\gamma \vdash A}
    \hypo{\grQ\gamma \vdash B}
    \infer3{\grR\gamma \vdash A \otimes B}
  }
\]

Some other transcriptions are as follows.
I unify the variable rules (the one for linear variables and the one for
intuitionistic variables) by introducing a coercibility ordering $\leq$ on usage
annotations.
We have $\gr\omega \leq \gr1$ because an intuitionistic variable can fill the
demand of a linear variable by dereliction.
We also have $\gr\omega \leq \gr0$, because intuitionistic variables can be
weakened away like $\gr0$-annotated variables.
All together, this means that at the unified variable rule, the variable being
used must have annotation less than or equal to $\gr1$, and every other variable
must have annotation less than or equal to $\gr0$.
Thus the basis vector $\langle x \rvert$ describes the upper bound on the
current usage context $\grR$.

\[
  \ebrule{%
    \infer0{\Theta; A \vdash A}
  }
  \quad\rightsquigarrow\quad
  \ebrule{%
    \infer0{\gr\omega\theta, \gr1A, \gr0\delta \vdash A}
  }
  \quad\rightsquigarrow\quad
  \ebrule{%
    \hypo{\grR \leq \bra x}
    \hypo{\gamma_x = A}
    \infer2{\grR\gamma \vdash A}
  }
\]

\[
  \ebrule{%
    \infer0{\Theta, A; {\cdot} \vdash A}
  }
  \quad\rightsquigarrow\quad
  \ebrule{%
    \infer0{\gr\omega\theta, \gr\omega A, \gr0\delta \vdash A}
  }
  \quad\rightsquigarrow\quad
  \ebrule{%
    \hypo{\grR \leq \bra x}
    \hypo{\gamma_x = A}
    \infer2{\grR\gamma \vdash A}
  }
\]

The final interesting rule form to cover is that found in DILL's
$\oc$-introduction rule.
I choose to see DILL's $\oc$-introduction as an $\gr\omega$-ary counterpart to
$\otimes$-introduction, though with the same premise each time rather than
$\gr\omega$-many premises.
This explanation recovers the behaviour that only $\gr\omega$- and
$\gr0$-annotated variables can appear in the conclusion of $\oc$-introduction,
and also justifies the designation of multiplication (vector scaling) as the
algebraic operation controlling the $\oc$-modality.

\[
  \ebrule{%
    \hypo{\Theta; {\cdot} \vdash A}
    \infer1{\Theta; {\cdot} \vdash \oc A}
  }
  \quad\rightsquigarrow\quad
  \ebrule{%
    \hypo{\gr\omega\theta, \gr0\delta \vdash A}
    \infer1{\gr\omega\theta, \gr0\delta \vdash \oc A}
  }
  \quad\rightsquigarrow\quad
  \ebrule{%
    \hypo{\grR \leq \gr\omega\grP}
    \hypo{\grP\gamma \vdash A}
    \infer2{\grR\gamma \vdash \oc_{\gr\omega} A}
  }
\]

In summary, the structure we have required of the set of usage annotations is
that they have addition (for $\otimes$-introduction and similar rules),
multiplication (for $\oc$-introduction), a $1$ (for a variable being used), a
$0$ (for a variable being discarded), and an ordering (allowing for subsumption
of usage restrictions).
Together, these form a \emph{partially ordered semiring} (posemiring), the laws
of which are both supported by examples and necessary for the syntax to be well
behaved.
\todo{Refer back to preliminaries for definition of posemiring}
For concreteness, I collect together the definition of the
$\{\gr0, \gr1, \gr\omega\}$ posemiring I have been using so far.

\begin{example}\label{def:lin-semiring}
  The \emph{$\{\gr0, \gr1, \gr\omega\}$ semiring}, also known as the
  \emph{linearity semiring}, has the operations given as follows, with
  $0 \coloneqq \gr0$ and $1 \coloneqq \gr1$:

  \makebox[\textwidth][s]{
    \begin{tabular}{c|ccc}
      $+$ & $\gr0$ & $\gr1$ & $\gr\omega$ \\ \hline
      $\gr0$ & $\gr0$ & $\gr1$ & $\gr\omega$ \\
      $\gr1$ & $\gr1$ & $\gr\omega$ & $\gr\omega$ \\
      $\gr\omega$ & $\gr\omega$ & $\gr\omega$ & $\gr\omega$ \\
    \end{tabular}
    \begin{tabular}{c|ccc}
      $*$ & $\gr0$ & $\gr1$ & $\gr\omega$ \\ \hline
      $\gr0$ & $\gr0$ & $\gr0$ & $\gr0$ \\
      $\gr1$ & $\gr0$ & $\gr1$ & $\gr\omega$ \\
      $\gr\omega$ & $\gr0$ & $\gr\omega$ & $\gr\omega$ \\
    \end{tabular}
    \begin{tikzpicture}[baseline]
      \node(omega) at (0,0) {$\gr\omega$};
      \node(0) [above left of=omega] {$\gr0$};
      \node(1) [above right of=omega] {$\gr1$};

      \draw (omega) -- (0);
      \draw (omega) -- (1);
    \end{tikzpicture}
  }
\end{example}

%In \cref{sec:simple}, we saw that the logical rules of simply typed
%$\lambda$-calculus can be described in terms of three basic premise combinators:
%$\dot1$, standing for no premises; $\dottimes$, allowing for multiple premises
%in the same context; and $\Theta \vdash A$, requiring a subterm of type $A$
%having bound the extra variables from context $\Theta$.
%However, we remember from \cref{sec:linearity} that in a substructural setting,
%we do not always want to copy assumptions for use in all subterms.
%This motivates me to introduce the additional premise combinators $I^*$, $\sep$,
%and $\cdot$ in \cref{sec:lnd}, allowing for the modes of usage exhibited in
%the introduction rules for $I$, $\otimes$, and $\oc$, respectively.

The rest of this chapter proceeds as follows.
In \cref{sec:lr}, I define the usage-annotated calculus \name{}, which has
appeared previously in my work with Atkey~\cite{WA21}, and can be seen as a
simply typed version of Atkey's dependently typed calculus QTT~\cite{Atkey18}.
The idea of \name{} is to augment simply typed $\lambda$-calculus with
annotations on free variables, and give enough types to manipulate these
annotations.
I use \cref{sec:lnd} to introduce some notation that allows us to restate the
typing rules of \name{} so as to not mention contexts explicitly.
The first problem I tackle with this new syntax is proving admissibility of
substitution.
I follow the methodology from \cref{sec:kits}, which

\section{A usage-annotated calculus}\label{sec:lr}
In this section, I introduce the syntax of the type theory \name{}, which makes
use of posemiring usage annotations.
I use this syntax to write some example programs, which will motivate the
denotational semantics explored in \cref{sec:wrel}.
For the rest of this thesis, \name{} will serve as both a prototypical
usage-constrained syntax and a target of semantic analyses.

The calculus \name{} is similar in spirit to intuitionistic linear logic (ILL).
The types of \name{}, listed in \cref{fig:lr-types}, are almost identical
to those of ILL, differing only in the exponential modality $\oc$
(read ``bang'').
In particular, I include distinguished tensor- and with-product types
($\otimes$, $\with$) and their units ($I$, $\top$), function types
($\multimap$), additive sum types and their unit ($\oplus$, $0$), and the
graded modality $\oc_{\gr r}$.
The idea of $\oc_{\gr r}$ is to internalise an annotation of $\gr r$ on a
variable in the context.
In this position, an assumption of type $\oc_{\gr r} A$ acts like an assumption
of type $A$ that is to be used according to $\gr r$ rather than the standard
$\gr1$.

\begin{figure}
  \begin{displaymath}
    A, B, C \Coloneqq I \mid A \otimes B \mid A \multimap B \mid \top
    \mid A \with B \mid 0 \mid A \oplus B \mid \oc_{\gr r} A
  \end{displaymath}
  \caption{The types of \name{}}
  \label{fig:lr-types}
\end{figure}

\begin{figure}
  \begin{displaymath}
    \begin{prooftree}
      \hypo{\gamma \ni x : A}
      \hypo{\grP \le \langle x \rvert}
      \infer2[Var]{\grP\gamma \vdash A}
    \end{prooftree}
  \end{displaymath}

  \begin{displaymath}
    \begin{matrix}
      \begin{prooftree}
        \hypo{\grP \le \gr0}
        \infer1[$I$-I]{\grP\gamma \vdash I}
      \end{prooftree}
      &&
      \begin{prooftree}
        \hypo{\grR \le \grP + \grQ}
        \hypo{\grP\gamma \vdash I}
        \hypo{\grQ\gamma \vdash C}
        \infer3[$I$-E]{\grR\gamma \vdash C}
      \end{prooftree}
      \\\\
      \begin{prooftree}
        \hypo{\grR \le \grP + \grQ}
        \hypo{%
          \begin{matrix}
            \grP\gamma \vdash A \\ \grQ\gamma \vdash B
          \end{matrix}%
        }
        \infer2[$\otimes$-I]{\grR\gamma \vdash A \otimes B}
      \end{prooftree}
      &&
      \begin{prooftree}
        \hypo{%
          \begin{matrix}
            \grR \le \grP + \grQ \\ \grP\gamma \vdash A \otimes B
          \end{matrix}%
        }
        \hypo{\grQ\gamma, \gr1A, \gr1B \vdash C}
        \infer2[$\otimes$-E]{\grR\gamma \vdash C}
      \end{prooftree}
      \\\\
      \begin{prooftree}
        \hypo{\grR\gamma, \gr1A \vdash B}
        \infer1[$\multimap$-I]{\grR\gamma \vdash A \multimap B}
      \end{prooftree}
      &&
      \begin{prooftree}
        \hypo{\grR \le \grP + \grQ}
        \hypo{\grP\gamma \vdash A \multimap B}
        \hypo{\grQ\gamma \vdash A}
        \infer3[$\multimap$-E]{\grR\gamma \vdash B}
      \end{prooftree}
      \\\\
      \begin{prooftree}
        \infer0[$\top$-I]{\grR\gamma \vdash \top}
      \end{prooftree}
      &&
      \textrm{(no $\top$-E)}
      \\\\
      \begin{prooftree}
        \hypo{\grR\gamma \vdash A}
        \hypo{\grR\gamma \vdash B}
        \infer2[$\with$-I]{\grR\gamma \vdash A \with B}
      \end{prooftree}
      &&
      \begin{prooftree}
        \hypo{\grR\gamma \vdash A_0 \with A_1}
        \infer1[$\with$-E$_i$, for $i \in \{0,1\}$]{\grR\gamma \vdash A_i}
      \end{prooftree}
      \\\\
      \textrm{(no $0$-I)}
      &&
      \begin{prooftree}
        \hypo{\grR \le \grP + \grQ}
        \hypo{\grP\gamma \vdash 0}
        \infer2[$0$-E]{\grR\gamma \vdash C}
      \end{prooftree}
      \\\\
      \begin{prooftree}
        \hypo{\grR\gamma \vdash A_i}
        \infer1[$\oplus$-I$_i$, for $i \in \{0,1\}$]%
        {\grR\gamma \vdash A_0 \oplus A_1}
      \end{prooftree}
      &&
      \begin{prooftree}
        \hypo{%
          \begin{matrix}
            \grR \le \grP + \grQ \\ \grP\gamma \vdash A \oplus B
          \end{matrix}%
        }
        \hypo{%
          \begin{matrix}
            \grQ\gamma, \gr1A \vdash C \\ \grQ\gamma, \gr1B \vdash C
          \end{matrix}%
        }
        \infer2[$\oplus$-E]{\grR\gamma \vdash C}
      \end{prooftree}
      \\\\
      \begin{prooftree}
        \hypo{\grR \le \gr r\grP}
        \hypo{\grP\gamma \vdash A}
        \infer2[$\oc$-I]{\grR\gamma \vdash \oc\gr rA}
      \end{prooftree}
      &&
      \begin{prooftree}
        \hypo{\grR \le \grP + \grQ}
        \hypo{\grP\gamma \vdash \oc\gr rA}
        \hypo{\grQ\gamma, \gr rA \vdash C}
        \infer3[$\oc$-E]{\grR\gamma \vdash C}
      \end{prooftree}
    \end{matrix}
  \end{displaymath}
  \caption{\name{}}
  \label{fig:lr}
\end{figure}

I will not cover the operational semantics or equational theory of \name{} in
this thesis.
I will discuss a denotational semantics in \cref{sec:wrel}.

The following features are of note.

\paragraph{Subusaging}
Several typing rules contain constraints of the form $\grP \leq \grQ$, for
certain usage vectors $\grP$ and $\grQ$.
To a first approximation, $\leq$ can simply be read as $=$.
The point of using $\leq$ rather than $=$ is to allow for \emph{subusaging},
i.e., subsumption of usage annotations.
For usage annotations $\gr r$ and $\gr s$, the inequality $\gr r \leq \gr s$
states that an assumption with annotation $\gr r$ can be used wherever an
assumption with annotation $\gr s$ is required.
A mnemonic is that $\gr r$ is less specific than $\gr s$.
The principle is reflected by the admissible subusaging rule.

\[
  \begin{prooftree}
    \hypo{\grP \leq \grQ}
    \hypo{\grQ\gamma \vdash A}
    \infer2[Subuse]{\grP\gamma \vdash A}
  \end{prooftree}
\]

Subusaging is essential to nearly all interesting choices of $\Ann$.
For example, we can capture intuitionistic linear logic by choosing $\Ann$ to
be $\{\gr0 > \gr\omega < \gr1\}$.
This allows variables annotated $\gr\omega$ (``unrestricted'') to be both
weakened/discarded (because $\gr\omega \leq \gr0$) and derelicted/used
(because $\gr\omega \leq \gr1$).

\paragraph{Tensor- and with-products}
Like intuitionistic linear logic (ILL), \name{} distinguishes tensor-products
($A \otimes B$) from with-products ($A \with B$).
Whereas in ILL, rules like $\otimes$-introduction involve splitting the
assumptions between the two subterms, in \name{}, this splitting is done by
choosing usage annotations which add up to the usage annotations of the
conclusion.
For example, we can derive $\vdash A \otimes B \multimap B \otimes A$ as
follows.
Notice that the assumption $A \otimes B$ is still present in all subderivations,
even after it has been ``used up''.
The only thing that stops us using the assumption again is that, for a general
choice of $\Ann$, we do not have $\gr0 \leq \gr1$ or $\gr1 \leq \gr1 + \gr1$.

\begin{small}
  \[
    \nabla \coloneqq
    \begin{prooftree}
      \infer0{\plr{\gr0\;\gr1\;\gr1} \leq
        \plr{\gr0\;\gr0\;\gr1} + \plr{\gr0\;\gr1\;\gr0}}
      \infer0{\plr{\gr0\;\gr0\;\gr1} \leq \plr{\gr0\;\gr0\;\gr1}}
      \infer1[Var]{\gr0\plr{A \otimes B}, \gr0A, \gr1B \vdash B}
      \infer0{\plr{\gr0\;\gr1\;\gr0} \leq \plr{\gr0\;\gr1\;\gr0}}
      \infer1[Var]{\gr0\plr{A \otimes B}, \gr1A, \gr0B \vdash A}
      \infer3[$\otimes$-I]%
      {\gr0\plr{A \otimes B}, \gr1A, \gr1B \vdash B \otimes A}
    \end{prooftree}
  \]

  \[
    \begin{prooftree}
      \infer0{\plr{\gr1} \leq \plr{\gr1} + \plr{\gr0}}
      \infer0{\plr{\gr1} \leq \plr{\gr1}}
      \infer1[Var]{\gr1\plr{A \otimes B} \vdash A \otimes B}
      \hypo{\nabla}
      \infer[no rule]1{\gr0\plr{A \otimes B}, \gr1A, \gr1B \vdash B \otimes A}
      \infer3[$\otimes$-E]{\gr1\plr{A \otimes B} \vdash B \otimes A}
      \infer1[$\multimap$-I]{\vdash A \otimes B \multimap B \otimes A}
    \end{prooftree}
  \]
\end{small}

\begin{example}
  Let $A \multimapboth B$ abbreviate
  $\plr{A \multimap B} \with \plr{B \multimap A}$.
  Then the following judgements hold for any partially ordered semiring.
  Derivations are left as an exercise to the reader.
  \begin{itemize}
    \item $\vdash A \oplus A \multimap A$
    \item $\vdash A \multimap A \with A$
    \item $\vdash A \oplus 0 \multimapboth A$
    \item $\vdash A \otimes 0 \multimapboth 0$
    \item $\vdash \oc\gr1A \multimapboth A$
    \item If $\gr r \leq \gr s$, then $\vdash \oc\gr rA \multimap \oc\gr sA$
  \end{itemize}
\end{example}

To get a feeling for \name{}, I will temporarily fix
$\Ann \coloneqq (\mathbb N, =, 0, +, 1, \times)$.
That is, the usual semiring of natural numbers with ordering given by equality.
Under this discipline, the usage constraints enforce a form of exact usage
counting.

\begin{example}
  The following judgements hold for the natural number semiring.
  Derivations are left as an exercise to the reader.
  \begin{itemize}
    \item $\vdash \oc\gr2A \multimap A \otimes A$
    \item $\vdash \oc\gr5A \multimap \oc\gr2A \otimes \oc\gr3A$
  \end{itemize}
\end{example}

\subsection{Other posemirings}\label{sec:example-posemirings}

Now that we have seen the role of usage annotations in $\name$, I will give more
examples of posemirings for tracking interesting usage patterns.

\begin{example}\label{def:trivial-posemiring}
  The singleton set gives rise to a posemiring in a unique way.
  When the usage annotations of $\name$ are taken from this trivial posemiring,
  we recover a version of intuitionistic simply typed $\lambda$-calculus.
\end{example}

{\color{red}TODO: cite rntz}
\begin{example}\label{def:monotonicity-posemiring}
  The \emph{monotonicity} posemiring is defined over the set of symbols
  $\{\gr{\wn\wn}, \gr{\uparrow\uparrow}, \gr{\downarrow\downarrow},
  \gr{\sim\sim}\}$, with $0 \coloneqq \gr{\wn\wn}$,
  $1 \coloneqq \gr{\uparrow\uparrow}$, and the following operations:

  \makebox[\textwidth][s]{
    \begin{tabular}{c|cccc}
      $+$ & $\gr{\wn\wn}$ & $\gr{\uparrow\uparrow}$ & $\gr{\downarrow\downarrow}$ & $\gr{\sim\sim}$ \\ \hline
      $\gr{\wn\wn}$ & $\gr{\wn\wn}$ & $\gr{\uparrow\uparrow}$ & $\gr{\downarrow\downarrow}$  & $\gr{\sim\sim}$ \\
      $\gr{\uparrow\uparrow}$ & $\gr{\uparrow\uparrow}$ & $\gr{\uparrow\uparrow}$ & $\gr{\sim\sim}$ & $\gr{\sim\sim}$ \\
      $\gr{\downarrow\downarrow}$ & $\gr{\downarrow\downarrow}$ & $\gr{\sim\sim}$ & $\gr{\downarrow\downarrow}$ & $\gr{\sim\sim}$ \\
      $\gr{\sim\sim}$ & $\gr{\sim\sim}$  & $\gr{\sim\sim}$ & $\gr{\sim\sim}$ & $\gr{\sim\sim}$ \\
    \end{tabular}
    \begin{tabular}{c|cccc}
      $*$ & $\gr{\wn\wn}$ & $\gr{\uparrow\uparrow}$ & $\gr{\downarrow\downarrow}$ & $\gr{\sim\sim}$ \\ \hline
      $\gr{\wn\wn}$ & $\gr{\wn\wn}$ & $\gr{\wn\wn}$ & $\gr{\wn\wn}$  & $\gr{\wn\wn}$ \\
      $\gr{\uparrow\uparrow}$ & $\gr{\wn\wn}$ & $\gr{\uparrow\uparrow}$ & $\gr{\downarrow\downarrow}$ & $\gr{\sim\sim}$ \\
      $\gr{\downarrow\downarrow}$ & $\gr{\wn\wn}$ & $\gr{\downarrow\downarrow}$ & $\gr{\uparrow\uparrow}$ & $\gr{\sim\sim}$ \\
      $\gr{\sim\sim}$ & $\gr{\wn\wn}$  & $\gr{\sim\sim}$ & $\gr{\sim\sim}$ & $\gr{\sim\sim}$ \\
    \end{tabular}
    \begin{tikzpicture}[baseline]
      \node(omega) at (0,-1) {$\gr{\sim\sim}$};
      \node(0) [above left of=omega] {$\gr{\uparrow\uparrow}$};
      \node(1) [above right of=omega] {$\gr{\downarrow\downarrow}$};
      \node(qq) [above right of=0] {$\gr{\wn\wn}$};

      \draw (omega) -- (0);
      \draw (omega) -- (1);
      \draw (0) -- (qq);
      \draw (1) -- (qq);
    \end{tikzpicture}
  }

  The idea is that each symbol represents the possible \emph{variance} of an
  input (free variable) with respect to some partial ordering on a semantic
  domain of elements.
  $\gr{\uparrow\uparrow}$ represents covariance (if that input goes up, the
  output goes up), $\gr{\downarrow\downarrow}$ represents contravariance
  (if that input goes \emph{down}, the output goes up), $\gr{\sim\sim}$ gives no
  guarantees (if that input remains constant, the output (trivially) goes up),
  and $\gr{\wn\wn}$ says that that input is irrelevant (whatever changes are
  made to that input, the output (trivially) goes up).

  Addition represents an intersection of guarantees.
  For example, if a variable is used covariantly in one subterm and
  contravariantly in another, we can only make the trivial guaratee represented
  by $\gr{\sim\sim}$.
  Multiplication is mainly interesting for multiplication by
  $\gr{\downarrow\downarrow}$, which flips the variance on any other annotation.
  As such, $\oc\gr{\downarrow\downarrow}A$ represents a contravariant $A$.
\end{example}

\begin{example}\label{def:sensitivity-posemiring}
  The \emph{sensitivity} posemiring~\citep{reed10distance} is given by
  $(\mathbb R^+, \geq, \gr0, +, \gr1, \times)$, where $\mathbb R^+$ is the
  non-negative real numbers extended with $\gr\infty$ (distances), and the rest
  of the structure comes from the standard operations on real numbers (except
  that $\gr0 \times \gr\infty = \gr\infty \times \gr0 = \gr0$).
  Note that the order is reversed, making $\gr\infty$ coercible to any other
  annotation and anything coercible to $\gr0$.

  This posemiring can be used for sensitivity analysis, where we want to bound
  the effect of a perturbation on inputs in terms of some semantic notion of
  distance between values.
  An annotation $\gr r$ says that if that input is perturbed by at most $r$,
  then the output will change by at most $1$.
  An $\gr\infty$-annotated variable gives the very strong guarantee that any
  change in the input will make a minimal change to the output, while a
  $\gr0$-annotated variable provides no guarantee at all.
  Addition forbids general contraction, which would otherwise allow arbitrary
  finite blow-up of the effect of any non-$\gr0$-annotated variable.
  However, the ordering, with $\gr0$ at the top, means that we have general
  weakening, so the resulting sensitivity calculus has an affine flavour.
\end{example}

\section{Bunched connectives}\label{sec:lnd}
%This should stay fairly general, so it could be compiled to a DILL-style
%calculus as well as the one with usage annotations.
%Our framework compiles a ND system down to a sequent calculus.
%It isolates the building blocks of typing rules ($\vdash$, $\wedge$, $*$,
%$\Box$, $\cdot$) so that we can do more generic programming (e.g., usage
%checker).

The typing rules presented in the previous section contain a lot of detail and
repeated patterns.
For example, nearly half of the rules include the premise
$\grR \leq \grP + \grQ$.
Also, the presence of usage annotations, which are often different in different
parts of a rule, means that we keep repeating the context.
Explicit contexts go against the style of natural deduction, which is based
around being parametric in the context, so that substitution is agnostic to the
details of typing rules.

To encapsulate the repeated patterns and facilitate an implicit context style,
I introduce the \emph{bunched connectives}.
These are inspired by bunched logic~\cite{oHP99}, and will not only be used for
stating the syntax, but will reappear in the semantics.
The idea is to generalise the space between premises from Gentzen's natural
deduction so as to allow for any linear manipulations of usage annotations.
Among other things, this generalisation will allow us to distinguish between
$\with$-introduction and $\otimes$-introduction by a choice of connective:
either \emph{sharing} or \emph{separating} conjunction.
Similar connectives, but with different interpretations, could be used to
define other linear-like type theories, like DILL, but here I will focus on the
usage annotation style.
The interpretations I use are defined in \cref{fig:bunched}.

\begin{figure}
  \begin{align*}
    \dot1\,\grR &\coloneqq 1 \\
    (T \dottimes U)\,\grR &\coloneqq T\,\grR \times U\,\grR \\
    (T \dotto U)\,\grR &\coloneqq T\,\grR \to U\,\grR \\
    I^*\,\grR &\coloneqq \grR \leq \gr0 \\
    (T \sep U)\,\grR &\coloneqq \Sigma \grP,\grQ.~ \plr{\grR \leq \grP + \grQ}
                       \times T\,\grP \times U\,\grQ \\
    (\gr r \cdot T)\,\grR &\coloneqq \Sigma \grP.~ \plr{\grR \leq \gr r\grP}
                       \times T\,\grP
  \end{align*}
  \caption{The bunched connectives}
  \label{fig:bunched}
\end{figure}

The simplest connectives are those we've already seen for intuitionistic
systems --- $\dot1$, $\dottimes$, and $\dotto$.
The absence of premises is encoded by $\dot1$, while the space between premises
sharing a context is encoded by $\dottimes$.
When we interpret a rule as a constructor of an inductive definition, $\dotto$
interprets the horizontal line, reflecting the fact that the usage annotations
we start off with in the premises are those of the conclusion.
The prototypical rules that use $\dot1$ and $\dottimes$ are the introduction
rules for $\top$ and $\with$, respectively.

\begin{align*}
  \begin{prooftree}
    \infer0{\grR\Gamma \vdash \top}
  \end{prooftree}
  &\quad\rightsquigarrow\quad
  \begin{prooftree}
    \set{separation=0.75em}
    \hypo{\dot1}
    \infer1{\vdash \top}
  \end{prooftree}
  \\\\
  \begin{prooftree}
    \hypo{\grR\Gamma \vdash A}
    \hypo{\grR\Gamma \vdash B}
    \infer2{\grR\Gamma \vdash A \with B}
  \end{prooftree}
  &\quad\rightsquigarrow\quad
  \begin{prooftree}
    \set{separation=0.75em}
    \hypo{\vdash A}
    \hypo{\dottimes}
    \hypo{\vdash B}
    \infer3{\vdash A \with B}
  \end{prooftree}
\end{align*}

The rest of the bunched connectives are for linear decompositions of the usage
annotations.
The three basic left semimodule operators --- zero, addition, and left-scaling
--- each get a bunched connective --- $I^*$, $\sep$, and $\gr r \cdot {}$,
respectively.
The prototypical typing rules for each of these three connectives are the
introduction rules for $I$, $\otimes$, and $\oc_{\gr r}$, respectively.

\begin{align*}
  \begin{prooftree}
    \hypo{\grR \leq \gr0}
    \infer1{\grR\Gamma \vdash I}
  \end{prooftree}
  &\quad\rightsquigarrow\quad
  \begin{prooftree}
    \set{separation=0.75em}
    \hypo{I^*}
    \infer1{\vdash I}
  \end{prooftree}
  \\\\
  \begin{prooftree}
    \hypo{\grP\Gamma \vdash A}
    \hypo{\grQ\Gamma \vdash B}
    \hypo{\grR \leq \grP + \grQ}
    \infer3{\grR\Gamma \vdash A \otimes B}
  \end{prooftree}
  &\quad\rightsquigarrow\quad
  \begin{prooftree}
    \set{separation=0.75em}
    \hypo{\vdash A}
    \hypo{\sep}
    \hypo{\vdash B}
    \infer3{\vdash A \otimes B}
  \end{prooftree}
  \\\\
  \begin{prooftree}
    \hypo{\grP\Gamma \vdash A}
    \hypo{\grR \leq \gr r\grP}
    \infer2{\grR\Gamma \vdash \oc_{\gr r} A}
  \end{prooftree}
  &\quad\rightsquigarrow\quad
  \begin{prooftree}
    \set{separation=0.75em}
    \hypo{\gr r \cdot (\vdash A)}
    \infer1{\vdash \oc_{\gr r} A}
  \end{prooftree}
\end{align*}

The full system \name{} is stated in terms of bunched connectives in
\cref{fig:lr-bunched}.
%I have not included a variable rule because, like with intuitionistic
%natural deduction,

These bunched connectives should not be confused for the linear connectives
they appear in the introduction rules of.
For example, it would make sense to define a bunched connective $\dotplus$,
defined analogously to $\dottimes$.
This $\dotplus$ could be used to rephrase the introduction rules for $\oplus$.
We then have maps both ways between $T \dottimes (U \dotplus V)$ and
$(T \dottimes U) \dotplus (T \dottimes V)$, reminiscent of bunched logic,
whereas linear logic only has a map from $(A \with B) \oplus (A \with C)$ to
$A \with (B \oplus C)$, and not a map the other way.
%For example, linearly we have
%$\oc_{\gr r}(A \with B) \simeq \oc_{\gr r} A \otimes \oc_{\gr r} B$, while the
%expressions $\gr r \cdot (A \dottimes B)$ and $\gr r \cdot A \sep \gr r \cdot B$
%are generally incomparable.
We will also see later\todo{Forward reference} that, while
$(\dottimes, \dotto)$ gives a Cartesian
closed structure, $\sep$ is monoidally closed, with an internal hom $\wand$.
Indeed, the Cartesian closed structure is sufficient to give us the interaction
between $\dotplus$ and $\dottimes$, by the fact that $A \dottimes {-}$ is a
left adjoint, and therefore preserves colimits.
Looking at the interpretations, the connection with bunched logic makes sense.
Instead of the partial commutative monoid (representing heaps) found in
standard semantics of bunched logic, we have a left semimodule of usage
contexts, which we are similarly interested in splitting and sharing between
various subterms.

\begin{figure}
  \ebproofset{separation=0.75em}
  \begin{displaymath}
    \begin{prooftree}
      \hypo{\sqni A}
      \infer1[Var]{\vdash A}
    \end{prooftree}
  \end{displaymath}

  \begin{displaymath}
    \begin{matrix}
      \begin{prooftree}
        \hypo{I^*}
        \infer1[$I$-I]{\vdash I}
      \end{prooftree}
      &&
      \begin{prooftree}
        \hypo{\vdash I}
        \hypo{\sep}
        \hypo{\vdash C}
        \infer3[$I$-E]{\vdash C}
      \end{prooftree}
      \\\\
      \begin{prooftree}
        \hypo{\vdash A}
        \hypo{\sep}
        \hypo{\vdash B}
        \infer3[$\otimes$-I]{\vdash A \otimes B}
      \end{prooftree}
      &&
      \begin{prooftree}
        \hypo{\vdash A \otimes B}
        \hypo{\sep}
        \hypo{\gr1A, \gr1B \vdash C}
        \infer3[$\otimes$-E]{\vdash C}
      \end{prooftree}
      \\\\
      \begin{prooftree}
        \hypo{\gr1A \vdash B}
        \infer1[$\multimap$-I]{\vdash A \multimap B}
      \end{prooftree}
      &&
      \begin{prooftree}
        \hypo{\vdash A \multimap B}
        \hypo{\sep}
        \hypo{\vdash A}
        \infer3[$\multimap$-E]{\vdash B}
      \end{prooftree}
      \\\\
      \begin{prooftree}
        \hypo{\dot1}
        \infer1[$\top$-I]{\vdash \top}
      \end{prooftree}
      &&
      \textrm{(no $\top$-E)}
      \\\\
      \begin{prooftree}
        \hypo{\vdash A}
        \hypo{\dottimes}
        \hypo{\vdash B}
        \infer3[$\with$-I]{\vdash A \with B}
      \end{prooftree}
      &&
      \begin{prooftree}
        \hypo{\vdash A_0 \with A_1}
        \infer1[$\with$-E$_i$, for $i \in \{0,1\}$]{\vdash A_i}
      \end{prooftree}
      \\\\
      \textrm{(no $0$-I)}
      &&
      \begin{prooftree}
        \hypo{\vdash 0}
        \hypo{\sep}
        \hypo{\dot1}
        \infer3[$0$-E]{\vdash C}
      \end{prooftree}
      \\\\
      \begin{prooftree}
        \hypo{\vdash A_i}
        \infer1[$\oplus$-I$_i$, for $i \in \{0,1\}$]{\vdash A_0 \oplus A_1}
      \end{prooftree}
      &&
      \begin{prooftree}
        \hypo{\vdash A \oplus B}
        \hypo{\sep}
        \hypo{(\gr1A \vdash C}
        \hypo{\dottimes}
        \hypo{\gr1B \vdash C)}
        \infer5[$\oplus$-E]{\vdash C}
      \end{prooftree}
      \\\\
      \begin{prooftree}
        \hypo{\gr r \cdot (\vdash A)}
        \infer1[$\oc$-I]{\vdash \oc_{\gr r} A}
      \end{prooftree}
      &&
      \begin{prooftree}
        \hypo{\vdash \oc_{\gr r} A}
        \hypo{\sep}
        \hypo{\gr rA \vdash C}
        \infer3[$\oc$-E]{\vdash C}
      \end{prooftree}
    \end{matrix}
  \end{displaymath}
  \ebproofset{separation=1.5em}
  \caption{\name{} stated using bunched connectives}
  \label{fig:lr-bunched}
\end{figure}

\section{What are linear renaming and substitution?}\label{sec:lrkits}
\paragraph{New explanation}
Recalling from \cref{sec:kits}, we have the following definition of
\emph{environments} for simple types.

\begin{definition}[Simple environment]
  A $\V$-\emph{environment} between simply typed contexts $\Gamma$ and $\Delta$
  is a function, polymorphic in type $A$, from variables of type $A$ in
  $\Delta$ to inhabitants of $\V\,\Gamma\,A$.
  We write the type of such environments as $\Gamma \env\V \Delta$.
\end{definition}

\begin{definition}[Simple recursive environment]
  A \emph{recursive $\V$-environment} between simply typed contexts $\Gamma$ and
  $\Delta$ is defined by cases on the shape of $\Delta$ (where
  $\Gamma \env\V_R \Delta$ is the notation for the type of recursive
  environments for given $\V$, $\Gamma$, and $\Delta$):
  \begin{itemize}
    \item If $\Delta$ is empty, then there is one environment.
    \item If $\Delta$ is a concatenation $\Delta_l, \Delta_r$, then an
      environment is a pair of environments of types
      $\Gamma \env\V_R \Delta_l$ and $\Gamma \env\V_R \Delta_r$.
    \item If $\Delta$ is a singleton $A$, then an environment is a value of
      type $\V\,\Gamma\,A$.
  \end{itemize}
\end{definition}

\begin{definition}[Usage-annotated recursive environment]
  A \emph{recursive $\V$-environment} between annotated contexts $\Gamma$ and
  $\Delta$ is defined by cases on the shape of $\Delta$ (where
  $\Gamma = \grP\gamma$ and $\Gamma \env\V_R \Delta$ is the notation for the
  type of recursive environments for given $\V$, $\Gamma$, and $\Delta$):
  \begin{itemize}
    \item If $\Delta$ is empty, then an environment exists if $\grP = \gr0$.
    \item If $\Delta$ is a concatenation $\Delta_l, \Delta_r$, then an
      environment is a choice of usage vectors $\grPl$ and $\grPr$ such that
      $\grP = \grPl + \grPr$ and we have a pair of environments of types
      $\grPl\gamma \env\V_R \Delta_l$ and $\grPr\gamma \env\V_R \Delta_r$.
    \item If $\Delta$ is a singleton $\gr rA$, then an environment is a choice
      of a usage vector $\grPprime$ such that $\grP = \gr r\grPprime$ and we
      have a value of type $\V\,\grPprime\gamma\,A$.
  \end{itemize}
\end{definition}

From this definition, we can recover a functional-style definition by
separating choices of usage vectors from the provision of $\V$-values.
In particular, the only choices of usage vectors that are essential are the
$\grPprime$s in the singleton case, with the choices in the concatenation case
being determined as scalings and sums of these $\grPprime$s.
I let $\gr\Psi$ collect up these $\size\Delta$-many choices of
$\size\Gamma$-length usage vectors and note that the constraint on $\gr\Psi$
generated by all the scaling and summing is
$\grP = \sum_{\plr{x : \gr rA} \in \Delta} \gr r\gr\Psi_x$.

\begin{definition}[Usage-annotated environment (tentative)]
  A \emph{$\V$-environment} between annotated contexts $\Gamma$ and $\Delta$
  (written $\grP\gamma$ and $\grQ\delta$, respectively, when convenient)
  is a matrix $\gr\Psi : \Ann^{\size\Delta \times \size\Gamma}$ such that
  $\grP = \sum_{\plr{x : \gr rA} \in \Delta} \gr r\gr\Psi_x$ and for each
  $\plr{x : A} \in \delta$ we have a value of type $\V\,\gr\Psi_x\gamma\,A$.
\end{definition}

We may note, further, that the constraint
$\grP = \sum_{\plr{x : \gr rA} \in \Delta} \gr r\gr\Psi_x$ can be stated as the
vector-matrix multiplication $\grP = \grQ\gr\Psi$.
Using the same operation, we have that $\gr\Psi_x = \langle x \rvert\gr\Psi$.
Because $\langle x \rvert$ is exactly the $\grQprime$ such that
$\plr{x : A} \sqin \grQprime\delta$, we can rephrase the function producing
$\V$-values as: for each $A$, $\grPprime$, and $\grQprime$ such that
$\grPprime = \grQprime\gr\Psi$, a function from $\grQprime\delta \sqni A$ to
$\V\,\grPprime\gamma\,A$.
Finally, I choose to switch from matrices and matrix multiplication to
linear maps and their actions, which are easier to work with.
All of these changes yield my primary definition of an environment for
usage-annotated calculi.

\begin{definition}[Usage-annotated environment]
  A \emph{$\V$-environment} between annotated contexts $\Gamma$ and $\Delta$
  (written $\grP\gamma$ and $\grQ\delta$, respectively, when convenient)
  is a linear map $\gr\Psi : \Ann^{\size\Delta} \to \Ann^{\size\Gamma}$ (written
  postfix) such that $\grP = \grQ\gr\Psi$ and for each $A$, $\grPprime$, and
  $\grQprime$ such that $\grPprime = \grQprime\gr\Psi$, a function from
  $\grQprime\delta \sqni A$ to $\V\,\grPprime\gamma\,A$.
\end{definition}

\paragraph{Old explanation}
As we discussed in \cref{sec:kits}, simultaneous substitution gives a
notion of derivability of one context from another, while simultaneous renaming
gives a similar notion of derivability restricted to structural rules.
To adapt these notions from an intuitionistic setting to our substructural
setting, we must examine what it means to derive one context from another
substructually.

In the intuitionistic case, we say that to derive a context $\Delta$ from a
context $\Gamma$ is to derive each element $\Delta_i$ from $\Gamma$.
We may justify this by an intermediate step --- noting that contexts are
understood to be Cartesian products of their elements, and giving a map into
a Cartesian product is the same as giving a map into each factor.
I picture this definition as the diagram below.

\begin{displaymath}
  \begin{tikzpicture}[baseline]
    \path
    (-1,1) node (Gtop) {}
    (-1,0) node (G) {$\Gamma$}
    (-1,-1) node (Gbot) {}
    ;
    \node[draw,dotted,fit=(Gtop) (G) (Gbot)] (GG) {};

    \path
    (1,1) node (Dtop) {}
    (1,0) node (D) {$\Delta$}
    (1,-1) node (Dbot) {}
    ;
    \node[draw,dotted,fit=(Dtop) (D) (Dbot)] (DD) {};

    \draw[->,double] (GG) -- (DD);
  \end{tikzpicture}
  \coloneqq
  \begin{tikzpicture}[baseline]
    \path
    (-1,1) node (Gtop) {}
    (-1,0) node (G) {$\Gamma$}
    (-1,-1) node (Gbot) {}
    ;
    \node[draw,dotted,fit=(Gtop) (G) (Gbot)] (GG) {};

    \path
    (1,1) node[draw] (Dtop) {$\Delta_1$}
    (1,0) node (D) {$\vdots$}
    (1,-1) node[draw] (Dbot) {$\Delta_n$}
    ;

    \fill[green!20!white,opacity=1] (GG.north east)
    parabola[bend at end] (Dtop.west)
    parabola[bend at start] (GG.south east)
    -- cycle;
    \fill[blue!40!white,opacity=.5] (GG.north east)
    parabola[bend at end] (Dbot.west)
    parabola[bend at start] (GG.south east)
    -- cycle;

    \draw[->] (GG.north east) parabola[bend at end] (Dtop.west);
    \draw (GG.south east) parabola[bend at end] (Dtop.west);
    \draw[->] (GG.north east) parabola[bend at end] (D.west);
    \draw (GG.south east) parabola[bend at end] (D.west);
    \draw[->] (GG.north east) parabola[bend at end] (Dbot.west);
    \draw (GG.south east) parabola[bend at end] (Dbot.west);
  \end{tikzpicture}
\end{displaymath}

To see how this definition works, let us construct an example substitution:
$A, B \to C, B \Longrightarrow B, C$.
Because the codomain is a Cartesian product, it suffices to give two separate
substitutions, $A, B \to C, B \Longrightarrow B$ and
$A, B \to C, B \Longrightarrow C$, with a substitution into a singleton
context being just a term.
We indeed have terms $x : A, y : B \to C, z : B \vdash y : B$ and
$x : A, y : B \to C, z : B \vdash y\,z : C$.
It is also instructive to look at an identity substitution (which is also a
renaming), $A, B \Longrightarrow A, B$, witnessed by terms
$x : A, y : B \vdash x : A$ and $x : A, y : B \vdash y : B$.

When working with our semiring-annotated calculus \name{}, contexts are no
longer understood as Cartesian products.
This means that substitutions of type $\Gamma \Longrightarrow \Delta$ are no
longer equivalent to collections of substitutions
$\Gamma \Longrightarrow \Delta_i$.
Indeed, notice that we should still have an identity substitution of type
$\gr1A, \gr1B \Longrightarrow \gr1A, \gr1B$, but we do not have terms proving
either $\gr1A, \gr1B \vdash A$ or $\gr1A, \gr1B \vdash B$.
What we do have are terms $x : \gr1A, y : \gr0B \vdash x : A$ and
$x : \gr0A, y : \gr1B \vdash y : B$, and if we pointwise add together the
annotations of the two terms, we get back the original context
$x : \gr1A, y : \gr1B$.
Furthermore, adding up the annotations is not just a random operation;
linear contexts are understood to be tensor products of their elements, and
introduction rule for the tensor product involves summing the annotations of
the two sides.

For any annotated context $\Delta$, we have
$\Delta \vdash \bigotimes_{(\gr rx : A) \in \Delta}\oc\gr rA$ by iterated
application of $\otimes$-I with $\oc$-I and Var at the leaves.
Let $\Gamma = \grP\gamma$ and $\Delta = \grQ\delta$.
If we are to produce substitutions from $\Gamma$ to $\Delta$ in this
pattern, we simulate the applications of $\otimes$-I by producing, for each
element in $\Delta$, a usage context for $\gamma$ such that the whole collection
sums to $\grP$, then simulate the applications of $\oc$-I by dividing each of
the new usage contexts by the corresponding annotation in $\Delta$, calling
the divided usage contexts $\gr\Psi_x$, and finally, instead of a variable
from $\delta$, we give a term of type $\gr\Psi_x\gamma \vdash \delta_x$.
In summary, the constraint on the collection of usage contexts $\gr\Psi$ is
that $\grP = \sum_{(\gr rx : A) \in \Delta}\gr r\gr\Psi_x$.
Moreover, if we take $\grP$ and $\grQ$ to be row vectors and $\gr\Psi$ to be a
matrix, the latter expression is equal to the vector-matrix multiplication
$\grQ\gr\Psi$.
The resulting definition of simultaneous substitution is depicted below.

\begin{displaymath}
  \begin{tikzpicture}[baseline]
    \path
    (-1,1) node (Gtop) {}
    (-1,0) node (G) {$\grP\gamma$}
    (-1,-1) node (Gbot) {}
    ;
    \node[draw,dotted,fit=(Gtop) (G) (Gbot)] (GG) {};

    \path
    (1,1) node (Dtop) {}
    (1,0) node (D) {$\grQ\delta$}
    (1,-1) node (Dbot) {}
    ;
    \node[draw,dotted,fit=(Dtop) (D) (Dbot)] (DD) {};

    \draw[->,double] (GG) -- (DD);
  \end{tikzpicture}
  \coloneqq
  \begin{tikzpicture}[baseline]
    \path
    (-1,1) node (Gtop) {}
    (-1,0) node (G) {$\grP\gamma$}
    (-1,-1) node (Gbot) {}
    ;
    \node[draw,dotted,fit=(Gtop) (G) (Gbot)] (GG) {};

    \path
    (1,3) node (G1top) {}
    (1,2) node (G1) {$\gr\Psi_1\gamma$}
    (1,1) node (G1bot) {}
    ;
    \node[draw,dotted,fit=(G1top) (G1) (G1bot)] (GG1) {};
    \draw[->] (GG) -- (GG1);

    \path (1,0) node {$\vdots$};

    \path
    (1,-1) node (Gntop) {}
    (1,-2) node (Gn) {$\gr\Psi_n\gamma$}
    (1,-3) node (Gnbot) {}
    ;
    \node[draw,dotted,fit=(Gntop) (Gn) (Gnbot)] (GGn) {};
    \draw[->] (GG) -- (GGn);

    \path
    (3,1) node[draw] (Dtop) {$\delta_1$}
    (3,0) node (D) {$\vdots$}
    (3,-1) node[draw] (Dbot) {$\delta_n$}
    ;

    \fill[green!20!white] (GG1.north east)
    parabola[bend at end] (Dtop.west)
    parabola[bend at start] (GG1.south east)
    -- cycle;
    \draw[->] (GG1.north east) parabola[bend at end] (Dtop.west);
    \draw (GG1.south east) parabola[bend at end] (Dtop.west);

    \fill[blue!20!white] (GGn.north east)
    parabola[bend at end] (Dbot.west)
    parabola[bend at start] (GGn.south east)
    -- cycle;
    \draw[->] (GGn.north east) parabola[bend at end] (Dbot.west);
    \draw (GGn.south east) parabola[bend at end] (Dbot.west);
  \end{tikzpicture}
  \quad\textrm{where }\grP = \grQ\gr\Psi
\end{displaymath}

In type theory, we write out the definition as follows.

\begin{displaymath}
  \sum_{\gr\Psi : \size\Delta \to \size\Gamma \to \Ann}
    \left(\grP = \grQ\gr\Psi\right) \times
    \prod_{(x : A) \in \delta}\gr\Psi_x\gamma \vdash A
\end{displaymath}

We can see the step-by-step construction of a substitution play out by adapting
the previous example to have type
$\gr0A, \gr2(B \multimap C), \gr3B \Longrightarrow \gr1B, \gr2C$.
To split the goal up, we note that $
\begin{pmatrix} \gr0 & \gr2 & \gr3 \end{pmatrix} =
\begin{pmatrix} \gr0 & \gr0 & \gr1 \end{pmatrix} +
\begin{pmatrix} \gr0 & \gr2 & \gr2 \end{pmatrix}
$, so it suffices to give substitutions of types
$\gr0A, \gr0(B \multimap C), \gr1B \Longrightarrow \gr1B$ and
$\gr0A, \gr2(B \multimap C), \gr2B \Longrightarrow \gr2C$.
Furthermore, our term calculus only supports $\gr1$-annotated conclusions,
so we divide the second substitution type through by $\gr2$.
Finally, we give the terms largely as before:
$\gr0x : A, \gr0y : B \to C, \gr1z : B \vdash y : B$ and
$\gr0x : A, \gr1y : B \to C, \gr1z : B \vdash z\,y : C$.

While we naturally derive a matrix as a fragmentation of a usage vector, we can
get a slightly cleaner presentation by instead using an abstract linear map.
Let $\gr\Psi$ now be a linear map of type
$\Ann^{\size\Delta} \to \Ann^{\size\Gamma}$, with application written postfix.
The equation $\grP = \grQ\gr\Psi$ remains unchanged.
Where we previously wrote $\gr\Psi_x$, the most direct replacement would be
$\langle x \rvert\gr\Psi$, with $\langle x \rvert$ being the $x$th basis row
vector.
But then we notice that $\langle x \rvert$ is exactly the $\grQprime$ satisfying
$\grQprime\delta \sqni x : B$.
This gives us the following definition, which can be verified by equationally
substituting $\grPprime$ and expanding the definition of $\sqni$.

\begin{displaymath}
  \sum_{\gr\Psi : \Ann^{\size\Delta} \to \Ann^{\size\Gamma}}
    \left(\grP = \grQ\gr\Psi\right) \times
    \prod_{A,\grQprime,\grPprime} \left(
    \grPprime = \grQprime\gr\Psi \to \grQprime\delta \sqni A \to
    \grPprime\gamma \vdash A\right)
\end{displaymath}

We now have a new reading for the interpretation of a linear substitution:
a linear map $\gr\Psi$ relating the two usage vectors $\grP$ and $\grQ$, and
for any two similarly related usage vectors $\grPprime$ and $\grQprime$, we
have a type-preserving function from variables in $\grQprime\delta$ to terms in
$\grPprime\gamma$.
Even though we don't use $\gr\Psi$ as a matrix containing fragmented usage
vectors, we can still justify why it should be a \emph{linear} map.
We need $\gr\Phi$ to respect all fragmentation of the usage context in a typing
rule, and we know that all such fragmentation is done by linear operations
zero, addition, and scaling by a constant.
\todo{Expand. Substitutions need to preserve everything done to the context,
and linear things are all we do to the context.}

Taking a lead from \cref{sec:kits}, we deduce a definition of
\emph{environment} by replacing the $\vdash$ in the definition of simultaneous
substitution by an arbitrary type family $\mathcal V$.
Letting $\mathcal V$ be $\sqni$ gives us a notion of simultaneous renaming,
allowing for renamings with types such as
$\gr6A, \gr0B, \gr1C \stackrel\sqni\Longrightarrow \gr1C, \gr2A, \gr4A$.

It is worth noting that, when contexts are Cartesian products, passing from
``a map into $\Delta$'' to ``for each $A \in \Delta$, a map into $A$'' does not
lose any generality because the universal property of Cartesian products
states that every map into a Cartesian product can be given factor-wise.
Hence, $\Delta$ and the one-element context $\prod_{A \in \Delta}A$ are
isomorphic in the category of contexts and simultaneous substitution.
However, tensor products are not limits, so don't have the same universal
property.
Indeed, many annotated contexts are not isomorphic to the weighted product of
their elements.
For example, we do not have a substitution of type
$\gr1(A \otimes B) \Longrightarrow \gr1A, \gr1B$ because we would first need
to pattern-match on the tensor product \emph{before} trying to derive the
target context.
This loss of generality is however justified when we consider the action of a
substitution.
Substitutions should only be replacing variables by terms, whereas if
substitutions were allowed to pattern-match before introducing the target, then
the substitution would have to replace the original term by a term that first
pattern-matches and then continues like the original term.

\section{Properties of linear environments}\label{sec:lenv}
\begin{remark}
  Given an environment $\rho : \grP\gamma \env\V \grQ\delta$ and a $\grPprime$
  and a $\grQprime$ such that $\grPprime = \grQprime\plr{\rho.\gr\Psi}$,
  there is also an environment of type $\grPprime\gamma \env\V \grQprime\delta$
  with the same linear map and action on variables.
\end{remark}
\begin{proof}
  The only part of the definition of an environment dependent on $\grP$ or
  $\grQ$ is the constraint $\grP = \grQ\gr\Psi$, which we are able to replace
  for $\grPprime$ and $\grQprime$.
\end{proof}

When constructing an environment, we can do so by cases on the shape of the
target context.
We can create an environment into the empty context when all usage annotations
on the source context are $\gr0$.
We can create an environment into a concatenated context when we can additively
split up the annotations of the source context and produce environments into
both halves from the split sources.
We can create an environment into a singleton context when there is a context
$\gr r$ times smaller than the source context in which we can produce a value
of the appropriate type.

\begin{lemma}\label{thm:construct-env}
  We can define all of the following equivalences for any values of the free
  variables.
  \begin{itemize}
    \item $\forallb{I \dotlr \plr{{-} \env\V {\cdot}}}$
    \item $\forallb{\plr{{-} \env\V \Gamma} \sep \plr{{-} \env\V \Delta}
      \dotlr \plr{{-} \env\V \Gamma, \Delta}}$
    \item
      $\forallb{\gr r \cdot \plr{\V\,(-)\,A} \dotlr \plr{{-} \env\V \gr rA}}$
  \end{itemize}
\end{lemma}
\begin{proof}
  There are 6 cases to check.
  Throughout, we write $\Gamma$ as $\grP\gamma$ and $\Delta$ as $\grQ\delta$
  when convenient.
  \begin{description}
    \item[$I(\to)$]
      Let $\gr\Psi$ be the unique linear map out of the zero space.
      By definition, $\gr0 = \grQ\gr\Psi$.
      There are no variables to act upon.
    \item[$I(\gets)$]
      $\grQ\gr\Psi$ is an empty sum, so if $\grP = \grQ\gr\Psi$ then
      $\grP = \gr0$.
    \item[$\sep(\to)$]
      Let the given environments be $\rho : \grRl\theta \env\V \Gamma$ and
      $\sigma : \grRr\theta \env\V \Delta$, with $\grR = \grRl + \grRr$.
      Define $\gr\Psi \coloneqq [\rho.\gr\Psi, \sigma.\gr\Psi]$, using the
      coproduct structure of the concatenated vector space.
      We have $\grR = \grRl + \grRr =
      \grP\plr{\rho.\gr\Psi} + \grQ\plr{\sigma.\gr\Psi} =
      \plr{\grP, \grQ}\gr\Psi$.
      To act on variables, we are given
      $\grRprime = \plr{\grPprime, \grQprime}\gr\Psi$ and
      $\grPprime\gamma, \grQprime\delta \sqni A$.
      Without loss of generality, let us have $\grPprime\gamma \sqni A$ and
      $\grQprime = \gr0$.
      Thus, $\grRprime = \grPprime\plr{\rho.\gr\Psi}$, and we can act on the
      variable using $\rho$.
    \item[$\sep(\gets)$]
      Let the unnamed context be $\Theta$, also written $\grR\theta$.
      The linear map
      $\gr\Psi : \Ann^{\size\Gamma + \size\Delta} \to \Ann^{\size\Theta}$ splits
      into
      $\gr\Psi_{\gr l} : \Ann^{\size\Gamma} \to \Ann^{\size\Theta}
      \coloneqq \langle \id, 0 \rangle; \gr\Psi$ and
      $\gr\Psi_{\gr r} : \Ann^{\size\Delta} \to \Ann^{\size\Theta}
      \coloneqq \langle 0, \id \rangle; \gr\Psi$, using the product structure of
      the concatenated vector space.
      Let $\grRl \coloneqq \grP\gr\Psi_{\gr l}$ and
      $\grRr \coloneqq \grQ\gr\Psi_{\gr r}$, by definition satisfying the
      required equations.
      For the action on variables, let us consider the left environment.
      We are given $\grRprime = \grPprime\gr\Psi_{\gr l}$ and
      $\grPprime\gamma \sqni A$.
      From these, we get
      $\grRprime = \grPprime\gr\Psi_{\gr l} = \plr{\grPprime, \gr0}\gr\Psi$ and
      $\grPprime\gamma, \gr0\delta \sqni A$.
      We can therefore act using the original environment.
    \item[$\cdot(\to)$]
      Let $\grP$ and $\grPprime$ be such that $\grP = \gr r\grPprime$ and let
      $v : \V\,\grPprime\gamma\,A$.
      Let $\gr\Psi : \Ann \to \Ann^{\size\gamma}
      \coloneqq \gr r\gr' \mapsto \gr r\gr'\grPprime$.
      By definition and the previous assumption, we have $\grP = \gr r\gr\Psi$.
      When acting on a variable, we have $\grP\gr{''} = \gr r\gr'\gr\Psi$
      and $\gr r\gr'A \sqni A'$.
      The latter tells us that $A = A'$ and $\gr r\gr' = \gr1$.
      Thus, $\grP\gr{''} = \grPprime$.
      We therefore need a value of type $\V\,\grPprime\gamma\,A$, which we can
      take to be $v$.
    \item[$\cdot(\gets)$]
      Let us have an environment of type $\grP\gamma \env\V \gr rA$.
      We want to use its action on variables to yield a value.
      To do this, we let $\grPprime \coloneqq \gr1\gr\Psi$, and use this
      equation, together with the fact that we have a variable of type
      $\gr1A \sqni A$, to get a value of type $\V\,\grPprime\gamma\,A$.
      Furthermore, we derive $\grP = \gr r\gr\Psi = \gr r\grPprime$, as
      required.
  \end{description}
\end{proof}

We could, indeed, use these three clauses to define what an environment is.
However, I find them difficult to work with, as it is often easier to do
linear algebraic proofs separately from the rest of an environment.
For identity and composition, as we are about to see, the original definition
is easier to use because we can rely on the identity and composition of linear
maps.
Concretely, an inductive proof of identity would, for example, involve
constructing an environment of type
$\grP\gamma, \grQ\delta \env\V \grP\gamma, \grQ\delta$ by constructing
environments of types $\grP\gamma, \gr0\delta \env\V \grP\gamma$ and
$\gr0\gamma, \grQ\delta \env\V \grQ\delta$.
These are not identity environments, so we would have to strengthen the
induction hypothesis.

One of the primary test cases for environments is simultaneous substitution,
which will look like the following rule.
The admissibility of substitution will be by induction on the derivation of
$\Delta \vdash A$, so we will need to be able to adapt any environment we are
given to work with any possible context of new premises.
In the simply typed case, the only change to the context we encountered was the
binding of new variables.
Now, with usage annotations, we furthermore have linear decompositions of the
context, necessitating changes to the environment whenever usage annotations
change.
I will deal first with linear decompositions.

\begin{displaymath}
  \begin{prooftree}
    \hypo{\Gamma \env{\vdash} \Delta}
    \hypo{\Delta \vdash A}
    \infer2[sub]{\Gamma \vdash A}
  \end{prooftree}
\end{displaymath}

There are three kinds of linear decompositions we have to deal with: zero,
addition, and scaling; corresponding to bunched connectives $I^*$, $\sep$, and
$\gr r \cdot {}$, respectively.
In each case, we have a simple preservation lemma, transforming an environment
of type $\Gamma \env\V \Delta$ and a decomposition of $\Delta$ into a
decomposition of $\Gamma$ and environments for all of the decomposed fragments
of $\Gamma$ and $\Delta$.

\begin{lemma}[environments preserve zero]\label{thm:lr-env-zero}
  Given an environment of type $\grP\gamma \env\V \grQ\delta$ such that
  $\grQ \leq \gr 0$, we also have that $\grP \leq \gr 0$.
\end{lemma}
\begin{proof}
  $\grP \leq \grQ\gr\Psi \leq \gr0\gr\Psi = \gr0$, by environment
  compatibility and monotonicity and linearity of $\gr\Psi$.
\end{proof}

\begin{lemma}[environments preserve addition]\label{thm:lr-env-add}
  Given an environment of type $\grP\gamma \env\V \grQ\delta$ such that
  $\grQ \leq \grQl + \grQr$ for some $\grQl$ and $\grQr$, we also have $\grPl$
  and $\grPr$ such that $\grP \leq \grPl + \grPr$ and there are environments
  of types $\grPl\gamma \env\V \grQl\delta$ and
  $\grPr\gamma \env\V \grQr\delta$.
\end{lemma}
\begin{proof}
  Let $\grPl \coloneqq \grQl\gr\Psi$ and $\grPr \coloneqq \grQr\gr\Psi$.
  Then, $\grP \leq \grQ\gr\Psi \leq \plr{\grQl + \grQr}\gr\Psi =
  \grQl\gr\Psi + \grQr\gr\Psi = \grPl + \grPr$, satisfying the first condition.
  Because clearly $\grPl \leq \grQl\gr\Psi$ and $\grPr \leq \grQr\gr\Psi$,
  \cref{thm:env-resize} on the original environment gives us the required
  pair of new environments.
\end{proof}

\begin{lemma}[environments preserve scaling]\label{thm:lr-env-scale}
  Given an environment of type $\grP\gamma \env\V \grQ\delta$ such that
  $\grQ \leq \gr r\grQprime$ for some $\grQprime$, we also have a $\grPprime$
  such that $\grP \leq \gr r\grPprime$ and there is an environment of type
  $\grPprime\gamma \env\V \grQprime\delta$.
\end{lemma}
\begin{proof}
  Let $\grPprime \coloneqq \grQprime\gr\Psi$.
  Then, $\grP \leq \grQ\gr\Psi \leq \plr{\gr r\grQprime}\gr\Psi =
  \gr r\plr{\grQprime\gr\Psi} = \gr r\grPprime$, satisfying the first condition.
  Because clearly $\grPprime \leq \grQprime\gr\Psi$,
  \cref{thm:env-resize} on the original environment gives us the required
  new environment.
\end{proof}

Finally, I will also take the opportunity to give the bind lemma, allowing
environments to incorporate newly bound variables.
In the intuitionistic case, the bind lemma had two requirements on $\V$: $\V$
admits weakening and we can map variables into $\V$-values.
With usage annotations, the former is unreasonable, but it turns out that we
only need weakening by variables whose usage annotation is less than or equal
to $\gr0$.
The latter stays as-is, with the note that ``variable'' now means a
usage-checked variable.

\begin{lemma}[bind]\label{thm:lr-bind}
  Given functions
  ${\swarrow^k} : \forall \Gamma, \grR, \theta.~\grR \leq \gr0 \to
  \forallb{\V\,\Gamma \dotto \V\,\plr{\Gamma, \grR\theta}}$ and
  $\mathrm{vr} : \forallb{{\sqni} \dotto \V}$, we can turn an environment of
  type $\Gamma \env\V \Delta$ into an environment of type
  $\Gamma, \Theta \env\V \Delta, \Theta$ for any context $\Theta$.
\end{lemma}
\begin{proof}
  Let $\grP\gamma \coloneqq \Gamma$, $\grQ\delta \coloneqq \Delta$, and
  $\grR\theta \coloneqq \Theta$.
  Let the new linear map $\gr\Psi\gr' : \Ann^{\size\Delta + \size\Theta} \to
  \Ann^{\size\Gamma + \size\Theta}$ be $\gr\Psi \oplus \gr I$.
  That is, in block matrix notation,
  $\begin{pmatrix} \gr\Psi & \gr0 \\ \gr0 & \gr I \end{pmatrix}$.
  Checking that this linear map fits, we have
  $\begin{pmatrix}\grP & \grR\end{pmatrix}
  \leq \begin{pmatrix}\grQ\gr\Psi & \grR\gr I\end{pmatrix}
  = \begin{pmatrix}\grQ & \grR\end{pmatrix}\plr{\gr\Psi \oplus \gr I}$.
  For the action on variables, we are given vectors $\grPprime$,
  $\grR\gr'_\grP$, $\grQprime$, and $\grR\gr'_\grQ$ such that
  $\begin{pmatrix} \grPprime & \grR\gr'_\grP \end{pmatrix} \leq
  \begin{pmatrix} \grQprime & \grR\gr'_\grQ \end{pmatrix}
  \plr{\gr\Psi \oplus \gr I}$ and we have a variable of type
  $\grQprime\delta, \grR\gr'_\grQ\theta \sqni A$ for some type $A$.
  The constraint on the new vectors reduces to $\grPprime \leq \grQprime\gr\Psi$
  and $\grR\gr'_\grP \leq \grR\gr'_\grQ$.
  From the variable we either have a variable $x$ in $\delta$ with
  $\grQprime \leq \langle x \rvert$ and $\grR\gr'_\grQ \leq \gr0$, or a
  variable $y$ in $\theta$ with $\grQprime \leq \gr0$ and
  $\grR\gr'_\grQ \leq \langle y \rvert$.
  In the former case, the action of the original environment on $x$ gives us a
  $\V$-value in $\grPprime\gamma$, and the $\gr0$-weakening principle
  $\swarrow^k$, noting that $\grR\gr'_\grP \leq \grR\gr'_\grQ \leq \gr0$, gives
  us a $\V$-value in $\grPprime\gamma, \grR\gr'_\grP\theta$.
  In the latter case, we have that
  $\begin{pmatrix} \grPprime & \grR\gr'_\grP \end{pmatrix}
  \leq \begin{pmatrix} \grQprime\gr\Psi & \grR\gr'_\grQ \end{pmatrix}
  \leq \begin{pmatrix} \gr0\gr\Psi & \langle y \rvert \end{pmatrix}
  = \begin{pmatrix} \gr0 & \langle y \rvert \end{pmatrix}
  = \left\langle {\searrow}y \right\rvert$, so $y$ also serves as a
  usage-checked variable in $\grPprime\gamma, \grR\gr'_\grP\theta$.
  From this usage-checked variable, we get a $\V$-value in the same context
  using $\mathrm{vr}$.
\end{proof}

The requirements for identity and composition of environments look a bit like
the unit and lift of a Kleisli triple.

\begin{lemma}[Identity environment]
  Given a function $\mathrm{vr} : \forallb{{\sqni} \dotto \V}$, for any
  $\Gamma$ we have an environment of type $\Gamma \env\V \Gamma$.
\end{lemma}
\begin{proof}
  Let $\gr\Psi$ be the identity map, which clearly satisfies
  $\grP = \grP\gr\Psi$.
  When acting on a variable, the equation $\grPprime = \grQprime\gr\Psi$ means
  that $\grPprime = \grQprime$, so we want, from a variable of type
  $\grPprime\gamma \sqni A$, a value of type $\V\,\grPprime\gamma\,A$, which
  we can get from $\mathrm{vr}$.
\end{proof}

\begin{lemma}\label{thm:env-comp-lemma}
  Given an environment $\rho : \Gamma \env\U \Delta$ for which we have, for any
  $\grPprime$ and $\grQprime$ such that
  $\grPprime = \grQprime\plr{\rho.\gr\Psi}$, we have a function
  $\mathrm{lift}_\rho :
  \forallb{\V\,\grQprime\delta \dotto \W\,\grPprime\gamma}$,
  we can map environments of type $\Delta \env\V \Theta$ into environments of
  type $\Gamma \env\W \Theta$.
\end{lemma}
\begin{proof}
  Let $\rho$ be as in the statement, and let $\sigma : \Delta \env\V \Theta$.
  For the environment we are constructing, let
  $\gr\Psi \coloneqq \sigma.\gr\Psi; \rho.\gr\Psi$, noting that
  $\grP = \grQ\plr{\rho.\gr\Psi} =
  \plr{\grR\plr{\sigma.\gr\Psi}}\plr{\rho.\gr\Psi}$.
  For the action on variables, we are given $\grPprime = \grRprime\gr\Psi$ with
  $\grRprime\theta \sqni A$.
  We can immediately apply the action of $\sigma$, giving us a value of type
  $\V\,\plr{\grRprime\plr{\sigma.\gr\Psi}}\,A$.
  We note that
  $\grPprime = \plr{\grRprime\plr{\sigma.\gr\Psi}}\plr{\rho.\gr\Psi}$, and
  apply $\mathrm{lift}_\rho$ to get the desired value.
\end{proof}

\begin{corollary}[Composition of environments]
  Given a function
  $\mathrm{lift} : \plr{\rho : \grP\gamma \env\U \grQ\delta} \to
  \forall \grPprime, \grQprime.~\grPprime = \grQprime\plr{\rho.\gr\Psi} \to
  \forallb{\V\,\grQprime\delta \dotto \W\,\grPprime\gamma}$, then we can
  compose environments of types $\Gamma \env\U \Delta$ and
  $\Delta \env\V \Theta$ into an environment of type $\Gamma \env\W \Theta$.
\end{corollary}

\begin{example}
  We can derive the following instances of environment composition.
  \begin{itemize}
    \item If $\U = \V = \W = {\sqni}$, then $\mathrm{lift}$ is given by the
      action of the renaming $\rho$ on variables.
      This allows us to derive composition of renamings.
    \item More generally, if $\V = {\sqni}$ and $\U = \W$, we can still use
      the action of the environment $\rho$.
      This means that renamings post-compose with any other sort of environment.
    \item If $\V = \W = {\vdash}$, then $\mathrm{lift}$ is given by a
      syntactic traversal.
      For example, if $\U = {\sqni}$, we need the action of renaming on terms
      to show that a renaming followed by a substitution composes to a
      substitution.
      If $\U = {\vdash}$, then the action of substitution on terms gives us that
      substitutions compose.
    \item More generally, if $\V = {\vdash}$ and we have a semantics from
      $\U$ to $\W$, then $\mathrm{lift}$ can be given by the semantic traversal
      of terms.
  \end{itemize}
\end{example}

% Concatenation is difficult; save to after I've talked about renamings.

% Finally for this section, we give the conditions under which the
% context-forming operations (empty, concatenation, and singleton) have a
% functorial action with respect to $\V$-environments.
%
% \begin{lemma}
%   For any $\V$, there is an environment ${\cdot} \env\V {\cdot}$.
% \end{lemma}
% \begin{proof}
%   By \autoref{thm:construct-env}, it suffices to show $I\,{\cdot}$, which is
%   trivially true.
% \end{proof}

\section{Substitution is admissible in \name{}}\label{sec:lrsub}
\def\LRKits{../agda/processed-latex/LRKits.tex}

I can now show that, using the notion of \emph{environment} derived in
\cref{sec:lrkits}, we can replicate the Agda proofs from
\cref{sec:syntactic-kits} in the usage-aware setting of $\name$.
From \cref{sec:lenv}, we know that environments are preserved under all
syntax-forming operations: zero, addition, scaling, and binding.
What is left is to show how these properties are deployed, and also how to
go on and prove the admissibility of simultaneous renaming, simultaneous
substitution, and then single substitution.

There are a few notational changes necessary in the Agda code, compared to the
typeset mathematics above.
Usage vectors, elsewhere called $\grP$, $\grQ$, and $\grR$ are rendered as
\AgdaBound{P}, \AgdaBound{Q}, and \AgdaBound{R}, respectively.
Usage contexts and typing contexts are tied together with the
\AgdaInductiveConstructor{ctx} constructor, rather than simple juxtaposition.
Environments, elsewhere notated $\Gamma \env\V \Delta$, are rendered as
\AgdaRecord{[}\AgdaSpace{}\AgdaBound{$\V$}\AgdaSpace{}\AgdaRecord{]}%
\AgdaSpace{}\AgdaBound{$\Gamma$}\AgdaSpace{}\AgdaRecord{$\Rightarrow^e$}%
\AgdaSpace{}\AgdaBound{$\Delta$}.

We start with a slightly modified definition of \AgdaRecord{Kit}.
We saw in \cref{thm:lr-bind} that in the usage-annotated context, we restrict
weakening of $\V$-values to just $\gr0$-use variables.
Meanwhile, the function $\mathrm{vr}$, also seen in \cref{thm:lr-bind}, maps
usage-checked variables to $\V$-values, and the function $\mathrm{tm}$, used
to coerce $V$-values yielded by the environment into terms, stays the same.
I state weakening in a slightly different way than previously, so as to help
unification against a known result type (avoiding the problem described by
\citet{McBride12} as \emph{green slime}).
The type \AgdaFunction{Weakening}\AgdaSpace{}\AgdaBound{$\V$} can be read as
saying that, for any context $\grP\gamma$ of shape $s + t$, if the right of
$\grP$ is below $\gr0$, then a value in the left part of $\grP\gamma$ weakens
to a value in the whole of $\grP\gamma$.

\ExecuteMetaData[\LRKits]{Kit}

To demonstrate the important points succinctly, I cut \name{} down to just the
$\oc\gr r$-fragment.
The introduction rule and pattern-matching eliminator feature scaling, addition,
and variable binding, missing out only on sharing (which is trivial) and zero
(which is simpler than, and analogous to, addition).
The resulting type of well typed terms is below.

\ExecuteMetaData[\LRKits]{Tm}

Given a \AgdaRecord{Kit}, \cref{thm:lr-bind} looks like the following.
The \AgdaField{lookup} clauses still contain essentially the same structure as
in the intuitionistic case: discriminating on whether the variable is old or
new, using the given environment \AgdaBound{$\rho$} and weakening on the old
variables, and using \AgdaField{vr} to repackage new variables.
I will not explain any of the algebraic manipulations here; see
\cref{thm:lr-bind}.

\ExecuteMetaData[\LRKits]{bindEnv}

Given \AgdaFunction{bindEnv} (\cref{thm:lr-bind}), \AgdaFunction{env-+}
(\cref{thm:lr-env-add}), and \AgdaFunction{env-*} (\cref{thm:lr-env-scale}),
we can reproduce the syntactic traversal \AgdaFunction{trav}.
With all these lemmas in place, writing \AgdaFunction{trav}
becomes routine.
When processing a rule, we work our way up through the
premise connectives, applying \AgdaFunction{env-*} wherever we see a
\AgdaFunction{$\cdot^c$}, \AgdaFunction{env-+} wherever we see a
\AgdaFunction{$*^c$}, and \AgdaFunction{bindEnv} wherever we see a
\AgdaFunction{Bind}.
We then use whatever environments (with names beginning with
\AgdaBound{$\rho$}) and whatever usage vector splitting facts (with names
beginning with \AgdaBound{sp}) come out of this process to recursively
traverse the subterms and recombine the results.

\ExecuteMetaData[\LRKits]{trav}

Instantiating the generic syntactic traversal \AgdaFunction{trav} to renaming
looks just like it did in the intuitionistic case.
I have consistently replaced intuitionistic variables by linear variables, so
\AgdaFunction{id} and \AgdaInductiveConstructor{var} still work to embed
variables into variables and terms, respectively.
Weakening for variables \AgdaFunction{$\swarrow^v$} (not pictured) has been
updated to note that, for $\grP \leq \bra x$ and $\grR \leq \gr0$, we also have
$\begin{pmatrix} \grP & \grR \end{pmatrix} \leq \bra{{\swarrow}x}$.

\ExecuteMetaData[\LRKits]{var-kit}

In the intuitionistic case, environments were just functions, so we passed the
variable weakening function \AgdaFunction{$\swarrow^v$} to the function
\AgdaFunction{ren} to yield a term weakening function.
However, a usage-aware environment is a function packed together with usage
distribution data.
As such, we must make an environment version of \AgdaFunction{$\swarrow^v$}.
I start with a general lemma \AgdaFunction{$\swarrow$\^{}Env}, stating that if
$\V$ supports weakening, then so do $\V$-environments (in their domain
context).
This lemma then specialises to variables, with the identity renaming
\AgdaFunction{id\^{}Env} on the left part of the context and the proof
\AgdaBound{R0} that the right part of the context is below $\gr0$ combining
to give the desired weakening environment.

\ExecuteMetaData[\LRKits]{dlv-env}

This is what we need to instantiate \AgdaFunction{trav} for substitution.
As a reminder, I also give the type of \AgdaFunction{sub} in rule form.

\ExecuteMetaData[\LRKits]{sub}
\[
  \ebrule{%
    \hypo{\Gamma \env\vdash \Delta}
    \hypo{\Delta \vdash B}
    \infer2[sub]{\Gamma \vdash B}
  }
\]

Finally, the simultaneous substitution \AgdaFunction{sub} specialises to
single substitution \AgdaFunction{sub[-]}.
Single substitution is stated as an admissible rule below.
To substitute in for $\gr r$-many $A$ in the second term, we need to derive
one $A$ with usages $\grP$, and then assert that the result can handle the
usages of the original term $\grQ$, plus $\gr r$-many copies of $\grP$.

\[
  \ebrule{%
    \hypo{\grR \leq \grQ + \gr r\grP}
    \hypo{\grP\gamma \vdash A}
    \hypo{\grQ\gamma, \gr rA \vdash B}
    \infer3[singleSub]{\grR\gamma \vdash B}
  }
\]

The proof strategy for producing the substitution \AgdaFunction{$\sigma$} is
to proceed structurally on the codomain context $\grQ\gamma, \gr rA$ using
\cref{thm:construct-env}, applying the identity substitution
\AgdaFunction{id\^{}Env} on the $\gamma$ half, and dropping the term
\AgdaBound{t} in place of the variable we are substituting for.

\ExecuteMetaData[\LRKits]{subSingle}

\section{Adding recursion to \name{}}\label{sec:rec}
Based on an intuitive understanding of ``usage'', recursion introduces a new
phenomenon relative to the forms of programs we have seen so far:
terms can be used an unbounded number of times.
For example, notice the following reduction in Agda.

\missingfigure{\texttt{foldr \_+\_ 0 (1 :: 2 :: 3 :: []) = 1 + (2 + (3 + 0))}}

The function \AgdaFunction{\_+\_} has been copied into 3 different places in
the running of the program.
This copying is despite no type telling us that \AgdaFunction{\_+\_} would be
used 3 times (both \verb|[1,2,3]| and \verb|[2,3]| have type
\AgdaDatatype{List}\AgdaSpace{}\AgdaDatatype{$\mathbb N$}, despite the
corresponding folds using \AgdaFunction{\_+\_} a different number of times).
As such, when checking an application of \AgdaFunction{foldr}, we need check
that we can use its functional argument (\AgdaFunction{\_+\_} in this case) an
arbitrary number of times.
If we were to fix $\Ann$ as the $\{\gr0, \gr1, \gr\omega\}$ posemiring, then
wrapping the type of the functional argument in $\oc\gr\omega$ would suffice.
However, we want to remain generic in the choice of semiring.

I propose the following additions to \name{} to support a broad class of
inductive types.
I define strictly positive functors syntactically, with the only notable
restriction being not being allowed to use the type variable $X$ in the domain
of a function type.
I then add least fixed points of such strictly positive functors to the syntax
of types.

\begin{align*}
  U &\Coloneqq A \multimap (-) \mid \oc\gr r(-) \\
  {\odot} &\Coloneqq {\otimes} \mid {\oplus} \mid {\with} \\
  F[X], G[X] &\Coloneqq X \mid A \mid U(F[X]) \mid F[X] \odot G[X] \\
  A &\Coloneqq \cdots \mid \mu X.~F[X]
\end{align*}

\begin{example}
  We may define $\mathrm{List}_A \coloneqq \mu X.~I \oplus \plr{A \otimes X}$.
\end{example}

In the typing rules, introduction of an inductive type is standard.
For the elimination rule, we follow a similar pattern to other pattern-matching
rules --- $\oplus$-E, $\otimes$-E, and $\oc$-E --- by splitting the context
and typing the eliminand in one half ($\grP$).
We type the continuation in the other half, but because the continuation may
be used multiple times, and in a modal context, we require that $\grQ$ is
preserved by all linear operations.

\begin{displaymath}
  \begin{prooftree}
    \hypo{\grR\gamma \vdash F[\mu X.~F[X]]}
    \infer1[$\mu$-I]{\grR\gamma \vdash \mu X.~F[X]}
  \end{prooftree}
\end{displaymath}
\begin{displaymath}
  \begin{prooftree}
    \hypo{\grR \leq \grP + \grQ}
    \hypo{\grP\gamma \vdash \mu X.~F[X]}
    \hypo{%
      \begin{matrix*}[l]
        \grQ \leq \gr0 \\
        \grQ \leq \grQ + \grQ \\
        \forall \gr r.~\grQ \leq \gr r\grQ
      \end{matrix*}%
    }
    \hypo{\grQ\gamma, \gr1F[C] \vdash C}
    \infer4[$\mu$-E]{\grR\gamma \vdash C}
  \end{prooftree}
\end{displaymath}

\begin{example}\label{thm:list-rules}
  For lists, we can derive the following introduction and elimination rules
  (with usage constraints omitted for brevity when obvious).

  \begin{align*}
    \begin{prooftree}
      \hypo{\grR \leq \gr0}
      \infer1[$I$-I]{I}
      \infer1[$\oplus$-I$_0$]%
      {\grR\gamma \vdash I \oplus \plr{A \otimes \mathrm{List}_A}}
      \infer1[$\mu$-I]{\grR\gamma \vdash \mathrm{List}_A}
    \end{prooftree}
    &&
    \begin{prooftree}
      \hypo{\grR \leq \grP + \grQ}
      \hypo{\grP\gamma \vdash A}
      \hypo{\grQ\gamma \vdash \mathrm{List}_A}
      \infer3[$\otimes$-I]{\grR\gamma \vdash A \otimes \mathrm{List}_A}
      \infer1[$\oplus$-I$_1$]%
      {\grR\gamma \vdash I \oplus \plr{A \otimes \mathrm{List}_A}}
      \infer1[$\mu$-I]{\grR\gamma \vdash \mathrm{List}_A}
    \end{prooftree}
  \end{align*}
  \begin{displaymath}
    \begin{prooftree}
      \hypo{\grP\gamma \vdash \mathrm{List}_A}
      \infer0[Var]{\gr0\gamma, \gr1\plr{I \oplus \plr{A \otimes C}}
        \vdash I \oplus \plr{A \otimes C}}
      \hypo{\nabla^n}
      \hypo{\nabla^c}
      \infer3[$\oplus$-E]{\grQ\gamma, \gr1\plr{I \oplus \plr{A \otimes C}}
        \vdash C}
      \infer2[$\mu$-E]{\grR\gamma \vdash C}
    \end{prooftree}
  \end{displaymath}
  \begin{align*}
    \textrm{where }\nabla^n &\coloneqq
    \begin{prooftree}
      \infer0[Var]{\gr0\gamma, \gr1I \vdash I}
      \hypo{\grQ\gamma \vdash C}
      \infer1[Wk]{\grQ\gamma, \gr0I \vdash C}
      \infer2[$I$-E]{\grQ\gamma, \gr1I \vdash C}
      \infer1[Wk]{\grQ\gamma, \gr0\plr{I \oplus \plr{A \otimes C}}, \gr1I
        \vdash C}
    \end{prooftree}
    \\\\
    \textrm{and }\nabla^c &\coloneqq
    \begin{prooftree}
      \infer0[Var]{\gr0\gamma, \gr1\plr{A \otimes C} \vdash A \otimes C}
      \hypo{\grQ\gamma, \gr1A, \gr1C \vdash C}
      \infer1[Wk]{\grQ\gamma, \gr0\plr{A \otimes C}, \gr1A, \gr1C \vdash C}
      \infer2[$\otimes$-E]{\grQ\gamma, \gr1\plr{A \otimes C} \vdash C}
      \infer1[Wk]%
      {\grQ\gamma, \gr0\plr{I \oplus \plr{A \otimes C}}, \gr1\plr{A \otimes C}
        \vdash C}
    \end{prooftree}
  \end{align*}
\end{example}

Following \cref{sec:lnd}, I want to turn the ad hoc constraints on $\grP$,
$\grQ$, and $\grR$ into the result of some premise combinators.
To do this, I introduce a new combinator $\Box^{0{+}{\times}}$ defined below,
along with the resulting implicit-context typing rules.

\begin{align*}
  \plr{\Box^{0{+}{\times}}\,T}\grR \coloneqq
  \plr{\grR \leq \gr0} \times \plr{\grR \leq \grR + \grR} \times
  \plr{\forall \gr r.~\grR \leq \gr r\grR} \times T\,\grR
\end{align*}

\begin{align*}
  \begin{prooftree}[comb]
    \hypo{\vdash F[\mu X.~F[X]]}
    \infer1[$\mu$-I]{\vdash \mu X.~F[X]}
  \end{prooftree}
  &&
  \begin{prooftree}[comb]
    \hypo{\vdash \mu X.~F[X]}
    \hypo{\sep}
    \hypo{\Box^{0{+}{\times}}\plr{\gr1F[C] \vdash C}}
    \infer3[$\mu$-E]{\vdash C}
  \end{prooftree}
\end{align*}

\begin{example}
  We can state the rules for lists derived in \cref{thm:list-rules} as follows.
  \begin{align*}
    \begin{prooftree}[comb]
      \hypo{I^*}
      \infer1{\vdash \mathrm{List}_A}
    \end{prooftree}
    &&
    \begin{prooftree}[comb]
      \hypo{\vdash A}
      \hypo{\sep}
      \hypo{\vdash \mathrm{List}_A}
      \infer3{\vdash \mathrm{List}_A}
    \end{prooftree}
    &&
    \begin{prooftree}[comb]
      \hypo{\vdash \mathrm{List}_A}
      \hypo{\sep}
      \hypo{\Box^{0{+}{\times}}
        \plr{\vdash C\hskip0.75em\dottimes\hskip0.75em\gr1A, \gr1C \vdash C}}
      \infer3{\vdash C}
    \end{prooftree}
  \end{align*}
\end{example}

\section{Addendum: (lack of) partiality}\label{sec:part}
As we have seen, the way additive and multiplicative rules are
realised algebraically is related to models of separation logic.
Models of separation logic typically use \emph{partial} commutative monoids to
model a heap, so it is tempting to generalise the commutative monoid of
addition in our semirings to a \emph{partial} commutative monoid.
However, we find that the most natural notion of \emph{partial semiring} is
degenerate, in the sense that all partial semirings are actually (total)
semirings.

Recall that a commutative monoid (or commutative monoid object) can be
defined in any symmetric monoidal category.
A partial commutative monoid is exactly a commutative monoid object in the
category of sets and partial functions with the usual monoidal product given
by pairing of objects and morphisms (like the Cartesian product in $\Set$).
However, semirings need a Cartesian category in order to state the interaction
equations between addition and multiplication.
While the category of sets and partial functions is not Cartesian, the
standard way to manufacture a Cartesian category out of a symmetric monoidal
category $\mathcal C$ is to take the category of cocommutative comonoids
$\mathrm{CComon}(\mathcal C)$.
Intuitively, the cocommutative comonoid structure equips the underlying
object $M$ with a \emph{delete} map $\eta : M \to I$ and a \emph{duplicate}
map $\delta : M \to M \otimes M$ which are coherent with respect to each other.
All morphisms in $\mathrm{CComon}(\mathcal C)$ must respect $\eta$ and
$\delta$; in particular, both addition and multiplication must separately
form bimonoids in $\mathcal C$ together with the cocommutative comonoid.

The distributivity laws of semirings are stated below.
I include these to show that the cocommutative comonoids of a monoidal category
give enough structure to state these laws.
The other laws --- that all morphisms respect $\eta$ and $\delta$, that addition
forms a commutative monoid, and that multiplication forms a monoid --- are
standard in symmetric monoidal category theory.

\[
  \begin{tikzpicture}[baseline]
    \path
    (-1,1) node(0) {0}
    (1,2) node(x) {}
    (0,0) node(*) {*}
    (0,-1) node(res) {}
    ;

    \draw (0) -- (*);
    \draw (x) to[out=270,in=45] (*);
    \draw (*) -- (res);
  \end{tikzpicture}
  =\quad
  \begin{tikzpicture}[baseline]
    \path
    (0,0) node(0) {0}
    (0,2) node(x) {}
    (0,-1) node(res) {}
    (0,1) node(del) {$\eta$}
    ;

    \draw (0) -- (res);
    \draw (x) -- (del);
  \end{tikzpicture}
  \quad=
  \begin{tikzpicture}[baseline]
    \path
    (1,1) node(0) {0}
    (-1,2) node(x) {}
    (0,0) node(*) {*}
    (0,-1) node(res) {}
    ;

    \draw (0) -- (*);
    \draw (x) to[out=270,in=135] (*);
    \draw (*) -- (res);
  \end{tikzpicture}
\]
\begin{displaymath}
  \begin{matrix}
    \begin{tikzpicture}[baseline]
      \path
      (-1,2) node(x) {}
      (0,2) node(y) {}
      (-0.5,1) node(+) {+}
      (1,2) node(z) {}
      (0,0) node(*) {*}
      (0,-1) node(res) {}
      ;

      \draw (x) to[out=270,in=135] (+);
      \draw (y) to[out=270,in=45] (+);
      \draw (+) to[out=270,in=135] (*);
      \draw (z) to[out=270,in=45] (*);
      \draw (*) -- (res);
    \end{tikzpicture}
    =
    \begin{tikzpicture}[baseline]
      \path
      (-2,3) node(x) {}
      (-1,3) node(y) {}
      (0,3) node(z) {}
      (0,2) node(dup) {$\delta$}
      (-1,1) node(x*) {*}
      (0,1) node(y*) {*}
      (-0.5,0) node(+) {+}
      (-0.5,-1) node(res) {}
      ;

      \draw (z) -- (dup);
      \draw (x) to[out=270,in=135] (x*);
      \draw (y) to[out=270,in=135] (y*);
      \draw (dup) to[out=-150,in=45] (x*);
      \draw (dup) -- (y*);
      \draw (x*) to[out=270,in=135] (+);
      \draw (y*) to[out=270,in=45] (+);
      \draw (+) -- (res);
    \end{tikzpicture}
    &\phantom{mmmm}&
    \begin{tikzpicture}[baseline]
      \path
      (1,2) node(x) {}
      (0,2) node(y) {}
      (0.5,1) node(+) {+}
      (-1,2) node(z) {}
      (0,0) node(*) {*}
      (0,-1) node(res) {}
      ;

      \draw (x) to[out=270,in=45] (+);
      \draw (y) to[out=270,in=135] (+);
      \draw (+) to[out=270,in=45] (*);
      \draw (z) to[out=270,in=135] (*);
      \draw (*) -- (res);
    \end{tikzpicture}
    =
    \begin{tikzpicture}[baseline]
      \path
      (2,3) node(x) {}
      (1,3) node(y) {}
      (0,3) node(z) {}
      (0,2) node(dup) {$\delta$}
      (1,1) node(x*) {*}
      (0,1) node(y*) {*}
      (0.5,0) node(+) {+}
      (0.5,-1) node(res) {}
      ;

      \draw (z) -- (dup);
      \draw (x) to[out=270,in=45] (x*);
      \draw (y) to[out=270,in=45] (y*);
      \draw (dup) to[out=-30,in=135] (x*);
      \draw (dup) -- (y*);
      \draw (x*) to[out=270,in=45] (+);
      \draw (y*) to[out=270,in=135] (+);
      \draw (+) -- (res);
    \end{tikzpicture}
  \end{matrix}
\end{displaymath}

It is well known that all commutative comonoids in $(\Set, \times)$, and indeed
any Cartesian monoidal category, are trivial, in the sense that every object of
$\Set$ gives rise to exactly one commutative comonoid.
We find in the following two lemmas that this property also holds of
$\plr{\Set_{\mathrm{part}}, \otimes}$.

\begin{lemma}\label{thm:ccomon-exists}
  For each object $X$ in $\plr{\Set_{\mathrm{part}}, {\otimes}}$, there is
  a cocommutative comonoid over $X$.
\end{lemma}
\begin{proof}
  Let $\eta(x) \coloneq ()$ and $\delta(x) \coloneq (x, x)$, with both
  being defined for all $x$.
  Checking that these satisfy the cocommutative comonoid laws is routine.
  Alternatively, we can see that both $\eta$ and $\delta$, being total, are
  morphisms in $\mathrm{Set}$, where it is well known that they form a
  cocommutative comonoid.
  The equations in $\mathrm{Set}$ carry over to $\mathrm{Set}_{\mathrm{part}}$.
\end{proof}

\begin{lemma}\label{thm:ccomon-unique}
  For each object $X$ in $\plr{\Set_{\mathrm{part}}, {\otimes}}$, any
  comonoid over $X$ is the one described in \cref{thm:ccomon-exists}.
\end{lemma}
\begin{proof}
  The left unit law says that, for all $x$ and $x'$, we have
  $\exists y.~\delta(x) = (y, x') \land \eta(y) = ()$ if and only if $x = x'$.
  Letting $x'$ be $x$ and reading from right to left, we get that there is
  some $y$ such that $\delta(x) = (y, x)$ and $\eta(y) = ()$.
  Symmetrically, from the right unit law, we get some $z$ such that
  $\delta(x) = (x, z)$ and $\eta(z) = ()$.
  But because $\delta$, being a partial function, is deterministic, we have
  $(y, x) = (x, z)$, giving us that $y = z = x$, and $\delta(x) = (x, x)$.
  Moreover, because the chosen $y$ is equal to $x$, we have for all $x$ that
  $\eta(x) = ()$.
\end{proof}

That a morphism $f$ respects the $\eta$ given in \cref{thm:ccomon-exists} is
equivalent to saying that $f$ is total.
Therefore, all possible semiring operators in
$\mathrm{CComon}\plr{\Set_{\mathrm{part}}, \otimes}$ are total, meaning that
there is a corresponding semiring in $\plr{\Set, \times}$.

The above reasoning shows that semirings in the category of sets and partial
functions are not worth studying.
If we want partiality, there appear to be two options.
The first option is to give up on multiplication.
We could imagine replacing the binary multiplication operator by a set of
unary modalities satisfying fewer laws.
In particular, I make little use of addition on the left of a multiplication,
and multiplying by $\gr0$ on the left (as done by $\oc\gr0$) is unwanted in some
cases (such as when encoding DILL and PD, as in \cref{sec:translation}).
With unary modalities, we could expect all of the required laws to be
expressible in a symmetric monoidal category.
The second option is to use a different notion of partiality.
The notion of partiality given by the category of sets and partial functions is
``strict'', in that composing with an everywhere-undefined function yields an
everywhere-undefined function.
With a non-strict notion of partial function, we may be able to have interesting
partial semirings.

