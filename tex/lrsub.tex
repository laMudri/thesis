\def\LRKits{../agda/processed-latex/LRKits.tex}

In this section, I show that, using the notion of \emph{environment} derived in
\cref{sec:lrkits}, we can replicate the Agda proofs from
\cref{sec:syntactic-kits} in the usage-aware setting.
From \cref{sec:lenv}, we know that environments are preserved under all
syntax-forming operations: zero, addition, scaling, and binding.
What is left is to show how these properties are deployed, and also how to
go on and prove the admissibility of simultaneous renaming, simultaneous
substitution, and then single substitution.

There are a few notational changes necessary in the Agda code.
Usage vectors, elsewhere called $\grP$, $\grQ$, and $\grR$ are rendered as
\AgdaBound{P}, \AgdaBound{Q}, and \AgdaBound{R}, respectively.
Environments, elsewhere notated $\Gamma \env\V \Delta$, are rendered as
\AgdaRecord{[}\AgdaSpace{}\AgdaBound{$\V$}\AgdaSpace{}\AgdaRecord{]}%
\AgdaSpace{}\AgdaBound{$\Gamma$}\AgdaSpace{}\AgdaRecord{$\Rightarrow^e$}%
\AgdaSpace{}\AgdaBound{$\Delta$}.

We start with a slightly modified definition of \AgdaRecord{Kit}.
We saw in \cref{thm:lr-bind} that in the usage-annotated context, we replace
general weakening of $\V$-values by weakening by $\gr0$-use variables.
Meanwhile, the function $\mathrm{vr}$, also seen in \cref{thm:lr-bind}, maps
usage-checked variables to $\V$-values, and the function $\mathrm{tm}$, used
to coerce $V$-values yielded by the environment into terms, stays the same.
I state weakening in a slightly different way than previously, so as to help
unification against a known result type.
The type \AgdaFunction{Weakening}\AgdaSpace{}\AgdaBound{$\V$} can be read as
saying that, for any context $\grP\gamma$ of shape $s + t$, if the right of
$\grP$ is below $\gr0$, then a value in the left part of $\grP\gamma$ weakens
to a value in the whole of $\grP\gamma$.

\ExecuteMetaData[\LRKits]{Kit}

To demonstrate the important points succinctly, I cut \name{} down to just the
$\oc\gr r$-fragment.
The introduction rule and pattern-matching eliminator feature scaling, addition,
and variable binding, missing out only on sharing (which is trivial) and zero
(which is simpler than, and analogous to, addition).
The resulting type of well typed terms is below.

\ExecuteMetaData[\LRKits]{Tm}

Given a \AgdaRecord{Kit}, \cref{thm:lr-bind} looks like the following.
The \AgdaField{lookup} clauses still contain essentially the same structure as
in the intuitionistic case: discriminating on whether the variable is old or
new, using the given environment \AgdaBound{$\rho$} and weakening on the old
variables, and using \AgdaField{vr} to repackage new variables.
I will not explain any of the algebraic manipulations here; see
\cref{thm:lr-bind}.

\ExecuteMetaData[\LRKits]{bindEnv}

Given \AgdaFunction{bindEnv}, along with \AgdaFunction{env-+} and
\AgdaFunction{env-*}, standing for \cref{thm:lr-env-add} and
\cref{thm:lr-env-scale}, respectively, we can reproduce the syntactic traversal
\AgdaFunction{trav}.
With all these lemmas in place, we can see how writing \AgdaFunction{trav}
becomes routine --- when processing a rule, we work our way up through the
premise connectives, applying \AgdaFunction{env-*} wherever we see a
\AgdaFunction{$\cdot^c$}, \AgdaFunction{env-+} wherever we see a
\AgdaFunction{$*^c$}, and \AgdaFunction{bindEnv} wherever we see a
\AgdaFunction{Bind}.
We when use whatever environments (with names beginning with
\AgdaBound{$\rho$}) and whatever usage vector splitting facts (with names
beginning with \AgdaBound{sp}) come out of this process to recursively
traverse the subterms and recombine the results.

\ExecuteMetaData[\LRKits]{trav}

Instantiating the generic syntactic traversal \AgdaFunction{trav} to renaming
looks just like it did in the intuitionistic case.
I have consistently replaced intuitionistic variables by linear variables, so
\AgdaFunction{id} and \AgdaInductiveConstructor{var} still work to embed
variables into variables and terms, respectively.
Weakening for variables \AgdaFunction{$\swarrow^v$} (not pictured) has been
updated to note that, for $\grP \leq \bra x$ and $\grR \leq \gr0$, we also have
$\begin{pmatrix} \grP & \grR \end{pmatrix} \leq \bra{{\swarrow}x}$.

\ExecuteMetaData[\LRKits]{var-kit}

In the intuitionistic case, environments were just functions, so we passed the
variable weakening function \AgdaFunction{$\swarrow^v$} to the function
\AgdaFunction{ren} to yield a term weakening function.
However, a usage-aware environment is a function packed together with some
extra data.
As such, we must make an environment version of \AgdaFunction{$\swarrow^v$}.
I start with a general lemma \AgdaFunction{$\swarrow$\^{}Env}, stating that if
$\V$ supports weakening, then so do $\V$-environments (in their domain
context).
This lemma then specialises to variables, with the identity renaming
\AgdaFunction{id\^{}Env} on the left part of the context and the proof
\AgdaBound{R0} that the right part of the context is below $\gr0$ combining
to give the desired weakening environment.

\ExecuteMetaData[\LRKits]{dlv-env}

This is what we need to instantiate \AgdaFunction{trav} for substitution.

\ExecuteMetaData[\LRKits]{sub}

Finally, the simultaneous substitution \AgdaFunction{sub} specialises to
single substitution \AgdaFunction{sub[-]}.
Single substitution is stated as an admissible rule below.
To substitute in for $\gr r$-many $A$ in the second term, we need to derive
one $A$ with usages $\grP$, and then assert that the result can handle the
usages of the original term $\grQ$, plus $\gr r$-many copies of $\grP$.

\[
  \begin{prooftree}
    \hypo{\grR \leq \grQ + \gr r\grP}
    \hypo{\grP\gamma \vdash A}
    \hypo{\grQ\gamma, \gr rA \vdash B}
    \infer3[singleSub]{\grR\gamma \vdash B}
  \end{prooftree}
\]

The proof strategy for producing the substitution \AgdaFunction{$\sigma$} is
to proceed structurally on the codomain context $\grQ\gamma, \gr rA$ using
\cref{thm:construct-env}, applying the identity substitution
\AgdaFunction{id\^{}Env} on the $\gamma$ half, and dropping the term
\AgdaBound{t} in place of the variable we are substituting for.

\ExecuteMetaData[\LRKits]{subSingle}
