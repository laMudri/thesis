\def\LRKits{../agda/processed-latex/LRKits.tex}

I now show that, using the notion of \emph{environment} derived in
\cref{sec:lrkits}, we can replicate the Agda proofs from
\cref{sec:syntactic-kits} in the usage-aware setting of $\name$.
From \cref{sec:lenv}, we know that environments are preserved under all
syntax-forming operations: zero, addition, scaling, and binding.
What is left is to show how these properties are deployed, and also how to
go on and prove the admissibility of simultaneous renaming, simultaneous
substitution, and then single substitution.

There are a few notational changes necessary in the Agda code, compared to the
typeset mathematics above.
Usage vectors, elsewhere called $\grP$, $\grQ$, and $\grR$ are rendered as
\AgdaBound{P}, \AgdaBound{Q}, and \AgdaBound{R}, respectively.
Usage contexts and typing contexts are tied together with the
\AgdaInductiveConstructor{ctx} constructor, rather than simple juxtaposition.
Environments, elsewhere notated $\Gamma \env\V \Delta$, are rendered as
\AgdaRecord{[}\AgdaSpace{}\AgdaBound{$\V$}\AgdaSpace{}\AgdaRecord{]}%
\AgdaSpace{}\AgdaBound{$\Gamma$}\AgdaSpace{}\AgdaRecord{$\Rightarrow^e$}%
\AgdaSpace{}\AgdaBound{$\Delta$}.

I start with the definition \AgdaFunction{Weakening}, which says what it means
for a family of values \AgdaBound{$\V$} to respect one step of weakening on the
right by $\gr0$-use variables.
I state weakening in a slightly different way to what appears in the statement
of \cref{thm:lr-bind}, so as to help
unification against a known result type (avoiding the problem described by
\citet{McBride12} as \emph{green slime}).
The type \AgdaFunction{Weakening}\AgdaSpace{}\AgdaBound{$\V$} can be read as
saying that, for any context $\grP\gamma$ of shape $s + t$, if the right of
$\grP$ is below $\gr0$, then a value in the left part of $\grP\gamma$ weakens
to a value in the whole of $\grP\gamma$.

\ExecuteMetaData[\LRKits]{Weakening}

Given this new definition of \AgdaFunction{Weakening}, the record
\AgdaRecord{Kit} remains largely unchanged relative to what we saw in
\cref{sec:syntactic-kits}.
We still want $\V$-values to respect weakening (as used in \cref{thm:lr-bind}),
and to support maps \AgdaField{vr} from variables (also used in
\cref{thm:lr-bind}) and \AgdaField{tm} into terms (used when traversal reaches a
variable, we get a value from the environment, and want to produce a term from
that value).
As well as \AgdaFunction{Weakening}, note that \AgdaRecord{\_$\sqni$\_} and, of
course, \AgdaDatatype{\_$\vdash$\_} have different definitions to the
corresponding intuitionistic notions, but they still represent morally the same
parts of the language and its metatheory.

\ExecuteMetaData[\LRKits]{Kit}

To demonstrate the important points succinctly, I cut \name{} down to just the
$\oc\gr r$-fragment.
The introduction rule and pattern-matching eliminator feature scaling, addition,
and variable binding, missing out only on sharing (which is trivial) and zero
(which is simpler than, and analogous to, addition).
The resulting type of well typed terms is below.

\ExecuteMetaData[\LRKits]{Tm}

Given a \AgdaRecord{Kit}\AgdaSpace{}\AgdaBound{$\V$},
\cref{thm:lr-bind} gives a function with the following type.

\ExecuteMetaData[\LRKits]{bindEnv}

Given \AgdaFunction{bindEnv} (\cref{thm:lr-bind}), \AgdaFunction{env-+}
(\cref{thm:lr-env-add}), and \AgdaFunction{env-*} (\cref{thm:lr-env-scale}),
we can reproduce the syntactic traversal \AgdaFunction{trav}.
Similarly to the unchanged high-level definition of \AgdaRecord{Kit}, we are
aiming for an unchanged traversal principle expressed by the rule below.
When $\V$ has a \AgdaRecord{Kit} structure, and we have a $\V$-environment from
$\Gamma$ to $\Delta$, we can transform a term in $\Delta$ to a term in $\Gamma$
of the same type.

\[
  \begin{prooftree}
    \hypo{\text{\AgdaRecord{Kit}\AgdaSpace{}}\V}
    \hypo{\Gamma \env\V \Delta}
    \hypo{\Delta \vdash A}
    \infer3[Trav]{\Gamma \vdash A}
  \end{prooftree}
\]

With all the lemmas of \cref{sec:lenv} in place, writing \AgdaFunction{trav}
becomes routine.
When processing a rule, we work our way up through the
premise connectives, applying \AgdaFunction{env-*} wherever we see a
\AgdaFunction{$\cdot^c$}, \AgdaFunction{env-+} wherever we see a
\AgdaFunction{$*^c$}, and \AgdaFunction{bindEnv} wherever we see a
\AgdaFunction{Bind}.
We then use whatever environments (with names beginning with
\AgdaBound{$\rho$}) and whatever usage vector splitting facts (with names
beginning with \AgdaBound{sp}) come out of this process to recursively
traverse the subterms and recombine the results.

\ExecuteMetaData[\LRKits]{trav}

Instantiating the generic syntactic traversal \AgdaFunction{trav} to renaming
looks just like it did in the intuitionistic case.
I have consistently replaced intuitionistic variables by linear variables, so
\AgdaFunction{id} and \AgdaInductiveConstructor{var} still work to embed
variables into variables and terms, respectively.
Weakening for variables \AgdaFunction{$\swarrow^v$} (not pictured) has been
updated to note that, for $\grP \leq \bra x$ and $\grR \leq \gr0$, we also have
$\begin{pmatrix} \grP & \grR \end{pmatrix} \leq \bra{{\swarrow}x}$.

\ExecuteMetaData[\LRKits]{var-kit}

In the intuitionistic case, environments were just functions, so we passed the
variable weakening function \AgdaFunction{$\swarrow^v$} to the function
\AgdaFunction{ren} to yield a term weakening function.
However, a usage-aware environment is a function packed together with usage
distribution data.
As such, we must make an environment version of \AgdaFunction{$\swarrow^v$}.
I start with a general lemma \AgdaFunction{$\swarrow$\^{}Env}, stating that if
$\V$ supports weakening, then so do $\V$-environments (in their domain
context).
This lemma then specialises to variables, with the identity renaming
\AgdaFunction{id\^{}Env} on the left part of the context and the proof
\AgdaBound{R0} that the right part of the context is below $\gr0$ combining
to give the desired weakening environment.

\ExecuteMetaData[\LRKits]{dlv-env}

This is what we need to instantiate \AgdaFunction{trav} for substitution.
As a reminder, I also give the type of \AgdaFunction{sub} in rule form.

\ExecuteMetaData[\LRKits]{sub}
\[
  \ebrule{%
    \hypo{\Gamma \env\vdash \Delta}
    \hypo{\Delta \vdash B}
    \infer2[sub]{\Gamma \vdash B}
  }
\]

Finally, the simultaneous substitution \AgdaFunction{sub} specialises to
single substitution.

\begin{corollary}[Single substitution]\label{thm:single-sub}
  The following equivalent rules are admissible.
  \begin{mathpar}
    \ebrule{%
      \hypo{\grR \leq \gr r\grP + \grQ}
      \hypo{\grP\gamma \vdash A}
      \hypo{\grQ\gamma, \gr rA \vdash B}
      \infer3{\grR\gamma \vdash B}
    }
    \and
    \ebrule[comb]{%
      \hypo{\gr r \cdot \plr{{} \vdash A}}
      \hypo{\sep}
      \hypo{\gr rA \vdash B}
      \infer3{{} \vdash B}
    }
  \end{mathpar}
\end{corollary}
\begin{proof}
  It is enough to construct a substitution of type
  $\grR\gamma \env\vdash \grQ\gamma, \gr rA$.
  To do this, we use \cref{thm:construct-env} cases $\sep(\rightarrowtriangle)$
  and $\cdot(\rightarrowtriangle)$ on inequalities
  $\grR \leq \grQ + \gr r\grP$ and $\gr r\grP \leq \gr r\grP$ respectively to
  leave us needing a substitution of type $\grQ\gamma \env\vdash \grQ\gamma$ and
  a term of type $\grP\gamma \vdash A$.
  For the substitution, we give the identity substitution (\cref{thm:env-id}),
  and we have the term as a hypothesis.
\end{proof}
