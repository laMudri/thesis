\def\LRKits{../agda/latex/LRKits.tex}

From \cref{sec:lrkits,sec:lenv}, we have fixed a notion of \emph{environment}
and proved that it is preserved under all syntax-forming operations.
In this section, I put these properties together in the style of
\cref{sec:syntactic-kits} to prove the admissibility of simultaneous renaming,
simultaneous substitution, and then single substitution.

We start with a slightly modified definition of \AgdaRecord{Kit}.
We saw in \cref{thm:lr-bind} that in the usage-annotated context, we replace
general weakening of $\V$-values by weakening by $\gr0$-use variables.
Meanwhile, the function $\mathrm{vr}$, also seen in \cref{thm:lr-bind}, maps
usage-checked variables to $\V$-values, and the function $\mathrm{tm}$, used
to coerce $V$-values yielded by the environment into terms, stays the same.
I state weakening in a slightly different way than previously, so as to help
unification against a known result type.
The type \AgdaFunction{Weakening}\AgdaSpace{}\AgdaBound{$\V$} can be read as
saying that, for any context $\grP\gamma$ of shape $s + t$, if the right of
$\grP$ is below $\gr0$, then a value in the left part of $\grP\gamma$ weakens
to a value in the whole of $\grP\gamma$.

\ExecuteMetaData[\LRKits]{Kit}

To demonstrate the important points succinctly, I cut \name{} down to just the
$\oc\gr r$-fragment.
The introduction rule and pattern-matching eliminator feature scaling, addition,
and variable binding, missing out only on sharing (which is trivial) and zero
(which is simpler than, and analogous to, addition).
The resulting type of well typed terms is below.

\ExecuteMetaData[\LRKits]{Tm}

Given a \AgdaRecord{Kit}, \cref{thm:lr-bind} looks like the following.
The \AgdaField{lookup} clauses still contain essentially the same structure as
in the intuitionistic case: discriminating on whether the variable is old or
new, using the given environment \AgdaBound{$\rho$} and weakening on the old
variables, and using \AgdaField{vr} to repackage new variables.
I will not explain any of the algebraic manipulations here; see
\cref{thm:lr-bind}.

\ExecuteMetaData[\LRKits]{bindEnv}

Given \AgdaFunction{bindEnv}, along with \AgdaFunction{env-+} and
\AgdaFunction{env-*}, standing for \cref{thm:lr-env-add} and
\cref{thm:lr-env-scale}, respectively, we can reproduce the syntactic traversal
\AgdaFunction{trav}.
With all these lemmas in place, we can see how writing \AgdaFunction{trav}
becomes routine --- when processing a rule, we work our way up through the
premise connectives, applying \AgdaFunction{env-*} wherever we see a
\AgdaFunction{$\cdot^c$}, \AgdaFunction{env-+} wherever we see a
\AgdaFunction{$*^c$}, and \AgdaFunction{bindEnv} wherever we see a
\AgdaFunction{Bind}.
We when use whatever environments (with names beginning with
\AgdaBound{$\rho$}) and whatever usage vector splitting facts (with names
beginning with \AgdaBound{sp}) come out of this process to recursively
traverse the subterms and recombine the results.

\ExecuteMetaData[\LRKits]{trav}
