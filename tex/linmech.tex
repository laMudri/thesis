In this section, I give an overview of techniques which have been used in
previous work to encode linear logic in proof assistants.
To my knowledge, there have been no other significant attempts to mechanise
semiring-annotated logics.

\subsection{Typing with leftovers}

Typing with leftovers~\cite{allais17} is a technique developed to capture a
straightforward approach to linearity checking.
The idea is to, at any point, consider an input context, a term, and an output
context.
The input context contains all of the variables in scope, while the output
context is the same minus any variables used by the term.
Checking of adjacent terms, for example in a $\otimes$-introduction, is done by
threading the context through from the output of the first term to the input of
the second.
Bound variables are introduced in the input context of the term in which they
are bound, and are expected to be absent in the output context of that term.
\Cref{fig:twl,fig:twl-mult} give rules in this style for the multiplicative
fragment of intuitionistic linear logic.

\begin{figure}
  \begin{align*}
    \Gamma, \Delta, \Theta
    &\Coloneqq {\cdot} \mid \Gamma, \gr1x : A \mid \Gamma, \gr0x : A \\
    \mathcal{S} &\Coloneqq \Gamma \vdash M : A \boxtimes \Delta
  \end{align*}
  \caption{Typing with leftovers, context and sequent syntax}
  \label{fig:twl}
\end{figure}

\begin{figure}
  \begin{mathpar}
    \ebrule{%
      \infer0[Var]{\Gamma, \gr1x : A \vdash x : A \boxtimes \Gamma, \gr0x : A}
    }
    \and
    \ebrule{%
      \infer0[$I$-I]{\Gamma \vdash {*} : I \boxtimes \Gamma}
    }
    \and
    \ebrule{%
      \hypo{\Gamma \vdash M : I \boxtimes \Delta}
      \hypo{\Delta \vdash N : C \boxtimes \Theta}
      \infer2[$I$-E]{\Gamma \vdash \text{let }{*} = M\text{ in }N : C
        \boxtimes \Theta}
    }
    \and
    \ebrule{%
      \hypo{\Gamma \vdash M : A \boxtimes \Delta}
      \hypo{\Delta \vdash N : B \boxtimes \Theta}
      \infer2[$\otimes$-I]{\Gamma \vdash (M, N) : A \otimes B \boxtimes \Theta}
    }
    \and
    \ebrule{%
      \hypo{\Gamma \vdash M : A \otimes B \boxtimes \Delta}
      \hypo{\Delta, \gr1x : A, \gr1y : B \vdash N : C \boxtimes
        \Theta, \gr0x : A, \gr0y : B}
      \infer2[$\otimes$-E]{\Gamma \vdash \text{let }(x, y) = M\text{ in }N : C
        \boxtimes \Theta}
    }
    \and
    \ebrule{%
      \hypo{\Gamma, \gr1x : A \vdash M : B \boxtimes \Delta, \gr0x : A}
      \infer1[$\multimap$-I]{\Gamma \vdash \lambda x.~M : A \multimap B
        \boxtimes \Delta}
    }
    \and
    \ebrule{%
      \hypo{\Gamma \vdash M : A \multimap B \boxtimes \Delta}
      \hypo{\Delta \vdash N : A \boxtimes \Theta}
      \infer2[$\multimap$-E]{\Gamma \vdash M\,N : B \boxtimes \Theta}
    }
  \end{mathpar}
  \caption{Typing with leftovers, multiplicative fragment}
  \label{fig:twl-mult}
\end{figure}

The original paper extends the logic covered to binary additives --- $\with$ and
$\oplus$ --- with rules that check that terms agree on output contexts, as seen
in \cref{fig:twl-add}.
However, it is less clear how to handle nullary additives --- $\top$ and $0$ ---
as we would have 0 (rather than 2) candidates for the output context.
At some level, this problem is unavoidable in a system modelling linearity
checking because any checking strategy will expose the ambiguity in sequents
like $\gr1x : A \vdash (\{\}, \{\}) : \top \otimes \top$ of whether the variable
$x$ was used in the left half or the right half.
As such, we may expect uses of $\top$-introduction (and similarly
$0$-elimination) to be annotated with which variables they use, in which case we
get a viable typing with leftovers rule.

\begin{figure}
  \begin{mathpar}
    \ebrule{%
      \hypo{\Gamma \vdash M : A \boxtimes \Delta}
      \hypo{\Gamma \vdash N : B \boxtimes \Delta}
      \infer2[$\with$-I]{\Gamma \vdash \{M,N\} : A \with B \boxtimes \Delta}
    }
    \and
    \ebrule{%
      \hypo{\Gamma \vdash M : A \oplus B \boxtimes \Delta}
      \hypo{\Delta, \gr1x : A \vdash N : C \boxtimes \Theta, \gr0x : A}
      \hypo{\Delta, \gr1y : B \vdash O : C \boxtimes \Theta, \gr0y : B}
      \infer3[$\oplus$-E]{\Gamma \vdash
        \text{case }M\text{ of }\{x.~N; y.~O\} : C \boxtimes \Theta}
    }
  \end{mathpar}
  \caption{Typing with leftovers, a selection of the additive rules}
  \label{fig:twl-add}
\end{figure}

The original paper also does not show how to capture the exponential modality
$\oc$.
We could perhaps, inspired by DILL, distinguish between linear variables and
intuitionistic variables.
This gives rules like those of \cref{fig:twl-exp}.
The important invariant is that linear and intuitionistic variables stay
distinct, so any intuitionistic variable in the input context (annotated by
$\gr\omega$) must be intuitionistic in the output context.
In the $\oc$-introduction rule, there are several choices we could make, but I
have chosen to keep all the linear variables in scope but used so as to match
the general style of variables staying in scope with the $\gr0$ annotation..

\begin{figure}
  \begin{mathpar}
    \ebrule{%
      \infer0[IVar]{\Gamma, \gr\omega x : A \vdash x : A
        \boxtimes \Gamma, \gr\omega x : A}
    }
    \and
    \ebrule{%
      \hypo{\gr0\gamma, \gr0\delta, \gr\omega\theta \vdash
        M : A \boxtimes \gr0\gamma, \gr0\delta, \gr\omega\theta}
      \infer1[$\oc$-I]{\gr0\gamma, \gr1\delta, \gr\omega\theta \vdash
        [M] : \oc A \boxtimes \gr0\gamma, \gr1\delta, \gr\omega\theta}
    }
    \and
    \ebrule{%
      \hypo{\Gamma \vdash M : \oc A \boxtimes \Delta}
      \hypo{\Delta, \gr\omega x : A \vdash N : C
        \boxtimes \Theta, \gr\omega x : A}
      \infer2[$\oc$-E]{\Gamma \vdash
        \text{let }[x] = M\text{ in }N : C \boxtimes \Theta}
    }
  \end{mathpar}
  \caption{Typing with leftovers, a possible way to capture $\oc$}
  \label{fig:twl-exp}
\end{figure}

However, this adaptation of the DILL style does not obviously generalise to
semiring annotations.
Even for the multiplicative fragment, we seem to be working against the
direction of addition, instead using a subtraction operation whenever we use a
variable.
For exponentials, though, and particularly the $\oc$-introduction rule, what I
have done seems ad-hoc, not based on any pointwise algebraic operation.

Also, the unusual form of sequents can cause some problems when working with a
typing with leftovers system.
For example, a traditional intuitionistic linear logic sequent
$\Gamma \vdash M : A$ corresponds to many different typing with leftovers
sequents:
\begin{itemize}
  \item $\gr1\Gamma \vdash M : A \boxtimes \gr0\Gamma$
  \item $\gr1\Gamma, \gr1x : B \vdash M : A \boxtimes \gr0\Gamma, \gr1x : B$
  \item $\gr1\Gamma, \gr0x : B \vdash M : A \boxtimes \gr0\Gamma, \gr0x : B$
  \item \&c.
\end{itemize}

Generally, any variable not used in the term can be added to both the input and
the output context with the same annotation.
Many of these variations are likely to appear in various typing derivations,
depending on what terms surround a given subterm.
This means that if we want to implement substitution, which involves putting a
term into an unknown surrounding, we have to navigate these different forms via
the \emph{framing property}.

The unusal form of sequents also somewhat obscures the denotational semantics of
a typing with leftovers system.
Though the $\boxtimes$ notation suggests a semantics into symmetric monoidal
closed categories where terms are morphisms from one iterated tensor product
(the input context) to another (the type and output context), the syntax is
incomplete for this semantics because we cannot produce anything interesting in
the output context.

\subsection{Yalla}

\Citet{laurent18} presents a collection of linear logics in a uniform style, and
various proofs relating them.
The logics share varying amounts of definitions and theorems --- for example,
the main linear logic is parametrised on whether to include mix rules and
whether to restrict exchange to cyclic permutations, whereas systems like the
Lambek calculus (with no exchange) and polarised linear logic are defined
separately from scratch.

The style used in Yalla is to realise sequents as lists of formulas.
The active formula tends to be forced to be the first formula in such a list,
with an explicit exchange rule being used to move such formulas into place.
\Citet{laurent18} explains why explicit permutations are necessary in such a
system, and why using multisets instead of lists would conflate distinct proofs
when there are repeated formulas in a sequent.
The system I will present in the next chapter avoids permutations essentially by
being a refinemet of the system presented in the previous chapter, which has an
ordered context and specific references (de Bruijn indices) into it.

Despite making sure to distinguish distinct proofs, the Yalla library does not
define any equational theories of proofs, so does not prove any results relying
on these distinctions.
While the complication of defining an equational theory in the presence of an
exchange rule is probably largely inevitable, having the exchange rule
introduces redundancy such that many equivalent proofs are not equal as data
structures.
The mechanisations presented in \cref{sec:simple} all sought to reduce this kind
of redundancy for intuitionistic logic, so that the only non-trivial equations
in the equational theory are $\beta$- and $\eta$-rules (i.e.\ computationally
interesting rules).
I will restore this property of the representation in \cref{sec:semirings}.

\begin{itemize}
  \item Sequent calculus
  \item Specialised to linear logic
\end{itemize}
