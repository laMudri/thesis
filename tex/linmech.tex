In this section, I give an overview of techniques which have been used in
previous work to mechanise linear logic in proof assistants.
Na\"{i}ve approaches often struggle to represent concatenation of contexts in a
way which is amenable to the way dependent type theory-based proof assistants
work.
Problems even arise when working rigorously on paper when trying to avoid an
explicit exchange rule, such as how \cref{def:DILL-split} is not a precise
definition of a binary operator on lists.

\subsection{Typing with leftovers}

Typing with leftovers, introduced by \citet{allais17}, is a technique developed
to specify an algorithm for linear type checking as a declarative type system.
The idea is to consider an input context, a term, and an output context, where
the input context contains all of the variables in scope, and the output
context is the same minus any variables used by the term.
Type-checking of adjacent subterms of, for example, an application of
the $\otimes$-introduction rule, is done by
threading the context through from the output of the first term to the input of
the second.
Bound variables are introduced in the input context of the term in which they
are bound, and are expected to be absent in the output context of that term.

\Cref{fig:twl,fig:twl-mult} give rules in the typing-with-leftovers style for
the multiplicative fragment of intuitionistic linear logic.
Where \citeauthor{allais17} marks \emph{fresh} and \emph{stale} variables, I use
the notation I will use starting in \cref{sec:semirings}, labelling such
variables with $\gr 1$ and $\gr 0$, respectively.
Intuitively, the number describes how many more times that variable is to be
used.

\begin{figure}
  \begin{align*}
    \Gamma, \Delta, \Theta
    &\Coloneqq {\cdot} \mid \Gamma, \gr1x : A \mid \Gamma, \gr0x : A \\
    \mathcal{S} &\Coloneqq \Gamma \vdash M : A \boxtimes \Delta
  \end{align*}
  \caption{Typing with leftovers, context and sequent syntax}
  \label{fig:twl}
\end{figure}

\begin{figure}
  \begin{mathpar}
    \ebrule{%
      \infer0[Var]{\Gamma, \gr1x : A \vdash x : A \boxtimes \Gamma, \gr0x : A}
    }
    \and
    \ebrule{%
      \infer0[$I$-I]{\Gamma \vdash {*} : I \boxtimes \Gamma}
    }
    \and
    \ebrule{%
      \hypo{\Gamma \vdash M : I \boxtimes \Delta}
      \hypo{\Delta \vdash N : C \boxtimes \Theta}
      \infer2[$I$-E]{\Gamma \vdash \text{let }{*} = M\text{ in }N : C
        \boxtimes \Theta}
    }
    \and
    \ebrule{%
      \hypo{\Gamma \vdash M : A \boxtimes \Delta}
      \hypo{\Delta \vdash N : B \boxtimes \Theta}
      \infer2[$\otimes$-I]{\Gamma \vdash (M, N) : A \otimes B \boxtimes \Theta}
    }
    \and
    \ebrule{%
      \hypo{\Gamma \vdash M : A \otimes B \boxtimes \Delta}
      \hypo{\Delta, \gr1x : A, \gr1y : B \vdash N : C \boxtimes
        \Theta, \gr0x : A, \gr0y : B}
      \infer2[$\otimes$-E]{\Gamma \vdash \text{let }(x, y) = M\text{ in }N : C
        \boxtimes \Theta}
    }
    \and
    \ebrule{%
      \hypo{\Gamma, \gr1x : A \vdash M : B \boxtimes \Delta, \gr0x : A}
      \infer1[$\multimap$-I]{\Gamma \vdash \lambda x.~M : A \multimap B
        \boxtimes \Delta}
    }
    \and
    \ebrule{%
      \hypo{\Gamma \vdash M : A \multimap B \boxtimes \Delta}
      \hypo{\Delta \vdash N : A \boxtimes \Theta}
      \infer2[$\multimap$-E]{\Gamma \vdash M\,N : B \boxtimes \Theta}
    }
  \end{mathpar}
  \caption{Typing with leftovers, multiplicative fragment}
  \label{fig:twl-mult}
\end{figure}

The original paper extends the logic covered to binary additives --- $\with$ and
$\oplus$ --- with rules that check that terms agree on output contexts, as seen
in \cref{fig:twl-add}.
However, it is less clear how to handle nullary additives --- $\top$ and $0$ ---
as we would have 0 (rather than 2) potential candidates for the output context.
At some level, this problem is unavoidable in a system modelling linearity
checking because any checking strategy will expose the ambiguity in sequents
like $\gr1x : A \vdash (\langle\rangle, \langle\rangle) : \top \otimes \top$ of
whether the variable
$x$ was consumed in the left half or the right half.
Such an example is also considered in related work on proof search for linear
logics, such as the work of \citet[p.\ 11]{WH94} and \citet[p.\ 150]{CHP00}.
It is not immediately clear whether the different solutions proposed by these
papers will apply to \citeauthor{allais17}' and my settings, given that they
both act on a set of formulas restricted to facilitate proof search.
The solutions also add significant, seemingly somewhat ad hoc, structure to the
syntax of sequents, with no semantic justification (rather being justified by
making their respective proof search algorithms efficient).
%As such, we may expect uses of $\top$-introduction (and similarly
%$0$-elimination) to be annotated with which variables they consume, in which case we
%get a viable typing with leftovers rule.

\begin{figure}
  \begin{mathpar}
    \ebrule{%
      \hypo{\Gamma \vdash M : A \boxtimes \Delta}
      \hypo{\Gamma \vdash N : B \boxtimes \Delta}
      \infer2[$\with$-I]{\Gamma \vdash \{M,N\} : A \with B \boxtimes \Delta}
    }
    \and
    \ebrule{%
      \hypo{\Gamma \vdash M : A \oplus B \boxtimes \Delta}
      \hypo{\Delta, \gr1x : A \vdash N : C \boxtimes \Theta, \gr0x : A}
      \hypo{\Delta, \gr1y : B \vdash O : C \boxtimes \Theta, \gr0y : B}
      \infer3[$\oplus$-E]{\Gamma \vdash
        \text{case }M\text{ of }\{x.~N; y.~O\} : C \boxtimes \Theta}
    }
  \end{mathpar}
  \caption{Typing with leftovers, a selection of the additive rules}
  \label{fig:twl-add}
\end{figure}

The original paper also does not show how to capture the exponential modality
$\oc$.
The solution given by both \citet{WH94} and \citet{CHP00} is, as in DILL, to
distinguish between linear variables and intuitionistic variables.
This gives rules like those of \cref{fig:twl-exp}.
The important invariant is that linear and intuitionistic variables stay
distinct, so any intuitionistic variable in the input context (annotated by
$\gr\omega$) must be intuitionistic in the output context.
%In the $\oc$-introduction rule, there are several choices we could make, but I
%have chosen to keep all the linear variables in scope but used so as to match
%the general style of variables staying in scope with the $\gr0$ annotation.

\begin{figure}
  \begin{mathpar}
    \ebrule{%
      \infer0[IVar]{\Gamma, \gr\omega x : A \vdash x : A
        \boxtimes \Gamma, \gr\omega x : A}
    }
    \and
    \ebrule{%
      \hypo{\gr0\gamma, \gr0\delta, \gr\omega\theta \vdash
        M : A \boxtimes \gr0\gamma, \gr0\delta, \gr\omega\theta}
      \infer1[$\oc$-I]{\gr0\gamma, \gr1\delta, \gr\omega\theta \vdash
        [M] : \oc A \boxtimes \gr0\gamma, \gr1\delta, \gr\omega\theta}
    }
    \and
    \ebrule{%
      \hypo{\Gamma \vdash M : \oc A \boxtimes \Delta}
      \hypo{\Delta, \gr\omega x : A \vdash N : C
        \boxtimes \Theta, \gr\omega x : A}
      \infer2[$\oc$-E]{\Gamma \vdash
        \text{let }[x] = M\text{ in }N : C \boxtimes \Theta}
    }
  \end{mathpar}
  \caption{Typing with leftovers, a possible way to capture $\oc$}
  \label{fig:twl-exp}
\end{figure}

However, this adaptation of the DILL style does not obviously generalise to
semiring annotations.
Even for the multiplicative fragment, we seem to be working against the
direction of addition, instead using a subtraction operation whenever we use a
variable.
An algebraic presentation of this fragment is given by \citet{ZD21}, in terms of
a partial functional cancellative addition operation.
For exponentials, though, and particularly the $\oc$-introduction rule, what I
have done seems ad-hoc, not based on any pointwise algebraic operation.

Also, the unusual form of sequents can cause some problems when working with a
typing with leftovers system.
For example, a traditional intuitionistic linear logic sequent
$\Gamma \vdash M : A$ corresponds to many different typing with leftovers
sequents, including:
\begin{itemize}
  \item $\gr1\Gamma \vdash M : A \boxtimes \gr0\Gamma$
  \item $\gr1\Gamma, \gr1x : B \vdash M : A \boxtimes \gr0\Gamma, \gr1x : B$
  \item $\gr1\Gamma, \gr0x : B \vdash M : A \boxtimes \gr0\Gamma, \gr0x : B$
\end{itemize}

Generally, any variable not used in the term can be added to both the input and
the output context with the same annotation.
Many of these variations are likely to appear in various typing derivations,
depending on what terms surround a given subterm.
In particular, many different variations can appear for the same term in
different subterms of the same derivation, even when the scope of the two
occurrences is the same.
This means that if we want to implement substitution, which involves putting a
term into an unknown surrounding, we have to navigate these different forms via
the \emph{framing property}.

The unusual form of sequents also somewhat obscures any attempt to interpret the
terms of a typing-with-leftovers system.
Though the $\boxtimes$ notation suggests a semantics into symmetric monoidal
closed categories where terms are morphisms from one iterated tensor product
(the input context) to another (the type and output context), the syntax is
incomplete for this semantics because we cannot produce anything interesting in
the output context.

Another piece of related work using a typing-with-leftovers style is that of
\citet{polakow15}.
There, \citeauthor{polakow15} encodes a linear embedded domain-specific language
inside Haskell using typeclass constraints.

\subsection{Yalla}

\Citet{laurent18} presents Yalla: Coq development of a collection of linear
logics in a uniform style, and various proofs relating them.
The logics share varying amounts of definitions and theorems --- for example,
the main linear logic is parametrised on whether to include mix rules and
whether to restrict exchange to cyclic permutations, whereas systems like the
Lambek calculus (with no exchange) and polarised linear logic are defined
separately from scratch.

The style used in Yalla is to realise sequents as lists of formulas.
The active formula tends to be forced to be the first formula in such a list,
with an explicit exchange rule being used to move such formulas into place.
\Citet{laurent18} points out that using multisets, as do some less formal
presentations of linear logic, is insufficient at distinguishing certain proofs
involving repeated assumptions or conclusions.
For example, we expect there to be two distinct derivations of
$A \otimes A \vdash A \otimes A$ (up to the appropriate equational theory):
the one that keeps the pair in the same order and the one that flips the order.
But if the \TirName{$\otimes$-L} rule unpacks the formula $A \otimes A$ into the
\emph{multiset} $\{A, A\}$, then we forget the order of the input pair.

Despite making sure to distinguish distinct proofs, the Yalla library does not
define any equational theories of proofs, so does not prove any results relying
on these distinctions.
While the complication of defining an equational theory in the presence of an
exchange rule is probably largely inevitable, having the exchange rule
introduces redundancy such that many equivalent proofs are not equal as data
structures.
The mechanisations presented in \cref{sec:simple} all sought to reduce this kind
of redundancy for intuitionistic logic, so that the only non-trivial equations
in the equational theory are $\beta$- and $\eta$-rules (i.e.\ computationally
interesting rules).
I will restore this property of the representation in \cref{sec:semirings}.

The relevance of Yalla to the work in this thesis is limited by the fact that
Yalla is based entirely on sequent calculi, whereas I am considering only
natural deduction calculi.

\subsection{Co-De-Bruijn syntax}

\Citet{McBride18} presents a mechanisation of the simply typed
$\lambda$-calculus in what he calls \emph{co-De-Bruijn} style.
He notes that syntax based on De Bruijn indices, as presented in
\cref{sec:simple}, finds a canonical way to place contractions and weakenings
by eagerly placing contractions wherever they could be needed (i.e., whenever a
rule has multiple premises) and leaving weakening as late as possible (i.e., at
the variable rule and at rules with no premises).
Co-De-Bruijn syntax, by contrast, finds a canonical way to place contractions
and weakenings by doing the reverse: weakening as early as possible (i.e., as
soon as the variable is bound) and contracting only where necessary.

Such a scheme straightforwardly adapts to multiplicative linear logic by
modifying the data structures presented by \citet{McBride18} to disallow all
contraction and weakening, as presented by \citet{RPKV20}.
With the additive rules, however, we get cases where variables appear multiple
times syntactically in a term but are still considered linear by the type
system, the simplest example being
$x : A \vdash \langle x, x\rangle : A \with A$.
Such rules are perhaps a new kind compared to what \citeauthor{McBride18}
considers, but it seems likely that just copying the same context to all the
premises would not be too hard to accommodate.
Meanwhile, we may consider implementing the $\oc$-modality as in DILL, with
intuitionistic variables handled using the regular co-De-Bruijn machinery from
the paper.
In summary, the co-De-Bruijn approach is promising for capturing linearity, but
has not been thoroughly investigated.

The other relevant contribution from \citet{RPKV20} is to recast the
context-splitting relations via connectives inspired by bunched
logic~\citep{oHP99}.
I adapt this method in \cref{sec:lnd}, but with a different notion of
context-splitting.

\subsection{Fitch-style modalities}\label{sec:fitch}

An alternative to the approach of \citet{judgmental} to adapt modal logics (and
particularly IS4) to natural deduction is using Fitch-style calculi.
Fitch-style calculi, as codified and studied by \citet{Borghuis-thesis}, are
distinguished by allowing for contexts containing \emph{locks}, written $\Lock$,
with the variable rule being restricted so that only variables not behind locks
are immediately accessible.

Below I give the main rules of Fitch-style IK (intuitionistic logic K).
Other normal modal logics are obtained by strengthening the $\Box$-elimination
rule to remove varying numbers of locks.
For example, we can add axiom \TirName{T} by allowing for 0 or 1 locks to be
removed, axiom \TirName{4} by allowing 1 or more locks to be removed, or both
axioms together by allowing any number of locks to be removed from the
right-hand end of the context (together with any variables to the right of a
removed lock).
The \TirName{$\Box$-I} rule stays the same, and forms part of an adjunction of
the form ``$\Lock \dashv \Box$''.

\begin{mathpar}
  \ebrule{%
    \hypo{\Lock \notin \Gamma'}
    \infer1[Var]{\Gamma, A, \Gamma' \vdash A}
  }
  \and
  \ebrule{%
    \hypo{\Gamma, \Lock \vdash A}
    \infer1[$\Box$-I]{\Gamma \vdash \Box A}
  }
  \and
  \ebrule{%
    \hypo{\Lock \notin \Gamma'}
    \hypo{\Gamma \vdash \Box A}
    \infer2[$\Box$-E$_K$]{\Gamma, \Lock, \Gamma' \vdash A}
  }
\end{mathpar}

A syntactic advantage of Fitch-style calculi over the calculus introduced by
\citet{judgmental} is that Fitch-style calculi support a projection-style
eliminator for $\Box$, which makes it easier to use than the pattern-matching
eliminator of \citeauthor{judgmental}.
A disadvantage is that $\Lock$ cannot be understood as a kind of hypothetical
judgement like the rest of the context, so many of the heuristics we relied on
in \cref{sec:simple} and \cref{sec:modal} fail.
In fact, the addition of locks represents a large change to the judgemental
structure of the calculus, apparently requiring a complete overhaul of the basic
metatheory.

\Citet{VRC22} have completed a significant mechanisation of the metatheory of
a Fitch-style calculus in Agda.
This work shows that, despite the change in the structure of the metatheory,
Fitch-style calculi are amenable to mechanised proofs.

I am not aware of any work discussing linear Fitch-style calculi.

\subsection{Systematisations of substructural logics}\label{sec:lin-sys}

Several pieces of prior work have aimed to give general accounts of a range of
substructural calculi, in a similar vein to existing accounts of aspects of
non-substructural calculi.
I review some of these, particularly as a way to give a comparison to the
adaptation of the methods of \cref{sec:simple} that I spend the rest of this
thesis on.

\paragraph{Linear Logical Framework}
In \cref{sec:lf}, I discussed logical frameworks based on the
$\lambda^\Pi$-calculus, and their use in the mechanisation of non-substructural
programming languages.
\Citet{CP02} extend $\lambda^\Pi$ to a calculus
$\lambda^{\Pi\multimap\with\top}$, and use that to create a logical framework
supporting linear logics: the Linear Logical Framework (LLF).
The $\Pi$ type former still forms intuitionistic weak dependent function spaces,
while the new $\multimap$ forms linear weak non-dependent function spaces.
They handle the distinction between intuitionistic and linear variables thus
introduced in the same style as DILL, with the argument of an intuitionistic
application having the same intuitionistic restriction as we saw in DILL's
\TirName{$\oc$-I} rule.

The additive connectives $\with$ and $\top$ provide a way to state rules whose
premises respectively share and arbitrarily consume linear variables, like in
the rules for additive connectives in linear logic.
I will revisit this method of stating typing rules in terms of \emph{sharing}
and \emph{separating} (as given by right-nested sequences of $\multimap$s)
conjunctions of premises in \cref{sec:lnd}, though in \cref{sec:bunched-logic}
I argue that there is a closer connection to bunched logics than linear logics.

The main focus of the original paper is to represent mutation in an ML-like
language via state updates mediated by $\multimap$, though they also mention
having mechanised some metatheory of linear calculi.
Some example programs are currently available at
\url{https://www.cs.cmu.edu/~iliano/projects/LLF/index.html}.

\paragraph{Encoding linearity in LF}
\Citet{crary10} gives a method of encoding linearity constraints in a
conventional, non-substructural, logical framework.
He implements this approach in the LF-based proof assistant Twelf~\citep{Twelf}.
He uses a predicate \texttt{linear : (term -> term) -> type}, where
\texttt{term} is a type of preterms, and thus \texttt{term -> term} is (thanks
to the weak function space of LF) the type of preterms with one free variable.
The predicate \texttt{linear} then says that that free variable is used linearly
in its term, which is defined inductively on the structure of preterms.
The \texttt{linear} predicate is used by the typing relation wherever variables
are bound.
The development handles all of intuitionistic linear logic, with the
$\oc$-modality treated with a DILL-style distinction between linear and
intuitionistic variables.
Intuitionistic variables are implemented simply by not checking for linearity in
the \TirName{$\oc$-E} rule.
\Citet[\S 4]{crary10} also shows how to adapt this technique to a PD-style
presentation of IS4.

As an example, let us look at the typing and linearity rules for the binary
tensor product.
Typing is given by a relation \texttt{of : term -> tp -> type}, where
\texttt{tp} is the type of object-level types.
Each syntactic form has a typing rule and potentially several linearity rules,
understood disjunctively as logic programming clauses.
The rules for the introduction form are listed below.
The typing relation is just like it would be for pairs in the simply typed
$\lambda$-calculus: $(M, N) : A \otimes B$ if $M : A$ and $N : B$.
The linearity rules are symmetric, so I will just consider
\texttt{linear/tpair1}.
It says that $x \vdash (M[x], N)$ is linear if $x \vdash M[x]$ is linear.
This rule is subtle in that not applying \texttt{x} to \texttt{N} implies that
\texttt{x} is fresh (and therefore unused) in \texttt{N}.
Where $\otimes$-pairs have two linearity rules, the $I$-unit, and also the
introduction form for $\oc$, have no linearity rules, meaning that no linear
variables can be used in or by them.

\begin{verbatim}
of/tpair : of (tpair M N) (tensor A B) <- of M A <- of N B.
linear/tpair1 : linear ([x] tpair (M x) N) <- linear ([x] M x).
linear/tpair2 : linear ([x] tpair M (N x)) <- linear ([x] N x).
\end{verbatim}

Meanwhile, the rules for the eliminator are somewhat more involved, thanks to
the bound variables.
First, the typing rule shows how \texttt{of} relies on \texttt{linear}, checking
each bound variable for linearity.
Because we have two bound variables, we need to check that the term \texttt{N}
is linear in both.
We do this by checking that, for all \texttt{y}, \texttt{N} is linear in
\texttt{x}, and that for all \texttt{x}, \texttt{N} is linear in \texttt{y}.
The linearity rules have the same choice and careful management of free
variables as they did for the introduction form.
In addition, in the rule \texttt{linear/lett2}, the bound variables in
\texttt{N} have to be universally quantified while we check for linearity in the
free variable \texttt{z}.

\begin{verbatim}
of/lett : of (lett M ([x] [y] N x y)) C
  <- of M (tensor A B) <- ({x} of x A -> {y} of y B -> of (N x y) C)
  <- ({y} linear ([x] N x y)) <- ({x} linear ([y] N x y))
linear/lett1 : linear ([z] lett (M z) ([x] [y] N x y))
  <- linear ([z] M z)
linear/lett2 : linear ([z] lett M ([x] [y] N z x y))
  <- ({x} {y} linear ([z] N z x y))
\end{verbatim}

\Citet{crary10} goes on to extend this encoding with intuitionistic dependent
$\Pi$-types, producing an object theory comparable to the
$\lambda^{\Pi\multimap\with\top}$ metatheory developed by \citet{CP02}.
If one wants to mix linearity and dependency following the methodology of
\citet{Atkey18}, then it is crucial that linear variables are still free in
subterms from which they have been discarded.
At first sight, it appears that \citeauthor{crary10}'s encoding violates this
because of its use of ``does not appear free'' to mean ``is not used'' in many
linearity rules.
However, one could imagine introducing an \texttt{unused} predicate similar to
\texttt{linear} in order to handle unused free variables, at the cost of a few
extra rules and a somewhat heavier encoding (scaling with the \emph{size} of the
term, rather than the depth).
Indeed, one could imagine parametrising the \texttt{linear} predicate so as to
encode the semiring-annotated systems I discuss in \cref{sec:semirings}.

The approach of \citet{crary10} removes the objection to the work of
\citet{CP02} that each new substructural discipline would need a new proof
assistant by encoding linearity in a standard intuitionistic logical framework.
However, the encoding makes linear variables second-class compared to
intuitionistic variables.
While intuitionistic variables are just there thanks to the metatheory, linear
variables must essentially be explicitly quantified.

\paragraph{Licata-Shulman-Riley}
\Citet{LicataSR17} describe a framework for specifying and working with a wide
range of substructural logics.
I discuss exactly what this range is in \cref{sec:semirings-conc}, in relation
to the calculi I describe in the rest of this thesis.
For now, it suffices to say that this framework is specified in enough detail
that it should be possible to mechanise it directly, but I am not aware of
anyone having done so.
\Citet{Restall1999} describes a similar approach.

\paragraph{Tanaka-Power}
The work of \citet{FPT99}, which I discussed in \cref{sec:fiore}, has been
extended to substructural syntaxes by \citet{TP06}.
This work gives a mechanism for turning a description of contexts and their
structural rules (expressed as a pseudo-monad on the 2-category of categories)
into a framework for defining substructural syntaxes, and more generally a
category of algebras of which the syntax is the initial object.
As examples, they give the untyped $\lambda$-calculus, an untyped multiplicative
linear logic, and a bunched logic.
These examples show a broad range of substructural disciplines they support ---
comparable to the work of \citet{LicataSR17} (which I discuss in
\cref{sec:semirings-conc}), and more than I discuss in this thesis.
However, they also show two of the limitations of their approach.
Firstly, this work provides no way to track types, though it should be possible
to incorporate types at the expense of complicating the categorical
constructions they use.
Secondly, it appears to be impossible to encode the syntactic forms used for the
additive connectives (i.e., the Cartesian product and coproduct) of linear
logic.
Essentially, subterms can only be combined into a larger term in the same ways
as how contexts can be appended together.
For example, in a bunched logic, contexts can be combined through both sharing
conjunction (the Cartesian product) and separating conjunction
(a monoidal product).
Correspondingly, the syntax descriptions of \citeauthor{TP06} allow for the
syntax of sharing pairs and separating pairs.
However, in the case of linear logic, contexts can only be combined via a
monoidal product, so we only get separating pairs (tensor-products) and not
sharing pairs (with-products).

\paragraph{Semiring-annotated systems}
There have been several appearances in the literature of calculi in which
variables are annotated with some algebraic usage information which is reflected
in the types of the calculus.
Such calculi appear in the work of \citet{abadi99core} and
\citet{reed10distance}, where annotations are used to control information flow
and sensitivity to perturbations, respectively.
Following these disparate calculi, the work of \citet{BrunelGMZ14},
\citet{GhicaS14}, and \citet{POM14} sought to unify these calculi using
(variations on) partially ordered semirings as the source of annotations.
Each of these papers also provided further examples of calculi with semiring
usage annotations.
I discuss calculi based on partially ordered semiring annotations further in the
following chapter, as it provides the basis for the rest of this thesis.
