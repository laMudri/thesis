Dual intuitionistic linear logic is a syntax for intuitionistic linear logic
introduced by \citet{Barber1996}.
Its key feature is sequents with \emph{dual contexts} --- splitting assumptions
into \emph{linear} assumptions and \emph{intuitionistic} assumptions.
The idea of the intuitionistic assumptions is that they will behave like the
variables of simply typed $\lambda$-calculus, in contrast to the linear
assumptions, which behave as in the linear calculus we saw in \autoref{sec:ill}.
For example, an intuitionistic assumption of $A$ in an instance of the
$\otimes$-introduction rule is automatically copied to both premises.
In contrast, an assumption of $\oc A$ in the purely linear calculus must first
be contracted into two assumptions, with one going to each premise.

The high-level purpose of intuitionistic assumptions is to universally
characterise the connective $\oc$.
That is to say, the existence of intuitionistic assumptions, differentiated from
linear assumptions, gives rise to $\oc$ in the same way that the existence of
multiple (linear) assumptions gives rise to $\otimes$.
We informally express the latter as ``$\otimes$ is the internalisation of
context concatenation (comma)'', and we can similarly express the former as
``$\oc$ is the internalisation of intuitionistic assumptions''.

The rules witnessing the characterisations of $\otimes$ and $\oc$ are listed
below.
In particular, these are elimination rules, as found in the intro-elim system
given by \citeauthor{Barber1996}.
I take $\uplus$ to be a partial function, with $\Gamma_0 \uplus \Gamma_1$
defined only when $\Gamma_0$ and $\Gamma_1$ have the same intuitionistic
entries (these being the same intuitionistic entries as in
$\Gamma_0 \uplus \Gamma_1$).
Linear assumptions are accumulated by $\uplus$.
The use of $\uplus$ in these two rules separates the eliminand from the
eliminator/continuation, rather than having anything particular to do with
the connectives $\otimes$ and $\oc$.

\[
  \begin{matrix}
    \begin{prooftree}
      \hypo{\Gamma_0 \vdash A \otimes B}
      \hypo{\Gamma_1, \mathbf{lin}\,A, \mathbf{lin}\,B \vdash C}
      \infer2[$\otimes$-E]{\Gamma_0 \uplus \Gamma_1 \vdash C}
    \end{prooftree}
    &&
    \begin{prooftree}
      \hypo{\Gamma_0 \vdash \oc A}
      \hypo{\Gamma_1, \mathbf{int}\,A \vdash C}
      \infer2[$\oc$-E]{\Gamma_0 \uplus \Gamma_1 \vdash C}
    \end{prooftree}
  \end{matrix}
\]

The introduction rule for $\oc$ follows the pattern of Promotion, allowing us
to conclude $\oc A$ from $A$ only when that proof of $A$ does not use linear
assumptions.
Now that the distinction between linear and intuitionistic assumptions is part
of the structure of the \emph{sequent}, we do not check the \emph{types} of any
assumptions in this rule.
I write $\mathbf{int}\,\Gamma$, where $\Gamma = A_1, \ldots, A_n$, for the
context $\mathbf{int}\,A_1, \ldots, \mathbf{int}\,A_n$.

\[
  \begin{prooftree}
    \hypo{\mathbf{int}\,\Gamma \vdash A}
    \infer1[$\oc$-I]{\mathbf{int}\,\Gamma \vdash \oc A}
  \end{prooftree}
\]
