Dual Intuitionistic Linear Logic (DILL) is a syntax for intuitionistic linear
logic introduced by \citet{Barber1996}.
Its key feature is splitting assumptions into \emph{linear} assumptions and
\emph{intuitionistic} assumptions --- sometimes called the \emph{dual context}
approach.
Intuitionistic assumptions behave like the variables of simply typed
$\lambda$-calculus.
In contrast, the linear assumptions behave as in the linear calculus we saw in
\cref{sec:ill}.
For example, an intuitionistic assumption of $A$ in an instance of the
$\otimes$-introduction rule is automatically copied to both premises.
This contrasts with an assumption of $\oc A$ in the purely linear calculus,
which must first
be contracted into two assumptions, with one going to each premise.
Compared to the modal system we saw in \cref{sec:modal}, linear assumptions
correspond to true assumptions, and intuitionistic assumptions correspond to
valid assumptions.

The new feature when dealing with linear logic, compared to modal logic, is that
linear and intuitionistic assumptions satisfy different structural rules.
In our de Bruijn index style, where every multi-premise rule does the maximal
contraction and weakening occurs in the leafwardmost possible position,
this means that all of the rules have to manage the substructurality of linear
assumptions.
To help manage linear assumptions, I introduce three pieces of notation.

\begin{definition}\label{def:DILL-empty}
  I write $\Gamma~\mathit{int}$ to state that context $\Gamma$ contains only
  intuitionistic assumptions, i.e., no linear assumptions.
\end{definition}

\begin{definition}\label{def:DILL-split}
  I write $\Gamma + \Delta$ to stand for a context combining $\Gamma$ and
  $\Delta$.
  Specifically, the operation is only defined when $\Gamma$ and $\Delta$ contain
  the same intuitionistic assumptions, and in that case the result contains the
  same intuitionistic assumptions.
  The linear assumptions of the result are the disjoint union of the linear
  assumptions of each of $\Gamma$ and $\Delta$.
\end{definition}

\begin{definition}\label{def:DILL-var}
  I write $\Gamma \ni A$, like in \cref{sec:modal}, to stand for the set of
  variables of type $A$ we can get from $\Gamma$.
  Specifically, if $\Gamma$ contains an assumption $x : A~\mathit{int}$, then
  $x$ gives rise to an inhabitant of $\Gamma \ni A$ if and only if
  $\Gamma~\mathit{int}$; and if $\Gamma$ contains an assumption
  $x : A~\mathit{lin}$, then $x$ gives rise to an inhabitant of $\Gamma \ni A$
  if and only if $x$ is the only linear assumption in $\Gamma$.
  These restrictions essentially encode the fact that we do not allow linear
  variables to be discarded via the use of another variable.
\end{definition}

\Citet{Barber1996} uses the notational convenience of dividing linear and
intuitionistic assumptions into two separate contexts, hence the name of the
\emph{dual context} approach.
I choose not to do this so as to draw out the connection with the approach of
\citet{judgmental}, as well as the approach I will take in later chapters.
I also think that it is more instructive to treat the context as a single
unified thing, so that we can clearly see what operations on the context have
to be preserved by environments (for use in renaming, substitution, and other
traversals).

I list the complete rules of DILL in \cref{fig:dill}.
Given \cref{def:DILL-empty,def:DILL-split,def:DILL-var}, the rules for
multiplicative and additive connectives look not too dissimilar to those in
\cref{fig:mall}.
However, their treatment of intuitionistic assumptions (contraction and
weakening automatically, as in non-substructural calculi) is new.
Meanwhile, the rules for the $\oc$-modality are the same as those for
the $\Box$-modality given in \cref{fig:PD}, except for the implicit structural
rules applying to linear assumptions.
For example, where the \TirName{$\Box$-I} rule discards true assumptions, the
\TirName{$\oc$-I} rule cannot discard linear assumptions, but instead requires
that there are no linear assumptions in the context when the rule is applied, so
that no linear assumptions can be used in the subproof.

\begin{figure}
  \begin{align*}
    A, B, C &\Coloneqq X \mid I \mid A \otimes B \mid A \multimap B
              \mid 0 \mid A \oplus B \mid \top \mid A \with B \mid \oc A \\
    \Gamma, \Delta, \Theta &\Coloneqq {\cdot} \mid \Gamma, A~\mathit{lin}
                             \mid \Gamma, A~\mathit{int} \\
    \mathcal S &\Coloneqq \Gamma \vdash A~\mathit{lin}
  \end{align*}
  \begin{mathpar}
    \ebrule{%
      \hypo{\Gamma \ni A}
      \infer1[Var]{\Gamma \vdash A~\mathit{lin}}
    }
    \and
    \ebrule{%
      \hypo{\Gamma \vdash I~\mathit{lin}}
      \hypo{\Delta \vdash C~\mathit{lin}}
      \infer2[$I$-E]{\Gamma + \Delta \vdash C~\mathit{lin}}
    }
    \and
    \ebrule{%
      \hypo{\Gamma~\mathit{int}}
      \infer1[$I$-I]{\Gamma \vdash I~\mathit{lin}}
    }
    \and
    \ebrule{%
      \hypo{\Gamma \vdash A \otimes B~\mathit{lin}}
      \hypo{\Delta, A~\mathit{lin}, B~\mathit{lin} \vdash C~\mathit{lin}}
      \infer2[$I$-E]{\Gamma + \Delta \vdash C~\mathit{lin}}
    }
    \and
    \ebrule{%
      \hypo{\Gamma \vdash A~\mathit{int}}
      \hypo{\Delta \vdash B~\mathit{int}}
      \infer2[$\otimes$-I]{\Gamma + \Delta \vdash A \otimes B~\mathit{lin}}
    }
    \and
    \ebrule{%
      \hypo{\Gamma \vdash A \multimap B~\mathit{int}}
      \hypo{\Delta \vdash A~\mathit{int}}
      \infer2[$\multimap$-E]{\Gamma + \Delta \vdash B~\mathit{lin}}
    }
    \and
    \ebrule{%
      \hypo{\Gamma, A~\mathit{lin} \vdash B~\mathit{int}}
      \infer1[$\multimap$-I]{\Gamma \vdash A \multimap B~\mathit{lin}}
    }
    \and
    \ebrule{%
      \hypo{\Gamma \vdash 0~\mathit{lin}}
      \infer1[$0$-E]{\Gamma + \Delta \vdash C~\mathit{lin}}
    }
    \and
    \text{(no \TirName{$0$-I})}
    \and
    \ebrule{%
      \hypo{\Gamma \vdash A \oplus B~\mathit{lin}}
      \hypo{\Delta, A~\mathit{lin} \vdash C~\mathit{lin}}
      \hypo{\Delta, B~\mathit{lin} \vdash C~\mathit{lin}}
      \infer3[$\oplus$-E]{\Gamma + \Delta \vdash C~\mathit{lin}}
    }
    \and
    \ebrule{%
      \hypo{\Gamma \vdash A_i~\mathit{lin}}
      \infer1[$\oplus$-I$_i$]{\Gamma \vdash A_0 \oplus A_1~\mathit{lin}}
    }
    \and
    \text{(no \TirName{$\top$-E})}
    \and
    \ebrule{%
      \infer0[$\top$-I]{\Gamma \vdash \top~\mathit{lin}}
    }
    \and
    \ebrule{%
      \hypo{\Gamma \vdash A_0 \with A_1~\mathit{lin}}
      \infer1[$\with$-E$_i$]{\Gamma \vdash A_i~\mathit{lin}}
    }
    \and
    \ebrule{%
      \hypo{\Gamma \vdash A~\mathit{lin}}
      \hypo{\Gamma \vdash B~\mathit{lin}}
      \infer2[$\with$-I]{\Gamma \vdash A \with B~\mathit{lin}}
    }
    \and
    \ebrule{%
      \hypo{\Gamma \vdash \oc A~\mathit{lin}}
      \hypo{\Delta, A~\mathit{int} \vdash \oc C~\mathit{lin}}
      \infer2[$\oc$-E]{\Gamma + \Delta \vdash C~\mathit{lin}}
    }
    \and
    \ebrule{%
      \hypo{\Gamma \vdash A~\mathit{lin}}
      \hypo{\Gamma~\mathit{int}}
      \infer2[$\oc$-I]{\Gamma \vdash \oc A~\mathit{lin}}
    }
  \end{mathpar}
  \caption{Dual Intuitionistic Linear Logic}
  \label{fig:dill}
\end{figure}

I will not give a direct substitution procedure for DILL like the one I gave for
the modal system of \cref{sec:modal}.
Suffice to say, where environments for the modal calculus had to be preserved by
binding of variables and pruning of all true variables, environments for DILL
have to be preserved by similar plus the operations of
\cref{def:DILL-empty,def:DILL-split}.
These definitions themselves are somewhat technical, and I do not believe that
delving into the further technicalities of how they interact with environments
will provide useful enough intuitions to justify including here.
However, I will revisit the idea of environments for linear calculi in a more
general semiring-annotated setting in \cref{sec:ren-sub-lr}.

Finally, notice that the rules given in \cref{fig:dill} characterise $\oc$ up to
logical equivalence.
The main feature ensuring this property is that each logical rule contains
exactly one occurrence of a logical connective, meaning that each logical rule
is only defining that connective in that place.
For the full characterisation result, we also require local soundness and
completeness of $\oc$, as discussed by \citet{judgmental} in the modal setting.
