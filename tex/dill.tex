Dual intuitionistic linear logic is a syntax for intuitionistic linear logic
introduced by \citep{Barber1997}.
Its key feature is sequents with \emph{dual contexts} --- splitting assumptions
into \emph{linear} assumptions and \emph{intuitionistic} assumptions.
The idea of the intuitionistic assumptions is that they will behave like the
variables of simply typed $\lambda$-calculus, in contrast to the linear
assumptions, which behave as in the linear calculus we saw in \autoref{sec:ill}.
For example, an intuitionistic assumption of $A$ in an instance of the
$\otimes$-introduction rule is automatically copied to both premises.
In contrast, an assumption of $\oc A$ in the purely linear calculus must first
be contracted into two assumptions, with one going to each premise.

The high-level purpose of intuitionistic assumptions is to universally
characterise the connective $\oc$.
That is to say, the existence of intuitionistic assumptions, differentiated from
linear assumptions, gives rise to $\oc$ in the same way that the existence of
multiple (linear) assumptions gives rise to $\otimes$.
We informally express the latter as ``$\otimes$ is the internalisation of
context concatenation (comma)'', and we can similarly express the former as
``$\oc$ is the internalisation of intuitionistic assumptions''.

%\begin{align*}
%  \begin{proof}
%    \hypo{\Gamma_0 \vdash \oc A}
%    \hypo{\Gamma_1, \mathbf{int}~A \vdash C}
%    \infer2{\Gamma_0 \cupdot \Gamma_1 \vdash C}
%  \end{proof}
%  &
%  \begin{proof}
%    \hypo{\Gamma_0 \vdash A \otimes B}
%    \hypo{\Gamma_1, \mathbf{lin}~A, \mathbf{lin}~B \vdash C}
%    \infer2{\Gamma_0 \cupdot \Gamma_1 \vdash C}
%  \end{proof}
%\end{align*}
