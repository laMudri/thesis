\chapter{Conclusions}\label{sec:conc}

In this thesis, I have developed a foundation for semiring-annotated calculi
presented in natural deduction style.
I have given a consolidated account of the semiring-annotated calculus $\name$,
including its relations to existing linear and modal calculi.
As part of this, I have adapted what I call \emph{bunched connectives} from
\cite{RPKV20} as a way to state the typing rules of the calculus as well as to
work with the metatheory.
The distinction between sharing and separating conjunction given by the bunched
connectives corresponds well to the notions of \emph{additive} and
\emph{multiplicative} connectives in the object language, respectively.
Following this, I have given a novel linear algebra-based definition of
\emph{environment} for semiring-annotated calculi, together with a motivation
which may serve as a basis for the corresponding definition for other
substructural systems.
The adequacy of this definition of environment is shown by my implementation of
simultaneous renaming and substitution, as well as other operations on
environments, like composition of renamings and substitutions.

With the details of $\name$ worked out, I then moved on to adapting the work of
\citet{AACMM21} so as to make it able to capture semiring-based usage
restrictions, as found in $\name$.
The syntax descriptions of the resulting system are based on the bunched
connectives, and are shown to be expressive enough to encode calculi of a
variety of forms.
In adapting the semantics, I am forced to be precise about \emph{sharing} and
\emph{separating} bunched connectives, but largely add these as a refinement of
the work of \citet{AACMM21}.
I provide the renaming and substitution operations for all expressible calculi.
I also provide more specialised examples of semantic traversals:
a usage elaborator, an NbE algorithm, a denotational semantics, and translations
between different calculi.
The usage elaborator gives an unexpected example of generic programming, which
one could not write without syntax descriptions.
% Other examples...
Together, these examples show the applicability and versatility of the framework
I have developed.

FIXME: the next two paragraphs are directly from the ESOP22 paper.

We have presented a framework for doing metatheory for a class of substructural
type systems in Agda.
The framework gives us renaming, substitution, and a usage elaborator for new
syntaxes for free, which we hope can facilitate prototyping and the
mechanisation of more interesting semantic results.
Beside the mechanised framework itself, we believe its methodology --- the use
of bunched premise combinators --- can guide and simplify the development of
(potentially unmechanised) substructural type systems.

Our account of substructurality is based on the linear algebraic
principles described by \citet{WA21}.
However, these details only really affect the definition of environment,
in which the use of linear maps is motivated by them being the standard notion
of morphism between vectors.
We could imagine that a similar notion of morphism is found for the kind of
annotations found in \citet{LicataSR17}, allowing a framework to consider
finer substructural systems.

\begin{itemize}
  \item Novel simultaneous substitution for linear type theories
  \item Generalise \citet{AACMM21} work to substructural logics
  \item Is this a reusable library? Yes, examples
  \item Methodology for metatheory of new calculi
  \item Semirings have been a reasonably robust basis.
  \item Can the code be directly generalised?
    Abstract away the algebraic structure of contexts.
\end{itemize}

\section{Future work}

\begin{itemize}
  \item Fitch-style systems (can't do).
\end{itemize}

\subsection{Further work}

\paragraph{Equality}
Perhaps the most fundamental missing piece from the metatheoretic account of
semiring-annotated calculi I have given in this thesis is equations between
terms.
Reasoning about equality between terms and environments is a problem I have
tried to solve, but I have not arrived at a satisfactory solution in the time
available to me.

I believe that the basic difficulty of giving an account for equality in a
linear setting is the proliferation of $\Sigma$-types.
For example, describing equality between two applications of
\TirName{$\with$-I} is immediate: $\Gamma \vdash (M, N) = (M', N') : A \with B$
if and only if $\Gamma \vdash M = M' : A$ and $\Gamma \vdash N = N' : B$.
However, to do the same with \TirName{$\otimes$-I} requires us to be careful
about the contexts of the subterms.
The two applications of \TirName{$\otimes$-I} may a priori split the context in
different ways, and should only be equated when those splittings are equal
(in the appropriate sense).
If the splittings are equal, then the contexts of the subterms will line up, and
only then can the subterms themselves be compared for equality.
These multiple stages come about because $\plr{T * U}\,\Gamma$ is a
$\Sigma$-type, and equality of elements of $\Sigma$-types always follows this
pattern.
Such reasoning becomes even more complex for environments, which are equivalent
to iterated $*$-families.
Additionally, it is unclear what effect subsumption of contexts (like subusaging
in this thesis or explicit structural rules in other calculi) should have on
equality, particularly when the subsumption commutes with parts of the subterms.

Previous work on linear logic has preferred to deal with equations only on
untyped terms. \todo{Check this.}

\paragraph{Polymorphism}
An important feature of most contemporary statically typed programming languages
is polymorphism.
In particular, parametric polymorphism over types can be used to significantly
improve code reuse, and is well undestood theoretically via System $F$ and its
variants.
I have not considered polymorphism in this thesis, and neither did
\citet{AACMM21} in their paper, so whether it can be supported in the framework
presented earlier is an open question.
However, \citet{Autosubst15} did apply their related system to System $F_{<:}$,
so it may be possible to support polymorphic systems using their techniques.

A separate but related question concerns polymorphism over usage annotations.
The status of polymorphism over usage annotations is less well established both
in practice and in theory.
\Citet{Granule18} present an implementation allowing for polymorphism of usage
annotations, and even polymorphism over semirings, but provide no more than
example programs to justify the feature.
This thesis provides no advance on understanding annotation polymorphism, unless
it can be encoded into a semiring to fit the framework.

%\subsection{Further generalisations}

\paragraph{Structure of contexts}
As I have presented it, the work of \citet{AACMM21} has two axes in which it is
generic: the syntax, which can be controlled through syntax descriptions to
produce a wide range of calculi and features; and the semantics, where we can
produce a wide range of maps out of terms with the help of environments.
To this, the work of this thesis has added a third axis of genericity: the usage
discipline of variables, as described by a partially ordered semiring.

Starting at least with the bunched connectives in \cref{sec:lnd}, if not earlier
when talking about usage contexts forming modules over the semiring of
annotations, I have made productive use of abstractions over the basic usage
annotations throughout this thesis.
These abstractions suggest a next step of completely abstracting away much of
the representation of contexts and their individual entries.
One may imagine that it is possible to develop a framework in which the required
operations and properties of contexts are axiomatised, similar to how usage
annotations are axiomatised to form a semiring in this thesis, and to how
categories-with-families models are defined~\citep{Dybjer95,CCD19}.
Instances of such a framework would include the work of \citet{RPKV20}, which
uses a very similar bunched connective abstraction over a very different
representation of contexts, based on relational interleaving of lists.

The use of semirings is motivated in this thesis and elsewhere largely because
they are general enough to cover a wide range of examples.
However, I cannot claim to have a derivation from first principles of why we
should choose partially ordered semirings over any of a range of similar
algebraic structures.
Additionally, some of the specific constructions done in this thesis fit
somewhat unnaturally with the semiring-based approach.
For example, when translating semiring-annotated systems to traditional systems,
I tended to need to make a \emph{bottom-up} assumption
(\cref{def:DILL-bottom-up,def:PD-bottom-up}) so as to avoid some ``junk'' facts
given by the semiring.
Meanwhile, the usage elaborator of \cref{sec:usage-elaborator} eschews the
``forward'' computation of semiring operations in favour of non-deterministic
backwards computation, e.g., from a sum to the collection of possible summands.
Possibly consciously working more abstractly, as described in the previous
paragraph, would make a more natural structure appear.

If we are to retain an annotation-based approach to usage restrictions, then a
possible feature request that falls out of the encoding of linear/non-linear
logic is to have some sort of kinding system by which different kinds of types
are annotated using different sets of annotations.
In the L/nL example, we would want linear types to be annotated with $\gr0$ and
$\gr1$, and intuitionistic types to be annotated with $\gr\omega$ (as the sole
element of a trivial instance of an algebraic structure), with no crossover
between the two kinds.
Algebraic means to handle such mixed-kind usage vectors may be inspired by the
work of \citet{Hart95,MF21} on dimensional analysis in linear algebra.

\subsubsection{Partiality}\label{sec:part}
As we have seen, the way additive and multiplicative rules are
realised algebraically is related to models of separation logic.
Models of separation logic typically use \emph{partial} commutative monoids to
model a heap, so it is tempting to generalise the commutative monoid of
addition in our semirings to a \emph{partial} commutative monoid.
However, we find that the most natural notion of \emph{partial semiring} is
degenerate, in the sense that all partial semirings are actually (total)
semirings.

Recall that a commutative monoid (or commutative monoid object) can be
defined in any symmetric monoidal category.
A partial commutative monoid is exactly a commutative monoid object in the
category of sets and partial functions with the usual monoidal product given
by pairing of objects and morphisms (like the Cartesian product in $\Set$).
However, semirings need a Cartesian category in order to state the interaction
equations between addition and multiplication.
While the category of sets and partial functions is not Cartesian, the
standard way to manufacture a Cartesian category out of a symmetric monoidal
category $\mathcal C$ is to take the category of cocommutative comonoids
$\mathrm{CComon}(\mathcal C)$.
Intuitively, the cocommutative comonoid structure equips the underlying
object $M$ with a \emph{delete} map $\eta : M \to I$ and a \emph{duplicate}
map $\delta : M \to M \otimes M$ which are coherent with respect to each other.
All morphisms in $\mathrm{CComon}(\mathcal C)$ must respect $\eta$ and
$\delta$; in particular, both addition and multiplication must separately
form bimonoids in $\mathcal C$ together with the cocommutative comonoid.

The distributivity laws of semirings are stated below.
I include these to show that the cocommutative comonoids of a monoidal category
give enough structure to state these laws.
The other laws --- that all morphisms respect $\eta$ and $\delta$, that addition
forms a commutative monoid, and that multiplication forms a monoid --- are
standard in symmetric monoidal category theory.

\[
  \begin{tikzpicture}[baseline]
    \path
    (-1,1) node(0) {0}
    (1,2) node(x) {}
    (0,0) node(*) {*}
    (0,-1) node(res) {}
    ;

    \draw (0) -- (*);
    \draw (x) to[out=270,in=45] (*);
    \draw (*) -- (res);
  \end{tikzpicture}
  =\quad
  \begin{tikzpicture}[baseline]
    \path
    (0,0) node(0) {0}
    (0,2) node(x) {}
    (0,-1) node(res) {}
    (0,1) node(del) {$\eta$}
    ;

    \draw (0) -- (res);
    \draw (x) -- (del);
  \end{tikzpicture}
  \quad=
  \begin{tikzpicture}[baseline]
    \path
    (1,1) node(0) {0}
    (-1,2) node(x) {}
    (0,0) node(*) {*}
    (0,-1) node(res) {}
    ;

    \draw (0) -- (*);
    \draw (x) to[out=270,in=135] (*);
    \draw (*) -- (res);
  \end{tikzpicture}
\]
\begin{displaymath}
  \begin{matrix}
    \begin{tikzpicture}[baseline]
      \path
      (-1,2) node(x) {}
      (0,2) node(y) {}
      (-0.5,1) node(+) {+}
      (1,2) node(z) {}
      (0,0) node(*) {*}
      (0,-1) node(res) {}
      ;

      \draw (x) to[out=270,in=135] (+);
      \draw (y) to[out=270,in=45] (+);
      \draw (+) to[out=270,in=135] (*);
      \draw (z) to[out=270,in=45] (*);
      \draw (*) -- (res);
    \end{tikzpicture}
    =
    \begin{tikzpicture}[baseline]
      \path
      (-2,3) node(x) {}
      (-1,3) node(y) {}
      (0,3) node(z) {}
      (0,2) node(dup) {$\delta$}
      (-1,1) node(x*) {*}
      (0,1) node(y*) {*}
      (-0.5,0) node(+) {+}
      (-0.5,-1) node(res) {}
      ;

      \draw (z) -- (dup);
      \draw (x) to[out=270,in=135] (x*);
      \draw (y) to[out=270,in=135] (y*);
      \draw (dup) to[out=-150,in=45] (x*);
      \draw (dup) -- (y*);
      \draw (x*) to[out=270,in=135] (+);
      \draw (y*) to[out=270,in=45] (+);
      \draw (+) -- (res);
    \end{tikzpicture}
    &\phantom{mmmm}&
    \begin{tikzpicture}[baseline]
      \path
      (1,2) node(x) {}
      (0,2) node(y) {}
      (0.5,1) node(+) {+}
      (-1,2) node(z) {}
      (0,0) node(*) {*}
      (0,-1) node(res) {}
      ;

      \draw (x) to[out=270,in=45] (+);
      \draw (y) to[out=270,in=135] (+);
      \draw (+) to[out=270,in=45] (*);
      \draw (z) to[out=270,in=135] (*);
      \draw (*) -- (res);
    \end{tikzpicture}
    =
    \begin{tikzpicture}[baseline]
      \path
      (2,3) node(x) {}
      (1,3) node(y) {}
      (0,3) node(z) {}
      (0,2) node(dup) {$\delta$}
      (1,1) node(x*) {*}
      (0,1) node(y*) {*}
      (0.5,0) node(+) {+}
      (0.5,-1) node(res) {}
      ;

      \draw (z) -- (dup);
      \draw (x) to[out=270,in=45] (x*);
      \draw (y) to[out=270,in=45] (y*);
      \draw (dup) to[out=-30,in=135] (x*);
      \draw (dup) -- (y*);
      \draw (x*) to[out=270,in=45] (+);
      \draw (y*) to[out=270,in=135] (+);
      \draw (+) -- (res);
    \end{tikzpicture}
  \end{matrix}
\end{displaymath}

It is well known that all commutative comonoids in $(\Set, \times)$, and indeed
any Cartesian monoidal category, are trivial, in the sense that every object of
$\Set$ gives rise to exactly one commutative comonoid.
We find in the following two lemmas that this property also holds of
$\plr{\Set_{\mathrm{part}}, \otimes}$.

\begin{lemma}\label{thm:ccomon-exists}
  For each object $X$ in $\plr{\Set_{\mathrm{part}}, {\otimes}}$, there is
  a cocommutative comonoid over $X$.
\end{lemma}
\begin{proof}
  Let $\eta(x) \coloneq ()$ and $\delta(x) \coloneq (x, x)$, with both
  being defined for all $x$.
  Checking that these satisfy the cocommutative comonoid laws is routine.
  Alternatively, we can see that both $\eta$ and $\delta$, being total, are
  morphisms in $\mathrm{Set}$, where it is well known that they form a
  cocommutative comonoid.
  The equations in $\mathrm{Set}$ carry over to $\mathrm{Set}_{\mathrm{part}}$.
\end{proof}

\begin{lemma}\label{thm:ccomon-unique}
  For each object $X$ in $\plr{\Set_{\mathrm{part}}, {\otimes}}$, any
  comonoid over $X$ is the one described in \cref{thm:ccomon-exists}.
\end{lemma}
\begin{proof}
  The left unit law says that, for all $x$ and $x'$, we have
  $\exists y.~\delta(x) = (y, x') \land \eta(y) = ()$ if and only if $x = x'$.
  Letting $x'$ be $x$ and reading from right to left, we get that there is
  some $y$ such that $\delta(x) = (y, x)$ and $\eta(y) = ()$.
  Symmetrically, from the right unit law, we get some $z$ such that
  $\delta(x) = (x, z)$ and $\eta(z) = ()$.
  But because $\delta$, being a partial function, is deterministic, we have
  $(y, x) = (x, z)$, giving us that $y = z = x$, and $\delta(x) = (x, x)$.
  Moreover, because the chosen $y$ is equal to $x$, we have for all $x$ that
  $\eta(x) = ()$.
\end{proof}

That a morphism $f$ respects the $\eta$ given in \cref{thm:ccomon-exists} is
equivalent to saying that $f$ is total.
Therefore, all possible semiring operators in
$\mathrm{CComon}\plr{\Set_{\mathrm{part}}, \otimes}$ are total, meaning that
there is a corresponding semiring in $\plr{\Set, \times}$.

The above reasoning shows that semirings in the category of sets and partial
functions are not worth studying.
If we want partiality, there appear to be two options.
The first option is to give up on multiplication.
We could imagine replacing the binary multiplication operator by a set of
unary modalities satisfying fewer laws.
In particular, I make little use of addition on the left of a multiplication,
and multiplying by $\gr0$ on the left (as done by $\oc\gr0$) is unwanted in some
cases (such as when encoding DILL and PD, as in \cref{sec:translation}).
With unary modalities, we could expect all of the required laws to be
expressible in a symmetric monoidal category.
The second option is to use a different notion of partiality.
The notion of partiality given by the category of sets and partial functions is
``strict'', in that composing with an everywhere-undefined function yields an
everywhere-undefined function.
With a non-strict notion of partial function, we may be able to have interesting
partial semirings.

