\chapter{Conclusions}\label{sec:conc}

FIXME: this chapter is directly from the ESOP22 paper.

We have presented a framework for doing metatheory for a class of substructural
type systems in Agda.
The framework gives us renaming, substitution, and a usage elaborator for new
syntaxes for free, which we hope can facilitate prototyping and the
mechanisation of more interesting semantic results.
Beside the mechanised framework itself, we believe its methodology --- the use
of bunched premise combinators --- can guide and simplify the development of
(potentially unmechanised) substructural type systems.

Our account of substructurality is based on the linear algebraic
principles described by \citet{WA21}.
However, these details only really affect the definition of environment,
in which the use of linear maps is motivated by them being the standard notion
of morphism between vectors.
We could imagine that a similar notion of morphism is found for the kind of
annotations found in \citet{LicataSR17}, allowing a framework to consider
finer substructural systems.
