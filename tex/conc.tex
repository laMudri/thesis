\chapter{Conclusions}\label{sec:conc}

In this thesis, I have developed a foundation for semiring-annotated calculi in
natural deduction style.

FIXME: this chapter is directly from the ESOP22 paper.

We have presented a framework for doing metatheory for a class of substructural
type systems in Agda.
The framework gives us renaming, substitution, and a usage elaborator for new
syntaxes for free, which we hope can facilitate prototyping and the
mechanisation of more interesting semantic results.
Beside the mechanised framework itself, we believe its methodology --- the use
of bunched premise combinators --- can guide and simplify the development of
(potentially unmechanised) substructural type systems.

Our account of substructurality is based on the linear algebraic
principles described by \citet{WA21}.
However, these details only really affect the definition of environment,
in which the use of linear maps is motivated by them being the standard notion
of morphism between vectors.
We could imagine that a similar notion of morphism is found for the kind of
annotations found in \citet{LicataSR17}, allowing a framework to consider
finer substructural systems.

\begin{itemize}
  \item Novel simultaneous substitution for linear type theories
  \item Generalise \citet{AACMM21} work to substructural logics
  \item Is this a reusable library? Yes, examples
  \item Fitch-style systems (can't do).
  \item Methodology for metatheory of new calculi
  \item Semirings have been a reasonably robust basis.
  \item Can the code be directly generalised?
    Abstract away the algebraic structure of contexts.
\end{itemize}

\section{Addendum: (lack of) partiality}\label{sec:part}
As we have seen, the way additive and multiplicative rules are
realised algebraically is related to models of separation logic.
Models of separation logic typically use \emph{partial} commutative monoids to
model a heap, so it is tempting to generalise the commutative monoid of
addition in our semirings to a \emph{partial} commutative monoid.
However, we find that the most natural notion of \emph{partial semiring} is
degenerate, in the sense that all partial semirings are actually (total)
semirings.

Recall that a commutative monoid (or commutative monoid object) can be
defined in any symmetric monoidal category.
A partial commutative monoid is exactly a commutative monoid object in the
category of sets and partial functions with the usual monoidal product given
by pairing of objects and morphisms (like the Cartesian product in $\Set$).
However, semirings need a Cartesian category in order to state the interaction
equations between addition and multiplication.
While the category of sets and partial functions is not Cartesian, the
standard way to manufacture a Cartesian category out of a symmetric monoidal
category $\mathcal C$ is to take the category of cocommutative comonoids
$\mathrm{CComon}(\mathcal C)$.
Intuitively, the cocommutative comonoid structure equips the underlying
object $M$ with a \emph{delete} map $\eta : M \to I$ and a \emph{duplicate}
map $\delta : M \to M \otimes M$ which are coherent with respect to each other.
All morphisms in $\mathrm{CComon}(\mathcal C)$ must respect $\eta$ and
$\delta$; in particular, both addition and multiplication must separately
form bimonoids in $\mathcal C$ together with the cocommutative comonoid.

The distributivity laws of semirings are stated below.
I include these to show that the cocommutative comonoids of a monoidal category
give enough structure to state these laws.
The other laws --- that all morphisms respect $\eta$ and $\delta$, that addition
forms a commutative monoid, and that multiplication forms a monoid --- are
standard in symmetric monoidal category theory.

\[
  \begin{tikzpicture}[baseline]
    \path
    (-1,1) node(0) {0}
    (1,2) node(x) {}
    (0,0) node(*) {*}
    (0,-1) node(res) {}
    ;

    \draw (0) -- (*);
    \draw (x) to[out=270,in=45] (*);
    \draw (*) -- (res);
  \end{tikzpicture}
  =\quad
  \begin{tikzpicture}[baseline]
    \path
    (0,0) node(0) {0}
    (0,2) node(x) {}
    (0,-1) node(res) {}
    (0,1) node(del) {$\eta$}
    ;

    \draw (0) -- (res);
    \draw (x) -- (del);
  \end{tikzpicture}
  \quad=
  \begin{tikzpicture}[baseline]
    \path
    (1,1) node(0) {0}
    (-1,2) node(x) {}
    (0,0) node(*) {*}
    (0,-1) node(res) {}
    ;

    \draw (0) -- (*);
    \draw (x) to[out=270,in=135] (*);
    \draw (*) -- (res);
  \end{tikzpicture}
\]
\begin{displaymath}
  \begin{matrix}
    \begin{tikzpicture}[baseline]
      \path
      (-1,2) node(x) {}
      (0,2) node(y) {}
      (-0.5,1) node(+) {+}
      (1,2) node(z) {}
      (0,0) node(*) {*}
      (0,-1) node(res) {}
      ;

      \draw (x) to[out=270,in=135] (+);
      \draw (y) to[out=270,in=45] (+);
      \draw (+) to[out=270,in=135] (*);
      \draw (z) to[out=270,in=45] (*);
      \draw (*) -- (res);
    \end{tikzpicture}
    =
    \begin{tikzpicture}[baseline]
      \path
      (-2,3) node(x) {}
      (-1,3) node(y) {}
      (0,3) node(z) {}
      (0,2) node(dup) {$\delta$}
      (-1,1) node(x*) {*}
      (0,1) node(y*) {*}
      (-0.5,0) node(+) {+}
      (-0.5,-1) node(res) {}
      ;

      \draw (z) -- (dup);
      \draw (x) to[out=270,in=135] (x*);
      \draw (y) to[out=270,in=135] (y*);
      \draw (dup) to[out=-150,in=45] (x*);
      \draw (dup) -- (y*);
      \draw (x*) to[out=270,in=135] (+);
      \draw (y*) to[out=270,in=45] (+);
      \draw (+) -- (res);
    \end{tikzpicture}
    &\phantom{mmmm}&
    \begin{tikzpicture}[baseline]
      \path
      (1,2) node(x) {}
      (0,2) node(y) {}
      (0.5,1) node(+) {+}
      (-1,2) node(z) {}
      (0,0) node(*) {*}
      (0,-1) node(res) {}
      ;

      \draw (x) to[out=270,in=45] (+);
      \draw (y) to[out=270,in=135] (+);
      \draw (+) to[out=270,in=45] (*);
      \draw (z) to[out=270,in=135] (*);
      \draw (*) -- (res);
    \end{tikzpicture}
    =
    \begin{tikzpicture}[baseline]
      \path
      (2,3) node(x) {}
      (1,3) node(y) {}
      (0,3) node(z) {}
      (0,2) node(dup) {$\delta$}
      (1,1) node(x*) {*}
      (0,1) node(y*) {*}
      (0.5,0) node(+) {+}
      (0.5,-1) node(res) {}
      ;

      \draw (z) -- (dup);
      \draw (x) to[out=270,in=45] (x*);
      \draw (y) to[out=270,in=45] (y*);
      \draw (dup) to[out=-30,in=135] (x*);
      \draw (dup) -- (y*);
      \draw (x*) to[out=270,in=45] (+);
      \draw (y*) to[out=270,in=135] (+);
      \draw (+) -- (res);
    \end{tikzpicture}
  \end{matrix}
\end{displaymath}

It is well known that all commutative comonoids in $(\Set, \times)$, and indeed
any Cartesian monoidal category, are trivial, in the sense that every object of
$\Set$ gives rise to exactly one commutative comonoid.
We find in the following two lemmas that this property also holds of
$\plr{\Set_{\mathrm{part}}, \otimes}$.

\begin{lemma}\label{thm:ccomon-exists}
  For each object $X$ in $\plr{\Set_{\mathrm{part}}, {\otimes}}$, there is
  a cocommutative comonoid over $X$.
\end{lemma}
\begin{proof}
  Let $\eta(x) \coloneq ()$ and $\delta(x) \coloneq (x, x)$, with both
  being defined for all $x$.
  Checking that these satisfy the cocomutative comonoid laws is routine.
  Alternatively, we can see that both $\eta$ and $\delta$, being total, are
  morphisms in $\mathrm{Set}$, where it is well known that they form a
  cocommutative comonoid.
  The equations in $\mathrm{Set}$ carry over to $\mathrm{Set}_{\mathrm{part}}$.
\end{proof}

\begin{lemma}\label{thm:ccomon-unique}
  For each object $X$ in $\plr{\Set_{\mathrm{part}}, {\otimes}}$, any
  comonoid over $X$ is the one described in \cref{thm:ccomon-exists}.
\end{lemma}
\begin{proof}
  The left unit law says that, for all $x$ and $x'$, we have
  $\exists y.~\delta(x) = (y, x') \land \eta(y) = ()$ if and only if $x = x'$.
  Letting $x'$ be $x$ and reading from right to left, we get that there is
  some $y$ such that $\delta(x) = (y, x)$ and $\eta(y) = ()$.
  Symmetrically, from the right unit law, we get some $z$ such that
  $\delta(x) = (x, z)$ and $\eta(z) = ()$.
  But because $\delta$, being a partial function, is deterministic, we have
  $(y, x) = (x, z)$, giving us that $y = z = x$, and $\delta(x) = (x, x)$.
  Moreover, because the chosen $y$ is equal to $x$, we have for all $x$ that
  $\eta(x) = ()$.
\end{proof}

That a morphism $f$ respects the $\eta$ given in \cref{thm:ccomon-exists} is
equivalent to saying that $f$ is total.
Therefore, all possible semiring operators in
$\mathrm{CComon}\plr{\Set_{\mathrm{part}}, \otimes}$ are total, meaning that
there is a corresponding semiring in $\plr{\Set, \times}$.

The above reasoning shows that semirings in the category of sets and partial
functions are not worth studying.
If we want partiality, there appear to be two options.
The first option is to give up on multiplication.
We could imagine replacing the binary multiplication operator by a set of
unary modalities satisfying fewer laws.
In particular, I make little use of addition on the left of a multiplication,
and multiplying by $\gr0$ on the left (as done by $\oc\gr0$) is unwanted in some
cases (such as when encoding DILL and PD, as in \cref{sec:translation}).
With unary modalities, we could expect all of the required laws to be
expressible in a symmetric monoidal category.
The second option is to use a different notion of partiality.
The notion of partiality given by the category of sets and partial functions is
``strict'', in that composing with an everywhere-undefined function yields an
everywhere-undefined function.
With a non-strict notion of partial function, we may be able to have interesting
partial semirings.

