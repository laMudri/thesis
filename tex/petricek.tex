A similar substitution lemma to the one presented in this chapter appears in the
PhD thesis of \citet[p.\ 138]{petricek-thesis} under the name
\emph{multi-nary substitution}.
In my notation, \citeauthor{petricek-thesis}'s substitution rule looks like the
following, up to permutation of the contexts containing $\Gamma$.
Note that if $\Delta = \grQ\delta$, then $\gr r\Delta$ denotes the context
$\plr{\gr r\grQ}\delta$.
This rule is essentially an iterated version of the standard linear single
substitution principle, and is used by \citeauthor{petricek-thesis} as a
strengthened induction hypothesis required to derive single substitution.

\[
  \ebrule{%
    \hypo{\Delta_1 \vdash A_1}
    \hypo{\cdots}
    \hypo{\Delta_n \vdash A_n}
    \hypo{\Gamma, \gr{r_1}A_1, \ldots, \gr{r_n}A_n \vdash B}
    \infer4{\Gamma, \gr{r_1}\Delta_1, \ldots, \gr{r_n}\Delta_n \vdash B}
  }
\]

We can derive \citeauthor{petricek-thesis}-style multi-nary substitution as a
corollary of my simultaneous substitution, using reasoning similar to that of
\cref{thm:single-sub}.

\begin{corollary}\label{thm:petricek-sub}
  \Citeauthor{petricek-thesis}'s multi-nary substitution, as stated above, is
  admissible in $\name$.
\end{corollary}
\begin{proof}
  It is enough to provide a substitution of type
  \[
    \Gamma, \gr{r_1}\Delta_1, \ldots, \gr{r_n}\Delta_n
    \env\vdash \Gamma, \gr{r_1}A_1, \ldots, \gr{r_n}A_n.
  \]
  To do this, we use \cref{thm:construct-env} repeatedly, leaving us needing a
  substitution of type
  $\Gamma, \gr0\Delta_1, \ldots, \gr0\Delta_n \env\vdash \Gamma$ and terms of
  types
  \begin{align*}
    \gr0\gamma, \Delta_1, \gr0\delta_2, &\ldots, \gr0\delta_{n-1}, \gr0\delta_n
    \vdash A_1 \\
    &\vdots \\
    \gr0\gamma, \gr0\delta_1, \gr0\delta_2, &\ldots, \gr0\delta_{n-1}, \Delta_n
    \vdash A_n.
  \end{align*}
  The identity substitution and weakening by $\gr0$-annotated variables is
  enough to make these requirements line up with the given hypotheses.
\end{proof}

My substitution principle is stronger than \citeauthor{petricek-thesis}'s.
Where \citeauthor{petricek-thesis} requires that distinct variables be
available for each hypothesis, I allow for separation of uses via addition of
contexts.
Below is a prototypical example.

\begin{example}
  Let $\Ann \coloneqq \plr{\mathbb N, =, 0, +, 1, \times}$, the exact
  usage-counting posemiring.
  Then, we can construct a substitution $\rho : \gr2A \env\vdash \gr1A, \gr1A$,
  yielding a transformation of terms of the following form:
  \[
    \ebrule{%
      \hypo{\gr1A, \gr1A \vdash B}
      \infer1{\gr2A \vdash B}
    }.
  \]
  To construct $\rho$, we use \cref{thm:construct-env} case
  $\sep(\rightarrowtriangle)$, using the fact that $\gr2 \leq \gr1 + \gr1$.
  From there, two identity substitutions suffice.
  The action of $\rho$ on terms is to merge the two variables into one.
  Note that a renaming, rather than a substitution, would also suffice.
\end{example}

Most notably, my (single) substitution principle more naturally fits the
requirement we would have for the reduct of the $\beta$-rule for functions in
$\name$, whereas \citeauthor{petricek-thesis}'s substitution principle would
need some additional transformation for it to fit properly.
This comes from the fact that the $\name$ function application rule introduces
an algebraic ($+$) separation between its premises, whereas
\citeauthor{petricek-thesis}'s substitution principle separates premises only
via concatenation.
