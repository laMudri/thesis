Based on an intuitive understanding of ``usage'', recursion introduces a new
phenomenon relative to the forms of programs we have seen so far:
terms can be used an unbounded number of times.
For example, notice the following reduction in Agda.

\missingfigure{\texttt{foldr \_+\_ 0 (1 :: 2 :: 3 :: []) = 1 + (2 + (3 + 0))}}

The function \AgdaFunction{\_+\_} has been copied into 3 different places in
the running of the program.
This copying is despite no type telling us that \AgdaFunction{\_+\_} would be
used 3 times (both \verb|[1,2,3]| and \verb|[2,3]| have type
\AgdaDatatype{List}\AgdaSpace{}\AgdaDatatype{$\mathbb N$}, despite the
corresponding folds using \AgdaFunction{\_+\_} a different number of times).
As such, when checking an application of \AgdaFunction{foldr}, we need check
that we can use its functional argument (\AgdaFunction{\_+\_} in this case) an
arbitrary number of times.
If we were to fix $\Ann$ as the $\{\gr0, \gr1, \gr\omega\}$ posemiring, then
wrapping the type of the functional argument in $\oc\gr\omega$ would suffice.
However, we want to remain generic in the choice of semiring.

I propose the following additions to \name{} to support a broad class of
inductive types.
I define strictly positive functors syntactically, with the only notable
restriction being not being allowed to use the type variable $X$ in the domain
of a function type.
I then add least fixed points of such strictly positive functors to the syntax
of types.

\begin{align*}
  U &\Coloneqq A \multimap (-) \mid \oc\gr r(-) \\
  {\odot} &\Coloneqq {\otimes} \mid {\oplus} \mid {\with} \\
  F[X], G[X] &\Coloneqq X \mid A \mid U(F[X]) \mid F[X] \odot G[X] \\
  A &\Coloneqq \cdots \mid \mu X.~F[X]
\end{align*}

\begin{example}
  We may define $\mathrm{List}_A \coloneqq \mu X.~I \oplus \plr{A \otimes X}$.
\end{example}

In the typing rules, introduction of an inductive type is standard.
For the elimination rule, we follow a similar pattern to other pattern-matching
rules --- $\oplus$-E, $\otimes$-E, and $\oc$-E --- by splitting the context
and typing the eliminand in one half ($\grP$).
We type the continuation in the other half, but because the continuation may
be used multiple times, and in a modal context, we require that $\grQ$ is
preserved by all linear operations.

\begin{displaymath}
  \begin{prooftree}
    \hypo{\grR\gamma \vdash F[\mu X.~F[X]]}
    \infer1[$\mu$-I]{\grR\gamma \vdash \mu X.~F[X]}
  \end{prooftree}
\end{displaymath}
\begin{displaymath}
  \begin{prooftree}
    \hypo{\grR \leq \grP + \grQ}
    \hypo{\grP\gamma \vdash \mu X.~F[X]}
    \hypo{%
      \begin{matrix*}[l]
        \grQ \leq \gr0 \\
        \grQ \leq \grQ + \grQ \\
        \forall \gr r.~\grQ \leq \gr r\grQ
      \end{matrix*}%
    }
    \hypo{\grQ\gamma, \gr1F[C] \vdash C}
    \infer4[$\mu$-E]{\grR\gamma \vdash C}
  \end{prooftree}
\end{displaymath}

\begin{example}\label{thm:list-rules}
  For lists, we can derive the following introduction and elimination rules
  (with usage constraints omitted for brevity when obvious).

  \begin{align*}
    \begin{prooftree}
      \hypo{\grR \leq \gr0}
      \infer1[$I$-I]{I}
      \infer1[$\oplus$-I$_0$]%
      {\grR\gamma \vdash I \oplus \plr{A \otimes \mathrm{List}_A}}
      \infer1[$\mu$-I]{\grR\gamma \vdash \mathrm{List}_A}
    \end{prooftree}
    &&
    \begin{prooftree}
      \hypo{\grR \leq \grP + \grQ}
      \hypo{\grP\gamma \vdash A}
      \hypo{\grQ\gamma \vdash \mathrm{List}_A}
      \infer3[$\otimes$-I]{\grR\gamma \vdash A \otimes \mathrm{List}_A}
      \infer1[$\oplus$-I$_1$]%
      {\grR\gamma \vdash I \oplus \plr{A \otimes \mathrm{List}_A}}
      \infer1[$\mu$-I]{\grR\gamma \vdash \mathrm{List}_A}
    \end{prooftree}
  \end{align*}
  \begin{displaymath}
    \begin{prooftree}
      \hypo{\grP\gamma \vdash \mathrm{List}_A}
      \infer0[Var]{\gr0\gamma, \gr1\plr{I \oplus \plr{A \otimes C}}
        \vdash I \oplus \plr{A \otimes C}}
      \hypo{\nabla^n}
      \hypo{\nabla^c}
      \infer3[$\oplus$-E]{\grQ\gamma, \gr1\plr{I \oplus \plr{A \otimes C}}
        \vdash C}
      \infer2[$\mu$-E]{\grR\gamma \vdash C}
    \end{prooftree}
  \end{displaymath}
  \begin{align*}
    \textrm{where }\nabla^n &\coloneqq
    \begin{prooftree}
      \infer0[Var]{\gr0\gamma, \gr1I \vdash I}
      \hypo{\grQ\gamma \vdash C}
      \infer1[Wk]{\grQ\gamma, \gr0I \vdash C}
      \infer2[$I$-E]{\grQ\gamma, \gr1I \vdash C}
      \infer1[Wk]{\grQ\gamma, \gr0\plr{I \oplus \plr{A \otimes C}}, \gr1I
        \vdash C}
    \end{prooftree}
    \\\\
    \textrm{and }\nabla^c &\coloneqq
    \begin{prooftree}
      \infer0[Var]{\gr0\gamma, \gr1\plr{A \otimes C} \vdash A \otimes C}
      \hypo{\grQ\gamma, \gr1A, \gr1C \vdash C}
      \infer1[Wk]{\grQ\gamma, \gr0\plr{A \otimes C}, \gr1A, \gr1C \vdash C}
      \infer2[$\otimes$-E]{\grQ\gamma, \gr1\plr{A \otimes C} \vdash C}
      \infer1[Wk]%
      {\grQ\gamma, \gr0\plr{I \oplus \plr{A \otimes C}}, \gr1\plr{A \otimes C}
        \vdash C}
    \end{prooftree}
  \end{align*}
\end{example}

Following \cref{sec:lnd}, I want to turn the ad hoc constraints on $\grP$,
$\grQ$, and $\grR$ into the result of some premise combinators.
To do this, I introduce a new combinator $\Box^{0{+}{\times}}$ defined below,
along with the resulting implicit-context typing rules.

\begin{align*}
  \plr{\Box^{0{+}{\times}}\,T}\grR \coloneqq
  \plr{\grR \leq \gr0} \times \plr{\grR \leq \grR + \grR} \times
  \plr{\forall \gr r.~\grR \leq \gr r\grR} \times T\,\grR
\end{align*}

\begin{align*}
  \begin{prooftree}[comb]
    \hypo{\vdash F[\mu X.~F[X]]}
    \infer1[$\mu$-I]{\vdash \mu X.~F[X]}
  \end{prooftree}
  &&
  \begin{prooftree}[comb]
    \hypo{\vdash \mu X.~F[X]}
    \hypo{\sep}
    \hypo{\Box^{0{+}{\times}}\plr{\gr1F[C] \vdash C}}
    \infer3[$\mu$-E]{\vdash C}
  \end{prooftree}
\end{align*}

\begin{example}
  We can state the rules for lists derived in \cref{thm:list-rules} as follows.
  \begin{align*}
    \begin{prooftree}[comb]
      \hypo{I^*}
      \infer1{\vdash \mathrm{List}_A}
    \end{prooftree}
    &&
    \begin{prooftree}[comb]
      \hypo{\vdash A}
      \hypo{\sep}
      \hypo{\vdash \mathrm{List}_A}
      \infer3{\vdash \mathrm{List}_A}
    \end{prooftree}
    &&
    \begin{prooftree}[comb]
      \hypo{\vdash \mathrm{List}_A}
      \hypo{\sep}
      \hypo{\Box^{0{+}{\times}}
        \plr{\vdash C\hskip0.75em\dottimes\hskip0.75em\gr1A, \gr1C \vdash C}}
      \infer3{\vdash C}
    \end{prooftree}
  \end{align*}
\end{example}
