\newcommand\instDILL{\gr{01\omega}}
\newcommand\instPD{\gr{01\Box}}

\newcommand\unused{\gr0}
\newcommand\true{\gr1}
\newcommand\valid{\gr\Box}

A motivating reason to consider the system $\name$ is that
instances of it correspond to previously studied systems.
In this section, I present translations from \name{} to Dual Intuitionistic
Linear Logic~\citep{Barber1996} and the modal system of \citet{judgmental},
and vice versa.
These translations are not mechanised, as part of the reason for developing
\name{} was to avoid mechanising these systems directly.
We cannot prove that the translations form an equivalence, because I have not
written down an equational theory for \name{}, but I expect this to be easy
enough to do.

\subsection{Dual Intuitionistic Linear Logic}\label{sec:dill}

Dual Intuitionistic Linear Logic is a particular formulation of intuitionistic
linear logic introduced by \citet{Barber1996}.
Its key feature, which simplifies the metatheory of linear logic, is the use of
separate contexts for linear and intuitionistic free variables.
Here I show that DILL is a fragment of the instantiation of \name{} at the
linearity semiring $\{\gr0, \gr1, \gr\omega\}$.

The types of DILL are the same as the types of \name, except for the
restriction of $\oc\gr{r}$ to just $\oc\gr\omega$.
I will write the latter simply as $\oc$ when it appears in DILL\@.
I add sums and with-products to the calculus of \cite{Barber1996}, with the
obvious rules (stated fully in \cref{fig:dill}).
These additive type formers present no additional difficulty to the translation.

\begin{figure}
  \begin{mathpar}
    \inferrule*[right=Int-Ax]{ }
    {\gamma, A; \cdot \vdash A}
    \and
    \inferrule*[right=Lin-Ax]{ }
    {\gamma; A \vdash A}
    \and
    \inferrule*[right=$I$-I]{ }
    {\gamma; \cdot \vdash I}
    \and
    \inferrule*[right=$I$-E]
    {\gamma; \delta_1 \vdash I \\ \gamma; \delta_2 \vdash A}
    {\gamma; \delta_1, \delta_2 \vdash A}
    \and
    \inferrule*[right=$\otimes$-I]
    {\gamma; \delta_1 \vdash A \\ \gamma, \delta_2 \vdash B}
    {\gamma; \delta_1, \delta_2 \vdash A \otimes B}
    \and
    \inferrule*[right=$\otimes$-E]
    {\gamma; \delta_1 \vdash A \otimes B \\ \gamma; \delta_2, A, B \vdash C}
    {\gamma; \delta_1, \delta_2 \vdash C}
    \and
    \inferrule*[right=$\multimap$-I]
    {\gamma; \delta, A \vdash B}
    {\gamma; \delta \vdash A \multimap B}
    \and
    \inferrule*[right=$\multimap$-E]
    {\gamma; \delta_1 \vdash A \multimap B \\ \gamma; \delta_2 \vdash A}
    {\gamma; \delta_1, \delta_2 \vdash B}
    \and
    \inferrule*[right=$\oc$-I]
    {\gamma; \cdot \vdash A}
    {\gamma; \cdot \vdash \oc A}
    \and
    \inferrule*[right=$\oc$-E]
    {\gamma; \delta_1 \vdash \oc A \\ \gamma, A; \delta_2 \vdash B}
    {\gamma; \delta_1, \delta_2 \vdash B}
    \and
    \inferrule*[right=$\top$-I]{ }
    {\gamma; \delta \vdash \top}
    \and
    \inferrule*[right=$\with$-I]
    {\gamma; \delta \vdash A \\ \gamma, \delta \vdash B}
    {\gamma; \delta \vdash A \with B}
    \and
    \inferrule*[right=$\with$-E$_i$]
    {\gamma; \delta \vdash A_0 \with A_1}
    {\gamma; \delta \vdash A_i}
    \and
    \inferrule*[right=$0$-E]
    {\gamma; \delta_1 \vdash 0}
    {\gamma; \delta_1, \delta_2 \vdash A}
    \and
    \inferrule*[right=$\oplus$-I$_i$]
    {\gamma; \delta \vdash A_i}
    {\gamma; \delta \vdash A_0 \oplus A_1}
    \and
    \inferrule*[right=$\oplus$-E]
    {
      \gamma; \delta_1 \vdash A \oplus B \\
      \gamma; \delta_2, A \vdash C \\
      \gamma; \delta_2, B \vdash C
    }
    {\gamma; \delta_1, \delta_2 \vdash C}
  \end{mathpar}
  \label{fig:dill}
  \caption{The rules of DILL, extended with additive connectives}
\end{figure}

\begin{figure}
  \begin{mathpar}
    \begin{eqns}
      \mathrm{DILL} &\hookrightarrow& \name_\instDILL \\
      Y &\mapsto& \iota_Y \\
      I &\mapsto& I \\
      A \otimes B &\mapsto& A \otimes B \\
      A \multimap B &\mapsto& A \multimap B \\
      \oc A &\mapsto& \oc{\gr\omega}{A} \\
      0 &\mapsto& 0 \\
      A \oplus B &\mapsto& A \oplus B \\
      \top &\mapsto& \top \\
      A \with B &\mapsto& A \with B
    \end{eqns}
    \and
    \begin{eqns}
      \mathrm{PD} &\hookrightarrow& \name_\instPD \\
      Y &\mapsto& \iota_Y \\
      \top &\mapsto& I \\
      A \wedge B &\mapsto& A \with B \\
      A \supset B &\mapsto& A \multimap B \\
      \Box A &\mapsto& \oc{\valid}{A} \\
      \bot &\mapsto& 0 \\
      A \vee B &\mapsto& A \oplus B
    \end{eqns}
  \end{mathpar}
  \label{fig:dill-pd}
  \caption{Embedding of DILL and PD types into \name}
\end{figure}

\begin{proposition}[DILL $\to$ \name]
  Given a DILL derivation of $\gamma; \delta \vdash A$, we can produce a
  $\name_{\instDILL}$ derivation of
  $\gr\omega\gamma, \gr1\delta \vdash A$.
\end{proposition}
\begin{proof}
  By induction on the derivation.
  We have $\gr\omega \leq \gr0$, which allows us to discard
  intuitionistic variables at the var rules, and both
  $\gr1 \leq \gr1$ and $\gr\omega \leq \gr1$, which allow
  us to use both linear and intuitionistic variables.

  Weakening is used when splitting linear variables between two premises.
  For example, \TirName{$\otimes$-I} in DILL is as follows.
  \[
    \inferrule*[right=$\otimes$-I]
    {\gamma; \delta_t \vdash t : A \\ \gamma; \delta_u \vdash u : B}
    {\gamma; \delta_t, \delta_u \vdash t \otimes u : A \otimes B}
  \]
  From this, our new derivation is as follows.
  \[
    \inferrule*[right=$\otimes$-I]
    {
      \inferrule*[right=Weak]
      {\mathit{ih}_t \\\\
        \gr\omega\gamma, \gr1\delta_t
        \vdash M_t : A}
      {\gr\omega\gamma, \gr1\delta_t,
        \gr0\delta_u
        \vdash M_t : A}
      \\
      \inferrule*[right=Weak]
      {\mathit{ih}_u \\\\
        \gr\omega\gamma, \gr1\delta_u
        \vdash M_u : A}
      {\gr\omega\gamma, \gr0\delta_t,
        \gr1\delta_u
        \vdash M_u : A}
    }
    {\gr\omega\gamma, \gr1\delta_t,
      \gr1\delta_u
      \vdash \plr{M_t, M_u} : A \otimes B}
  \]
\end{proof}

When translating from \name{} to DILL, we first coerce the \name{} derivation
to be in a form easily amenable to translation into DILL\@.
An example of a \name{} derivation with no direct translation into DILL is the
following.
In DILL terms, the intuitionistic variable of the conclusion becomes a linear
variable in the premises.
Such a move is admissible in DILL, but does not come naturally.

\[
  \inferrule*[right=$\otimes$-I]
  {
    \inferrule*[right=var]{ }
    {\gr1A : x \vdash A}
    \\
    \inferrule*[right=var]{ }
    {\gr1A : x \vdash A}
    \\
    \gr\omega \leq \gr1 + \gr1
  }
  {\gr\omega A : x \vdash A \otimes A}
\]

To avoid such situations, and therefore manipulations on DILL derivations, I
show that all $\name_{\instDILL}$ derivations can be made in \emph{bottom-up}
style.
In bottom-up style, the algebraic facts we make use of are dictated by making
most general choices based on the conclusions of rules.
Bottom-up style corresponds to a (non-deterministic) form of
\emph{usage checking}, and the following lemma can be understood as saying
that that form of usage checking is sufficiently general.

\begin{definition}
  A derivation is said to be \emph{$\instDILL$-bottom-up} if only the following
  facts about addition and multiplication are used, and all proofs of
  inequalities not at leaves are by reflexivity (i.e, not using the facts that
  $\gr\omega \leq \gr0$ and $\gr\omega \leq \gr1$).

  \makebox[\textwidth][s]{
    \begin{tabular}{c|ccc}
      $+$ & $\gr0$ & $\gr1$ & $\gr\omega$ \\ \hline
      $\gr0$ & $\gr0$ & $\gr1$ & - \\
      $\gr1$ & $\gr1$ & - & - \\
      $\gr\omega$ & - & - & $\gr\omega$ \\
    \end{tabular}
    \begin{tabular}{c|ccc}
      $*$ & $\gr0$ & $\gr1$ & $\gr\omega$ \\ \hline
      $\gr0$ & - & - & $\gr0$ \\
      $\gr1$ & $\gr0$ & $\gr1$ & $\gr\omega$ \\
      $\gr\omega$ & $\gr0$ & - & $\gr\omega$ \\
    \end{tabular}
  }
\end{definition}

Bottom-up style enforces that whenever we split a context into two (for
example, in the rule \TirName{$\otimes$-I}) all unused variables in the
conclusion stay unused in the premises, intuitionistic variables stay
intuitionistic, and linear variables go either left or right.
Multiplication is only used in the rule \TirName{$\oc\gr{r}$-I}, at which point
both the result and left argument are available.
Here, the bottom-up style enforces that linear variables never appear in the
premise of \TirName{$\oc{\gr\omega}$-I}.

\begin{lemma}
  Every $\name_{\instDILL}$ derivation can be translated into a bottom-up
  $\name_{\instDILL}$ derivation.
\end{lemma}
\begin{proof}
  By induction on the shape of the derivation.
  When we come across a non-bottom-up use of addition, it must be that the
  corresponding variable in the conclusion has annotation $\gr\omega$.
  By subusaging, we can give this variable annotation $\gr\omega$ in
  the premises too, before translating the subderivations to bottom-up
  style.
  A similar argument applies to uses of multiplication, remembering that both
  the left argument and result are fixed.
\end{proof}

\begin{proposition}[\name{} $\to$ DILL]
  Given a $\name_{\instDILL}$ derivation of
  $\gr\omega\gamma, \gr1\delta,
  \gr0\theta \vdash A$ which contains only types expressible
  in DILL, we can produce a DILL derivation of $\gamma; \delta \vdash A$.
\end{proposition}
\begin{proof}
  By induction on the derivation having been translated to bottom-up form.

  In the case of \TirName{var}, all of the unused variables have annotation
  greater than $\gr0$, i.e., $\gr0$ or $\gr\omega$.
  Those annotated $\gr0$ are absent from the DILL derivation, and those
  annotated $\gr\omega$ are in the intuitionistic context.
  The used variable is annotated either $\gr1$ or $\gr\omega$.
  In the first case, we use \TirName{Lin-Ax}, and in the second case,
  \TirName{Int-Ax}.

  All binding of variables in \name{} maps directly onto DILL\@.

  Because we translated to bottom-up form, additions, as seen in, for example,
  the \TirName{$\otimes$-I} rule, can be handled straightforwardly.
  Any intuitionistic variables in the conclusion correspond to intuitionistic
  variables in both premises.
  Any linear variables in the conclusion correspond to a linear variable in
  exactly one of the premises, and is absent in the other premise.

  The only remaining rule is \TirName{$\oc\gr{r}$-I}, of which we only cover
  \TirName{$\oc{\gr\omega}$-I} (the other two targeting types not found
  in DILL).
  In this case, we know that every variable in the conclusion is annotated
  either $\gr0$ or $\gr\omega$, and every variable in the premise is
  annotated the same way.
  This corresponds exactly to the restrictions of DILL's \TirName{$\oc$-I}.
\end{proof}

\subsection{Pfenning Davies}\label{sec:pd}

The translation to and from the modal system of Pfenning and Davies
\cite{judgmental} (henceforth \emph{PD}) is similar to the translation to and
from DILL\@.
I present my variant of PD, again adding some common connectives, in
\cref{fig:pd}
The main difference is the algebra at which \name{} is instantiated.

\begin{figure}
  \begin{mathpar}
    \inferrule*[right=hyp]{ }
    {\gamma; \delta, A\;\mathit{true} \vdash A\;\mathit{true}}
    \and
    \inferrule*[right=hyp*]{ }
    {\gamma, A\;\mathit{valid}; \delta \vdash A\;\mathit{true}}
    \and
    \inferrule*[right=$\supset$I]
    {\gamma; \delta, A\;\mathit{true} \vdash B\;\mathit{true}}
    {\gamma; \delta \vdash A \supset B\;\mathit{true}}
    \and
    \inferrule*[right=$\supset$E]
    {
      \gamma; \delta \vdash A \supset B\;\mathit{true} \\
      \gamma; \delta \vdash A\;\mathit{true}
    }
    {\gamma; \delta \vdash B\;\mathit{true}}
    \and
    \inferrule*[right=$\Box$I]
    {\gamma; \cdot \vdash A\;\mathit{true}}
    {\gamma; \delta \vdash \Box A\;\mathit{true}}
    \and
    \inferrule*[right=$\Box$-E]
    {
      \gamma; \delta \vdash \Box A\;\mathit{true} \\
      \gamma, A\;\mathit{valid}; \delta \vdash B\;\mathit{true}
    }
    {\gamma; \delta \vdash B\;\mathit{true}}
    \and
    \inferrule*[right=$\top$-I]{ }
    {\gamma; \delta \vdash \top\;\mathit{true}}
    \and
    \inferrule*[right=$\wedge$-I]
    {
      \gamma; \delta \vdash A\;\mathit{true} \\
      \gamma, \delta \vdash B\;\mathit{true}
    }
    {\gamma; \delta \vdash A \wedge B\;\mathit{true}}
    \and
    \inferrule*[right=$\wedge$-E$_i$]
    {\gamma; \delta \vdash A_0 \wedge A_1\;\mathit{true}}
    {\gamma; \delta \vdash A_i\;\mathit{true}}
    \and
    \inferrule*[right=$\bot$-E]
    {\gamma; \delta \vdash \bot\;\mathit{true}}
    {\gamma; \delta \vdash A\;\mathit{true}}
    \and
    \inferrule*[right=$\vee$-I$_i$]
    {\gamma; \delta \vdash A_i\;\mathit{true}}
    {\gamma; \delta \vdash A_0 \vee A_1\;\mathit{true}}
    \and
    \inferrule*[right=$\vee$-E]
    {
      \gamma; \delta \vdash A \vee B\;\mathit{true} \\
      \gamma; \delta, A \vdash C\;\mathit{true} \\
      \gamma; \delta, B \vdash C\;\mathit{true}
    }
    {\gamma; \delta \vdash C\;\mathit{true}}
  \end{mathpar}
  \label{fig:pd}
  \caption{The rules of PD, extended with several standard connectives}
\end{figure}

\begin{definition}
  Let $\instPD$ denote the following semiring on the partially ordered set
  $\{\valid \triangleleft \true \triangleleft \unused\}$.
  \begin{itemize}
    \item $0 := \unused$.
    \item $+$ is the meet ($\wedge$) according to the subusaging order.
    \item $1 := \true$.
    \item
      \begin{tabular}{c|ccc}
        $*$ & $\unused$ & $\true$ & $\valid$ \\ \hline
        $\unused$ & $\unused$ & $\unused$ & $\unused$ \\
        $\true$ & $\unused$ & $\true$ & $\valid$ \\
        $\valid$ & $\unused$ & $\valid$ & $\valid$ \\
      \end{tabular}
  \end{itemize}
\end{definition}

The $\unused$ annotation plays only a formal role in this example.
Meanwhile, $\true$ and $\valid$ correspond to the judgement forms
$\mathit{true}$ and $\mathit{valid}$ from PD\@.
Addition being the meet makes it idempotent.
Furthermore, it gives us that $\true + \valid = \valid$ --- if somewhere we
require an assumption to be true, and elsewhere require it to be valid, then
ultimately it must be valid (from which we can deduce that it is true).
Multiplication is designed to make $\oc{\valid}$ act like PD's $\Box$.
In particular, $\valid * \valid = \valid$ says that the valid assumptions are
available before and after \TirName{$\oc{\valid}$-I}, whereas
$\valid * \true = \valid$ says that valid assumptions in the conclusion can be
weakened to true assumptions in the premise.
The latter fact does not appear in PD, and will be excluded from
\emph{bottom-up} derivations.

To keep my notation consistent with that of DILL, I swap the roles of
$\gamma$ and $\delta$ in PD compared to what they were in the original paper.
Thus, my PD judgements are of the form $\gamma; \delta \vdash A~\mathit{true}$,
where $\gamma$ contains valid assumptions and $\delta$ contains true
assumptions.

\begin{proposition}[PD $\to$ \name]
  Given a PD derivation of $\gamma; \delta \vdash t : A~\mathit{true}$, we can
  produce a $\name_{\instPD}$ derivation of
  $\valid\gamma, \true\delta \vdash A$.
\end{proposition}
\begin{proof}
  By induction on the PD derivation.
  Most PD rules have direct $\name$ counterparts, noting that variables of any
  annotation can be discarded and duplicated because we have both
  $\gr r \leq \gr 0$ and
  $\gr r \leq \gr r + \gr r$ for all
  $\gr r$.

  Care must be taken with the \TirName{$\Box$I} rule.
  We have, from the induction hypothesis, a $\name$ derivation of
  $\gr\Box\gamma \vdash A$.
  By \TirName{$\oc\valid$-I}, we have
  $\gr\Box\gamma \vdash \oc\valid A$.
  To get the desired conclusion, we must use \TirName{Weak} to get
  $\gr\Box\gamma, \gr\unused\delta \vdash \oc\valid A$, and
  then \TirName{Subuse} on the variables we just introduced (noting that
  $\true \leq \unused$) to get
  $\gr\Box\gamma, \gr\true\delta \vdash \oc\valid A$.
\end{proof}

For translating from $\name_{\instPD}$ to PD, I introduce a similar notion of
\emph{bottom-up} derivations as I did for DILL\@.
Every $\name_{\instPD}$ derivation can be translated into bottom-up style, and
then be directly translated into PD\@.

\begin{definition}
  A derivation is said to be \emph{$\instPD$-bottom-up} if only the following
  facts about addition and multiplication are used, and all proofs of
  inequalities not at leaves are by reflexivity.

  \makebox[\textwidth][s]{
    \begin{tabular}{c|ccc}
      $+$ & $\unused$ & $\true$ & $\valid$ \\ \hline
      $\unused$ & $\unused$ & - & - \\
      $\true$ & - & $\true$ & - \\
      $\valid$ & - & - & $\valid$ \\
    \end{tabular}
    \begin{tabular}{c|ccc}
      $*$ & $\unused$ & $\true$ & $\valid$ \\ \hline
      $\unused$ & - & - & $\unused$ \\
      $\true$ & $\unused$ & $\true$ & $\valid$ \\
      $\valid$ & $\unused$ & - & $\valid$ \\
    \end{tabular}
  }
\end{definition}

\begin{lemma}
  Every $\name_{\instPD}$ derivation can be translated into a bottom-up
  $\name_{\instPD}$ derivation.
\end{lemma}
\begin{proof}
  By induction on the shape of the derivation.
  Given that addition is a meet, it is clear that any non-bottom-up uses of
  addition come from one of the arguments being greater than the result.
  Therefore, it is safe to make this argument smaller in the corresponding
  premise (via subusaging), before translating that subderivation.
  For multiplication, again, there is always a lesser value of the right
  argument that will take us from a non-bottom-up fact to a bottom-up fact with
  the same left argument and result.
\end{proof}

\begin{proposition}[\name{} $\to$ PD]
  Given a $\name_{\instPD}$ derivation of
  $\valid\gamma, \true\delta, \unused\theta
  \vdash M : A$ which does not contain types using $\oc{\unused}{}$ or
  $\oc{\true}$, we can produce a PD derivation of
  $\gamma; \delta \vdash A~\mathit{true}$.
\end{proposition}
\begin{proof}
  We translate away tensor products and tensor units using
  \cref{thm:top-meet}, and translate the resulting derivation to bottom-up
  form.
  The proof proceeds by induction on the resulting derivation in the obvious
  way.
\end{proof}

As mentioned in \cref{sec:alt}, the system of \citet{AbelBernardy2020}
is unable to embed PD in this way, as it would prove
$\Box(A \wedge B) \to \Box A \wedge \Box B$, where PD and $\name$ do not.
In fact, this example shows that, even when weakening and contraction are
admissible, with- and tensor-products are distinct in their system in the
presence of modalities.
