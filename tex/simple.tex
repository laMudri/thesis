\chapter{Mechanisation of simple types}\label{sec:simple}

In this chapter, I review and justify the family of approaches usually used to
represent simple type systems inside dependently typed proof assistants.
These approaches were first presented by \citet{AR99}, who showed a way of
representing well \emph{scoped} terms in a language with polymorphic recursion,
and extended the representation to well \emph{typed} terms in a language with
dependent types.
The representation of terms relies on indexing on both a context --- giving the
types of all the free variables --- and a type for the term itself.
A basic operation on terms is \emph{simultaneous substitution}, which replaces
each variable in the context by a term in another context.
This simultaneous substitution operation fits the form of composition in a
multicategory.
Specifically, in the case of a simple intuitionistic type system, this is the
composition of a Cartesian multicategory, which allows all of the terms being
substituted in to share a context.

\section{Term representation}
We could mechanise Gentzen's definition of a natural deduction system directly,
but this definition is quite complicated.
In particular, if we want to give derivations an inductive definition, the use
of the discharge mechanism means that we actually need an inductive-inductive
type --- derivations, particularly those using $\to$-introduction, can involve
references to assumptions within their subderivations.
An inductive-inductive definition of derivations would complicate our programs
and proofs about natural deduction derivations, so I choose an alternative
representation.

Indeed, most authors since Gentzen, whether mechanising their work or not,
have opted to replace discharge of assumptions by explicit \emph{contexts} and
a variable rule.
Contexts can be justified as a way to keep track of undischarged assumptions.
In particular, we only produce derivations in the presence of a known collection
of \emph{free variables} specified by the context.
In other words, derivations are \emph{indexed} over their free variables and
their types.
When using an assumption within a derivation, we must say which free variable
it corresponds to.
Free variables are introduced by \emph{variable-binding} rules, like
$\to$-introduction.
\cref{fig:explicit-contexts} gives an example of the same derivation written
in Gentzen's style and in the explicit context style.

\begin{sidewaysfigure}
  \centering
  \begin{prooftree}
    \hypo{[A \to A \to B]^f}
    \hypo{[A]^x}
    \infer2[$\to$-E]{A \to B}
    \hypo{[A]^x}
    \infer2[$\to$-E]{B}
    \infer1[$\to$-I$^x$]{A \to B}
    \infer1[$\to$-I$^f$]{\plr{A \to A \to B} \to \plr{A \to B}}
  \end{prooftree}

  \vspace{2em}

  \begin{prooftree}
    \infer0[var$^f$]{{\color{red}f : A \to A \to B, x : A} \vdash A \to A \to B}
    \infer0[var$^x$]{{\color{red}f : A \to A \to B, x : A} \vdash A}
    \infer2[$\to$-E]{{\color{red}f : A \to A \to B, x : A} \vdash A \to B}
    \infer0[var$^x$]{{\color{red}f : A \to A \to B, x : A} \vdash A}
    \infer2[$\to$-E]{{\color{red}f : A \to A \to B, x : A} \vdash B}
    \infer1[$\to$-I$^x$]{{\color{red}f : A \to A \to B} \vdash A \to B}
    \infer1[$\to$-I$^f$]{\vdash \plr{A \to A \to B} \to \plr{A \to B}}
  \end{prooftree}
  \caption{A proof in Gentzen's natural deduction syntax, and a proof using
    explicit contexts (contexts coloured {\color{red}red})}
  \label{fig:explicit-contexts}
\end{sidewaysfigure}

Explicit contexts can be seen as a mechanism for encoding a natural deduction
system as a sequent calculus.
However, the natural deduction character of the system is maintained by
ensuring that the resultant sequent calculus is really an encoding of a
natural deduction system.
Concretely, this means that rules can only interact with the context in
restricted ways:

\begin{itemize}
  \item There is a designated \emph{variable rule}, stating that any variable
    in the context can serve as a derivation of its type.
  \item Non-variable rules may only require subterms with \emph{extended}
    contexts, i.e., subterms in which new variables have been bound.
    Non-variable rules are parametric in the existing free variables.
\end{itemize}

Having chosen to use explicit contexts, the mechanisation must have a chosen
representation of contexts as a data structure.
While the notation in \cref{fig:explicit-contexts} uses names $f$ and $x$
for variables, I opt for a nameless representation.
In a nameless representation, variables are identified by their position in
the context, rather than by a name.
The absence of names means that $\alpha$-equivalence is just on-the-nose
equality, and also that we never have to reason about freshness of names.
Agda does not have support for nominal techniques~\cite{GP02}, which may have
made names a better option.

Most mechanisations choose contexts to be an inductive list of types.
However, I instead choose a functional, tree-shaped representation, as shown
with the type \AgdaRecord{Ctx}.
The type \AgdaDatatype{LTree} is the inductive type generated by leaves and
nullary \& binary nodes, and serves as a generalised ``length'' of the context.
The contents of the context --- the types --- are then stored in the functional
vector \AgdaField{ty-ctx}, which is a mapping from leaves in \AgdaField{shape}
to object language types \AgdaDatatype{Ty}.
The advantages of the functional vector representation will not become clear
until later chapters\todo{forward reference}, where I make use of the ease of
look-up and $\eta$-law of functions.
However, I claim for now that there is little to no disadvantage in the
functional vector representation --- in particular, we have no need for
function extensionality principles because we never talk about equality of
contexts.
For example, instead of using an equality of contexts to coerce a term, we can
use renaming.
As for the tree shape, this makes context concatenation definitionally
injective, so that in cases where multiple new variables are bound in a subterm
(for example, $\otimes$-elimination), Agda's unification-based solving will
be more able to infer which variables have just been bound.

\ExecuteMetaData[\LTreetex]{LTree}
\Ctx{}

Our first data structure involving contexts is that of intrinsically typed
variables.
A variable of type
\AgdaBound{$\Gamma$}\AgdaSpace{}\AgdaRecord{$\ni$}\AgdaSpace{}\AgdaBound{A}
is given by a path \AgdaField{idx} to a type in \AgdaBound{$\Gamma$}, together
with a proof \AgdaField{tyq} that this type is equal to \AgdaBound{A}.
Variables embed into terms via the \AgdaInductiveConstructor{var} constructor of
the family \AgdaDatatype{\_⊢\_} of intrinsically simply typed terms.
The only other syntactic forms we consider for now are the eliminator and
constructor of function types \AgdaInductiveConstructor{\_`→\_} ---
\AgdaInductiveConstructor{app} and \AgdaInductiveConstructor{lam}.
Application \AgdaInductiveConstructor{app} takes two subterms of the appropriate
types, while the subterm of $\lambda$-abstraction \AgdaInductiveConstructor{lam}
is in an extended context \GA{} --- \AgdaBound{$\Gamma$} concatenated with a
singleton context containing the type \AgdaBound{A}.

\Var{}
\Term{}

\section{Renaming and substitution}\label{sec:kits}
\def\prefix{../agda/latex}
\CatchFileBetweenTags{\Var}{\prefix/SimpleKits.tex}{Var}
\CatchFileBetweenTags{\Term}{\prefix/SimpleKits.tex}{Term}
\CatchFileBetweenTags{\Ren}{\prefix/SimpleKits.tex}{Ren}
\CatchFileBetweenTags{\bindRen}{\prefix/SimpleKits.tex}{bindRen}
\CatchFileBetweenTags{\rename}{\prefix/SimpleKits.tex}{rename}
\CatchFileBetweenTags{\leftTerm}{\prefix/SimpleKits.tex}{leftTerm}
\CatchFileBetweenTags{\Sub}{\prefix/SimpleKits.tex}{Sub}
\CatchFileBetweenTags{\bindSub}{\prefix/SimpleKits.tex}{bindSub}
\CatchFileBetweenTags{\substitute}{\prefix/SimpleKits.tex}{substitute}
\CatchFileBetweenTags{\Env}{\prefix/SimpleKits.tex}{Env}
\CatchFileBetweenTags{\RenSub}{\prefix/SimpleKits.tex}{RenSub}
\CatchFileBetweenTags{\Kit}{\prefix/SimpleKits.tex}{Kit}
\CatchFileBetweenTags{\trav}{\prefix/SimpleKits.tex}{trav}
\CatchFileBetweenTags{\renKit}{\prefix/SimpleKits.tex}{renKit}
\CatchFileBetweenTags{\subKit}{\prefix/SimpleKits.tex}{subKit}

\CatchFileBetweenTags{\RenGD}{\prefix/SimpleKits.tex}{RenGD}
\CatchFileBetweenTags{\RenGADA}{\prefix/SimpleKits.tex}{RenGADA}
\CatchFileBetweenTags{\GTh}{\prefix/SimpleKits.tex}{GTh}
\CatchFileBetweenTags{\DTh}{\prefix/SimpleKits.tex}{DTh}
\CatchFileBetweenTags{\GD}{\prefix/SimpleKits.tex}{GD}

\subsection{Simultaneous renaming and simultaneous substitution}

A simultaneous renaming from $\Gamma$ to $\Delta$ is a type-preserving map from
variables in $\Delta$ to \emph{variables} in $\Gamma$, while a simultaneous
substitution is a map into \emph{terms} in $\Gamma$.
While simultaneous substitution gives us a notion of one context being
\emph{derivable} from another, simultaneous renaming gives a similar notion
of derivability restricted to structural rules.

\begin{mathpar}
  \inferrule*[right=Subst]
  {%
    \inferrule*[right=$\to$-E]
    {%
      \inferrule*[right=Var]{~}{A \to B, A \vdash A \to B}
      \\
      \inferrule*[Right=Var]{~}{A \to B, A \vdash A}
    }
    {A \to B, A \vdash B}
    \\
    B \vdash C
  }
  {A \to B, A \vdash C}
\end{mathpar}

\begin{displaymath}
  \begin{prooftree}
    \infer0[Var]{A \to B, A \vdash A \to B}
    \infer0[Var]{A \to B, A \vdash A}
    \infer2[$\to$-E]{A \to B, A \vdash B}
    \hypo{B \vdash C}
    \infer2[Subst]{A \to B, A \vdash C}
  \end{prooftree}
\end{displaymath}

\subsection{Proofs of renaming and substitution}

We start with a data type \AgdaDatatype{\_⊢\_} of intrinsically simply typed
terms.
Beside base types, the only type former we have is the function type constructor
\AgdaInductiveConstructor{\_`→\_}.
Contexts (of type \AgdaRecord{Ctx}) are implemented as the free magma on types
(\AgdaDatatype{Ty}).
Context concatenation is \AgdaFunction{\_++ᶜ\_}, and \AgdaFunction{[\_]ᶜ}
embeds types into contexts.
Typed variables in a context are given by \AgdaRecord{\_∋\_}.
A variable in
\AgdaBound{$\Gamma$}\AgdaSpace{}\AgdaRecord{$\ni$}\AgdaSpace{}\AgdaBound{A}
is given by a path \AgdaField{idx} to a type in \AgdaBound{$\Gamma$}, together
with a proof \AgdaField{tyq} that this type is equal to \AgdaBound{A}.
Variables embed into terms via the \AgdaInductiveConstructor{var} constructor.

\Var{}
\Term{}

For this syntax, a renaming from \AgdaBound{$\Gamma$} to \AgdaBound{$\Delta$}
is a map from variables in \AgdaBound{$\Delta$} to variables in
\AgdaBound{$\Gamma$}.
Substitutions instead map into terms in \AgdaBound{$\Gamma$}.

\Ren{}
\Sub{}

In the following, \AgdaFunction{ren} gives the action of a renaming on terms.
We replace variables by new variables given by the renaming \AgdaBound{$\rho$}.
For applications, we rename each subterm using the same renaming.
For $\lambda$-abstractions, we want to rename the body, but the renaming we
have has type \RenGD{}, whereas the renaming we need is of type \RenGADA{}.
To make up the difference, we introduce the \AgdaFunction{bindRen} lemma,
saying that any renaming can be extended to the right by a context
\AgdaBound{$\Theta$}~\footnote{%
  We only require extension to the right because our syntax only has binding
  on the right.
  We could also extend on the left.
}.
To produce such an extended renaming, we receive a variable in \DTh{}, with the
two cases being that this variable is either in \AgdaBound{$\Delta$} or
\AgdaBound{$\Theta$}.
In the first case, we can use the original renaming \AgdaBound{$\rho$} to get
a variable in \AgdaBound{$\Gamma$}, which can be extended in the obvious way
to a variable in \GTh{}.
In the second case, we have a variable in \AgdaBound{$\Theta$}, and just need
to extend it to be a variable in \GTh{}.

\bindRen{}
\rename{}

We use renaming to show that, like variables, we can extend the context of terms
to the right.
The operation below renames a term by replacing all variables with variables
pointing to the left of the term's new context \GD{}.

\leftTerm{}

The action of a substitution on a term is given below.
The structure is very similar to that of renaming.
The differences are the following:

\begin{itemize}
  \item A substitution gives us terms, rather than variables.
        This means that when we use the substitution on a variable, we already
        have a term, and don't need to do any embedding into terms.
  \item When dealing with newly bound variables in \AgdaFunction{bindSub}, we
        need to produce terms, so we embed the newly bound variables into terms
        using \AgdaInductiveConstructor{var}.
  \item We use \AgdaFunction{$↙ᵗ$} instead of \AgdaFunction{$↙ᵛ$}, because we
        are dealing with terms rather than variables.
\end{itemize}

\bindSub{}
\substitute{}

\subsection{Kits}

To abstract over the similarities between renaming and substitution, we can use
\emph{kits} as introduced by McBride~\cite{McBride05,BHKM12}.
Each of the three differences above is turned into a parameter of
\AgdaFunction{trav} (generalising \AgdaFunction{ren} and \AgdaFunction{sub})
and \AgdaFunction{bindEnv} (generalising \AgdaFunction{bindRen} and
\AgdaFunction{bindSub}).
In the types, \AgdaFunction{Ren} and \AgdaFunction{Sub} are generalised by
\AgdaFunction{Env}\AgdaSpace{}\AgdaBound{K} --- a function from variables in
\AgdaBound{$\Delta$} to \AgdaBound{K}-things in \AgdaBound{$\Gamma$}.

\Env{}

The function \AgdaFunction{trav} produces a term-to-term mapping based on an
environment \AgdaBound{$\rho$}.
Like renaming and substitution, the traversal \AgdaFunction{trav} replaces
variables according to \AgdaBound{$\rho$}, while keeping the rest of the
syntactic forms intact.
The three differences between renaming and substitution present themselves as
requirements of the notion of \emph{kit} we can choose:

\begin{itemize}
  \item In the \AgdaInductiveConstructor{var} case of \AgdaFunction{trav}, we
        apply the environment \AgdaBound{$\rho$} to get a \AgdaBound{K}-thing,
        and need something to turn this \AgdaBound{K}-thing into a term.
        We let \AgdaField{tm} be this context- and type-preserving map from
        \AgdaBound{K}-things to terms.
  \item When dealing with newly bound variables in \AgdaFunction{bindEnv}, we
        need to convert the new variable into a \AgdaBound{K}-thing in order to
        put it into the environment.
        We let \AgdaField{vr} be this context- and type-preserving map from
        variables to \AgdaBound{K}-things.
  \item When working out where to map old variables in an extended context, we
        need \AgdaBound{K}-things to be stable under context extensions.
        We let \AgdaField{↙ᵏ} be the function embedding a \AgdaBound{K}-thing
        into an extended context.
\end{itemize}

\trav{}

The three parameters can be given to these functions by filling the fields of
the \AgdaRecord{Kit} record.

\Kit{}

We may now redefine the types \AgdaFunction{Ren} and \AgdaFunction{Sub} via
\AgdaFunction{Env}, and derive the actions of renaming and substitution from
\AgdaFunction{trav}.
Notice that \AgdaFunction{↙ᵗ} (written out as
\AgdaFunction{ren}\AgdaSpace{}\AgdaFunction{↙ᵛ}) still relies on renaming, but
because \AgdaFunction{ren} is only being used to fill a parameter of
\AgdaFunction{trav}, \AgdaFunction{trav} itself can be used to define
\AgdaFunction{ren} in a non-circular way.
Thus, we have succeeded in avoiding the duplication of code between
\AgdaFunction{ren} and \AgdaFunction{sub}.
%\todo{Distribute code horizontally}

\begin{multicols}{3}
  \noindent\RenSub{} \columnbreak

  \noindent\renKit{} \columnbreak

  \noindent\subKit{}
\end{multicols}

%\noindent
%\begin{tabular}{l|l|l}
%  \begin{minipage}{.25\textwidth}
%    \RenSub{}
%  \end{minipage}
%  &
%  \begin{minipage}{.25\textwidth}
%    \renKit{}
%  \end{minipage}
%  &
%  \begin{minipage}{.25\textwidth}
%    \subKit{}
%  \end{minipage}
%\end{tabular}

\section{Generic semantics}\label{sec:gen-sem}
\def\SimpleSem{../agda/latex/SimpleSem.tex}

The traversal \AgdaFunction{trav} from the last section is generic in the sense
that $\V$, the type of entries in an environment, can be instantiated to many
different things.
However, in practice we only use $\ni$ and $\vdash$, giving us renaming and
substitution, respectively.
This is because \AgdaFunction{trav} only targets terms, and does so by keeping
term constructors intact and replacing only the variables by things from the
environment.
This makes substitution the most general possible traversal.

If we want to capture a broader range of traversals, including not just
syntactic but also \emph{semantic} operations, we must be able to target things
other than terms, and act in an interesting way on term constructors.
Doing a straight generalisation of the type of \AgdaFunction{trav}, this
suggests that we want a function with the following type, where
\AgdaBound{$\C$} is the type family we are targeting.

\ExecuteMetaData[\SimpleSem]{semType}

Following the implementation of \AgdaFunction{trav}, we see that
\AgdaBound{$\C$} will need to support a semantic counterpart of each syntactic
form (\AgdaInductiveConstructor{var}, \AgdaInductiveConstructor{app}, and
\AgdaInductiveConstructor{lam}).
With syntactic kits, we already asked for the field \AgdaField{tm} to interpret
\AgdaBound{$\V$}-values as terms.
We rename \AgdaField{tm} to \AgdaField{⟦var⟧} to reflect its role in the
semantic traversal \AgdaFunction{sem}.
Now, we will also ask for fields to replace the right-hand side applications of
the other term constructors.
For application, we can stick with the obvious thing: we should be able to
combine a semantic function and its semantic argument to get the semantic
result.

\ExecuteMetaData[\SimpleSem]{semVarApp}

However, we want to treat binding constructs specially, particularly because
there are semanticses with no notion of binding.
We instead provide a function from values to computations that works in any
\emph{extension} of the current context.
Keeping \AgdaField{$\swarrow^k$} as before, we get the following semantic
replacement for \AgdaRecord{Kit}.

\ExecuteMetaData[\SimpleSem]{SemanticsExplicit}

With the aim of abstracting away from explicit contexts, bringing us closer to
natural deduction, we can use some new notation to rephrase these requirements.
We will work in \AgdaRecord{Ctx}\AgdaSpace{}\AgdaSymbol{$\to$}\AgdaSpace{}%
\AgdaPrimitive{Set} rather than \AgdaPrimitive{Set}.
One of the basic connectives in this setting is the \emph{pointwise} arrow
\AgdaFunction{\_$\Rightarrow$\_}, which acts in
\AgdaRecord{Ctx}\AgdaSpace{}\AgdaSymbol{$\to$}\AgdaSpace{}\AgdaPrimitive{Set}
like the non-dependent arrow does in \AgdaPrimitive{Set}.
Another basic component is the \AgdaFunction{$\forall[\_]$} notation, which
embeds \AgdaRecord{Ctx}\AgdaSpace{}\AgdaSymbol{$\to$}\AgdaSpace{}%
\AgdaPrimitive{Set} into \AgdaPrimitive{Set} by using an implicit $\Pi$-type
to quantify over \emph{all} contexts.
Finally, at this stage, I introduce a modality \AgdaFunction{$\bigcirc$}
encapsulating the pattern of considering arbitrary \emph{extensions} of a
context.
To facilitate working in this point-free setting, I give infix versions of
the families \AgdaBound{$\V$} and \AgdaBound{$\C$} (respectively
\AgdaFunction{\_$\V\vdash$\_} and \AgdaFunction{\_$\C\vdash$\_}).
The principal use of these aliases is to fill the right argument with a type
(occuring explicitly), and leave the left argument as \AgdaFunction{\_}, i.e.,
a context given through the point-free machinery.

\ExecuteMetaData[\SimpleSem]{Circle}

\ExecuteMetaData[\SimpleSem]{SemanticsCircle}

To illustrate this definition, I will discuss a syntactic traversal ---
renaming --- and a semantic traversal --- a standard $\mathrm{Set}$ semantics.

For the renaming semantics, as with the renaming kit, we specify that
environments hold variables (\AgdaDatatype{\_$\ni$\_}) and show that variables
satisfy the required form of weakening (\AgdaFunction{$\swarrow^v$}).
Meanwhile, whereas all syntactic kits target terms
(\AgdaDatatype{\_$\vdash$\_}), with a semantic traversal we must specify the
target.
The fields \AgdaField{$\llbracket$var$\rrbracket$} and
\AgdaField{$\llbracket$app$\rrbracket$} follow straightforwardly, with variables
embedding into terms and a pair of terms of the right types giving an
application term in the same context, via the relevant constructors.
For the \AgdaField{$\llbracket$lam$\rrbracket$} case, we are given
\ExecuteMetaData[\SimpleSem]{bRenSemCircle}, and after producing a
\AgdaInductiveConstructor{lam}, are left needing a term in
\ExecuteMetaData[\SimpleSem]{resRenSemCircle}.
That the type of \AgdaBound{b} is wrapped in \AgdaFunction{$\bigcirc$} gives
us the ability to use \AgdaBound{b} in the extended context
\ExecuteMetaData[\SimpleKits]{GA}.
In particular, we point at the new variable to yield the desired term in the
same context.

\ExecuteMetaData[\SimpleSem]{RenSemCircle}

To produce a $\mathrm{Set}$ semantics, we shift from targeting terms to
targeting the interpretation of terms.
In particular, \ExecuteMetaData[\SimpleSem]{semGA}\ is the type of functions
from the interpretation of \AgdaBound{$\Gamma$} to the interpretation of
\AgdaBound{A}.
The interpretation of a type is defined as usual, by recursion on the structure
of the type.
The interpretation of a context is the interpretation for each of its types.
We still have environments storing variables, which delays the interpretation of
variables to the \AgdaField{$\llbracket$var$\rrbracket$} case and allows newly
bound variables to be referred to directly as variables, rather than fetching
them up-front from an environment of interpretations.
In the \AgdaField{$\llbracket$app$\rrbracket$} case, we have
\ExecuteMetaData[\SimpleSem]{mSetSemCircle}\ and
\ExecuteMetaData[\SimpleSem]{nSetSemCircle}, and combine them in the usual way
by distributing the interpretation of the context
\ExecuteMetaData[\SimpleSem]{gammaSetSemCircle}.
The \AgdaField{$\llbracket$lam$\rrbracket$} case involves the same placement of
the new variable into the environment as in \AgdaFunction{RenSem}.
Finally, we get the function \AgdaFunction{set} from terms to their
interpretations by passing in the identity $\ni$-environment \AgdaFunction{id}.

\ExecuteMetaData[\SimpleSem]{SetSemCircle}

The definition of \AgdaRecord{Semantics} above essentially enforces that the
term being traversed and the result of the traversal share the same binding
structure.
Concretely, \AgdaInductiveConstructor{lam} is the only case where we can bind
new variables, and at that point we must do exactly one binding.
This is fine for renaming and substitution, which preserve the binding
structure, and also for a standard denotational semantics, which is sufficiently
abstracted from binding.
However, if we want to do other syntactic translations --- for example,
converting from a syntax with $n$-ary functions to a syntax with only unary
Curried functions --- it would be useful to allow more choices when going under
a binder.
To this end, I replace the one-step binding modality \AgdaFunction{$\bigcirc$}
by the all-possible-renamings modality \AgdaFunction{$\Box$}.
\AgdaFunction{$\Box$}\AgdaSpace{}\AgdaBound{T}\AgdaSpace{}\AgdaBound{$\Gamma$}
states that \AgdaBound{T} holds not only at \AgdaBound{$\Gamma$}, but also at
any context \AgdaBound{$\Gamma^+$} containing \AgdaBound{$\Gamma$} (including
\ExecuteMetaData[\SimpleKits]{GD}\ for any \AgdaBound{$\Delta$}).

\ExecuteMetaData[\SimpleSem]{Box}

As well as the flexibility in binding structure, the \AgdaFunction{$\Box$}
modality allows us to use the somewhat more well behaved and standard
relation of renaming, rather than strict context extension.
The resulting definition of \AgdaRecord{Semantics} is as follows, and is
simply a version of the previous definition where \AgdaFunction{$\bigcirc$}
has been replaced by \AgdaFunction{$\Box$}.
It will become apparent when we implement the traversal \AgdaFunction{sem} why
the first field also changes to include a \AgdaFunction{$\Box$}.

\ExecuteMetaData[\SimpleSem]{Semantics}

Writing a \AgdaFunction{$\Box$}-based semantics is very similar to writing a
\AgdaFunction{$\bigcirc$}-based semantics, so I will only give one further
example.
I generalise the renaming example to derive a semantic traversal from any
syntactic traversal.
We need a slightly modified definition of \AgdaRecord{Kit} to provide
\emph{renaming} of \AgdaBound{$\V$}-values, rather than just extension.

\ExecuteMetaData[\SimpleSem]{Kit}

The interesting feature of the corresponding \AgdaRecord{Semantics} is that
we now pass \AgdaBound{b} the renaming \AgdaFunction{$\swarrow^v$}, projecting
the original context \AgdaBound{$\Gamma$} out of the
\AgdaInductiveConstructor{lam}-extended context
\ExecuteMetaData[\SimpleKits]{GA}.

\ExecuteMetaData[\SimpleSem]{kit}

\section{Generic syntax}
We have seen in previous sections a method for defining well typed terms,
providing them with the basic operations of renaming and substitution, and
defining type-preserving semantic traversals over those terms.
However, the Agda code we have seen only deals with one specific kind of terms
--- simply typed $\lambda$-calculus with a base type and function types.
The aim of this section is to write some code to which we can pass a
\emph{description} or \emph{signature} of a syntax and have it produce all of
the same machinery.

The description of a syntax will closely resemble the logical rules Gentzen gave
for natural deduction systems NJ and NK, but we give them a revised
interpretation.
Where Gentzen intended his rules to be applied schematically, and hypothetical
proofs to be handled via \emph{discharge} of hypotheses, we will take the rules
formally to produce a system with explicit contexts and a variable rule.
However, knowing that this resulting system came from such a description means
that we can derive variable-handling features, such as substitution, in a
generic way.

I will present a scheme based on the work of \citet{AACMM21} such that
\cref{fig:app-lam} is interpreted as the type
system we studied in the previous sections (simply typed $\lambda$-calculus with
a base type and function types).
Remember that, while these look like inference rules, I am treating them
entirely formally, collected together into a \emph{syntax description}.
The information presented in \cref{fig:app-lam} is essentially all of the
information needed for the type system sans any details about variables.
In particular, notice:
\begin{itemize}
  \item Contexts, in particular the context of a rule's conclusion, which is
    shared in all premises in the resulting type system, are elided.
    The only part of any context I record is the newly bound variables in
    premises, such as the variable bound by a $\lambda$-abstraction.
  \item There is no explicit variable rule.
    It is understood that any $x : A$ in the context of the resulting type
    system can be used to yield a term with type $A$.
\end{itemize}

\begin{figure}
  \begin{mathpar}
    \ebrule{%
      \hypo{\vdash A \to B}
      \hypo{\vdash A}
      \infer2[app]{\vdash B}
    }
    \and
    \ebrule{%
      \hypo{A \vdash B}
      \infer1[lam]{\vdash A \to B}
    }
  \end{mathpar}
  \caption{An example syntax description}
  \label{fig:app-lam}
\end{figure}

Such a scheme commits us to a certain approach to variable binding and context
management, but does not commit us to anything about the meaning of types.
For example, we do not declare that \TirName{app} and \TirName{lam} are
``elimination'' and ``introduction'' forms for the function type former.
This limits our generic results to matters of syntax and variables, but provides
a platform upon which a future semantic scheme could rest.

\begin{figure}
  \begin{align*}
    \text{Premises} && \mathit{ps}, \mathit{qs} &\Coloneqq
      \Delta \vdash A \mid {} \mid \mathit{ps} \quad \mathit{qs}
    \\
    \text{Rule} && r, s &\Coloneqq {\mathit{ps} \over \vdash A}
  \end{align*}
  \caption{The grammar of typing rules}
  \label{fig:simple-syntax}
\end{figure}

A syntax description is a set of fully instantiated rules.
In our running example, this set contains a \TirName{app}-rule and a
\TirName{lam}-rule for each pair of types $A$ and $B$.

To construct the syntax given by a description, we keep
\AgdaInductiveConstructor{var} as before, and have another constructor
\AgdaInductiveConstructor{con} for all of the logical rules.
\AgdaInductiveConstructor{con} takes a rule \AgdaBound{r} with premises
$\Delta_1 \vdash A_1; \ldots; \Delta_n \vdash A_n$ and conclusion $A$, and the
remainder of its type is as follows.
\[
  \AgdaInductiveConstructor{con}~\AgdaBound{r}~\AgdaSymbol{:}~%
  \forall\Gamma.~(\Gamma, \Delta_1 \vdash A_1) \times \cdots
  \times (\Gamma, \Delta_n \vdash A_n) \to \Gamma \vdash A
\]
Note that, in this type, $\vdash$ is the type family of terms we are
inductively constructing, as opposed to the description syntax found in the
premises.

In our generic version of \AgdaRecord{Semantics}, we keep the
\AgdaField{ren\textasciicircum$\V$} and \AgdaField{$\llbracket$var$\rrbracket$}
fields as before, and replace \AgdaField{$\llbracket$app$\rrbracket$} and
\AgdaField{$\llbracket$lam$\rrbracket$} by a
\AgdaField{$\llbracket$con$\rrbracket$} field as follows.
\[
  \AgdaField{$\llbracket$con$\rrbracket$}~\AgdaBound{r}~\AgdaSymbol{:}~%
  \forallb{%
    \Box\plr{{} \env\V \Delta_1 \dotto {} \sdtstile{}\C A_1} \dottimes
    \cdots \dottimes
    \Box\plr{{} \env\V \Delta_n \dotto {} \sdtstile{}\C A_n} \dotto
    {} \sdtstile{}\C A}
\]
I use ${} \sdtstile{}\C A$ for the Agda notation
\AgdaFunction{\_$\C\vDash$}\AgdaSpace{}\AgdaBound{A}, while ${} \env\V \Delta$
stands for the type family of environments
\AgdaSymbol{$\lambda$}\AgdaSpace{}\AgdaBound{$\Gamma$}\AgdaSpace{}%
\AgdaSymbol{$\to$}\AgdaSpace{}\AgdaFunction{Env}\AgdaSpace{}\AgdaBound{$\V$}%
\AgdaSpace{}\AgdaBound{$\Gamma$}\AgdaSpace{}\AgdaBound{$\Delta$}.
Environments appear in this definition simply as a way to write a product of
$\V$-values --- one value for each element of $\Delta$.
We could make a special case of premises which do not bind any variables, as did
\citet{AACMM21}, eliding the $\Box$ and empty environment, but I choose not to
for uniformity and simplicity of presentation.

To generate the expressions involving ellipses, I give an interpretation of the
formal rule descriptions.
The interpretation is parametrised on some
\ExecuteMetaData[\SimpleSyntaxtex]{WithScope}, where, in
\AgdaBound{,}\AgdaSpace{}\AgdaBound{$\Delta$}\AgdaSpace{}%
\AgdaBound{$\llbracket$}\AgdaSpace{}\AgdaBound{$\Gamma$}\AgdaSpace{}%
\AgdaBound{$\vdash$}\AgdaSpace{}\AgdaBound{A}\AgdaSpace{}%
\AgdaBound{$\rrbracket$}, the context \AgdaBound{$\Delta$} stands for the newly
bound variables of a premise, \AgdaBound{$\Gamma$} is the context as it was
below the rule's horizontal line, and \AgdaBound{A} is the type of the premise.
In the \AgdaInductiveConstructor{con} constructor for terms, the parameter is
$\Gamma, \Delta \vdash A$, and in the \AgdaField{$\llbracket$con$\rrbracket$}
field for semantics, the parameter is
$\Box\plr{{} \env\V \Delta \dotto {} \sdtstile{}\C A}\,\Gamma$.

A single premise with newly bound variables is interpreted by shuffling the
parts into the right place, while multiple premises are interpreted as pointwise
products of the individual premises (giving the ellipses above).

\ExecuteMetaData[\SimpleSyntaxtex]{semp}

A rule, with all its parameters instantiated, targets a specific type
\AgdaBound{A$'$}, which we check to match the desired type \AgdaBound{A}.
Finally, a whole \AgdaRecord{System} comprises a set \AgdaBound{L} of rule
labels, and \AgdaBound{rs}\AgdaSpace{}\AgdaSymbol{:}\AgdaSpace{}\AgdaBound{L}%
\AgdaSpace{}\AgdaSymbol{$\to$}\AgdaSpace{}\AgdaRecord{Rule}.
The interpretation of these data is to pick a rule label \AgdaBound{l}, and then
take the interpretation of the rule \AgdaBound{rs}\AgdaSpace{}\AgdaBound{l}.
For the sake of defining terms as a least fixed point, it is important to note
that the interpretation of a syntax description is strictly positive in the
parameter \AgdaBound{,\_$\llbracket$\_$\vdash$\_$\rrbracket$}.

\ExecuteMetaData[\SimpleSyntaxtex]{semr}
\ExecuteMetaData[\SimpleSyntaxtex]{sems}

The interpretation of a system description as a single layer of syntax is
functorial, supporting the \AgdaFunction{map-s} function when the parameter
\AgdaBound{,\_$\llbracket$\_$\vdash$\_$\rrbracket$} is given as an extra
argument named \AgdaBound{X} or \AgdaBound{Y} (which are both fixed as
parameters of \AgdaFunction{map-s}, together with \AgdaBound{$\Gamma$} and
\AgdaBound{$\Delta$}).

\ExecuteMetaData[\SimpleSyntaxtex]{map-s-type}

The implementation of \AgdaFunction{map-s} is straightforward, so I do not list
it here.
We preserve
the shape of the syntactic layer, applying the function to each \AgdaBound{X}
we find (wherever the description contains \AgdaInductiveConstructor{$\langle$}%
\AgdaSpace{}\AgdaBound{$\Delta$}\AgdaSpace{}%
\AgdaInductiveConstructor{`$\vdash$}\AgdaSpace{}\AgdaBound{A}\AgdaSpace{}%
\AgdaInductiveConstructor{$\rangle$}).

This \AgdaFunction{map-s} will be used in the generic syntax version of
\AgdaFunction{sem} to recursively apply \AgdaFunction{sem} to all subterms.
However, a major distinction between generic syntax and the specific syntax of
previous sections is that the subterms found by \AgdaFunction{map-s} are not
recognised by Agda's termination checker as \emph{structurally smaller} than
the original term.
Therefore, a na\"{i}vely written \AgdaFunction{sem} will fail Agda's termination
check.

To make \AgdaFunction{sem} pass the termination check, we have four major
options:
\begin{enumerate}
  \item Assert \AgdaFunction{sem} to be terminating, bypassing the termination
    check.
  \item Use Agda's \emph{sized types} to remember that the subterms are smaller.
  \item Avoid sized types, and index terms over some user-defined type (for
    example, natural numbers or ordinal notations) which is structurally smaller
    at subterms.
  \item Inline a new, instantiated version of \AgdaFunction{map-s} wherever it
    is used.
\end{enumerate}

Each of these approaches has drawbacks.
Approach 1 is clearly unsafe, in the sense that the fundamental lemma
\AgdaFunction{sem} is not being completely checked for type-correctness.
Approach 2 is also unsafe, because Agda's sized type implementation is known to
make the system inconsistent~\citep{AgdaIssue1201}.
Meanwhile, approach 3 is safe, but entails a lot of manually extracted and
supplied extra arguments, which I think would distract from the presentation and
make the resulting code harder to use.
Finally, approach 4 is safe, but limits code reuse (both of the function
\AgdaFunction{map-s} itself and any lemmas we may prove about it).
I choose to follow \citet{AACMM21} in using sized types, justified by the idea
that Agda may eventually have a sound implementation of sized types, at which
point I would want my code to be as easy to update for that new version of Agda
as possible.
\Citet{FS22} use approach 4, and in fact have only one use of (their equivalent
of) \AgdaFunction{map-s} in their development.

Using sized types, my type family of \AgdaRecord{System}-generic terms is as
below.
\AgdaFunction{Scope} is a name for the transformation of an
\AgdaFunction{OpenFam} into a \AgdaRecord{Ctx}\AgdaSpace{}\AgdaSymbol{$\to$}%
\AgdaSpace{}\AgdaFunction{OpenFam} which appends the extra context to the
existing context before applying the original \AgdaFunction{OpenFam}
(in this case, producing something like $\Gamma, \Delta \vdash A$ from
$\vdash$).
The Agda builtin \AgdaPostulate{$\uparrow$} produces a bigger size from an
existing size \AgdaBound{sz}, giving us here that the size of a term is 1 bigger
than the size of all of its immediate subterms.
Agda's elaborator and termination checker have special support for sizes, so we
do not have to worry much about them from this point on.

\ExecuteMetaData[\SimpleTermtex]{Tm}

Corresponding to the generic \AgdaInductiveConstructor{`con} constructor for
terms, we have a generic field \AgdaField{$\llbracket$con$\rrbracket$} in the
updated \AgdaRecord{Semantics} record.
In place of where one might expect
\AgdaFunction{Scope}\AgdaSpace{}\AgdaBound{$\C$}, we instead have
\AgdaFunction{Kripke}\AgdaSpace{}\AgdaBound{$\V$}\AgdaSpace{}\AgdaBound{$\C$},
with \AgdaFunction{Kripke} defined below.
The form of \AgdaFunction{Kripke} follows from the shape we saw in the type of
the \AgdaField{$\llbracket$lam$\rrbracket$} field we saw in \cref{sec:gen-sem},
where I use an environment targeting \AgdaBound{$\Delta$} as a way to say
``a value for each type in \AgdaBound{$\Delta$}''.

\ExecuteMetaData[\SimpleSemanticstex]{Kripke}
\ExecuteMetaData[\SimpleSemanticstex]{Semantics}

Finally, we get a generic semantic traversal as follows.
The function \AgdaFunction{bindEnv} is unchanged from \cref{sec:gen-sem}, as it
never mentions the syntax.
The type of the traversal \AgdaFunction{sem} is also basically unchanged --- we
just need to account for arbitrary term sizes (\AgdaBound{sz}), which will get
smaller when recursing on subterms.
I have chosen, as did \citet{AACMM21}, to define \AgdaFunction{sem} mutually
with a function \AgdaFunction{body}, which is like a counterpart to
\AgdaFunction{sem} dealing with newly bound variables.
Note that the mutual recursion is not essential --- for example,
\AgdaFunction{body} could simply be inlined.
The \AgdaInductiveConstructor{`var} case of \AgdaFunction{sem} is as before.
The \AgdaInductiveConstructor{`con} case, if viewed appropriately, is a direct
generalisation of the \AgdaInductiveConstructor{lam} case from earlier.
We recursively apply \AgdaFunction{sem} to all immediate subterms contained in
\AgdaBound{M} (as found by \AgdaFunction{map-s}), with an environment updated to
reflect the newly bound variables in each premise of the rule that was applied.

\ExecuteMetaData[\SimpleSemanticstex]{sem}

%\section{Natural deduction vs sequent calculus}
%In a seminal paper~\cite{Gentzen64}, Gerhard Gentzen introduces two syntactic
paradigms which remain among the most studied to this day.
These paradigms are \emph{natural deduction} and \emph{sequent calculus}, as
exemplified by natural deduction calculi NJ and NK, and sequent calculi LJ and
LK.
Restricting attention to just the intuitionistic systems NJ and LJ, these
actually differ in two orthogonal ways, which I shall prise apart in this
section.
The simpler distinction is that the logical rules in NJ are introduction and
elimination rules, whereas the logical rules in LJ are left and right rules.
But the more important distinction for this thesis is that where NJ has
assumptions, LJ has sequents explictly manipulated by structural rules.
I take the latter distinction to define natural deduction and sequent calculus,
and wherever I need to make the former distinction I shall speak of
\emph{intro-elim systems} and \emph{left-right systems}.

I will use the rest of this section as follows.
First, I introduce NJ and LJ, and some of their basic metatheory.
Then, I will give examples of systems intermediate between Gentzen's natural
deduction and sequent calculi: a left-right natural deduction calculus
$\mu\tilde\mu$, and an intro-elim sequent calculus BBdPH\@.
Both of these examples will reappear in later chapters.

\subsection{Intro-elim natural deduction: NJ}

\subsection{Left-right sequent calculus: LJ}

\subsection{Left-right natural deduction: $\mu\tilde\mu$}
The $\mu\tilde\mu$-calculus~\cite{CH00} (also known as
$\overline\lambda\mu\tilde\mu$ or system L, and closely related to Wadler's
dual calculus~\cite{Wadler03}) can be seen as an adaptation of natural deduction
to classical logic.
Though originally presented as a sequent calculus, the underlying natural
deduction calculus was later given by Herbelin~\cite[p.\ 12]{herbelin-hab}, and
I will follow the latter.
While Gentzen gave a natural deduction calculus NK for classical logic, NK
relies on adding the \emph{axiom} of excluded middle.
As axioms are not systematic parts of the calculus, they can hinder or break
metatheoretic properties like normalisation.
In contrast, the $\mu\tilde\mu$-calculus allows us to \emph{derive} excluded
middle from entirely systematic components.

In NJ, a derivation of $A$ from assumptions $\Gamma$ tells us that if each
formula in $\Gamma$ is \emph{true}, then $A$ is also \emph{true}.
The $\mu\tilde\mu$-calculus generalises the picture by allowing us to have
both \emph{true} and \emph{false} assumptions, and allowing us to conclude that
some $A$ is \emph{true}, that some $A$ is \emph{false}, or that we have reached
a contradiction.
% A similar judgement of contradiction appears in Prawitz' classical natural
% deduction calculus~\cite{Prawitz65}.
Following Herbelin, we notate the judgement that $A$ is true as ${}\vdash A$,
that $A$ is false as $A \vdash{}$, and of contradiction as $\vdash$.
The only way to derive a contradiction is to find some $A$ such that
${}\vdash A$ and $A \vdash{}$.
Meanwhile, we can derive ${}\vdash A$ by assuming $A \vdash{}$ and deriving
$\vdash$, i.e., we can prove $A$ by assuming that $A$ is false and deriving a
contradiction.
Dually, we can derive $A \vdash{}$ by assuming ${}\vdash A$ and deriving
$\vdash$.
These three methods of derivation are encoded in the following rules.

\begin{mathpar}
  \inferrule*[right=Cut]
  {{}\vdash A \\ A \vdash{}}
  {\vdash}

  \and

  \inferrule*[right=$\mu$]
  {
    [A \vdash{}] \\\\ \vdots \\\\ \vdash
    %\inferrule*[fraction={~~~}]
    %{[A \vdash{}] \\\\ \vdots}
    %{\vdash}
  }
  {{}\vdash A}

  \and

  \inferrule*[right=$\tilde\mu$]
  {[{}\vdash A] \\\\ \vdots \\\\ \vdash}
  {A \vdash{}}
\end{mathpar}

The rules for logical connectives describe how to \emph{prove} and how to
\emph{refute} a formula whose principal connective is that connective.
These correspond strongly with the right and left rules, respectively, of LJ,
and for this reason, $\mu\tilde\mu$ is usually described elsewhere as a
sequent calculus.
For example, we could choose the following rules for disjunction.
To prove $A \vee B$, we can assume that both $A$ and $B$ are false, and derive
a contradiction.
To refute $A \vee B$, we can refute $A$ and $B$ separately.

\begin{mathpar}
  \inferrule*[right=$\vee$-r]
  {[A \vdash{}][B \vdash{}] \\\\ \vdots \\\\ \vdash}
  {{}\vdash A \vee B}

  \and

  \inferrule*[right=$\vee$-l]
  {A \vdash{} \\ B \vdash{}}
  {A \vee B \vdash{}}
\end{mathpar}

\subsection{Intro-elim sequent calculus: BBdPH}

Term assignment system introduced in~\cite{BBdPH93}.

\section{Related work}
There is a vast literature on formalisations of syntaxes with binding, which I
cannot possibly do justice to in a reasonably sized thesis chapter.
Instead, I limit myself to comparisons of the \citet{AACMM21} method I follow in
this thesis to just its closest related work.

\subsection{Autosubst}

\citet{Autosubst15} present the system \emph{Autosubst}, which provides various
tools for working with syntaxes with binding in the Coq proof assistant.
Autosubst is based on similar ideas to those \citeauthor{AACMM21} use:
De Bruijn-indexed terms with a distinguished variable rule and notion of
binding, acted upon by simultaneous renaming and substitution.

The simplest differences are essentially matters of choosing the encoding that
best fits the proof assistant being used.
Coq users tend to prefer using unindexed types and propositions indexed over
them --- in this case, a type of unscoped and untyped terms plus a
``well typed'' predicate --- whereas Agda users prefer to work with only well
formed data (well scoped and well typed terms).
The latter approach more readily allows us to show generically that substitution
preserves scoping and typing, but the former approach, conversely, allows for
bespoke proofs of such facts.
For example, one theorem of \citet{Autosubst15} is type preservation for
$\mathrm{CC}_\omega$, a dependent type system we cannot express using the
machinery of \citet{AACMM21}.
In principle, one could use \citeauthor{AACMM21}'s machinery as the basis of a
similar bespoke proof, but as far as I am aware, this has not been tried.

Another main difference is that Autosubst is presented to the user largely as
a black-box implementation of substitution and related lemmas, in contrast to
\citeauthor{AACMM21}'s work exposing the \AgdaRecord{Semantics} bundle to the
user, and having substitution be just one instance.
\citet{ACMM17} and \citet{AACMM21} provide many examples of traversals over
syntax using the same generic environment management as used by substitution.
However, the focus on substitution in Autosubst has meant that reasoning about
substitutions has been given more developed support.
For example, the library provides a tactic \texttt{autosubst} which automates
many equational proofs involving substitutions based on the $\sigma$-calculus of
\citet{ACCL91}.

An interesting feature of Autosubst is \emph{heterogeneous substitution}.
The motivation for heterogeneous substitution is to handle systems like system
F, where types and terms are syntactically distinct, but both feature binding
and require a substitution operation.
Furthermore, binding and substitution of types also affects the syntax of terms,
thanks to $\Lambda$ terms.
\citeauthor{AACMM21} provide no direct equivalent to heterogeneous substitution,
and it is unclear how well their work can handle polymorphic calculi.

\citet{Autosubst18} propose some modifications to Autosubst which, as far as I
can tell, have not yet been incorporated, although all of the case studies in
the paper are mechanised in Coq.
In the paper, they adapt and extend the work of \citeauthor{ACMM17} on generic
semantic traversals to cover a variant of system F.
They use the term \emph{multivariate traversal} for the generalisation of
heterogeneous substitution to semantic traversals.
It appears that this work could be followed through to produce syntax
descriptions covering polymorphic calculi, which would provide a route for this
work to be incorporated into the Autosubst library.

\subsection{Second order abstract syntax}\label{sec:fiore}

Marcelo Fiore and various collaborators have a long line of work aiming for a
categorical account of variable-binding~\citep{FPT99,Fiore08,FH13,FH10,FM10}.
A recent, particularly relevant paper from this line of work is that of
\citet{FS22}, which mechanises some of this work to obtain a framework similar
in scope to the work of \citet{AACMM21}.
However, there are several differences between the resulting mechanised
frameworks, of methodological, mathematical, and technical nature.

Though \citet{AACMM21} did not state their results in categorical terms, it is
still useful to infer what the category-theoretic statement would have been, and
compare it to the statement actually given by \citet{FPT99} and \citet{FS22}.
When we do this, we see that while \citeauthor{FPT99} deal with the category of
contexts under renaming, and presheaves on that category.
This means that every model must be shown to respect renaming before touching
the framework of \citeauthor{FS22}.
Meanwhile, \citeauthor{AACMM21} make use of the interplay between the discrete
category of contexts and the category of contexts under renaming.
For example, the models (\AgdaRecord{Semantics}) of \citet{AACMM21} require only
that the family of semantic values $\V$ be shown to respect renaming, while the
fact that the result family $\C$ respects renaming follows as a corollary of the
traversal function \AgdaFunction{trav}.
\Citeauthor{FS22} make no distinction between $\V$ and $\C$, essentially only
having $\C$, but requiring it to respect renaming before getting the morphism
to that model from the initial model (the syntax).
In particular, this interplay allows us to derive renaming for terms in the way
we saw in \cref{sec:gen-syn} --- by making $\V$ the family of variables, which
trivially respects renaming.
%\todo{Every model is an environment model?}

Perhaps the relative complexity of the categorical account of the work of
\citet{AACMM21} is why the authors decided not to state it in such terms.
However, it is also likely that \citeauthor{AACMM21} developed their work in
quite a different style to how \citeauthor{FPT99} did, despite arriving at
similar theories.
The work of \citet{AACMM21} is designed first and foremost to facilitate
programming language mechanisations, and thus pays a lot of attention to syntax
and traversals of it, making sure that the results compute well in Agda.
On the other hand, \citeauthor{FPT99} started from theoretical investigations
about the category of models of theories with binding, and only applied their
work to the mechanisation of programming languages in the much later work of
\citet{FS22}.

In terms of the underlying theory, both works extend multi-sorted universal
algebra%\todo{cite}
with variable-binding.
However, the two extensions are subtly different.
Universal algebra already has a notion of variable, which supports renaming and
substitution, and which allows a given term to be evaluated when the free
variables of that term are assigned semantic values.
\citeauthor{AACMM21} reuse this notion of variable, and allow binding of such
variables in terms.
On the other hand, \citeauthor{FS22} recast the existing variables as
\emph{metavariables}, and introduce a new notion of (bindable) variables
separately into the syntax.
The resulting metavariables can then stand for arbitrary \emph{open} terms, and
thus each one remembers its context and requires an explicit substitution
whenever it is used.

%Compared to the work of \citet{AACMM21}, which is what I chose to guide this
%chapter and the rest of this thesis, \citet{FS22} can be understood as providing
%many extra features.
%I happened not to be interested in these specific features, so have not
%implemented them, but future development of the work I present in this thesis
%could benefit from such features.
%The extra features include an externally compiled language of syntax and theory
%descriptions, internal support for metavariables, and a treatment of the
%category of models (of which terms are the initial model and what I call
%\AgdaFunction{sem} forms the universal morphism from terms to any other model).

\Citeauthor{FS22} use metavariables to form descriptions of relations over
terms, like an equational theory over the simply typed $\lambda$-calculus.
Terms have, as well as their context of variables, a context of metavariables,
and terms can contain a metavariable wherever they could contain a subterm.
There is then an operation of metavariable substitution, which substitutes terms
in the place of metavariables.
Metavariable substitution is used to instantiate the rules of described
relations/theories.
In contrast, \citeauthor{AACMM21} make do with the variables of the metalanguage
(i.e.\ Agda variables).

More concrete work by \citet{MMS18}, building on the approach to semantics of
\citet{ACMM17} and \cref{sec:gen-sem}, also deals with syntactic contexts into
which terms can be substituted (as part of developing notions of contextual
equivalence).
However, this work makes use of several different notions of syntactic context,
rather than just the one given naturally by metavariables in the framework of
\citeauthor{FS22}.
This suggests that more research is needed about the various roles of
metavariables before any particular approach is standardised.

Technologically, \citeauthor{FS22} provide an external domain-specific language
for syntax descriptions.
From such a description, a Python program generates some boilerplate Agda code
providing the types, the algebraic signature, the well typed terms, and a proof
that the terms are the initial model.
Using code generation like this resembles parts of the Autosubst plugin, as
opposed to the purely internal, generic programming-based solution given by
\citeauthor{AACMM21} and in this thesis.

\Citet[p.\ 19]{FS22} avoid sized types by defining their equivalent of the
functorial mapping \AgdaFunction{map-s} specialised to \AgdaFunction{sem}, with
these two functions being mutually recursive.
This is a standard solution, where code reuse is traded away in favour of
satisfying the termination checker and avoiding sized types.

Despite all of these differences,
I chose to base the work of this thesis on that of \citet{AACMM21} for largely
circumstantial reasons.
I only became aware of the work of \citet{FS22} when it was published, by which
time I had already completed the bulk of the work presented in this thesis.
Also, one of the authors of the former paper is my PhD supervisor, and the other
authors too were working nearby, so it was easy to discuss their work quickly
and informally.
I think it is clear what it would mean to adapt the later work of this thesis to
a framework in the style of \citeauthor{FS22} rather than \citeauthor{AACMM21},
but it may not be obvious how usage restrictions (for ordinary variables) and
metavariables should interact.

\subsection{Substitution-based semantics}

A feature shared by the frameworks of \citet{AACMM21} and \citet{FPT99} (and
the rest of the work described in \cref{sec:fiore}) is that they are both often
concerned about how renamings act upon semantics/models.
The basic currency of both systems is presheaves on the category of contexts
under renaming, and when the user wants to produce a semantics, they must show
that the desired semantics forms such a presheaf --- amounting to showing that
the relevant type family respects renamings (in the direction that may introduce
new, unused variables to the context).

In contrast, the work of \citet{HHLM22} is based around the category of contexts
under \emph{substitution} or, more precisely, the morphisms given by semantic
environments of whatever semantics is being defined.
The focus on substitution makes this work simpler than renaming-based
frameworks, because we get to avoid talking about renaming, whereas we always
ultimately need to talk about substitution.
However, what is a gain for internal simplicity is a loss for usability.
For the syntax, starting from substitution means that the framework provides no
help in \emph{proving} the admissibility of substitution, because the syntax is
not considered a model \emph{until} we can show that it admits substitution.
Similarly, for semantic models, we have to prove that each model we consider
respects semantic substitution, which is a stronger requirement than
respecting renaming.

Additionally, \citet{HHLM22} gain further simplicity of definitions by not
keeping track of contexts.
For example, untyped substitutions on a set (as opposed to scope-indexed family)
$X$ are functions from $\mathbb N$ to $X$.
Similarly, the simply typed families introduced in section 6 of the
aforementioned paper are indexed on their type, but not their context.
Ignoring scopes/contexts lets one talk about monads (as in
\emph{De Bruijn monads}) rather than a more complex notion like relative
monads~\citep{ACU15}.
However, it also means losing potentially useful and important information.

\subsection{Nominal techniques}

There have been essentially two approaches when it comes to incorporating
nominal techniques into proof assistants.
The first is to develop a new foundational theory and develop a new proof
assistant on top of it.
The second is to take an existing proof assistant and provide a library,
possibly based on unsafe features.

The best-developed nominal foundational theory at the time of writing is
\emph{Fraenkel-Mostowski (FM) set theory}, as introduced by
\citet{Gabbay-thesis} and \citet{GP02}.
\citeauthor{Gabbay-thesis}'s thesis presents FM set theory, and then uses it as
the basis for a proof assistant Isabelle/FM and programming language FreshML\@.
FM is a variant of ZF set theory containing an infinite set $\mathbb A$ of
\emph{atoms} or \emph{names}, and the \emph{equivariance axiom}, stating that
every FM proposition is invariant under consistent permutation of $\mathbb A$.
In this setting, one can define the quantifier
$\reflectbox{$\mathsf N$}a.~\phi$,
read as introducing a \emph{new} or \emph{fresh} name $a$ to be used in the
proposition $\phi$.
As a material set theory, FM is unusual in that it refutes the axiom of choice.
In terms of applications to proof assistants, this means that FM is incompatible
with much existing Isabelle-based work, including anything using the Hilbert
$\varepsilon$-operator, which is used liberally in Isabelle/HOL\@.
For example, if the syntax of a programming language is formalised in FM, then
there is no formal connection to the kind of foundations in which interesting
denotational semantics have likely been formalised.

Related to nominal set theories is the recent work of \citet{Pitts23} about
theories of \emph{locally nameless sets}.
This behaves quite similarly to nominal theories, in that free variables have
names, and renaming is given by primitive operations.
In fact, \citet{Pitts23} shows that every locally nameless set is a nominal set.

The other approach --- implementing a nominal library within an existing proof
assistant --- is exemplified by the Nominal Isabelle library of \citet{Urban08}.
This work is based around having a countably infinite discrete type
\texttt{name} (which could be a type of natural numbers or strings, for example)
which is treated abstractly, and then defining liftings of permutations on
\texttt{name} to permutations on any other types involving \texttt{name}s.
Because these liftings are defined explicitly within Isabelle/HOL, the Nominal
Isabelle setup amounts to an explicit model of a nominal logic within
Isabelle/HOL\@.
This setup resembles what \citet{AACMM21} did within Agda, and similarly what I
do later in this thesis --- creating a library within an existing proof
assistant.
The main downside, compared to working internally to a nominal set theory, is
that the external constructions of Nominal Isabelle are more complicated, and
involve keeping track of more properties that do not come as part of the theory
the proof assistant understands.

There has also been an effort to replicate the Nominal Isabelle work in Coq by
\citet{ABW06}.
However, this work appears to have been abandoned before publication, and the
approach is unclear from the work-in-progress paper.
Nominal libraries have also been made in general-purpose programming languages,
like in Haskell with the nom package~\citep{Gabbay20}.

From a broader perspective, I claim that nominal techniques solve a subtly
different problem to that solved by the likes of Autosubst, the work of Fiore,
and the techniques discussed in detail in this chapter.
The names found in nominal techniques are more general than (names of)
variables in formal systems.
This means that nominal techniques have been applicable to problems outside
standard programming language theory, such as in the representation of graphs
and in topology~\citep{GG17}.
However, in certain ways, this is arguably not a good fit for type systems.
In a basic nominal formalisation of a syntax with binding, there is no real
distinction between the context of a term and the collection of that term's free
variables.
This, for example, leaves no natural place to put the types of variables, except
as either a partial function from the set of names to types, or from the
computed set of free variables to types.
The discrepancy becomes even more apparent in substructural systems, as we see
later, where we want tight control over the context, as opposed to the scope.

\subsection{Logical frameworks}\label{sec:lf}

Logical frameworks based on higher order abstract syntax are another approach
which has been used to mechanise the metatheory of logics and programming
languages.
Primary among logical frameworks in recent study is the Edinburgh Logical
Framework~\citep{HHP93}, also known as \emph{LF}.
The underlying theory of LF is a dependent type theory $\lambda^\Pi$ featuring
\emph{weak} function spaces.
Weak function spaces differs from the usual strong function spaces in that
functions in a weak function space cannot inspect their arguments --- only
place them whole in their result.
Therefore, variables of the logical framework can be used as if they were the
variables of the object language.

Below I give two example constructors of a type family $\mathrm{tm}$.
These declarations are similar to the rule descriptions I gave in
\cref{fig:app-lam} for their lack of explicit contexts.
In this example, $\Pi$ is the type former for dependent weak function spaces,
with $\to$ being the non-dependent specialisation of $\Pi$.
The quoted arrow $\qto$ is a type former of the object language, as in the
previous examples in this chapter.

\begin{align*}
  \mathrm{lam} &: \Pi A,B.~\plr{\mathrm{tm}\,A \to \mathrm{tm}\,B} \to
                 \mathrm{tm}\,\plr{A \qto B} \\
  \mathrm{app} &: \Pi A,B.~\mathrm{tm}\,\plr{A \qto B} \to
                 \mathrm{tm}\,A \to \mathrm{tm}\,B
\end{align*}

Like several other approaches I have described in this chapter, logical
frameworks aim to manage variables, variable binding, and contexts in a generic
way, giving the user only the choice of what logical rules they want to add to a
basic calculus.
In fact, logical frameworks go further than any other approach I have described,
never giving the user reified access to the context of object language terms,
and giving immediate access to (single) substitution simply as application of
weak functions.
LF also natively supports more complex calculi than the simply typed
$\lambda$-calculus --- for example allowing us to implement and reason about
System F.

The main drawback of logical frameworks is that making one requires making a new
proof assistant, typically with no compatibility with existing proof assistants
and their libraries of useful mathematical definitions and proofs.
Particularly, if you want to consider calculi not (conveniently) expressible in
$\lambda^\Pi$ (as I do, with substructural calculi), then you need to make a new
logical framework which is probably also incompatible with existing logical
frameworks.
As such, no single logical framework acts as a convenient and natural foundation
upon which to build a broad range of mathematics including substructural logics
and their metatheory.

