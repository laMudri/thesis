\chapter{Mechanisation of simple types}\label{sec:simple}

In this chapter, I review and justify the family of approaches usually used to
represent simple type systems inside dependently typed proof assistants:
intrinsically well scoped and well typed representations.
The line of work leading to intrinsically well typed representations of syntax
starts with \citet{BH94}, who defined, in Standard ML, a type of
$\lambda$-calculus terms indexed on the set/type of variables they contain, and
gave some operations on them.
Later, \citet{AR99} (working from a suggestion by Hook) and \citet{BP99DeBruijn}
used Haskell's support for nested
datatypes and polymorphic recursion over them to define types of terms indexed
only over \emph{free} variables, with a $\lambda$-abstraction having one fewer
free variable than its immediate subterm.
\Citet{AR99} also showed how to extend this well \emph{scoped} representation of
terms to a well \emph{typed} representation, given a metalanguage with dependent
types.
The representation of terms relies on indexing on both a context --- giving the
types of all the free variables --- and a type for the term itself.
A basic operation on terms is \emph{simultaneous substitution}, which replaces
each variable in the context by a term in another context.
%This simultaneous substitution operation fits the form of composition in a
%multicategory, giving us a connection to categorical logic~\citep{}.
%Specifically, in the case of a simple intuitionistic type system, this is the
%composition of a Cartesian multicategory, which allows all of the terms being
%substituted in to share a context.

This chapter reviews prior work, except for a mildly novel connecting stage at
the start of \cref{sec:gen-sem} introducing the
\AgdaFunction{$\bigcirc$}-modality.
First, in \cref{sec:agda-primer}, I give an introduction to Agda, the proof
assistant I work in throughout this thesis.
Then, in \cref{sec:terms}, I use a presentation of the dependently typed
encoding of the simply typed $\lambda$-calculus from \citet{AR99} to set
notational conventions.
\Cref{sec:kits} presents the work of \citet{BHKM12},
successively deriving simultaneous renaming and simultaneous substitution for
the terms defined in \cref{sec:terms} from a single traversal of the syntax.
The rest of the chapter generalises the shared core of renaming and substitution
in two dimensions: in \cref{sec:gen-sem} following \citet{ACMM17} to cover
semantic traversals, and in \cref{sec:gen-syn} following \citet{AACMM21} to
cover a whole range of simply typed syntaxes with binding, rather than just
a specific syntax.
Finally, I review some related work in \cref{sec:mech-related}.

The Agda code displayed in this chapter is available for further study at
\url{https://github.com/laMudri/thesis/tree/master/agda}, except for the code
in \cref{sec:gen-syn}, which is available at
\url{https://github.com/laMudri/generic-lr/tree/thesis/src/Generic/Simple}.

\section{Agda primer}\label{sec:agda-primer}

I use the proof assistant and programming language Agda throughout this thesis,
with Agda code being used particularly in this chapter and
\cref{sec:framework,sec:semantics,sec:example-semantics}.
As such, it is important for the reader to be able to read basic Agda syntax in
order to benefit from the parts of the exposition that reside in code listings.
The syntax of Agda is broadly similar to that of Haskell~\citep{Haskell}, and
relatively close to that of Standard ML, OCaml, and Coq version 8's Gallina
sublanguage~\citep{SML,OCaml,Coq}.
I will assume that the reader is able to read basic Haskell code, and spend most
time explaining differences thereof.

\subsection{Lexical structure}

Agda is extremely liberal in its set of allowed names.
There is just a single lexical class (unlike in Haskell, where, for example,
constructors start with a capital letter and definitions start with a lowercase
letter), and names can be any string of Unicode characters except whitespace and
special characters \verb|.;{}()@"|, apart from those strings reserved as
keywords or literals.
Therefore, we can introduce names like
\AgdaFunction{0x-+-$\uplambda\rightarrow$}
to stand for any kind of identifiable thing.
With such free-form names, ample spacing is required between identifiers.
For example, while \AgdaNumber{0}\AgdaSpace{}\AgdaDatatype{$\leq$}\AgdaSpace{}%
\AgdaNumber{1} is a possible expression containing three
identifiers, \AgdaInductiveConstructor{0$\leq$1} is a single
valid identifier.
Only the special characters may appear next to names without being separated by
whitespace.

A character with unique behaviour in Agda's syntax is the underscore
(\AgdaSymbol{\_}).
Within a name, an underscore signifies that the name will function as a mixfix
operator, allowing for an argument in the position of the underscore.
For example, the full name of the \AgdaDatatype{$\leq$} operator used in the
previous paragraph is \AgdaDatatype{\_$\leq$\_}, signifying that it can take an
argument to its left and its right.
We can also introduce closed operators, like \AgdaDatatype{[\_]}, which can take
an argument between the square brackets (e.g.\ \AgdaDatatype{[}\AgdaSpace{}%
\AgdaNumber{1}\AgdaSpace{}\AgdaDatatype{]}, with spaces still being important).
Mixfix operators can be partially applied by leaving underscores in the name in
the application.
For example, \AgdaDatatype{\_$\leq$}\AgdaSpace{}\AgdaNumber{1} could be the
predicate asserting that a number is less than or equal to 1.

On its own, an underscore has a completely different meaning, which can depend
on context.
In patterns, an underscore has the same meaning as it has in Haskell and ML ---
it holds the place of a pattern variable, but does not name that variable.
In expressions, an underscore stands for an unspecified subterm which will be
solved by unification~\citep{AP11,Miller92}.
The solving of unspecified terms is canonical and respects $\beta\eta$-equality,
unlike in Coq.

Spacing is important particularly important when dealing with underscores.
For example,
\AgdaDatatype{\_$\leq$}\AgdaSpace{}\AgdaSymbol{\_} (with a space after the
\AgdaDatatype{$\leq$} but not before) standing for the predicate asserting that
a number is less than or equal to some unspecified number.

Like Haskell, Agda's syntax is indentation-sensitive.
The distinctions conveyed by indentation are largely obvious or intuitive to
human readers (for example, allowing for line-continuation or delineating nested
modules), so I will not discuss them explicitly here.

\subsection{Functions, $\Pi$-types}\label{sec:pi-types}

Simple function types take the form
\AgdaArgument{A}\AgdaSpace{}\AgdaSymbol{$\to$}\AgdaSpace{}\AgdaArgument{B},
coinciding with Haskell's syntax.
Also as in Haskell and ML, the function arrow nests to the right.
However, Agda has a termination checker ensuring that all definable functions
are total, so many Haskell functions do not have a corresponding Agda function.

The key feature distinguishing Agda from Haskell is the presence of arbitrary
dependent types, including dependent function types ($\Pi$-types).
The basic syntax for $\Pi$-types is
\AgdaSymbol{(}\AgdaBound{x}\AgdaSpace{}\AgdaSymbol{:}\AgdaSpace{}%
\AgdaArgument{A}\AgdaSymbol{)}\AgdaSpace{}\AgdaSymbol{$\to$}\AgdaSpace{}%
\AgdaArgument{B},
where variable \AgdaBound{x} can occur free in expression \AgdaArgument{B}.
However, there are several syntactic conveniences I use throughout the code
listings.
For one, iterated $\Pi$-types can be abbreviated so that
\AgdaSymbol{(}\AgdaBound{x}\AgdaSpace{}\AgdaSymbol{:}\AgdaSpace{}%
\AgdaArgument{A}\AgdaSymbol{)}\AgdaSpace{}\AgdaSymbol{$\to$}\AgdaSpace{}%
\AgdaSymbol{(}\AgdaBound{y}\AgdaSpace{}\AgdaSymbol{:}\AgdaSpace{}%
\AgdaArgument{B}\AgdaSymbol{)}\AgdaSpace{}\AgdaSymbol{$\to$}\AgdaSpace{}%
\AgdaArgument{C}
is written just
\AgdaSymbol{(}\AgdaBound{x}\AgdaSpace{}\AgdaSymbol{:}\AgdaSpace{}%
\AgdaArgument{A}\AgdaSymbol{)}\AgdaSpace{}%
\AgdaSymbol{(}\AgdaBound{y}\AgdaSpace{}\AgdaSymbol{:}\AgdaSpace{}%
\AgdaArgument{B}\AgdaSymbol{)}\AgdaSpace{}\AgdaSymbol{$\to$}\AgdaSpace{}%
\AgdaArgument{C},
omitting the first arrow.
For another, prefixing an arrow with the \AgdaSymbol{$\forall$} symbol allows us
to omit domain types.
For example,
\AgdaSymbol{$\forall$}\AgdaSpace{}\AgdaBound{x}\AgdaSpace{}\AgdaSymbol{$\to$}%
\AgdaSpace{}\AgdaArgument{B}
is equivalent to
\AgdaSymbol{(}\AgdaBound{x}\AgdaSpace{}\AgdaSymbol{:}\AgdaSpace{}%
\AgdaSymbol{\_}\AgdaSymbol{)}\AgdaSpace{}\AgdaSymbol{$\to$}%
\AgdaSpace{}\AgdaArgument{B}.
Notice that this is a very different type to
\AgdaBound{x}\AgdaSpace{}\AgdaSymbol{$\to$}\AgdaSpace{}\AgdaArgument{B},
which is a non-dependent function type equivalent to
\AgdaSymbol{(}\AgdaSymbol{\_}\AgdaSpace{}\AgdaSymbol{:}\AgdaSpace{}%
\AgdaBound{x}\AgdaSymbol{)}\AgdaSpace{}\AgdaSymbol{$\to$}\AgdaSpace{}%
\AgdaArgument{B}.
When writing
\AgdaSymbol{$\forall$}\AgdaSpace{}\AgdaBound{x}\AgdaSpace{}\AgdaSymbol{$\to$}%
\AgdaSpace{}\AgdaArgument{B},
we assume that the occurrence of \AgdaBound{x} in \AgdaArgument{B} tells us what
type \AgdaBound{x} should have (i.e.\ there is enough information to solve the
underscore in
\AgdaSymbol{(}\AgdaBound{x}\AgdaSpace{}\AgdaSymbol{:}\AgdaSpace{}%
\AgdaSymbol{\_}\AgdaSymbol{)}\AgdaSpace{}\AgdaSymbol{$\to$}%
\AgdaSpace{}\AgdaArgument{B}).

Just like in Haskell, functions in Agda can be introduced via
\AgdaSymbol{$\uplambda$}-abstractions and clausal definitions, and are applied
by juxtaposition.
Agda also includes \emph{extended} \AgdaSymbol{$\uplambda$}-abstractions,
introduced via equivalent syntaxes
\AgdaSymbol{$\uplambda$}\AgdaSpace{}\AgdaKeyword{where}\AgdaSpace{}%
\AgdaBound{x}\AgdaSpace{}\AgdaSymbol{$\to$}\AgdaSpace{}\AgdaArgument{M} and
\AgdaSymbol{$\uplambda$}\AgdaSpace{}\AgdaSymbol{\{}\AgdaSpace{}%
\AgdaBound{x}\AgdaSpace{}\AgdaSymbol{$\to$}\AgdaSpace{}\AgdaArgument{M}%
\AgdaSpace{}\AgdaSymbol{\}},
which allow for pattern-matching on the variable \AgdaBound{x} (or all of the
variables, if there are multiple variables).

Agda allows for arbitrary function arguments to be marked as \emph{implicit} by
replacing the round brackets in the type by curly braces.
For example, if we have
\AgdaBound{f}\AgdaSpace{}\AgdaSymbol{:}\AgdaSpace{}%
\AgdaSymbol{\{}\AgdaBound{x}\AgdaSpace{}\AgdaSymbol{:}\AgdaSpace{}%
\AgdaArgument{A}\AgdaSymbol{\}}\AgdaSpace{}\AgdaSymbol{$\to$}\AgdaSpace{}%
\AgdaArgument{B},
then the argument to \AgdaBound{f} is implicit.
Being implicit means that an occurrence of \AgdaBound{f} is treated as if it has
been applied to an underscore, giving the expression \AgdaBound{f} the type
$\AgdaArgument{B}[\AgdaSymbol{\_}/\AgdaBound{x}]$ (i.e.\ \AgdaArgument{B} with
\AgdaSymbol{\_} substituted in for \AgdaBound{x}; the substitution syntax is not
part of Agda syntax).
An implicit argument can also be given explicitly in two ways.
The first of a series of implicit arguments can be given by surrounding the
argument in curly braces, and any other implicit arguments in the series can be
given by including the name of the argument.
For example,
\AgdaBound{f}\AgdaSpace{}\AgdaSymbol{\{}\AgdaSymbol{\_}\AgdaSymbol{\}},
\AgdaBound{f}\AgdaSpace{}\AgdaSymbol{\{}\AgdaArgument{x}\AgdaSpace{}%
\AgdaSymbol{=}\AgdaSpace{}\AgdaSymbol{\_}\AgdaSymbol{\}}, and just
\AgdaBound{f} are all equivalent expressions, and the underscore can be filled
in in either of the first two expressions to provide an actual value for the
implicit argument.
Implicit arguments are usually left out of \AgdaSymbol{$\uplambda$}-abstractions
and clausal definitions, but can be bound to names and pattern-matched on using
the same syntax as in expressions.

There are a few other places in the syntax using single curly braces, all of
which have meanings related to implicit arguments.
I also make a small amount of use of double curly braces
(\AgdaSymbol{\{\{} and \AgdaSymbol{\}\}}), which denote arguments which are to
be solved by instance resolution.
Instance resolution is very similar to Haskell's typeclass resolution ---
finding non-canonical solutions based on the instances in scope.

Agda uses $\Pi$-types where in Haskell we would use polymorphism.
For example, we can define an identity function as below.
The definition relies on quantifying over terms of type \AgdaPrimitive{Set},
i.e.\ (small) types.
This definition also gives an example of defining a function with an implicit
argument (\AgdaBound{X}), which can typically be inferred from either the
argument type or the return type, so can be omitted.

\ExecuteMetaData[\SnippetsTwotex]{id0}

An unfortunate feature of the definition \AgdaFunction{id$_0$} is that we cannot
apply it to the expression \AgdaPrimitive{Set}, because \AgdaPrimitive{Set}
contains only small types, and itself is a large type.
We can work around these size issues using
\emph{universe level polymorphism}~\citep{BCDE22},
as in the following definition.

\ExecuteMetaData[\SnippetsTwotex]{id}

Universe levels start at \AgdaPrimitive{0$\ell$}, with \AgdaPrimitive{Set} being
an alias for \AgdaPrimitive{Set}\AgdaSpace{}\AgdaPrimitive{0$\ell$} (and also
\AgdaPrimitive{Set$_0$}).
Larger levels can be produced with the successor operator \AgdaPrimitive{suc},
and we can take the least upper bound of two levels using the operator
\AgdaPrimitive{\_$\sqcup$\_}.

\subsection{Data types}

Agda's \AgdaKeyword{data}-declarations are similar in scope to Haskell's, with
the addition of indexing by terms of arbitrary type.
\AgdaKeyword{data}-declarations give us indexed inductive sum-of-product types.

All \AgdaKeyword{data}-declarations use GADT syntax.
The body of a declaration comprises a list of constructor names paired with
their types.
Where two constructors have the same type, they may be written on the same line
with their names separated by whitespace, as I do with the two constructors of
\AgdaDatatype{Bool} below.
\AgdaDatatype{Bool} has two constructors --- \AgdaInductiveConstructor{true} and
\AgdaInductiveConstructor{false} --- both of which have type
\AgdaDatatype{Bool}.
\AgdaDatatype{$\mathbb N$} also has two constructors, where
\AgdaInductiveConstructor{zero} has type \AgdaDatatype{$\mathbb N$} and
\AgdaInductiveConstructor{suc} is inductive, with type
\AgdaDatatype{$\mathbb N$}\AgdaSpace{}\AgdaKeyword{$\to$}\AgdaSpace{}%
\AgdaDatatype{$\mathbb N$}.

\noindent
\begin{minipage}[t]{0.5\textwidth}
  \ExecuteMetaData[\SnippetsThreetex]{Bool}
\end{minipage}
\begin{minipage}[t]{0.5\textwidth}
  \ExecuteMetaData[\SnippetsThreetex]{Nat}
\end{minipage}

\AgdaDatatype{Bool} and \AgdaDatatype{$\mathbb N$} are both types, and indeed
small types, as we can see by the fact that they are annotated to have type
\AgdaPrimitive{Set}.
We can also use \AgdaKeyword{data}-declarations to define type \emph{families}
in various ways.
The simplest is to add \emph{parameters}, as in the type family
\AgdaDatatype{List} below.
Parameters always appear to the left of the colon of the first line of the
\AgdaKeyword{data}-declaration, and are constant throughout the
\AgdaKeyword{data}-declaration.
Variables to the left of the colon can appear in the body of the
\AgdaKeyword{data}-declaration without further quantification.

\ExecuteMetaData[\SnippetsThreetex]{List}

Slightly more flexible than parameters are \emph{Protestant indices}%
\footnote{The terminology of Protestant/Catholic indices is due to Peter
  Hancock. The mnemonic is that Catholics believe in transubstantiation, which
  is seen as analogous to the instantiation of Catholic indices with expressions
  that occurs during dependent pattern matching.}.
Protestant indices also appear to the left of the colon, and also must appear
unmodified in the \emph{return type} of all of the constructors.
However, they may take different values in inductive appearances of the type
family in the argument types of constructors.
Protestant indices give a generalisation of polymorphic
recursion to indices of arbitrary type~\citep{Mycroft84,Henglein93}.

I give two examples of type families with Protestant indices.
The first, \AgdaDatatype{NestedList} is standard from the polymorphic recursion
literature.
It is worth noting at this point that Agda permits overloading of constructors,
which are disambiguated by the type family they are being used to construct.
This overloading allows \AgdaDatatype{List} and \AgdaDatatype{NestedList} to
have constructors with the same names without confusion.
The second example, \AgdaDatatype{ScopedTerm} is a data structure representing
well scoped untyped $\lambda$-calculus terms.
The Protestant index \AgdaBound{s} describes the number of variables in scope,
which increases by 1 when we introduce a $\lambda$-abstraction.
I will introduce \AgdaDatatype{Fin}, a type family with a specified natural
number of inhabitants, in the next set of examples.
As a syntactic note, in the type of the \AgdaInductiveConstructor{app}
constructor, I use the two variable names \AgdaBound{M} and \AgdaBound{N}
separated by whitespace to name two arguments with the same type.

\ExecuteMetaData[\SnippetsThreetex]{NestedList}
\ExecuteMetaData[\SnippetsThreetex]{ScopedTerm}

The most general way to make a type family is to introduce a
\emph{Catholic index}.
The types of Catholic indices are specified to the right of the colon, and can
be instantiated arbitrarily throughout the \AgdaKeyword{data}-declaration.
Catholic indices are not in scope for the body of the
\AgdaKeyword{data}-declaration, so the values filling them may need to be
quantified over in each constructor.
When this quantification is over a large type, like \AgdaPrimitive{Set}, the
type family being defined will itself need to be large, e.g.\ inhabiting
\AgdaPrimitive{Set$_1$}.
This is a major reason for not defining types like \AgdaDatatype{List} and
\AgdaDatatype{NestedList} using Catholic indices.

I give two examples of type families with Catholic indices.
The first is the \AgdaDatatype{Fin} family, as used in \AgdaDatatype{ScopedTerm}
above.
By inspection of the return types of the constructors, there is no way to
produce a canonical inhabitant of
\AgdaDatatype{Fin}\AgdaSpace{}\AgdaInductiveConstructor{zero}.
For
\AgdaDatatype{Fin}\AgdaSpace{}\AgdaSymbol(\AgdaInductiveConstructor{suc}%
\AgdaSpace{}\AgdaBound{n}\AgdaSymbol),
we can potentially use either of the constructors.
Either we use \AgdaInductiveConstructor{zero} to get a canonical inhabitant, or
if we can make a number with a smaller bound (i.e.\ an inhabitant of
\AgdaDatatype{Fin}\AgdaSpace{}\AgdaBound{n}), we can use
\AgdaInductiveConstructor{suc} to produce a larger number.

\ExecuteMetaData[\SnippetsThreetex]{Fin}

The second example of a type family with Catholic indices is more general in
nature.
Below I define \emph{propositional equality}, written
\AgdaDatatype{\_$\equiv$\_}.
It has two parameters and one Catholic index (though the standard library
version of propositional equality I use throughout this thesis has an extra
level parameter for the sake of universe level polymorphism).
The constructor \AgdaInductiveConstructor{refl} constructs an inhabitant of
\AgdaArgument{M}\AgdaSpace{}\AgdaDatatype{$\equiv$}\AgdaSpace{}%
\AgdaArgument{N} only when \AgdaArgument{N} is definitionally equal to
\AgdaArgument{M} (because terms are considered ``the same'' to the type checker
exactly when they are definitionally equal).
Notice that \AgdaInductiveConstructor{refl} does not quantify over \AgdaBound{x}
because \AgdaBound{x} is already in scope as a parameter.

\ExecuteMetaData[\SnippetsThreetex]{Eq}

It is through type families like \AgdaDatatype{\_$\equiv$\_} that we can state
and prove mathematical theorems in Agda.
In the following subsection, I show how to use such indexed type families.

\subsection{Clausal definitions}

Clausal definitions of functions in Agda look very similar to their equivalents
in Haskell.
However, definitions in Agda regularly make use of
\emph{dependent pattern matching}, which is our primary way of using indexed
data types.
Recursive definitions are also conservatively checked for termination.

I will explain the salient aspects of clausal function definitions via two
examples.
The first, unimaginatively named \AgdaFunction{lemma}, shows a simple case where
pattern matching modifies the context through unification of Catholic indices.
The second, named \AgdaFunction{elim-Fin-zero}, gives an example of proper
dependent pattern matching.

In the following definition \AgdaFunction{lemma}, we want to chase equations in
order to prove that \AgdaBound{x} is propositionally equal to \AgdaBound{z}.
We start with the following incomplete definition, where the expression
\AgdaHole{\{ \}0} marks an interaction point, or \emph{hole}, in the program, to
which we can apply interactive commands to complete the program.

\begin{code}
\>[0]\AgdaFunction{lemma}\AgdaSpace{}%
\AgdaSymbol{:}\AgdaSpace{}%
\AgdaSymbol{∀}\AgdaSpace{}%
\AgdaSymbol{\{}\AgdaBound{A}\AgdaSpace{}%
\AgdaSymbol{:}\AgdaSpace{}%
\AgdaPrimitive{Set}\AgdaSymbol{\}}\AgdaSpace{}%
\AgdaSymbol{\{}\AgdaBound{x}\AgdaSpace{}%
\AgdaBound{y}\AgdaSpace{}%
\AgdaBound{z}\AgdaSpace{}%
\AgdaSymbol{:}\AgdaSpace{}%
\AgdaBound{A}\AgdaSymbol{\}}\AgdaSpace{}%
\AgdaSymbol{→}\AgdaSpace{}%
\AgdaBound{x}\AgdaSpace{}%
\AgdaOperator{\AgdaDatatype{≡}}\AgdaSpace{}%
\AgdaBound{y}\AgdaSpace{}%
\AgdaSymbol{→}\AgdaSpace{}%
\AgdaBound{z}\AgdaSpace{}%
\AgdaOperator{\AgdaDatatype{≡}}\AgdaSpace{}%
\AgdaBound{y}\AgdaSpace{}%
\AgdaSymbol{→}\AgdaSpace{}%
\AgdaBound{x}\AgdaSpace{}%
\AgdaOperator{\AgdaDatatype{≡}}\AgdaSpace{}%
\AgdaBound{z}\<%
\\
\>[0]\AgdaFunction{lemma}\AgdaSpace{}%
\AgdaBound{p}\AgdaSpace{}%
\AgdaBound{q}\AgdaSpace{}%
\AgdaSymbol{=}\AgdaSpace{}%
\AgdaHole{\{ \}0}\<%
\end{code}

As a first step, I choose to match on the variable
\AgdaBound{p}\AgdaSpace{}\AgdaSymbol:\AgdaSpace{}%
\AgdaBound{x}\AgdaSpace{}%
\AgdaOperator{\AgdaDatatype{≡}}\AgdaSpace{}%
\AgdaBound{y}.
The only applicable pattern is \AgdaInductiveConstructor{refl}.
Doing this match has the effect of unifying \AgdaBound{y} --- which is taking
the position of the Catholic index of \AgdaDatatype{\_$\equiv$\_} --- with
\AgdaBound{x} --- which is the value of the index specified in the type of
\AgdaInductiveConstructor{refl}.
Local variables act as unification variables, so the unification succeeds with
most general unifier $[x \coloneqq x, y \coloneqq x]$.
Therefore, the type of \AgdaBound{q} becomes
\AgdaBound{z}\AgdaSpace{}%
\AgdaOperator{\AgdaDatatype{≡}}\AgdaSpace{}%
\AgdaBound{x}.

\begin{code}
\>[0]\AgdaFunction{lemma}\AgdaSpace{}%
\AgdaSymbol{:}\AgdaSpace{}%
\AgdaSymbol{∀}\AgdaSpace{}%
\AgdaSymbol{\{}\AgdaBound{A}\AgdaSpace{}%
\AgdaSymbol{:}\AgdaSpace{}%
\AgdaPrimitive{Set}\AgdaSymbol{\}}\AgdaSpace{}%
\AgdaSymbol{\{}\AgdaBound{x}\AgdaSpace{}%
\AgdaBound{y}\AgdaSpace{}%
\AgdaBound{z}\AgdaSpace{}%
\AgdaSymbol{:}\AgdaSpace{}%
\AgdaBound{A}\AgdaSymbol{\}}\AgdaSpace{}%
\AgdaSymbol{→}\AgdaSpace{}%
\AgdaBound{x}\AgdaSpace{}%
\AgdaOperator{\AgdaDatatype{≡}}\AgdaSpace{}%
\AgdaBound{y}\AgdaSpace{}%
\AgdaSymbol{→}\AgdaSpace{}%
\AgdaBound{z}\AgdaSpace{}%
\AgdaOperator{\AgdaDatatype{≡}}\AgdaSpace{}%
\AgdaBound{y}\AgdaSpace{}%
\AgdaSymbol{→}\AgdaSpace{}%
\AgdaBound{x}\AgdaSpace{}%
\AgdaOperator{\AgdaDatatype{≡}}\AgdaSpace{}%
\AgdaBound{z}\<%
\\
\>[0]\AgdaFunction{lemma}\AgdaSpace{}%
\AgdaInductiveConstructor{refl}\AgdaSpace{}%
\AgdaBound{q}\AgdaSpace{}%
\AgdaSymbol{=}\AgdaSpace{}%
\AgdaHole{\{ \}0}\<%
\end{code}

The next step is to match on \AgdaBound{q}.
This similarly unifies \AgdaBound{z} and \AgdaBound{x}, making the conclusion
type
\AgdaBound{z}\AgdaSpace{}%
\AgdaOperator{\AgdaDatatype{≡}}\AgdaSpace{}%
\AgdaBound{z}.
Finally, this conclusion type is in the image of the
\AgdaInductiveConstructor{refl} constructor, so we may fill the hole with
\AgdaInductiveConstructor{refl}.

\ExecuteMetaData[\SnippetsThreetex]{lemma}

Full \emph{dependent} pattern matching, as described by \citet{MM04}, is when
the unification of indices described above takes account of constructors.
In particular, the constructors of a data type satisfy the ``no confusion''
property --- constructors are injective and mutually disjoint.
Where we encounter disjoint constructors during unification, we may dismiss the
corresponding case as impossible.
Consider the following example (\AgdaFunction{elim-Fin-zero}).
We start with an argument
\AgdaBound{i}\AgdaSpace{}\AgdaSymbol:\AgdaSpace{}\AgdaDatatype{Fin}\AgdaSpace{}%
\AgdaInductiveConstructor{zero}, and consider which constructors could possibly
construct such a value.
However, as noted earlier, both constructors of \AgdaDatatype{Fin} target
successor values of the index, from which \AgdaInductiveConstructor{zero} is
disjoint.
Therefore, both cases are impossible.
The notation when all cases are impossible is to place empty round brackets
\AgdaSymbol{()} in the place of the impossible argument, and to not provide a
right-hand side to the clause.

\ExecuteMetaData[\SnippetsThreetex]{elim-Fin-zero}

As an example of the injectivity of constructors, the obvious example is to
internalise the proof of injectivity for a given constructor, as I do in
\AgdaFunction{suc-injective}.
We start with an argument
\AgdaBound{p}\AgdaSpace{}\AgdaSymbol:\AgdaSpace{}%
\AgdaInductiveConstructor{suc}\AgdaSpace{}\AgdaBound{m}%
\AgdaSpace{}\AgdaOperator{\AgdaDatatype{$\equiv$}}\AgdaSpace{}%
\AgdaInductiveConstructor{suc}\AgdaSpace{}\AgdaBound{n}
and match on it.
This time, we do have a possible pattern --- \AgdaInductiveConstructor{refl} ---
but working out how to change the context relies on unifying
\AgdaInductiveConstructor{suc}\AgdaSpace{}\AgdaBound{m} with
\AgdaInductiveConstructor{suc}\AgdaSpace{}\AgdaBound{n}.
We are justified in doing this, with most general unifier
$[m \coloneqq m, n \coloneqq m]$, because \AgdaInductiveConstructor{suc} is
injective (with respect to propositional equality).
If the checker for dependent pattern matching did not know that
\AgdaInductiveConstructor{suc} was injective --- for example, if it were instead
a defined function --- then the unification would fail.
This leads to the intuition that constructors and variables are well behaved
with respect to dependent pattern matching, while other expressions are not.

\ExecuteMetaData[\SnippetsThreetex]{suc-injective}

Ordinarily, each clause of a definition gives rise to a \emph{definitional}
equation between its left-hand side and right-hand side.
In intensional type theory, as implemented by Agda, definitional and
propositional equality are contrasted to each other.
Definitional equality corresponds to a decidable fragment of the natural
equational theory of the type theory.
As such, definitional equality is an entirely metatheoretic notion, and we can
neither assume nor prove directly definitional equations within the language.
Definitional equality is sometimes also called \emph{judgemental equality},
because it forms a judgement which plays a part in the rules of the type theory.
As well as from the clauses of definitions, we also get definitional equations
from $\beta$-reductions of $\lambda$-abstractions and $\eta$-laws of functions
and records.
Because the type checker treats definitionally equal terms equivalently, we are
able to refactor up to definitional equality without changing any downstream
code.

On the other hand, propositional equality is a notion internal to the language,
as we have seen by defining propositional equality (\AgdaDatatype{\_$\equiv$\_})
and proving things about it (\AgdaFunction{lemma}).
Propositional equality is sometimes known as \emph{typal equality} or
\emph{mathematical equality}.
The latter name comes from the fact that propositional equality is the closest
notion to what mathematicians usually call \emph{equality}, because, for
example, it allows us to prove things like
\AgdaBound{m}\AgdaSpace{}\AgdaFunction{+}\AgdaSpace{}\AgdaBound{n}%
\AgdaSpace{}\AgdaDatatype{$\equiv$}\AgdaSpace{}%
\AgdaBound{n}\AgdaSpace{}\AgdaFunction{+}\AgdaSpace{}\AgdaBound{m} for all
natural numbers \AgdaBound{m} and \AgdaBound{n}.
Propositional equality satisfies Leibniz' law, meaning that an inhabitant of a
type \AgdaBound{A} can be coerced into an inhabitant of any type propositionally
equal to \AgdaBound{A}.
However, this cast requires marking in the code, so is less convenient to use
than definitional equality.

Definitional equality between two terms implies their propositional equality,
because exactly when two terms are definitionally equal, the type checker is
happy to accept \AgdaInductiveConstructor{refl} as a proof.
This relationship between the two is simple, but can still be deceptive.
For example, consider the notion of injectivity with respect to definitional
and propositional equality.
A function $f$ is injective (with respect to some notion of equality $\approx$)
when, for all $x$ and $y$, we have $f\,x \approx f\,y \to x \approx y$.
Because $\approx$ appears both covariantly and contravariantly in this
definition, we have implications in neither direction between definitional
injectivity and propositional injectivity.
Indeed, we can find examples of all four possibilities:
constructors are injective in both senses; type formers, like \AgdaDatatype{Fin}
and \AgdaDatatype{List}, are definitionally injective but not propositionally
injective;
\ExecuteMetaData[\SnippetsThreetex]{double}\ can be proven to be
propositionally injective, but is not definitionally injective because
\AgdaFunction{\_+\_} is not injective;
and nearly everything else is not injective in either sense.

Because the notions of definitional and propositional injectivity are
incomparable, so too are the corresponding unification procedures.
Propositional unification (using only the injectivity of constructors) is used
during dependent pattern matching, while solving of implicit arguments and
underscores in expressions is done by definitional unification.

\subsection{Records, $\Sigma$-types}

While Agda provides built-in basic $\Pi$-types, with special syntax described in
\cref{sec:pi-types}, it does not do the same for $\Sigma$-types.
Instead, the default way to get the functionality of $\Sigma$-types is to
declare record types, similarly to how we get sums via
\AgdaKeyword{data}-declarations.
However, the standard library does provide $\Sigma$-types, via record types,
using the following declaration.

\ExecuteMetaData[\SnippetsThreetex]{Sigma}

As does the standard library, I will begin to use universe level polymorphism in
these example definitions.
Here, \AgdaBound{a} and \AgdaBound{b} are the levels of the two projections.
The level of the record type must be at least the level of the type of each
field, and in this case, the smallest such level is
\AgdaBound{a}\AgdaSpace{}\AgdaOperator{\AgdaPrimitive{$\sqcup$}}\AgdaSpace{}%
\AgdaBound{b}.
As for the main points of interest in this \AgdaKeyword{record}-declaration, it
contains two fields.
The first, \AgdaField{proj$_1$}, simply has type \AgdaBound{A}.
The second, \AgdaField{proj$_2$}, then has a type dependent on the value of the
first field.
Additionally, we give this record type a named constructor
\AgdaInductiveConstructor{\_,\_}.
Any record type can also be constructed using the more verbose syntax
\begin{code}[inline]%
\>[0]\AgdaKeyword{record}\AgdaSpace{}%
\AgdaSymbol{\{}\AgdaSpace{}%
\AgdaField{proj$_1$}\AgdaSpace{}%
\AgdaSymbol{=}\AgdaSpace{}%
\AgdaHole{\{ \}1}\AgdaSpace{}%
\AgdaSymbol{;}\AgdaSpace{}%
\AgdaField{proj$_2$}\AgdaSpace{}%
\AgdaSymbol{=}\AgdaSpace{}%
\AgdaHole{\{ \}2}\AgdaSpace{}%
\AgdaSymbol{\}}\<%
\end{code}.

The standard library provides various notations for
\AgdaRecord{$\Upsigma$}-types, useful in various situations.
In this thesis, I use \ExecuteMetaData[\SnippetsFourtex]{Sigma-syntax}\ and
\ExecuteMetaData[\SnippetsFourtex]{exists}\ as equivalent notations for
\AgdaRecord{$\Upsigma$}\AgdaSpace{}\AgdaBound{A}\AgdaSpace{}\AgdaBound{B}.
Indeed, the $\eta$-contracted form can be used with \AgdaFunction{$\exists$}, as
in \AgdaFunction{$\exists$}\AgdaSpace{}\AgdaBound{B} (\AgdaSymbol{\textbackslash}
is an alternative notation for \AgdaSymbol{$\uplambda$}, as in Haskell).
\AgdaRecord{$\Upsigma$} also specialises to non-dependent products, as given by
the infix operator \AgdaFunction{\_$\times$\_}.
This is achieved by setting the parameter \AgdaBound{B} to be a constant type
family.
The resulting operator \AgdaFunction{\_$\times$\_}, as well as the non-dependent
function type, behave better than their dependent counterparts with respect to
unification because they allow us to remain in the first order fragment of
higher order unification.

There are two main ways of using the fields of a record.
The first is to put the projections into scope using
\AgdaKeyword{open}\AgdaSpace{}\AgdaRecord{$\Upsigma$}, and then to use the field
names to project out of arbitrary terms of \AgdaRecord{$\Upsigma$}-type.
This is what I will always do when using the \AgdaRecord{$\Upsigma$}-type
family.
Within this paradigm, there are two further notational choices.
Either, we can use the field names as functions, so that
\AgdaBound{z}\AgdaSpace{}\AgdaSymbol{=}\AgdaSpace{}\AgdaField{proj$_1$}%
\AgdaSpace{}\AgdaBound{z}\AgdaSpace{}\AgdaOperator{\AgdaInductiveConstructor,}%
\AgdaSpace{}\AgdaField{proj$_2$}\AgdaSpace{}\AgdaBound{z},
or we can use postfix projections via the space-dot notation, as in
\AgdaBound{z}\AgdaSpace{}\AgdaSymbol{=}\AgdaSpace{}\AgdaBound{z}%
\AgdaSpace{}\AgdaSymbol.\AgdaField{proj$_1$}\AgdaSpace{}%
\AgdaOperator{\AgdaInductiveConstructor,}\AgdaSpace{}\AgdaBound{z}%
\AgdaSpace{}\AgdaSymbol.\AgdaField{proj$_2$}.
I tend to prefer the latter, also using it occasionally in ordinary mathematical
notation (without the space).
Both notations can also be used on the left-hand side of a clausal definition as
\emph{copatterns}.
Copatterns let us think of records as being function-like, with the fields of a
record type being the possible arguments we can pass to such a function.

The second way of using the fields of a record requires a motivating example.
Consider the below definition of the type of semigroups at universe level
\AgdaBound{$\ell$}.
A semigroup has a carrier set, a binary operation on that set, and an
associativity law for that binary operation.

\ExecuteMetaData[\SnippetsFourtex]{Semigroup}

In order to use the fields of \AgdaRecord{Semigroup} in the intended way, we
do not open them into global scope.
Doing so would mean that, for example, \AgdaField{\_$\bullet$\_} would take
three arguments: the semigroup and its two intended arguments.
Instead, we get to the point where we have a semigroup \AgdaBound{G} in scope
and use
\AgdaKeyword{open}\AgdaSpace{}\AgdaRecord{Semigroup}\AgdaSpace{}\AgdaBound{G}
to put into scope the components \emph{of \AgdaBound{G}}.
Then, the name \AgdaField{Carrier} in scope will refer to the carrier set of
\AgdaBound{G}, the name \AgdaField{\_$\bullet$\_} will refer to the binary
operator (which really takes two arguments), et cetera.
Doing this gives the impression of working ``inside'' \AgdaBound{G}, which is
the way I typically work with algebraic sets with structure.

By $\eta$-equality, two inhabitants of a record type are definitionally equal
exactly when they agree definitionally on all fields.
This often makes record types much more convenient to work with than the
corresponding single-constructor data types, which do not enjoy any $\eta$-laws.
Notably, all inhabitants of the record type \AgdaRecord{$\top$} with no fields
are definitionally equal.

Along with \AgdaRecord{$\Upsigma$} and \AgdaRecord{$\top$}, there are two more
general-purpose record types I need to cover which take advantage of two special
features of \AgdaKeyword{record}-declarations (and also
\AgdaKeyword{data}-declarations, but I use \AgdaKeyword{record}-declarations for
the convenience reason given in the previous paragraph).
The first feature is that the universe level of a record type has a lower bound
(the level of each field) but no upper bound.
Therefore, we can introduce the following declaration \AgdaRecord{Lift}, which
takes a type \AgdaBound{A} at level \AgdaBound{a} and produces an equivalent
type at a potentially higher level
\AgdaBound{a}\AgdaSpace{}\AgdaOperator{\AgdaPrimitive{$\sqcup$}}\AgdaSpace{}%
\AgdaBound{$\ell$}.
This type former is useful in situations which require a type at a specific
level, such as when constructing a type using a function.

\ExecuteMetaData[\SnippetsFourtex]{Lift}

The other interesting property we get from \AgdaKeyword{record}-declarations is
that the resulting type family is definitionally injective in its parameters.
Therefore, record types behave well in the form of unification that solves
implicit arguments.
We can use this property to take any type family \AgdaBound{F} and produce an
equivalent family \AgdaRecord{Wrap}\AgdaSpace{}\AgdaBound{F} which is
definitionally injective.

\ExecuteMetaData[\SnippetsFourtex]{Wrap}

As an example, if we have a variable
\AgdaBound{f}\AgdaSpace{}\AgdaSymbol{:}\AgdaSpace{}\AgdaRecord{Wrap}%
\AgdaSpace{}\AgdaFunction{F}\AgdaSpace{}\AgdaBound{y}
and pass it to a function with a type of the form
\AgdaSymbol{$\forall$}\AgdaSpace{}\AgdaSymbol\{\AgdaBound{x}\AgdaSymbol\}%
\AgdaSpace{}\AgdaSymbol{$\to$}\AgdaSpace{}\AgdaRecord{Wrap}\AgdaSpace{}%
\AgdaFunction{F}\AgdaSpace{}\AgdaBound{x}\AgdaSpace{}\AgdaSymbol{$\to$}%
\AgdaSpace{}\AgdaSymbol{\_},
Agda will successfully unify the type of \AgdaBound{f} with the expected type of
the argument, setting $[x \coloneqq y]$.
However, without the \AgdaRecord{Wrap}, we would need to unify
\AgdaFunction{F}\AgdaSpace{}\AgdaBound{y} with
\AgdaFunction{F}\AgdaSpace{}\AgdaBound{x}, which would fail if \AgdaFunction{F}
were not injective, because there may be multiple acceptable values of
\AgdaBound{x} up to definitional equality.

The version of \AgdaRecord{Wrap} found in Agda's standard library is
significantly more complicated to allow for type families with arbitrarily many
arguments in a convenient syntax, using the $n$-ary functions of
\citet{Allais19}.
The version in the standard library is the one I use in this thesis.
In fact, both versions of the \AgdaRecord{Wrap} type family are the first pieces
of novel work to be presented in this thesis.

\subsection{Colours}

I use the ``Conor colours'' option for Agda syntax highlighting.
This set of colours is inspired by Conor McBride's syntax highlighting for
Epigram 2.
The colour given to a name is determined by the type of declaration that name
is bound to.
The main colours are \AgdaDatatype{blue} for types and type families,
\AgdaField{red} for constructors of data types and fields of records,
\AgdaFunction{green} for definitions which may unfold/compute, and
\AgdaBound{purple} for local variables.

Separately, I use \gr{green} in many places for usage annotations in traditional
typeset mathematical notation.
This usage of green contrasts only with ordinary black text.

\section{Term representation}\label{sec:terms}
We could mechanise Gentzen's definition of a natural deduction system directly,
but this definition is quite complicated.
In particular, if we want to give derivations an inductive definition, the use
of the discharge mechanism means that we actually need an inductive-inductive
type --- derivations, particularly those using $\to$-introduction, can involve
references to assumptions within their subderivations.
An inductive-inductive definition of derivations would complicate our programs
and proofs about natural deduction derivations, so I choose an alternative
representation.

Indeed, most authors since Gentzen, whether mechanising their work or not,
have opted to replace discharge of assumptions by explicit \emph{contexts} and
a variable rule.
Contexts can be justified as a way to keep track of undischarged assumptions.
In particular, we only produce derivations in the presence of a known collection
of \emph{free variables} specified by the context.
In other words, derivations are \emph{indexed} over their free variables and
their types.
When using an assumption within a derivation, we must say which free variable
it corresponds to.
Free variables are introduced by \emph{variable-binding} rules, like
$\to$-introduction.
\cref{fig:explicit-contexts} gives an example of the same derivation written
in Gentzen's style and in the explicit context style.

\begin{sidewaysfigure}
  \centering
  \begin{prooftree}
    \hypo{[A \to A \to B]^f}
    \hypo{[A]^x}
    \infer2[$\to$-E]{A \to B}
    \hypo{[A]^x}
    \infer2[$\to$-E]{B}
    \infer1[$\to$-I$^x$]{A \to B}
    \infer1[$\to$-I$^f$]{\plr{A \to A \to B} \to \plr{A \to B}}
  \end{prooftree}

  \vspace{2em}

  \begin{prooftree}
    \infer0[var$^f$]{{\color{red}f : A \to A \to B, x : A} \vdash A \to A \to B}
    \infer0[var$^x$]{{\color{red}f : A \to A \to B, x : A} \vdash A}
    \infer2[$\to$-E]{{\color{red}f : A \to A \to B, x : A} \vdash A \to B}
    \infer0[var$^x$]{{\color{red}f : A \to A \to B, x : A} \vdash A}
    \infer2[$\to$-E]{{\color{red}f : A \to A \to B, x : A} \vdash B}
    \infer1[$\to$-I$^x$]{{\color{red}f : A \to A \to B} \vdash A \to B}
    \infer1[$\to$-I$^f$]{\vdash \plr{A \to A \to B} \to \plr{A \to B}}
  \end{prooftree}
  \caption{A proof in Gentzen's natural deduction syntax, and a proof using
    explicit contexts (contexts coloured {\color{red}red})}
  \label{fig:explicit-contexts}
\end{sidewaysfigure}

Explicit contexts can be seen as a mechanism for encoding a natural deduction
system as a sequent calculus.
However, the natural deduction character of the system is maintained by
ensuring that the resultant sequent calculus is really an encoding of a
natural deduction system.
Concretely, this means that rules can only interact with the context in
restricted ways:

\begin{itemize}
  \item There is a designated \emph{variable rule}, stating that any variable
    in the context can serve as a derivation of its type.
  \item Non-variable rules may only require subterms with \emph{extended}
    contexts, i.e., subterms in which new variables have been bound.
    Non-variable rules are parametric in the existing free variables.
\end{itemize}

Having chosen to use explicit contexts, the mechanisation must have a chosen
representation of contexts as a data structure.
While the notation in \cref{fig:explicit-contexts} uses names $f$ and $x$
for variables, I opt for a nameless representation.
In a nameless representation, variables are identified by their position in
the context, rather than by a name.
The absence of names means that $\alpha$-equivalence is just on-the-nose
equality, and also that we never have to reason about freshness of names.
Agda does not have support for nominal techniques~\cite{GP02}, which may have
made names a better option.

Most mechanisations choose contexts to be an inductive list of types.
However, I instead choose a functional, tree-shaped representation, as shown
with the type \AgdaRecord{Ctx}.
The type \AgdaDatatype{LTree} is the inductive type generated by leaves and
nullary \& binary nodes, and serves as a generalised ``length'' of the context.
The tree shape makes concatenation definitionally
injective, so that in cases where multiple new variables are bound in a subterm
(for example, $\otimes$-elimination), Agda's unification-based solving will
be more able to infer which variables have just been bound.
Within a given \AgdaBound{t}\AgdaSpace{}\AgdaSymbol:\AgdaSpace{}%
\AgdaDatatype{LTree}, we can define the positions of \AgdaBound{t} using
\AgdaDatatype{Ptr}.
A \emph{pointer} (\AgdaDatatype{Ptr}) into a tree picks out a leaf
(\AgdaInductiveConstructor{[-]}) following a path of lefts
(\AgdaInductiveConstructor{$\swarrow$}) and rights
(\AgdaInductiveConstructor{$\searrow$}) at any binary nodes encountered.

\ExecuteMetaData[\LTreetex]{LTree}
\ExecuteMetaData[\LTreetex]{Ptr}

The contents of the context --- the types --- are then stored in the functional
vector \AgdaField{ty-ctx}, which is a mapping from leaves in \AgdaField{shape}
to object language types \AgdaDatatype{Ty}.
The advantages of the functional vector representation will not become clear
until later chapters --- particularly the example in
\cref{sec:usage-elaborator}, where I make use of the ease of look-up and the
$\eta$-law of functions.
However, I claim for now that there is little to no disadvantage in the
functional vector representation --- in particular, we have no need for
function extensionality principles because we never talk about equality of
contexts.
For example, instead of using an equality of contexts to coerce a term, we can
use renaming.

\ExecuteMetaData[\Vectortex]{Vector}
\Ctx{}

Our first data structure involving contexts is that of intrinsically typed
variables.
A variable of type
\AgdaBound{$\Gamma$}\AgdaSpace{}\AgdaRecord{$\ni$}\AgdaSpace{}\AgdaBound{A}
is given by a path \AgdaField{idx} to a type in \AgdaBound{$\Gamma$}, together
with a proof \AgdaField{tyq} that this type is equal to \AgdaBound{A}.

\Var{}

Variables embed into terms via the \AgdaInductiveConstructor{var} constructor of
the family \AgdaDatatype{\_⊢\_} of intrinsically simply typed terms.
The only other syntactic forms we consider for now are the eliminator and
constructor of function types \AgdaInductiveConstructor{\_`→\_} ---
\AgdaInductiveConstructor{app} and \AgdaInductiveConstructor{lam}.
Application \AgdaInductiveConstructor{app} takes two subterms of the appropriate
types, while the subterm of $\lambda$-abstraction \AgdaInductiveConstructor{lam}
is in an extended context \GA{} --- \AgdaBound{$\Gamma$} concatenated with a
singleton context containing the type \AgdaBound{A}.

\Term{}

Using this encoding, the Church numeral for 2 appears as follows.
In standard notation, this would be
$\lambda f.~\lambda x.~f\,(f\,x)$.
To refer to $f$ in the main body of the expression, we skip one binder (using
\AgdaInductiveConstructor{$\swarrow$}) and pick the next one
(using \AgdaInductiveConstructor{$\searrow$}) and pick its only bound variable
(using \AgdaInductiveConstructor{here}).
To refer to $x$, we do not skip its binder, instead picking it and its only
bound variable.

\ExecuteMetaData[../agda/processed-latex/SimpleKits.tex]{two}

\section{Renaming and substitution}\label{sec:kits}
\def\SimpleKits{../agda/processed-latex/SimpleKits.tex}

%Explain:
%
%\begin{itemize}
%  \item Specific uses of renaming/substitution in $\lambda$-calculus semantics.
%  \item General role of renaming/substitution in abstract algebra/syntax with
%    binding.
%\end{itemize}

A basic operation on any syntax with variables is \emph{substitution} --- the
replacement of variables in a term by terms with the same type as the variables.
In a sense, this is the defining operation of variables --- a variable is a
placeholder for a term, or equivalently in logic, a hypothesis is a placeholder
for an arbitrary proof.
In a type theory or logic, terms can bind variables, and we will typically have
operational semantics rules combining a term binding a variable with a term that
is to be substituted into the place of that variable, like the $\beta$-rule for
$\lambda$-calculus functions.

While substitution has this extra role in a lot of the syntaxes with binding we
care about, variable-binding also significantly complicates the substitution
operation.
Substitution acts on the free variables of a term, replacing them by terms, but
binders mean that some subterms have \emph{more} free variables than our
original term.
This causes different challenges for different representations of terms.
For example, with named variables and shadowing, na\"{i}vely defined
substitution could fall foul of variable capture.
In our approach, based on de Bruijn indices, the difficulty is that an index $i$
outside a binder of $n$ variables corresponds to an index $n + i$ inside the
binder.
Therefore, when substituting under a binder, we must first increment any free
variables contained in terms we are substituting in, which is a form of
\emph{renaming}.
Renaming replaces each free variable by another free variable, and is a special
case of substitution.
We must, however, define renaming before substitution, so as to avoid the
definition of substitution being circular.
Renaming avoids a similar circularity because when renaming goes under a binder,
we only have to increment each variable being renamed in, rather than each
variable \emph{in each term} being substituted in.

In this section, I formally implement simultaneous renaming and substitution for
the terms defined in the previous section.
Simultaneous substitution turns out to have a simple definition, which
generalises into other algorithms over terms with binders.
The section concludes with a unified implementation of renaming and
substitution, leaving further generalisation to the next section.

\subsection{Simultaneous renaming and simultaneous substitution}

A simultaneous renaming from $\Gamma$ to $\Delta$ is a type-preserving map from
variables in $\Delta$ to \emph{variables} in $\Gamma$, while a simultaneous
substitution is a map into \emph{terms} in $\Gamma$.
While simultaneous substitution gives us a notion of one context being
\emph{derivable} from another, simultaneous renaming gives a similar notion
of derivability restricted to structural rules.

In the derivation below, we assume the existence of a derivation of
$B, C \to C \vdash C$, and by the admissibility of substitution we thus have a
derivation of $A \to B, A \vdash C$.
Intuitively, the context $A \to B, A$ derives the context $B, C \to C$, so
anything derived from $B, C \to C$ can also be derived from $A \to B, A$.
We see formally that $A \to B, A$ derives $B, C \to C$ by deriving each element
of the latter from the former --- hence the first two premises of the \TirName{Subst}
rule below, deriving $B$ and $C \to C$ from $A \to B, A$.

\begin{align*}
  &\begin{prooftree}
    \hypo{\Pi}
    \infer[no rule]1{A \to B, A \vdash B}
    \infer0[Var]{A \to B, A, C \vdash C}
    \infer1[$\to$-I]{A \to B, A \vdash C \to C}
    \hypo{B, C \to C \vdash C}
    \infer3[Subst]{A \to B, A \vdash C}
  \end{prooftree}
  \\\\
  &\textrm{where }\Pi \coloneqq
  \begin{prooftree}
    \infer0[Var]{A \to B, A \vdash A \to B}
    \infer0[Var]{A \to B, A \vdash A}
    \infer2[$\to$-E]{A \to B, A \vdash B}
  \end{prooftree}
\end{align*}

\subsection{Proofs of admissibility of renaming and substitution}

A renaming from \AgdaBound{$\Gamma$} to \AgdaBound{$\Delta$}
is a map from variables in \AgdaBound{$\Delta$} to variables in
\AgdaBound{$\Gamma$}, represented in Agda as follows.

\Ren{}

The action of a renaming \AgdaBound{$\rho$} on terms is given by
\AgdaFunction{ren}\AgdaSpace{}\AgdaBound{$\rho$}, with \AgdaFunction{ren}
defined below.
The idea of simultaneous renaming is to preserve the structure of the term, but
replace all of the variables from \AgdaBound{$\Delta$} by variables from
\AgdaBound{$\Gamma$}, with the mapping given by the renaming \AgdaBound{$\rho$}.

\rename{}

The \AgdaInductiveConstructor{var} case is where the action of the renaming
happens: the variable \AgdaBound{x} from \AgdaBound{$\Delta$} is mapped to the
variable \AgdaBound{$\rho$}\AgdaSpace{}\AgdaBound{x} from \AgdaBound{$\Gamma$}.
In the \AgdaInductiveConstructor{app} case, we have terms \AgdaBound{M}
\AgdaSymbol{:} \AgdaBound{$\Delta$} \AgdaDatatype{$\vdash$}
\AgdaBound{A} \AgdaInductiveConstructor{`$\to$} \AgdaBound{B} and \AgdaBound{N}
\AgdaSymbol{:} \AgdaBound{$\Delta$} \AgdaDatatype{$\vdash$} \AgdaBound{A}.
We may apply \AgdaFunction{ren} \AgdaBound{$\rho$} recursively to both
\AgdaBound{M} and \AgdaBound{N} to change their contexts from
\AgdaBound{$\Delta$} to \AgdaBound{$\Gamma$}, and the
\AgdaInductiveConstructor{app} constructor then produces the desired term in
\AgdaBound{$\Gamma$}.
Finally, in the \AgdaInductiveConstructor{lam} case, we get a term
\AgdaBound{M} \AgdaSymbol{:} \DA{} \AgdaDatatype{$\vdash$} \AgdaBound{B} and,
after introducing a \AgdaInductiveConstructor{lam} on the right, are in need
of a term of type \GA{} \AgdaDatatype{$\vdash$} \AgdaBound{B}.
To recursively apply \AgdaFunction{ren} to \AgdaBound{M}, we must thus extend
the renaming \AgdaBound{$\rho$} \AgdaSymbol{:} \RenGD{} with the newly bound
variable.
For this, we need an auxiliary function \AgdaFunction{bindRen} such that
\AgdaFunction{bindRen} \AgdaBound{$\rho$} \AgdaSymbol{:} \RenGADA{}.
This new renaming will act like \AgdaBound{$\rho$} for variables in
\AgdaBound{$\Delta$}, and map the new variable of type \AgdaBound{A} to the
corresponding new variable in \GA{}.

\bindRen{}

The \AgdaFunction{bindRen} given here has a slightly generalised type, where
instead of binding just a single variable of type \AgdaBound{A}, we could
bind a whole context \AgdaBound{$\Theta$} of new variables.
The first case of \AgdaFunction{bindRen} is for old variables from
\AgdaBound{$\Delta$}, where we apply \AgdaBound{$\rho$} to get a variable in
\AgdaBound{$\Gamma$}, and then use \AgdaFunction{$↙ᵛ$} to embed that variable
into \GTh{}.
The second case is for new variables from \AgdaBound{$\Theta$}, which embed
straight into \GTh{}.

Meanwhile, a substitution from \AgdaBound{$\Gamma$} to \AgdaBound{$\Delta$} is
an inhabitant of \AgdaFunction{Sub}\AgdaSpace{}\AgdaBound{$\Gamma$}\AgdaSpace{}%
\AgdaBound{$\Delta$}, as defined below.
This definition is identical to the definition of \AgdaFunction{Ren}, except
that it gives us \emph{terms} in \AgdaBound{$\Gamma$} rather than variables.

\Sub{}

The \AgdaFunction{sub} function below gives the action of a substitution.
Similarly to renaming, we want to preserve the structure of the term, except
now variables in the original term are replaced by \emph{terms} in the new
context.

\substitute{}

Given that this time, \AgdaBound{$\rho$} is a substitution rather than a
renaming, \AgdaBound{$\rho$} \AgdaBound{x} is a term, and is sufficient in the
\AgdaInductiveConstructor{var} case.
The \AgdaInductiveConstructor{app} case again deals with the subterms
recursively and then recombines them with \AgdaInductiveConstructor{app}.
In the \AgdaInductiveConstructor{lam} case, we again have a mismatch if we
want to apply \AgdaFunction{sub} recursively to the subterm \AgdaBound{M} with
an extra free variable.
We have \AgdaBound{$\rho$} \AgdaSymbol{:} \SubGD{} but need a substitution of
type \SubGADA{}, so we introduce the auxiliary definition
\AgdaFunction{bindSub}.

\bindSub{}

For the old variables in the first case, we have \AgdaBound{$\rho$} to turn
them into terms in \AgdaBound{$\Gamma$}.
Turning a term in \AgdaBound{$\Gamma$} into a term in \GTh{} requires a form
of weakening we have not yet proved, so I write \AgdaFunction{$↙ᵗ$} in analogy
with \AgdaFunction{$↙ᵛ$}, and prove it below.
In the second case, we want to substitute the new variable by the \emph{term}
referring to this new variable in \GTh{}.

The final piece to define substitution is to define the function that weakens
a term by some newly bound variables \AgdaBound{$\Delta$}.
For this, we use the action of renaming, which we have fully defined already,
and in particular rename each variable in the term from a variable in
\AgdaBound{$\Gamma$} to a variable in \GD{}.

\leftTerm{}

With this, the action of substitution is defined, and depends on the action
of renaming.

\subsection{Syntactic kits}\label{sec:syntactic-kits}

As observed by \citet{McBride05,BHKM12},
the statements of simultaneous renaming and simultaneous substitution are
very similar, with substitution being the generalisation that allows
replacement of variables by terms rather than just other variables.
Following \citet{McBride05},
I will introduce a type family \AgdaFunction{Env} of \emph{environments}, and
redefine \AgdaFunction{Ren} and \AgdaFunction{Sub} as environments of
variables and terms, respectively.

\Env{}
\RenSub{}

The processes I described for constructing proofs of the admissibility of
renaming and substitution were also similar.
Indeed, when we line up the resulting functions, \AgdaFunction{ren} and
\AgdaFunction{sub}, and their auxiliaries, \AgdaFunction{bindRen} and
\AgdaFunction{bindSub}, we notice only three key
differences:

\begin{itemize}
  \item In the first cases of \AgdaFunction{bindRen} and \AgdaFunction{bindSub},
    we do \AgdaFunction{$↙ᵛ$} and \AgdaFunction{$↙ᵗ$}, respectively, based on
    whether we are weakening a variable or a term.
  \item In the second case of \AgdaFunction{bindSub}, we do an extra wrapping of
    the new variable by \AgdaInductiveConstructor{var}, so as to make it a term
    to go in the substitution.
  \item In the \AgdaInductiveConstructor{var} case of \AgdaFunction{ren}, we
    have \AgdaInductiveConstructor{var} \AgdaSymbol(\AgdaBound{$\rho$}
    \AgdaBound{x}\AgdaSymbol) rather than just \AgdaBound{$\rho$} \AgdaBound{x},
    because the renaming \AgdaBound{$\rho$} gives us a variable rather than a
    term.
\end{itemize}

We may abstract over these three differences using the record \AgdaRecord{Kit}.
As in \AgdaFunction{Env}, we think of \AgdaBound{K} as being either
\AgdaDatatype{\_$\ni$\_} or \AgdaDatatype{\_$\vdash$\_}.
The fields of \AgdaRecord{Kit} are given in the same order as the points of
difference above.
Wherever the difference was presence or absence of
\AgdaInductiveConstructor{var}, we will be able to fill that field with either
\AgdaInductiveConstructor{var} or the identity function \AgdaFunction{id}.

\Kit{}

The field \AgdaField{$\swarrow^k$} can be seen as a property of the
judgement form \AgdaBound{K}, saying that it supports a form of weakening.
We use \AgdaField{vr} when adding a newly bound variable to an environment, and
use \AgdaField{tm} when we do a lookup from an environment and want to get a
term out.
Given a \AgdaRecord{Kit} \AgdaBound{K}, we can write the syntactic traversal
function \AgdaFunction{trav} and its auxiliary \AgdaFunction{bindEnv}, in the
model of \AgdaFunction{ren}, \AgdaFunction{sub}, and their auxiliaries.

\trav{}
\bindEnv{}

Concrete kits can be given for variables and terms either by inspecting
\AgdaFunction{ren} and \AgdaFunction{sub} or by following the types.
Notice that the kit for terms requires the admissibility of renaming so as to
achieve weakening of a substitution by newly bound variables.
Fortunately, this can be the \AgdaFunction{ren} defined below in terms of
\AgdaFunction{trav}, so we can keep \AgdaFunction{trav} as the only syntactic
traversal we have to write.

\begin{multicols}{2}
  \noindent\renKit{} \columnbreak

  \noindent\subKit{}
\end{multicols}

\section{Generic semantics}\label{sec:gen-sem}
The traversal \AgdaFunction{trav} from the last section is generic in the sense
that $\V$, the type of entries in an environment, can be instantiated to many
different things.
However, in practice we only use $\ni$ and $\vdash$, giving us renaming and
substitution, respectively.
This is because \AgdaFunction{trav} only targets terms, and does so by keeping
term constructors intact and replacing only the variables by things from the
environment.
This makes substitution the most general possible traversal.

If we want to capture a broader range of traversals, including not just
syntactic but also \emph{semantic} operations, we must be able to target things
other than terms, and act in an interesting way on term constructors.
Doing a straight generalisation of the type of \AgdaFunction{trav}, this
suggests that we want a function with the following type.

\missingfigure{Semantic traversal type signature}

\section{Generic syntax}\label{sec:gen-syn}
We have seen in previous sections a method for defining well typed terms,
providing them with the basic operations of renaming and substitution, and
defining type-preserving semantic traversals over those terms.
However, the Agda code we have seen only deals with one specific kind of terms
--- simply typed $\lambda$-calculus with a base type and function types.
The aim of this section is to write some code to which we can pass a
\emph{description} or \emph{signature} of a syntax and have it produce all of
the same machinery.

The description of a syntax will closely resemble the logical rules Gentzen gave
for natural deduction systems NJ and NK, but we give them a revised
interpretation.
Where Gentzen intended his rules to be applied schematically, and hypothetical
proofs to be handled via \emph{discharge} of hypotheses, we will take the rules
formally to produce a system with explicit contexts and a variable rule.
However, knowing that this resulting system came from such a description means
that we can derive variable-handling features, such as substitution, in a
generic way.

I will present a scheme based on the work of \citet{AACMM21} such that
\cref{fig:app-lam} is interpreted as the type
system we studied in the previous sections (simply typed $\lambda$-calculus with
a base type and function types).
Remember that, while these look like inference rules, I am treating them
entirely formally, collected together into a \emph{syntax description}.
The information presented in \cref{fig:app-lam} is essentially all of the
information needed for the type system sans any details about variables.
In particular, notice:
\begin{itemize}
  \item Contexts, in particular the context of a rule's conclusion, which is
    shared in all premises in the resulting type system, are elided.
    The only part of any context I record is the newly bound variables in
    premises, such as the variable bound by a $\lambda$-abstraction.
  \item There is no explicit variable rule.
    It is understood that any $x : A$ in the context of the resulting type
    system can be used to yield a term with type $A$.
\end{itemize}

\begin{figure}
  \begin{mathpar}
    \ebrule{%
      \hypo{\vdash A \to B}
      \hypo{\vdash A}
      \infer2[app]{\vdash B}
    }
    \and
    \ebrule{%
      \hypo{A \vdash B}
      \infer1[lam]{\vdash A \to B}
    }
  \end{mathpar}
  \caption{An example syntax description}
  \label{fig:app-lam}
\end{figure}

Such a scheme commits us to a certain approach to variable binding and context
management, but does not commit us to anything about the meaning of types.
For example, we do not declare that \TirName{app} and \TirName{lam} are
``elimination'' and ``introduction'' forms for the function type former.
This limits our generic results to matters of syntax and variables, but provides
a platform upon which a future semantic scheme could rest.

\begin{figure}
  \begin{align*}
    \text{Premises} && \mathit{ps}, \mathit{qs} &\Coloneqq
      \Delta \vdash A \mid {} \mid \mathit{ps} \quad \mathit{qs}
    \\
    \text{Rule} && r, s &\Coloneqq {\mathit{ps} \over \vdash A}
  \end{align*}
  \caption{The grammar of typing rules}
  \label{fig:simple-syntax}
\end{figure}

A syntax description is a set of fully instantiated rules.
In our running example, this set contains a \TirName{app}-rule and a
\TirName{lam}-rule for each pair of types $A$ and $B$.

To construct the syntax given by a description, we keep
\AgdaInductiveConstructor{var} as before, and have another constructor
\AgdaInductiveConstructor{con} for all of the logical rules.
\AgdaInductiveConstructor{con} takes a rule \AgdaBound{r} with premises
$\Delta_1 \vdash A_1; \ldots; \Delta_n \vdash A_n$ and conclusion $A$, and the
remainder of its type is as follows.
\[
  \AgdaInductiveConstructor{con}~\AgdaBound{r}~\AgdaSymbol{:}~%
  \forall\Gamma.~(\Gamma, \Delta_1 \vdash A_1) \times \cdots
  \times (\Gamma, \Delta_n \vdash A_n) \to \Gamma \vdash A
\]
Note that, in this type, $\vdash$ is the type family of terms we are
inductively constructing, as opposed to the description syntax found in the
premises.

In our generic version of \AgdaRecord{Semantics}, we keep the
\AgdaField{ren\textasciicircum$\V$} and \AgdaField{$\llbracket$var$\rrbracket$}
fields as before, and replace \AgdaField{$\llbracket$app$\rrbracket$} and
\AgdaField{$\llbracket$lam$\rrbracket$} by a
\AgdaField{$\llbracket$con$\rrbracket$} field as follows.
\[
  \AgdaField{$\llbracket$con$\rrbracket$}~\AgdaBound{r}~\AgdaSymbol{:}~%
  \forallb{%
    \Box\plr{{} \env\V \Delta_1 \dotto {} \sdtstile{}\C A_1} \dottimes
    \cdots \dottimes
    \Box\plr{{} \env\V \Delta_n \dotto {} \sdtstile{}\C A_n} \dotto
    {} \sdtstile{}\C A}
\]
I use ${} \sdtstile{}\C A$ for the Agda notation
\AgdaFunction{\_$\C\vDash$}\AgdaSpace{}\AgdaBound{A}, while ${} \env\V \Delta$
stands for the type family of environments
\AgdaSymbol{$\lambda$}\AgdaSpace{}\AgdaBound{$\Gamma$}\AgdaSpace{}%
\AgdaSymbol{$\to$}\AgdaSpace{}\AgdaFunction{Env}\AgdaSpace{}\AgdaBound{$\V$}%
\AgdaSpace{}\AgdaBound{$\Gamma$}\AgdaSpace{}\AgdaBound{$\Delta$}.
Environments appear in this definition simply as a way to write a product of
$\V$-values --- one value for each element of $\Delta$.
We could make a special case of premises which do not bind any variables, as did
\citet{AACMM21}, eliding the $\Box$ and empty environment, but I choose not to
for uniformity and simplicity of presentation.

To generate the expressions involving ellipses, I give an interpretation of the
formal rule descriptions.
The interpretation is parametrised on some
\ExecuteMetaData[\SimpleSyntaxtex]{WithScope}, where, in
\AgdaBound{,}\AgdaSpace{}\AgdaBound{$\Delta$}\AgdaSpace{}%
\AgdaBound{$\llbracket$}\AgdaSpace{}\AgdaBound{$\Gamma$}\AgdaSpace{}%
\AgdaBound{$\vdash$}\AgdaSpace{}\AgdaBound{A}\AgdaSpace{}%
\AgdaBound{$\rrbracket$}, the context \AgdaBound{$\Delta$} stands for the newly
bound variables of a premise, \AgdaBound{$\Gamma$} is the context as it was
below the rule's horizontal line, and \AgdaBound{A} is the type of the premise.
In the \AgdaInductiveConstructor{con} constructor for terms, the parameter is
$\Gamma, \Delta \vdash A$, and in the \AgdaField{$\llbracket$con$\rrbracket$}
field for semantics, the parameter is
$\Box\plr{{} \env\V \Delta \dotto {} \sdtstile{}\C A}\,\Gamma$.

A single premise with newly bound variables is interpreted by shuffling the
parts into the right place, while multiple premises are interpreted as pointwise
products of the individual premises (giving the ellipses above).

\ExecuteMetaData[\SimpleSyntaxtex]{semp}

A rule, with all its parameters instantiated, targets a specific type
\AgdaBound{A$'$}, which we check to match the desired type \AgdaBound{A}.
Finally, a whole \AgdaRecord{System} comprises a set \AgdaBound{L} of rule
labels, and \AgdaBound{rs}\AgdaSpace{}\AgdaSymbol{:}\AgdaSpace{}\AgdaBound{L}%
\AgdaSpace{}\AgdaSymbol{$\to$}\AgdaSpace{}\AgdaRecord{Rule}.
The interpretation of these data is to pick a rule label \AgdaBound{l}, and then
take the interpretation of the rule \AgdaBound{rs}\AgdaSpace{}\AgdaBound{l}.
For the sake of defining terms as a least fixed point, it is important to note
that the interpretation of a syntax description is strictly positive in the
parameter \AgdaBound{,\_$\llbracket$\_$\vdash$\_$\rrbracket$}.

\ExecuteMetaData[\SimpleSyntaxtex]{semr}
\ExecuteMetaData[\SimpleSyntaxtex]{sems}

The interpretation of a system description as a single layer of syntax is
functorial, supporting the \AgdaFunction{map-s} function when the parameter
\AgdaBound{,\_$\llbracket$\_$\vdash$\_$\rrbracket$} is given as an extra
argument named \AgdaBound{X} or \AgdaBound{Y} (which are both fixed as
parameters of \AgdaFunction{map-s}, together with \AgdaBound{$\Gamma$} and
\AgdaBound{$\Delta$}).

\ExecuteMetaData[\SimpleSyntaxtex]{map-s-type}

The implementation of \AgdaFunction{map-s} is straightforward, so I do not list
it here.
We preserve
the shape of the syntactic layer, applying the function to each \AgdaBound{X}
we find (wherever the description contains \AgdaInductiveConstructor{$\langle$}%
\AgdaSpace{}\AgdaBound{$\Delta$}\AgdaSpace{}%
\AgdaInductiveConstructor{`$\vdash$}\AgdaSpace{}\AgdaBound{A}\AgdaSpace{}%
\AgdaInductiveConstructor{$\rangle$}).

This \AgdaFunction{map-s} will be used in the generic syntax version of
\AgdaFunction{sem} to recursively apply \AgdaFunction{sem} to all subterms.
However, a major distinction between generic syntax and the specific syntax of
previous sections is that the subterms found by \AgdaFunction{map-s} are not
recognised by Agda's termination checker as \emph{structurally smaller} than
the original term.
Therefore, a na\"{i}vely written \AgdaFunction{sem} will fail Agda's termination
check.

To make \AgdaFunction{sem} pass the termination check, we have four major
options:
\begin{enumerate}
  \item Assert \AgdaFunction{sem} to be terminating, bypassing the termination
    check.
  \item Use Agda's \emph{sized types} to remember that the subterms are smaller.
  \item Avoid sized types, and index terms over some user-defined type (for
    example, natural numbers or ordinal notations) which is structurally smaller
    at subterms.
  \item Inline a new, instantiated version of \AgdaFunction{map-s} wherever it
    is used.
\end{enumerate}

Each of these approaches has drawbacks.
Approach 1 is clearly unsafe, in the sense that the fundamental lemma
\AgdaFunction{sem} is not being completely checked for type-correctness.
Approach 2 is also unsafe, because Agda's sized type implementation is known to
make the system inconsistent~\citep{AgdaIssue1201}.
Meanwhile, approach 3 is safe, but entails a lot of manually extracted and
supplied extra arguments, which I think would distract from the presentation and
make the resulting code harder to use.
Finally, approach 4 is safe, but limits code reuse (both of the function
\AgdaFunction{map-s} itself and any lemmas we may prove about it).
I choose to follow \citet{AACMM21} in using sized types, justified by the idea
that Agda may eventually have a sound implementation of sized types, at which
point I would want my code to be as easy to update for that new version of Agda
as possible.
\Citet{FS22} use approach 4, and in fact have only one use of (their equivalent
of) \AgdaFunction{map-s} in their development.

Using sized types, my type family of \AgdaRecord{System}-generic terms is as
below.
\AgdaFunction{Scope} is a name for the transformation of an
\AgdaFunction{OpenFam} into a \AgdaRecord{Ctx}\AgdaSpace{}\AgdaSymbol{$\to$}%
\AgdaSpace{}\AgdaFunction{OpenFam} which appends the extra context to the
existing context before applying the original \AgdaFunction{OpenFam}
(in this case, producing something like $\Gamma, \Delta \vdash A$ from
$\vdash$).
The Agda builtin \AgdaPostulate{$\uparrow$} produces a bigger size from an
existing size \AgdaBound{sz}, giving us here that the size of a term is 1 bigger
than the size of all of its immediate subterms.
Agda's elaborator and termination checker have special support for sizes, so we
do not have to worry much about them from this point on.

\ExecuteMetaData[\SimpleTermtex]{Tm}

Corresponding to the generic \AgdaInductiveConstructor{`con} constructor for
terms, we have a generic field \AgdaField{$\llbracket$con$\rrbracket$} in the
updated \AgdaRecord{Semantics} record.
In place of where one might expect
\AgdaFunction{Scope}\AgdaSpace{}\AgdaBound{$\C$}, we instead have
\AgdaFunction{Kripke}\AgdaSpace{}\AgdaBound{$\V$}\AgdaSpace{}\AgdaBound{$\C$},
with \AgdaFunction{Kripke} defined below.
The form of \AgdaFunction{Kripke} follows from the shape we saw in the type of
the \AgdaField{$\llbracket$lam$\rrbracket$} field we saw in \cref{sec:gen-sem},
where I use an environment targeting \AgdaBound{$\Delta$} as a way to say
``a value for each type in \AgdaBound{$\Delta$}''.

\ExecuteMetaData[\SimpleSemanticstex]{Kripke}
\ExecuteMetaData[\SimpleSemanticstex]{Semantics}

Finally, we get a generic semantic traversal as follows.
The function \AgdaFunction{bindEnv} is unchanged from \cref{sec:gen-sem}, as it
never mentions the syntax.
The type of the traversal \AgdaFunction{sem} is also basically unchanged --- we
just need to account for arbitrary term sizes (\AgdaBound{sz}), which will get
smaller when recursing on subterms.
I have chosen, as did \citet{AACMM21}, to define \AgdaFunction{sem} mutually
with a function \AgdaFunction{body}, which is like a counterpart to
\AgdaFunction{sem} dealing with newly bound variables.
Note that the mutual recursion is not essential --- for example,
\AgdaFunction{body} could simply be inlined.
The \AgdaInductiveConstructor{`var} case of \AgdaFunction{sem} is as before.
The \AgdaInductiveConstructor{`con} case, if viewed appropriately, is a direct
generalisation of the \AgdaInductiveConstructor{lam} case from earlier.
We recursively apply \AgdaFunction{sem} to all immediate subterms contained in
\AgdaBound{M} (as found by \AgdaFunction{map-s}), with an environment updated to
reflect the newly bound variables in each premise of the rule that was applied.

\ExecuteMetaData[\SimpleSemanticstex]{sem}

%\section{Natural deduction vs sequent calculus}
%In a seminal paper~\cite{Gentzen64}, Gerhard Gentzen introduces two syntactic
paradigms which remain among the most studied to this day.
These paradigms are \emph{natural deduction} and \emph{sequent calculus}, as
exemplified by natural deduction calculi NJ and NK, and sequent calculi LJ and
LK.
Restricting attention to just the intuitionistic systems NJ and LJ, these
actually differ in two orthogonal ways, which I shall prise apart in this
section.
The simpler distinction is that the logical rules in NJ are introduction and
elimination rules, whereas the logical rules in LJ are left and right rules.
But the more important distinction for this thesis is that where NJ has
assumptions, LJ has sequents explictly manipulated by structural rules.
I take the latter distinction to define natural deduction and sequent calculus,
and wherever I need to make the former distinction I shall speak of
\emph{intro-elim systems} and \emph{left-right systems}.

I will use the rest of this section as follows.
First, I introduce NJ and LJ, and some of their basic metatheory.
Then, I will give examples of systems intermediate between Gentzen's natural
deduction and sequent calculi: a left-right natural deduction calculus
$\mu\tilde\mu$, and an intro-elim sequent calculus BBdPH\@.
Both of these examples will reappear in later chapters.

\subsection{Intro-elim natural deduction: NJ}

\subsection{Left-right sequent calculus: LJ}

\subsection{Left-right natural deduction: $\mu\tilde\mu$}
The $\mu\tilde\mu$-calculus~\cite{CH00} (also known as
$\overline\lambda\mu\tilde\mu$ or system L, and closely related to Wadler's
dual calculus~\cite{Wadler03}) can be seen as an adaptation of natural deduction
to classical logic.
Though originally presented as a sequent calculus, the underlying natural
deduction calculus was later given by Herbelin~\cite[p.\ 12]{Herbelin-hab}, and
I will follow the latter.
While Gentzen gave a natural deduction calculus NK for classical logic, NK
relies on adding the \emph{axiom} of excluded middle.
As axioms are not systematic parts of the calculus, they can hinder or break
metatheoretic properties like normalisation.
In contrast, the $\mu\tilde\mu$-calculus allows us to \emph{derive} excluded
middle from entirely systematic components.

In NJ, a derivation of $A$ from assumptions $\Gamma$ tells us that if each
formula in $\Gamma$ is \emph{true}, then $A$ is also \emph{true}.
The $\mu\tilde\mu$-calculus generalises the picture by allowing us to have
both \emph{true} and \emph{false} assumptions, and allowing us to conclude that
some $A$ is \emph{true}, that some $A$ is \emph{false}, or that we have reached
a contradiction.
% A similar judgement of contradiction appears in Prawitz' classical natural
% deduction calculus~\cite{Prawitz65}.
Following Herbelin, we notate the judgement that $A$ is true as ${}\vdash A$,
that $A$ is false as $A \vdash{}$, and of contradiction as $\vdash$.
The only way to derive a contradiction is to find some $A$ such that
${}\vdash A$ and $A \vdash{}$.
Meanwhile, we can derive ${}\vdash A$ by assuming $A \vdash{}$ and deriving
$\vdash$, i.e., we can prove $A$ by assuming that $A$ is false and deriving a
contradiction.
Dually, we can derive $A \vdash{}$ by assuming ${}\vdash A$ and deriving
$\vdash$.
These three methods of derivation are encoded in the following rules.

\begin{mathpar}
  \inferrule*[right=Cut]
  {{}\vdash A \\ A \vdash{}}
  {\vdash}

  \and

  \inferrule*[right=$\mu$]
  {
    [A \vdash{}] \\\\ \vdots \\\\ \vdash
    %\inferrule*[fraction={~~~}]
    %{[A \vdash{}] \\\\ \vdots}
    %{\vdash}
  }
  {{}\vdash A}

  \and

  \inferrule*[right=$\tilde\mu$]
  {[{}\vdash A] \\\\ \vdots \\\\ \vdash}
  {A \vdash{}}
\end{mathpar}

The rules for logical connectives describe how to \emph{prove} and how to
\emph{refute} a formula whose principal connective is that connective.
These correspond strongly with the right and left rules, respectively, of LJ,
and for this reason, $\mu\tilde\mu$ is usually described elsewhere as a
sequent calculus.
For example, we could choose the following rules for disjunction.
To prove $A \vee B$, we can assume that both $A$ and $B$ are false, and derive
a contradiction.
To refute $A \vee B$, we can refute $A$ and $B$ separately.

\begin{mathpar}
  \inferrule*[right=$\vee$-r]
  {[A \vdash{}][B \vdash{}] \\\\ \vdots \\\\ \vdash}
  {{}\vdash A \vee B}

  \and

  \inferrule*[right=$\vee$-l]
  {A \vdash{} \\ B \vdash{}}
  {A \vee B \vdash{}}
\end{mathpar}

\subsection{Intro-elim sequent calculus: BBdPH}

Term assignment system introduced in~\cite{BBdPH93}.

\section{Related work}\label{sec:mech-related}
There is a vast literature on formalisations of syntaxes with binding, which I
cannot possibly do justice to in a reasonably sized thesis chapter.
Instead, I limit myself to comparisons of the \citet{AACMM21} method I follow in
this thesis to just its closest related work.

\subsection{Autosubst}

\citet{Autosubst15} present the system \emph{Autosubst}, which provides various
tools for working with syntaxes with binding in the Coq proof assistant.
Autosubst is based on similar ideas to those \citeauthor{AACMM21} use:
de Bruijn-indexed terms with a distinguished variable rule and notion of
binding, acted upon by simultaneous renaming and substitution.

The simplest differences are essentially matters of choosing the encoding that
best fits the proof assistant being used.
Coq users tend to prefer using unindexed types and propositions indexed over
them --- in this case, a type of unscoped and untyped terms plus a
``well typed'' predicate --- whereas Agda users prefer to work with only well
formed data (well scoped and well typed terms).
The latter approach more readily allows us to show generically that substitution
preserves scoping and typing, but the former approach, conversely, allows for
bespoke proofs of such facts.
For example, one theorem of \citet{Autosubst15} is type preservation for
$\mathrm{CC}_\omega$, a dependent type system we cannot express using the
machinery of \citet{AACMM21}.
In principle, one could use \citeauthor{AACMM21}'s machinery as the basis of a
similar bespoke proof, but as far as I am aware, this has not been tried.

Another main difference is that Autosubst is presented to the user largely as
a black-box implementation of substitution and related lemmas, in contrast to
\citeauthor{AACMM21}'s work exposing the \AgdaRecord{Semantics} bundle to the
user, and having substitution be just one instance.
\citet{ACMM17} and \citet{AACMM21} provide many examples of traversals over
syntax using the same generic environment management as used by substitution.
However, the focus on substitution in Autosubst has meant that reasoning about
substitutions has been given more developed support.
For example, the library provides a tactic \texttt{autosubst} which automates
many equational proofs involving substitutions based on the $\sigma$-calculus of
\citet{ACCL91}.

An interesting feature of Autosubst is \emph{heterogeneous substitution}.
The motivation for heterogeneous substitution is to handle systems like system
F, where types and terms are syntactically distinct, but both feature binding
and require a substitution operation.
Furthermore, binding and substitution of types also affects the syntax of terms,
thanks to $\Lambda$ terms.
\citeauthor{AACMM21} provide no direct equivalent to heterogeneous substitution,
and it is unclear how well their work can handle polymorphic calculi.

\citet{Autosubst18} propose some modifications to Autosubst which, as far as I
can tell, have not yet been incorporated, but are presented in mechanised form
for the paper.
Some of these modifications aim to bring Autosubst into line with
\citeauthor{AACMM21}'s work, in particular taking semantic traversals as a
theoretical basis.
TODO: more.

\subsection{Fiore, Plotkin, Turi}

TODO: cite \citet{FPT99} and \citet{FS22}.

