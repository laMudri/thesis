\chapter{Mechanisation of simple types}\label{sec:simple}

In this chapter, I review and justify the family of approaches usually used to
represent simple type systems inside dependently typed proof assistants.
These approaches were first presented by \citet{AR99}, who showed a way of
representing well \emph{scoped} terms in a language with polymorphic recursion,
and extended the representation to well \emph{typed} terms in a language with
dependent types.
The representation of terms relies on indexing on both a context --- giving the
types of all the free variables --- and a type for the term itself.
A basic operation on terms is \emph{simultaneous substitution}, which replaces
each variable in the context by a term in another context.
This simultaneous substitution operation fits the form of composition in a
multicategory.
Specifically, in the case of a simple intuitionistic type system, this is the
composition of a Cartesian multicategory, which allows all of the terms being
substituted in to share a context.

\section{Term representation}
We could mechanise Gentzen's definition of a natural deduction system directly,
but this definition is quite complicated.
In particular, if we want to give derivations an inductive definition, the use
of the discharge mechanism means that we actually need an inductive-inductive
type --- derivations, particularly those using $\to$-introduction, can involve
references to assumptions within their subderivations.
An inductive-inductive definition of derivations would complicate our programs
and proofs about natural deduction derivations, so I choose an alternative
representation.

Indeed, most authors since Gentzen, whether mechanising their work or not,
have opted to replace discharge of assumptions by explicit \emph{contexts} and
a variable rule.
Contexts can be justified as a way to keep track of undischarged assumptions.
In particular, we only produce derivations in the presence of a known collection
of \emph{free variables} specified by the context.
In other words, derivations are \emph{indexed} over their free variables and
their types.
When using an assumption within a derivation, we must say which free variable
it corresponds to.
Free variables are introduced by \emph{variable-binding} rules, like
$\to$-introduction.
\cref{fig:explicit-contexts} gives an example of the same derivation written
in Gentzen's style and in the explicit context style.

\begin{sidewaysfigure}
  \centering
  \begin{prooftree}
    \hypo{[A \to A \to B]^f}
    \hypo{[A]^x}
    \infer2[$\to$-E]{A \to B}
    \hypo{[A]^x}
    \infer2[$\to$-E]{B}
    \infer1[$\to$-I$^x$]{A \to B}
    \infer1[$\to$-I$^f$]{\plr{A \to A \to B} \to \plr{A \to B}}
  \end{prooftree}

  \vspace{2em}

  \begin{prooftree}
    \infer0[var$^f$]{{\color{red}f : A \to A \to B, x : A} \vdash A \to A \to B}
    \infer0[var$^x$]{{\color{red}f : A \to A \to B, x : A} \vdash A}
    \infer2[$\to$-E]{{\color{red}f : A \to A \to B, x : A} \vdash A \to B}
    \infer0[var$^x$]{{\color{red}f : A \to A \to B, x : A} \vdash A}
    \infer2[$\to$-E]{{\color{red}f : A \to A \to B, x : A} \vdash B}
    \infer1[$\to$-I$^x$]{{\color{red}f : A \to A \to B} \vdash A \to B}
    \infer1[$\to$-I$^f$]{\vdash \plr{A \to A \to B} \to \plr{A \to B}}
  \end{prooftree}
  \caption{A proof in Gentzen's natural deduction syntax, and a proof using
    explicit contexts (contexts coloured {\color{red}red})}
  \label{fig:explicit-contexts}
\end{sidewaysfigure}

Explicit contexts can be seen as a mechanism for encoding a natural deduction
system as a sequent calculus.
However, the natural deduction character of the system is maintained by
ensuring that the resultant sequent calculus is really an encoding of a
natural deduction system.
Concretely, this means that rules can only interact with the context in
restricted ways:

\begin{itemize}
  \item There is a designated \emph{variable rule}, stating that any variable
    in the context can serve as a derivation of its type.
  \item Non-variable rules may only require subterms with \emph{extended}
    contexts, i.e., subterms in which new variables have been bound.
    Non-variable rules are parametric in the existing free variables.
\end{itemize}

Having chosen to use explicit contexts, the mechanisation must have a chosen
representation of contexts as a data structure.
While the notation in \cref{fig:explicit-contexts} uses names $f$ and $x$
for variables, I opt for a nameless representation.
In a nameless representation, variables are identified by their position in
the context, rather than by a name.
The absence of names means that $\alpha$-equivalence is just on-the-nose
equality, and also that we never have to reason about freshness of names.
Agda does not have support for nominal techniques~\cite{GP02}, which may have
made names a better option.

Most mechanisations choose contexts to be an inductive list of types.
However, I instead choose a functional, tree-shaped representation, as shown
with the type \AgdaRecord{Ctx}.
The type \AgdaDatatype{LTree} is the inductive type generated by leaves and
nullary \& binary nodes, and serves as a generalised ``length'' of the context.
The contents of the context --- the types --- are then stored in the functional
vector \AgdaField{ty-ctx}, which is a mapping from leaves in \AgdaField{shape}
to object language types \AgdaDatatype{Ty}.
The advantages of the functional vector representation will not become clear
until later chapters\todo{forward reference}, where I make use of the ease of
look-up and $\eta$-law of functions.
However, I claim for now that there is little to no disadvantage in the
functional vector representation --- in particular, we have no need for
function extensionality principles because we never talk about equality of
contexts.
For example, instead of using an equality of contexts to coerce a term, we can
use renaming.
As for the tree shape, this makes context concatenation definitionally
injective, so that in cases where multiple new variables are bound in a subterm
(for example, $\otimes$-elimination), Agda's unification-based solving will
be more able to infer which variables have just been bound.

\ExecuteMetaData[\LTreetex]{LTree}
\Ctx{}

Our first data structure involving contexts is that of intrinsically typed
variables.
A variable of type
\AgdaBound{$\Gamma$}\AgdaSpace{}\AgdaRecord{$\ni$}\AgdaSpace{}\AgdaBound{A}
is given by a path \AgdaField{idx} to a type in \AgdaBound{$\Gamma$}, together
with a proof \AgdaField{tyq} that this type is equal to \AgdaBound{A}.
Variables embed into terms via the \AgdaInductiveConstructor{var} constructor of
the family \AgdaDatatype{\_⊢\_} of intrinsically simply typed terms.
The only other syntactic forms we consider for now are the eliminator and
constructor of function types \AgdaInductiveConstructor{\_`→\_} ---
\AgdaInductiveConstructor{app} and \AgdaInductiveConstructor{lam}.
Application \AgdaInductiveConstructor{app} takes two subterms of the appropriate
types, while the subterm of $\lambda$-abstraction \AgdaInductiveConstructor{lam}
is in an extended context \GA{} --- \AgdaBound{$\Gamma$} concatenated with a
singleton context containing the type \AgdaBound{A}.

\Var{}
\Term{}

\section{Renaming and substitution}\label{sec:kits}
\def\prefix{../agda/latex}
\CatchFileBetweenTags{\Var}{\prefix/SimpleKits.tex}{Var}
\CatchFileBetweenTags{\Term}{\prefix/SimpleKits.tex}{Term}
\CatchFileBetweenTags{\Ren}{\prefix/SimpleKits.tex}{Ren}
\CatchFileBetweenTags{\bindRen}{\prefix/SimpleKits.tex}{bindRen}
\CatchFileBetweenTags{\rename}{\prefix/SimpleKits.tex}{rename}
\CatchFileBetweenTags{\leftTerm}{\prefix/SimpleKits.tex}{leftTerm}
\CatchFileBetweenTags{\Sub}{\prefix/SimpleKits.tex}{Sub}
\CatchFileBetweenTags{\bindSub}{\prefix/SimpleKits.tex}{bindSub}
\CatchFileBetweenTags{\substitute}{\prefix/SimpleKits.tex}{substitute}
\CatchFileBetweenTags{\Env}{\prefix/SimpleKits.tex}{Env}
\CatchFileBetweenTags{\RenSub}{\prefix/SimpleKits.tex}{RenSub}
\CatchFileBetweenTags{\Kit}{\prefix/SimpleKits.tex}{Kit}
\CatchFileBetweenTags{\trav}{\prefix/SimpleKits.tex}{trav}
\CatchFileBetweenTags{\renKit}{\prefix/SimpleKits.tex}{renKit}
\CatchFileBetweenTags{\subKit}{\prefix/SimpleKits.tex}{subKit}

\CatchFileBetweenTags{\RenGD}{\prefix/SimpleKits.tex}{RenGD}
\CatchFileBetweenTags{\RenGADA}{\prefix/SimpleKits.tex}{RenGADA}
\CatchFileBetweenTags{\GTh}{\prefix/SimpleKits.tex}{GTh}
\CatchFileBetweenTags{\DTh}{\prefix/SimpleKits.tex}{DTh}
\CatchFileBetweenTags{\GD}{\prefix/SimpleKits.tex}{GD}

\subsection{Simultaneous renaming and simultaneous substitution}

A simultaneous renaming from $\Gamma$ to $\Delta$ is a type-preserving map from
variables in $\Delta$ to \emph{variables} in $\Gamma$, while a simultaneous
substitution is a map into \emph{terms} in $\Gamma$.
While simultaneous substitution gives us a notion of one context being
\emph{derivable} from another, simultaneous renaming gives a similar notion
of derivability restricted to structural rules.

\begin{mathpar}
  \inferrule*[right=Subst]
  {%
    \inferrule*[right=$\to$-E]
    {%
      \inferrule*[right=Var]{~}{A \to B, A \vdash A \to B}
      \\
      \inferrule*[Right=Var]{~}{A \to B, A \vdash A}
    }
    {A \to B, A \vdash B}
    \\
    B \vdash C
  }
  {A \to B, A \vdash C}
\end{mathpar}

\begin{displaymath}
  \begin{prooftree}
    \infer0[Var]{A \to B, A \vdash A \to B}
    \infer0[Var]{A \to B, A \vdash A}
    \infer2[$\to$-E]{A \to B, A \vdash B}
    \hypo{B \vdash C}
    \infer2[Subst]{A \to B, A \vdash C}
  \end{prooftree}
\end{displaymath}

\subsection{Proofs of renaming and substitution}

We start with a data type \AgdaDatatype{\_⊢\_} of intrinsically simply typed
terms.
Beside base types, the only type former we have is the function type constructor
\AgdaInductiveConstructor{\_`→\_}.
Contexts (of type \AgdaRecord{Ctx}) are implemented as the free magma on types
(\AgdaDatatype{Ty}).
Context concatenation is \AgdaFunction{\_++ᶜ\_}, and \AgdaFunction{[\_]ᶜ}
embeds types into contexts.
Typed variables in a context are given by \AgdaRecord{\_∋\_}.
A variable in
\AgdaBound{$\Gamma$}\AgdaSpace{}\AgdaRecord{$\ni$}\AgdaSpace{}\AgdaBound{A}
is given by a path \AgdaField{idx} to a type in \AgdaBound{$\Gamma$}, together
with a proof \AgdaField{tyq} that this type is equal to \AgdaBound{A}.
Variables embed into terms via the \AgdaInductiveConstructor{var} constructor.

\Var{}
\Term{}

For this syntax, a renaming from \AgdaBound{$\Gamma$} to \AgdaBound{$\Delta$}
is a map from variables in \AgdaBound{$\Delta$} to variables in
\AgdaBound{$\Gamma$}.
Substitutions instead map into terms in \AgdaBound{$\Gamma$}.

\Ren{}
\Sub{}

In the following, \AgdaFunction{ren} gives the action of a renaming on terms.
We replace variables by new variables given by the renaming \AgdaBound{$\rho$}.
For applications, we rename each subterm using the same renaming.
For $\lambda$-abstractions, we want to rename the body, but the renaming we
have has type \RenGD{}, whereas the renaming we need is of type \RenGADA{}.
To make up the difference, we introduce the \AgdaFunction{bindRen} lemma,
saying that any renaming can be extended to the right by a context
\AgdaBound{$\Theta$}~\footnote{%
  We only require extension to the right because our syntax only has binding
  on the right.
  We could also extend on the left.
}.
To produce such an extended renaming, we receive a variable in \DTh{}, with the
two cases being that this variable is either in \AgdaBound{$\Delta$} or
\AgdaBound{$\Theta$}.
In the first case, we can use the original renaming \AgdaBound{$\rho$} to get
a variable in \AgdaBound{$\Gamma$}, which can be extended in the obvious way
to a variable in \GTh{}.
In the second case, we have a variable in \AgdaBound{$\Theta$}, and just need
to extend it to be a variable in \GTh{}.

\bindRen{}
\rename{}

We use renaming to show that, like variables, we can extend the context of terms
to the right.
The operation below renames a term by replacing all variables with variables
pointing to the left of the term's new context \GD{}.

\leftTerm{}

The action of a substitution on a term is given below.
The structure is very similar to that of renaming.
The differences are the following:

\begin{itemize}
  \item A substitution gives us terms, rather than variables.
        This means that when we use the substitution on a variable, we already
        have a term, and don't need to do any embedding into terms.
  \item When dealing with newly bound variables in \AgdaFunction{bindSub}, we
        need to produce terms, so we embed the newly bound variables into terms
        using \AgdaInductiveConstructor{var}.
  \item We use \AgdaFunction{$↙ᵗ$} instead of \AgdaFunction{$↙ᵛ$}, because we
        are dealing with terms rather than variables.
\end{itemize}

\bindSub{}
\substitute{}

\subsection{Kits}

To abstract over the similarities between renaming and substitution, we can use
\emph{kits} as introduced by McBride~\cite{McBride05,BHKM12}.
Each of the three differences above is turned into a parameter of
\AgdaFunction{trav} (generalising \AgdaFunction{ren} and \AgdaFunction{sub})
and \AgdaFunction{bindEnv} (generalising \AgdaFunction{bindRen} and
\AgdaFunction{bindSub}).
In the types, \AgdaFunction{Ren} and \AgdaFunction{Sub} are generalised by
\AgdaFunction{Env}\AgdaSpace{}\AgdaBound{K} --- a function from variables in
\AgdaBound{$\Delta$} to \AgdaBound{K}-things in \AgdaBound{$\Gamma$}.

\Env{}

The function \AgdaFunction{trav} produces a term-to-term mapping based on an
environment \AgdaBound{$\rho$}.
Like renaming and substitution, the traversal \AgdaFunction{trav} replaces
variables according to \AgdaBound{$\rho$}, while keeping the rest of the
syntactic forms intact.
The three differences between renaming and substitution present themselves as
requirements of the notion of \emph{kit} we can choose:

\begin{itemize}
  \item In the \AgdaInductiveConstructor{var} case of \AgdaFunction{trav}, we
        apply the environment \AgdaBound{$\rho$} to get a \AgdaBound{K}-thing,
        and need something to turn this \AgdaBound{K}-thing into a term.
        We let \AgdaField{tm} be this context- and type-preserving map from
        \AgdaBound{K}-things to terms.
  \item When dealing with newly bound variables in \AgdaFunction{bindEnv}, we
        need to convert the new variable into a \AgdaBound{K}-thing in order to
        put it into the environment.
        We let \AgdaField{vr} be this context- and type-preserving map from
        variables to \AgdaBound{K}-things.
  \item When working out where to map old variables in an extended context, we
        need \AgdaBound{K}-things to be stable under context extensions.
        We let \AgdaField{↙ᵏ} be the function embedding a \AgdaBound{K}-thing
        into an extended context.
\end{itemize}

\trav{}

The three parameters can be given to these functions by filling the fields of
the \AgdaRecord{Kit} record.

\Kit{}

We may now redefine the types \AgdaFunction{Ren} and \AgdaFunction{Sub} via
\AgdaFunction{Env}, and derive the actions of renaming and substitution from
\AgdaFunction{trav}.
Notice that \AgdaFunction{↙ᵗ} (written out as
\AgdaFunction{ren}\AgdaSpace{}\AgdaFunction{↙ᵛ}) still relies on renaming, but
because \AgdaFunction{ren} is only being used to fill a parameter of
\AgdaFunction{trav}, \AgdaFunction{trav} itself can be used to define
\AgdaFunction{ren} in a non-circular way.
Thus, we have succeeded in avoiding the duplication of code between
\AgdaFunction{ren} and \AgdaFunction{sub}.
%\todo{Distribute code horizontally}

\begin{multicols}{3}
  \noindent\RenSub{} \columnbreak

  \noindent\renKit{} \columnbreak

  \noindent\subKit{}
\end{multicols}

%\noindent
%\begin{tabular}{l|l|l}
%  \begin{minipage}{.25\textwidth}
%    \RenSub{}
%  \end{minipage}
%  &
%  \begin{minipage}{.25\textwidth}
%    \renKit{}
%  \end{minipage}
%  &
%  \begin{minipage}{.25\textwidth}
%    \subKit{}
%  \end{minipage}
%\end{tabular}

\section{Generic semantics}
\def\SimpleSem{../agda/latex/SimpleSem.tex}

The traversal \AgdaFunction{trav} from the last section is generic in the sense
that $\V$, the type of entries in an environment, can be instantiated to many
different things.
However, in practice we only use $\ni$ and $\vdash$, giving us renaming and
substitution, respectively.
This is because \AgdaFunction{trav} only targets terms, and does so by keeping
term constructors intact and replacing only the variables by things from the
environment.
This makes substitution the most general possible traversal.

If we want to capture a broader range of traversals, including not just
syntactic but also \emph{semantic} operations, we must be able to target things
other than terms, and act in an interesting way on term constructors.
Doing a straight generalisation of the type of \AgdaFunction{trav}, this
suggests that we want a function with the following type, where
\AgdaBound{$\C$} is the type family we are targeting.

\ExecuteMetaData[\SimpleSem]{semType}

Following the implementation of \AgdaFunction{trav}, we see that
\AgdaBound{$\C$} will need to support a semantic counterpart of each syntactic
form (\AgdaInductiveConstructor{var}, \AgdaInductiveConstructor{app}, and
\AgdaInductiveConstructor{lam}).
With syntactic kits, we already asked for the field \AgdaField{tm} to interpret
\AgdaBound{$\V$}-values as terms.
We rename \AgdaField{tm} to \AgdaField{⟦var⟧} to reflect its role in the
semantic traversal \AgdaFunction{sem}.
Now, we will also ask for fields to replace the right-hand side applications of
the other term constructors.
For application, we can stick with the obvious thing: we should be able to
combine a semantic function and its semantic argument to get the semantic
result.

\ExecuteMetaData[\SimpleSem]{semVarApp}

However, we want to treat binding constructs specially, particularly because
there are semanticses with no notion of binding.
We instead provide a function from values to computations that works in any
\emph{extension} of the current context.
Keeping \AgdaField{$\swarrow^k$} as before, we get the following semantic
replacement for \AgdaRecord{Kit}.

\ExecuteMetaData[\SimpleSem]{SemanticsExplicit}

With the aim of abstracting away from explicit contexts, bringing us closer to
natural deduction, we can use some new notation to rephrase these requirements.
We will work in \AgdaRecord{Ctx}\AgdaSpace{}\AgdaSymbol{$\to$}\AgdaSpace{}%
\AgdaPrimitive{Set} rather than \AgdaPrimitive{Set}.
One of the basic connectives in this setting is the \emph{pointwise} arrow
\AgdaFunction{\_$\Rightarrow$\_}, which acts in
\AgdaRecord{Ctx}\AgdaSpace{}\AgdaSymbol{$\to$}\AgdaSpace{}\AgdaPrimitive{Set}
like the non-dependent arrow does in \AgdaPrimitive{Set}.
Another basic component is the \AgdaFunction{$\forall[\_]$} notation, which
embeds \AgdaRecord{Ctx}\AgdaSpace{}\AgdaSymbol{$\to$}\AgdaSpace{}%
\AgdaPrimitive{Set} into \AgdaPrimitive{Set} by using an implicit $\Pi$-type
to quantify over \emph{all} contexts.
Finally, at this stage, I introduce a modality \AgdaFunction{$\bigcirc$}
encapsulating the pattern of considering arbitrary \emph{extensions} of a
context.
To facilitate working in this point-free setting, I give infix versions of
the families \AgdaBound{$\V$} and \AgdaBound{$\C$} (respectively
\AgdaFunction{\_$\V\vdash$\_} and \AgdaFunction{\_$\C\vdash$\_}).
The principal use of these aliases is to fill the right argument with a type
(occuring explicitly), and leave the left argument as \AgdaFunction{\_}, i.e.,
a context given through the point-free machinery.

\ExecuteMetaData[\SimpleSem]{Circle}

\ExecuteMetaData[\SimpleSem]{SemanticsCircle}

To illustrate this definition, I will discuss a syntactic traversal ---
renaming --- and a semantic traversal --- a standard $\mathrm{Set}$ semantics.

For the renaming semantics, as with the renaming kit, we specify that
environments hold variables (\AgdaDatatype{\_$\ni$\_}) and show that variables
satisfy the required form of weakening (\AgdaFunction{$\swarrow^v$}).
Meanwhile, whereas all syntactic kits target terms
(\AgdaDatatype{\_$\vdash$\_}), with a semantic traversal we must specify the
target.
The fields \AgdaField{$\llbracket$var$\rrbracket$} and
\AgdaField{$\llbracket$app$\rrbracket$} follow straightforwardly, with variables
embedding into terms and a pair of terms of the right types giving an
application term in the same context, via the relevant constructors.
For the \AgdaField{$\llbracket$lam$\rrbracket$} case, we are given
\ExecuteMetaData[\SimpleSem]{bRenSemCircle}, and after producing a
\AgdaInductiveConstructor{lam}, are left needing a term in
\ExecuteMetaData[\SimpleSem]{resRenSemCircle}.
That the type of \AgdaBound{b} is wrapped in \AgdaFunction{$\bigcirc$} gives
us the ability to use \AgdaBound{b} in the extended context
\ExecuteMetaData[\SimpleKits]{GA}.
In particular, we point at the new variable to yield the desired term in the
same context.

\ExecuteMetaData[\SimpleSem]{RenSemCircle}

To produce a $\mathrm{Set}$ semantics, we shift from targeting terms to
targeting the interpretation of terms.
In particular, \ExecuteMetaData[\SimpleSem]{semGA}\ is the type of functions
from the interpretation of \AgdaBound{$\Gamma$} to the interpretation of
\AgdaBound{A}.
The interpretation of a type is defined as usual, by recursion on the structure
of the type.
The interpretation of a context is the interpretation for each of its types.
We still have environments storing variables, which delays the interpretation of
variables to the \AgdaField{$\llbracket$var$\rrbracket$} case and allows newly
bound variables to be referred to directly as variables, rather than fetching
them up-front from an environment of interpretations.
In the \AgdaField{$\llbracket$app$\rrbracket$} case, we have
\ExecuteMetaData[\SimpleSem]{mSetSemCircle}\ and
\ExecuteMetaData[\SimpleSem]{nSetSemCircle}, and combine them in the usual way
by distributing the interpretation of the context
\ExecuteMetaData[\SimpleSem]{gammaSetSemCircle}.
The \AgdaField{$\llbracket$lam$\rrbracket$} case involves the same placement of
the new variable into the environment as in \AgdaFunction{RenSem}.
Finally, we get the function \AgdaFunction{set} from terms to their
interpretations by passing in the identity $\ni$-environment \AgdaFunction{id}.

\ExecuteMetaData[\SimpleSem]{SetSemCircle}

The definition of \AgdaRecord{Semantics} above essentially enforces that the
term being traversed and the result of the traversal share the same binding
structure.
Concretely, \AgdaInductiveConstructor{lam} is the only case where we can bind
new variables, and at that point we must do exactly one binding.
This is fine for renaming and substitution, which preserve the binding
structure, and also for a standard denotational semantics, which is sufficiently
abstracted from binding.
However, if we want to do other syntactic translations --- for example,
converting from a syntax with $n$-ary functions to a syntax with only unary
Curried functions --- it would be useful to allow more choices when going under
a binder.
To this end, I replace the one-step binding modality \AgdaFunction{$\bigcirc$}
by the all-possible-renamings modality \AgdaFunction{$\Box$}.
\AgdaFunction{$\Box$}\AgdaSpace{}\AgdaBound{T}\AgdaSpace{}\AgdaBound{$\Gamma$}
states that \AgdaBound{T} holds not only at \AgdaBound{$\Gamma$}, but also at
any context \AgdaBound{$\Gamma^+$} containing \AgdaBound{$\Gamma$} (including
\ExecuteMetaData[\SimpleKits]{GD}\ for any \AgdaBound{$\Delta$}).

\ExecuteMetaData[\SimpleSem]{Box}

As well as the flexibility in binding structure, the \AgdaFunction{$\Box$}
modality allows us to use the somewhat more well behaved and standard
relation of renaming, rather than strict context extension.
The resulting definition of \AgdaRecord{Semantics} is as follows, and is
simply a version of the previous definition where \AgdaFunction{$\bigcirc$}
has been replaced by \AgdaFunction{$\Box$}.
It will become apparent when we implement the traversal \AgdaFunction{sem} why
the first field also changes to include a \AgdaFunction{$\Box$}.

\ExecuteMetaData[\SimpleSem]{Semantics}

Writing a \AgdaFunction{$\Box$}-based semantics is very similar to writing a
\AgdaFunction{$\bigcirc$}-based semantics, so I will only give one further
example.
I generalise the renaming example to derive a semantic traversal from any
syntactic traversal.
We need a slightly modified definition of \AgdaRecord{Kit} to provide
\emph{renaming} of \AgdaBound{$\V$}-values, rather than just extension.

\ExecuteMetaData[\SimpleSem]{Kit}

The interesting feature of the corresponding \AgdaRecord{Semantics} is that
we now pass \AgdaBound{b} the renaming \AgdaFunction{$\swarrow^v$}, projecting
the original context \AgdaBound{$\Gamma$} out of the
\AgdaInductiveConstructor{lam}-extended context
\ExecuteMetaData[\SimpleKits]{GA}.

\ExecuteMetaData[\SimpleSem]{kit}

\section{Generic syntax}
\section{Natural deduction vs sequent calculus}
In a seminal paper~\cite{Gentzen64}, Gerhard Gentzen introduces two syntactic
paradigms which remain among the most studied to this day.
These paradigms are \emph{natural deduction} and \emph{sequent calculus}, as
exemplified by natural deduction calculi NJ and NK, and sequent calculi LJ and
LK.
Restricting attention to just the intuitionistic systems NJ and LJ, these
actually differ in two orthogonal ways, which I shall prise apart in this
section.
The simpler distinction is that the logical rules in NJ are introduction and
elimination rules, whereas the logical rules in LJ are left and right rules.
But the more important distinction for this thesis is that where NJ has
assumptions, LJ has sequents explictly manipulated by structural rules.
I take the latter distinction to define natural deduction and sequent calculus,
and wherever I need to make the former distinction I shall speak of
\emph{intro-elim systems} and \emph{left-right systems}.

I will use the rest of this section as follows.
First, I introduce NJ and LJ, and some of their basic metatheory.
Then, I will give examples of systems intermediate between Gentzen's natural
deduction and sequent calculi: a left-right natural deduction calculus
$\mu\tilde\mu$, and an intro-elim sequent calculus BBdPH\@.
Both of these examples will reappear in later chapters.

\subsection{Intro-elim natural deduction: NJ}

\subsection{Left-right sequent calculus: LJ}

\subsection{Left-right natural deduction: $\mu\tilde\mu$}
The $\mu\tilde\mu$-calculus~\cite{CH00} (also known as
$\overline\lambda\mu\tilde\mu$ or system L, and closely related to Wadler's
dual calculus~\cite{Wadler03}) can be seen as an adaptation of natural deduction
to classical logic.
Though originally presented as a sequent calculus, the underlying natural
deduction calculus was later given by Herbelin~\cite[p.\ 12]{herbelin-hab}, and
I will follow the latter.
While Gentzen gave a natural deduction calculus NK for classical logic, NK
relies on adding the \emph{axiom} of excluded middle.
As axioms are not systematic parts of the calculus, they can hinder or break
metatheoretic properties like normalisation.
In contrast, the $\mu\tilde\mu$-calculus allows us to \emph{derive} excluded
middle from entirely systematic components.

In NJ, a derivation of $A$ from assumptions $\Gamma$ tells us that if each
formula in $\Gamma$ is \emph{true}, then $A$ is also \emph{true}.
The $\mu\tilde\mu$-calculus generalises the picture by allowing us to have
both \emph{true} and \emph{false} assumptions, and allowing us to conclude that
some $A$ is \emph{true}, that some $A$ is \emph{false}, or that we have reached
a contradiction.
% A similar judgement of contradiction appears in Prawitz' classical natural
% deduction calculus~\cite{Prawitz65}.
Following Herbelin, we notate the judgement that $A$ is true as ${}\vdash A$,
that $A$ is false as $A \vdash{}$, and of contradiction as $\vdash$.
The only way to derive a contradiction is to find some $A$ such that
${}\vdash A$ and $A \vdash{}$.
Meanwhile, we can derive ${}\vdash A$ by assuming $A \vdash{}$ and deriving
$\vdash$, i.e., we can prove $A$ by assuming that $A$ is false and deriving a
contradiction.
Dually, we can derive $A \vdash{}$ by assuming ${}\vdash A$ and deriving
$\vdash$.
These three methods of derivation are encoded in the following rules.

\begin{mathpar}
  \inferrule*[right=Cut]
  {{}\vdash A \\ A \vdash{}}
  {\vdash}

  \and

  \inferrule*[right=$\mu$]
  {
    [A \vdash{}] \\\\ \vdots \\\\ \vdash
    %\inferrule*[fraction={~~~}]
    %{[A \vdash{}] \\\\ \vdots}
    %{\vdash}
  }
  {{}\vdash A}

  \and

  \inferrule*[right=$\tilde\mu$]
  {[{}\vdash A] \\\\ \vdots \\\\ \vdash}
  {A \vdash{}}
\end{mathpar}

The rules for logical connectives describe how to \emph{prove} and how to
\emph{refute} a formula whose principal connective is that connective.
These correspond strongly with the right and left rules, respectively, of LJ,
and for this reason, $\mu\tilde\mu$ is usually described elsewhere as a
sequent calculus.
For example, we could choose the following rules for disjunction.
To prove $A \vee B$, we can assume that both $A$ and $B$ are false, and derive
a contradiction.
To refute $A \vee B$, we can refute $A$ and $B$ separately.

\begin{mathpar}
  \inferrule*[right=$\vee$-r]
  {[A \vdash{}][B \vdash{}] \\\\ \vdots \\\\ \vdash}
  {{}\vdash A \vee B}

  \and

  \inferrule*[right=$\vee$-l]
  {A \vdash{} \\ B \vdash{}}
  {A \vee B \vdash{}}
\end{mathpar}

\subsection{Intro-elim sequent calculus: BBdPH}

Term assignment system introduced in~\cite{BBdPH93}.

