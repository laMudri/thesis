\chapter{Mechanisation of simple types}\label{sec:simple}

In this chapter, I review and justify the family of approaches usually used to
represent simple type systems inside dependently typed proof assistants.
These approaches were first presented by \citet{AR99}, who showed a way of
representing well \emph{scoped} terms in a language with polymorphic recursion,
and extended the representation to well \emph{typed} terms in a language with
dependent types.
The representation of terms relies on indexing on both a context --- giving the
types of all the free variables --- and a type for the term itself.
A basic operation on terms is \emph{simultaneous substitution}, which replaces
each variable in the context by a term in another context.
This simultaneous substitution operation fits the form of composition in a
multicategory.
Specifically, in the case of a simple intuitionistic type system, this is the
composition of a Cartesian multicategory, which allows all of the terms being
substituted in to share a context.

\section{Term representation}
We could mechanise Gentzen's definition of a natural deduction system directly,
but this definition is quite complicated.
In particular, if we want to give derivations an inductive definition, the use
of the discharge mechanism means that we actually need an inductive-inductive
type --- derivations, particularly those using $\to$-introduction, can involve
references to assumptions within their subderivations.
An inductive-inductive definition of derivations would complicate our programs
and proofs about natural deduction derivations, so I choose an alternative
representation.

Indeed, most authors since Gentzen, whether mechanising their work or not,
have opted to replace discharge of assumptions by explicit \emph{contexts} and
a variable rule.
Contexts can be justified as a way to keep track of undischarged assumptions.
In particular, we only produce derivations in the presence of a known collection
of \emph{free variables} specified by the context.
In other words, derivations are \emph{indexed} over their free variables and
their types.
When using an assumption within a derivation, we must say which free variable
it corresponds to.
Free variables are introduced by \emph{variable-binding} rules, like
$\to$-introduction.
\cref{fig:explicit-contexts} gives an example of the same derivation written
in Gentzen's style and in the explicit context style.

\begin{sidewaysfigure}
  \centering
  \begin{prooftree}
    \hypo{[A \to A \to B]^f}
    \hypo{[A]^x}
    \infer2[$\to$-E]{A \to B}
    \hypo{[A]^x}
    \infer2[$\to$-E]{B}
    \infer1[$\to$-I$^x$]{A \to B}
    \infer1[$\to$-I$^f$]{\plr{A \to A \to B} \to \plr{A \to B}}
  \end{prooftree}

  \vspace{2em}

  \begin{prooftree}
    \infer0[var$^f$]{{\color{red}f : A \to A \to B, x : A} \vdash A \to A \to B}
    \infer0[var$^x$]{{\color{red}f : A \to A \to B, x : A} \vdash A}
    \infer2[$\to$-E]{{\color{red}f : A \to A \to B, x : A} \vdash A \to B}
    \infer0[var$^x$]{{\color{red}f : A \to A \to B, x : A} \vdash A}
    \infer2[$\to$-E]{{\color{red}f : A \to A \to B, x : A} \vdash B}
    \infer1[$\to$-I$^x$]{{\color{red}f : A \to A \to B} \vdash A \to B}
    \infer1[$\to$-I$^f$]{\vdash \plr{A \to A \to B} \to \plr{A \to B}}
  \end{prooftree}
  \caption{A proof in Gentzen's natural deduction syntax, and a proof using
    explicit contexts (contexts coloured {\color{red}red})}
  \label{fig:explicit-contexts}
\end{sidewaysfigure}

Explicit contexts can be seen as a mechanism for encoding a natural deduction
system as a sequent calculus.
However, the natural deduction character of the system is maintained by
ensuring that the resultant sequent calculus is really an encoding of a
natural deduction system.
Concretely, this means that rules can only interact with the context in
restricted ways:

\begin{itemize}
  \item There is a designated \emph{variable rule}, stating that any variable
    in the context can serve as a derivation of its type.
  \item Non-variable rules may only require subterms with \emph{extended}
    contexts, i.e., subterms in which new variables have been bound.
    Non-variable rules are parametric in the existing free variables.
\end{itemize}

Having chosen to use explicit contexts, the mechanisation must have a chosen
representation of contexts as a data structure.
While the notation in \cref{fig:explicit-contexts} uses names $f$ and $x$
for variables, I opt for a nameless representation.
In a nameless representation, variables are identified by their position in
the context, rather than by a name.
The absence of names means that $\alpha$-equivalence is just on-the-nose
equality, and also that we never have to reason about freshness of names.
Agda does not have support for nominal techniques~\cite{GP02}, which may have
made names a better option.

Most mechanisations choose contexts to be an inductive list of types.
However, I instead choose a functional, tree-shaped representation, as shown
with the type \AgdaRecord{Ctx}.
The type \AgdaDatatype{LTree} is the inductive type generated by leaves and
nullary \& binary nodes, and serves as a generalised ``length'' of the context.
The tree shape makes concatenation definitionally
injective, so that in cases where multiple new variables are bound in a subterm
(for example, $\otimes$-elimination), Agda's unification-based solving will
be more able to infer which variables have just been bound.
Within a given \AgdaBound{t}\AgdaSpace{}\AgdaSymbol:\AgdaSpace{}%
\AgdaDatatype{LTree}, we can define the positions of \AgdaBound{t} using
\AgdaDatatype{Ptr}.
A \emph{pointer} (\AgdaDatatype{Ptr}) into a tree picks out a leaf
(\AgdaInductiveConstructor{[-]}) following a path of lefts
(\AgdaInductiveConstructor{$\swarrow$}) and rights
(\AgdaInductiveConstructor{$\searrow$}) at any binary nodes encountered.

\ExecuteMetaData[\LTreetex]{LTree}
\ExecuteMetaData[\LTreetex]{Ptr}

The contents of the context --- the types --- are then stored in the functional
vector \AgdaField{ty-ctx}, which is a mapping from leaves in \AgdaField{shape}
to object language types \AgdaDatatype{Ty}.
The advantages of the functional vector representation will not become clear
until later chapters --- particularly the example in
\cref{sec:usage-elaborator}, where I make use of the ease of look-up and the
$\eta$-law of functions.
However, I claim for now that there is little to no disadvantage in the
functional vector representation --- in particular, we have no need for
function extensionality principles because we never talk about equality of
contexts.
For example, instead of using an equality of contexts to coerce a term, we can
use renaming.

\ExecuteMetaData[\Vectortex]{Vector}
\Ctx{}

Our first data structure involving contexts is that of intrinsically typed
variables.
A variable of type
\AgdaBound{$\Gamma$}\AgdaSpace{}\AgdaRecord{$\ni$}\AgdaSpace{}\AgdaBound{A}
is given by a path \AgdaField{idx} to a type in \AgdaBound{$\Gamma$}, together
with a proof \AgdaField{tyq} that this type is equal to \AgdaBound{A}.

\Var{}

Variables embed into terms via the \AgdaInductiveConstructor{var} constructor of
the family \AgdaDatatype{\_⊢\_} of intrinsically simply typed terms.
The only other syntactic forms we consider for now are the eliminator and
constructor of function types \AgdaInductiveConstructor{\_`→\_} ---
\AgdaInductiveConstructor{app} and \AgdaInductiveConstructor{lam}.
Application \AgdaInductiveConstructor{app} takes two subterms of the appropriate
types, while the subterm of $\lambda$-abstraction \AgdaInductiveConstructor{lam}
is in an extended context \GA{} --- \AgdaBound{$\Gamma$} concatenated with a
singleton context containing the type \AgdaBound{A}.

\Term{}

Using this encoding, the Church numeral for 2 appears as follows.
In standard notation, this would be
$\lambda f.~\lambda x.~f\,(f\,x)$.
To refer to $f$ in the main body of the expression, we skip one binder (using
\AgdaInductiveConstructor{$\swarrow$}) and pick the next one
(using \AgdaInductiveConstructor{$\searrow$}) and pick its only bound variable
(using \AgdaInductiveConstructor{here}).
To refer to $x$, we do not skip its binder, instead picking it and its only
bound variable.

\ExecuteMetaData[../agda/processed-latex/SimpleKits.tex]{two}

\section{Renaming and substitution}\label{sec:kits}
\def\SimpleKits{../agda/processed-latex/SimpleKits.tex}

%Explain:
%
%\begin{itemize}
%  \item Specific uses of renaming/substitution in $\lambda$-calculus semantics.
%  \item General role of renaming/substitution in abstract algebra/syntax with
%    binding.
%\end{itemize}

A basic operation on any syntax with variables is \emph{substitution} --- the
replacement of variables in a term by terms with the same type as the variables.
In a sense, this is the defining operation of variables --- a variable is a
placeholder for a term, or equivalently in logic, a hypothesis is a placeholder
for an arbitrary proof.
In a type theory or logic, terms can bind variables, and we will typically have
operational semantics rules combining a term binding a variable with a term that
is to be substituted into the place of that variable, like the $\beta$-rule for
$\lambda$-calculus functions.

While substitution has this extra role in a lot of the syntaxes with binding we
care about, variable-binding also significantly complicates the substitution
operation.
Substitution acts on the free variables of a term, replacing them by terms, but
binders mean that some subterms have \emph{more} free variables than our
original term.
This causes different challenges for different representations of terms.
For example, with named variables and shadowing, na\"{i}vely defined
substitution could fall foul of variable capture.
In our approach, based on de Bruijn indices, the difficulty is that an index $i$
outside a binder of $n$ variables corresponds to an index $n + i$ inside the
binder.
Therefore, when substituting under a binder, we must first increment any free
variables contained in terms we are substituting in, which is a form of
\emph{renaming}.
Renaming replaces each free variable by another free variable, and is a special
case of substitution.
We must, however, define renaming before substitution, so as to avoid the
definition of substitution being circular.
Renaming avoids a similar circularity because when renaming goes under a binder,
we only have to increment each variable being renamed in, rather than each
variable \emph{in each term} being substituted in.

In this section, I formally implement simultaneous renaming and substitution for
the terms defined in the previous section.
Simultaneous substitution turns out to have a simple definition, which
generalises into other algorithms over terms with binders.
The section concludes with a unified implementation of renaming and
substitution, leaving further generalisation to the next section.

\subsection{Simultaneous renaming and simultaneous substitution}

A simultaneous renaming from $\Gamma$ to $\Delta$ is a type-preserving map from
variables in $\Delta$ to \emph{variables} in $\Gamma$, while a simultaneous
substitution is a map into \emph{terms} in $\Gamma$.
While simultaneous substitution gives us a notion of one context being
\emph{derivable} from another, simultaneous renaming gives a similar notion
of derivability restricted to structural rules.

In the derivation below, we assume the existence of a derivation of
$B, C \to C \vdash C$, and by the admissibility of substitution we thus have a
derivation of $A \to B, A \vdash C$.
Intuitively, the context $A \to B, A$ derives the context $B, C \to C$, so
anything derived from $B, C \to C$ can also be derived from $A \to B, A$.
We see formally that $A \to B, A$ derives $B, C \to C$ by deriving each element
of the latter from the former --- hence the first two premises of the \TirName{Subst}
rule below, deriving $B$ and $C \to C$ from $A \to B, A$.

\begin{align*}
  &\begin{prooftree}
    \hypo{\Pi}
    \infer[no rule]1{A \to B, A \vdash B}
    \infer0[Var]{A \to B, A, C \vdash C}
    \infer1[$\to$-I]{A \to B, A \vdash C \to C}
    \hypo{B, C \to C \vdash C}
    \infer3[Subst]{A \to B, A \vdash C}
  \end{prooftree}
  \\\\
  &\textrm{where }\Pi \coloneqq
  \begin{prooftree}
    \infer0[Var]{A \to B, A \vdash A \to B}
    \infer0[Var]{A \to B, A \vdash A}
    \infer2[$\to$-E]{A \to B, A \vdash B}
  \end{prooftree}
\end{align*}

\subsection{Proofs of admissibility of renaming and substitution}

A renaming from \AgdaBound{$\Gamma$} to \AgdaBound{$\Delta$}
is a map from variables in \AgdaBound{$\Delta$} to variables in
\AgdaBound{$\Gamma$}, represented in Agda as follows.

\Ren{}

The action of a renaming \AgdaBound{$\rho$} on terms is given by
\AgdaFunction{ren}\AgdaSpace{}\AgdaBound{$\rho$}, with \AgdaFunction{ren}
defined below.
The idea of simultaneous renaming is to preserve the structure of the term, but
replace all of the variables from \AgdaBound{$\Delta$} by variables from
\AgdaBound{$\Gamma$}, with the mapping given by the renaming \AgdaBound{$\rho$}.

\rename{}

The \AgdaInductiveConstructor{var} case is where the action of the renaming
happens: the variable \AgdaBound{x} from \AgdaBound{$\Delta$} is mapped to the
variable \AgdaBound{$\rho$}\AgdaSpace{}\AgdaBound{x} from \AgdaBound{$\Gamma$}.
In the \AgdaInductiveConstructor{app} case, we have terms \AgdaBound{M}
\AgdaSymbol{:} \AgdaBound{$\Delta$} \AgdaDatatype{$\vdash$}
\AgdaBound{A} \AgdaInductiveConstructor{`$\to$} \AgdaBound{B} and \AgdaBound{N}
\AgdaSymbol{:} \AgdaBound{$\Delta$} \AgdaDatatype{$\vdash$} \AgdaBound{A}.
We may apply \AgdaFunction{ren} \AgdaBound{$\rho$} recursively to both
\AgdaBound{M} and \AgdaBound{N} to change their contexts from
\AgdaBound{$\Delta$} to \AgdaBound{$\Gamma$}, and the
\AgdaInductiveConstructor{app} constructor then produces the desired term in
\AgdaBound{$\Gamma$}.
Finally, in the \AgdaInductiveConstructor{lam} case, we get a term
\AgdaBound{M} \AgdaSymbol{:} \DA{} \AgdaDatatype{$\vdash$} \AgdaBound{B} and,
after introducing a \AgdaInductiveConstructor{lam} on the right, are in need
of a term of type \GA{} \AgdaDatatype{$\vdash$} \AgdaBound{B}.
To recursively apply \AgdaFunction{ren} to \AgdaBound{M}, we must thus extend
the renaming \AgdaBound{$\rho$} \AgdaSymbol{:} \RenGD{} with the newly bound
variable.
For this, we need an auxiliary function \AgdaFunction{bindRen} such that
\AgdaFunction{bindRen} \AgdaBound{$\rho$} \AgdaSymbol{:} \RenGADA{}.
This new renaming will act like \AgdaBound{$\rho$} for variables in
\AgdaBound{$\Delta$}, and map the new variable of type \AgdaBound{A} to the
corresponding new variable in \GA{}.

\bindRen{}

The \AgdaFunction{bindRen} given here has a slightly generalised type, where
instead of binding just a single variable of type \AgdaBound{A}, we could
bind a whole context \AgdaBound{$\Theta$} of new variables.
The first case of \AgdaFunction{bindRen} is for old variables from
\AgdaBound{$\Delta$}, where we apply \AgdaBound{$\rho$} to get a variable in
\AgdaBound{$\Gamma$}, and then use \AgdaFunction{$↙ᵛ$} to embed that variable
into \GTh{}.
The second case is for new variables from \AgdaBound{$\Theta$}, which embed
straight into \GTh{}.

Meanwhile, a substitution from \AgdaBound{$\Gamma$} to \AgdaBound{$\Delta$} is
an inhabitant of \AgdaFunction{Sub}\AgdaSpace{}\AgdaBound{$\Gamma$}\AgdaSpace{}%
\AgdaBound{$\Delta$}, as defined below.
This definition is identical to the definition of \AgdaFunction{Ren}, except
that it gives us \emph{terms} in \AgdaBound{$\Gamma$} rather than variables.

\Sub{}

The \AgdaFunction{sub} function below gives the action of a substitution.
Similarly to renaming, we want to preserve the structure of the term, except
now variables in the original term are replaced by \emph{terms} in the new
context.

\substitute{}

Given that this time, \AgdaBound{$\rho$} is a substitution rather than a
renaming, \AgdaBound{$\rho$} \AgdaBound{x} is a term, and is sufficient in the
\AgdaInductiveConstructor{var} case.
The \AgdaInductiveConstructor{app} case again deals with the subterms
recursively and then recombines them with \AgdaInductiveConstructor{app}.
In the \AgdaInductiveConstructor{lam} case, we again have a mismatch if we
want to apply \AgdaFunction{sub} recursively to the subterm \AgdaBound{M} with
an extra free variable.
We have \AgdaBound{$\rho$} \AgdaSymbol{:} \SubGD{} but need a substitution of
type \SubGADA{}, so we introduce the auxiliary definition
\AgdaFunction{bindSub}.

\bindSub{}

For the old variables in the first case, we have \AgdaBound{$\rho$} to turn
them into terms in \AgdaBound{$\Gamma$}.
Turning a term in \AgdaBound{$\Gamma$} into a term in \GTh{} requires a form
of weakening we have not yet proved, so I write \AgdaFunction{$↙ᵗ$} in analogy
with \AgdaFunction{$↙ᵛ$}, and prove it below.
In the second case, we want to substitute the new variable by the \emph{term}
referring to this new variable in \GTh{}.

The final piece to define substitution is to define the function that weakens
a term by some newly bound variables \AgdaBound{$\Delta$}.
For this, we use the action of renaming, which we have fully defined already,
and in particular rename each variable in the term from a variable in
\AgdaBound{$\Gamma$} to a variable in \GD{}.

\leftTerm{}

With this, the action of substitution is defined, and depends on the action
of renaming.

\subsection{Syntactic kits}\label{sec:syntactic-kits}

As observed by \citet{McBride05,BHKM12},
the statements of simultaneous renaming and simultaneous substitution are
very similar, with substitution being the generalisation that allows
replacement of variables by terms rather than just other variables.
Following \citet{McBride05},
I will introduce a type family \AgdaFunction{Env} of \emph{environments}, and
redefine \AgdaFunction{Ren} and \AgdaFunction{Sub} as environments of
variables and terms, respectively.

\Env{}
\RenSub{}

The processes I described for constructing proofs of the admissibility of
renaming and substitution were also similar.
Indeed, when we line up the resulting functions, \AgdaFunction{ren} and
\AgdaFunction{sub}, and their auxiliaries, \AgdaFunction{bindRen} and
\AgdaFunction{bindSub}, we notice only three key
differences:

\begin{itemize}
  \item In the first cases of \AgdaFunction{bindRen} and \AgdaFunction{bindSub},
    we do \AgdaFunction{$↙ᵛ$} and \AgdaFunction{$↙ᵗ$}, respectively, based on
    whether we are weakening a variable or a term.
  \item In the second case of \AgdaFunction{bindSub}, we do an extra wrapping of
    the new variable by \AgdaInductiveConstructor{var}, so as to make it a term
    to go in the substitution.
  \item In the \AgdaInductiveConstructor{var} case of \AgdaFunction{ren}, we
    have \AgdaInductiveConstructor{var} \AgdaSymbol(\AgdaBound{$\rho$}
    \AgdaBound{x}\AgdaSymbol) rather than just \AgdaBound{$\rho$} \AgdaBound{x},
    because the renaming \AgdaBound{$\rho$} gives us a variable rather than a
    term.
\end{itemize}

We may abstract over these three differences using the record \AgdaRecord{Kit}.
As in \AgdaFunction{Env}, we think of \AgdaBound{K} as being either
\AgdaDatatype{\_$\ni$\_} or \AgdaDatatype{\_$\vdash$\_}.
The fields of \AgdaRecord{Kit} are given in the same order as the points of
difference above.
Wherever the difference was presence or absence of
\AgdaInductiveConstructor{var}, we will be able to fill that field with either
\AgdaInductiveConstructor{var} or the identity function \AgdaFunction{id}.

\Kit{}

The field \AgdaField{$\swarrow^k$} can be seen as a property of the
judgement form \AgdaBound{K}, saying that it supports a form of weakening.
We use \AgdaField{vr} when adding a newly bound variable to an environment, and
use \AgdaField{tm} when we do a lookup from an environment and want to get a
term out.
Given a \AgdaRecord{Kit} \AgdaBound{K}, we can write the syntactic traversal
function \AgdaFunction{trav} and its auxiliary \AgdaFunction{bindEnv}, in the
model of \AgdaFunction{ren}, \AgdaFunction{sub}, and their auxiliaries.

\trav{}
\bindEnv{}

Concrete kits can be given for variables and terms either by inspecting
\AgdaFunction{ren} and \AgdaFunction{sub} or by following the types.
Notice that the kit for terms requires the admissibility of renaming so as to
achieve weakening of a substitution by newly bound variables.
Fortunately, this can be the \AgdaFunction{ren} defined below in terms of
\AgdaFunction{trav}, so we can keep \AgdaFunction{trav} as the only syntactic
traversal we have to write.

\begin{multicols}{2}
  \noindent\renKit{} \columnbreak

  \noindent\subKit{}
\end{multicols}

\section{Generic semantics}
The traversal \AgdaFunction{trav} from the last section is generic in the sense
that $\V$, the type of entries in an environment, can be instantiated to many
different things.
However, in practice we only use $\ni$ and $\vdash$, giving us renaming and
substitution, respectively.
This is because \AgdaFunction{trav} only targets terms, and does so by keeping
term constructors intact and replacing only the variables by things from the
environment.
This makes substitution the most general possible traversal.

If we want to capture a broader range of traversals, including not just
syntactic but also \emph{semantic} operations, we must be able to target things
other than terms, and act in an interesting way on term constructors.
Doing a straight generalisation of the type of \AgdaFunction{trav}, this
suggests that we want a function with the following type.

\missingfigure{Semantic traversal type signature}

\section{Generic syntax}
\section{Natural deduction vs sequent calculus}
In a seminal paper~\cite{Gentzen64}, Gerhard Gentzen introduces two syntactic
paradigms which remain among the most studied to this day.
These paradigms are \emph{natural deduction} and \emph{sequent calculus}, as
exemplified by natural deduction calculi NJ and NK, and sequent calculi LJ and
LK.
Restricting attention to just the intuitionistic systems NJ and LJ, these
actually differ in two orthogonal ways, which I shall prise apart in this
section.
The simpler distinction is that the logical rules in NJ are introduction and
elimination rules, whereas the logical rules in LJ are left and right rules.
But the more important distinction for this thesis is that where NJ has
assumptions, LJ has sequents explictly manipulated by structural rules.
I take the latter distinction to define natural deduction and sequent calculus,
and wherever I need to make the former distinction I shall speak of
\emph{intro-elim systems} and \emph{left-right systems}.

I will use the rest of this section as follows.
First, I introduce NJ and LJ, and some of their basic metatheory.
Then, I will give examples of systems intermediate between Gentzen's natural
deduction and sequent calculi: a left-right natural deduction calculus
$\mu\tilde\mu$, and an intro-elim sequent calculus BBdPH\@.
Both of these examples will reappear in later chapters.

\subsection{Intro-elim natural deduction: NJ}

\subsection{Left-right sequent calculus: LJ}

\subsection{Left-right natural deduction: $\mu\tilde\mu$}
The $\mu\tilde\mu$-calculus~\cite{CH00} (also known as
$\overline\lambda\mu\tilde\mu$ or system L, and closely related to Wadler's
dual calculus~\cite{Wadler03}) can be seen as an adaptation of natural deduction
to classical logic.
Though originally presented as a sequent calculus, the underlying natural
deduction calculus was later given by Herbelin~\cite[p.\ 12]{Herbelin-hab}, and
I will follow the latter.
While Gentzen gave a natural deduction calculus NK for classical logic, NK
relies on adding the \emph{axiom} of excluded middle.
As axioms are not systematic parts of the calculus, they can hinder or break
metatheoretic properties like normalisation.
In contrast, the $\mu\tilde\mu$-calculus allows us to \emph{derive} excluded
middle from entirely systematic components.

In NJ, a derivation of $A$ from assumptions $\Gamma$ tells us that if each
formula in $\Gamma$ is \emph{true}, then $A$ is also \emph{true}.
The $\mu\tilde\mu$-calculus generalises the picture by allowing us to have
both \emph{true} and \emph{false} assumptions, and allowing us to conclude that
some $A$ is \emph{true}, that some $A$ is \emph{false}, or that we have reached
a contradiction.
% A similar judgement of contradiction appears in Prawitz' classical natural
% deduction calculus~\cite{Prawitz65}.
Following Herbelin, we notate the judgement that $A$ is true as ${}\vdash A$,
that $A$ is false as $A \vdash{}$, and of contradiction as $\vdash$.
The only way to derive a contradiction is to find some $A$ such that
${}\vdash A$ and $A \vdash{}$.
Meanwhile, we can derive ${}\vdash A$ by assuming $A \vdash{}$ and deriving
$\vdash$, i.e., we can prove $A$ by assuming that $A$ is false and deriving a
contradiction.
Dually, we can derive $A \vdash{}$ by assuming ${}\vdash A$ and deriving
$\vdash$.
These three methods of derivation are encoded in the following rules.

\begin{mathpar}
  \inferrule*[right=Cut]
  {{}\vdash A \\ A \vdash{}}
  {\vdash}

  \and

  \inferrule*[right=$\mu$]
  {
    [A \vdash{}] \\\\ \vdots \\\\ \vdash
    %\inferrule*[fraction={~~~}]
    %{[A \vdash{}] \\\\ \vdots}
    %{\vdash}
  }
  {{}\vdash A}

  \and

  \inferrule*[right=$\tilde\mu$]
  {[{}\vdash A] \\\\ \vdots \\\\ \vdash}
  {A \vdash{}}
\end{mathpar}

The rules for logical connectives describe how to \emph{prove} and how to
\emph{refute} a formula whose principal connective is that connective.
These correspond strongly with the right and left rules, respectively, of LJ,
and for this reason, $\mu\tilde\mu$ is usually described elsewhere as a
sequent calculus.
For example, we could choose the following rules for disjunction.
To prove $A \vee B$, we can assume that both $A$ and $B$ are false, and derive
a contradiction.
To refute $A \vee B$, we can refute $A$ and $B$ separately.

\begin{mathpar}
  \inferrule*[right=$\vee$-r]
  {[A \vdash{}][B \vdash{}] \\\\ \vdots \\\\ \vdash}
  {{}\vdash A \vee B}

  \and

  \inferrule*[right=$\vee$-l]
  {A \vdash{} \\ B \vdash{}}
  {A \vee B \vdash{}}
\end{mathpar}

\subsection{Intro-elim sequent calculus: BBdPH}

Term assignment system introduced in~\cite{BBdPH93}.

