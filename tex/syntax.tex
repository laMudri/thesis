The aim of this chapter is to produce a domain-specific language of
\emph{syntax descriptions}.
One may also use the terminology ``universe'' in place of ``domain-specific
language'', alluding to the \emph{universe pattern} common in dependently typed
programming~\citep{BDJ03}.
A syntax description is essentially a concise, high-level representation of a
type system's syntactic rules.
The information contained in a syntax description is comparable to what is
written (informally) in \cref{fig:lr-bunched}.
The key features allowing these descriptions to capture semiring-annotated
calculi are the distinction between sharing ($\dottimes$) and separating ($*$)
conjunction of premises, modal scaling by a semiring element ($\cdot$), and the
inclusion of semiring annotations on newly bound variables.

%We take the insights of the previous section and use them to build a
%generic framework for posemiring-annotated substructural systems in
%Agda. We will first show \emph{descriptions} of systems, which are
%comprised of rules that have premises combined using the bunched
%combinators. We then show how to construct the Agda data type of
%intrinsically well scoped, typed, and resourced terms for any given
%system of our framework. We use the prototypical system from
%\cref{fig:lr-comb} as a running example. \Cref{sec:other-syntaxes}
%presents further examples that our framework can express.

%We now start to use Agda notation for record and data type
%declarations, to emphasise that our framework has been implemented.

\section{Descriptions of Systems}

% We capture the form of rules exemplified previously\todo{Previously?} via
% \emph{descriptions} of rules.
% The key to making these descriptions work is that they only allow syntactic
% forms that preserve environments.
% These forms are: absence and multiplicity of subterms with the same usage
% annotations, absence and multiplicity of subterms with summed usage annotations,
% scaling of a subterm, and variable binding.\todo{Switching to Agda}

% \paragraph{\AgdaDatatype{Premises}, \AgdaRecord{Rule}s, and \AgdaRecord{System}s.}

I introduce syntax descriptions in three layers: \AgdaRecord{System},
\AgdaRecord{Rule}, and \AgdaRecord{Premises}.
A type \AgdaRecord{System} is made up of multiple \AgdaRecord{Rule}s.
Each \AgdaRecord{Rule} comprises a \AgdaDatatype{Premises} and
a conclusion type. We assume that there is a
$\AgdaBound{Ty} : \AgdaPrimitiveType{Set}$ of types for the system in
scope.

The \AgdaDatatype{Premise} data type describes premises of rules,
using the bunched combinators from \cref{fig:bunched}. A single
premise is introduced by the
\AgdaInductiveConstructor{$\langle$\_`$\vdash$\_$\rangle$}
constructor.  This allows binding of additional variables
\AgdaBound{$\Delta$} (with specified types and usage annotations) and
the specification of a conclusion type \AgdaBound{A} for this premise.
The remaining constructors are descriptions for the
bunched connectives. %, and will be interpreted directly as such, below.

\ExecuteMetaData[\Syntaxtex]{Premises}

A \AgdaRecord{Rule} is a pair of some \AgdaDatatype{Premises} and a
conclusion.
I suggestively use a quoted version of the ``universal entailment'' arrow
$\rightarrowtriangle$, the unquoted version of which interprets the horizontal
line in a traditionally presented typing rule.

\ExecuteMetaData[\Syntaxtex]{Rule}

Finally, a \AgdaRecord{System} consists of a set of rule labels (i.e.,
constructor names), and for each label a description of the
corresponding rule. We use $\rhd$ as infix notation for systems to
associate the label set with the rules.

\ExecuteMetaData[\Syntaxtex]{System}

% \paragraph{\Cref{fig:lr-comb} as a \AgdaRecord{System}.}

As an example, we transcribe a fragment of $\name$ (as defined in
\cref{fig:lr} and \cref{fig:lr-bunched}) into a description.
We give the set of types of
this system as a data type \AgdaDatatype{Ty} (together with a base
type \AgdaInductiveConstructor{$\iota$}). We assume that there is a
posemiring \AgdaInductiveConstructor{Ann} in scope for the
annotations.
There is one label for each instantiation of a logical
rule, but the labels contain no further information about subterms or
restrictions on the context. This will be provided when we associate
labels with \AgdaRecord{Rule}s in a \AgdaRecord{System}.

\noindent
\begin{minipage}[t]{0.5\textwidth}
  \ExecuteMetaData[\PaperExamplestex]{Ty}
  \ExecuteMetaData[\Handtex]{Hand}
\end{minipage}
\begin{minipage}[t]{0.5\textwidth}
  \ExecuteMetaData[\PaperExamplestex]{qlR}
\end{minipage}

To build a system, we associate with each label a rule:

\ExecuteMetaData[\PaperExamplestex]{lR}

Compared to \cref{fig:lr-bunched}, modulo the Agda notation, we can see
that the fundamental structure has been preserved: The rules match
one-to-one, and the bunched premises are the same. A major difference
is that we do not include a counterpart to the
\AgdaInductiveConstructor{var} rule in a
\AgdaRecord{System}. Variables are common to all the systems
representable in our framework.

\section{Terms of a \AgdaRecord{System}}\label{sec:terms-of-system}

The next thing we want to do is to build terms in the described type system.
The following definitions are useful for talking about types indexed over
contexts, judgement forms, and judgement forms admitting newly bound variables,
respectively.

\ExecuteMetaData[\Syntaxtex]{OpenFam}

To specify the meaning of descriptions, we assume some
\AgdaBound{X}\AgdaSpace{}\AgdaSymbol:\AgdaSpace{}\AgdaFunction{ExtOpenFam}%
\AgdaSpace{}\AgdaSymbol{\_},
% \ExecuteMetaData[\Interpretationtex]{X},
over which we form one layer of syntax, using the function
\AgdaFunction{$\llbracket$\_$\rrbracket$p} that interprets
\AgdaDatatype{Premises} defined below.  The first argument to
\AgdaBound{X} is the new variables bound by this layer of syntax, as
exemplified in the first clause of
\AgdaFunction{$\llbracket$\_$\rrbracket$p}.  The second argument is
the context containing the variables being carried over from the
previous layer.  Notice that this is not, in general, the same as the
context from the previous layer, because the usage annotations may
have been changed by connectives like
\AgdaInductiveConstructor{\_`$*$\_} and
\AgdaInductiveConstructor{\_`$\cdot$\_}.  The third argument is the
type of subterm required.

The remainder of the clauses of \AgdaFunction{$\llbracket$\_$\rrbracket$p}
are given by the bunched connectives, as listed below.

\ExecuteMetaData[\Interpretationtex]{semp}

The interpretation of a \AgdaRecord{Rule} checks that the rule targets
the desired type and then interprets the rule's premises
\AgdaBound{ps}.  Notice that the interpretation of the premises is
independent of the conclusion of the rule, which accounts for the use
of \AgdaFunction{OpenType} in
\AgdaFunction{$\llbracket$\_$\rrbracket$p} versus
\AgdaFunction{OpenFam} in \AgdaFunction{$\llbracket$\_$\rrbracket$r}.

\ExecuteMetaData[\Interpretationtex]{semr}

The interpretation of a \AgdaRecord{System} is to choose a rule label
\AgdaBound{l} from \AgdaBound{L} and interpret the corresponding rule
\AgdaBound{rs}\AgdaSpace{}\AgdaBound{l} in the same context and for the same
conclusion.

\ExecuteMetaData[\Interpretationtex]{sems}

The most obvious way to make an
\AgdaBound{X}\AgdaSpace{}\AgdaSymbol:\AgdaSpace{}\AgdaFunction{ExtOpenFam}%
\AgdaSpace{}\AgdaSymbol{\_}
is to use some existing
\AgdaFunction{OpenFam} on an extended context.
I define \AgdaFunction{Scope} to do this: take the new variables
\AgdaBound{$\Delta$}, concatenate them onto the existing context
\AgdaBound{$\Gamma$}, and pass the extended context onto the judgement
\AgdaBound{T}.

\ExecuteMetaData[\Syntaxtex]{Scope}

%{\color{red}(Forward ref: for now, we could have inlined \texttt{Scope}.)}

I use \AgdaFunction{Scope} to deal with new variables in syntax.
Terms resemble the free monad over a layer-of-syntax functor, though
that picture is complicated by variable binding.  A term is either a
variable or a use of a logical rule together with terms for each of
the required subterms.
The \AgdaFunction{Size} argument \AgdaBound{sz} is a use of
Agda's sized types to record that subterms are smaller than the
surrounding term for the termination checker.

\ExecuteMetaData[\Termtex]{Term}

%This definition uses \AgdaFunction{$\dotto$}, which, analogously to
%\AgdaFunction{$\dottimes$}, is an index-preserving version of the function
%space.
%We take \AgdaFunction{$\dotto$} to handle $n$ many indices --- in this
%case two (the context and the type).
%The notation
%\AgdaFunction{$\forall[$}\AgdaSpace{}\AgdaBound{T}\AgdaSpace{}\AgdaFunction{]}
%stands for
%\AgdaSymbol{$\forall$}\AgdaSpace{}\AgdaSymbol{\{}%
%\AgdaBound{x$_1$}\AgdaSpace{}\AgdaSymbol{$\ldots$}\AgdaSpace{}\AgdaBound{x$_n$}%
%\AgdaSymbol{\}}\AgdaSpace{}\AgdaSymbol{$\to$}\AgdaSpace{}\AgdaBound{T}%
%\AgdaSpace{}%
%\AgdaBound{x$_1$}\AgdaSpace{}\AgdaSymbol{$\ldots$}\AgdaSpace{}\AgdaBound{x$_n$},
%where \AgdaBound{T} is a type family with $n$ many indices.

Terms in this data type are difficult to write by hand, due to the
need for proofs that the usage contexts are handled correctly. For
example, the following term is needed to show that, in the $\{\gr0,
\gr1, \gr\omega\}$ (linearity) posemiring of \cref{def:lin-semiring},
$\oc\gr\omega$ forms a comonad.
Pattern synonyms \AgdaInductiveConstructor{$\multimap$I},
\AgdaInductiveConstructor{!E$'$}, and
\AgdaInductiveConstructor{!I$'$} stand for applications of
\AgdaInductiveConstructor{`con}, with the latter two taking explicit usage
contexts and proofs.
On concrete
posemirings (as in this example), unification is particularly poor at
inferring the usage contexts from the proofs because addition and
multiplication are no longer (judgementally) injective.
The function \AgdaFunction{var\#} is a way of turning a statically known de
Bruijn level and a usage proof into an application of \AgdaInductiveConstructor{`var}.

\ExecuteMetaData[\HeavyItex]{cojoin-explicit}

 Writing terms like this
is clearly unsustainable. We will see a way of automating the
necessary proofs via a \AgdaRecord{System}-generic elaborator in
\cref{sec:usage-elaborator}.

%Here is an example term, using the \AgdaFunction{$\lambda$R} system.
%First, for ease of writing, we introduce pattern synonyms for each of the
%typing rules we use.

%\ExecuteMetaData[\PaperExamplestex]{patterns}

%Our example term is a function that flips a tagged union wrapped in an
%arbitrarily annotated \emph{bang}.
%Much of the effort in writing such a term goes into writing the various
%relatedness proofs between usage contexts --- observing, for example, that two
%usage contexts sum together to make a third, or that a usage context used for
%a variable is a basis vector.
%We give a method of automating these proofs in \cref{sec:usage-elaborator}.
%\todo{To be clear, we don't actually write this.}

%\ExecuteMetaData[\HeavyItex]{lR-term}

% A layer of syntax supports the following functorial action.

% \ExecuteMetaData[\Maptex]{map-s-type}

\section{More example syntaxes}\label{sec:other-syntaxes}

With the range of representable syntaxes now formalised, we can explore encoding
techniques for syntaxes more exotic than ST$\lambda$C and $\name$.
As well as the variations presented in \cref{sec:variant}, we can represent a
usage-annotated $\mu\tilde\mu$-calculus and a Linear/non Linear system.

\subsection{An encoding of graphs}

As a non-logical example of a syntax that can be encoded using linear syntax
descriptions, I consider a language of directed acyclic hypergraphs, which can
be used as string diagrams for symmetric monoidal categories.
I want to represent hypergraphs like the following, made up of operators
$\{
\tikz{\node (n) [circle,draw,minimum size=2mm,inner sep=0] {};%
  \draw[->] (n) to (0,-3mm)},
\tikz{\node (n) [circle,draw,minimum size=2mm,inner sep=0] {};%
  \draw[->] (2.12mm,2.12mm) to (n); \draw[->] (-2.12mm,2.12mm) to (n);%
  \draw[->] (n) to (0,-3mm)},
\tikz{\node (n) [circle,fill,minimum size=2mm,inner sep=0] {};%
  \draw[->] (0,3mm) to (n)},
\tikz{\node (n) [circle,fill,minimum size=2mm,inner sep=0] {};%
  \draw[->] (n) to (2.12mm,-2.12mm); \draw[->] (n) to (-2.12mm,-2.12mm);%
  \draw[->] (0,3mm) to (n)}
\}$
and wires between them with no fan-in or fan-out.
I have given the wires names (a, b, w, x, y, and z), which I use in a textual
representation of the graph on the right.
The textual representation is in A-normal form, where the two nullary operators
have names beginning with \texttt{eta}, the two binary operators have names
beginning with \texttt{mu}, and names having the suffix \texttt{white} or
\texttt{black} based on the corresponding operator's colour.

\noindent
\begin{minipage}[t]{0.5\textwidth}
  \begin{tikzpicture}[baseline]
    \node (i0) {};
    \node[right=2cm of i0] (i1) {};
    \node[below=1cm of i0] (mub) [circle,fill] {};
    \node[below right=1cm of mub] (muw) [circle,draw] {};
    \node[below left=1cm of muw] (etaw) [circle,draw] {};
    \node[below right=1cm of etaw] (etab) [circle,fill] {};
    \node[below left=1cm of etab] (o0) {};
    \node[below right=1cm of etab] (o1) {};
    \draw[->] (i0) to node[auto] {a} (mub);
    \draw[->] (i1) to[out=-90,in=45] node[auto] {b} (muw);
    \draw[->] (mub) to node[auto] {x} (muw);
    \draw[->] (muw) to node[above right] {y} (etab);
    \draw[->] (etaw) to node[auto] {z} (o0);
    \draw[->] (mub) to[out=-135,in=90] node[left,near start] {w} (o1);
  \end{tikzpicture}
\end{minipage}
\begin{minipage}[t]{0.5\textwidth}
  \vspace{1.5em}
\begin{verbatim}
a, b |-
let w, x := mu_black(a) in
let y := mu_white(x, b) in
let z := eta_white() in
let := eta_black(y) in
z, w
\end{verbatim}
\end{minipage}

In the textual representation, I treat inputs a and b as free variables, while
outputs z and w are listed on the final line.
The rest of the expression is a series of \texttt{let}-expressions binding zero
or more variables to the results of an operator applied to other variables.
All of the variables are used linearly, avoiding any fan-out.

To turn this representation into a type system, I do the following.
First, I fix the usage annotation semiring as the
$\{\gr0, \gr1, \gr\omega\}$-semiring to achieve linearity.
Then, I set \AgdaBound{Ty} to be the following type.
All of the variables get type \AgdaInductiveConstructor{wire}, with a context
full of \AgdaInductiveConstructor{wire}-type variables corresponding to the
inputs of the graph.
Meanwhile, the whole expression has type
\AgdaInductiveConstructor{bundle}\AgdaSpace{}\AgdaBound{s}, where \AgdaBound{s}
gives the shape of the outputs.

\ExecuteMetaData[\Graphtex]{Sort}

The type system itself is as follows.
Each operator gets its own syntactic construct, with essentially the same naming
conventions as for the textual representation above.
Each of these constructs is an entire \texttt{let}-expression.
There is also an \AgdaInductiveConstructor{`end} construct, allowing us to
finish and list the outputs.

To see how the encodings of the operations work, let us look at the
\AgdaInductiveConstructor{`$\upmu\bullet$}-construct
($\tikz{\node (n) [circle,fill,minimum size=2mm,inner sep=0] {};%
  \draw[->] (n) to (2.12mm,-2.12mm); \draw[->] (n) to (-2.12mm,-2.12mm);%
  \draw[->] (0,3mm) to (n)}$).
It has two premises conjoined by $\sep$ (in fact, all premises in this language
are separated, with no use of sharing conjunction).
The first premise expects a simple value, corresponding to the input of the
operator.
The second premise is more complex, and represents the body/continuation of the
\texttt{let}-expression.
It binds two variables, corresponding to the two outputs, which can then be used
in later \texttt{let}-expressions or the outputs.
The bound variables all have type \AgdaInductiveConstructor{wire} and usage
annotation $\gr1$ (written \AgdaInductiveConstructor{u1} in Agda) to make them
behave linearly.

\ExecuteMetaData[\Graphtex]{qGraphL}
\ExecuteMetaData[\Graphtex]{GraphL}

The required premises of the \AgdaInductiveConstructor{`end}-rule are calculated
by \AgdaFunction{end-premises}.
This follows the tree structure of the shape, requiring a
\AgdaInductiveConstructor{wire}-term in an unextended context for each leaf.

\ExecuteMetaData[\Graphtex]{end-premises}

A graph is a term of the graph language.
Specifically, a graph with inputs of shape \AgdaBound{s} and outputs of shape
\AgdaBound{t} has the following type --- a term in a context of shape
\AgdaBound{s}, all of whose entries are wire variables with annotation $\gr1$,
whose result type is \AgdaInductiveConstructor{bundle}\AgdaSpace{}\AgdaBound{t}.

\ExecuteMetaData[\Graphtex]{Graph}

Finally, I give an example term \AgdaFunction{myGraph}.
As I said at the end of \cref{sec:terms-of-system}, writing a term out in full
is tedious, so I instead choose to use the machinery I discuss in detail in
\cref{sec:usage-elaborator}.
The \AgdaFunction{elab-unique} tool elaborates a well typed term into a well
typed and usage-correct term as long as it can infer assignments of usage
annotations that satisfy the constraints.
I add the prefix \texttt{u} to signify the unannotated (just well typed) terms.
All of the checks required by the elaboration procedure are done at Agda's type
checking time, so we know that the program listed below can be elaborated by
virtue of the whole Agda definition being type-checked.

The \AgdaFunction{uvar\#} tool is like the \AgdaFunction{var\#} tool I used
earlier --- allowing us to refer to variables by counting from the lefthand end
of the context.
I add these numbers to the picture of the graph for convenience.
Variables never go out of scope, despite being used, so these numbers do not
change with any bindings.

\noindent
\begin{minipage}[t]{0.33\textwidth}
  \begin{tikzpicture}[baseline]
    \node (i0) {};
    \node[right=2cm of i0] (i1) {};
    \node[below=1cm of i0] (mub) [circle,fill] {};
    \node[below right=1cm of mub] (muw) [circle,draw] {};
    \node[below left=1cm of muw] (etaw) [circle,draw] {};
    \node[below right=1cm of etaw] (etab) [circle,fill] {};
    \node[below left=1cm of etab] (o0) {};
    \node[below right=1cm of etab] (o1) {};
    \draw[->] (i0) to node[auto] {0} (mub);
    \draw[->] (i1) to[out=-90,in=45] node[auto] {1} (muw);
    \draw[->] (mub) to node[auto] {3} (muw);
    \draw[->] (muw) to node[above right] {4} (etab);
    \draw[->] (etaw) to node[auto] {5} (o0);
    \draw[->] (mub) to[out=-135,in=90] node[left,near start] {2} (o1);
  \end{tikzpicture}
\end{minipage}
\begin{minipage}[t]{0.67\textwidth}
  \AgdaNoSpaceAroundCode{}
  \ExecuteMetaData[\Graphtex]{myGraph}
  \AgdaSpaceAroundCode{}
\end{minipage}

\subsection{The system $\mu\tilde\mu$}\label{sec:mu-mu-tilde}
I encode a usage-annotated version of System $L$/the
$\mu\tilde\mu$-calculus~\citep{CH00} --- a syntax for classical logic --- in
such a way that contexts capture the undistinguished parts of the sequent.
As such, the generic substitution lemma we get in \cref{sec:kit-to-sem} is the
form of substitution required in standard $\mu\tilde\mu$-calculus metatheory.
Though the $\mu\tilde\mu$-calculus is originally described as a sequent
calculus~\citep{CH00}, I use the techniques of
\citet[p.~12]{herbelin-hab} and \citet{LC06} to present it using hypothetical
judgements, thus giving a notion of \emph{variable} to the system.

Unlike the single judgement form of \name{} and standard simply typed
$\lambda$-calculi, the $\mu\tilde\mu$-calculus has three judgement forms:
terms (traditional notation: $\Gamma \vdash A \mid \Delta$), coterms
($\Gamma \mid A \vdash \Delta$), and commands ($\Gamma \vdash \Delta$).
Read logically, terms and coterms are seen to, respectively, prove and refute
propositions (types), while commands exhibit contradictions.
This means that the abstract \AgdaBound{Ty} in the generic framework is
instantiated to \AgdaDatatype{Conc} (for \emph{conclusion}) as below, with
\AgdaDatatype{Ty} not being exposed directly to the generic framework.
For now, I just consider multiplicative disjunction $\parr$ (\emph{par}) and
negation/duality, beside an uninterpreted base type.
These are enough to exhibit classical behaviour.

\noindent
\begin{minipage}[t]{0.5\textwidth}
  \ExecuteMetaData[\MuMuTildetex]{Ty}
\end{minipage}
\begin{minipage}[t]{0.5\textwidth}
  \ExecuteMetaData[\MuMuTildetex]{Conc}
\end{minipage}

With \AgdaBound{Ty} instantiated as \AgdaDatatype{Conc}, all terms are assigned
\AgdaDatatype{Conc} type, as are all the variables.
No variables are given \AgdaInductiveConstructor{com} type, similar to how in
the bidirectional typing syntax of \citet[p.~25]{AACMM21}, no variables are
given \AgdaInductiveConstructor{Check} type.
How to observe this invariant is covered in the latter paper, so we will not
repeat it here (having not yet seen how to write traversals on terms).

The syntax comprises a \emph{cut} between a term and a coterm of the same type,
the eponymous $\mu$ and $\tilde\mu$ constructs for proof by contradiction, and
then term and coterm (introduction and elimination) forms for negation and
\emph{par}.
I give a traditional presentation of this fragment of the
$\mu\tilde\mu$-calculus in \cref{fig:mmt-orig}, with my encoding of the rules
below.

\ExecuteMetaData[\MuMuTildetex]{MMT}

\begin{figure}
  \begin{mathpar}
    \ebrule{%
      \hypo{\grP \leq \grPl + \grPr}
      \hypo{\grQ \leq \grQl + \grQr}
      \hypo{\grPl\gamma \vdash A \mid \grQl\delta}
      \hypo{\grPr\gamma \mid A \vdash \grQr\delta}
      \infer4[Cut$_A$]{\grP\gamma \vdash \grQ\delta}
    } \and
    \ebrule{%
      \hypo{\Gamma \vdash \gr1A, \Delta}
      \infer1[$\mu$]{\Gamma \vdash A \mid \Delta}
    } \and
    \ebrule{%
      \hypo{\Gamma, \gr1A \vdash \Delta}
      \infer1[$\tilde\mu$]{\Gamma \mid A \vdash \Delta}
    } \and
    \ebrule{%
      \hypo{\Gamma \mid A \vdash \Delta}
      \infer1[$\lambda$]{\Gamma \vdash A^\bot \mid \Delta}
    } \and
    \ebrule{%
      \hypo{\Gamma \vdash A \mid \Delta}
      \infer1[$\tilde\lambda$]{\Gamma \mid A^\bot \vdash \Delta}
    } \and
    \ebrule{%
      \hypo{\grP \leq \gr r\grPl + \gr s\grPr}
      \hypo{\grQ \leq \gr r\grQl + \gr s\grQr}
      \hypo{\grPl\gamma \vdash A \mid \grQl\delta}
      \hypo{\grPr\gamma \vdash B \mid \grQr\delta}
      \infer4[$\alr{-,-}$]%
      {\grP\gamma \vdash \gr rA \parr \gr sB \mid \grQ\delta}
    } \and
    \ebrule{%
      \hypo{\Gamma, \gr rA, \gr sB \vdash \Delta}
      \infer1[$\mu\alr{-,-}$]{\Gamma \mid \gr rA \parr \gr sB \vdash \Delta}
    }
  \end{mathpar}
  \caption{A fragment of a usage-annotated $\mu\tilde\mu$-calculus presented in
    traditional sequent notation}
  \label{fig:mmt-orig}
\end{figure}

%With a collection of pattern synonyms and the machinery from
%\cref{sec:usage-elaborator}, we can write an example term: a function which
%flips the disjuncts of a \emph{par}.

%\ExecuteMetaData[\MuMuTildeTermtex]{patterns}
%\ExecuteMetaData[\MuMuTildeTermtex]{myComm}

\subsection{Duplicability and L/nL}\label{sec:dup-lnl}
In \cref{sec:rec}, I introduced a bunched connective $\Box^{0{+}}$, used to
ensure that such a premise is derived in a context that allows it to be
duplicated and discarded.
Syntax for this connective can be added to \AgdaDatatype{Premises} and worked
through the various parts of the framework, though I do not show these parts in
the text of this thesis for the sake of simplicity and brevity.
Additionally, variations such as $\Box^0$ (requiring only that the context is
${} \leq \gr0$) and $\Box^{0{+}{*}}$ (requiring, in addition to
$\Box^{0{+}}$, closure under multiplication by any scalar) can be added to the
framework independently of each other and of $\Box^{0{+}}$.

As well as the use case of inductive types, as shown in \cref{sec:rec}, I use
the $\Box^{0{+}{*}}$-modality to encode the restriction to intuitionistic
variables in the intuitionistic judgement forms of the L/nL
calculus~\citep{Benton94}.
I will explain this in detail in due course.

Linear/non-Linear Logic has two sorts of types: linear types $A, B, C$ and
intuitionistic types $X, Y, Z$.
I encode these two sorts in the Agda types
\AgdaDatatype{Ty}\AgdaSpace{}\AgdaInductiveConstructor{lin} and
\AgdaDatatype{Ty}\AgdaSpace{}\AgdaInductiveConstructor{int}, respectively.
Using an indexed type lets me take the total space of \AgdaDatatype{Ty}, named
\AgdaFunction{$\Sigma$Ty}, as the evident dependent pair type.
\AgdaFunction{$\Sigma$Ty} becomes the type of types seen by the framework.
The types themselves are, on the linear side,
a tensor unit \AgdaInductiveConstructor{tI},
a tensor product \AgdaInductiveConstructor{\_t$\otimes$\_}, and
a linear function space \AgdaInductiveConstructor{\_t$\multimap$\_};
on the intuitionistic side,
a unit \AgdaInductiveConstructor{t1},
a product \AgdaInductiveConstructor{\_t$\times$\_}, and
an intuitionistic function space \AgdaInductiveConstructor{\_t$\to$\_};
and adjoint type formers \AgdaInductiveConstructor{tF} and
\AgdaInductiveConstructor{tG}.
I also include a linear base type \AgdaInductiveConstructor{$\iota$}.

\ExecuteMetaData[\LnLtex]{Frag}
\ExecuteMetaData[\LnLtex]{Ty}
%\ExecuteMetaData[\LnLtex]{SigTy}

As in the other examples, I define a data type of rule labels
\AgdaDatatype{`LnL} before defining the type system \AgdaFunction{LnL}.
The system \AgdaFunction{LnL} has types \AgdaFunction{$\Sigma$Ty} and usage
annotations $\{\gr0, \gr1, \gr\omega\}$, as I have used before for linear
systems.
For reference, I give a standard presentation of the rules of L/nL in
\cref{fig:LnL-orig}.

\begin{figure}
  \begin{align*}
    A, B, C &\Coloneqq I \mid A \otimes B \mid A \multimap B \mid FX \\
    X, Y, Z &\Coloneqq 1 \mid X \times Y \mid X \to Y \mid GA \\
    \mathcal J &\Coloneqq \; \Theta; \Gamma \vdashL A \; \mid \; \Theta \vdashC X
  \end{align*}
  \begin{mathpar}
    \ebrule{%
      \infer0[$\mathcal L$-var]{\Theta; A \vdashL A}
    } \and
    \ebrule{%
      \infer0[$\mathcal C$-var]{\Theta, X \vdashC X}
    } \and
    \ebrule{%
      \infer0[$I$i]{\Theta; {\cdot} \vdashL I}
    } \and
    \ebrule{%
      \hypo{\Theta; \Gamma \vdashL I}
      \hypo{\Theta; \Delta \vdashL C}
      \infer2[$I$e]{\Theta; \Gamma, \Delta \vdashL C}
    } \and
    \ebrule{%
      \hypo{\Theta; \Gamma \vdashL A}
      \hypo{\Theta; \Delta \vdashL B}
      \infer2[$\otimes$i]{\Theta; \Gamma, \Delta \vdashL A \otimes B}
    } \and
    \ebrule{%
      \hypo{\Theta; \Gamma \vdashL A \otimes B}
      \hypo{\Theta; \Delta, A, B \vdashL C}
      \infer2[$\otimes$e]{\Theta; \Gamma, \Delta \vdashL C}
    } \and
    \ebrule{%
      \hypo{\Theta; \Gamma, A \vdashL B}
      \infer1[$\multimap$i]{\Theta; \Gamma \vdashL A \multimap B}
    } \and
    \ebrule{%
      \hypo{\Theta; \Gamma \vdashL A \multimap B}
      \hypo{\Theta; \Delta \vdashL A}
      \infer2[$\multimap$e]{\Theta; \Gamma, \Delta \vdashL B}
    } \and
    \ebrule{%
      \hypo{\Theta \vdashC X}
      \infer1[$F$i]{\Theta; {\cdot} \vdashL FX}
    } \and
    \ebrule{%
      \hypo{\Theta; \Gamma \vdashL FX}
      \hypo{\Theta, X; \Delta \vdashL C}
      \infer2[$F$e]{\Theta; \Gamma, \Delta \vdashL C}
    } \and
    \ebrule{%
      \infer0[$1$i]{\Theta \vdashC 1}
    } \and
    \text{(no \TirName{$1$e})} \and
    \ebrule{%
      \hypo{\Theta \vdashC X}
      \hypo{\Theta \vdashC Y}
      \infer2[$\times$i]{\Theta \vdashC X \times Y}
    } \and
    \ebrule{%
      \hypo{\Theta \vdashC X_0 \times X_1}
      \infer1[$\times$e$_i$]{\Theta \vdashC X_i}
    } \and
    \ebrule{%
      \hypo{\Theta, X \vdashC Y}
      \infer1[$\to$i]{\Theta \vdashC X \to Y}
    } \and
    \ebrule{%
      \hypo{\Theta \vdashC X \to Y}
      \hypo{\Theta \vdashC X}
      \infer2[$\to$e]{\Theta \vdashC Y}
    } \and
    \ebrule{%
      \hypo{\Theta; {\cdot} \vdashL A}
      \infer1[$G$i]{\Theta \vdashC GA}
    } \and
    \ebrule{%
      \hypo{\Theta \vdashC GA}
      \infer1[$G$e]{\Theta; {\cdot} \vdashL A}
    }
  \end{mathpar}
  \caption{Linear/non-Linear Logic in traditional sequent notation}
  \label{fig:LnL-orig}
\end{figure}

We want to be able to embed the L/nL judgement forms
$\Theta; \Gamma \vdashL A$ and $\Theta \vdashC X$, where
$\Theta$ is made of intuitionistic variables and $\Gamma$ is made of linear
variables.
I take the subscript on $\vdashL$ and $\vdashC$ to be a redundant annotation of
whether the conclusion type is \AgdaInductiveConstructor{lin} or
\AgdaInductiveConstructor{int}, so it does not appear in the Agda code, but I
retain it in all non-mechanised presentations.

I will generally ensure that all intuitionistic variables will all have usage
annotation $\gr\omega$, while linear variables will each be annotated
either $\gr1$ or $\gr0$.
These assignments of usage annotations are enforced wherever the rules bind new
variables.
By usage subsumption, intuitionistic variables may also end up with annotation
$\gr1$ or $\gr0$, but we see this as a (meaningless) refinement of the
$\gr\omega$ annotation, not causing a soundness problem.
The key invariant is that linear variables never get annotation $\gr\omega$.

The key difference in form between linear and intuitionistic sequents is that
intuitionistic sequents cannot contain linear variables.
However, syntax descriptions give us no ability to directly state such a
restriction on the kinds of variables.
Therefore, any encoded syntax will allow us to propose ill formed sequents like
$\gr1A \vdashC X$, where a linear variable is to be used for an intuitionistic
conclusion.
To make sure that this is not a problem I use the $\Boxzpt$ bunched modality to
guard many of the rules, making sure that variables with usage annotation $\gr1$
cannot occur in \emph{provable} intuitionistic sequents.
Specifically, every rule targeting an intuitionistic judgement has its premises
wrapped in a $\Boxzpt$.
Additionally, rules \TirName{$F$i} and \TirName{$G$e}, which target linear
judgements but have intuitionistic premises, similarly get a $\Boxzpt$.
Note that having the annotation $\gr0$ means that well formed intuitionistic
sequents can contain linear variables annotated $\gr0$, but these variables are
not usable.

For ease of \emph{producing} terms, we may prefer a ``garbage in; garbage out''
style with a minimal number of uses of $\Boxzpt$, which could be achieved in two
different ways:
\begin{itemize}
  \item
    Ensure that no sequent of the form $\Gamma, \gr1A \vdashC X$ is derivable.
    This leaves boxes only on rules \TirName{$1$-I} and \TirName{$G$-I} --- the
    two logical rules where an intuitionistic conclusion is derived without any
    intuitionistic premises.
  \item
    Ensure that any rule targeting a well formed conclusion has only well formed
    premises.
    This leaves boxes only on rules \TirName{$F$-I} and \TirName{$G$-E} --- the
    two rules where a linear conclusion is derived from intuitionistic premises.
\end{itemize}
It would be interesting to show the equivalence of these two approaches with the
one I have taken in this thesis, but that is left to future work.

The rules are presented in Agda code below, and also using bunched inference
rule notation in \cref{fig:LnL-bunched}.

\begin{figure}
  \begin{mathpar}
    \ebrule[comb]{%
      \hypo{I^{\sep}}
      \infer1[$I$i]{\vdashL I}
    } \and
    \ebrule[comb]{%
      \hypo{\vdashL I}
      \hypo{\sep}
      \hypo{\vdashL C}
      \infer3[$I$e]{\vdashL C}
    } \and
    \ebrule[comb]{%
      \hypo{\vdashL A}
      \hypo{\sep}
      \hypo{\vdashL B}
      \infer3[$\otimes$i]{\vdashL A \otimes B}
    } \and
    \ebrule[comb]{%
      \hypo{\vdashL A \otimes B}
      \hypo{\sep}
      \hypo{\gr1A, \gr1B \vdashL C}
      \infer3[$\otimes$e]{\vdashL C}
    } \and
    \ebrule[comb]{%
      \hypo{\gr1A \vdashL B}
      \infer1[$\multimap$i]{\vdashL A \multimap B}
    } \and
    \ebrule[comb]{%
      \hypo{\vdashL A \multimap B}
      \hypo{\sep}
      \hypo{\vdashL A}
      \infer3[$\multimap$e]{\vdashL B}
    } \and
    \ebrule[comb]{%
      \hypo{\Box\plr{\vdashC X}}
      \infer1[$F$i]{\vdashL FX}
    } \and
    \ebrule[comb]{%
      \hypo{\vdashL FX}
      \hypo{\sep}
      \hypo{\gr\omega X \vdashL C}
      \infer3[$F$e]{\vdashL C}
    } \and
    \ebrule[comb]{%
      \hypo{\Box\dot1}
      \infer1[$1$i]{\vdashC 1}
    } \and
    \text{(no \TirName{$1$e})} \and
    \ebrule[comb]{%
      \hypo{\Box(\vdashC X}
      \hypo{\dottimes}
      \hypo{\vdashC Y)}
      \infer3[$\times$i]{\vdashC X \times Y}
    } \and
    \ebrule[comb]{%
      \hypo{\Box\plr{\vdashC X_0 \times X_1}}
      \infer1[$\times$e$_i$]{\vdashC X_i}
    } \and
    \ebrule[comb]{%
      \hypo{\Box\plr{\gr\omega X \vdashC Y}}
      \infer1[$\to$i]{\vdashC X \to Y}
    } \and
    \ebrule[comb]{%
      \hypo{\Box(\vdashC X \to Y}
      \hypo{\dottimes}
      \hypo{\vdashC X)}
      \infer3[$\to$e]{\vdashC Y}
    } \and
    \ebrule[comb]{%
      \hypo{\Box\plr{\vdashL A}}
      \infer1[$G$i]{\vdashC GA}
    } \and
    \ebrule[comb]{%
      \hypo{\Box\plr{\vdashC GA}}
      \infer1[$G$e]{\vdashL A}
    }
  \end{mathpar}
  \caption{Linear/non-Linear Logic in bunched notation using
    $\{\gr0, \gr1, \gr\omega\}$ usage annotations and with $\Boxzpt$ abbreviated
    to $\Box$}
  \label{fig:LnL-bunched}
\end{figure}

\ExecuteMetaData[\LnLtex]{quoteLnL}
\ExecuteMetaData[\LnLtex]{LnL}

In \cref{sec:lnl}, I will give translations between the original L/nL system,
this encoding of L/nL, and $\name$, providing some assurance that the correct
logic has been encoded.

%There is one more bunched combinator we have experimented with adding to the
%framework:
%
%\[
%  \plr{\Box T}\,\grR \coloneqq \Sigma\grRprime.~\plr{\grRprime \leq \grR}
%  \times \plr{\grRprime \leq \gr0}
%  \times \plr{\grRprime \leq \grRprime + \grRprime}
%  \times T\,\grRprime
%\]
%
%The idea of $\plr{\Box T}\,\grR$ is to assert that $\grR$, or some refinement
%of it, can be both discarded and duplicated indefinitely, and in the
%refinement we have a $T$.
%We use this combinator to introduce subterms that are used an unknown number of
%times, for example the continuations of the eliminator of an inductive type,
%or other fixed points.
%We can also use it in linear/non-linear style systems~\cite{Benton94} to make
%sure linear variables are not available in the intuitionistic fragment.
%
%Adding the $\Box$ combinator is the only thing we have found that requires our
%linear maps be functional rather than merely relational.

\section{Conclusion}

I have shown via examples that the syntax descriptions defined in this chapter
can capture a significant range of simply typed usage-aware calculi.
Together with the examples of posemirings given in \cref{sec:semirings}, the
whole framework is capable of encoding many specific substructural calculi.
However, the language of descriptions has several restrictions, with more and
less clear practical consequences.
Many of these restrictions are shared with the syntax descriptions of
\citet{AACMM21}, but some pertain to usage annotations.

One clear limitation is that the forms of judgements do not allow enough
interdependency to capture dependently typed calculi.
In particular, for any given calculus, there is one fixed set of types, with no
indexing over contexts.
This allows contexts to have a flat list-like structure rather than a telescope,
and means that the type of a term does not depend on that context, but clearly
restricts the expressiveness of the calculi which can be encoded.
As well as dependently typed calculi, this also prevents the encoding of the
well formed types of System $F_\omega$.

Less clear than describability of System $F_\omega$ is describability of System
$F$.
Though I have not presented the details, I believe it should be possible to
encode calculi with polymorphism, such as System $F$ and the usage-annotated
system found in \citet{AbelBernardy2020}, via a two-stage process.
First, we make a syntax description for the language of types, and then use
the well typed terms arising from that description as the set of types when
making the description for the System $F$ terms.
Such an encoding would yield all of the benefits of the framework presented in
this thesis separately to the language of types and the language of terms, but
how these two languages interact is unclear.
For example, in \cref{sec:semantics}, I provide a syntax-generic substitution
operation.
This operation would yield substitution for type variables in types and
substitution for term variables in terms, but substitution for type variables in
terms would have to be provided separately, and would likely be more
complicated.
It would be interesting to see future work in this direction, possibly inspired
by how Autosubst~\citep{Autosubst15} handles the same issue.

As for restrictions peculiar to the usage-aware setting, we of course have the
general limitations of posemiring annotations listed in
\cref{sec:semirings-conc}, which I will not repeat here.
Additionally, there are things we could wish to do with posemiring-based usage
annotations which we do not have access to with the descriptions given in this
chapter.
For example, an issue one may have with the encoding of Linear/non-Linear Logic
given in \cref{sec:dup-lnl} is that linear variables and intuitionistic
variables, despite having morally distinct usage disciplines, necessarily share
the $\{\gr0, \gr1, \gr\omega\}$ posemiring of usage annotations.
If one wanted to generalise this construction to different pairs of usage
disciplines, one may well find that they do not combine adequately into a single
posemiring.
A possible solution would be to have a range of kinds, and for each kind to have
its own set of types and its own posemiring --- for example, a \emph{linear}
kind and an \emph{intuitionistic} kind, using $\{\gr0, \gr1, \gr\omega\}$ and
the 1-element posemiring, respectively.
However, the descriptions given in this thesis only allow for one posemiring.

Finally, while there is a simple and well motivated core of premise combinators
--- $\dot1$, $\dottimes$, $I^{\sep}$, $\sep$, and $\gr r \cdot \plr{-}$ --- the
various $\Box$-modalities feel like ad hoc additions.
While I have shown that $\Boxzp$ and $\Boxzpt$ have several practical uses, it
is unclear whether there should be more modalities.
Furthermore, I have not made an effort to isolate the properties that make these
modalities well behaved.
We will see in \cref{sec:functorial} that any such modality will need to be
functorial in the appropriate sense, but other examples of generic programming,
such as the usage elaborator which will be detailed in
\cref{sec:usage-elaborator}, may need further properties.
