We take the insights of the previous section and use them to build a
generic framework for posemiring-annotated substructural systems in
Agda. We will first show \emph{descriptions} of systems, which are
comprised of rules that have premises combined using the bunched
combinators. We then show how to construct the Agda data type of
intrinsically well scoped, typed, and resourced terms for any given
system of our framework. We use the prototypical system from
\cref{fig:lr-comb} as a running example. \cref{sec:other-syntaxes}
presents further examples that our framework can express.

We now start to use Agda notation for record and data type
declarations, to emphasise that our framework has been implemented.

\subsection{Descriptions of Systems}

% We capture the form of rules exemplified previously\todo{Previously?} via
% \emph{descriptions} of rules.
% The key to making these descriptions work is that they only allow syntactic
% forms that preserve environments.
% These forms are: absence and multiplicity of subterms with the same usage
% annotations, absence and multiplicity of subterms with summed usage annotations,
% scaling of a subterm, and variable binding.\todo{Switching to Agda}

\paragraph{\AgdaDatatype{Premises}, \AgdaRecord{Rule}s, and \AgdaRecord{System}s.}

A type \AgdaRecord{System} is made up of multiple \AgdaRecord{Rule}s.
Each \AgdaRecord{Rule} comprises a tree of \AgdaDatatype{Premises} and
a type of conclusion. We assume that there is a
$\AgdaBound{Ty} : \AgdaPrimitiveType{Set}$ of types for the system in
scope.

The \AgdaDatatype{Premise} data type describes premises of rules,
using the bunched combinators from the previous section. A single
premise is introduced by the
\AgdaInductiveConstructor{$\langle$\_`$\vdash$\_$\rangle$}
constructor.  This allows binding of additional variables
\AgdaBound{$\Delta$} (with specified types and usage annotations) and
the specification of a conclusion type \AgdaBound{A} for this premise.
The remaining constructors are descriptions for the first-order
bunched connectives, and will be interpreted directly as such, below.

\ExecuteMetaData[\Syntaxtex]{Premises}

A \AgdaRecord{Rule} is a pair of some \AgdaDatatype{Premises} and a
conclusion. We use an infix arrow as a suggestive notation for rules.

\ExecuteMetaData[\Syntaxtex]{Rule}

Finally, a \AgdaRecord{System} consists of a set of rule labels (i.e.,
constructor names), and for each label a decsription of the
corresponding rule. We use $\rhd$ as infix notation for systems to
associate the label set with the rules.

\ExecuteMetaData[\Syntaxtex]{System}

\paragraph{\cref{fig:lr-comb} as a \AgdaRecord{System}.}

As an example, we transcribe the system defined in
\cref{fig:lr-comb} into a description.  We give the set of types of
this system as a data type \AgdaDatatype{Ty} (together with a base
type \AgdaInductiveConstructor{$\iota$}). We assume that there is a
posemiring \AgdaInductiveConstructor{Ann} in scope for the
annotations.There is one label for each instantiation of a logical
rule, but the labels contain no further information about subterms or
restrictions on the context. This will be provided when we associate
labels with \AgdaRecord{Rule}s in a \AgdaRecord{System}.

\noindent
\begin{minipage}[t]{0.5\textwidth}
  \ExecuteMetaData[\PaperExamplestex]{Ty}
  \ExecuteMetaData[\PaperExamplestex]{Side}
\end{minipage}
\begin{minipage}[t]{0.5\textwidth}
  \ExecuteMetaData[\PaperExamplestex]{qlR}
\end{minipage}

To build a system, we associate with each label a rule:

\ExecuteMetaData[\PaperExamplestex]{lR}

Compared to \cref{fig:lr-comb}, modulo the Agda notation, we can see
that the fundamental structure has been preserved: the rules match
one-to-one, and the bunched premises are the same. A major difference
is that we do not include a counterpart to the
\AgdaInductiveConstructor{var} rule in a
\AgdaRecord{System}. Variables are common to all the systems
representable in our framework.

\paragraph{Terms of a \AgdaRecord{System}.}

The next thing we want to do is to build terms in the described type system.
The following definitions are useful for talking about types indexed over
contexts, judgement forms, and judgement forms admitting newly bound variables,
respectively.

\ExecuteMetaData[\Syntaxtex]{OpenFam}

To specify the meaning of descriptions, we assume some \AgdaBound{X} : \AgdaFunction{ExtOpenFam},
% \ExecuteMetaData[\Interpretationtex]{X},
over which we form one layer of syntax, using the function
\AgdaFunction{$\llbracket$\_$\rrbracket$p} that interprets
\AgdaDatatype{Premises} defined below.  The first argument to
\AgdaBound{X} is the new variables bound by this layer of syntax, as
exemplified in the first clause of
\AgdaFunction{$\llbracket$\_$\rrbracket$p}.  The second argument is
the context containing the variables being carried over from the
previous layer.  Notice that this is not, in general, the same as the
context from the previous layer, because the usage annotations may
have been changed by connectives like
\AgdaInductiveConstructor{\_`$*$\_} and
\AgdaInductiveConstructor{\_`$\cdot$\_}.  The third argument is the
type of subterm required.

With the first clause of \AgdaFunction{$\llbracket$\_$\rrbracket$p} explained,
the rest are simply interpretations of premises into bunched combinators.

\ExecuteMetaData[\Interpretationtex]{semp}

The interpretation of a \AgdaRecord{Rule} checks that the rule targets
the desired type and then interprets the rule's premises \AgdaBound{ps}.
Notice that the interpretation of the premises is independent of the conclusion
of the rule, which accounts for the difference in type between
\AgdaFunction{$\llbracket$\_$\rrbracket$p} and
\AgdaFunction{$\llbracket$\_$\rrbracket$r}.

\ExecuteMetaData[\Interpretationtex]{semr}

The interpretation of a \AgdaRecord{System} is to choose a rule label
\AgdaBound{l} from \AgdaBound{L} and interpret the corresponding rule
\AgdaBound{rs}\AgdaSpace{}\AgdaBound{l} in the same context and for the same
conclusion.

\ExecuteMetaData[\Interpretationtex]{sems}

The most obvious way to make such an \AgdaBound{X} is to use some existing
\AgdaFunction{OpenFam} on an extended context.
We defined \AgdaFunction{Scope} to do this: take the new variables
\AgdaBound{$\Delta$}, concatenate them onto the existing context
\AgdaBound{$\Gamma$}, and pass the extended context onto the judgement
\AgdaBound{T}.

\ExecuteMetaData[\Syntaxtex]{Scope}

%{\color{red}(Forward ref: for now, we could have inlined \texttt{Scope}.)}

We use \AgdaFunction{Scope} to deal with new variables in syntax.
Terms resemble the free monad over a layer-of-syntax functor, though
that picture is complicated by variable binding.  A term is either a
variable or a use of a logical rule together with terms for each of
the required subterms. The \AgdaFunction{Size} argument is where we
use sized types to convince Agda that this type is strictly positive.

\ExecuteMetaData[\Termtex]{Term}

Terms defined like this are still quite difficult to write, mainly because of
frequently changing usage contexts and the need for proofs that they all match
up.
We will see how to automate these proofs in \cref{sec:usage-elaborator}.

%Here is an example term, using the \AgdaFunction{$\lambda$R} system.
%First, for ease of writing, we introduce pattern synonyms for each of the
%typing rules we use.

%\ExecuteMetaData[\PaperExamplestex]{patterns}

%Our example term is a function that flips a tagged union wrapped in an
%arbitrarily annotated \emph{bang}.
%Much of the effort in writing such a term goes into writing the various
%relatedness proofs between usage contexts --- observing, for example, that two
%usage contexts sum together to make a third, or that a usage context used for
%a variable is a basis vector.
%We give a method of automating these proofs in \cref{sec:usage-elaborator}.
%\todo{To be clear, we don't actually write this.}

%\ExecuteMetaData[\HeavyItex]{lR-term}

% A layer of syntax supports the following functorial action.

% \ExecuteMetaData[\Maptex]{map-s-type}

\subsection{Other syntaxes and syntactic forms}\label{sec:other-syntaxes}

\paragraph{The system $\mu\tilde\mu$.}
We can encode a usage-annotated version of System $L$/the
$\mu\tilde\mu$-calculus~\cite{CH00} --- a syntax for classical logic --- in
such a way that contexts capture the undistinguished parts of the sequent.
As such, the generic substitution lemma we get in \cref{sec:kits} is the form
of substitution required in standard $\mu\tilde\mu$-calculus metatheory.
Though the $\mu\tilde\mu$-calculus is originally described as a sequent
calculus~\cite{CH00}, we use the techniques of
\citet[p.~12]{herbelin-hab} and \citet{LC06} to present it as a natural
deduction system, thus giving a notion of \emph{variable} to the system.

Unlike the single judgement form of \name{} and standard simply typed
$\lambda$-calculi, the $\mu\tilde\mu$-calculus has three judgement forms:
terms, coterms, and commands.
Read logically, terms and coterms are seen to, respectively, prove and refute
propositions (types), while commands exhibit contradictions.
This means that the abstract \AgdaBound{Ty} in the generic framework is
instantiated to \AgdaDatatype{Conc} (for \emph{conclusion}) as below, with
\AgdaDatatype{Ty} not being exposed directly to the generic framework.
For now, we just consider multiplicative disjunction $\parr$ (\emph{par}) and
negation/duality, beside an uninterpreted base type.
These are enough to exhibit classical behaviour.

\noindent
\begin{minipage}[t]{0.5\textwidth}
  \ExecuteMetaData[\MuMuTildetex]{Ty}
\end{minipage}
\begin{minipage}[t]{0.5\textwidth}
  \ExecuteMetaData[\MuMuTildetex]{Conc}
\end{minipage}

With \AgdaBound{Ty} instantiated as \AgdaDatatype{Conc}, all terms are assigned
\AgdaDatatype{Conc} type, as are all the variables.
No variables are given \AgdaInductiveConstructor{com} type, similar to how in
the bidirectional typing syntax of \citet[p.~25]{AACMM20}, no variables are
given \AgdaInductiveConstructor{Check} type.
How to observe this invariant is covered in the latter paper, so we will not
repeat it here (having not yet seen how to write traverals on terms).

The syntax comprises a \emph{cut} between a term and a coterm of the same type,
the eponymous $\mu$ and $\tilde\mu$ constructs for proof by contradiction, and
then term and coterm (introduction and elimination) forms for negation and
\emph{par}.

\ExecuteMetaData[\MuMuTildetex]{MMT}

%With a collection of pattern synonyms and the machinery from
%\cref{sec:usage-elaborator}, we can write an example term: a function which
%flips the disjuncts of a \emph{par}.

%\ExecuteMetaData[\MuMuTildeTermtex]{patterns}
%\ExecuteMetaData[\MuMuTildeTermtex]{myComm}

\paragraph{Duplicability}
There is one more bunched combinator we have experimented with adding to the
framework:

\[
  \plr{\Box T}\,\grR \coloneqq \Sigma\grRprime.~\plr{\grRprime \leq \grR}
  \times \plr{\grRprime \leq \gr0}
  \times \plr{\grRprime \leq \grRprime + \grRprime}
  \times T\,\grRprime
\]

The idea of $\plr{\Box T}\,\grR$ is to assert that $\grR$, or some refinement
of it, can be both discarded and duplicated indefinitely, and in the
refinement we have a $T$.
We use this combinator to introduce subterms that are used an unknown number of
times, for example the continuations of the eliminator of an inductive type,
or other fixed points.
We can also use it in linear/non-linear style systems~\cite{Benton94} to make
sure linear variables are not available in the intuitionistic fragment.

Adding the $\Box$ combinator is the only thing we have found that requires our
linear maps be functional rather than merely relational.
