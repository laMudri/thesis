In this chapter, I provide four example uses of semantic traversals:
a usage elaborator, a normalisation by evaluation algorithm, a denotational
semantics, and translations between calculi.
The reader is also encouraged to see the far greater range of examples in the
work of \citet{AACMM21}, which should adapt to our usage-annotated setting.
The usage elaborator (\cref{sec:usage-elaborator}) gives an example of a program
that is generic in both syntax and usage annotations, as well as being an
essential tool for practical use of the framework.
The normalisation by evaluation algorithm (\cref{sec:nbe}) is for a specific
syntax, but exercises parts of the generic semantic traversal --- particularly,
its handling of environments --- to good effect.
The denotational semantics in terms of world-indexed relations
(\cref{sec:den-sem}) presents an interesting semantics for semiring-annotated
type systems, but with minimal use of operations provided by the framework.
Finally, the translations between $\name$ and L/nL (\cref{sec:lnl}) both justify
the encoding of L/nL given in \cref{sec:dup-lnl} and give examples of syntactic
work benefiting from the generic renaming and substitution operations.


\section{A usage elaborator}\label{sec:usage-elaborator}

Using the constructs we have seen so far, producing example terms soon becomes
extremely tedious.
We can achieve some abbreviation by using pattern synonyms to wrap around
\AgdaInductiveConstructor{`con} expressions, but we still have to
produce essentially bespoke proofs whenever we use a usage-sensitive part of the
syntax.
The size of each of these proofs is roughly proportional to the number of free
variables, so the amount of proof we have to write grows roughly quadratically
with the size of terms.
An additional factor, which we can't see on paper but is nonetheless
significant, is that type checking time
for these proofs soon becomes prohibitive to interactive development.

Our aim in this section is to automate usage constraint proofs, making terms
both easier to write and more performant to check.
We invoke the automation by writing terms in a syntax where usage constraints
have been trivialised, and then use a semantic traversal over the simplified
syntax to try to produce a fully elaborated term in the original syntax.
We write the automation in a way that is generic in the syntax description, thus
avoiding repetition and facilitating the prototyping of new type systems.

The type of syntax descriptions depends on the type of usage annotations because
of variable binding.
For example, in the $\oc{\gr r}$-E rule of \cref{fig:lr-bunched}, the right
premise binds a new variable with annotation $\gr r$, where $\gr r$ is drawn
from the ambient posemiring.
The scaling connective also makes direct reference to the posemiring.
To produce a simplified syntax description, where usage constraints are
trivialised, we set the ambient posemiring to the 1-element $\mathbf0$
posemiring.
In contrast to syntax descriptions, even though types can contain usage
annotations, the type of types does not depend on the type of usage annotations.
This means that, in our simplified syntax, terms have types from the original
system even though variables have trivial usage annotations.
We define the $\mathbf0$ posemiring as follows, being careful to use the
0-field record type \AgdaRecord{$\top$} so that everything algebraic gets
solved by Agda's $\eta$-laws.
Indeed, in this very definition, all of the semiring operations and laws are
canonically inferred.

\ExecuteMetaData[\UsageChecktex]{0-poSemiring}

The elaboration process is monadic.
In particular, we use the \AgdaDatatype{List}/non-determinism monad to give
\emph{all} of the possible annotation choices on the free variables of a term.
We believe the commitment to multiple solutions is inherent when the syntax
contains \AgdaInductiveConstructor{`$\dot1$}.
For example, in the intermediate stages of elaborating
$\plr{\vdash \lambda x.~\plr{*,*}} : A \multimap \top \otimes \top$ with a
usage counting posemiring (assuming reasonable rules for $\top$ and $\otimes$),
it is unclear whether to use the variable $x$ in the left $*$ or the right $*$.
This uncertainty should be reflected in the final result.

The non-deterministic choices we make during elaboration are enumerated by
the fields of \AgdaRecord{NonDetInverses}.
These choices are driven by the typing rules and a candidate usage vector for
the conclusion.
For example, \AgdaField{+$^{-1}$}\AgdaSpace{}\AgdaBound{r} is needed when we
encounter a \AgdaInductiveConstructor{`$\sep$} in the syntax and the candidate
usage annotation we are considering is \AgdaBound{r}.
Then, \AgdaField{+$^{-1}$}\AgdaSpace{}\AgdaBound{r} is a list of pairs of
annotations \AgdaBound{p} and \AgdaBound{q} that \AgdaBound{r} can split into,
together with a proof of the splitting.
For \AgdaField{0\#$^{-1}$} and \AgdaField{1\#$^{-1}$}, inverses to constants,
we are given the candidate \AgdaBound{r} and typically return an empty list if
the constraint cannot be satisfied, or a singleton list containing a proof.
\AgdaField{*$^{-1}$} is used when we encounter scaling, in which case we know
both the scaling factor \AgdaBound{r} (from the syntax description) and the
candidate \AgdaBound{q}.
These inverse operations combine monadically (in fact, applicatively) to give
inverses to the vector operations of zero, addition, scaling, and basis.

\ExecuteMetaData[\UsageChecktex]{NonDetInverses}

We choose the \AgdaBound{$\V$} of our semantics to be (unannotated) variables.
For the \AgdaBound{$\C$}, we consider \emph{functions} from candidate usage
vectors \AgdaBound{R} to the list of elaborated derivations with usage
annotations given by \AgdaBound{R}.
The protocol this encodes is that the user will provide an unannotated term
together with a candidate usage context \AgdaBound{R}, and usage elaboration
returns a list of possible ways the term could be annotated such that the
conclusion has usage context \AgdaBound{R}.
The module name \AgdaModule{U} refers to the fact that we are taking the
ambient posemiring to be $\mathbf0$ in \AgdaFunction{OpenFam}.
The effect on \AgdaFunction{OpenFam} is that the usage annotations of any
contexts we consider are uninformative (hence the \AgdaSymbol{\_} on the left).

\ExecuteMetaData[\UsageChecktex]{C}

To traverse the unannotated terms, we produce a \AgdaRecord{Semantics} over the
unannotated system \AgdaFunction{uSystem}\AgdaSpace{}\AgdaBound{sys}.
To write it, we make use of idiom brackets
\AgdaSymbol{\ensuremath{\llparenthesis}}\AgdaSpace{}\AgdaSymbol{$\ldots$}\AgdaSpace{}\AgdaSymbol{\ensuremath{\rrparenthesis}},
which have the effect of replacing top-level spines of applications by
(\AgdaDatatype{List}-)applicative applications.
Field by field, we already know that variables are renameable.
To interpret a variable, we consider all the possible proofs that such a
variable could be well annotated, and package them up as a variable term via
the applicative machinery.
Finally, for compound terms, we first reify the unannotated subterms, and then
combine the subterms via a \AgdaFunction{lemma}.

\ExecuteMetaData[\UsageChecktex]{elab-sem}

The \AgdaFunction{lemma} essentially goes through the shape of the premises,
combining the collections of subterms in the natural way.
For example, at each
\AgdaInductiveConstructor{\AgdaUnderscore{}$\dottimes$\AgdaUnderscore{}},
we take the Cartesian product of the possibilities of each half, and at each
\AgdaInductiveConstructor{\AgdaUnderscore{}$\sep$\AgdaUnderscore{}},
we non-deterministically split the usage annotations coming in, and then take
the Cartesian product.
When it comes to newly bound variables, the \emph{syntax description} tells us
their annotations, so there is no further non-determinism introduced here.

\ExecuteMetaData[\UsageChecktex]{lemma-type}

To actually use \AgdaFunction{elab-sem} on terms, we take the associated
\AgdaFunction{semantics} and pass it the identity environment (an identity
\emph{renaming} in this case, because $\V$ is a family of variables).
We use \AgdaFunction{elab-unique}, which further
checks statically that exactly one derivation is returned.
The candidate usage vector \AgdaBound{R} will be \AgdaFunction{[]} for closed
terms, and otherwise we have to supply the intended usage annotations.

We can now use the elaborator to automatically infer the usage
annotations for the example at the end of \cref{sec:terms-of-system}.
This allows us to write:
\ExecuteMetaData[\PaperExamplestex]{cojoin}
We have instantiated the usage elaborator so that:
\AgdaField{0\#$^{-1}$} is a singleton on $\gr0$ and $\gr\omega$, and
empty on $\gr1$; \AgdaField{1\#$^{-1}$} is a singleton on $\gr1$ and
$\gr\omega$, and empty on $\gr0$; \AgdaField{+$^{-1}$} gives $\gr0
\mapsto [(\gr0,\gr0)]$, $\gr1 \mapsto [(\gr0,\gr1),(\gr1,\gr0)]$, and
$\gr\omega \mapsto [(\gr\omega,\gr\omega)]$; and \AgdaField{*$^{-1}$}
gives $(\gr\omega, \gr0) \mapsto [\gr0]$, $(\gr\omega, \gr1) \mapsto
[]$, and $(\gr\omega, \gr\omega) \mapsto [\gr\omega]$ (omitting
$(\gr0, \_)$ and $(\gr1, \_)$ cases for brevity). Note that we do not
consider splitting $\gr\omega$ up as, say, $\gr1 + \gr\omega$, because
this splitting would introduce more non-determinism but not allow any
more terms to be typed. As such, the only non-determinism comes when
we have variables annotated $\gr1$ and need to do an additive split,
like when we apply the \AgdaInductiveConstructor{!E} rule below. At
this point, the variable can become either $\gr0$-annotated in the
left subterm and $\gr1$-annotated on the right, or vice-versa. We will
find that, because the left subterm wants to use that variable, the
former choice will be rejected. The function \AgdaFunction{var\#} is a
convenience for converting statically known natural numbers,
representing de Bruijn \emph{levels}, into variable terms.


\section{Normalisation by evaluation}\label{sec:nbe}

To give an example of a common operation on programming languages, I present a
normalisation by evaluation (NbE) algorithm~\citep{BS91} for a fragment of
$\name$.
The algorithm contains not only a structural induction on a term using an
environment, but also functions which create terms.
It therefore gives an example of various parts of the framework.

Though it is easy to define a syntax of normal and neutral forms, I do not do so
here.
Converting between terms and normal/neutral forms introduces significant
overheads, and the syntax of normal/neutral forms is somewhat complicated by the
invariant that all variables are of neutral type.
It would be worthwhile future work to work out this refinement, thereby showing
that the normaliser presented in this section actually puts terms into normal
form.
Another limitation is that I do no reasoning about equivalence of terms, so
there is no proof that the resultant term is equivalent to the original.

For the sake of simplicity and conciseness, I consider just the fragment
containing a unary function type, but whose argument carries a usage annotation.
The partially ordered semiring of usage annotations is arbitrary throughout.
I also include an arbitrary set of base types \AgdaBound{BaseTy} via constructor
\AgdaInductiveConstructor{$\iota$}.

\ExecuteMetaData[\NbESimpletex]{Ty}

With only function types, the only term formers are application and
$\lambda$-abstraction.
These are similar to the corresponding term formers in $\name$, but application
scales its argument in accordance with the usage annotation of the function
type, and $\lambda$-abstraction binds a variable with the given usage
annotation.
For convenience, I define an alias \AgdaFunction{Term} for the open family of
terms in this type system of unspecified (\AgdaPostulate{$\infty$}) size.

\ExecuteMetaData[\NbESimpletex]{system}

The NbE model of types is given by \AgdaFunction{\_$\vDash$\_}.
It is defined by induction on the type.
The base type is interpreted by terms of base type, while
the function type is interpreted by the Kripke function space (essentially a
special case of \AgdaFunction{Kripke} with $\Delta = \gr rA$, though reusing
the \AgdaFunction{Kripke} definition causes difficulty for the termination
checker).
The interpretation of the base type additionally requires a \AgdaRecord{Lift}
of its universe level, because \AgdaFunction{$\Box^r$} produces a large type
(a type in \AgdaPrimitive{Set$_1$}).
The largeness of \AgdaFunction{$\Box^r$} comes from the largeness of
environments, and hence renamings, because of the relationally specified linear
map.

\ExecuteMetaData[\NbESimpletex]{vDash}

\AgdaFunction{\_$\vDash$\_} is a renameable family, as can be seen by cases on
the type.
It is renameable at the base type because terms are renameable, and it is
renameable at function type because any open type formed using
\AgdaFunction{$\Box^r$} is renameable by composition of renamings.
I refer to this proof of renameability as
\AgdaFunction{ren\textasciicircum$\vDash$}.

Now that I am working with a specific syntax, I introduce the function
\AgdaFunction{reify[]}.
This is a variant of the \AgdaFunction{reify} function given in \cref{sec:reify}
usable only for rules that bind no new variables --- i.e., the parameter
\AgdaBound{$\Delta$} to \AgdaFunction{Kripke} is the empty context.

\begin{lemma}[\AgdaFunction{reify[]}]
  We have a function of type
  $\mathrm{Kripke}\,\V\,\C\,\plr{\cdot} \rightarrowtriangle \C$ for any open
  families $\V$ and $\C$.
\end{lemma}
\begin{proof}
  Let $b : \mathrm{Kripke}\,\V\,\C\,\plr{\cdot}\,\Gamma\,A$.
  We give $b$ the identity renaming (corresponding to no extension of the
  context), yielding
  $b\,1 : \plr{\plr{-} \env\V {\cdot} \wand \plr{-} \sdtstile{}\C A}\,\Gamma$.
  By \cref{thm:construct-env}, we have that $\plr{-} \env\V {\cdot}$ is
  equivalent to $I^*$, and by monoidal closure,
  $\plr{I^* \wand \plr{-} \sdtstile{}\C A}$ is equivalent to
  $\plr{-} \sdtstile{}\C A$, so the result follows.
\end{proof}

The first part of the NbE algorithm I give is the evaluator.
The evaluator interprets terms into the model via an environment of values from
the model, making it a clear instance of the generic semantic traversal.
Specifically, I fix both \AgdaBound{$\V$} and \AgdaBound{$\C$} to
\AgdaFunction{\_$\vDash$\_} (allowing the conversion from semantic values to
semantic terms to be given by the identity function \AgdaFunction{id}), and
provide the other required data as below.
The case for $\lambda$-abstraction is almost trivial, given that its job is to
convert the Kripke function \AgdaBound{M} coming from the generic traversal into
a value in the model of function type (i.e.\ a Kripke function).
The main thing we need to do is to wrap the argument to the function into a
singleton environment using \AgdaFunction{[-]$^e$}.
In the case for function application, we use \AgdaFunction{reify[]} to get rid of
the extraneous Kripke functions coming from the generic traversal.
Having done so for both subterms, we apply the value corresponding to
\AgdaBound{M} (which is a Kripke function) to the value corresponding to
\AgdaBound{N} in an unextended context (\AgdaFunction{1$^r$}), with usage
information divided up as in the original application.

\ExecuteMetaData[\NbESimpletex]{evalSem}

The function \AgdaFunction{eval} is given by the appropriate instantiation of
the generic traversal \AgdaFunction{semantics}.
Note that we have not yet shown that \AgdaFunction{\_$\vDash$\_} supports
identity environments (which requires the reify and reflect functions), so we
cannot yet evaluate arbitrary open terms.

\ExecuteMetaData[\NbESimpletex]{eval}

The final pieces of NbE are the \emph{reify} and \emph{reflect} functions.
I give these the names \AgdaFunction{nbeReify} and \AgdaFunction{nbeReflect}
respectively to disambiguate from the \AgdaFunction{reify} and
\AgdaFunction{reify[]} functions of the framework.
\AgdaFunction{nbeReify} and \AgdaFunction{nbeReflect} are defined mutually by
recursion on the type, as is standard in NbE.
Therefore, neither uses the generic semantic traversal, though both make use of
some of the general components I have developed in this thesis.

At base type, values in the model are terms (modulo \AgdaRecord{Lift}), so the
conversion between these and terms is trivial.
At function type, \AgdaFunction{nbeReify} is given a Kripke function
\AgdaBound{v} and aims to turn it into a normal form of function type.
It does so in the canonical way --- introducing a $\lambda$-abstraction --- and
then aims to produce a term in an extended context.
Because of the context extension, we apply \AgdaBound{v} using the renaming
\AgdaFunction{$\swarrow^r$}\AgdaSpace{}\AgdaSymbol{:}\AgdaSpace{}%
\AgdaBound{$\Gamma$}\AgdaSpace{}\AgdaFunction{++$^c$}\AgdaSpace{}%
\AgdaFunction{[}\AgdaSpace{}\AgdaField{0\#}\AgdaSpace{}\AgdaFunction{$\bullet$}%
\AgdaSpace{}\AgdaBound{A}\AgdaSpace{}\AgdaFunction{]$^c$}
to get it ``back'' to the original context.
The splitting \AgdaBound{sp+} adds together $\Gamma, \gr0A$ and
$\gr0\Gamma, \gr rA$ to get $\Gamma, \gr rA$, while \AgdaBound{sp*} scales down
$\gr0\Gamma, \gr rA$ to $\gr0\Gamma, \gr1A$.
Semantic value \AgdaBound{v} then wants a semantic value of type $A$, which it
gets from reflecting the newly bound variable into the model.
Finally, \AgdaBound{v} produces a semantic value, which a recursive call to
\AgdaFunction{nbeReify} at the smaller type \AgdaBound{B} converts into a term
(in fact, a normal form).
Meanwhile, \AgdaFunction{nbeReflect} at function type is given a syntactic
function (in fact, a neutral form) and aims to turn it into a Kripke function.
To produce a Kripke function, we consider a renaming \AgdaBound{$\rho$} into
\AgdaBound{$\Gamma$}, splittings of the resultant context, and a semantic value
\AgdaBound{N} of type \AgdaBound{A}.
To produce the value of type \AgdaBound{B}, we recursively call
\AgdaFunction{nbeReflect} at smaller type \AgdaBound{B}, but with a larger term
formed by reifying \AgdaBound{N} (again of smaller type) and applying it to the
term \AgdaBound{M}.
To make the application work, we must rename both \AgdaBound{M} and
\AgdaBound{N} to ignore the (empty) context extensions introduced by the
application rule, and we must also rename \AgdaBound{M} by \AgdaBound{$\rho$} to
put \AgdaBound{M} into the extended context the reflection is being done at.

\ExecuteMetaData[\NbESimpletex]{reify-reflect}

Finally, we compose \AgdaFunction{eval}, \AgdaFunction{nbeReify}, and
\AgdaFunction{nbeReflect} to get the normaliser \AgdaFunction{nbe}.
The definition of \AgdaFunction{nbe} uses \AgdaFunction{eval} to interpret the
term into the model, and then uses \AgdaFunction{nbeReify} to read the semantic
value back as a term (a normal form).
However, as noted earlier, we need to give an environment of type
$\Gamma \env\vDash \Gamma$ to \AgdaFunction{eval} in order to interpret an open
term in its own context.
I provide this environment via \cref{thm:env-id}, which says that if we have a
mapping from variables in $\Gamma$ to values in $\Gamma$, we can get the desired
identity environment.
The required map is given in
\AgdaFunction{identityEnv\textasciicircum{}$\vDash$}
by turning the variable into a term using \AgdaInductiveConstructor{`var} and
then applying \AgdaFunction{nbeReflect}.
The definition \AgdaFunction{identityEnv\textasciicircum{}$\vDash$} is marked as
an \AgdaKeyword{instance} so that it is picked up by
\AgdaFunction{id\textasciicircum{}Env}.

\ExecuteMetaData[\NbESimpletex]{nbe}

The main limitation of this example is that the results are not specified as
being in normal form by their type.
Ideally, we would have a syntax of normal and neutral forms for
\AgdaFunction{nbe} and supporting functions to target.
We can produce such a syntax --- similarly to how \citet{AACMM21} produce a
syntax for bidirectional type checking --- but this syntax is difficult to work
with because of the crucial restriction on the allowable kind of variables ---
namely that variables can only appear as neutral forms, and not normal forms.
I believe that this difficulty could be worked around via carefully placed
assertions that various contexts are well formed --- and in particular that the
extended context introduced by a \AgdaFunction{$\Box^r$} is well formed.

Also, I have not given an account of many of the connectives of $\name$.
Of particular interest would be the types with pattern-matching eliminators ---
$I$, $\otimes$, $0$, $\oplus$, and $\oc$.
I believe these could be handled by ordinary techniques for NbE for sums ---
probably not doing $\eta$-expansion for these types.


\section{A denotational semantics}\label{sec:den-sem}

Standard denotational semantics falls out as a somewhat trivial case of
\AgdaRecord{Semantics}.
A lot of work is done in the generic traversal \AgdaFunction{semantics} to
maintain a $\V$-environment, where $\V$ is a reasonably variable-like semantic
family.
For a simple denotational semantics, we set $\V$ to be variables

As an example denotational semantics for a semiring-annotated calculus, I take
a semantics in world-indexed relations.
This is a refinement of the semantics given by \citet{AbelBernardy2020},
which gives a way to extract parametricity theorems from substructurally typed
programs.
Example theorems are that all linear terms act as permutations on some fixed
set of resources, and all monotonically typed terms are really monotonic in the
way the typing suggests they are.

For the sake of brevity, I use the same term language as in \cref{sec:nbe}, with
a set of base types and function types with a usage-annotated domain type.

As a running example, I take the usage annotations to be the 4-element
variance posemiring (\cref{def:monotonicity-posemiring}).
I establish the property that all terms are monotonic in their free variables.
This monotonicity can be covariant or contravariant (or neither or both)
depending on the annotation of each free variable (respectively,
\AgdaInductiveConstructor{$\uparrow\uparrow$},
\AgdaInductiveConstructor{$\downarrow\downarrow$},
\AgdaInductiveConstructor{$\wn\wn$}, and
\AgdaInductiveConstructor{\textasciitilde\textasciitilde}).
This provides an additional example to those of \citet{AbelBernardy2020}.

In the semantics of this section, types are interpreted as
\emph{world-indexed relations}~\citep{AbelBernardy2020,context-constrained}.
A world-indexed relation (\AgdaRecord{WRel}) over a poset of worlds
\AgdaBound{W} is a set \AgdaField{set} over which
we have a \AgdaBound{W}-indexed binary relation \AgdaField{ren} satisfying a
presheaf-like property \AgdaBound{subres} with respect to the order on
\AgdaBound{W}.
I declare the record type \AgdaRecord{WRel} to have
\AgdaKeyword{no-eta-equality} so that the usual $\eta$-law of records does not
apply to it.
The lack of an $\eta$-law allows the connectives on \AgdaRecord{WRel}
(e.g.\ \AgdaFunction{\_$\otimes^R$\_} and \AgdaFunction{\_$\multimap^R$\_},
defined below) to be definitionally injective, helping the unification in Agda's
elaborator.

\ExecuteMetaData[\WReltex]{WRel}

\begin{example}
  When \AgdaBound{W} is the 1-element set, a world-indexed relation is just a
  set equipped with a binary relation.
\end{example}

Morphisms (\AgdaRecord{WRelMor}) between world-indexed relations \AgdaBound{R}
and \AgdaBound{S} consist of a mapping \AgdaField{set$\Rightarrow$} between the
underlying sets such that, at each world \AgdaBound{w}, the mapping preserves
relatedness from \AgdaBound{R} to \AgdaBound{S}.
In the type of the field \AgdaField{rel$\Rightarrow$}, the world index is
handled implicitly via \AgdaFunction{\_$\rightarrowtriangle$\_}.

\ExecuteMetaData[\WReltex]{WRelMor}

When the poset of worlds forms a (relational) commutative monoid, such
world-indexed relations support a symmetric monoidal closed structure, with
objects denoted \AgdaFunction{I$^R$},
\AgdaFunction{\AgdaUnderscore{}$\otimes^R$\AgdaUnderscore{}}, and
\AgdaFunction{\AgdaUnderscore{}$\multimap^R$\AgdaUnderscore{}},.
These reuse the bunched connectives \AgdaRecord{$I^*$}, \AgdaRecord{$\sep$}, and
\AgdaRecord{$\wand$}, now over worlds rather than contexts.
Their definitions are listed below.

\ExecuteMetaData[\WReltex]{IR}
\ExecuteMetaData[\WReltex]{tensorR}
\ExecuteMetaData[\WReltex]{lollyR}

The final piece of semantics we need is a \emph{bang} operator.
I allow the
semantic \emph{bang} to be an arbitrary annotation-indexed functor on
world-indexed relations.
This functor must respect all of the structure on the indices, making it a
graded comonad over multiplication, as well as being lax monoidal at any
particular index \AgdaBound{r}.

\ExecuteMetaData[\WReltex]{Bang}

The properties required of the semantic \emph{bang} operator are slightly
weaker than those given by \citet{AbelBernardy2020}.
Specifically, the morphisms given by \AgdaField{!$^R$-1}, \AgdaField{!$^R$-I},
and \AgdaField{!$^R$-$\otimes$} only go one way, rather than being
bi-implications as they are for \citet{AbelBernardy2020}.
\citeauthor{AbelBernardy2020} need the stronger laws because of their strong
eliminator for tensor products, discussed in \cref{sec:alt}.
Also, I do not use the \emph{non-idempotent intersection} operators (which
handle worlds in the same way as \AgdaFunction{I$^R$} and
\AgdaFunction{\AgdaUnderscore{}$\otimes^R$\AgdaUnderscore{}}, but keep the
underlying set the same, rather than using products of the underlying sets),
and thus do not have axioms for them.
I avoid the need for non-idempotent intersections by giving the semantics all at
once, rather than first giving a $\mathrm{Set}$-semantics and then giving a
$\mathrm{WRel}$-semantics on top of it.

\begin{example}
  With \AgdaBound{W} being the 1-element set and annotations coming from the
  variance semiring, we can define the following \emph{bang}.
  It is always the identity on the set component, while the relation component
  consists of flipping the relation for contravariance and taking conjunctions
  to achieve both covariance and contravariance.
  When we want neither covariance nor contravariance, we use the always true
  predicate on worlds \AgdaFunction{$\dot1$}.
  With the worlds being trivial, so too is the property \AgdaField{subres}.

  \ExecuteMetaData[\Monotonicitytex]{BangR}
\end{example}

%To associate semantics to syntax, we start as standard by associating
%world-indexed relations to types.
%We also extend the semantics of types to contexts, using \AgdaFunction{I$^R$},
%\AgdaFunction{$\otimes^R$}, and \AgdaField{!$^R$} to interpret the empty
%context, concatenation, and usage annotations on singletons, respectively.
%
%\ExecuteMetaData[\WReltex]{sem}
%
%The semantics of a term is then to be a morphism from the interpretation of the
%context to the interpretation of the term's type.
%
%\ExecuteMetaData[\WReltex]{sem-vdash}

Let us assume that we have a mapping
\AgdaBound{$\upiota\llbracket$\_$\rrbracket$} from base types to world-indexed
relations.
We extend this interpretation to all types via
\AgdaFunction{$\llbracket$\_$\rrbracket$}, shown below, with
the function type being interpreted using \AgdaFunction{$\multimap^R$} and
\AgdaField{!$^R$}.
Contexts are interpreted by \AgdaFunction{$\llbracket$\_$\rrbracket^c$}, using
\AgdaFunction{$\otimes^R$} and \AgdaFunction{I$^R$}, and \AgdaField{!$^R$}
for singleton contexts.
Terms are interpreted as morphisms by the open family
\AgdaFunction{$\llbracket$\_$\vdash$\_$\rrbracket$}.
Variables are interpreted by \AgdaFunction{lookup$^R$} (definition omitted).

\ExecuteMetaData[\WReltex]{sem}
\ExecuteMetaData[\WReltex]{sem-vdash}
\ExecuteMetaData[\WReltex]{lookupR-type}

The laws \AgdaField{!$^R$-0}, \AgdaField{!$^R$-+}, and \AgdaField{!$^R$-*} lift
from singleton contexts to all contexts via lemmas \AgdaFunction{ctx-0},
\AgdaFunction{ctx-+}, and \AgdaFunction{ctx-*}, with the latter saying that we
have morphisms of type
\mbox{$\sem{\plr{\gr r\grP}\gamma} \to \oc^R\gr r\sem{\grP\gamma}$}.
We can see each of these three lemmas as ways to turn the algebraic structure of
contexts into structured world-indexed relations --- allowing us to use general
facts about \AgdaFunction{I$^R$}, \AgdaFunction{\_$\otimes^R$\_}, and
\AgdaField{!$^R$}, respectively.

Now I give a \AgdaRecord{Semantics}.
The choice of \AgdaBound{$\V$} as
\AgdaRecord{\AgdaUnderscore{}$\sqni$\AgdaUnderscore{}} is somewhat arbitrary,
given that a standard denotational semantics would not use intermediate
environments in the same sense as renaming and substitution, but it allows us to
reuse the standard facts that variables support renaming and identity
environments.
With this choice of \AgdaBound{$\V$} and \AgdaBound{$\C$}, we interpret
environment entries by \AgdaFunction{lookup$^R$}.

Meanwhile, for the logical rules, we ignore environments by using
\AgdaFunction{reify} and \AgdaFunction{reify[]} to just deal with morphisms in
an extended context.
The crucial structure of world-indexed relations is given by lemmas
\AgdaFunction{curry$^R$} and \AgdaFunction{uncurry$^R$}, which translate between
\mbox{\AgdaRecord{WRelMor}\AgdaSpace{}\AgdaSymbol(\AgdaBound{R}\AgdaSpace{}%
\AgdaFunction{$\otimes^R$}\AgdaSpace{}\AgdaBound{S}\AgdaSymbol)%
\AgdaSpace{}\AgdaBound{T}} and
\mbox{\AgdaRecord{WRelMor}\AgdaSpace{}\AgdaBound{R}\AgdaSpace{}\AgdaSymbol(%
\AgdaBound{S}\AgdaSpace{}\AgdaFunction{$\multimap^R$}\AgdaSpace{}\AgdaBound{T}%
\AgdaSymbol)}.
Also, \AgdaFunction{map-$\otimes^R$} implements the tensor
product of world-indexed relation morphisms.
With these lemmas, the interpretations of $\lambda$-abstraction and application
are straightforward.
For $\lambda$-abstraction, we have
\mbox{$\sem{\Gamma, \gr rA} = \sem\Gamma \otimes^R \oc^R\gr r\sem{A}$},
so we just need to use \AgdaFunction{curry$^R$} to get the desired semantic
function in the original context.
For application, we are producing the composition pictured in
\cref{fig:Wrel-app}.
In this picture, I ignore the possible subusaging between $\grP + \gr r\grQ$ and
the original context for the sake of notational suggestiveness.

\ExecuteMetaData[\WReltex]{Wrel}

\begin{figure}
  \centering
  \begin{tikzpicture}
    \node(0) {$\sem{\plr{\grP + \gr r\grQ}\gamma}$};
    \node(1) [right=of 0] {$\otimes$};
    \node(2) [above=1em of 1] {$\sem{\grP\gamma}$};
    \node(3) [below=1em of 1] {$\sem{\gr r\grQ\gamma}$};
    \node(4) [right=of 3] {$\oc\gr r\sem{\grQ\gamma}$};
    \node(5) [right=of 4] {$\oc\gr r\sem{A}$};
    \node(6) [above=1em of 5] {$\otimes$};
    \node(7) [above=1em of 6] {$\sem{\grP\gamma}$};
    \node(8) [right=of 6] {$\sem{B}$};

    \draw[->] (0) to node[above] {\AgdaFunction{ctx-+}} (1);
    \draw[->] (2) to node[above] {\AgdaFunction{id$^R$}} (7);
    \draw[->] (3) to node[below] {\AgdaFunction{ctx-*}} (4);
    \draw[->] (4) to node[below] {\AgdaBound{n}} (5);
    \draw[->] (6) to node[above] {\AgdaBound{m}} (8);
  \end{tikzpicture}
  \caption{Interpretation of application in the world-indexed relation
    semantics}
  \label{fig:Wrel-app}
\end{figure}

Then, the semantics of terms is given by the function
\AgdaFunction{semantics}\AgdaSpace{}\AgdaFunction{Wrel}\AgdaSpace{}%
\AgdaFunction{1$^r$}, where \AgdaFunction{1$^r$} is the identity renaming.
%In order to map open terms to interpretations, we take the action of the
%semantics and give the identity renaming as the starting environment.

\ExecuteMetaData[\WReltex]{wrel}

\begin{example}\label{thm:minus}
  We can make a subtraction function from primitive addition and negation on
  integers.
  Subtraction is covariant in its first argument and contravariant in its
  second argument.
  We give the definition in pseudocode, though it is also amenable to the
  usage elaborator of \cref{sec:usage-elaborator}, suitably instantiated.

  \begin{align*}
    &{\sim\sim}p :
      {\uparrow\uparrow}\mathbb Z \multimap
      {\uparrow\uparrow}\mathbb Z \multimap \mathbb Z,
      {\sim\sim}n : {\downarrow\downarrow}\mathbb Z \multimap \mathbb Z
      \vdash \mathnormal{minus} :
      {\uparrow\uparrow}\mathbb Z \multimap
      {\downarrow\downarrow}\mathbb Z \multimap
      \mathbb Z
    \\
    &\mathnormal{minus} \coloneqq \lambda x.~\lambda y.~p\,x\,(n\,y)
  \end{align*}

  After feeding in Agda's addition and negation functions as the
  interpretations of the free variables (noting that they are both monotonic
  in the required way), we get the following free theorem.

  \ExecuteMetaData[\Monotonicitytex]{thm-type}

  %We observe that the set component of this term's semantics is just the
  %expected Agda function when the two free variables are given appropriate
  %interpretations.

  %\ExecuteMetaData[\Monotonicitytex]{minus-set}

  %Furthermore, the relational component of the semantics yields the free
  %theorem that the Agda subtraction so defined is monotonic in the expected way.
  %This relies on library proofs that addition and negation are suitably
  %monotonic.

  %\ExecuteMetaData[\Monotonicitytex]{thm}
\end{example}

\section{Translating between $\name$ and L/nL}\label{sec:lnl}
We can express Benton's linear/non-linear logic~\cite{Benton94} in the
framework.
There are two apparent discrepancies between L/nL and what we have seen so far
to be expressible in the framework: the existence of multiple judgement modes
and the restriction on the kinds of assumptions based on the mode.
We will start by addressing these discrepancies, before presenting the encoding
in full and proving the resulting system logically equivalent to
$\lambda\gr{\mathcal R}$.

\subsection{Encoding L/nL}

In L/nL, we have a \emph{linear} and an \emph{intuitionistic} mode, with
separate types (respectively $A$ and $X$) and typing judgements (respectively
$\vdash_{\mathcal L}$ and $\vdash_{\mathcal C}$) connected by modalities $F$ and
$G$.
The mode of the conclusion type is the same as the mode of the judgement, so
only a distinction in the conclusion mode is necessary.
To achieve this distinction in the types, I have an indexed type family
$\mathrm{Ty} : \mathrm{Frag} \to \mathrm{Set}$, and set the framework types to
be the type $(f : \mathrm{Frag}) \times \mathrm{Ty}\,f$.
% \AgdaDatatype{Ty}\AgdaSymbol{:}\AgdaDatatype{Frag}\AgdaSymbol{$\to$}
% \AgdaDatatype{Set}
I present these judgements in a standard paper style in \cref{fig:lnl-types}.

In L/nL, while linear judgements can have both linear and intuitionistic
assumptions, intuitionistic judgements can only have intuitionistic assumptions.
I encode this invariant on intuitionistic judgements using the $\Box$
premise combinator on all intuitionistic conclusions and wherever
intuitionistic premises occur for linear conclusions, as seen in
\cref{fig:lnl-desc}.
This scheme means that the rules contain the maximal number of boxes, which is
convenient when consuming terms.
For ease of \emph{producing} terms, we would want a ``garbage in; garbage out''
style with a minimal number of boxes, which could be achieved in two different
ways:
\begin{itemize}
  \item Ensure that $\Gamma, \gr1\,\lin A \vdash \intu X$ is never derivable.
    This leaves boxes only on rules $1$-I and $G$-I.
  \item Ensure that, when working bottom-up, we never transition from a linear
    to an intuitionistic conclusion with linear assumptions present.
    This leaves boxes only on rules $F$-I and $G$-E.
\end{itemize}
It would be interesting to show the equivalence of these three approaches, but
that is left to future work.

\begin{figure}
  \begin{align*}
    A, B, C &\Coloneqq I \mid A \otimes B \mid A \multimap B \mid FX \\
    X, Y, Z &\Coloneqq 1 \mid X \times Y \mid X \to Y \mid GA \\
    \mathcal J &\Coloneqq \lin A \mid \intu X
  \end{align*}
  \caption{Types and judgement forms of L/nL}
  \label{fig:lnl-types}
\end{figure}

\begin{figure}
  \begin{mathpar}
    % lin rules
    \ebrule[comb]{%
      \hypo{I^*}
      \infer1[$I$-I]{\vdash \lin I}
    }
    \and
    \ebrule[comb]{%
      \hypo{\vdash \lin I}
      \hypo{*}
      \hypo{\vdash \lin C}
      \infer3[$I$-E]{\vdash \lin C}
    }
    \and
    \ebrule[comb]{%
      \hypo{\vdash \lin A}
      \hypo{*}
      \hypo{\vdash \lin B}
      \infer3[$\otimes$-I]{\vdash \lin A \otimes B}
    }
    \and
    \ebrule[comb]{%
      \hypo{\vdash \lin A \otimes B}
      \hypo{*}
      \hypo{\gr1\,\lin A, \gr1\,\lin B \vdash \lin C}
      \infer3[$\otimes$-E]{\vdash \lin C}
    }
    \and
    \ebrule[comb]{%
      \hypo{\gr1\,\lin A \vdash \lin B}
      \infer1[$\multimap$-I]{\vdash \lin A \multimap B}
    }
    \and
    \ebrule[comb]{%
      \hypo{\vdash \lin A \multimap B}
      \hypo{*}
      \hypo{\vdash \lin A}
      \infer3[$\multimap$-E]{\vdash \lin B}
    }
    \and
    \ebrule[comb]{%
      \hypo{\Box\plr{\vdash \intu X}}
      \infer1[$F$-I]{\vdash \lin FX}
    }
    \and
    \ebrule[comb]{%
      \hypo{\vdash \lin FX}
      \hypo{*}
      \hypo{\gr\omega\,\intu X \vdash \lin C}
      \infer3[$F$-E]{\vdash \lin C}
    }
    \and
    % int rules
    \ebrule[comb]{%
      \hypo{\Box\plr{\dot1}}
      \infer1[$1$-I]{\vdash \intu 1}
    }
    \and
    \text{(no $1$-E)}
    \and
    \ebrule[comb]{%
      \hypo{\Box(\vdash \intu X}
      \hypo{\dottimes}
      \hypo{\vdash \intu Y)}
      \infer3[$\times$-I]{\vdash \intu X \times Y}
    }
    \and
    \ebrule[comb]{%
      \hypo{\Box\plr{\vdash \intu X \times Y}}
      \infer1[$\times$-El]{\vdash \intu X}
    }
    \and
    \ebrule[comb]{%
      \hypo{\Box\plr{\vdash \intu X \times Y}}
      \infer1[$\times$-Er]{\vdash \intu Y}
    }
    \and
    \ebrule[comb]{%
      \hypo{\Box\plr{\gr\omega\,\intu A \vdash \intu B}}
      \infer1[$\to$-I]{\vdash \intu A \to B}
    }
    \and
    \ebrule[comb]{%
      \hypo{\Box(\vdash \intu A \to B}
      \hypo{\dottimes}
      \hypo{\vdash \intu A)}
      \infer3[$\to$-E]{\vdash \intu B}
    }
    \and
    \ebrule[comb]{%
      \hypo{\Box\plr{\vdash \lin A}}
      \infer1[$G$-I]{\vdash \intu GA}
    }
    \and
    \ebrule[comb]{%
      \hypo{\Box\plr{\vdash \intu GA}}
      \infer1[$G$-E]{\vdash \lin A}
    }
  \end{mathpar}
  \caption{Description of the logical rules of L/nL}
  \label{fig:lnl-desc}
\end{figure}

\begin{proposition}
  We can construct the following translations.
  \begin{align}
    (\Theta \vdash_{\mathcal C} X) &\to (\gr\omega\Theta \vdash \intu X) \\
    (\Theta; \Gamma \vdash_{\mathcal L} A) &\to
    (\gr\omega\Theta, \gr1\Gamma \vdash \lin A)
  \end{align}
\end{proposition}

\begin{proposition}
  We can construct the following translations, where $\Gamma_{\gr r}$ is the
  list of types in $\Gamma$ that are annotated $\gr r$.%
  \todo{I'm not sure the resultant contexts are well formed.}
  \begin{align}
    (\Gamma \vdash \intu X) &\to (\Gamma_{\gr\omega} \vdash_{\mathcal C} X) \\
    (\Gamma \vdash \lin A) &\to
    (\Gamma_{\gr\omega}; \Gamma_{\gr1} \vdash_{\mathcal L} A)
  \end{align}
\end{proposition}

\subsection{Translating between L/nL and $\lambda\gr{\mathcal R}$}

The translations largely follow Benton's originals.
In this section, I take $\name$ to be the system described in
\cref{fig:lr-bunched} instantiated to the $\{\gr0 > \gr\omega < \gr1\}$
posemiring and restricted to the fragment with only connectives $I$, $\otimes$,
$\multimap$, and $\oc\gr\omega$.
Notably, this excludes $\oc\gr0$ and $\oc\gr1$.
I write $\oc\gr\omega$ as just $\oc$, as in traditional linear logic.
Under these restrictions, Benton's translations of types can be used verbatim
(\cref{fig:lnl-lr-types}).

\begin{figure}
  \centering
  \begin{subfigure}{.49\linewidth}
    \centering
    \begin{align*}
      (-)^\circ : \mathrm{Ty}_{\name} \to \mathrm{Ty}_{\lin} \\
      \begin{aligned}
        I^\circ &= I \\
        \plr{A \otimes B}^\circ &= A^\circ \otimes B^\circ \\
        \plr{A \multimap B}^\circ &= A^\circ \multimap B^\circ \\
        \plr{\oc A}^\circ &= GFA^\circ
      \end{aligned}
    \end{align*}
  \end{subfigure}
  \begin{subfigure}{.49\linewidth}
    \centering
    \begin{align*}
      (-)^* : \mathrm{Ty}_\lnl \to \mathrm{Ty}_{\name} \\
      \begin{aligned}
        I^* &= I \\
        \plr{A \otimes B}^* &= A^* \otimes B^* \\
        \plr{A \multimap B}^* &= A^* \multimap B^* \\
        \plr{FX}^* &= \oc X^* \\
        1^* &= I \\
        \plr{X \times Y}^* &= \oc X^* \otimes \oc Y^* \\
        \plr{X \to Y}^* &= \oc X^* \multimap Y^* \\
        \plr{GA}^* &= A^*
      \end{aligned}
    \end{align*}
  \end{subfigure}
  \caption{Translation of types between L/nL and $\name$}
  \label{fig:lnl-lr-types}
\end{figure}

We extend both $(-)^\circ$ and $(-)^*$ to contexts pointwise on the types.
For $(-)^\circ$, this means that the translation lands outside what is usually
expressible in L/nL whenever the context contains annotation $\gr\omega$.
For example, $\plr{\gr\omega I}^\circ = \plr{\gr\omega\,\lin I}$, and we do not
see ``$\gr\omega\,\lin$'' in translations from Benton's L/nL.
This could be fixed by inserting $G$ in such cases.
As for $(-)^*$, note that we do not insert a $\oc$ on intuitionistic assumptions
as Benton does.
Instead, the annotation $\gr\omega$ achieves the same effect for us.
This discrepancy essentially amounts to the fact that $\name$ is more like DILL
than Girard's ILL, and Benton's translations act upon the latter.

\begin{theorem}
  We can translate from $\name$ to the linear fragment of L/nL.
  \begin{align}
    (\Gamma \vdash_{\name} A) \to (\Gamma^\circ \vdash_\lnl \lin A^\circ)
  \end{align}
\end{theorem}

\begin{theorem}
  We can translate any L/nL term to a $\name$ term as follows.
  \begin{align}
    (\Gamma \vdash_\lnl \intu X) &\to (\Gamma^* \vdash_{\name} X^*) \\
    (\Gamma \vdash_\lnl \lin A) &\to (\Gamma^* \vdash_{\name} A^*)
  \end{align}
\end{theorem}

