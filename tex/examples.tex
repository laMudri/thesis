In this chapter, I provide three example uses of semantic traversals: generic
renaming and substitution\todo{Moved}, a usage elaborator, and a denotational semantics.
The reader is also encouraged to see the far greater range of examples in the
work of \citet{AACMM21}, which should adapt to our usage-annotated setting.
Renaming and substitution are essential results, while the latter two examples
focus on usage annotations.


\section{A usage elaborator}\label{sec:usage-elaborator}

Using the constructs we have seen so far, producing example terms soon becomes
extremely tedious.
We can achieve some abbreviation by using pattern synonyms to wrap around
\AgdaInductiveConstructor{`con} expressions, but we still have to
produce essentially bespoke proofs whenever we use a usage-sensitive part of the
syntax.
The size of each of these proofs is roughly proportional to the number of free
variables, so the amount of proof we have to write grows roughly quadratically
with the size of terms.
An additional factor, which we can't see on paper but is nonetheless
significant, is that type checking time
for these proofs soon becomes prohibitive to interactive development.

Our aim in this section is to automate usage constraint proofs, making terms
both easier to write and more performant to check.
We invoke the automation by writing terms in a syntax where usage constraints
have been trivialised, and then use a semantic traversal over the simplified
syntax to try to produce a fully elaborated term in the original syntax.
We write the automation in a way that is generic in the syntax description, thus
avoiding repetition and facilitating the prototyping of new type systems.

The type of syntax descriptions depends on the type of usage annotations because
of variable binding.
For example, in the $\oc{\gr r}$-E rule of \cref{fig:lr-comb}, the right
premise binds a new variable with annotation $\gr r$, where $\gr r$ is drawn
from the ambient posemiring.
The scaling combinator also makes direct reference to the posemiring.
To produce a simplified syntax description, where usage constraints are
trivialised, we set the ambient posemiring to the 1-element $\mathbf0$
posemiring.
In contrast to syntax descriptions, even though types can contain usage
annotations, the type of types does not depend on the type of usage annotations.
This means that, in our simplified syntax, terms have types from the original
system even though variables have trivial usage annotations.
We define the $\mathbf0$ posemiring as follows, being careful to use the
0-field record type \AgdaRecord{$\top$} so that everything algebraic gets
solved by Agda's $\eta$-laws.
Indeed, in this very definition, all of the semiring operations and laws are
canonically inferred.

\ExecuteMetaData[\UsageChecktex]{0-poSemiring}

The elaboration process is monadic.
In particular, we use the \AgdaDatatype{List}/non-determinism monad to give
\emph{all} of the possible annotation choices on the free variables of a term.
We believe the commitment to multiple solutions is inherent when the syntax
contains \AgdaInductiveConstructor{`$\dot1$}.
For example, in the intermediate stages of elaborating
$\plr{\vdash \lambda x.~\plr{*,*}} : A \multimap \top \otimes \top$ with a
usage counting posemiring (assuming reasonable rules for $\top$ and $\otimes$),
it is unclear whether to use the variable $x$ in the left $*$ or the right $*$.
This uncertainty should be reflected in the final result.

The non-deterministic choices we make during elaboration are enumerated by
the fields of \AgdaRecord{NonDetInverses}.
These choices are driven by the typing rules and a candidate usage vector for
the conclusion.
For example, \AgdaField{+$^{-1}$}\AgdaSpace{}\AgdaBound{r} is needed when we
encounter a \AgdaInductiveConstructor{`$\sep$} in the syntax and the candidate
usage annotation we are considering is \AgdaBound{r}.
Then, \AgdaField{+$^{-1}$}\AgdaSpace{}\AgdaBound{r} is a list of pairs of
annotations \AgdaBound{p} and \AgdaBound{q} that \AgdaBound{r} can split into,
together with a proof of the splitting.
For \AgdaField{0\#$^{-1}$} and \AgdaField{1\#$^{-1}$}, inverses to constants,
we are given the candidate \AgdaBound{r} and typically return an empty list if
the constraint cannot be satisfied, or a singleton list containing a proof.
\AgdaField{*$^{-1}$} is used when we encounter scaling, in which case we know
both the scaling factor \AgdaBound{r} (from the syntax description) and the
candidate \AgdaBound{q}.
These inverse operations combine monadically (in fact, applicatively) to give
inverses to the vector operations of zero, addition, scaling, and basis.

\ExecuteMetaData[\UsageChecktex]{NonDetInverses}

We choose the \AgdaBound{$\V$} of our semantics to be (unannotated) variables.
For the \AgdaBound{$\C$}, we consider \emph{functions} from candidate usage
vectors \AgdaBound{R} to the list of elaborated derivations with usage
annotations given by \AgdaBound{R}.
The protocol this encodes is that the user will provide an unannotated term
together with a candidate usage context \AgdaBound{R}, and usage elaboration
returns a list of possible ways the term could be annotated such that the
conclusion has usage context \AgdaBound{R}.
The module name \AgdaModule{U} refers to the fact that we are taking the
ambient posemiring to be $\mathbf0$ in \AgdaFunction{OpenFam}.
The effect on \AgdaFunction{OpenFam} is that the usage annotations of any
contexts we consider are uninformative (hence the \AgdaSymbol{\_} on the left).

\ExecuteMetaData[\UsageChecktex]{C}

To traverse the unannotated terms, we produce a \AgdaRecord{Semantics} over the
unannotated system \AgdaFunction{uSystem}\AgdaSpace{}\AgdaBound{sys}.
To write it, we make use of idiom brackets
\AgdaSymbol{\ensuremath{\llparenthesis}}\AgdaSpace{}\AgdaSymbol{$\ldots$}\AgdaSpace{}\AgdaSymbol{\ensuremath{\rrparenthesis}},
which have the effect of replacing top-level spines of applications by
(\AgdaDatatype{List}-)applicative applications.
Field by field, we already know that variables are renameable.
To interpret a variable, we consider all the possible proofs that such a
variable could be well annotated, and package them up as a variable term via
the applicative machinery.
Finally, for compound terms, we first reify the unannotated subterms, and then
combine the subterms via a \AgdaFunction{lemma}.

\ExecuteMetaData[\UsageChecktex]{elab-sem}

The \AgdaFunction{lemma} essentially goes through the shape of the premises,
combining the collections of subterms in the natural way.
For example, at each
\AgdaInductiveConstructor{\AgdaUnderscore{}$\dottimes$\AgdaUnderscore{}},
we take the Cartesian product of the possibilities of each half, and at each
\AgdaInductiveConstructor{\AgdaUnderscore{}$\sep$\AgdaUnderscore{}},
we non-deterministically split the usage annotations coming in, and then take
the Cartesian product.
When it comes to newly bound variables, the \emph{syntax description} tells us
their annotations, so there is no further non-determinism introduced here.

\ExecuteMetaData[\UsageChecktex]{lemma-type}

To actually use \AgdaFunction{elab-sem} on terms, we take the associated
\AgdaFunction{semantics} and pass it the identity environment (an identity
\emph{renaming} in this case, because $\V$ is a family of variables).
We use \AgdaFunction{elab-unique}, which further
checks statically that exactly one derivation is returned.
The candidate usage vector \AgdaBound{R} will be \AgdaFunction{[]} for closed
terms, and otherwise we have to supply the intended usage annotations.

We can now use the elaborator to automatically infer the usage
annotations for the example at the end of \cref{sec:terms}. This
allows us to write:
\ExecuteMetaData[\PaperExamplestex]{cojoin}
We have instantiated the usage elaborator so that:
\AgdaField{0\#$^{-1}$} is a singleton on $\gr0$ and $\gr\omega$, and
empty on $\gr1$; \AgdaField{1\#$^{-1}$} is a singleton on $\gr1$ and
$\gr\omega$, and empty on $\gr0$; \AgdaField{+$^{-1}$} gives $\gr0
\mapsto [(\gr0,\gr0)]$, $\gr1 \mapsto [(\gr0,\gr1),(\gr1,\gr0)]$, and
$\gr\omega \mapsto [(\gr\omega,\gr\omega)]$; and \AgdaField{*$^{-1}$}
gives $(\gr\omega, \gr0) \mapsto [\gr0]$, $(\gr\omega, \gr1) \mapsto
[]$, and $(\gr\omega, \gr\omega) \mapsto [\gr\omega]$ (omitting
$(\gr0, \_)$ and $(\gr1, \_)$ cases for brevity). Note that we do not
consider splitting $\gr\omega$ up as, say, $\gr1 + \gr\omega$, because
this splitting would introduce more non-determinism but not allow any
more terms to be typed. As such, the only non-determinism comes when
we have variables annotated $\gr1$ and need to do an additive split,
like when we apply the \AgdaInductiveConstructor{!E} rule below. At
this point, the variable can become either $\gr0$-annotated in the
left subterm and $\gr1$-annotated on the right, or vice-versa. We will
find that, because the left subterm wants to use that variable, the
former choice will be rejected. The function \AgdaFunction{var\#} is a
convenience for converting statically known natural numbers,
representing de Bruijn \emph{levels}, into variable terms.


\section{A denotational semantics}\label{sec:den-sem}

Standard denotational semantics falls out as a somewhat trivial case of
\AgdaRecord{Semantics}.
A lot of work is done in the generic traversal \AgdaFunction{semantics} to
maintain a $\V$-environment, where $\V$ is a reasonably variable-like semantic
family.

{\color{red}Old stuff below\ldots}

To justify the name \emph{semantics}, we give an example traversal that is a
denotational semantics in the usual sense.
The semantics we take is a refinement of that of \citet{AbelBernardy2020},
which gives a way to extract parametricity theorems from substructurally typed
programs.
Example theorems are that all linear terms act as permutations on some fixed
set of resources, and all monotonically typed terms are really monotonic in the
way the typing suggests they are.

To abbreviate this section, we use a simplified syntax compared to \name{}.
We allow for an arbitrary family of base types \AgdaBound{BaseTy}, and a single
type former \mbox{\ExecuteMetaData[\WReltex]{rAToB}}, equivalent to
\mbox{\ExecuteMetaData[\PaperExamplestex]{BangrAToB}} from the earlier system.

\ExecuteMetaData[\WReltex]{Ty}

In the term syntax, $\lambda$-abstraction now binds a variable with annotation
\AgdaBound{r}, while application needs to scale its argument by \AgdaBound{r}
(both in accordance with the function type they are acting on).

\ExecuteMetaData[\WReltex]{AnnArr}

As a running example, we take the usage annotations to be the 4-element
variance posemiring (\cref{thm:variance}).
We establish the property that all terms are monotonic in their free variables.
This monotonicity can be covariant or contravariant (or neither or both)
depending on the annotation of each free variable.
This provides an additional example to those of \citeauthor{AbelBernardy2020}.

We will take semantics of this system into
\emph{world-indexed relations}~\cite{AbelBernardy2020,context-constrained}.
A world-indexed relation (\AgdaRecord{WRel}) over a poset of worlds
\AgdaBound{W} is a set over which
we have a \AgdaBound{W}-indexed binary relation satisfying a presheaf-like
property with respect to the order on \AgdaBound{W}.

\ExecuteMetaData[\WReltex]{WRel}

\begin{example}
  When \AgdaBound{W} is the 1-element set, a world-indexed relation is just a
  set equipped with a binary relation.
\end{example}

Morphisms (\AgdaRecord{WRelMor}) between world-indexed relations \AgdaBound{R}
and \AgdaBound{S} consist of a mapping between the underlying sets such that, at
each fixed world \AgdaBound{w}, the mapping preserves relatedness from
\AgdaBound{R} to \AgdaBound{S}.

\ExecuteMetaData[\WReltex]{WRelMor}

When the poset of worlds forms a (relational) commutative monoid, such
world-indexed relations support a symmetric monoidal closed structure, with
objects denoted \AgdaFunction{I$^R$},
\AgdaFunction{\AgdaUnderscore{}$\otimes^R$\AgdaUnderscore{}}, and
\AgdaFunction{\AgdaUnderscore{}$\multimap^R$\AgdaUnderscore{}},.
These reuse the bunched connectives \AgdaRecord{$I^*$}, \AgdaRecord{$\sep$}, and
\AgdaRecord{$\wand$}, now over worlds rather than contexts.

The final piece of semantics we need is a \emph{bang} operator.
We allow the
semantic \emph{bang} to be an arbitrary annotation-indexed functor on
world-indexed relations.
This functor must respect all of the structure on the indices, making it a
graded comonad over multiplication, as well as being lax monoidal at any
particular index \AgdaBound{r}.

\ExecuteMetaData[\WReltex]{Bang}

\begin{example}
  With \AgdaBound{W} being the 1-element set and annotations coming from the
  variance semiring, we can define the following \emph{bang}.
  It is always the identity on the set component, while the relation component
  consists of flipping the relation for contravariance and taking conjunctions
  to achieve both covariance and contravariance.
  When we want neither covariance nor contravariance, we use the always true
  predicate on worlds \AgdaFunction{$\dot1$}.

  \ExecuteMetaData[\Monotonicitytex]{BangR}
\end{example}

%To associate semantics to syntax, we start as standard by associating
%world-indexed relations to types.
%We also extend the semantics of types to contexts, using \AgdaFunction{I$^R$},
%\AgdaFunction{$\otimes^R$}, and \AgdaField{!$^R$} to interpret the empty
%context, concatenation, and usage annotations on singletons, respectively.
%
%\ExecuteMetaData[\WReltex]{sem}
%
%The semantics of a term is then to be a morphism from the interpretation of the
%context to the interpretation of the term's type.
%
%\ExecuteMetaData[\WReltex]{sem-vdash}

The semantics of a type is given by \AgdaFunction{$\llbracket$\_$\rrbracket$},
which maps into world-indexed relations.
The function type is interpreted using \AgdaFunction{$\multimap^R$} and
\AgdaField{!$^R$}.
Contexts are interpreted by \AgdaFunction{$\llbracket$\_$\rrbracket^c$}, using
\AgdaFunction{$\otimes^R$} and \AgdaFunction{I$^R$}.
Terms are interpreted as morphisms by the open family
\AgdaFunction{$\llbracket$\_$\vdash$\_$\rrbracket$}.
Variables are interpreted by \AgdaFunction{lookup$^R$} (definition omitted).

\ExecuteMetaData[\WReltex]{lookupR-type}

Now we give a \AgdaRecord{Semantics}.
The choice of \AgdaBound{$\V$} as
\AgdaRecord{\AgdaUnderscore{}$\sqni$\AgdaUnderscore{}} is somewhat arbitrary,
given that a standard denotational semantics would not use intermediate
environments in the same sense as renaming and substitution, but it allows us to
reuse the standard facts that variables support renaming and identity
environments.
With this choice of \AgdaBound{$\V$} and \AgdaBound{$\C$}, we interpret
environment entries by \AgdaFunction{lookup$^R$}.
Meanwhile, for the logical rules, we ignore environments by using
\AgdaFunction{reify} to just deal with morphisms in an extended context.
As such, $\lambda$-abstractions are easy to interpret, while applications
require some massaging to remove the extension by an empty context, followed by
some plumbing to split the interpretation of the context according to the usage
constraints and feed the interpretation of the argument \AgdaBound{n$'$} into
the interpretation of the function \AgdaBound{m$'$}.

\ExecuteMetaData[\WReltex]{Wrel}

Then, the semantics of terms is given by the function
\AgdaFunction{semantics}\AgdaSpace{}\AgdaFunction{Wrel}\AgdaSpace{}%
\AgdaFunction{1$^r$}, where \AgdaFunction{1$^r$} is the identity renaming.
%In order to map open terms to interpretations, we take the action of the
%semantics and give the identity renaming as the starting environment.

\ExecuteMetaData[\WReltex]{wrel}

\begin{example}\label{thm:minus}
  We can make a subtraction function from primitive addition and negation on
  integers.
  Subtraction is covariant in its first argument and contravariant in its
  second argument.
  We give the definition in pseudocode, though it is also amenable to the
  usage elaborator of \cref{sec:usage-elaborator}, suitably instantiated.

  \begin{align*}
    &{\sim\sim}p :
      {\uparrow\uparrow}\mathbb Z \multimap
      {\uparrow\uparrow}\mathbb Z \multimap \mathbb Z,
      {\sim\sim}n : {\downarrow\downarrow}\mathbb Z \multimap \mathbb Z
      \vdash \mathnormal{minus} :
      {\uparrow\uparrow}\mathbb Z \multimap
      {\downarrow\downarrow}\mathbb Z \multimap
      \mathbb Z
    \\
    &\mathnormal{minus} \coloneqq \lambda x.~\lambda y.~p\,x\,(n\,y)
  \end{align*}

  After feeding in Agda's addition and negation functions as the
  interpretations of the free variables (noting that they are both monotonic
  in the required way), we get the following free theorem.

  \ExecuteMetaData[\Monotonicitytex]{thm-type}

  %We observe that the set component of this term's semantics is just the
  %expected Agda function when the two free variables are given appropriate
  %interpretations.

  %\ExecuteMetaData[\Monotonicitytex]{minus-set}

  %Furthermore, the relational component of the semantics yields the free
  %theorem that the Agda subtraction so defined is monotonic in the expected way.
  %This relies on library proofs that addition and negation are suitably
  %monotonic.

  %\ExecuteMetaData[\Monotonicitytex]{thm}
\end{example}

\section{Translating between $\name$ and L/nL}\label{sec:lnl}
We can express Benton's linear/non-linear logic~\cite{Benton94} in the
framework.
There are two apparent discrepancies between L/nL and what we have seen so far
to be expressible in the framework: the existence of multiple judgement modes
and the restriction on the kinds of assumptions based on the mode.
We will start by addressing these discrepancies, before presenting the encoding
in full and proving the resulting system logically equivalent to
$\lambda\gr{\mathcal R}$.

\subsection{Encoding L/nL}

In L/nL, we have a \emph{linear} and an \emph{intuitionistic} mode, with
separate types (respectively $A$ and $X$) and typing judgements (respectively
$\vdash_{\mathcal L}$ and $\vdash_{\mathcal C}$) connected by modalities $F$ and
$G$.
The mode of the conclusion type is the same as the mode of the judgement, so
only a distinction in the conclusion mode is necessary.
To achieve this distinction in the types, I have an indexed type family
$\mathrm{Ty} : \mathrm{Frag} \to \mathrm{Set}$, and set the framework types to
be the type $(f : \mathrm{Frag}) \times \mathrm{Ty}\,f$.
% \AgdaDatatype{Ty}\AgdaSymbol{:}\AgdaDatatype{Frag}\AgdaSymbol{$\to$}
% \AgdaDatatype{Set}
I present these judgements in a standard paper style in \cref{fig:lnl-types}.

In L/nL, while linear judgements can have both linear and intuitionistic
assumptions, intuitionistic judgements can only have intuitionistic assumptions.
I encode this invariant on intuitionistic judgements using the $\Box$
premise combinator on all intuitionistic conclusions and wherever
intuitionistic premises occur for linear conclusions, as seen in
\cref{fig:lnl-desc}.
This scheme means that the rules contain the maximal number of boxes, which is
convenient when consuming terms.
For ease of \emph{producing} terms, we would want a ``garbage in; garbage out''
style with a minimal number of boxes, which could be achieved in two different
ways:
\begin{itemize}
  \item Ensure that $\Gamma, \gr1\,\lin A \vdash \intu X$ is never derivable.
    This leaves boxes only on rules $1$-I and $G$-I.
  \item Ensure that, when working bottom-up, we never transition from a linear
    to an intuitionistic conclusion with linear assumptions present.
    This leaves boxes only on rules $F$-I and $G$-E.
\end{itemize}
It would be interesting to show the equivalence of these three approaches, but
that is left to future work.

\begin{figure}
  \begin{align*}
    A, B, C &\Coloneqq I \mid A \otimes B \mid A \multimap B \mid FX \\
    X, Y, Z &\Coloneqq 1 \mid X \times Y \mid X \to Y \mid GA \\
    \mathcal J &\Coloneqq \lin A \mid \intu X
  \end{align*}
  \caption{Types and judgement forms of L/nL}
  \label{fig:lnl-types}
\end{figure}

\begin{figure}
  \begin{mathpar}
    % lin rules
    \ebrule[comb]{%
      \hypo{I^*}
      \infer1[$I$-I]{\vdash \lin I}
    }
    \and
    \ebrule[comb]{%
      \hypo{\vdash \lin I}
      \hypo{*}
      \hypo{\vdash \lin C}
      \infer3[$I$-E]{\vdash \lin C}
    }
    \and
    \ebrule[comb]{%
      \hypo{\vdash \lin A}
      \hypo{*}
      \hypo{\vdash \lin B}
      \infer3[$\otimes$-I]{\vdash \lin A \otimes B}
    }
    \and
    \ebrule[comb]{%
      \hypo{\vdash \lin A \otimes B}
      \hypo{*}
      \hypo{\gr1\,\lin A, \gr1\,\lin B \vdash \lin C}
      \infer3[$\otimes$-E]{\vdash \lin C}
    }
    \and
    \ebrule[comb]{%
      \hypo{\gr1\,\lin A \vdash \lin B}
      \infer1[$\multimap$-I]{\vdash \lin A \multimap B}
    }
    \and
    \ebrule[comb]{%
      \hypo{\vdash \lin A \multimap B}
      \hypo{*}
      \hypo{\vdash \lin A}
      \infer3[$\multimap$-E]{\vdash \lin B}
    }
    \and
    \ebrule[comb]{%
      \hypo{\Box\plr{\vdash \intu X}}
      \infer1[$F$-I]{\vdash \lin FX}
    }
    \and
    \ebrule[comb]{%
      \hypo{\vdash \lin FX}
      \hypo{*}
      \hypo{\gr\omega\,\intu X \vdash \lin C}
      \infer3[$F$-E]{\vdash \lin C}
    }
    \and
    % int rules
    \ebrule[comb]{%
      \hypo{\Box\plr{\dot1}}
      \infer1[$1$-I]{\vdash \intu 1}
    }
    \and
    \text{(no $1$-E)}
    \and
    \ebrule[comb]{%
      \hypo{\Box(\vdash \intu X}
      \hypo{\dottimes}
      \hypo{\vdash \intu Y)}
      \infer3[$\times$-I]{\vdash \intu X \times Y}
    }
    \and
    \ebrule[comb]{%
      \hypo{\Box\plr{\vdash \intu X \times Y}}
      \infer1[$\times$-El]{\vdash \intu X}
    }
    \and
    \ebrule[comb]{%
      \hypo{\Box\plr{\vdash \intu X \times Y}}
      \infer1[$\times$-Er]{\vdash \intu Y}
    }
    \and
    \ebrule[comb]{%
      \hypo{\Box\plr{\gr\omega\,\intu A \vdash \intu B}}
      \infer1[$\to$-I]{\vdash \intu A \to B}
    }
    \and
    \ebrule[comb]{%
      \hypo{\Box(\vdash \intu A \to B}
      \hypo{\dottimes}
      \hypo{\vdash \intu A)}
      \infer3[$\to$-E]{\vdash \intu B}
    }
    \and
    \ebrule[comb]{%
      \hypo{\Box\plr{\vdash \lin A}}
      \infer1[$G$-I]{\vdash \intu GA}
    }
    \and
    \ebrule[comb]{%
      \hypo{\Box\plr{\vdash \intu GA}}
      \infer1[$G$-E]{\vdash \lin A}
    }
  \end{mathpar}
  \caption{Description of the logical rules of L/nL}
  \label{fig:lnl-desc}
\end{figure}

\begin{proposition}
  We can construct the following translations.
  \begin{align}
    (\Theta \vdash_{\mathcal C} X) &\to (\gr\omega\Theta \vdash \intu X) \\
    (\Theta; \Gamma \vdash_{\mathcal L} A) &\to
    (\gr\omega\Theta, \gr1\Gamma \vdash \lin A)
  \end{align}
\end{proposition}

\begin{proposition}
  We can construct the following translations, where $\Gamma_{\gr r}$ is the
  list of types in $\Gamma$ that are annotated $\gr r$.%
  \todo{I'm not sure the resultant contexts are well formed.}
  \begin{align}
    (\Gamma \vdash \intu X) &\to (\Gamma_{\gr\omega} \vdash_{\mathcal C} X) \\
    (\Gamma \vdash \lin A) &\to
    (\Gamma_{\gr\omega}; \Gamma_{\gr1} \vdash_{\mathcal L} A)
  \end{align}
\end{proposition}

\subsection{Translating between L/nL and $\lambda\gr{\mathcal R}$}

The translations largely follow Benton's originals.
In this section, I take $\name$ to be the system described in
\cref{fig:lr-bunched} instantiated to the $\{\gr0 > \gr\omega < \gr1\}$
posemiring and restricted to the fragment with only connectives $I$, $\otimes$,
$\multimap$, and $\oc\gr\omega$.
Notably, this excludes $\oc\gr0$ and $\oc\gr1$.
I write $\oc\gr\omega$ as just $\oc$, as in traditional linear logic.
Under these restrictions, Benton's translations of types can be used verbatim
(\cref{fig:lnl-lr-types}).

\begin{figure}
  \centering
  \begin{subfigure}{.49\linewidth}
    \centering
    \begin{align*}
      (-)^\circ : \mathrm{Ty}_{\name} \to \mathrm{Ty}_{\lin} \\
      \begin{aligned}
        I^\circ &= I \\
        \plr{A \otimes B}^\circ &= A^\circ \otimes B^\circ \\
        \plr{A \multimap B}^\circ &= A^\circ \multimap B^\circ \\
        \plr{\oc A}^\circ &= GFA^\circ
      \end{aligned}
    \end{align*}
  \end{subfigure}
  \begin{subfigure}{.49\linewidth}
    \centering
    \begin{align*}
      (-)^* : \mathrm{Ty}_\lnl \to \mathrm{Ty}_{\name} \\
      \begin{aligned}
        I^* &= I \\
        \plr{A \otimes B}^* &= A^* \otimes B^* \\
        \plr{A \multimap B}^* &= A^* \multimap B^* \\
        \plr{FX}^* &= \oc X^* \\
        1^* &= I \\
        \plr{X \times Y}^* &= \oc X^* \otimes \oc Y^* \\
        \plr{X \to Y}^* &= \oc X^* \multimap Y^* \\
        \plr{GA}^* &= A^*
      \end{aligned}
    \end{align*}
  \end{subfigure}
  \caption{Translation of types between L/nL and $\name$}
  \label{fig:lnl-lr-types}
\end{figure}

We extend both $(-)^\circ$ and $(-)^*$ to contexts pointwise on the types.
For $(-)^\circ$, this means that the translation lands outside what is usually
expressible in L/nL whenever the context contains annotation $\gr\omega$.
For example, $\plr{\gr\omega I}^\circ = \plr{\gr\omega\,\lin I}$, and we do not
see ``$\gr\omega\,\lin$'' in translations from Benton's L/nL.
This could be fixed by inserting $G$ in such cases.
As for $(-)^*$, note that we do not insert a $\oc$ on intuitionistic assumptions
as Benton does.
Instead, the annotation $\gr\omega$ achieves the same effect for us.
This discrepancy essentially amounts to the fact that $\name$ is more like DILL
than Girard's ILL, and Benton's translations act upon the latter.

\begin{theorem}
  We can translate from $\name$ to the linear fragment of L/nL.
  \begin{align}
    (\Gamma \vdash_{\name} A) \to (\Gamma^\circ \vdash_\lnl \lin A^\circ)
  \end{align}
\end{theorem}

\begin{theorem}
  We can translate any L/nL term to a $\name$ term as follows.
  \begin{align}
    (\Gamma \vdash_\lnl \intu X) &\to (\Gamma^* \vdash_{\name} X^*) \\
    (\Gamma \vdash_\lnl \lin A) &\to (\Gamma^* \vdash_{\name} A^*)
  \end{align}
\end{theorem}

