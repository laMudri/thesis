We could mechanise Gentzen's definition of a natural deduction system directly,
but this definition is quite complicated.
In particular, if we want to give derivations an inductive definition, the use
of the discharge mechanism means that we actually need an inductive-inductive
type --- derivations, particularly those using $\to$-introduction, can involve
references to assumptions within their subderivations.
An inductive-inductive definition of derivations would complicate our programs
and proofs about natural deduction derivations, so I choose an alternative
representation.

Indeed, most authors since Gentzen, whether mechanising their work or not,
have opted to replace discharge of assumptions by explicit \emph{contexts} and
a variable rule.
Contexts can be justified as a way to keep track of undischarged assumptions.
In particular, we only produce derivations in the presence of a known collection
of \emph{free variables} specified by the context.
In other words, derivations are \emph{indexed} over their free variables and
their types.
When using an assumption within a derivation, we must say which free variable
it corresponds to.
Free variables are introduced by \emph{variable-binding} rules, like
$\to$-introduction.
\cref{fig:explicit-contexts} gives an example of the same derivation written
in Gentzen's style and in the explicit context style.

\begin{sidewaysfigure}
  \centering
  \begin{prooftree}
    \hypo{[A \to A \to B]^f}
    \hypo{[A]^x}
    \infer2[$\to$-E]{A \to B}
    \hypo{[A]^x}
    \infer2[$\to$-E]{B}
    \infer1[$\to$-I$^x$]{A \to B}
    \infer1[$\to$-I$^f$]{\plr{A \to A \to B} \to \plr{A \to B}}
  \end{prooftree}

  \vspace{2em}

  \begin{prooftree}
    \infer0[var$^f$]{{\color{red}f : A \to A \to B, x : A} \vdash A \to A \to B}
    \infer0[var$^x$]{{\color{red}f : A \to A \to B, x : A} \vdash A}
    \infer2[$\to$-E]{{\color{red}f : A \to A \to B, x : A} \vdash A \to B}
    \infer0[var$^x$]{{\color{red}f : A \to A \to B, x : A} \vdash A}
    \infer2[$\to$-E]{{\color{red}f : A \to A \to B, x : A} \vdash B}
    \infer1[$\to$-I$^x$]{{\color{red}f : A \to A \to B} \vdash A \to B}
    \infer1[$\to$-I$^f$]{\vdash \plr{A \to A \to B} \to \plr{A \to B}}
  \end{prooftree}
  \caption{A proof in Gentzen's natural deduction syntax, and a proof using
    explicit contexts (contexts coloured {\color{red}red})}
  \label{fig:explicit-contexts}
\end{sidewaysfigure}

Explicit contexts can be seen as a mechanism for encoding a natural deduction
system as a sequent calculus.
However, the natural deduction character of the system is maintained by
ensuring that the resultant sequent calculus is really an encoding of a
natural deduction system.
Concretely, this means that rules can only interact with the context in
restricted ways:

\begin{itemize}
  \item There is a designated \emph{variable rule}, stating that any variable
    in the context can serve as a derivation of its type.
  \item Non-variable rules may only require subterms with \emph{extended}
    contexts, i.e., subterms in which new variables have been bound.
    Non-variable rules are parametric in the existing free variables.
\end{itemize}

Having chosen to use explicit contexts, the mechanisation must have a chosen
representation of contexts as a data structure.
While the notation in \cref{fig:explicit-contexts} uses names $f$ and $x$
for variables, I opt for a nameless representation.
In a nameless representation, variables are identified by their position in
the context, rather than by a name.
The absence of names means that $\alpha$-equivalence is just on-the-nose
equality, and also that we never have to reason about freshness of names.
Agda does not have support for nominal techniques~\cite{GP02}, which may have
made names a better option.

Most mechanisations choose contexts to be an inductive list of types.
However, I instead choose a functional, tree-shaped representation, as shown
with the type \AgdaRecord{Ctx}.
The type \AgdaDatatype{LTree} is the inductive type generated by leaves and
nullary \& binary nodes, and serves as a generalised ``length'' of the context.
The contents of the context --- the types --- are then stored in the functional
vector \AgdaField{ty-ctx}, which is a mapping from leaves in \AgdaField{shape}
to object language types \AgdaDatatype{Ty}.
The advantages of the functional vector representation will not become clear
until later chapters\todo{forward reference}, where I make use of the ease of
look-up and $\eta$-law of functions.
However, I claim for now that there is little to no disadvantage in the
functional vector representation --- in particular, we have no need for
function extensionality principles because we never talk about equality of
contexts.
For example, instead of using an equality of contexts to coerce a term, we can
use renaming.
As for the tree shape, this makes context concatenation definitionally
injective, so that in cases where multiple new variables are bound in a subterm
(for example, $\otimes$-elimination), Agda's unification-based solving will
be more able to infer which variables have just been bound.

\missingfigure{LTree}
\Ctx{}

We start with a data type \AgdaDatatype{\_⊢\_} of intrinsically simply typed
terms.
Beside base types, the only type former we have is the function type constructor
\AgdaInductiveConstructor{\_`→\_}.
Contexts (of type \AgdaRecord{Ctx}) are implemented as the free magma on types
(\AgdaDatatype{Ty}).
Context concatenation is \AgdaFunction{\_++ᶜ\_}, and \AgdaFunction{[\_]ᶜ}
embeds types into contexts.
Typed variables in a context are given by \AgdaRecord{\_∋\_}.
A variable in
\AgdaBound{$\Gamma$}\AgdaSpace{}\AgdaRecord{$\ni$}\AgdaSpace{}\AgdaBound{A}
is given by a path \AgdaField{idx} to a type in \AgdaBound{$\Gamma$}, together
with a proof \AgdaField{tyq} that this type is equal to \AgdaBound{A}.
Variables embed into terms via the \AgdaInductiveConstructor{var} constructor.

\Var{}
\Term{}
