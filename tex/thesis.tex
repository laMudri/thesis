%%%%%%%%%%%%%%%%%%%%%%%%%%%%%%%%%%%%%%%%%%%%%%%%
%
% Strath PhD Thesis Template
%  by Jethro Browell [jethro.browell@strath.ac.uk]
%
%  Guidelines for thesis format, submission and content are found in
%  General and Course Regulations for Graduate and Postgraduate
%  Awards and Degrees, section 20.6.
%
%  Using .eps or .pdf is recomended to prduce high quality figures etc.
%
%  The Strathclyde logo can be found in other formats at www.strath.ac.uk.
%
%%%%%%%%%%%%%%%%%%%%%%%%%%%%%%%%%%%%%%%%%%%%%%%%

\documentclass[a4paper,oneside,11pt]{book}

\usepackage{amsbsy}
\usepackage{amsmath}
\usepackage{amsfonts}
\usepackage{graphicx}
\usepackage{multirow}
\usepackage{mathrsfs}
\usepackage{xcolor}
\usepackage[hidelinks]{hyperref}
\usepackage{cite}
\usepackage{enumitem}
\usepackage{epsfig}
\usepackage{caption}
\usepackage{subcaption}
\usepackage[strict]{changepage}

\usepackage{amssymb}
\usepackage{catchfilebetweentags}
\usepackage{cmll}
\usepackage[utf8]{inputenc}
\usepackage{mathtools}
\usepackage{mathpartir}
\usepackage{newunicodechar}
\usepackage{stmaryrd}
\usepackage{todonotes}

\usepackage[conor]{agda}

\newunicodechar{λ}{\ensuremath{\mathnormal\lambda}}
\newunicodechar{→}{\ensuremath{\mathnormal\to}}
\newunicodechar{∀}{\ensuremath{\mathnormal\forall}}
\newunicodechar{ι}{\ensuremath{\mathnormal\iota}}
\newunicodechar{·}{\ensuremath{\mathnormal\cdot}}
\newunicodechar{⊸}{\ensuremath{\mathnormal\multimap}}
\newunicodechar{⊕}{\ensuremath{\mathnormal\oplus}}
\newunicodechar{─}{\textrm{---}}
\newunicodechar{ᶜ}{\ensuremath{\mathnormal{^c}}}
\newunicodechar{⊢}{\ensuremath{\mathnormal\vdash}}
\newunicodechar{∧}{\ensuremath{\mathnormal\wedge}}
\newunicodechar{⟦}{\ensuremath{\mathnormal\llbracket}}
\newunicodechar{⟧}{\ensuremath{\mathnormal\rrbracket}}
\newunicodechar{∩}{\ensuremath{\mathnormal\cap}}
\newunicodechar{✴}{\ensuremath{\mathnormal*}}
\newunicodechar{ℓ}{\ensuremath{\mathnormal\ell}}
\newunicodechar{Γ}{\ensuremath{\mathnormal\Gamma}}
\newunicodechar{Δ}{\ensuremath{\mathnormal\Delta}}
\newunicodechar{Σ}{\ensuremath{\mathnormal\Sigma}}
\newunicodechar{∈}{\ensuremath{\mathnormal\in}}
\newunicodechar{′}{\ensuremath{\mathnormal'}}
\newunicodechar{≡}{\ensuremath{\mathnormal\equiv}}
\newunicodechar{⊤}{\ensuremath{\mathnormal\top}}
\newunicodechar{⊥}{\ensuremath{\mathnormal\bot}}
\newunicodechar{⇒}{\ensuremath{\mathnormal\Rightarrow}}
\newunicodechar{▹}{\ensuremath{\mathnormal\triangleright}}
\newunicodechar{₁}{\ensuremath{\mathnormal{_1}}}
\newunicodechar{ℑ}{\ensuremath{\mathfrak I}}
\newunicodechar{□}{\ensuremath{\mathnormal\Box}}

\definecolor{use}{HTML}{008000}
\newcommand\gr[1]{{\color{use}#1}}
\newcommand\grctx[1]{\gr{\mathcal{#1}}}
\newcommand\grP{\grctx P}
\newcommand\grQ{\grctx Q}
\newcommand\grR{\grctx R}
\newcommand\grPprime{\grP\gr'}
\newcommand\grQprime{\grQ\gr'}
\newcommand\name{\ensuremath{\lambda\grR}}
\newcommand\grctxsub[2]{\grctx{#1}_{\gr{#2}}}
\newcommand\grPe{\grctxsub P e}
\newcommand\grPf{\grctxsub P f}
\newcommand\grQe{\grctxsub Q e}
\newcommand\grQf{\grctxsub Q f}
\newcommand\sem[1]{\left\llbracket{#1}\right\rrbracket}
\newcommand\size[1]{\left\lvert{#1}\right\rvert}
\newcommand\ps{\mathit{ps}}
\newcommand\qs{\mathit{qs}}
\newcommand\rs{\mathit{rs}}
\newcommand\dotto{\mathrel{\dot\to}}
\newcommand\dottimes{\mathbin{\dot\times}}
\newcommand\wand{\mathrel{\mathord{-}\hspace{-0.75ex}*}}
\newcommand\sep{\mathbin{*}}
\newcommand\env[1]{(#1\mathrm{-Env})}
\newcommand\thinningN{\mathrm{Thinning}}
\newcommand\thinning[2]{\thinningN~#1~#2}
\newcommand\V{\mathcal V}
\newcommand\C{\mathcal C}
\newcommand\sqin{\mathrel{\mathrlap{\sqsubset}{\mathord{-}}}}
\newcommand\sqni{\mathrel{\mathrlap{\sqsupset}{\mathord{-}}}}
\newcommand\subres{=}
\newcommand\Ann{\mathscr R}

\renewcommand\land{~\wedge~}
\renewcommand\lor{~\vee~}

\DeclareMathOperator\obj{Obj}
\let\hom\relax
\DeclareMathOperator\hom{Hom}
\DeclareMathOperator\id{id}
\DeclareMathOperator\sub{Sub}


% Page Margins - Strath Requirement
\usepackage[left=4cm,right=2.5cm,top=2cm,bottom=4cm,includehead,includefoot,headheight=15pt]{geometry}

% Page Headers
\usepackage{fancyhdr}
\fancyhf{}
\renewcommand{\headrulewidth}{0pt} % optional
%\fancyhead[L]{\nouppercase{\leftmark} \hfill Section \nouppercase{\rightmark}}
\fancyhead[L]{\nouppercase{\leftmark}}
\cfoot{\thepage}
\pagestyle{fancy}

% Draft Watermark
%\usepackage[draft=true,allpages=true,fontfamily=cmr,angle=90,scale=0.1,mark={\fboxsep=35pt\fboxrule=0pt\relax\fbox{-- DRAFT -- \today~--}},xcoord=-80,ycoord=-20]{draftwatermark}


% Line Spacing
%\def\baselinestretch{1.5}
\usepackage{setspace}
\setstretch{1.5}


% Place UoS Logo on Title Page (this package modifies the "\maketitel" command.)
\usepackage{titling}
\pretitle{%
  \begin{flushright}
  \vspace{-9.5cm}
%  \includegraphics[width=5cm,natwidth=472,natheight=531]{logo} \\[7cm]
  \includegraphics[width=5cm]{logo} \\[6cm]
  \end{flushright}
  \begin{center}
  \LARGE
}
\posttitle{\end{center}}

\title{PhD Title \\ PhD Thesis}
\author{James Wood
\\ \small Mathematically Structured Programming\\[-0.8ex]
\small Computer and Information Sciences\\[-0.8ex]
\small University of Strathclyde, Glasgow\\
}




%%%%%%%%%%%%%%%%%%%%%%%%%%%%%%%%%%%%%%%%%%%%%%%%%%%%%%%%%%%%%%
\begin{document}

\maketitle


%%%%%%%%%%%%%%%%%%%%%%%%%%%%%%%%%%%%%%%%%%%%%%%%%%%%%%%%%%%%%%
\frontmatter
%%%%%%%%%%%%%%%%%%%%%%%%%%%%%%%%%%%%%%%%%%%%%%%%%%%%%%%%%%%%%%

% Declaration
\topskip0pt
\vspace*{\fill}
\noindent
\begin{quote}
  \centering
  This thesis is the result of the author's original research. It has been composed by the author and has not been previously submitted for examination which has led to the award of a degree. \\[5pt]
  %
  The copyright of this thesis belongs to the author under the terms of the United Kingdom Copyright Acts as qualified by University of Strathclyde Regulation 3.50. Due acknowledgement must always be made of the use of any material contained in, or derived from, this thesis. \\[5pt]
  %
  %Signed: \\
  %Date:
\end{quote}
\vspace*{\fill}



\chapter{Abstract}



\tableofcontents

\addcontentsline{toc}{chapter}{List of Figures}
\listoffigures

\addcontentsline{toc}{chapter}{List of Tables}
\listoftables



\chapter{Preface/Acknowledgements}
I would like to acknowledge...

This document is adapted from the template by Jethro Browell
(\url{https://www.overleaf.com/latex/templates/thesis-template-for-university-of-strathclyde/nfnrnmjqyxqg}),
which was licensed under CC BY 4.0.



%%%%%%%%%%%%%%%%%%%%%%%%%%%%%%%%%%%%%%%%%%%%%%%%%%%%%%%%%%%%%%
\mainmatter
%%%%%%%%%%%%%%%%%%%%%%%%%%%%%%%%%%%%%%%%%%%%%%%%%%%%%%%%%%%%%%


\chapter{Introduction}

\chapter{Simply typed $\lambda$-calculus}
  \section{Basics}
  \section{Kits}
  \section{Natural deduction vs sequent calculus}
  In a seminal paper~\cite{Gentzen64}, Gerhard Gentzen introduces two syntactic
paradigms which remain among the most studied to this day.
These paradigms are \emph{natural deduction} and \emph{sequent calculus}, as
exemplified by natural deduction calculi NJ and NK, and sequent calculi LJ and
LK.
Restricting attention to just the intuitionistic systems NJ and LJ, these
actually differ in two orthogonal ways, which I shall prise apart in this
section.
The simpler distinction is that the logical rules in NJ are introduction and
elimination rules, whereas the logical rules in LJ are left and right rules.
But the more important distinction for this thesis is that where NJ has
assumptions, LJ has sequents explictly manipulated by structural rules.
I take the latter distinction to define natural deduction and sequent calculus,
and wherever I need to make the former distinction I shall speak of
\emph{intro-elim systems} and \emph{left-right systems}.

I will use the rest of this section as follows.
First, I introduce NJ and LJ, and some of their basic metatheory.
Then, I will give examples of systems intermediate between Gentzen's natural
deduction and sequent calculi: a left-right natural deduction calculus
$\mu\tilde\mu$, and an intro-elim sequent calculus BBdPH\@.
Both of these examples will reappear in later chapters.

\subsection{Intro-elim natural deduction: NJ}

\subsection{Left-right sequent calculus: LJ}

\subsection{Left-right natural deduction: $\mu\tilde\mu$}
The $\mu\tilde\mu$-calculus~\cite{CH00} (also known as
$\overline\lambda\mu\tilde\mu$ or system L, and closely related to Wadler's
dual calculus~\cite{Wadler03}) can be seen as an adaptation of natural deduction
to classical logic.
Though originally presented as a sequent calculus, the underlying natural
deduction calculus was later given by Herbelin~\cite[p.\ 12]{Herbelin-hab}, and
I will follow the latter.
While Gentzen gave a natural deduction calculus NK for classical logic, NK
relies on adding the \emph{axiom} of excluded middle.
As axioms are not systematic parts of the calculus, they can hinder or break
metatheoretic properties like normalisation.
In contrast, the $\mu\tilde\mu$-calculus allows us to \emph{derive} excluded
middle from entirely systematic components.

In NJ, a derivation of $A$ from assumptions $\Gamma$ tells us that if each
formula in $\Gamma$ is \emph{true}, then $A$ is also \emph{true}.
The $\mu\tilde\mu$-calculus generalises the picture by allowing us to have
both \emph{true} and \emph{false} assumptions, and allowing us to conclude that
some $A$ is \emph{true}, that some $A$ is \emph{false}, or that we have reached
a contradiction.
% A similar judgement of contradiction appears in Prawitz' classical natural
% deduction calculus~\cite{Prawitz65}.
Following Herbelin, we notate the judgement that $A$ is true as ${}\vdash A$,
that $A$ is false as $A \vdash{}$, and of contradiction as $\vdash$.
The only way to derive a contradiction is to find some $A$ such that
${}\vdash A$ and $A \vdash{}$.
Meanwhile, we can derive ${}\vdash A$ by assuming $A \vdash{}$ and deriving
$\vdash$, i.e., we can prove $A$ by assuming that $A$ is false and deriving a
contradiction.
Dually, we can derive $A \vdash{}$ by assuming ${}\vdash A$ and deriving
$\vdash$.
These three methods of derivation are encoded in the following rules.

\begin{mathpar}
  \inferrule*[right=Cut]
  {{}\vdash A \\ A \vdash{}}
  {\vdash}

  \and

  \inferrule*[right=$\mu$]
  {
    [A \vdash{}] \\\\ \vdots \\\\ \vdash
    %\inferrule*[fraction={~~~}]
    %{[A \vdash{}] \\\\ \vdots}
    %{\vdash}
  }
  {{}\vdash A}

  \and

  \inferrule*[right=$\tilde\mu$]
  {[{}\vdash A] \\\\ \vdots \\\\ \vdash}
  {A \vdash{}}
\end{mathpar}

The rules for logical connectives describe how to \emph{prove} and how to
\emph{refute} a formula whose principal connective is that connective.
These correspond strongly with the right and left rules, respectively, of LJ,
and for this reason, $\mu\tilde\mu$ is usually described elsewhere as a
sequent calculus.
For example, we could choose the following rules for disjunction.
To prove $A \vee B$, we can assume that both $A$ and $B$ are false, and derive
a contradiction.
To refute $A \vee B$, we can refute $A$ and $B$ separately.

\begin{mathpar}
  \inferrule*[right=$\vee$-r]
  {[A \vdash{}][B \vdash{}] \\\\ \vdots \\\\ \vdash}
  {{}\vdash A \vee B}

  \and

  \inferrule*[right=$\vee$-l]
  {A \vdash{} \\ B \vdash{}}
  {A \vee B \vdash{}}
\end{mathpar}

\subsection{Intro-elim sequent calculus: BBdPH}

Term assignment system introduced in~\cite{BBdPH93}.

  \section{Generic syntax}

\chapter{Linearity (the struggle)}
  \section{Intuitionistic linear logic}
  \section{Mechanisation survey}
  \section{DILL}
  DILL provides a syntax in which $\oc$ is characterised by a universal
  property.
  There is no need for any rule to depend on the entire context and rebind
  variables (as done by BBdPH), so is more ND-like.
  DILL makes $\oc$ behave more like a positive type former (see the deep
  identity function $\oc A \vdash \oc A$).
  DILL has been extended to dependent types: Vakar15.

  The idea of split contexts can also be applied to left-/right-rule systems.
  (Define this DILL variant.)

  \section{Notice what additive/multiplicative/exponential rules look like}

\chapter{Usage restriction via semirings}
  \section{Why semirings?}
  Addition gives the structural rules.
  Multiplication gives modalities.

  Compare with other work.
  About here, I can talk about Granule-style $\otimes$-pattern matching.

  \section{What are linear renaming and substitution?}
  \section{$\lambda\mathcal{R}$}

  \section{Linear natural deduction}
  This should stay fairly general, so it could be compiled to a DILL-style
  calculus as well as the one with usage annotations.
  Our framework compiles a ND system down to a sequent calculus.
  It isolates the building blocks of typing rules ($\vdash$, $\wedge$, $*$,
  $\Box$, $\cdot$) so that we can do more generic programming (e.g., usage
  checker).

\chapter{Semantics in worldly relations}

\chapter{Weighted multicategories}

\chapter{A framework for usage-restricted calculi}
  \chapter{A framework for usage-restricted calculi}\label{sec:framework}

In \cref{sec:semirings}, we saw how to use parametrisation over a partially
ordered semiring to recreate a range of usage-aware calculi.
However, $\name$, with its fixed set of type formers and syntactic forms, is a
long way from capturing the full range of linear-like programming languages
studied in the literature and required in practice.

In this chapter, I take the framework for typed syntaxes with binding developed
by \citet{AACMM21} and apply the principles we discovered in
\cref{sec:semirings} to yield a framework allowing for semiring-based usage
restrictions on variables.
Syntactically, I claim that this framework ranges over all finitiary
variable-based simply typed semiring-annotated calculi, with justification by
comparison to the framework of \citet{AACMM21} and some novel examples in
\cref{sec:other-syntaxes}.
I also derive analogues to some of the semantic results of \citet{AACMM21},
strengthening them to take advantage of usage restrictions.

The work in this chapter is fully mechanised in Agda, which allows me to be
precise about the various levels of domain-specific languages which appear.
I assume that the reader is familiar with the bunched connectives introduced in
\cref{fig:bunched} and the usage-aware environments of \cref{def:lr-env}.

\section{Syntax}

We take the insights of the previous section and use them to build a
generic framework for posemiring-annotated substructural systems in
Agda. We will first show \emph{descriptions} of systems, which are
comprised of rules that have premises combined using the bunched
combinators. We then show how to construct the Agda data type of
intrinsically well scoped, typed, and resourced terms for any given
system of our framework. We use the prototypical system from
\cref{fig:lr-comb} as a running example. \cref{sec:other-syntaxes}
presents further examples that our framework can express.

We now start to use Agda notation for record and data type
declarations, to emphasise that our framework has been implemented.

\subsection{Descriptions of Systems}

% We capture the form of rules exemplified previously\todo{Previously?} via
% \emph{descriptions} of rules.
% The key to making these descriptions work is that they only allow syntactic
% forms that preserve environments.
% These forms are: absence and multiplicity of subterms with the same usage
% annotations, absence and multiplicity of subterms with summed usage annotations,
% scaling of a subterm, and variable binding.\todo{Switching to Agda}

\paragraph{\AgdaDatatype{Premises}, \AgdaRecord{Rule}s, and \AgdaRecord{System}s.}

A type \AgdaRecord{System} is made up of multiple \AgdaRecord{Rule}s.
Each \AgdaRecord{Rule} comprises a tree of \AgdaDatatype{Premises} and
a type of conclusion. We assume that there is a
$\AgdaBound{Ty} : \AgdaPrimitiveType{Set}$ of types for the system in
scope.

The \AgdaDatatype{Premise} data type describes premises of rules,
using the bunched combinators from the previous section. A single
premise is introduced by the
\AgdaInductiveConstructor{$\langle$\_`$\vdash$\_$\rangle$}
constructor.  This allows binding of additional variables
\AgdaBound{$\Delta$} (with specified types and usage annotations) and
the specification of a conclusion type \AgdaBound{A} for this premise.
The remaining constructors are descriptions for the first-order
bunched connectives, and will be interpreted directly as such, below.

\ExecuteMetaData[\Syntaxtex]{Premises}

A \AgdaRecord{Rule} is a pair of some \AgdaDatatype{Premises} and a
conclusion. We use an infix arrow as a suggestive notation for rules.

\ExecuteMetaData[\Syntaxtex]{Rule}

Finally, a \AgdaRecord{System} consists of a set of rule labels (i.e.,
constructor names), and for each label a decsription of the
corresponding rule. We use $\rhd$ as infix notation for systems to
associate the label set with the rules.

\ExecuteMetaData[\Syntaxtex]{System}

\paragraph{\cref{fig:lr-comb} as a \AgdaRecord{System}.}

As an example, we transcribe the system defined in
\cref{fig:lr-comb} into a description.  We give the set of types of
this system as a data type \AgdaDatatype{Ty} (together with a base
type \AgdaInductiveConstructor{$\iota$}). We assume that there is a
posemiring \AgdaInductiveConstructor{Ann} in scope for the
annotations.There is one label for each instantiation of a logical
rule, but the labels contain no further information about subterms or
restrictions on the context. This will be provided when we associate
labels with \AgdaRecord{Rule}s in a \AgdaRecord{System}.

\noindent
\begin{minipage}[t]{0.5\textwidth}
  \ExecuteMetaData[\PaperExamplestex]{Ty}
  \ExecuteMetaData[\PaperExamplestex]{Side}
\end{minipage}
\begin{minipage}[t]{0.5\textwidth}
  \ExecuteMetaData[\PaperExamplestex]{qlR}
\end{minipage}

To build a system, we associate with each label a rule:

\ExecuteMetaData[\PaperExamplestex]{lR}

Compared to \cref{fig:lr-comb}, modulo the Agda notation, we can see
that the fundamental structure has been preserved: the rules match
one-to-one, and the bunched premises are the same. A major difference
is that we do not include a counterpart to the
\AgdaInductiveConstructor{var} rule in a
\AgdaRecord{System}. Variables are common to all the systems
representable in our framework.

\paragraph{Terms of a \AgdaRecord{System}.}

The next thing we want to do is to build terms in the described type system.
The following definitions are useful for talking about types indexed over
contexts, judgement forms, and judgement forms admitting newly bound variables,
respectively.

\ExecuteMetaData[\Syntaxtex]{OpenFam}

To specify the meaning of descriptions, we assume some \AgdaBound{X} : \AgdaFunction{ExtOpenFam},
% \ExecuteMetaData[\Interpretationtex]{X},
over which we form one layer of syntax, using the function
\AgdaFunction{$\llbracket$\_$\rrbracket$p} that interprets
\AgdaDatatype{Premises} defined below.  The first argument to
\AgdaBound{X} is the new variables bound by this layer of syntax, as
exemplified in the first clause of
\AgdaFunction{$\llbracket$\_$\rrbracket$p}.  The second argument is
the context containing the variables being carried over from the
previous layer.  Notice that this is not, in general, the same as the
context from the previous layer, because the usage annotations may
have been changed by connectives like
\AgdaInductiveConstructor{\_`$*$\_} and
\AgdaInductiveConstructor{\_`$\cdot$\_}.  The third argument is the
type of subterm required.

With the first clause of \AgdaFunction{$\llbracket$\_$\rrbracket$p} explained,
the rest are simply interpretations of premises into bunched combinators.

\ExecuteMetaData[\Interpretationtex]{semp}

The interpretation of a \AgdaRecord{Rule} checks that the rule targets
the desired type and then interprets the rule's premises \AgdaBound{ps}.
Notice that the interpretation of the premises is independent of the conclusion
of the rule, which accounts for the difference in type between
\AgdaFunction{$\llbracket$\_$\rrbracket$p} and
\AgdaFunction{$\llbracket$\_$\rrbracket$r}.

\ExecuteMetaData[\Interpretationtex]{semr}

The interpretation of a \AgdaRecord{System} is to choose a rule label
\AgdaBound{l} from \AgdaBound{L} and interpret the corresponding rule
\AgdaBound{rs}\AgdaSpace{}\AgdaBound{l} in the same context and for the same
conclusion.

\ExecuteMetaData[\Interpretationtex]{sems}

The most obvious way to make such an \AgdaBound{X} is to use some existing
\AgdaFunction{OpenFam} on an extended context.
We defined \AgdaFunction{Scope} to do this: take the new variables
\AgdaBound{$\Delta$}, concatenate them onto the existing context
\AgdaBound{$\Gamma$}, and pass the extended context onto the judgement
\AgdaBound{T}.

\ExecuteMetaData[\Syntaxtex]{Scope}

%{\color{red}(Forward ref: for now, we could have inlined \texttt{Scope}.)}

We use \AgdaFunction{Scope} to deal with new variables in syntax.
Terms resemble the free monad over a layer-of-syntax functor, though
that picture is complicated by variable binding.  A term is either a
variable or a use of a logical rule together with terms for each of
the required subterms. The \AgdaFunction{Size} argument is where we
use sized types to convince Agda that this type is strictly positive.

\ExecuteMetaData[\Termtex]{Term}

Terms defined like this are still quite difficult to write, mainly because of
frequently changing usage contexts and the need for proofs that they all match
up.
We will see how to automate these proofs in \cref{sec:usage-elaborator}.

%Here is an example term, using the \AgdaFunction{$\lambda$R} system.
%First, for ease of writing, we introduce pattern synonyms for each of the
%typing rules we use.

%\ExecuteMetaData[\PaperExamplestex]{patterns}

%Our example term is a function that flips a tagged union wrapped in an
%arbitrarily annotated \emph{bang}.
%Much of the effort in writing such a term goes into writing the various
%relatedness proofs between usage contexts --- observing, for example, that two
%usage contexts sum together to make a third, or that a usage context used for
%a variable is a basis vector.
%We give a method of automating these proofs in \cref{sec:usage-elaborator}.
%\todo{To be clear, we don't actually write this.}

%\ExecuteMetaData[\HeavyItex]{lR-term}

% A layer of syntax supports the following functorial action.

% \ExecuteMetaData[\Maptex]{map-s-type}

\subsection{Other syntaxes and syntactic forms}\label{sec:other-syntaxes}

\paragraph{The system $\mu\tilde\mu$.}
We can encode a usage-annotated version of System $L$/the
$\mu\tilde\mu$-calculus~\cite{CH00} --- a syntax for classical logic --- in
such a way that contexts capture the undistinguished parts of the sequent.
As such, the generic substitution lemma we get in \cref{sec:kits} is the form
of substitution required in standard $\mu\tilde\mu$-calculus metatheory.
Though the $\mu\tilde\mu$-calculus is originally described as a sequent
calculus~\cite{CH00}, we use the techniques of
\citet[p.~12]{herbelin-hab} and \citet{LC06} to present it as a natural
deduction system, thus giving a notion of \emph{variable} to the system.

Unlike the single judgement form of \name{} and standard simply typed
$\lambda$-calculi, the $\mu\tilde\mu$-calculus has three judgement forms:
terms, coterms, and commands.
Read logically, terms and coterms are seen to, respectively, prove and refute
propositions (types), while commands exhibit contradictions.
This means that the abstract \AgdaBound{Ty} in the generic framework is
instantiated to \AgdaDatatype{Conc} (for \emph{conclusion}) as below, with
\AgdaDatatype{Ty} not being exposed directly to the generic framework.
For now, we just consider multiplicative disjunction $\parr$ (\emph{par}) and
negation/duality, beside an uninterpreted base type.
These are enough to exhibit classical behaviour.

\noindent
\begin{minipage}[t]{0.5\textwidth}
  \ExecuteMetaData[\MuMuTildetex]{Ty}
\end{minipage}
\begin{minipage}[t]{0.5\textwidth}
  \ExecuteMetaData[\MuMuTildetex]{Conc}
\end{minipage}

With \AgdaBound{Ty} instantiated as \AgdaDatatype{Conc}, all terms are assigned
\AgdaDatatype{Conc} type, as are all the variables.
No variables are given \AgdaInductiveConstructor{com} type, similar to how in
the bidirectional typing syntax of \citet[p.~25]{AACMM21}, no variables are
given \AgdaInductiveConstructor{Check} type.
How to observe this invariant is covered in the latter paper, so we will not
repeat it here (having not yet seen how to write traverals on terms).

The syntax comprises a \emph{cut} between a term and a coterm of the same type,
the eponymous $\mu$ and $\tilde\mu$ constructs for proof by contradiction, and
then term and coterm (introduction and elimination) forms for negation and
\emph{par}.

\ExecuteMetaData[\MuMuTildetex]{MMT}

%With a collection of pattern synonyms and the machinery from
%\cref{sec:usage-elaborator}, we can write an example term: a function which
%flips the disjuncts of a \emph{par}.

%\ExecuteMetaData[\MuMuTildeTermtex]{patterns}
%\ExecuteMetaData[\MuMuTildeTermtex]{myComm}

\paragraph{Duplicability}
There is one more bunched combinator we have experimented with adding to the
framework:

\[
  \plr{\Box T}\,\grR \coloneqq \Sigma\grRprime.~\plr{\grRprime \leq \grR}
  \times \plr{\grRprime \leq \gr0}
  \times \plr{\grRprime \leq \grRprime + \grRprime}
  \times T\,\grRprime
\]

The idea of $\plr{\Box T}\,\grR$ is to assert that $\grR$, or some refinement
of it, can be both discarded and duplicated indefinitely, and in the
refinement we have a $T$.
We use this combinator to introduce subterms that are used an unknown number of
times, for example the continuations of the eliminator of an inductive type,
or other fixed points.
We can also use it in linear/non-linear style systems~\cite{Benton94} to make
sure linear variables are not available in the intuitionistic fragment.

Adding the $\Box$ combinator is the only thing we have found that requires our
linear maps be functional rather than merely relational.


\section{Semantics}

Given a $\V$-environment $\Gamma \Rightarrow \Delta$, the function
\AgdaFunction{semantics} we define in this section assigns a
$\C$-value in context $\Gamma$ to every term in context $\Delta$,
where $\C$ is an \AgdaFunction{OpenFam} being the carrier of the
semantic interpretation of terms ($\V$ being the semantic
interpretation of variables). Before we can define
\AgdaFunction{semantics}, we need to treat recursion through rules'
premises (\cref{sec:functorial}) and extension of environments when
going under variable binders (\cref{sec:kripke}).
\todo{Maybe split the chapter here. Syntax/semantics}

% Our goal in this section is to define \AgdaFunction{semantics}, a
% recursor that turns a term into a \AgdaBound{$\C$}-value using a
% \AgdaBound{$\V$}-environment, in a type preserving way:\bob{Get rid of
%   ``body'' here}

% \ExecuteMetaData[\Semanticstex]{semantics-type}

% The \AgdaBound{$\V$} and \AgdaBound{$\C$} are \AgdaFunction{OpenFam}s,
% representing the interpretations of variables and terms
% respectively. In \cref{sec:traversal} we will see the data that must
% be provided to make a \AgdaFunction{semantics} for a given
% system. Before that, we must see how to deal with the two complicated
% features of our syntax: the usage annotations (\cref{sec:functorial})
% and variable binding (\cref{sec:kripke}). \todo{fwd ref to where these are used}

\subsection{A layer of syntax is functorial}\label{sec:functorial}

A basic property of the universe of syntaxes we described in \cref{sec:syntax}
is that every syntax supports a functorial action on subterms, realised by the function \AgdaFunction{map-s}.
Its type says that to map a function \AgdaBound{f}
over a layer of syntax, there must be a linear map \AgdaBound{F} relating the
domain and codomain usage contexts, and \AgdaBound{f} should be usable
wherever the domain and codomain usage contexts are similarly related by
\AgdaBound{F}.

\ExecuteMetaData[\Maptex]{map-s-type}

This generality is needed because usage contexts change between
a term and its immediate subterms---they are decomposed according to the bunched connectives used in the rules.
\AgdaBound{X} and \AgdaBound{Y} are \AgdaFunction{ExtOpenFam}s, with
\AgdaBound{$\Theta$} being the context extension for a subterm (i.e., the
variables newly bound in that subterm).
Unlike usage annotations, types in the contexts \AgdaBound{$\gamma$} and \AgdaBound{$\delta$}, and the conclusion types implicit here, are preserved throughout.
This is the essence of the usage annotation based approach---we use traditional techniques for variable binding, with the usage annotations layered on top.

The heart of \AgdaFunction{map-s} is \AgdaFunction{map-p}, which recursively
works through the structure \AgdaBound{ps} of premises of the rule applied,
acting on each subterm it finds.
Here, particularly in the clauses for \AgdaInductiveConstructor{`$\sep$} and
\AgdaInductiveConstructor{`$\cdot$}, we see why it is not enough for the
function on subterms to apply at usage contexts \AgdaBound{P} and \AgdaBound{Q}
--- rather, it also needs to apply at any similarly related \AgdaBound{P$'$}
and \AgdaBound{Q$'$}.
In the case of \AgdaInductiveConstructor{`$\sep$}, we have that
$\grP \leq \grP_M + \grP_N$, with \AgdaBound{M} and \AgdaBound{N} being
collections of subterms in usage contexts $\grP_M$ and $\grP_N$, respectively.
Linearity of \AgdaBound{F} yields $\grQ_M$ and $\grQ_N$ such that
$\grQ \leq \grQ_M + \grQ_N$ and we use \AgdaFunction{map-p} recursively at
$(\grP_M, \grQ_M)$ and $(\grP_N, \grQ_N)$ on \AgdaBound{M} and \AgdaBound{N}.
The cases for \AgdaInductiveConstructor{`$\cdot$} and
\AgdaInductiveConstructor{`$I^*$} are similar, each using a different aspect
of linearity.
In contrast, the cases for \AgdaInductiveConstructor{`$\dot1$} and
\AgdaInductiveConstructor{`$\dot\times$}, which are the only constructors used in fully structural
systems, do not involve any changes in the usage contexts.

\ExecuteMetaData[\Maptex]{map-p}

\subsection{The Kripke function space}\label{sec:kripke}

At this point we introduce a minor generalisation to
\AgdaFunction{OpenFam} and \AgdaFunction{ExtOpenFam}:
\AgdaBound{I}\AgdaSpace{}\AgdaFunction{---OpenFam} and
\AgdaBound{I}\AgdaSpace{}\AgdaFunction{---ExtOpenFam}.  We obtain the
definition of \AgdaBound{I}\AgdaSpace{}\AgdaFunction{---OpenFam} by
replacing the textual occurrence of \AgdaBound{Ty} by the parameter
\AgdaBound{I}.

The definition \AgdaFunction{Kripke}\,$\V$\,$\C$\,$\Delta$ is a kind
of function space that describes a $\C$ value parametrised by
$\Delta$-many additional $\V$s (all correctly typed and usage
annotated). It is used to describe how to go under binders in a
Higher-Order Abstract Syntax style---to go under a binder we must
provide semantic interpretations for all the additional variables:

% When going under binders during a recursion, the context will be extended by some $\Theta$. This means that the current environment must be extended with $\Theta$s-worth of $\V$s

% we need the ability to say that

% Kripke V C is given the extension \Theta

% In \cref{sec:terms}, we defined \AgdaFunction{Scope} to let a
% judgement-indexed family admit context extensions. However, a key
% component of our generic semantic traversal is to make use of the open
% family \AgdaBound{$\V$} of \emph{values}, which are the sort of thing
% we store in an environment.  The definition \AgdaFunction{Kripke}
% gives an alternative to \AgdaFunction{Scope} which interprets the
% newly bound variables via a requirement of $\V$-values rather than
% extra assumptions for the $\C$-computation.

\ExecuteMetaData[\Semanticstex]{Kripke}

\AgdaFunction{Wrap}
is a device that turns any type family into an equivalent type family
that is judgementally injective in its indices, which helps with
Agda's type inference.
It turns the type family into a parametrised
record with a single field \AgdaField{get} whose type is the type in
the body of the $\lambda$-abstraction.
For understanding the meaning of
\AgdaFunction{Kripke}, \AgdaFunction{Wrap} can be ignored.

If $\Delta$ is of the form $\gr{s_1}B_1, \ldots, \gr{s_n}B_n$, then
\ExecuteMetaData[\Snippetstex]{KripkeVCDGA}\ is equivalent to
\ExecuteMetaData[\Snippetstex]{KripkeExpanded}\ by Currying.  That is
to say, the Kripke function is expecting a value for each newly bound
variable, at the multiplicity of its annotation, together with the
resources supporting each of those values. We use the ``magic wand''
function space here to enforce the invariant that the freshly bound
variables have usage annotations that are added to the existing
variables, not shared with them. The use of the
\AgdaFunction{$\Box^r$} modality ensures that we can still use it in
the presence of additional variables introduced by weakening.

\AgdaFunction{Kripke} is functorial in the \AgdaBound{$\C$} argument,
as witnessed by the \AgdaFunction{mapK$\C$} function, which is essentially
post-composition:

\ExecuteMetaData[\Semanticstex]{mapKC}

% is exemplified by the following construct
% \AgdaFunction{reify}, where we weaken \AgdaBound{$\Gamma$} by a $\gr0$ed-out
% version of \AgdaBound{$\Delta$}.
% The \AgdaBound{$\Delta$} then gets filled in by the $\V$-values.

% \bob{Move this para}
% This means that \AgdaBound{A} in the definition of \AgdaFunction{Kripke} has
% type \AgdaBound{I}, rather than specifically \AgdaBound{Ty}.
% We use this generality later in \AgdaFunction{extend}, setting \AgdaBound{I}
% to \AgdaDatatype{Ctx}.

\subsection{Semantic traversal}\label{sec:traversal}

We can now state the data required to implement a traversal assigning
semantics to terms. For open families $\V$ and $\C$, interpreting
variables and terms respectively, we assume that $\V$ is renameable,
that $\V$ is embeddable in $\C$, and that we have an algebra for a
layer of syntax, where bound variables are handled using the Kripke
function space:

% The aim of this subsection is to give an alternative recursion principle for
% terms that incorporates some of the environment-handling seen in the
% implementations of renaming and substitution.
% The rest of this section assumes the following: a renameable open family
% \AgdaBound{$\V$} that embeds into the open family \AgdaBound{$\C$}, and an
% algebra for a layer of syntax at \AgdaBound{$\C$}.

\ExecuteMetaData[\Semanticstex]{Semantics}

%\ExecuteMetaData[\Semanticstex]{Comp}

We mutually define the action \AgdaFunction{semantics} and its lemma
\AgdaFunction{body}.
The purpose of \AgdaFunction{semantics} is to turn a term into a
\AgdaBound{$\C$}-value using a \AgdaBound{$\V$}-environment and the fields of
\AgdaRecord{Semantics}.
Meanwhile, \AgdaFunction{body} does a similar job, but also deals with
newly bound variables.
In particular, \AgdaFunction{body} takes a term in a context extended by
\AgdaBound{$\Theta$}, and produces a Kripke function from
\AgdaBound{$\V$}-values for \AgdaBound{$\Theta$} to \AgdaBound{$\C$}-values.

\ExecuteMetaData[\Semanticstex]{semantics-type}

To implement the new recursor \AgdaFunction{semantics}, we use the standard
recursor, which in one case gives us a variable \AgdaBound{v}, and in the other
gives us a structure of subterms \AgdaBound{M}, each of which is in an extended
context.
To deal with a variable \AgdaBound{v}, we look it
up in the environment \AgdaBound{$\rho$}, then use the
\AgdaField{$\sem{\text{var}}$} field to map the resulting
\AgdaBound{$\V$}-value to a \AgdaBound{$\C$}-value.
To deal with a structure of subterms \AgdaBound{M}, we use the functoriality of
the syntactic structure to consider each subterm separately.
On a subterm, we apply \AgdaFunction{body}, which amounts to a recursive call
to \AgdaFunction{semantics} with an extended environment.
Recall that \AgdaFunction{relocate} (\cref{thm:env-resize}) adjusts the
environment \AgdaBound{$\rho$} to work in the usage contexts of the subterms.

\ExecuteMetaData[\Semanticstex]{semantics}

For \AgdaFunction{body}, we are given a subterm \AgdaBound{M}, to
which we want to apply \AgdaFunction{semantics}.  To do so, we need an
extended version of the initial environment \AgdaBound{$\rho$}. We
express this as the generation of a Kripke function that produces the
extended environment given interpretations of the fresh variables. We
take \AgdaBound{$\rho$}, which is an environment covering
\AgdaBound{$\Delta$}, and \AgdaBound{$\sigma$}, which is an
environment covering \AgdaBound{$\Theta$}, and glue them together
using the inductive rules for generating environments, after having
renamed \AgdaBound{$\rho$} via \cref{thm:env-ren} to make it fit the
new context \AgdaBound{$\Gamma^+$} (intended to be
\ExecuteMetaData[\Snippetstex]{GT}):

\ExecuteMetaData[\Semanticstex]{extend}

% The best we can achieve without identity environments for \AgdaBound{$\V$} is
% a Kripke function returning an extended environment.
To define \AgdaFunction{body}, we use \AgdaFunction{mapK$\C$} to
post-compose the environment extension by the
\AgdaSymbol{$\lambda$}-function taking an extended environment and
acting with it on \AgdaBound{M}.

\ExecuteMetaData[\Semanticstex]{body}

% \todo{FIX} Under the assumption that \AgdaBound{$\V$} is renameable, we can decompose
% \cref{thm:lr-bind} as
% \AgdaFunction{reify}\AgdaSpace{}\AgdaOperator{\AgdaFunction{$\circ$}}%
% \AgdaSpace{}\AgdaFunction{extend}, with \AgdaFunction{extend} defined below.
% We can think of \AgdaFunction{extend} as our best effort to extend an
% environment by \AgdaBound{$\Theta$} without access to an identity environment
% at \AgdaBound{$\Theta$}.


\section{Example semantics}\label{sec:example-semantics}

\subsection{Renaming and substitution}\label{sec:kits}

In an unpublished note, \citet{McBride05} gives a parametrised traversal
yielding homomorphisms of syntax.
The parameters are collected in the record \AgdaRecord{Kit}.
We make a minor change to the original presentation, where instead of our
\AgdaField{ren\textasciicircum{}$\V$} field, \citeauthor{McBride05} has the
field \AgdaField{wk} allowing only context extensions.
As for the other two fields, \AgdaField{vr} allows us to map variables to
$\V$-values, so as to put newly bound variables in environments; and
\AgdaField{tm} allows us to extract terms from $\V$-values, as required when
we use the environment to evaluate a free variable.

\ExecuteMetaData[\Syntactictex]{Kit}

Where \citeauthor{McBride05} gave the traversal explicitly, we go via our
generic semantic traversal.
The first two fields of \AgdaRecord{Semantics} derive directly from fields of
\AgdaRecord{Kit}.
Meanwhile, to handle term constructors, we first \AgdaFunction{reify} to get a
collection of traversed subterms, and then use \AgdaInductiveConstructor{`con}
to assemble these subterms into a similarly shaped syntactic form as we started
with.
The \AgdaField{vr} field is used implicitly in \AgdaFunction{reify}, as it is
used to show that $\V$-identity environments exist.

\ExecuteMetaData[\Syntactictex]{kit-to-sem}

The action of a syntactic traversal on logical rules is basically fixed: we
preserve the logical rule and extend the environment with any newly bound
variables according to \AgdaField{vr}.
Meanwhile, the action on variables is relatively unconstrained: we look up the
variable in the environment to get a $\V$-value, then transform that $\V$-value
into a term using \AgdaField{tm}.

The idea of renaming is that variables replace variables, whereas with
substitution, terms replace variables.
This translates to environments for renaming containing $\sqni$-values
(variables), and environments for substitution containing $\vdash$-values
(terms).

%To implement renaming and substitution for terms, we now just implement
%syntactic kits for variables and terms, respectively.

\ExecuteMetaData[\Syntactictex]{Ren-Kit}

Notice that \AgdaFunction{ren\textasciicircum$\vdash$}, witnessing the fact
that terms are renameable, is a corollary of \AgdaFunction{Ren-Kit}.

\ExecuteMetaData[\Syntactictex]{Sub-Kit}

\subsection{A denotational semantics}

\todo{Introduction}
To abbreviate this section, we use a simplified syntax compared to \name{}.
We allow for an arbitrary family of base types \AgdaBound{BaseTy}, and a single
type former \mbox{\ExecuteMetaData[\WReltex]{rAToB}}, equivalent to
\mbox{\ExecuteMetaData[\PaperExamplestex]{BangrAToB}} from the earlier system.

\ExecuteMetaData[\WReltex]{Ty}

In the term syntax, $\lambda$-abstraction now binds a variable with annotation
\AgdaBound{r}, while application needs to scale its argument by \AgdaBound{r}
(both in accordance with the function type they are acting on).

\ExecuteMetaData[\WReltex]{AnnArr}

In this subsection, we take the usage annotations to be the 4-element variance
posemiring.
\todo{This works for any semiring.}
We establish the property that all terms are monotonic in their free variables.
This monotonicity can be covariant or contravariant (or neither or both)
depending on the annotation of each free variable.
This provides an additional example to those of \citet{AbelBernardy2020}.
\todo{Cite before here.}

We will take semantics of this system into
\emph{worldly relations}~\cite{AbelBernardy2020}.
A worldly relation over a poset of worlds \AgdaBound{W} is a set over which
we have a \AgdaBound{W}-indexed binary relation satisfying a presheaf-like
property with respect to the order on \AgdaBound{W}.

\ExecuteMetaData[\WReltex]{WRel}

\begin{example}
  When \AgdaBound{W} is the 1-element set, a worldly relation is just a set
  equipped with a binary relation.
\end{example}

Morphisms between worldly relations \AgdaBound{R} and \AgdaBound{S} consist of
a mapping between the underlying sets such that that mapping preserves
relatedness from \AgdaBound{R} to \AgdaBound{S}.

\ExecuteMetaData[\WReltex]{WRelMor}

\todo{Define big intersection.}
When the poset of worlds forms a (relational) commutative monoid, such worldly
relations support a symmetric monoidal closed structure.
We reuse the bunched connectives \AgdaRecord{$I^*$}, \AgdaRecord{$\sep$}, and
\AgdaRecord{$\wand$}, now over worlds rather than contexts.

\ExecuteMetaData[\WReltex]{IR}
\ExecuteMetaData[\WReltex]{tensorR}
\ExecuteMetaData[\WReltex]{lollyR}

The final piece of sematics we need is a \emph{bang} operator.
\todo{No instead}
Instead of requiring extra algebraic structure on the worlds, we allow the
semantic \emph{bang} to be an arbitrary annotation-indexed functor on worldly
relations.
This functor must respect all of the structure on the indices, making it a
graded comonad over multiplication, as well as being lax monoidal at any
particular index \AgdaBound{r}.

\ExecuteMetaData[\WReltex]{Bang}

\begin{example}
  With \AgdaBound{W} being the 1-element set and annotations coming from the
  variance semiring, we can define the following \emph{bang}.
  It is always the identity on the set component, while the relation component
  consists of flipping the relation for contravariance and taking conjunctions
  to achieve both covariance and contravariance.
  When we want neither covariance nor contravariance, we use the always true
  predicate on worlds \AgdaFunction{$\dot1$}.

  \ExecuteMetaData[\Monotonicitytex]{BangR}
\end{example}

To associate semantics to syntax, we start as standard by associating worldly
relations to types.
We also extend the semantics of types to contexts, using \AgdaFunction{I$^R$},
\AgdaFunction{$\otimes^R$}, and \AgdaField{!$^R$} to interpret the empty
context, concatenation, and usage annotations on singletons, respectively.

\ExecuteMetaData[\WReltex]{sem}

The semantics of a term is then to be a morphism from the interpretation of the
context to the interpretation of the term's type.

\ExecuteMetaData[\WReltex]{sem-vdash}

Variables are given semantics by \AgdaFunction{lookup$^R$} (definition omitted).

\ExecuteMetaData[\WReltex]{lookupR-type}

Now, we give a \AgdaRecord{Semantics}.
The choice of \AgdaBound{$\V$} as
\AgdaRecord{\AgdaUnderscore{}$\sqni$\AgdaUnderscore{}} is somewhat arbitrary,
given that a standard denotational semantics would not use intermediate
environments in the same sense as renaming and substitution, but allows us to
reuse the standard facts that variables support renaming and identity
environments.
With this choice of \AgdaBound{$\V$} and \AgdaBound{$\C$}, we interpret
environment entries by \AgdaFunction{lookup$^R$}.
Meanwhile, for the logical rules, we ignore environments by using
\AgdaFunction{reify} to just deal with morphisms in an extended context.
As such, $\lambda$-abstractions are easy to interpret, while applications
require some massaging to remove the extension by an empty context, followed by
some plumbing to split the interpretation of the context according to the usage
constraints and feed the interpretation of the argument \AgdaBound{n$'$} into
the interpretation of the function \AgdaBound{m$'$}.

\ExecuteMetaData[\WReltex]{Wrel}

In order to map open terms to interpretations, we take the action of the
semantics and give the identity renaming as the starting environment.

\ExecuteMetaData[\WReltex]{wrel}

\begin{example}
  We can make a subtraction function from primitive addition and negation on
  integers.
  Subtraction is covariant in its first argument and contravariant in its
  second argument.
  We give the definition in pseudocode, as we have not yet seen how to
  conveniently write terms (\cref{sec:usage-elaborator}).

  \begin{align*}
    &{\sim\sim}p :
      {\uparrow\uparrow}\mathbb Z \multimap
      {\uparrow\uparrow}\mathbb Z \multimap \mathbb Z,
      {\sim\sim}n : {\downarrow\downarrow}\mathbb Z \multimap \mathbb Z
      \vdash \mathnormal{minus} :
      {\uparrow\uparrow}\mathbb Z \multimap
      {\downarrow\downarrow}\mathbb Z \multimap
      \mathbb Z
    \\
    &\mathnormal{minus} \coloneqq \lambda x.~\lambda y.~p\,x\,(n\,y)
  \end{align*}

  We observe that the set component of this term's semantics is just the
  expected Agda function when the two free variables are given appropriate
  interpretations.

  \ExecuteMetaData[\Monotonicitytex]{minus-set}

  Furthermore, the relational component of the semantics yields the free
  theorem that the Agda subtraction so defined is monotonic in the expected way.
  This relies on library proofs that addition and negation are suitably
  monotonic.

  \ExecuteMetaData[\Monotonicitytex]{thm}
\end{example}

\subsection{A usage elaborator}\label{sec:usage-elaborator}

Using the constructs we have seen so far, producing example terms soon becomes
extremely tedious.
We achieved some abbreviation by using pattern synonyms, but we still have to
produce essentially bespoke proofs whenever we use a usage-sensitive part of the
syntax.
The size of each of these proofs is roughly proportional to the number of free
variables, so the amount of proof we have to write grows roughly quadratically
with the size of terms.
An additional factor, which we can't see on paper, is that type checking time
for these proofs soon becomes prohibitive to interactive development.

Our aim in this subsection is to automate usage constraint proofs, making terms
both easier to write and more performant to check.
We invoke the automation by writing terms in a syntax where usage constraints
have been trivialised, and then use a semantic traversal over the simplified
syntax to try to produce a fully elaborated term in the original syntax.
We write the automation in a way that is generic in the syntax description, thus
avoiding repetition and facilitating the prototyping of new type systems.

The type of syntax descriptions depends on the type of usage annotations because
of variable binding.
For example, in the $\oc_{\gr r}$-E rule of \cref{fig:lr-comb}, the right
premise binds a new variable with annotation $\gr r$, where $\gr r$ is drawn
from the ambient posemiring.
The scaling combinator also makes direct reference to the posemiring.
To produce a simplified syntax description, where usage constraints are
trivialised, we set the ambient posemiring to the 1-element $\mathbf0$
posemiring.
In contrast to syntax descriptions, even though types can contain usage
annotations, the type of types does not depend on the type of usage annotations.
This means that, in our simplified syntax, terms have types from the original
system even though variables have trivial usage annotations.
We define the $\mathbf0$ posemiring as follows, being careful to use the
0-field record type \AgdaRecord{$\top$} so that everything algebraic gets
solved by Agda's $\eta$-laws.
Indeed, in this very definition, all of the semiring operations and laws are
canonically inferred.

\ExecuteMetaData[\UsageChecktex]{0-poSemiring}

The elaboration process is monadic.
In particular, we use the \AgdaDatatype{List}/non-determinism monad to give
\emph{all} of the possible annotation choices on the free variables of a term.
We believe the commitment to multiple solutions is inherent when the syntax
contains \AgdaInductiveConstructor{`$\dot1$}.
For example, in the intermediate stages of elaborating
$\plr{\vdash \lambda x.~\plr{*,*}} : A \multimap \top \otimes \top$ with a
usage counting posemiring (assuming reasonable rules for $\top$ and $\otimes$),
it is unclear whether to use the variable $x$ in the left $*$ or the right $*$.
This uncertainty should be reflected in the final result.

The non-deterministic choices we make during elaboration are enumerated by
the fields of \AgdaRecord{NonDetInverses}.
These choices are driven by the typing rules and a candidate usage vector for
the conclusion.
For example, \AgdaField{+$^{-1}$}\AgdaSpace{}\AgdaBound{r} is needed when we
encounter a \AgdaInductiveConstructor{`$\sep$} in the syntax and the candidate
usage annotation we are considering is \AgdaBound{r}.
Then, \AgdaField{+$^{-1}$}\AgdaSpace{}\AgdaBound{r} is a list of pairs of
annotations \AgdaBound{p} and \AgdaBound{q} that \AgdaBound{r} can split into,
together with a proof of the splitting.
For \AgdaField{0\#$^{-1}$} and \AgdaField{1\#$^{-1}$}, inverses to constants,
we are given the candidate \AgdaBound{r} and typically return an empty list if
the constraint cannot be satisfied, or a singleton list containing a proof.
\AgdaField{*$^{-1}$} is used when we encounter scaling, in which case we know
both the scaling factor \AgdaBound{r} (from the syntax description) and the
candidate \AgdaBound{q}.
These inverse operations combine monadically (in fact, applicatively) to give
inverses to the vector operations of zero, addition, scaling, and basis.

\ExecuteMetaData[\UsageChecktex]{NonDetInverses}

We choose the \AgdaBound{$\V$} of our semantics to be (unannotated) variables.
For the \AgdaBound{$\C$}, we consider \emph{functions} from candidate usage
vectors \AgdaBound{R} to the list of elaborated derivations with usage
annotations given by \AgdaBound{R}.
The module name \AgdaModule{U} refers to the fact that we are taking the
ambient posemiring to be $\mathbf0$ in \AgdaFunction{OpenFam}.
The effect on \AgdaFunction{OpenFam} is that the usage annotations of any
contexts we consider are uninformative (hence the \AgdaSymbol{\_} on the left).

\ExecuteMetaData[\UsageChecktex]{C}

To traverse the unannotated terms, we produce a \AgdaRecord{Semantics} over the
unannotated system \AgdaFunction{uSystem}\AgdaSpace{}\AgdaBound{sys}.
We already know that variables are renameable.
To interpret a variable, we consider all the possible proofs that such a
variable could be well annotated, and package them up as a variable term.

\ExecuteMetaData[\UsageChecktex]{elab-sem}

\ExecuteMetaData[\UsageChecktex]{lemma-type}

To actually use \AgdaFunction{elab-sem} on terms, we take the associated
\AgdaFunction{semantics} and pass it the identity environment (an identity
\emph{renaming} in this case, because $\V$ is a family of variables).
The candidate usage vector \AgdaBound{R} will be empty for closed terms, and
otherwise we have to supply the intended usage annotations.



\chapter{Generic programs}
  \section{Usage checker}

\chapter{Investigations that are now easier}
  \section{Linear/non-linear logic}
  In \cref{sec:dup-lnl}, I gave what I claimed to be an encoding of
Linear/non-Linear logic~\citep{Benton94} as a syntax description.
In this section, I rigorously state and prove the correspondence between
\citeauthor{Benton94}'s definition of L/nL and my encoding of it.
Then, I give translations from this encoding to my encoding of $\name$, and
vice versa, using two generic semantic traversals.
These results together should give us confidence that the encoding of L/nL is
correct up to logical equivalence.

\subsection{Encoding L/nL}\label{sec:encoding-LnL}

I will present translations between the systems given by
\cref{fig:LnL-orig,fig:LnL-bunched}.

I use $S$ to range over both linear and intuitionistic variables.
In this section, I use the notations $\Gamma \vdash A$ and $\Gamma \vdash X$
without subscripts on the turnstile to refer to the encoded version of the
calculus.
This notation keeps the encoded calculus distinct from the reference L/nL
calculus I am translating from and to.

The main difference between the original L/nL calculus and the encoded version
is that the encoded version contains some extra ``junk'', not corresponding to
anything in the original L/nL calculus.
This junk includes all of the wrinkles we saw when translating to and from DILL
in \cref{sec:trans-dill} --- where in the semiring-annotated system, variables
annotated $\gr\omega$ (corresponding to intuitionistic variables) can slip into
having annotation $\gr1$ or $\gr0$ whenever there are any algebraic
manipulations of the context.
In linear/non-linear logic, this slipping causes extra problems because
intuitionistic variables are supposed to be of a distinct sort to other
(i.e., linear) variables.
Additionally, we have no means in the framework to correlate types with usage
annotations, so we must deal with free variables carefully to ensure the
required correlation between linear and intuitionistic types and annotations.

With the above remarks in mind, I take it as clear how to translate the original
L/nL calculus into the encoded version, and just state the type of the
translation in \cref{thm:lnl-to-enc} without including a proof.
In contrast, I spend most of this subsection on the reverse translation, which
I provide a proof sketch of in \cref{thm:enc-to-lnl}.

\begin{proposition}\label{thm:lnl-to-enc}
  We can construct the following translations.
  \begin{align}
    (\Theta \vdashC X) &\to (\gr\omega\Theta \vdash X) \\
    (\Theta; \Gamma \vdashL A) &\to (\gr\omega\Theta, \gr1\Gamma \vdash A)
  \end{align}
\end{proposition}

The key property needed to sensibly do the reverse translation is
\emph{linear well-formedness}, as given by \cref{def:lin-well-formed}.
Linear well-formedness says that variables of linear type have linear usage
annotations.
It does not say anything about intuitionistic types and usage annotations for
two reasons.
First, talking about $\gr\omega$ is not sufficiently stable.
As we work up a derivation, the ``slip'' described earlier says that usage
annotations will tend to get larger, i.e.\ more precise.
Therefore, it makes more sense to make conditions of being greater than or equal
to some collection of usage annotations.
Second, it turns out to be unnecessary to add any conditions on variables with
intuitionistic type, because we can just treat them as if they were annotated
$\gr\omega$.
We can forget the specificness of annotations $\gr0$ and $\gr1$ when not needed,
because whatever a specifically annotated variable can do can be done by an
$\gr\omega$-annotated variable.

\begin{definition}\label{def:lin-well-formed}
  A semiring-annotated context for L/nL is \emph{linearly well formed} when all
  linear variables are annotated either $\gr0$ or $\gr1$.
\end{definition}

\begin{lemma}\label{thm:lwf}
  If $\Gamma$ is linearly well formed and $M : \plr{\Gamma \vdash S}$, then the
  context of every subterm of $M$ is linearly well formed.
\end{lemma}
\begin{proof}
  This lemma follows by inspection of the syntax description
  (\cref{fig:LnL-bunched}).
  In the subterms, the linear variables in $\Gamma$ are only changed by binding
  new variables (all instances of which maintain linear well formedness) and by
  existing variables being shared out or coerced (which never produces
  $\gr\omega$ annotations from $\gr0$ or $\gr1$).
\end{proof}

Linear well-formedness is the only condition needed for the translation given by
\cref{thm:enc-to-lnl}.
We can translate a derivation with any such context, with no further
specification of its shape.
In particular, the shape is not calculated from an original L/nL context.

\begin{proposition}\label{thm:enc-to-lnl}
  Let $\Gamma_{\mathcal C}$ be the list of variables of intuitionistic type in
  $\Gamma$.
  Let $\Gamma_{\mathcal L}$ be the list of variables of linear type in $\Gamma$
  with usage annotation $\gr1$.
  Then, whenever $\Gamma$ is linearly well formed, we can construct the
  following translations.
  \begin{align}
    (\Gamma \vdash X) &\to \plr{\Gamma_{\mathcal C} \vdashC X} \\
    (\Gamma \vdash A) &\to
      \plr{\Gamma_{\mathcal C}; \Gamma_{\mathcal L} \vdashL A}
  \end{align}
\end{proposition}
\begin{proof}
  We proceed by mutual induction on the derivations.

  First, I consider variables.
  Suppose we have $\Gamma \sqni S$.
  If $S$ is a linear type $A$, then $\Gamma$ contains one variable of type $A$
  annotated $\gr1$, while all of the other linear variables are annotated $\gr0$
  (by linear well formedness, we have no linear variables annotated
  $\gr\omega$).
  Therefore, $\Gamma_{\mathcal L} = A$, and \TirName{$\mathcal L$-var} applies.
  Otherwise, if $S$ is an intuitionistic type $X$, then
  \TirName{$\mathcal C$-var} applies to yield $\Gamma_{\mathcal C} \vdashC X$.

  For the logical rules, linear well formedness means that all variables of
  linear type act linearly.
  Additionally, every L/nL rule requiring there to be no linear variables in
  scope is guarded by $\Boxzpt$ or $I^*$ in the syntax description, excluding
  linear variables.
  Given these behaviours, the two calculi correspond closely, and it is a matter
  of inspection (and use of \cref{thm:lwf} when using the induction hypothesis)
  to complete the proof.
\end{proof}

\subsection{Translating between L/nL and $\lambda\gr{\mathcal R}$}

\Citet[\S 3.3]{Benton94} gives syntactic translations back and forth between
Linear/non-Linear logic and the presentation of intuitionistic linear logic
given by \citet{BBdePH93}.
In this section, I give analogous translations between my encodings of L/nL and
$\name$ as instances of the generic traversal \AgdaFunction{semantics}.
More precisely, I instantiate $\name$ to the $\{\gr0 > \gr\omega < \gr1\}$
posemiring and restrict it to the fragment containing connectives $I$,
$\otimes$, $\multimap$, and $\oc\gr\omega$, matching the fragment of L/nL
presented by \citeauthor{Benton94} and in this section.
Notably, this fragment of $\name$ excludes $\oc\gr0$ and $\oc\gr1$.
I write $\oc\gr\omega$ as just $\oc$, as in traditional linear logic.

Under the above restrictions and conventions, \citeauthor{Benton94}'s
translations between the types of ILL and L/nL can be used verbatim,
and are reproduced in \cref{fig:lnl-lr-types}.
Notably, the image of each ILL type under the $\plr{-}^o$-translation falls in
the \emph{linear} types of L/nL.
In the other direction (the $\plr{-}^*$-translation), we make extensive use of
the $\oc$-type former to translate the intuitionistic types of L/nL.

\begin{figure}
  \centering
  \begin{subfigure}{.49\linewidth}
    \centering
    \begin{align*}
      &(-)^* : \mathrm{Ty}_\lnl \to \mathrm{Ty}_{\name} \\
      &\begin{aligned}
        I^* &= I \\
        \plr{A \otimes B}^* &= A^* \otimes B^* \\
        \plr{A \multimap B}^* &= A^* \multimap B^* \\
        \plr{FX}^* &= \oc X^* \\
        1^* &= I \\
        \plr{X \times Y}^* &= \oc X^* \otimes \oc Y^* \\
        \plr{X \to Y}^* &= \oc X^* \multimap Y^* \\
        \plr{GA}^* &= A^*
      \end{aligned}
    \end{align*}
    \begin{align*}
      &(-)^* : \mathrm{Ctx}_\lnl \to \mathrm{Ctx}_{\name} \\
      &\plr{\plr{\grR\gamma}^*}_i = \grR_i\gamma_i^*
    \end{align*}
  \end{subfigure}
  \begin{subfigure}{.49\linewidth}
    \centering
    \begin{align*}
      &(-)^\circ : \mathrm{Ty}_{\name} \to \mathrm{Ty}_{\lin} \\
      &\begin{aligned}
        I^\circ &= I \\
        \plr{A \otimes B}^\circ &= A^\circ \otimes B^\circ \\
        \plr{A \multimap B}^\circ &= A^\circ \multimap B^\circ \\
        \plr{\oc A}^\circ &= GFA^\circ
      \end{aligned}
    \end{align*}
    \begin{align*}
      &(-)^\circ : \oiw \times \mathrm{Ty}_{\name} \to \Sigma_f~\mathrm{Ty}_f \\
      &\begin{aligned}
        \plr{\gr0A}^\circ &= \plr{\lin, A^\circ} \\
        \plr{\gr1A}^\circ &= \plr{\lin, A^\circ} \\
        \plr{\gr\omega A}^\circ &= \plr{\intu, GA^\circ}
      \end{aligned}
    \end{align*}
    \begin{align*}
      &(-)^\circ : \mathrm{Ctx}_{\name} \to \mathrm{Ctx}_\lnl \\
      &\plr{\plr{\grR\gamma}^\circ}_i = \grR_i\plr{\grR_i\gamma_i}^\circ
    \end{align*}
  \end{subfigure}
  \caption{Translation of types between L/nL and $\name$}
  \label{fig:lnl-lr-types}
\end{figure}

I extend $\plr{-}^*$ to contexts pointwise on the type context.
Note that this differs from \citeauthor{Benton94}'s translation of contexts in
that the intuitionistic variables of $\lnl$ are translated using usage
annotation $\gr\omega$, rather than type former $\oc$.
A translation from $\lnl$ to DILL would probably similarly use DILL's
intuitionistic variables rather than $\oc$.

For $\plr{-}^\circ$, I use an extra step to avoid producing contexts
that are not linearly well formed per \cref{def:lin-well-formed}.
Specifically, wherever there is an annotation $\gr\omega$ in a $\name$ context,
the corresponding type is wrapped in a $G$ to make it intuitionistic.
For example, $\plr{\gr0A, \gr1B, \gr\omega C}^\circ =
\plr{\gr0A^\circ, \gr1B^\circ, \gr\omega GC^\circ}$.

In Agda code, I define the $\plr{-}^\circ$ operator on
$\oiw \times \mathrm{Ty}_{\name}$ (the one that introduces a $G$ for each
$\gr\omega$) via a \emph{view} \AgdaDatatype{LIView}~\citep{MM04}.
I define usage annotations $\gr0$ and $\gr1$ (\AgdaInductiveConstructor{0\#} and
\AgdaInductiveConstructor{1\#} in Agda) to be \emph{linear}, with only
$\gr\omega$ (\AgdaInductiveConstructor{$\upomega$\#}) being
\emph{intuitionistic}.
\AgdaDatatype{LIView} is a view because of the existence of
\ExecuteMetaData[\LinIntViewtex]{liview-type}, and well behaved in the sense
that any two views of the same usage annotation are equal (witnessed by
\AgdaFunction{liview-prop}, not shown here).
The translation from $\name$ to $\lnl$ takes cases between linear and non-linear
annotations at many points, so having these cases expressed as a view avoids
duplication of arguments between the cases for $\gr0$ and $\gr1$.

\ExecuteMetaData[\LinIntViewtex]{Linear}
\ExecuteMetaData[\LinIntViewtex]{LIView}

\begin{theorem}\label{thm:lnl-to-lr}
  Let $S$ be an $\lnl$ type, either linear or intuitionistic.
  Then, we can translate any L/nL term to a $\name$ term as follows.
  \begin{align}
    (\Gamma \vdash_\lnl S) &\to (\Gamma^* \vdash_{\name} S^*)
  \end{align}
\end{theorem}
\begin{proof}
  The proof is mostly straightforward, largely following Benton's translation.
  Similarly to the denotational semantics of \cref{sec:den-sem}, we may use a
  \AgdaRecord{Semantics} with $\V$ set to $\sqni_{\lnl}$, and then set
  $\C\,\Gamma\,S \coloneqq \Gamma^* \vdash_{\name} S^*$.
  Whenever we have induction hypotheses of \AgdaFunction{Kripke} type, we use
  the \AgdaFunction{reify} function for $\name$ to get $\name$ terms.
  Therefore, we are essentially just doing a proof by induction on the structure
  of the input term.

  Translating the \TirName{$F$i} rule into \TirName{$\oc$i} relies
  on the equivalence between duplicability (as expressed by the $\Box$ premise
  connective) and having been scaled by $\gr\omega$ (as expressed by the
  $\gr\omega \cdot {}$ premise connective).
  This holds of the $\oiw$ semiring, but not of general semirings (and is not
  even stateable for general semirings because of the mention of $\gr\omega$).
  The same reasoning is needed when translating the introduction rules for
  intuitionistic connectives, because they always have $\Box$ed premises and
  are translated using $\oc$.

  As an example of translating an intuitionistic elimination rule, let us
  consider the \TirName{$\times$e$_0$} rule.
  I reproduce it here with explicit contexts.
  \[
    \ebrule{%
      \hypo{\Gamma' \leq \Gamma}
      \hypo{\Gamma\text{ duplicable}}
      \hypo{\Gamma \vdash_{\lnl} X \times Y}
      \infer3[$\times$e$_0$]{\Gamma' \vdash_{\lnl} X}
    }
  \]

  Recall that $\plr{X \times Y}^* = \oc X^* \otimes \oc Y^*$.
  This means that we must pattern-match the hypothesis to get variables
  $\gr\omega X^*, \gr\omega Y^*$ so as to be able to use the $X^*$ for the
  conclusion and discard the $Y^*$.
  The formal derivation is as follows.
  We are able to copy $\Gamma^*$ between all of these subterms because its usage
  annotations are the same as those of $\Gamma$, which is duplicable by the
  assumption in \TirName{$\times$e$_0$}.
  The distinction between $\Gamma'$ and $\Gamma$ is minor.
  When we apply the \TirName{$\otimes$e} rule, we use the fact that
  $\Gamma' \leq \Gamma + \Gamma$, by the fact that $\Gamma' \leq \Gamma$ and
  $\Gamma \leq \Gamma + \Gamma$.
  From then on, we need only think about $\Gamma$, which behaves well because it
  is duplicable.

  \begin{align*}
    &\ebrule{%
      \hypo{\text{IH}}
      \infer[no rule]1{\Gamma^* \vdash_{\name} \oc X^* \otimes \oc Y^*}
      \infer0[Var]{\Gamma^*, \gr1\oc X^*, \gr0\oc Y^* \vdash_{\name} \oc X^*}
      \hypo{\nabla}
      \infer2[$\oc$e]{\Gamma^*, \gr1\oc X^*, \gr1\oc Y^* \vdash_{\name} X^*}
      \infer2[$\otimes$e]{\Gamma^* \vdash_{\name} X^*}
    }
    \\
    &\text{where }\nabla \coloneqq
    \\
    &\ebrule{%
      \infer0[Var]{\Gamma^*, \gr0\oc X^*, \gr1\oc Y^*, \gr\omega X^* \vdash_{\name} \oc Y^*}
      \infer0[Var]{\Gamma^*, \gr0\oc X^*, \gr0\oc Y^*, \gr\omega X^*, \gr\omega Y^* \vdash_{\name} X^*}
      \infer2[$\oc$e]{\Gamma'^*, \gr0\oc X^*, \gr1\oc Y^*, \gr\omega X^* \vdash_{\name} X^*}
    }
  \end{align*}

  In the Agda proof, renaming is required to perform lots of minor functions
  that we would ignore on paper.
  For example, the equation $\plr{\Gamma, \Delta}^* = \Gamma^*, \Delta^*$ ---
  required when induction hypotheses have newly bound variables --- is not true
  definitionally.
  Furthermore, because of the functional representation I use for contexts, it
  is not even provable without function extensionality.
  Therefore, renaming is the simplest way to get the required coercion.
  Such renamings perhaps could be avoided in most cases if contexts were
  represented as non-functional lists.
\end{proof}

\begin{theorem}\label{thm:lr-to-lnl}
  We can translate from $\name$ to the linear fragment of $\lnl$.
  \begin{align}
    (\Gamma \vdash_{\name} A) \to (\Gamma^\circ \vdash_\lnl A^\circ)
  \end{align}
\end{theorem}
\begin{proof}
  We use the same simple induction scheme as in \cref{thm:lnl-to-lr}, but with
  the places of $\name$ and $\lnl$ switched.
  However, some of the rule translations are more complex, mainly caused by the
  complexity of the $\plr{-}^\circ$ translation on contexts.

  For variables, we must consider separately the cases where the variable being
  used is annotated $\gr1$ and $\gr\omega$.
  The case for a variable $\gr1A$ is straightforward, and translates directly
  to an $\lnl$ variable $\gr1A^\circ$.
  A variable $\gr\omega A$, however, is translated to $\gr\omega GA^\circ$, so
  we must eliminate the $G$ in order to get the conclusion $A^\circ$.
  The \TirName{$G$e} rule requires its context to be duplicable, which is true
  by the fact that all of the unused variables are annotated either $\gr0$ or
  $\gr\omega$ (because they are all less than or equal to $\gr0$), and the
  variable being used is annotated $\gr\omega$.

  Most logical rules are handled very similarly to each other, so I will just
  translate the \TirName{$\otimes$i} rule as an example.
  I reproduce it here with explicit contexts (split into their usage and typing
  contexts).

  \[
    \ebrule{%
      \hypo{\grR \leq \grP + \grQ}
      \hypo{\grP\gamma \vdash_{\name} A}
      \hypo{\grQ\gamma \vdash_{\name} B}
      \infer3[$\otimes$i]{\grR\gamma \vdash_{\name} A \otimes B}
    }
  \]

  We cannot do a na\"{i}ve translation to the corresponding application of the
  \TirName{$\otimes$i} rule of $\lnl$ because $\plr{\grR\gamma}^\circ$,
  $\plr{\grP\gamma}^\circ$, and $\plr{\grQ\gamma}^\circ$ may all have different
  typing contexts.
  For example, consider the following instance.
  There are two problems.
  First, our induction hypotheses give us contexts
  containing $C^\circ$, where our conclusion wants a context containing
  $GC^\circ$.
  Second, $\gr1GC^\circ$ is not linearly well formed, because an intuitionistic
  type is given a linear annotation.
  It is therefore difficult to work with such a sequent.

  \[
    \ebrule{%
      \hypo{\gr1 C \vdash_{\name} A}
      \hypo{\gr1 C \vdash_{\name} B}
      \infer2[$\otimes$i]{\gr\omega C \vdash_{\name} A \otimes B}
    }
    \quad\rightsquigarrow\quad
    \ebrule{%
      \hypo{\gr1C^\circ \vdash_{\lnl} A^\circ}
      \ellipsis{?}{\gr1 GC^\circ \vdash_{\lnl} A^\circ}
      \hypo{\gr1C^\circ \vdash_{\lnl} B^\circ}
      \ellipsis{?}{\gr1 GC^\circ \vdash_{\lnl} B^\circ}
      \infer2[$\otimes$i]{\gr\omega GC^\circ \vdash_{\lnl} A^\circ \otimes B^\circ}
    }
  \]

  To fix these issues, firstly I overwrite the context-splitting given by the
  input term to conform to the bottom-up style of \cref{def:DILL-bottom-up}.
  This means precisely that wherever $\gr\omega$ appears in the context of the
  conclusion, it will also appear in the context of all the hypotheses.
  This gives the following partial derivation.

  \[
    \ebrule{%
      \hypo{\gr1C^\circ \vdash_{\lnl} A^\circ}
      \ellipsis{?}{\gr\omega GC^\circ \vdash_{\lnl} A^\circ}
      \hypo{\gr1C^\circ \vdash_{\lnl} B^\circ}
      \ellipsis{?}{\gr\omega GC^\circ \vdash_{\lnl} B^\circ}
      \infer2[$\otimes$i]{\gr\omega GC^\circ \vdash_{\lnl} A^\circ \otimes B^\circ}
    }
  \]

  Then, I fix the discrepancy in types using substitution.
  In the running example, the substitution needed for both subterms is of the
  type $\gr\omega GC^\circ \env{\vdash_{\lnl}} \gr1C^\circ$, which amounts to a
  term of type $\gr\omega GC^\circ \vdash_{\lnl} C^\circ$, as follows.
  Note that \TirName{$G$e} is only applicable thanks to changing the usage
  annotation from $\gr1$ to $\gr\omega$ in the previous step.

  \[
    \ebrule{%
      \infer0[Var]{\gr\omega GC^\circ \vdash_{\lnl} GC^\circ}
      \infer1[$G$e]{\gr\omega GC^\circ \vdash_{\lnl} C^\circ}
    }
  \]

  More generally, we may have to produce substitutions of type
  $$\gr0A^\circ, \gr1B^\circ, \gr\omega GC^\circ, \gr\omega GD^\circ
  \env{\vdash_{\lnl}}
  \gr0A^\circ, \gr1B^\circ, \gr1C^\circ, \gr\omega GD^\circ.$$
  These can be produced pointwise, from substitutions of types
  $\gr0A^\circ \env{\vdash_{\lnl}} \gr0A^\circ$ and
  $\gr1B^\circ \env{\vdash_{\lnl}} \gr1B^\circ$ and
  $\gr\omega GC^\circ \env{\vdash_{\lnl}} \gr1C^\circ$ and
  $\gr\omega GD^\circ \env{\vdash_{\lnl}} \gr\omega GD^\circ$.
  We have just seen how to produce the third of these, and the rest are
  identity substitutions.

  The two sui generis rules to translate are the rules for the $\oc$-modality,
  with the $\oc$ of intuitionistic linear logic becoming the composite $FG$ in
  linear/non-linear logic.
  The way of handling \TirName{$\oc$i} is similar to the way of handling rules
  like \TirName{$\otimes$i}, but involving multiplication/scaling by $\gr\omega$
  rather than addition.
  We have a similar difficult instance, shown below, where an $\gr\omega$ gets
  specialised to a $\gr1$ via the algebraic operation.

  \[
    \ebrule{%
      \hypo{\gr\omega \leq \gr\omega \cdot \gr1}
      \hypo{\gr1 C \vdash_{\name} A}
      \infer2[$\oc$i]{\gr\omega C \vdash_{\name} \oc A}
    }
  \]

  Again, the solution is to fix up the operation to keep the more general
  $\gr\omega$, allowing us to apply \TirName{$F$i} and \TirName{$G$i}, and
  the same substitution as before possibly including an application of
  \TirName{$G$e} as necessary.

  Finally, the translation of \TirName{$\oc$e} is simple, but worth checking.
  I reproduce the rule below with explicit contexts.

  \[
    \ebrule{%
      \hypo{\grR \leq \grP + \grQ}
      \hypo{\grP\gamma \vdash_{\name} \oc A}
      \hypo{\grQ\gamma, \gr\omega A \vdash_{\name} B}
      \infer3[$\oc$e]{\grR\gamma \vdash_{\name} B}
    }
  \]

  The main thing to note is that the translation of the context of the
  right-hand premise, $\grQ\gamma, \gr\omega A$, is
  $\plr{\grQ\gamma}^\circ, \gr\omega GA^\circ$ --- i.e., the translation of
  contexts gives us a $G$ thanks to the $\gr\omega$ usage annotation.
  Therefore, we do not have to eliminate the $G$, because the right-hand subterm
  is expected to do it for us.
  Indeed, we have seen in previous cases many uses of \TirName{$G$e} already.

  I translate the \TirName{$\oc$e} rule as follows.
  As in the \TirName{$\otimes$i} case, I pick $\grPprime$ and $\grQprime$ to fit
  bottom-up form, and use the same substitutions as in that case to mend the
  terms arriving from the induction hypotheses.
  Then, just \TirName{$F$e} suffices.

  \[
    \ebrule{%
      \infer0{\grR \leq \grPprime + \grQprime}
      \hypo{}
      \ellipsis{}{\plr{\grPprime\gamma}^\circ \vdash_{\lnl} FGA^\circ}
      \hypo{}
      \ellipsis{}{\plr{\grQprime\gamma}^\circ, \gr\omega GA^\circ \vdash_{\lnl} B^\circ}
      \infer3[$F$e]{\plr{\grR\gamma}^\circ \vdash_{\lnl} B^\circ}
    }
  \]

  %In the Agda version of the translation, manipulations are complicated by the
  %fact that the $\plr{-}^\circ$ translation for contexts does not distribute
\end{proof}

  \section{$\mu\tilde\mu$-calculus}
  \section{Graphs}

\chapter{Conclusion}



%%%%%%%%%%%%%%%%%%%%%%%%%%%%%%%%%%%%%%%%%%%%%%%%%%%%%%%%%%%%%%
\appendix
\chapter{Stuff That Didn't Fit Anywhere Else}
%%%%%%%%%%%%%%%%%%%%%%%%%%%%%%%%%%%%%%%%%%%%%%%%%%%%%%%%%%%%%%


%%%%%%%%%%%%%%%%%%%%%%%%%%%%%%%%%%%%%%%%%%%%%%%%%%%%%%%%%%%%%%
\addcontentsline{toc}{chapter}{Bibliography}
\bibliographystyle{alpha}
\bibliography{quantitative}

\end{document}
