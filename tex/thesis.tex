%%%%%%%%%%%%%%%%%%%%%%%%%%%%%%%%%%%%%%%%%%%%%%%%
%
% Strath PhD Thesis Template
%  by Jethro Browell [jethro.browell@strath.ac.uk]
%
%  Guidelines for thesis format, submission and content are found in
%  General and Course Regulations for Graduate and Postgraduate
%  Awards and Degrees, section 20.6.
%
%  Using .eps or .pdf is recomended to prduce high quality figures etc.
%
%  The Strathclyde logo can be found in other formats at www.strath.ac.uk.
%
%%%%%%%%%%%%%%%%%%%%%%%%%%%%%%%%%%%%%%%%%%%%%%%%

\documentclass[a4paper,oneside,11pt]{book}

\usepackage{amsbsy}
\usepackage{amsmath}
\usepackage{amsfonts}
\usepackage{graphicx}
\usepackage{multirow}
\usepackage{mathrsfs}
\usepackage{color}
\usepackage[hidelinks]{hyperref}
\usepackage{cite}
\usepackage{enumitem}
\usepackage{epsfig}
\usepackage{caption}
\usepackage{subcaption}
\usepackage[strict]{changepage}

\usepackage{cmll}

% Page Margins - Strath Requirement
\usepackage[left=4cm,right=2.5cm,top=2cm,bottom=4cm,includehead,includefoot,headheight=15pt]{geometry}

% Page Headers
\usepackage{fancyhdr}
\fancyhf{}
\renewcommand{\headrulewidth}{0pt} % optional
%\fancyhead[L]{\nouppercase{\leftmark} \hfill Section \nouppercase{\rightmark}}
\fancyhead[L]{\nouppercase{\leftmark}}
\cfoot{\thepage}
\pagestyle{fancy}

% Draft Watermark
%\usepackage[draft=true,allpages=true,fontfamily=cmr,angle=90,scale=0.1,mark={\fboxsep=35pt\fboxrule=0pt\relax\fbox{-- DRAFT -- \today~--}},xcoord=-80,ycoord=-20]{draftwatermark}


% Line Spacing
%\def\baselinestretch{1.5}
\usepackage{setspace}
\setstretch{1.5}


% Place UoS Logo on Title Page (this package modifies the "\maketitel" command.)
\usepackage{titling}
\pretitle{%
  \begin{flushright}
  \vspace{-9.5cm}
%  \includegraphics[width=5cm,natwidth=472,natheight=531]{logo} \\[7cm]
  \includegraphics[width=5cm]{logo} \\[6cm]
  \end{flushright}
  \begin{center}
  \LARGE
}
\posttitle{\end{center}}

\title{PhD Title \\ PhD Thesis}
\author{A. PhD Student
\\ \small Research Group\\[-0.8ex]
\small Department\\[-0.8ex]
\small University of Strathclyde, Glasgow\\
}




%%%%%%%%%%%%%%%%%%%%%%%%%%%%%%%%%%%%%%%%%%%%%%%%%%%%%%%%%%%%%%
\begin{document}

\maketitle


%%%%%%%%%%%%%%%%%%%%%%%%%%%%%%%%%%%%%%%%%%%%%%%%%%%%%%%%%%%%%%
\frontmatter
%%%%%%%%%%%%%%%%%%%%%%%%%%%%%%%%%%%%%%%%%%%%%%%%%%%%%%%%%%%%%%

% Declaration
\topskip0pt
\vspace*{\fill}
\noindent
\begin{quote}
  \centering
  This thesis is the result of the author's original research. It has been composed by the author and has not been previously submitted for examination which has led to the award of a degree. \\[5pt]
  %
  The copyright of this thesis belongs to the author under the terms of the United Kingdom Copyright Acts as qualified by University of Strathclyde Regulation 3.50. Due acknowledgement must always be made of the use of any material contained in, or derived from, this thesis. \\[5pt]
  %
  %Signed: \\
  %Date:
\end{quote}
\vspace*{\fill}



\chapter{Abstract}



\tableofcontents

\addcontentsline{toc}{chapter}{List of Figures}
\listoffigures

\addcontentsline{toc}{chapter}{List of Tables}
\listoftables



\chapter{Preface/Acknowledgements}
I would like to acknowledge...

This document is adapted from the template by Jethro Browell
(\url{https://www.overleaf.com/latex/templates/thesis-template-for-university-of-strathclyde/nfnrnmjqyxqg}),
which was licensed under CC BY 4.0.



%%%%%%%%%%%%%%%%%%%%%%%%%%%%%%%%%%%%%%%%%%%%%%%%%%%%%%%%%%%%%%
\mainmatter
%%%%%%%%%%%%%%%%%%%%%%%%%%%%%%%%%%%%%%%%%%%%%%%%%%%%%%%%%%%%%%


\chapter{Introduction}

\chapter{Simply typed $\lambda$-calculus}
  \section{Basics}
  \section{Kits}
  \section{Natural deduction vs sequent calculus}
  \section{Generic syntax}

\chapter{Linearity (the struggle)}
  \section{Intuitionistic linear logic}
  \section{DILL}
  DILL provides a syntax in which $\oc$ is characterised by a universal
  property.
  There is no need for any rule to depend on the entire context and rebind
  variables (as done by BBdPH), so is more ND-like.
  DILL makes $\oc$ behave more like a positive type former (see the deep
  identity function $\oc A \vdash \oc A$).
  DILL has been extended to dependent types: Vakar15.

  The idea of split contexts can also be applied to left-/right-rule systems.
  (Define this DILL variant.)

\chapter{Usage restriction via semirings}
  \section{Why semirings?}
  Addition gives the structural rules.
  Multiplication gives modalities.

  Compare with other work.
  About here, I can talk about Granule-style $\otimes$-pattern matching.

  \section{What are linear renaming and substitution?}
  \section{$\lambda\mathcal{R}$}

\chapter{A framework for usage-restricted calculi}
  \section{Linear natural deduction}
  This should stay fairly general, so it could be compiled to a DILL-style
  calculus as well as the one with usage annotations.
  Our framework compiles a ND system down to a sequent calculus.
  It isolates the building blocks of typing rules ($\vdash$, $\wedge$, $*$,
  $\Box$, $\cdot$) so that we can do more generic programming (e.g., usage
  checker).
  \section{Gory details}

\chapter{Investigations that are now easier}
  \section{$\mu\tilde\mu$-calculus}
  \section{Graphs}

\chapter{Conclusion}



%%%%%%%%%%%%%%%%%%%%%%%%%%%%%%%%%%%%%%%%%%%%%%%%%%%%%%%%%%%%%%
\appendix
\chapter{Stuff That Didn't Fit Anywhere Else}
%%%%%%%%%%%%%%%%%%%%%%%%%%%%%%%%%%%%%%%%%%%%%%%%%%%%%%%%%%%%%%


%%%%%%%%%%%%%%%%%%%%%%%%%%%%%%%%%%%%%%%%%%%%%%%%%%%%%%%%%%%%%%
\addcontentsline{toc}{chapter}{Bibliography}
\bibliographystyle{alpha}
\bibliography{quantitative}

\end{document}
