There is a vast literature on formalisations of syntaxes with binding, which I
cannot possibly do justice to in a reasonably sized thesis chapter.
Instead, I limit myself to comparisons of the \citet{AACMM21} method I follow in
this thesis to just its closest related work.

\subsection{Autosubst}

\citet{Autosubst15} present the system \emph{Autosubst}, which provides various
tools for working with syntaxes with binding in the Coq proof assistant.
Autosubst is based on similar ideas to those \citeauthor{AACMM21} use:
de Bruijn-indexed terms with a distinguished variable rule and notion of
binding, acted upon by simultaneous renaming and substitution.

The simplest differences are essentially matters of choosing the encoding that
best fits the proof assistant being used.
Coq users tend to prefer using unindexed types and propositions indexed over
them --- in this case, a type of unscoped and untyped terms plus a
``well typed'' predicate --- whereas Agda users prefer to work with only well
formed data (well scoped and well typed terms).
The latter approach more readily allows us to show generically that substitution
preserves scoping and typing, but the former approach, conversely, allows for
bespoke proofs of such facts.
For example, one theorem of \citet{Autosubst15} is type preservation for
$\mathrm{CC}_\omega$, a dependent type system we cannot express using the
machinery of \citet{AACMM21}.
In principle, one could use \citeauthor{AACMM21}'s machinery as the basis of a
similar bespoke proof, but as far as I am aware, this has not been tried.

Another main difference is that Autosubst is presented to the user largely as
a black-box implementation of substitution and related lemmas, in contrast to
\citeauthor{AACMM21}'s work exposing the \AgdaRecord{Semantics} bundle to the
user, and having substitution be just one instance.
\citet{ACMM17} and \citet{AACMM21} provide many examples of traversals over
syntax using the same generic environment management as used by substitution.
However, the focus on substitution in Autosubst has meant that reasoning about
substitutions has been given more developed support.
For example, the library provides a tactic \texttt{autosubst} which automates
many equational proofs involving substitutions based on the $\sigma$-calculus of
\citet{ACCL91}.

An interesting feature of Autosubst is \emph{heterogeneous substitution}.
The motivation for heterogeneous substitution is to handle systems like system
F, where types and terms are syntactically distinct, but both feature binding
and require a substitution operation.
Furthermore, binding and substitution of types also affects the syntax of terms,
thanks to $\Lambda$ terms.
\citeauthor{AACMM21} provide no direct equivalent to heterogeneous substitution,
and it is unclear how well their work can handle polymorphic calculi.

\citet{Autosubst18} propose some modifications to Autosubst which, as far as I
can tell, have not yet been incorporated, but are presented in mechanised form
for the paper.
Some of these modifications aim to bring Autosubst into line with
\citeauthor{AACMM21}'s work, in particular taking semantic traversals as a
theoretical basis.
TODO: more.

\subsection{Fiore, Plotkin, Turi}

TODO: cite \citet{FPT99} and \citet{FS22}.
