\chapter{Linearity and modality}\label{sec:linearity}

The techniques of \cref{sec:simple} were developed to handle idealised core
calculi, like the simply typed $\lambda$-calculus and its many variants.
Such core calculi are fairly close to some real intermediate languages,
\todo{cite} though many intermediate languages also have polymorphism in their
type systems, which I have not discussed, but has been considered by much prior
work~\citep{POPLmark}.

Another potential feature --- required for many desired semantics but only
relatively recently appearing in programming languages --- that is not handled
by the techniques of \cref{sec:simple} is restrictions on how variables can be
used.
In the simple type systems covered in \cref{sec:simple}, the only restriction on
the use of a variable is that it is used as an expression of the appropriate
type.
However, we could also consider having somehow ``guarded'' subexpressions in
which some variables bound outside cannot occur, or systems in which the use of
a variable in one part of an expression precludes its use in another part of
that expression (the variable having been ``used up'').
I will introduce and motivate some such systems in this chapter, and argue that
they are impossible to capture directly in the framework presented in
\cref{sec:gen-syn}.

Languages and systems implementing features inspired by linear logic include
Rust~\citep{MK14,Rust}, ATS~\citep{Xi04,ZX05}, and various implementations of
session types~\citep{HLVCCDMPRTVTZ16}.
Additionally, forms of linearity have been used in theoretical work, for example
to bound computational complexity~\citep{GSS92,Hofmann03} or to write programs
which support incremental updates to computations~\citep{ER03,Ehrhard18}.
As for modality, \citet{IMO20} show that common data-flow analyses for
imperative languages can be recast as modal type systems, though they have
generally been developed in an ad hoc fashion.
Such a recasting may be useful to understand the metatheoretic properties of
programs passing such static analyses.
More user-facing implementations of modal type systems include the system for
stack allocation in OCaml~\citep{DW22} and the approach to programmer-annotated
erasure found in Agda and Idris~\citep{Atkey18}.

In this chapter, I look at two standard type-theoretic features that motivate
considering calculi going beyond simple type theories as deliniated in
\cref{sec:simple} or equivalently by \citet{AACMM21}.
The first feature is a $\Box$-modality --- specifically, the $\Box$-modality of
intuitionistic S4, which I discuss in \cref{sec:modal}.
The second feature is linearity, which I discuss in \cref{sec:ill}.
In particular, I discuss possible syntaxes for each, with a mind to
being able to apply the techniques described in \cref{sec:simple}.
Then, I\ldots

\section{Intuitionistic S4 modal logic}\label{sec:modal}
Modal logics, and in particular their modalities, are usually presented in
philosophical and mathematical logic in an axiomatic style.
For example, the common modal logic S4 may be presented by taking a presentation
of classical logic, adding a unary $\Box$ (``box'') operator to the formulas,
keeping the logical rules as-are, and adding all of the axioms and rules
listed in \cref{fig:S4-axioms}.
Note that, unlike elsewhere in this thesis, the empty context in the
necessitation rule \TirName{N} is a proper restriction:
While the axioms hold in any context, thanks to weakening, and ordinary logical
rules hold in any context, the necessitation rule only holds in the empty
context.
We may apply weakening to the conclusion of \TirName{N} to embed such a
derivation into a larger derivation, but we may not use hypotheses from the
outside in such a subderivation.

\begin{figure}
  \begin{mathpar}
    \ebrule{%
      \hypo{\vdash A}
      \infer1[N]{\vdash \Box A}
    }
    \and
    \ebrule{%
      \infer0[K]{\vdash \Box\plr{A \to B} \to \plr{\Box A \to \Box B}}
    }
    \and
    \ebrule{%
      \infer0[T]{\vdash \Box A \to A}
    }
    \and
    \ebrule{%
      \infer0[4]{\vdash \Box A \to \Box\plr{\Box A}}
    }
  \end{mathpar}
  \caption{The axioms and rules required in a traditional presentation of S4}
  \label{fig:S4-axioms}
\end{figure}

In this thesis, I am working with intuitionistic logics, so I start with a base
of intuitionistic logic, and study intuitionistic S4 (IS4).
The axiomatic presentation can be introduced on top of various presentations of
intuitionistic/classical logic --- including natural deduction, sequent
calculi, Hilbert systems, and abstractly taking the set of
intuitionistic/classical tautologies.
For concreteness and rigour, I will assume a natural deduction system NJ
allowing for explicit weakening.

The axiomatic presentation is convenient for talking about semantics, with
different choices of axioms corresponding fairly directly to natural
restrictions on models.
However, the syntax is quite poorly behaved, at least for programming language
applications.
As mentioned earlier, the necessitation rule makes requirements of its context,
which makes it incompatible with the methods discussed in \cref{sec:simple}.
Essentially, we are required to make necessitation a special case in renaming,
substitution, and all the other traversals.
Additionally, axioms require us to modify the notion of canonical
form~\citep[p.\ 79]{Prawitz65}, and redo
any proofs that rely on all closed terms of function type being
$\lambda$-abstractions.

The methodology I use to produce a nicer syntax is that of \citet{judgmental}:
capturing the desired modality in the judgemental structure of the syntax.
To start, note the following two facts about a ``promotion'' rule similar to one
mentioned by \citet{BBdePH93}.

\begin{proposition}\label{thm:NKT4-P}
  When $\Gamma = B_1, \ldots, B_n$, let $\Box\Gamma$ abbreviate
  $\Box B_1, \ldots, \Box B_n$.
  Then, in NJ + \TirName{N} + \TirName{K} + \TirName{4}, the following principle
  is admissible.
  \[
    \ebrule{%
      \hypo{\Box\Gamma \vdash A}
      \infer1[Promotion]{\Box\Gamma \vdash \Box A}
    }
  \]
\end{proposition}
\begin{proof}
  By induction on $n$.
  When $n = 0$, \TirName{Promotion} becomes just the necessitation rule \TirName{N}.
  For $n = \lvert\Gamma\rvert + 1$, we have the following derivation, where
  \TirName{4$'$} and \TirName{K$'$} abbreviate the obvious combinations of
  \TirName{4} and \TirName{K}, respectively, with \TirName{$\to$-E}.
  I silently apply weakening throughout the derivation for the sake of concision.
  \[
    \begin{prooftree}
      \hypo{\Box\Gamma, \Box B \vdash A}
      \infer1[$\to$-I]{\Box\Gamma \vdash \Box B \to A}
      \infer1[IH]{\Box\Gamma \vdash \Box\plr{\Box B \to A}}
      \infer0[Var]{\Box B \vdash \Box B}
      \infer1[4$'$]{\Box B \vdash \Box\Box B}
      \infer2[K$'$]{\Box\Gamma, \Box B \vdash \Box A}
    \end{prooftree}
  \]
\end{proof}

\begin{proposition}\label{thm:PT-NKT4}
  In NJ + \TirName{Promotion} + \TirName{T}, we can derive the original axioms
  and rule of S4: \TirName{N}, \TirName{K}, \TirName{T}, and \TirName{4}
\end{proposition}
\begin{proof}
  The rule \TirName{N} and axiom \TirName{T} are immediate.
  For axioms \TirName{K} and \TirName{4}, we have the following derivations,
  where \TirName{T$'$} abbreviates the obvious combination of \TirName{T} and
  \TirName{$\to$-E}.
  \begin{mathpar}
    \ebrule{%
      \infer0[T$'$]{\Box\plr{A \to B} \vdash A \to B}
      \infer0[T$'$]{\Box A \vdash A}
      \infer2[$\to$-E]{\Box\plr{A \to B}, \Box A \vdash B}
      \infer1[Promotion]{\Box\plr{A \to B}, \Box A \vdash \Box B}
      \infer1[$\to$-I]{\Box\plr{A \to B} \vdash \Box A \to \Box B}
      \infer1[$\to$-I]{\vdash \Box\plr{A \to B} \to \Box A \to \Box B}
    }
    \and
    \ebrule{%
      \infer0[Var]{\Box A \vdash \Box A}
      \infer1[Promotion]{\Box A \vdash \Box\Box A}
      \infer1[$\to$-I]{\vdash \Box A \to \Box\Box A}
    }
  \end{mathpar}
\end{proof}

\Cref{thm:NKT4-P,thm:PT-NKT4} suggest that a \TirName{Promotion}-like rule to
introduce the $\Box$ operator and a \TirName{T}-like rule to eliminate the
$\Box$ operator may be a possible approach to capture S4 in a natural deduction
style.
However, we do not want to add such a rule to our system because it is not
stable under substitution.
For example, let
$\sigma : \Box A \wedge \Box B \env\vdash \Box A, \Box B$ be the substitution
whose two components are given by left projection and right projection,
respectively.
The substitution principle says that we should be able to turn any derivation of
$\Box A, \Box B \vdash \Box C$ into a derivation of
$\Box A \wedge \Box B \vdash \Box C$.
However, if we concluded $\Box A, \Box B \vdash \Box C$ by \TirName{Promotion},
then it becomes unclear how to conclude $\Box A \wedge \Box B \vdash \Box C$.
In particular, \TirName{Promotion} does not apply because the formula
$\Box A \wedge \Box B$ is headed by $\wedge$, rather than $\Box$.

Instead of constraining the types of all of the items in the context, the
solution given by \citet{judgmental} is to diversify the shapes of contexts, and
constrain those shapes when giving our necessitation rule.
In this case, we are interested in hypotheses being $\Box$-like, so we introduce
a judgement form $A~\mathit{valid}$, contrasting with the judgement we had
previously written just $A$, but will now write $A~\mathit{true}$.
We construct the system so that $A~\mathit{valid}$ is equivalent to
$\Box A~\mathit{true}$.
%Note that, in S4, $\Box\Box A$ is logically equivalent to $\Box A$
Because the choice of $\mathit{valid}$ or $\mathit{true}$ is part of the
structure of the context, rather than being part of the type, we are able to
define substitution between contexts in a way that treats valid and true entries
differently --- restricting substitution so that it is compatible with an
adapted version of \TirName{Promotion}.

Let us take the natural deduction presentation of the simply typed
$\lambda$-calculus given in \cref{sec:terms}.
Where that system has sequents of the form $A_1, \ldots, A_n \vdash B$, we
instead now write
$A_1~\mathit{true}, \ldots, A_n~\mathit{true} \vdash B~\mathit{true}$.
We may also have hypothetical occurrences of $A~\mathit{valid}$, but all logical
rules will target judgements of the form $A~\mathit{true}$.
Hypotheses of the form $A~\mathit{valid}$ arise from the elimination rule for
the $\Box$ operator, \TirName{$\Box$-E}.
The only special thing about valid hypotheses is that, once they are bound, they
are preserved by the \TirName{$\Box$-I} rule, which is our recasting of
\TirName{Promotion} as the introduction rule for $\Box$.
Otherwise, like true hypotheses, an $A~\mathit{valid}$ hypothesis can be used to
obtain $A~\mathit{true}$, via the \TirName{vVar} rule, as justified by axiom
\TirName{4}.

The three new rules are listed in \cref{fig:PD}.
Let $\Gamma_v$ be the context $\Gamma$ but with all true hypotheses removed,
leaving only valid hypotheses.
Because I am working with a De Bruijn index-style presentation, weakening needs
to be admissible.
Therefore, the \TirName{$\Box$-I} rule needs to remove its true hypotheses,
rather than only applying in a context in which there are no true hypotheses.

\begin{figure}
  \begin{mathpar}
    \ebrule{%
      \hypo{\Gamma_v \vdash A~\mathit{true}}
      \infer1[$\Box$-I]{\Gamma \vdash \Box A~\mathit{true}}
    }
    \and
    \ebrule{%
      \hypo{\Gamma \vdash \Box A~\mathit{true}}
      \hypo{\Gamma, A~\mathit{valid} \vdash B~\mathit{true}}
      \infer2[$\Box$-E]{\Gamma \vdash B~\mathit{true}}
    }
    \and
    \ebrule{%
      \hypo{\Gamma \ni \plr{A~\mathit{valid}}}
      \infer1[vVar]{\Gamma \vdash A~\mathit{true}}
    }
  \end{mathpar}
  \caption{The new rules of the \citeauthor{judgmental} presentation of IS4.}
  \label{fig:PD}
\end{figure}

Whereas \citet{judgmental} prove single substitution for both true and valid
hypotheses (as well as \emph{possible} hypotheses, and for possible judgements,
in their system dealing with a possibility modality $\Diamond$) separately, I
build on the simultaneous substitution procedure given in
\cref{sec:syntactic-kits} to give a unified simultaneous substitution procedure.
Recall that the basic definition was that of \emph{$\vDash$-environment},
defined in the intuitionistic case by
$\Gamma \env\vDash \Delta \coloneqq \Pi A.~\Delta \ni A \to \Gamma \vDash A$.
Essentially the same formula deals with true hypotheses, but for valid
hypotheses, we need to think of something new.
We want to be able to instantiate $\vDash$ to $\vdash$ so as to derive
substitution, but $\Gamma \vdash A~\mathit{valid}$ is not a valid sequent in our
system.
Instead, I consider $\Gamma \vdash \Box A~\mathit{true}$, and in particular the
canonical way of deriving such a sequent: the \TirName{$\Box$-I} rule.
\TirName{$\Box$-I} says that we can get
$\Gamma \vdash \Box A~\mathit{true}$ from $\Gamma_v \vdash A~\mathit{true}$.
I take the form of latter sequent to replace $\Gamma \vDash A~\mathit{valid}$,
giving the following definition.

\begin{definition}\label{def:PD-env}
  A \emph{(modal) $\vDash$-environment} from $\Gamma$ to $\Delta$ is given as
  follows.
  \[
    \Gamma \env\vDash \Delta \coloneqq
    \Pi A.~\plr{\Delta \ni A~\mathit{true} \to \Gamma \vDash A}
    \times \plr{\Delta \ni A~\mathit{valid} \to \Gamma_v \vDash A}
  \]
\end{definition}

With this definition, I reproduce the kit-based machinery from
\cref{sec:syntactic-kits}.
However, I make crucial use of the \emph{stability under renaming} property ---
rather than \emph{stability under context extension} --- to deal with not just
context extensions, but also discarding of true hypotheses by the
\TirName{$\Box$-I} rule.
To state this property, I firstly need to be precise about what a renaming is.

\begin{definition}\label{def:PD-var}
  The set of variables in context $\Gamma$ of type $A$, notated $\Gamma \ni A$,
  is the disjoint union of the sets $\Gamma \ni A~\mathit{true}$ and
  $\Gamma \ni A~\mathit{valid}$ --- respectively, the true variables and the
  valid variables in $\Gamma$ of type $A$.

  The set of \emph{renamings} from $\Gamma$ to $\Delta$ is
  $\Gamma \env\ni \Delta$.
\end{definition}

\Cref{def:PD-var} allows us to state a unified variable rule: For each
$x : \Gamma \ni A$, we have a corresponding term
$x : \Gamma \vdash A~\mathit{true}$.
However, with the new definition of both environments and variables, we must
check that they interact in the desired way.

\begin{lemma}[lookup]\label{thm:PD-lookup}
  If $\vDash$ respects renaming (i.e., we have a function
  $\mathrm{ren}^\vDash :
  \Gamma \env\ni \Delta \to \Delta \vDash A \to \Gamma \vDash A$
  for each $\Gamma$, $\Delta$, and $A$), then
  from an environment $\rho : \Gamma \env\vDash \Delta$ and a variable
  $x : \Delta \ni A$, we get a value $\rho(x) : \Gamma \vDash A$.
\end{lemma}
\begin{proof}
  I proceed by cases on whether $x$ is a true or valid variable.
  In the true case, the result is straightforward.
  In the valid case, \cref{def:PD-env} gives us a value in $\Gamma_v \vDash A$.
  We need a renaming of type $\Gamma \env\vDash \Gamma_v$ to get the desired
  value in $\Gamma \vDash A$.
  The renaming is the one that, conceptually, discards the true hypotheses.
\end{proof}

We then need to prove that environments are preserved by everything we do to
contexts in a derivation.
One thing we do, like in the simply typed case, is to bind new variables, and
this is handled by \cref{thm:PD-bindEnv}.
The other thing, which is new for modal logic, is to discard the true
hypotheses, as in the \TirName{$\Box$-I} rule, which is handled by
\cref{thm:PD-vEnv}.

\begin{lemma}[bindEnv]\label{thm:PD-bindEnv}
  If we have $\mathrm{vr} : \Gamma \ni A \to \Gamma \vDash A$ for all
  $\Gamma$ and $A$, and
  $\mathrm{ren}^\vDash :
  \Gamma \env\ni \Delta \to \Delta \vDash A \to \Gamma \vDash A$
  for all $\Gamma$, $\Delta$, and $A$,
  then from an environment $\rho : \Gamma \env\vDash \Delta$
  we can get an environment of type $\Gamma, \Theta \env\vDash \Delta, \Theta$
  for all $\Gamma$, $\Delta$, and $\Theta$.
\end{lemma}
\begin{proof}
  I consider separately the part of the environment we are constructing that
  deals with true hypotheses and the part that deals with valid hypotheses.
  These cases correspond to the two factors in the expression in
  \cref{def:PD-env}.

  The case for true hypotheses is essentially the same as in
  \cref{sec:syntactic-kits}.
  We take cases on whether the $x : \Delta, \Theta \ni A~\mathit{true}$ is in
  $\Delta$ or $\Theta$.
  In the $\Delta$ case, we apply $\rho$ and then rename via
  $\mathrm{ren}^\vDash$ using the obvious renaming of type
  $\Gamma, \Theta \env\ni \Gamma$.
  In the $\Theta$ case, we embed our $z : \Theta \ni A~\mathit{true}$ as
  $\searrow z : \Gamma, \Theta \ni A~\mathit{true}$, and use $\mathrm{vr}$ to
  get the required value.

  The case for valid hypotheses is similar.
  We take the same cases, with the $\Delta$ case differing only in which part of
  $\rho$ we need to use (the valid part, rather than the true part).
  In the $\Theta$ case, we have $z : \Theta \ni A~\mathit{valid}$.
  The $_v$ operation keeps all valid hypotheses, so we have
  $z_v : \Theta_v \ni A~\mathit{valid}$.
  Using the fact that $(\Gamma, \Theta)_v = \Gamma_v, \Theta_v$, we have
  $\searrow z_v : (\Gamma, \Theta)_v \ni A~\mathit{valid}$, and we use
  $\mathrm{vr}$ to get the required value.
\end{proof}

\begin{lemma}\label{thm:PD-vEnv}
  From an environment $\rho : \Gamma \env\vDash \Delta$, we can get an
  environment $\rho_v : \Gamma_v \env\vDash \Delta_v$.
\end{lemma}
\begin{proof}
  Suppose $x : \Delta_v \ni A$.
  Because $\Delta_v$ contains only and all of the valid hypotheses from
  $\Delta$, we actually have $x' : \Delta \ni A~\mathit{valid}$.
  Applying the second part of $\rho$ on $x'$ gives us a value in
  $\Gamma_v \vDash A$.
  We actually wanted a value in $\plr{\Gamma_v}_v \vDash A$, but $_v$ is
  idempotent, so we already have this.
\end{proof}

Then, we use all of the previous features to prove the key theorem.

\begin{theorem}[trav]\label{thm:PD-trav}
  For any notion of value $\vDash$ which is stable under renaming, admits
  variables (via some $\mathrm{vr}$ as in \cref{thm:PD-bindEnv}), and
  embeds into terms (via some
  $\mathrm{tm} : \Gamma \vDash A \to \Gamma \vdash A~\mathit{true}$ for all
  $\Gamma$ and $A$),
  an environment $\rho : \Gamma \env\vDash \Delta$ and a term
  $M : \Delta \vdash A~\mathit{true}$ yield a term
  $M[\rho] : \Gamma \vdash A~\mathit{true}$.
\end{theorem}
\begin{proof}
  I proceed by induction on the term $M$.
  Here I consider only variables and the two rules governing the $\Box$
  connective.
  It is easy to adapt this procedure to any of the types from simply typed
  $\lambda$-calculus.

  % var
  I consider variables being given by the unified variable rule proposed in the
  paragraph before \cref{thm:PD-lookup}.
  Given this, from a variable $x : \Delta \ni A$, \cref{thm:PD-lookup} gives us
  a value $\rho(x) : \Gamma \vDash A$, which $\mathrm{tm}$ turns into a term of
  the desired form.

  % \Box-I
  If the term was made from the \TirName{$\Box$-I} rule, we use that rule to
  produce the resulting term, and need to get from a term
  $M : \Delta_v \vdash A~\mathit{true}$ to a term in
  $\Gamma_v \vdash A~\mathit{true}$.
  It is enough to use the induction hypothesis, updating the environment using
  \cref{thm:PD-vEnv}.

  % \Box-E
  If the term was made from the \TirName{$\Box$-E} rule, we again use the same
  rule to construct the output, but have to get from a term
  $M : \Delta \vdash \Box A~\mathit{true}$ to a term in
  $\Gamma \vdash \Box A~\mathit{true}$, and from a term
  $N : \Delta, A~\mathit{valid} \vdash B~\mathit{true}$ to a term in
  $\Gamma, A~\mathit{valid} \vdash B~\mathit{true}$.
  The former follows from a straightforward use of the induction hypothesis,
  while the latter needs \cref{thm:PD-bindEnv} before applying the induction
  hypothesis.
\end{proof}

To get our desired corollaries of simultaneous renaming and substitution, we
first have to show that variables respect renaming.
In the pure intuitionistic case of \cref{sec:simple}, this was trivial because
renamings were precisely functions between sets of variables.
In the modal case, we have to be careful about the distinction between true and
valid hypotheses, and how the context is restricted in the valid case.

\begin{lemma}[variables respect renaming]\label{thm:PD-ren-var}
  Given a renaming $\rho : \Gamma \env\ni \Delta$ and a variable
  $x : \Delta \ni A$, we get a variable in $\Gamma \ni A$.
\end{lemma}
\begin{proof}
  Note that we cannot use \cref{thm:PD-lookup} with $\vDash$ instantiated to
  $\ni$ because it would circularly require that variables respect renaming.
  However, the procedure in this case is similar to that of
  \cref{thm:PD-lookup}.

  We take cases on whether $x$ is true or valid, and the result in the true case
  comes immediately by applying the true part of $\rho$.
  For $x : \Delta \ni A~\mathit{valid}$, $\rho$ gives us a variable in
  $\Gamma_v \ni A$.
  This variable must be valid ($\Gamma_v \ni A~\mathit{valid}$), and from there
  it is easy to get a variable in the larger context $\Gamma$.
\end{proof}

\begin{corollary}[renaming]\label{thm:PD-ren}
  Given a renaming $\rho : \Gamma \env\ni \Delta$ and a term
  $M : \Delta \vdash A~\mathit{true}$, we get a term in
  $\Gamma \vdash A~\mathit{true}$.
\end{corollary}
\begin{proof}
  We use \cref{thm:PD-trav}, with $\vDash$ being $\ni$, $\mathrm{vr}$ being the
  identity function, $\mathrm{tm}$ being the unified variable rule, and renaming
  of $\ni$ being given by \cref{thm:PD-ren-var}.
\end{proof}

\begin{corollary}[substitution]\label{thm:PD-sub}
  Given a substitution $\rho : \Gamma \env\vdash \Delta$ and a term
  $M : \Delta \vdash A~\mathit{true}$, we get a term in
  $\Gamma \vdash A~\mathit{true}$.
\end{corollary}
\begin{proof}
  We use \cref{thm:PD-trav}, with $\vDash$ being $- \vdash -~\mathit{true}$,
  $\mathrm{vr}$ being the unified variable rule, $\mathrm{tm}$ being the
  identity function, and renaming being given by \cref{thm:PD-ren}.
\end{proof}

With these basic syntactic lemmas proved in a manner largely following the
proofs of \cref{sec:simple}, it is plausible that this approach could be
extended to handle generic semantics and generic syntax, following
\cref{sec:gen-sem,sec:gen-syn}, respectively.
However, I hold off from developing this until phrasing it more generally in
\cref{sec:framework}.


\section{Intuitionistic Linear Logic}\label{sec:ill}
Linear logic is a good starting point for our investigations into usage
restrictions because it is well known, well understood, and exhibits most of the
difficulties we will come across.
Linear logic's lack of general weakening and contraction will be a typical
feature of the calculi we will see, and the exponential modality $\oc$ (bang)
will be the prototype for a range of modalities I will consider.

\subsection{Motivation of linearity}

While classical linear logic is perhaps better understood, for the sake of this
thesis I focus on intuitionistic linear logic.
The syntactic difference between the two is analogous to the difference between
classical and intuitionistic logic: The latter restricts the former by only
allowing one formula on the right of a sequent.
Classical linear logic is not irrelevant to this thesis, and appears as an
instance of the framework presented in \cref{sec:framework}.
However, the connection of this work to intuitionistic linear logic is more
direct, and reflects the basis of existing semiring-annotated calculi on
(intuitionistic) typed $\lambda$-calculi.

Intuitionistic linear logic is often motivated by its sensitivity to resources.
The fact that, in intuitionistic (and classical) logic, \emph{any} hypothesis
can be discarded or duplicated means that propositions cannot be used to model
things that may be ``possessed'' or ``spent'', despite hypotheses often being
described as things we ``have''.
In contrast, linear logic allows propositions to behave like physical objects,
being moved around but not being created or destroyed (a priori).

Programming applications of linearity include stateful protocols
\citep{Wadler12}, mutually exclusive capabilities (TODO: cite),
and mutation (TODO: cite).
In each of these cases, the act of using up a hypothesis allows us to divide
time into ``before'' and ``after'', with lack of duplication avoiding confusion
over which half we are in, and lack of deletion allowing us to know whether the
corresponding action actually happened.
For example, when modelling mutation, a variable refers to a single state of a
mutable value.
The \texttt{write} operation uses up such a variable, making the old state
inaccessible, and returns a new linear value to represent the new state of the
mutable value.
Such a protocol naturally also supports a \emph{freezing} operation, by which we
relinquish the ability to mutate the value in return for an immutable reference
to the value produced.

%In LJ, two distinct, but equiderivable, canonical definitions of $\wedge$ can
%be given.
%
%In the first approach, $\wedge^-$ is conceived to be a
%\emph{categorial product}.
%Such a product is constructed by providing two maps in from the same antecedents
%$\Gamma$.
%It is used by projecting out one of the sides.
%
%\begin{mathpar}
%  \inferrule*[right=$\wedge^-$-r]
%  {\Gamma \vdash A \\ \Gamma \vdash B}
%  {\Gamma \vdash A \wedge^- B}
%
%  \and
%
%  \inferrule*[right=$\wedge^-$-l$_0$]
%  {\Gamma, A \vdash C}
%  {\Gamma, A \wedge^- B \vdash C}
%
%  \and
%
%  \inferrule*[right=$\wedge^-$-l$_1$]
%  {\Gamma, B \vdash C}
%  {\Gamma, A \wedge^- B \vdash C}
%\end{mathpar}
%
%In the second approach, $\wedge^+$ is conceived to be a \emph{tensor product}.
%Intuitively, $\wedge^+$ internalises the comma of context concatenation.
%Such a product is used by giving access to both sides simultaneously.
%It is constructed by constructing both sides and combining the antecedents
%required by each side.
%
%\begin{mathpar}
%  \inferrule*[right=$\wedge^+$-r]
%  {\Gamma \vdash A \\ \Delta \vdash B}
%  {\Gamma, \Delta \vdash A \wedge^+ B}
%
%  \and
%
%  \inferrule*[right=$\wedge^+$-l]
%  {\Gamma, A, B \vdash C}
%  {\Gamma, A \wedge^+ B \vdash C}
%\end{mathpar}
%
%When we prove the logical equivalence of these two formulations, we notice that
%the structural rules of weakening and contraction are essential.
%When we remove weakening and contraction, $\wedge^-$ and $\wedge^+$ become
%logically distinct connectives, which we notate $\&$ and $\otimes$,
%respectively.
%
%\begin{mathpar}
%  \inferrule*[right=$\wedge^-$-r]
%  {%
%    \inferrule*[right=Weak$^*$,fraction={\cdot\cdots\cdot}]
%    {\Gamma \vdash A}
%    {\Gamma, \Delta \vdash A}
%    \\
%    \inferrule*[Right=Weak$^*$,fraction={\cdot\cdots\cdot}]
%    {\Delta \vdash B}
%    {\Gamma, \Delta \vdash B}
%  }
%  {\Gamma, \Delta \vdash A \wedge^- B}
%
%  \and
%
%  \inferrule*[right=Contr]
%  {%
%    \inferrule*[Right=$\wedge^-$-l$_0$]
%    {%
%      \inferrule*[Right=$\wedge^-$-l$_1$]
%      {\Gamma, A, B \vdash C}
%      {\Gamma, A, A \wedge^- B \vdash C}
%    }
%    {\Gamma, A \wedge^- B, A \wedge^- B \vdash C}
%  }
%  {\Gamma, A \wedge^- B \vdash C}
%\end{mathpar}
%
%\begin{mathpar}
%  \inferrule*[right=Contr$^*$,fraction={\cdot\cdots\cdot}]
%  {%
%    \mprset{defaultfraction}
%    \inferrule*[Right=$\wedge^+$-r]
%    {\Gamma \vdash A \\ \Gamma \vdash B}
%    {\Gamma, \Gamma \vdash A \wedge^+ B}
%  }
%  {\Gamma \vdash A \wedge^+ B}
%
%  \and
%
%  \inferrule*[right=$\wedge^+$-l]
%  {%
%    \inferrule*[Right=Weak]
%    {\Gamma, A \vdash C}
%    {\Gamma, A, B \vdash C}
%  }
%  {\Gamma, A \wedge^+ B \vdash C}
%
%  \and
%
%  \inferrule*[right=$\wedge^+$-l]
%  {%
%    \inferrule*[Right=Weak]
%    {\Gamma, B \vdash C}
%    {\Gamma, A, B \vdash C}
%  }
%  {\Gamma, A \wedge^+ B \vdash C}
%\end{mathpar}

\subsection{The multiplicative-additive fragment}

The multiplicative-additive fragment of linear logic (MALL) is the fragment
where all hypotheses are linear (must be used exactly once).
We will extend MALL with the \emph{exponential} modality in
\cref{sec:bang-modality}.
MALL is unable to embed intuitionistic or classical logic (collectively
known as \emph{structural logics}), as MALL is unable to reflect any of the
discarding or duplication that can be done in proofs in structural logics.

In short, the syntax of intuitionistic MALL can be described as intuitionistic
logic with the structural rules of \emph{weakening} and \emph{contraction}
removed.
However, without the presence of weakening and contraction, we have to be more
careful about the rules we state, so as not to accidentally introduce weakening
and contraction admissibly.
The lack of these structural rules also allows us to observe a new phenomenon:
the distinction (at the level of provability) between \emph{additive} and
\emph{multiplicative} formulations of existing connectives (in particular, in
the intuitionistic case, the conjunction connective).

I present MALL in \cref{fig:mall} in a sequent calculus style, as it was
presented by \citet{girard87linear}.

To encode what it means to use a hypothesis \emph{exactly once}, we first need
to decide what counts as a use.
The simplest case is that the identity sequent counts as a single use of its
sole hypothesis, and conversely does not count as a use of any other hypotheses.
For sequential proofs, created by the \TirName{Cut} rule, if we have a proof of
$A$ using $\Gamma$, and a proof of $B$ using $\Delta$ and $A$, then we have a
proof of $B$ transitively using $\Delta$ and $\Gamma$.
The exchange rule Exch says that use is invariant under permutation.

For the logical connectives, we have genuine choices as to what it means to use
them.
Two cases --- disjunction ($\oplus$) and (linear) implication ($\multimap$) ---
are somewhat intuitive from intuitionistic logic.
A canonical proof of a disjunction is a tag and a proof of one of the two
disjuncts.
This suggests that a proof of a disjunction only uses the same hypotheses as
the proof of the disjunct we actually choose, with the other disjunct being
irrelevant.
Correspondingly, when we use a disjunction hypothesis, we will only actually use
one of the cases, so each branch should use the same hypotheses.
For implication, use is sequential like with the \TirName{Cut} rule, and its
left rule is more or less the only choice that allows use of the surrounding
hypotheses.

For conjunction, there are two choices: Either the conjuncts \emph{together} use
all of the hypotheses, or each of the conjuncts \emph{individually} uses all of
the hypotheses.
The former choice gives us the tensor-product ($\otimes$), and the latter choice
gives us the with-product ($\with$).
These products are equivalent up to provability in structural logics but
distinct in linear logic.

\begin{figure}
  \begin{align*}
    A, B, C &\Coloneqq X \mid I \mid A \otimes B \mid A \multimap B
              \mid 0 \mid A \oplus B \mid \top \mid A \with B \\
    \Gamma, \Delta, \Theta &\Coloneqq {\cdot} \mid \Gamma, A
  \end{align*}
  \begin{mathpar}
    \ebrule{%
      \infer0[Id]{A \vdash A}
    }

    \and

    \ebrule{%
      \hypo{\Gamma \vdash A}
      \hypo{\Delta, A \vdash B}
      \infer2[Cut]{\Gamma, \Delta \vdash B}
    }

    \and

    \ebrule{%
      \hypo{\Gamma, B, A, \Delta \vdash C}
      \infer1[Exch]{\Gamma, A, B, \Delta \vdash C}
    }

    \and

    \ebrule{%
      \hypo{\Gamma \vdash C}
      \infer1[$I$-L]{\Gamma, I \vdash C}
    }

    \and

    \ebrule{%
      \infer0[$I$-R]{{\cdot} \vdash I}
    }

    \and

    \ebrule{%
      \hypo{\Gamma, A, B \vdash C}
      \infer1[$\otimes$-L]{\Gamma, A \otimes B \vdash C}
    }

    \and

    \ebrule{%
      \hypo{\Gamma \vdash A}
      \hypo{\Delta \vdash B}
      \infer2[$\otimes$-R]{\Gamma, \Delta \vdash A \otimes B}
    }

    \and

    \ebrule{%
      \hypo{\Gamma \vdash A}
      \hypo{\Delta, B \vdash C}
      \infer2[$\multimap$-L]{\Gamma, \Delta, A \multimap B \vdash C}
    }

    \and

    \ebrule{%
      \hypo{\Gamma, A \vdash B}
      \infer1[$\multimap$-R]{\Gamma \vdash A \multimap B}
    }

    \and

    \ebrule{%
      \infer0[$0$-L]{\Gamma, 0 \vdash C}
    }

    \and

    \text{(no $0$-R)}

    \and

    \ebrule{%
      \hypo{\Gamma, A \vdash C}
      \hypo{\Gamma, B \vdash C}
      \infer2[$\oplus$-L]{\Gamma, A \oplus B \vdash C}
    }

    \and

    \ebrule{%
      \hypo{\Gamma \vdash A_i}
      \infer1[$\oplus$-R$_i$]{\Gamma \vdash A_0 \oplus A_1}
    }

    \and

    \text{(no $\top$-L)}

    \and

    \ebrule{%
      \infer0[$\top$-R]{\Gamma \vdash \top}
    }

    \and

    \ebrule{%
      \hypo{\Gamma, A_i \vdash C}
      \infer1[$\with$-L$_i$]{\Gamma, A_0 \with A_1 \vdash C}
    }

    \and

    \ebrule{%
      \hypo{\Gamma \vdash A}
      \hypo{\Gamma \vdash B}
      \infer2[$\with$-R]{\Gamma \vdash A \with B}
    }
  \end{mathpar}
  \caption{Multiplicative-additive fragment of linear logic}
  \label{fig:mall}
\end{figure}

Implication ($\multimap$) and the tensor-product ($\otimes$, $I$) comprise the
\emph{multiplicative} fragment, while disjunction ($\oplus$, $0$) and the
with-product ($\with$, $\top$) comprise the \emph{additive} fragment.
Categorically, the additive fragment corresponds to products and coproducts,
while the multiplicative fragment corresponds to multicategorical tensor
products and closure.

\subsection{The $\oc$-modality}\label{sec:bang-modality}

\begin{mathpar}
  \ebrule{%
    \hypo{\oc\Gamma \vdash A}
    \infer1[Promotion]{\oc\Gamma \vdash \oc A}
  }

  \and

  \ebrule{%
    \hypo{\Gamma, A \vdash B}
    \infer1[Dereliction]{\Gamma, \oc A \vdash B}
  }

  \and

  \ebrule{%
    \hypo{\Gamma \vdash B}
    \infer1[Weakening]{\Gamma, \oc A \vdash B}
  }

  \and

  \ebrule{%
    \hypo{\Gamma, \oc A, \oc A \vdash B}
    \infer1[Contraction]{\Gamma, \oc A \vdash B}
  }
\end{mathpar}

In the intuitionistic linear logic sequent calculus ILL, $\oc A$ is defined
to be a proposition whose occurrences as antecedents can be deleted
(\TirName{Weakening}) and duplicated (\TirName{Contraction}), from which we can
extract $A$ (\TirName{Dereliction}), and which we can form from a conclusion
$A$ only when all antecedents are of the form $\oc X$ for some proposition $X$
(\TirName{Promotion}).
In short, $\oc A$ can be seen as an intuitionistic version of $A$, supporting
all of the structural rules of LJ, and only being able to be formed when it
does not depend on anything linear.

While this definition of $\oc$ works, in the sense that it gives the intended
class of models and cut elimination is maintained, it has some disadvantages.
Firstly, while the multiplicative and additive connectives are all characterised
by universal properties, $\oc$ is not.
This can be seen by the fact that taking the rules for $\oc$ and replacing each
occurrence of $\oc$ by a fresh connective $\oc'$ produces a logically distinct
connective.
One cannot produce any derivation of $\oc' A \vdash \oc A$ because
\TirName{Promotion} does not apply when there are antecedents not of the form
$\oc X$.
Finally, while $\oc$ is supposed to be a positive connective, it sometimes
behaves like a negative connective.
For example, for a positive connective like $\otimes$, the normal form proof
of the identity sequent $P \otimes Q \vdash P \otimes Q$ starts (from the
bottom) with a left rule and then, with the left in a more amenable form,
applies the right rule.
In contrast, the normal form proof of $\oc P \vdash \oc P$ starts with the
right rule, as it needs to have everything on the left be of the form $\oc X$.

\begin{mathpar}
  \ebrule{%
    \infer0[Id]{P \vdash P}
    \infer0[Id]{Q \vdash Q}
    \infer2[$\otimes$-r]{P, Q \vdash P \otimes Q}
    \infer1[$\otimes$-l]{P \otimes Q \vdash P \otimes Q}
  }

  \and

  \ebrule{%
    \infer0[Id]{P \vdash P}
    \infer1[Der]{\oc P \vdash P}
    \infer1[Pro]{\oc P \vdash \oc P}
  }
\end{mathpar}


\section{Mechanisations and systemitisations of substructural logics}
\label{sec:linmech}
In this section, I give an overview of techniques which have been used in
previous work to mechanise linear logic in proof assistants.
Na\"{i}ve approaches often struggle to represent concatenation of contexts in a
way which is amenable to the way dependent type theory-based proof assistants
work.
Problems even arise when working rigorously on paper when trying to avoid an
explicit exchange rule, such as how \cref{def:DILL-split} is not a precise
definition of a binary operator on lists.

\subsection{Typing with leftovers}

Typing with leftovers, introduced by \citet{allais17}, is a technique developed
to specify an algorithm for linear type checking as a declarative type system.
The idea is to consider an input context, a term, and an output context, where
the input context contains all of the variables in scope, and the output
context is the same minus any variables used by the term.
Type-checking of adjacent subterms of, for example, an application of
the $\otimes$-introduction rule, is done by
threading the context through from the output of the first term to the input of
the second.
Bound variables are introduced in the input context of the term in which they
are bound, and are expected to be absent in the output context of that term.

\Cref{fig:twl,fig:twl-mult} give rules in the typing-with-leftovers style for
the multiplicative fragment of intuitionistic linear logic.
Where \citeauthor{allais17} marks \emph{fresh} and \emph{stale} variables, I use
the notation I will use starting in \cref{sec:semirings}, labelling such
variables with $\gr 1$ and $\gr 0$, respectively.
Intuitively, the number describes how many more times that variable is to be
used.

\begin{figure}
  \begin{align*}
    \Gamma, \Delta, \Theta
    &\Coloneqq {\cdot} \mid \Gamma, \gr1x : A \mid \Gamma, \gr0x : A \\
    \mathcal{S} &\Coloneqq \Gamma \vdash M : A \boxtimes \Delta
  \end{align*}
  \caption{Typing with leftovers, context and sequent syntax}
  \label{fig:twl}
\end{figure}

\begin{figure}
  \begin{mathpar}
    \ebrule{%
      \infer0[Var]{\Gamma, \gr1x : A \vdash x : A \boxtimes \Gamma, \gr0x : A}
    }
    \and
    \ebrule{%
      \infer0[$I$-I]{\Gamma \vdash {*} : I \boxtimes \Gamma}
    }
    \and
    \ebrule{%
      \hypo{\Gamma \vdash M : I \boxtimes \Delta}
      \hypo{\Delta \vdash N : C \boxtimes \Theta}
      \infer2[$I$-E]{\Gamma \vdash \text{let }{*} = M\text{ in }N : C
        \boxtimes \Theta}
    }
    \and
    \ebrule{%
      \hypo{\Gamma \vdash M : A \boxtimes \Delta}
      \hypo{\Delta \vdash N : B \boxtimes \Theta}
      \infer2[$\otimes$-I]{\Gamma \vdash (M, N) : A \otimes B \boxtimes \Theta}
    }
    \and
    \ebrule{%
      \hypo{\Gamma \vdash M : A \otimes B \boxtimes \Delta}
      \hypo{\Delta, \gr1x : A, \gr1y : B \vdash N : C \boxtimes
        \Theta, \gr0x : A, \gr0y : B}
      \infer2[$\otimes$-E]{\Gamma \vdash \text{let }(x, y) = M\text{ in }N : C
        \boxtimes \Theta}
    }
    \and
    \ebrule{%
      \hypo{\Gamma, \gr1x : A \vdash M : B \boxtimes \Delta, \gr0x : A}
      \infer1[$\multimap$-I]{\Gamma \vdash \lambda x.~M : A \multimap B
        \boxtimes \Delta}
    }
    \and
    \ebrule{%
      \hypo{\Gamma \vdash M : A \multimap B \boxtimes \Delta}
      \hypo{\Delta \vdash N : A \boxtimes \Theta}
      \infer2[$\multimap$-E]{\Gamma \vdash M\,N : B \boxtimes \Theta}
    }
  \end{mathpar}
  \caption{Typing with leftovers, multiplicative fragment}
  \label{fig:twl-mult}
\end{figure}

The original paper extends the logic covered to binary additives --- $\with$ and
$\oplus$ --- with rules that check that terms agree on output contexts, as seen
in \cref{fig:twl-add}.
However, it is less clear how to handle nullary additives --- $\top$ and $0$ ---
as we would have 0 (rather than 2) potential candidates for the output context.
At some level, this problem is unavoidable in a system modelling linearity
checking because any checking strategy will expose the ambiguity in sequents
like $\gr1x : A \vdash (\langle\rangle, \langle\rangle) : \top \otimes \top$ of
whether the variable
$x$ was consumed in the left half or the right half.
Such an example is also considered in related work on proof seach for linear
logics, such as the work of \citet[p.\ 11]{WH94} and \citet[p.\ 150]{CHP00}.
It is not immediately clear whether the different solutions proposed by these
papers will apply to \citeauthor{allais17}' and my settings, given that they
both act on a set of formulas restricted to facilitate proof search.
The solutions also add significant, seemingly somewhat ad hoc, structure to the
syntax of sequents, with no semantic justification (rather being justified by
making their respective proof search algorithms efficient).
%As such, we may expect uses of $\top$-introduction (and similarly
%$0$-elimination) to be annotated with which variables they consume, in which case we
%get a viable typing with leftovers rule.

\begin{figure}
  \begin{mathpar}
    \ebrule{%
      \hypo{\Gamma \vdash M : A \boxtimes \Delta}
      \hypo{\Gamma \vdash N : B \boxtimes \Delta}
      \infer2[$\with$-I]{\Gamma \vdash \{M,N\} : A \with B \boxtimes \Delta}
    }
    \and
    \ebrule{%
      \hypo{\Gamma \vdash M : A \oplus B \boxtimes \Delta}
      \hypo{\Delta, \gr1x : A \vdash N : C \boxtimes \Theta, \gr0x : A}
      \hypo{\Delta, \gr1y : B \vdash O : C \boxtimes \Theta, \gr0y : B}
      \infer3[$\oplus$-E]{\Gamma \vdash
        \text{case }M\text{ of }\{x.~N; y.~O\} : C \boxtimes \Theta}
    }
  \end{mathpar}
  \caption{Typing with leftovers, a selection of the additive rules}
  \label{fig:twl-add}
\end{figure}

The original paper also does not show how to capture the exponential modality
$\oc$.
The solution given by both \citet{WH94} and \citet{CHP00} is, as in DILL, to
distinguish between linear variables and intuitionistic variables.
This gives rules like those of \cref{fig:twl-exp}.
The important invariant is that linear and intuitionistic variables stay
distinct, so any intuitionistic variable in the input context (annotated by
$\gr\omega$) must be intuitionistic in the output context.
%In the $\oc$-introduction rule, there are several choices we could make, but I
%have chosen to keep all the linear variables in scope but used so as to match
%the general style of variables staying in scope with the $\gr0$ annotation.

\begin{figure}
  \begin{mathpar}
    \ebrule{%
      \infer0[IVar]{\Gamma, \gr\omega x : A \vdash x : A
        \boxtimes \Gamma, \gr\omega x : A}
    }
    \and
    \ebrule{%
      \hypo{\gr0\gamma, \gr0\delta, \gr\omega\theta \vdash
        M : A \boxtimes \gr0\gamma, \gr0\delta, \gr\omega\theta}
      \infer1[$\oc$-I]{\gr0\gamma, \gr1\delta, \gr\omega\theta \vdash
        [M] : \oc A \boxtimes \gr0\gamma, \gr1\delta, \gr\omega\theta}
    }
    \and
    \ebrule{%
      \hypo{\Gamma \vdash M : \oc A \boxtimes \Delta}
      \hypo{\Delta, \gr\omega x : A \vdash N : C
        \boxtimes \Theta, \gr\omega x : A}
      \infer2[$\oc$-E]{\Gamma \vdash
        \text{let }[x] = M\text{ in }N : C \boxtimes \Theta}
    }
  \end{mathpar}
  \caption{Typing with leftovers, a possible way to capture $\oc$}
  \label{fig:twl-exp}
\end{figure}

However, this adaptation of the DILL style does not obviously generalise to
semiring annotations.
Even for the multiplicative fragment, we seem to be working against the
direction of addition, instead using a subtraction operation whenever we use a
variable.
For exponentials, though, and particularly the $\oc$-introduction rule, what I
have done seems ad-hoc, not based on any pointwise algebraic operation.

Also, the unusual form of sequents can cause some problems when working with a
typing with leftovers system.
For example, a traditional intuitionistic linear logic sequent
$\Gamma \vdash M : A$ corresponds to many different typing with leftovers
sequents:
\begin{itemize}
  \item $\gr1\Gamma \vdash M : A \boxtimes \gr0\Gamma$
  \item $\gr1\Gamma, \gr1x : B \vdash M : A \boxtimes \gr0\Gamma, \gr1x : B$
  \item $\gr1\Gamma, \gr0x : B \vdash M : A \boxtimes \gr0\Gamma, \gr0x : B$
  \item \&c.
\end{itemize}

Generally, any variable not used in the term can be added to both the input and
the output context with the same annotation.
Many of these variations are likely to appear in various typing derivations,
depending on what terms surround a given subterm.
This means that if we want to implement substitution, which involves putting a
term into an unknown surrounding, we have to navigate these different forms via
the \emph{framing property}.

The unusual form of sequents also somewhat obscures any attempt to interpret the
terms of a typing-with-leftovers system.
Though the $\boxtimes$ notation suggests a semantics into symmetric monoidal
closed categories where terms are morphisms from one iterated tensor product
(the input context) to another (the type and output context), the syntax is
incomplete for this semantics because we cannot produce anything interesting in
the output context.

Another piece of related work using a typing-with-leftovers style is that of
\citet{polakow15}.
There, \citeauthor{polakow15} encodes a linear embedded domain-specific language
inside Haskell using typeclass constraints.

\subsection{Yalla}

\Citet{laurent18} presents a collection of linear logics in a uniform style, and
various proofs relating them.
The logics share varying amounts of definitions and theorems --- for example,
the main linear logic is parametrised on whether to include mix rules and
whether to restrict exchange to cyclic permutations, whereas systems like the
Lambek calculus (with no exchange) and polarised linear logic are defined
separately from scratch.

The style used in Yalla is to realise sequents as lists of formulas.
The active formula tends to be forced to be the first formula in such a list,
with an explicit exchange rule being used to move such formulas into place.
\Citet{laurent18} points out that using multisets, as do some less formal
presentations of linear logic, is insufficient at distinguishing certain proofs
involving repeated assumptions or conclusions.
For example, we expect there to be two distinct derivations of
$A \otimes A \vdash A \otimes A$ (up to the appropriate equational theory):
the one that keeps the pair in the same order and the one that flips the order.
But if the \TirName{$\otimes$-L} rule unpacks the formula $A \otimes A$ into the
\emph{multiset} $\{A, A\}$, then we forget the order of the input pair.

Despite making sure to distinguish distinct proofs, the Yalla library does not
define any equational theories of proofs, so does not prove any results relying
on these distinctions.
While the complication of defining an equational theory in the presence of an
exchange rule is probably largely inevitable, having the exchange rule
introduces redundancy such that many equivalent proofs are not equal as data
structures.
The mechanisations presented in \cref{sec:simple} all sought to reduce this kind
of redundancy for intuitionistic logic, so that the only non-trivial equations
in the equational theory are $\beta$- and $\eta$-rules (i.e.\ computationally
interesting rules).
I will restore this property of the representation in \cref{sec:semirings}.

The relevance of Yalla to the work in this thesis is limited by the fact that
Yalla is based entirely on sequent calculi, whereas I am considering only
natural deduction calculi.

\subsection{Co-de-Bruijn syntax}

\Citet{McBride18} presents a mechanisation of the simply typed
$\lambda$-calculus in what he calls \emph{co-de-Bruijn} style.
He notes that syntax based on de Bruijn indices, as presented in
\cref{sec:simple}, finds a canonical way to place contractions and weakenings
by eagerly placing contractions wherever they could be needed (i.e., whenever a
rule has multiple premises) and leaving weakening as late as possible (i.e., at
the variable rule and at rules with no premises).
Co-de-Bruijn syntax, by contrast, finds a canonical way to place contractions
and weakenings by doing the reverse: weakening as early as possible (i.e., as
soon as the variable is bound) and contracting only where necessary.

Such a scheme straightforwardly adapts to multiplicative linear logic by
modifying the data structures presented by \citet{McBride18} to disallow all
contraction and weakening.
With the additive rules, however, we get cases where variables appear multiple
times syntactically in a term but are still considered linear by the type
system, the simplest example being
$x : A \vdash \langle x, x\rangle : A \with A$.
Such rules are perhaps a new kind compared to what \citeauthor{McBride18}
considers, but it seems likely that just copying the same context to all the
premises would not be too hard to accommodate.
Meanwhile, we may consider implementing the $\oc$-modality as in DILL, with
intuitionistic variables handled using the regular co-de-Bruijn machinery from
the paper.
In summary, the co-de-Bruijn approach is promising for capturing linearity, but
has not been thoroughly investigated.

\subsection{Fitch-style modalities}

An alternative to the approach of \citet{judgmental} to adapt modal logics (and
particularly IS4) to natural deduction is using Fitch-style calculi.
Fitch-style calculi, as codified and studied by \citet{Borghuis-thesis}, are
distinguished by allowing for contexts containing \emph{locks}, written $\Lock$,
with the variable rule being restricted so that only variables not behind locks
are immediately accessible.

Below I give the main rules of Fitch-style IK (intuitionistic logic K).
Other normal modal logics are obtained by strengthening the $\Box$-elimination
rule to remove varying numbers of locks.
For example, we can add axiom \TirName{T} by allowing for 0 or 1 locks to be
removed, axiom \TirName{4} by allowing 1 or more locks to be removed, or both
axioms together by allowing any number of locks to be removed from the
right-hand end of the context (together with any variables to the right of a
removed lock).
The \TirName{$\Box$-I} rule stays the same, and forms part of an adjunction of
the form ``$\Lock \dashv \Box$''.

\begin{mathpar}
  \ebrule{%
    \hypo{\Lock \notin \Gamma'}
    \infer1[Var]{\Gamma, A, \Gamma' \vdash A}
  }
  \and
  \ebrule{%
    \hypo{\Gamma, \Lock \vdash A}
    \infer1[$\Box$-I]{\Gamma \vdash \Box A}
  }
  \and
  \ebrule{%
    \hypo{\Lock \notin \Gamma'}
    \hypo{\Gamma \vdash \Box A}
    \infer2[$\Box$-E$_K$]{\Gamma, \Lock, \Gamma' \vdash A}
  }
\end{mathpar}

A syntactic advantage of Fitch-style calculi over the calculus introduced by
\citet{judgmental} is that Fitch-style calculi support a projection-style
eliminator for $\Box$, which makes it easier to use than the pattern-matching
eliminator of \citeauthor{judgmental}.
A disadvantage is that $\Lock$ cannot be understood as a kind of hypothetical
judgement like the rest of the context, so many of the heuristics we relied on
in \cref{sec:simple} and \cref{sec:modal} fail.
In fact, the addition of locks represents a large change to the judgemental
structure of the calculus, apparently requiring a complete overhaul of the basic
metatheory.

\Citet{VRC22} have completed a significant mechanisation of the metatheory of
a Fitch-style calculus in Agda.
This work shows that, despite the change in the structure of the metatheory,
Fitch-style calculi are amenable to mechanised proofs.

I am not aware of any work discussing linear Fitch-style calculi.

\subsection{Systematisations of substructural logics}

Several pieces of prior work have aimed to give general accounts of a range of
substructural calculi, in a similar vein to existing accounts of aspects of
non-substructural calculi.
I review some of these, particularly as a way to give a comparison to the
adaptation of the methods of \cref{sec:simple} that I spend the rest of this
thesis on.

\paragraph{Linear Logical Framework}
In \cref{sec:lf}, I discussed logical frameworks based on the
$\lambda^\Pi$-calculus, and their use in the mechanisation of non-substructural
programming languages.
\Citet{CP02} extend $\lambda^\Pi$ to a calculus
$\lambda^{\Pi\multimap\with\top}$, and use that to create a logical framework
supporting linear logics: the Linear Logical Framework (LLF).
The $\Pi$ type former still forms intuitionistic weak dependent function spaces,
while the new $\multimap$ forms linear weak non-dependent function spaces.
They handle the distinction between intuitionistic and linear variables thus
introduced in the same style as DILL, with the argument of an intuitionistic
application having the same intuitionistic restriction as we saw in DILL's
\TirName{$\oc$-I} rule.

The additive connectives $\with$ and $\top$ provide a way to state rules whose
premises respectively share and arbitrarily consume linear variables, like in
the rules for additive connectives in linear logic.
I will revisit this method of stating typing rules in terms of \emph{sharing}
and \emph{separating} (as given by right-nested sequences of $\multimap$s)
conjunctions of premises in \cref{sec:lnd}, though in \cref{sec:bunched-logic}
I argue that there is a closer connection to bunched logics than linear logics.

The main focus of the original paper is to represent mutation in an ML-like
language via state updates mediated by $\multimap$, though they also mention
having mechanised some metatheory of linear calculi.
Some example programs are currently available at
\url{https://www.cs.cmu.edu/~iliano/projects/LLF/index.html}.

\paragraph{Encoding linearity in LF}
\Citet{crary10} gives a method of encoding linearity constraints in a
conventional, non-substructural, logical framework.
He implements this approach in the LF-based proof assistant Twelf~\citep{Twelf}.
He uses a predicate \texttt{linear : (term -> term) -> type}, where
\texttt{term} is a type of preterms, and thus \texttt{term -> term} is (thanks
to the weak function space of LF) the type of preterms with one free variable.
The predicate \texttt{linear} then says that that free variable is used linearly
in its term, which is defined inductively on the structure of preterms.
The \texttt{linear} predicate is used by the typing relation wherever variables
are bound.
The development handles all of intuitionistic linear logic, with the
$\oc$-modality treated with a DILL-style distinction between linear and
intuitionistic variables.
Intuitionistic variables are implemented simply by not checking for linearity in
the \TirName{$\oc$-E} rule.
\Citet[\S 4]{crary10} also shows how to adapt this technique to a PD-style
presentation of IS4.

As an example, let us look at the typing and linearity rules for the binary
tensor product.
Typing is given by a relation \texttt{of : term -> tp -> type}, where
\texttt{tp} is the type of object-level types.
Each syntactic form has a typing rule and potentially several linearity rules,
understood disjunctively as logic programming clauses.
The rules for the introduction form are listed below.
The typing relation is just like it would be for pairs in the simply typed
$\lambda$-calculus: $(M, N) : A \otimes B$ if $M : A$ and $N : B$.
The linearity rules are symmetric, so I will just consider
\texttt{linear/tpair1}.
It says that $x \vdash (M[x], N)$ is linear if $x \vdash M[x]$ is linear.
This rule is subtle in that not applying \texttt{x} to \texttt{N} implies that
\texttt{x} is fresh (and therefore unused) in \texttt{N}.
Where $\otimes$-pairs have two linearity rules, the $I$-unit, and also the
introduction form for $\oc$, have no linearity rules, meaning that no linear
variables can be used in or by them.

\begin{verbatim}
of/tpair : of (tpair M N) (tensor A B) <- of M A <- of N B.
linear/tpair1 : linear ([x] tpair (M x) N) <- linear ([x] M x).
linear/tpair2 : linear ([x] tpair M (N x)) <- linear ([x] N x).
\end{verbatim}

Meanwhile, the rules for the eliminator are somewhat more involved, thanks to
the bound variables.
First, the typing rule shows how \texttt{of} relies on \texttt{linear}, checking
each bound variable for linearity.
Because we have two bound variables, we need to check that the term \texttt{N}
is linear in both.
We do this by checking that, for all \texttt{y}, \texttt{N} is linear in
\texttt{x}, and that for all \texttt{x}, \texttt{N} is linear in \texttt{y}.
The linearity rules have the same choice and careful management of free
variables as they did for the introduction form.
In addition, in the rule \texttt{linear/lett2}, the bound variables in
\texttt{N} have to be universally quantified while we check for linearity in the
free variable \texttt{z}.

\begin{verbatim}
of/lett : of (lett M ([x] [y] N x y)) C
  <- of M (tensor A B) <- ({x} of x A -> {y} of y B -> of (N x y) C)
  <- ({y} linear ([x] N x y)) <- ({x} linear ([y] N x y))
linear/lett1 : linear ([z] lett (M z) ([x] [y] N x y))
  <- linear ([z] M z)
linear/lett2 : linear ([z] lett M ([x] [y] N z x y))
  <- ({x} {y} linear ([z] N z x y))
\end{verbatim}

\Citet{crary10} goes on to extend this encoding with intuitionistic dependent
$\Pi$-types, producing an object theory comparable to the
$\lambda^{\Pi\multimap\with\top}$ metatheory developed by \citet{CP02}.
If one wants to mix linearity and dependency following the methodology of
\citet{Atkey18}, then it is crucial that linear variables are still free in
subterms from which they have been discarded.
At first sight, it appears that \citeauthor{crary10}'s encoding violates this
because of its use of ``does not appear free'' to mean ``is not used'' in many
linearity rules.
However, one could imagine introducing an \texttt{unused} predicate similar to
\texttt{linear} in order to handle unused free variables, at the cost of a few
extra rules and a somewhat heavier encoding (scaling with the \emph{size} of the
term, rather than the depth).
Indeed, one could imagine parametrising the \texttt{linear} predicate so as to
encode the semiring-annotated systems I discuss in \cref{sec:semirings}.

The approach of \citet{crary10} removes the objection to the work of
\citet{CP02} that each new substructural discipline would need a new proof
assistant by encoding linearity in a standard intuitionistic logical framework.
However, the encoding makes linear variables second-class compared to
intuitionistic variables.
While intuitionistic variables are just there thanks to the metatheory, linear
variables must essentially be explicitly quantified.

\paragraph{Licata-Shulman-Riley}
\Citet{LicataSR17} describe a framework for specifying and working with a wide
range of substructural logics.
I discuss exactly what this range is in \cref{sec:semirings-conc}, in relation
to the calculi I describe in the rest of this thesis.
For now, it suffices to say that this framework is specified in enough detail
that it should be possible to mechanise it directly, but I am not aware of
anyone having done so.
\Citet{Restall1999} describes a similar approach.

\paragraph{Tanaka-Power}
The work of \citet{FPT99}, which I discussed in \cref{sec:fiore}, has been
extended to substructural syntaxes by \citet{TP06}.
This work gives a mechanism for turning a description of contexts and their
structural rules (expressed as a pseudo-monad on the 2-category of categories)
into a framework for defining substructural syntaxes, and more generally a
category of algebras of which the syntax is the initial object.
As examples, they give the untyped $\lambda$-calculus, an untyped multiplicative
linear logic, and a bunched logic.
These examples show a broad range of substructural disciplines they support ---
comparable to the work of \citet{LicataSR17} (which I discuss in
\cref{sec:semirings-conc}), and more than I discuss in this thesis.
However, they also show two of the limitations of their approach.
Firstly, this work provides no way to track types, though it should be possible
to incorporate types at the expense of complicating the categorical
constructions they use.
Secondly, it appears to be impossible to encode the syntactic forms used for the
additive connectives (i.e., the Cartesian product and coproduct) of linear
logic.
Essentially, subterms can only be combined into a larger term in the same ways
as how contexts can be appended together.
For example, in a bunched logic, contexts can be combined through both sharing
conjunction (the Cartesian product) and separating conjunction
(a monoidal product).
Correspondingly, the syntax descriptions of \citeauthor{TP06} allow for the
syntax of sharing pairs and separating pairs.
However, in the case of linear logic, contexts can only be combined via a
monoidal product, so we only get separating pairs (tensor-products) and not
sharing pairs (with-products).

\paragraph{Granule}
\citet{BrunelGMZ14}, \citet{GhicaS14}, \ldots

