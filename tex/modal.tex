Modal logics, and in particular their modalities, are usually presented in
philosophical and mathematical logic in an axiomatic style.
For example, the common modal logic S4 may be presented by taking a presentation
of classical logic, adding a unary $\Box$ (``box'') operator to the formulas,
keeping the logical rules as-are, and adding all of the axioms and rules
listed in \cref{fig:S4-axioms}.
Note that, unlike elsewhere in this thesis, the empty context in the
necessitation rule \TirName{N} is a proper restriction:
While the axioms hold in any context, thanks to weakening, and ordinary logical
rules hold in any context, the necessitation rule only holds in the empty
context.
We may apply weakening to the conclusion of \TirName{N} to embed such a
derivation into a larger derivation, but we may not use hypotheses from the
outside in such a subderivation.

\begin{figure}
  \begin{mathpar}
    \ebrule{%
      \hypo{\vdash A}
      \infer1[N]{\vdash \Box A}
    }
    \and
    \ebrule{%
      \infer0[K]{\vdash \Box\plr{A \to B} \to \plr{\Box A \to \Box B}}
    }
    \and
    \ebrule{%
      \infer0[T]{\vdash \Box A \to A}
    }
    \and
    \ebrule{%
      \infer0[4]{\vdash \Box A \to \Box\plr{\Box A}}
    }
  \end{mathpar}
  \caption{The axioms and rules required in a traditional presentation of S4}
  \label{fig:S4-axioms}
\end{figure}

In this thesis, I am working with intuitionistic logics, so I start with a base
of intuitionistic logic, and study intuitionistic S4 (IS4).
The axiomatic presentation can be introduced on top of various presentations of
intuitionistic/classical logic --- including natural deduction, sequent
calculi, Hilbert systems, and abstractly taking the set of
intuitionistic/classical tautologies.
For concreteness and rigour, I will assume a natural deduction system NJ
allowing for explicit weakening.

The axiomatic presentation is convenient for talking about semantics, with
different choices of axioms corresponding fairly directly to natural
restrictions on models.
However, the syntax is quite poorly behaved, at least for programming language
applications.
As mentioned earlier, the necessitation rule makes requirements of its context,
which makes it incompatible with the methods discussed in \cref{sec:simple}.
Essentially, we are required to make necessitation a special case in renaming,
substitution, and all the other traversals.
Additionally, axioms require us to modify the notion of canonical
form~\citep[p.\ 79]{Prawitz65}, and redo
any proofs that rely on all closed terms of function type being
$\lambda$-abstractions.

The methodology I use to produce a nicer syntax is that of \citet{judgmental}:
capturing the desired modality in the judgemental structure of the syntax.
To start, note the following two facts about a ``promotion'' rule similar to one
mentioned by \citet{BBdPH93}.

\begin{proposition}\label{thm:NKT4-P}
  In NJ + \TirName{N} + \TirName{K} + \TirName{4}, the following principle is
  admissible.
  \[
    \ebrule{%
      \hypo{\Box B_1, \ldots, \Box B_n \vdash A}
      \infer1[Promotion]{\Box B_1, \ldots, \Box B_n \vdash \Box A}
    }
  \]
\end{proposition}
\begin{proof}
  By induction on $n$.
  When $n = 0$, \TirName{Promotion} becomes just the necessitation rule \TirName{N}.
  For $n = \lvert\Gamma\rvert + 1$, we have the following derivation, where
  $\Box\Gamma$ abbreviates
  $\Box\Gamma_1, \ldots, \Box\Gamma_{\lvert\Gamma\rvert}$ and \TirName{4$'$} and
  \TirName{K$'$} abbreviate the obvious combinations of
  \TirName{4} and \TirName{K}, respectively, and \TirName{$\to$-E}.
  I silently apply weakening throughout the derivation for the sake of concision.
  \[
    \begin{prooftree}
      \hypo{\Box\Gamma, \Box B \vdash A}
      \infer1[$\to$-I]{\Box\Gamma \vdash \Box B \to A}
      \infer1[IH]{\Box\Gamma \vdash \Box\plr{\Box B \to A}}
      \infer0[Var]{\Box B \vdash \Box B}
      \infer1[4$'$]{\Box B \vdash \Box\Box B}
      \infer2[K$'$]{\Box\Gamma, \Box B \vdash \Box A}
    \end{prooftree}
  \]
\end{proof}

\begin{proposition}\label{thm:PT-NKT4}
  In NJ + \TirName{Promotion} + \TirName{T}, we can derive the original axioms
  and rule of S4: \TirName{N}, \TirName{K}, \TirName{T}, and \TirName{4}
\end{proposition}
\begin{proof}
  The rule \TirName{N} and axiom \TirName{T} are immediate.
  For axioms \TirName{K} and \TirName{4}, we have the following derivations,
  where \TirName{T$'$} abbreviates the obvious combination of \TirName{T} and
  \TirName{$\to$-E}.
  \begin{mathpar}
    \ebrule{%
      \infer0[T$'$]{\Box\plr{A \to B} \vdash A \to B}
      \infer0[T$'$]{\Box A \vdash A}
      \infer2[$\to$-E]{\Box\plr{A \to B}, \Box A \vdash B}
      \infer1[Promotion]{\Box\plr{A \to B}, \Box A \vdash \Box B}
      \infer1[$\to$-I]{\Box\plr{A \to B} \vdash \Box A \to \Box B}
      \infer1[$\to$-I]{\vdash \Box\plr{A \to B} \to \Box A \to \Box B}
    }
    \and
    \ebrule{%
      \infer0[Var]{\Box A \vdash \Box A}
      \infer1[Promotion]{\Box A \vdash \Box\Box A}
      \infer1[$\to$-I]{\vdash \Box A \to \Box\Box A}
    }
  \end{mathpar}
\end{proof}

\Cref{thm:NKT4-P,thm:PT-NKT4} suggest that a \TirName{Promotion}-like rule to
introduce the $\Box$ operator and a \TirName{T}-like rule to eliminate the
$\Box$ operator may be a possible approach to capture S4 in a natural deduction
style.
However, we do not want to add such a rule to our system because it is not
stable under substitution.
For example, let
$\sigma : \Box A \wedge \Box B \env\vdash \Box A, \Box B$ be the substitution
whose two components are given by left projection and right projection,
respectively.
The substitution principle says that we should be able to turn any derivation of
$\Box A, \Box B \vdash \Box C$ into a derivation of
$\Box A \wedge \Box B \vdash \Box C$.
However, if we concluded $\Box A, \Box B \vdash \Box C$ by \TirName{Promotion},
then it becomes unclear how to conclude $\Box A \wedge \Box B \vdash \Box C$.
In particular, \TirName{Promotion} does not apply because the formula
$\Box A \wedge \Box B$ is headed by $\wedge$, rather than $\Box$.

Instead of constraining the types of all of the items in the context, the
solution given by \citet{judgmental} is to diversify the shapes of contexts, and
constrain those shapes when giving our necessitation rule.
In this case, we are interested in hypotheses being $\Box$-like, so we introduce
a judgement form $A~\mathit{valid}$, contrasting with the judgement we had
previously written just $A$, but will now write $A~\mathit{true}$.
We construct the system so that $A~\mathit{valid}$ is equivalent to
$\Box A~\mathit{true}$.
%Note that, in S4, $\Box\Box A$ is logically equivalent to $\Box A$
Because the choice of $\mathit{valid}$ or $\mathit{true}$ is part of the
structure of the context, rather than being part of the type, we are able to
define substitution between contexts in a way that treats valid and true entries
differently --- restricting substitution so that it is compatible with an
adapted version of \TirName{Promotion}.

Let us take the natural deduction presentation of the simply typed
$\lambda$-calculus given in \cref{sec:terms}.
Where that system has sequents of the form $A_1, \ldots, A_n \vdash B$, we
instead now write
$A_1~\mathit{true}, \ldots, A_n~\mathit{true} \vdash B~\mathit{true}$.
We may also have hypothetical occurrences of $A~\mathit{valid}$, but all logical
rules will target judgements of the form $A~\mathit{true}$.
Hypotheses of the form $A~\mathit{valid}$ arise from the elimination rule for
the $\Box$ operator, \TirName{$\Box$-E}.
The only special thing about valid hypotheses is that, once they are bound, they
are preserved by the \TirName{$\Box$-I} rule, which is our recasting of
\TirName{Promotion} as the introduction rule for $\Box$.
Otherwise, like true hypotheses, an $A~\mathit{valid}$ hypothesis can be used to
obtain $A~\mathit{true}$, via the \TirName{vVar} rule, as justified by axiom
\TirName{4}.

The three new rules are listed in \cref{fig:PD}.
Let $\Gamma_v$ be the context $\Gamma$ but with all true hypotheses removed,
leaving only valid hypotheses.
Because I am working with a de Bruijn index-style presentation, weakening needs
to be admissible.
Therefore, the \TirName{$\Box$-I} rule needs to remove its true hypotheses,
rather than only applying in a context in which there are no true hypotheses.

\begin{figure}
  \begin{mathpar}
    \ebrule{%
      \hypo{\Gamma_v \vdash A~\mathit{true}}
      \infer1[$\Box$-I]{\Gamma \vdash \Box A~\mathit{true}}
    }
    \and
    \ebrule{%
      \hypo{\Gamma \vdash \Box A~\mathit{true}}
      \hypo{\Gamma, A~\mathit{valid} \vdash B~\mathit{true}}
      \infer2[$\Box$-E]{\Gamma \vdash B~\mathit{true}}
    }
    \and
    \ebrule{%
      \hypo{\Gamma \ni \plr{A~\mathit{valid}}}
      \infer1[vVar]{\Gamma \vdash A~\mathit{true}}
    }
  \end{mathpar}
  \caption{The new rules of the \citeauthor{judgmental} presentation of IS4.}
  \label{fig:PD}
\end{figure}

Whereas \citet{judgmental} prove single substitution for both true and valid
hypotheses (as well as \emph{possible} hypotheses, and for possible judgements,
in their system dealing with a possibility modality $\Diamond$) separately, I
build on the simultaneous substitution procedure given in
\cref{sec:syntactic-kits} to give a unified simultaneous substitution procedure.
Recall that the basic definition was that of \emph{$\vDash$-environment},
defined in the intuitionistic case by
$\Gamma \env\vDash \Delta \coloneqq \Pi A.~\Delta \ni A \to \Gamma \vDash A$.
Essentially the same formula deals with true hypotheses, but for valid
hypotheses, we need to think of something new.
We want to be able to instantiate $\vDash$ to $\vdash$ so as to derive
substitution, but $\Gamma \vdash A~\mathit{valid}$ is not a valid sequent in our
system.
Instead, I consider $\Gamma \vdash \Box A~\mathit{true}$, and in particular the
canonical way of deriving such a sequent: the \TirName{$\Box$-I} rule.
\TirName{$\Box$-I} says that we can get
$\Gamma \vdash \Box A~\mathit{true}$ from $\Gamma_v \vdash A~\mathit{true}$.
I take the form of latter sequent to replace $\Gamma \vDash A~\mathit{valid}$,
giving the following definition.

\begin{definition}\label{def:PD-env}
  A \emph{(modal) $\vDash$-environment} from $\Gamma$ to $\Delta$ is given as
  follows.
  \[
    \Gamma \env\vDash \Delta \coloneqq
    \Pi A.~\plr{\Delta \ni A~\mathit{true} \to \Gamma \vDash A}
    \times \plr{\Delta \ni A~\mathit{valid} \to \Gamma_v \vDash A}
  \]
\end{definition}

With this definition, I reproduce the kit-based machinery from
\cref{sec:syntactic-kits}.
However, I make crucial use of the \emph{stability under renaming} property ---
rather than \emph{stability under context extension} --- to deal with not just
context extensions, but also discarding of true hypotheses by the
\TirName{$\Box$-I} rule.
To state this property, I firstly need to be precise about what a renaming is.

\begin{definition}\label{def:PD-var}
  The set of variables in context $\Gamma$ of type $A$, notated $\Gamma \ni A$,
  is the disjoint union of the sets $\Gamma \ni A~\mathit{true}$ and
  $\Gamma \ni A~\mathit{valid}$ --- respectively, the true variables and the
  valid variables in $\Gamma$ of type $A$.

  The set of \emph{renamings} from $\Gamma$ to $\Delta$ is
  $\Gamma \env\ni \Delta$.
\end{definition}

\Cref{def:PD-var} allows us to state a unified variable rule: For each
$x : \Gamma \ni A$, we have a corresponding term
$x : \Gamma \vdash A~\mathit{true}$.
However, with the new definition of both environments and variables, we must
check that they interact in the desired way.

\begin{lemma}[lookup]\label{thm:PD-lookup}
  If $\vDash$ respects renaming (i.e., we have a function
  $\mathrm{ren}^\vDash :
  \Gamma \env\ni \Delta \to \Delta \vDash A \to \Gamma \vDash A$
  for each $\Gamma$, $\Delta$, and $A$), then
  from an environment $\rho : \Gamma \env\vDash \Delta$ and a variable
  $x : \Delta \ni A$, we get a value $\rho(x) : \Gamma \vDash A$.
\end{lemma}
\begin{proof}
  I proceed by cases on whether $x$ is a true or valid variable.
  In the true case, the result is straightforward.
  In the valid case, \cref{def:PD-env} gives us a value in $\Gamma_v \vDash A$.
  We need a renaming of type $\Gamma \env\vDash \Gamma_v$ to get the desired
  value in $\Gamma \vDash A$.
  The renaming is the one that, conceptually, discards the true hypotheses.
\end{proof}

We then need to prove that environments are preserved by everything we do to
contexts in a derivation.
One thing we do, like in the simply typed case, is to bind new variables, and
this is handled by \cref{thm:PD-bindEnv}.
The other thing, which is new for modal logic, is to discard the true
hypotheses, as in the \TirName{$\Box$-I} rule, which is handled by
\cref{thm:PD-vEnv}.

\begin{lemma}[bindEnv]\label{thm:PD-bindEnv}
  If we have $\mathrm{vr} : \Gamma \ni A \to \Gamma \vDash A$ for all
  $\Gamma$ and $A$, and
  $\mathrm{ren}^\vDash :
  \Gamma \env\ni \Delta \to \Delta \vDash A \to \Gamma \vDash A$
  for all $\Gamma$, $\Delta$, and $A$,
  then from an environment $\rho : \Gamma \env\vDash \Delta$
  we can get an environment of type $\Gamma, \Theta \env\vDash \Delta, \Theta$
  for all $\Gamma$, $\Delta$, and $\Theta$.
\end{lemma}
\begin{proof}
  I consider separately the part of the environment we are constructing that
  deals with true hypotheses and the part that deals with valid hypotheses.
  These cases correspond to the two factors in the expression in
  \cref{def:PD-env}.

  The case for true hypotheses is essentially the same as in
  \cref{sec:syntactic-kits}.
  We take cases on whether the $x : \Delta, \Theta \ni A~\mathit{true}$ is in
  $\Delta$ or $\Theta$.
  In the $\Delta$ case, we apply $\rho$ and then rename via
  $\mathrm{ren}^\vDash$ using the obvious renaming of type
  $\Gamma, \Theta \env\ni \Gamma$.
  In the $\Theta$ case, we embed our $z : \Theta \ni A~\mathit{true}$ as
  $\searrow z : \Gamma, \Theta \ni A~\mathit{true}$, and use $\mathrm{vr}$ to
  get the required value.

  The case for valid hypotheses is similar.
  We take the same cases, with the $\Delta$ case differing only in which part of
  $\rho$ we need to use (the valid part, rather than the true part).
  In the $\Theta$ case, we have $z : \Theta \ni A~\mathit{valid}$.
  The $_v$ operation keeps all valid hypotheses, so we have
  $z_v : \Theta_v \ni A~\mathit{valid}$.
  Using the fact that $(\Gamma, \Theta)_v = \Gamma_v, \Theta_v$, we have
  $\searrow z_v : (\Gamma, \Theta)_v \ni A~\mathit{valid}$, and we use
  $\mathrm{vr}$ to get the required value.
\end{proof}

\begin{lemma}\label{thm:PD-vEnv}
  From an environment $\rho : \Gamma \env\vDash \Delta$, we can get an
  environment $\rho_v : \Gamma_v \env\vDash \Delta_v$.
\end{lemma}
\begin{proof}
  Suppose $x : \Delta_v \ni A$.
  Because $\Delta_v$ contains only and all of the valid hypotheses from
  $\Delta$, we actually have $x' : \Delta \ni A~\mathit{valid}$.
  Applying the second part of $\rho$ on $x'$ gives us a value in
  $\Gamma_v \vDash A$.
  We actually wanted a value in $\plr{\Gamma_v}_v \vDash A$, but $_v$ is
  idempotent, so we already have this.
\end{proof}

Then, we use all of the previous features to prove the key theorem.

\begin{theorem}[trav]\label{thm:PD-trav}
  For any notion of value $\vDash$ which is stable under renaming, admits
  variables (via some $\mathrm{vr}$ as in \cref{thm:PD-bindEnv}), and
  embeds into terms (via some
  $\mathrm{tm} : \Gamma \vDash A \to \Gamma \vdash A~\mathit{true}$ for all
  $\Gamma$ and $A$),
  an environment $\rho : \Gamma \env\vDash \Delta$ and a term
  $M : \Delta \vdash A~\mathit{true}$ yield a term
  $M[\rho] : \Gamma \vdash A~\mathit{true}$.
\end{theorem}
\begin{proof}
  I proceed by induction on the term $M$.
  Here I consider only variables and the two rules governing the $\Box$
  connective.
  It is easy to adapt this procedure to any of the types from simply typed
  $\lambda$-calculus.

  % var
  I consider variables being given by the unified variable rule proposed in the
  paragraph before \cref{thm:PD-lookup}.
  Given this, from a variable $x : \Delta \ni A$, \cref{thm:PD-lookup} gives us
  a value $\rho(x) : \Gamma \vDash A$, which $\mathrm{tm}$ turns into a term of
  the desired form.

  % \Box-I
  If the term was made from the \TirName{$\Box$-I} rule, we use that rule to
  produce the resulting term, and need to get from a term
  $M : \Delta_v \vdash A~\mathit{true}$ to a term in
  $\Gamma_v \vdash A~\mathit{true}$.
  It is enough to use the induction hypothesis, updating the environment using
  \cref{thm:PD-vEnv}.

  % \Box-E
  If the term was made from the \TirName{$\Box$-E} rule, we again use the same
  rule to construct the output, but have to get from a term
  $M : \Delta \vdash \Box A~\mathit{true}$ to a term in
  $\Gamma \vdash \Box A~\mathit{true}$, and from a term
  $N : \Delta, A~\mathit{valid} \vdash B~\mathit{true}$ to a term in
  $\Gamma, A~\mathit{valid} \vdash B~\mathit{true}$.
  The former follows from a straightforward use of the induction hypothesis,
  while the latter needs \cref{thm:PD-bindEnv} before applying the induction
  hypothesis.
\end{proof}

To get our desired corollaries of simultaneous renaming and substitution, we
first have to show that variables respect renaming.
In the pure intuitionistic case of \cref{sec:simple}, this was trivial because
renamings were precisely functions between sets of variables.
In the modal case, we have to be careful about the distinction between true and
valid hypotheses, and how the context is restricted in the valid case.

\begin{lemma}[variables respect renaming]\label{thm:PD-ren-var}
  Given a renaming $\rho : \Gamma \env\ni \Delta$ and a variable
  $x : \Delta \ni A$, we get a variable in $\Gamma \ni A$.
\end{lemma}
\begin{proof}
  Note that we cannot use \cref{thm:PD-lookup} with $\vDash$ instantiated to
  $\ni$ because it would circularly require that variables respect renaming.
  However, the procedure in this case is similar to that of
  \cref{thm:PD-lookup}.

  We take cases on whether $x$ is true or valid, and the result in the true case
  comes immediately by applying the true part of $\rho$.
  For $x : \Delta \ni A~\mathit{valid}$, $\rho$ gives us a variable in
  $\Gamma_v \ni A$.
  This variable must be valid ($\Gamma_v \ni A~\mathit{valid}$), and from there
  it is easy to get a variable in the larger context $\Gamma$.
\end{proof}

\begin{corollary}[renaming]\label{thm:PD-ren}
  Given a renaming $\rho : \Gamma \env\ni \Delta$ and a term
  $M : \Delta \vdash A~\mathit{true}$, we get a term in
  $\Gamma \vdash A~\mathit{true}$.
\end{corollary}
\begin{proof}
  We use \cref{thm:PD-trav}, with $\vDash$ being $\ni$, $\mathrm{vr}$ being the
  identity function, $\mathrm{tm}$ being the unified variable rule, and renaming
  of $\ni$ being given by \cref{thm:PD-ren-var}.
\end{proof}

\begin{corollary}[substitution]\label{thm:PD-sub}
  Given a substitution $\rho : \Gamma \env\vdash \Delta$ and a term
  $M : \Delta \vdash A~\mathit{true}$, we get a term in
  $\Gamma \vdash A~\mathit{true}$.
\end{corollary}
\begin{proof}
  We use \cref{thm:PD-trav}, with $\vDash$ being $- \vdash -~\mathit{true}$,
  $\mathrm{vr}$ being the unified variable rule, $\mathrm{tm}$ being the
  identity function, and renaming being given by \cref{thm:PD-ren}.
\end{proof}

With these basic syntactic lemmas proved in a manner largely following the
proofs of \cref{sec:simple}, it is plausible that this approach could be
extended to handle generic semantics and generic syntax, following
\cref{sec:gen-sem,sec:gen-syn}, respectively.
However, I hold off from developing this until phrasing it more generally in
\cref{sec:framework}.
