\subsection{Motivation of linearity}

In LJ, two distinct, but equiderivable, canonical definitions of $\wedge$ can
be given.

In the first approach, $\wedge^-$ is conceived to be a
\emph{categorial product}.
Such a product is constructed by providing two maps in from the same antecedents
$\Gamma$.
It is used by projecting out one of the sides.

\begin{mathpar}
  \inferrule*[right=$\wedge^-$-r]
  {\Gamma \vdash A \\ \Gamma \vdash B}
  {\Gamma \vdash A \wedge^- B}

  \and

  \inferrule*[right=$\wedge^-$-l$_0$]
  {\Gamma, A \vdash C}
  {\Gamma, A \wedge^- B \vdash C}

  \and

  \inferrule*[right=$\wedge^-$-l$_1$]
  {\Gamma, B \vdash C}
  {\Gamma, A \wedge^- B \vdash C}
\end{mathpar}

In the second approach, $\wedge^+$ is conceived to be a \emph{tensor product}.
Intuitively, $\wedge^+$ internalises the comma of context concatenation.
Such a product is used by giving access to both sides simultaneously.
It is constructed by constructing both sides and combining the antecedents
required by each side.

\begin{mathpar}
  \inferrule*[right=$\wedge^+$-r]
  {\Gamma \vdash A \\ \Delta \vdash B}
  {\Gamma, \Delta \vdash A \wedge^+ B}

  \and

  \inferrule*[right=$\wedge^+$-l]
  {\Gamma, A, B \vdash C}
  {\Gamma, A \wedge^+ B \vdash C}
\end{mathpar}

When we prove the logical equivalence of these two formulations, we notice that
the structural rules of weakening and contraction are essential.
When we remove weakening and contraction, $\wedge^-$ and $\wedge^+$ become
logically distinct connectives, which we notate $\&$ and $\otimes$,
respectively.

\begin{mathpar}
  \inferrule*[right=$\wedge^-$-r]
  {%
    \inferrule*[right=Weak$^*$,fraction={\cdot\cdots\cdot}]
    {\Gamma \vdash A}
    {\Gamma, \Delta \vdash A}
    \\
    \inferrule*[Right=Weak$^*$,fraction={\cdot\cdots\cdot}]
    {\Delta \vdash B}
    {\Gamma, \Delta \vdash B}
  }
  {\Gamma, \Delta \vdash A \wedge^- B}

  \and

  \inferrule*[right=Contr]
  {%
    \inferrule*[Right=$\wedge^-$-l$_0$]
    {%
      \inferrule*[Right=$\wedge^-$-l$_1$]
      {\Gamma, A, B \vdash C}
      {\Gamma, A, A \wedge^- B \vdash C}
    }
    {\Gamma, A \wedge^- B, A \wedge^- B \vdash C}
  }
  {\Gamma, A \wedge^- B \vdash C}
\end{mathpar}

\begin{mathpar}
  \inferrule*[right=Contr$^*$,fraction={\cdot\cdots\cdot}]
  {%
    \mprset{defaultfraction}
    \inferrule*[Right=$\wedge^+$-r]
    {\Gamma \vdash A \\ \Gamma \vdash B}
    {\Gamma, \Gamma \vdash A \wedge^+ B}
  }
  {\Gamma \vdash A \wedge^+ B}

  \and

  \inferrule*[right=$\wedge^+$-l]
  {%
    \inferrule*[Right=Weak]
    {\Gamma, A \vdash C}
    {\Gamma, A, B \vdash C}
  }
  {\Gamma, A \wedge^+ B \vdash C}

  \and

  \inferrule*[right=$\wedge^+$-l]
  {%
    \inferrule*[Right=Weak]
    {\Gamma, B \vdash C}
    {\Gamma, A, B \vdash C}
  }
  {\Gamma, A \wedge^+ B \vdash C}
\end{mathpar}

\subsection{The $\oc$-modality}

\begin{mathpar}
  \inferrule*[right=Promotion]
  {\oc\Gamma \vdash A}
  {\oc\Gamma \vdash \oc A}

  \and

  \inferrule*[right=Dereliction]
  {\Gamma, A \vdash B}
  {\Gamma, \oc A \vdash B}

  \and

  \inferrule*[right=Weakening]
  {\Gamma \vdash B}
  {\Gamma, \oc A \vdash B}

  \and

  \inferrule*[right=Contraction]
  {\Gamma, \oc A, \oc A \vdash B}
  {\Gamma, \oc A \vdash B}
\end{mathpar}

In the intuitionistic linear logic sequent calculus ILL, $\oc A$ is defined
to be a proposition whose occurrences as antecedents can be deleted
(\TirName{Weakening}) and duplicated (\TirName{Contraction}), from which we can
extract $A$ (\TirName{Dereliction}), and which we can form from a conclusion
$A$ only when all antecedents are of the form $\oc X$ for some proposition $X$
(\TirName{Promotion}).
In short, $\oc A$ can be seen as an intuitionistic version of $A$, supporting
all of the structural rules of LJ, and only being able to be formed when it
does not depend on anything linear.

While this definition of $\oc$ works, in the sense that it gives the intended
class of models and cut elimination is maintained, it has some disadvantages.
Firstly, while the multiplicative and additive connectives are all characterised
by universal properties, $\oc$ is not.
This can be seen by the fact that taking the rules for $\oc$ and replacing each
occurrence of $\oc$ by a fresh connective $\oc'$ produces a logically distinct
connective.
One cannot produce any derivation of $\oc' A \vdash \oc A$ because
\TirName{Promotion} does not apply when there are antecedents not of the form
$\oc X$.
Finally, while $\oc$ is supposed to be a positive connective, it sometimes
behaves like a negative connective.
For example, for a positive connective like $\otimes$, the normal form proof
of the identity sequent $P \otimes Q \vdash P \otimes Q$ starts (from the
bottom) with a left rule and then, with the left in a more amenable form,
applies the right rule.
In contrast, the normal form proof of $\oc P \vdash \oc P$ starts with the
right rule, as it needs to have everything on the left be of the form $\oc X$.

\begin{mathpar}
  \inferrule*[right=$\otimes$-l]
  {%
    \inferrule*[Right=$\otimes$-r]
    {%
      \inferrule*[right=Id]{ }{P \vdash P} \\
      \inferrule*[Right=Id]{ }{Q \vdash Q}
    }
    {P, Q \vdash P \otimes Q}
  }
  {P \otimes Q \vdash P \otimes Q}

  \and

  \inferrule*[right=Pro]
  {%
    \inferrule*[Right=Der]
    {%
      \inferrule*[Right=Id]{ }{P \vdash P}
    }
    {\oc P \vdash P}
  }
  {\oc P \vdash \oc P}
\end{mathpar}
