Where modal logics can be seen as additions to an underlying classical or
intuitionistic logic, linear logic is a much more radical change.
Formally, many modal logics, including (I)S4, are conservative extensions of
the underlying classical or intuitionistic logic, meaning that statements not
mentioning the modal operators $\Box$ and $\Diamond$ are provable in modal logic
if and only if they are provable in the underlying non-modal logic.
In linear logic, however, we a priori severely restrict provability of basic
formulae, and then use modalities to recover the strength of classical or
intuitionistic logic.

In this thesis, I consider intuitionistic linear logic (ILL).
ILL can be understood as a logic of resources.
Whereas in classical logic we read a proposition $A$ to implicitly mean ``$A$ is
true'', we may read a proposition $A$ of linear logic to implicitly mean ``I
have an $A$''.
A sequent $A \vdash B$ can then be understood as saying ``if I have an $A$, I
can give it up so as to have a $B$'', and the corresponding implication
$A \multimap B$ can be understood as saying ``I have a \emph{mechanism} for
having a $B$ if I have and am willing to give up an $A$''.
\todo{Camels example}
In this section, I will carefully introduce intuitionistic linear logic so as to
make the intuitive readings more precise.

%Linear logic is a good starting point for our investigations into usage
%restrictions because it is well known, well understood, and exhibits most of the
%difficulties we will come across.
%Linear logic's lack of general weakening and contraction will be a typical
%feature of the calculi we will see, and the exponential modality $\oc$ (bang)
%will be the prototype for a range of modalities I will consider.

\subsection{Motivation of linearity}

\todo{Incorporate into chapter introduction.}
Programming applications of linearity include stateful protocols
\citep{Wadler12}, mutually exclusive capabilities (TODO: cite),
and mutation (TODO: cite).
In each of these cases, the act of using up a hypothesis allows us to divide
time into ``before'' and ``after'', with lack of duplication avoiding confusion
over which half we are in, and lack of deletion allowing us to know whether the
corresponding action actually happened.
For example, when modelling mutation, a variable refers to a single state of a
mutable value.
The \texttt{write} operation uses up such a variable, making the old state
inaccessible, and returns a new linear value to represent the new state of the
mutable value.
Such a protocol naturally also supports a \emph{freezing} operation, by which we
relinquish the ability to mutate the value in return for an immutable reference
to the value produced.

\subsection{The multiplicative-additive fragment}

The multiplicative-additive fragment of linear logic (MALL) is the fragment
where all hypotheses are linear (must be used exactly once).
I will extend MALL with the \emph{exponential} modality in
\cref{sec:bang-modality}.
MALL is unable to embed intuitionistic or classical logic, as MALL is unable to
reflect any of the discarding or duplication that can be done in proofs using
weakening or contraction.

In short, the syntax of intuitionistic MALL can be described as intuitionistic
logic with the structural rules of \emph{weakening} and \emph{contraction}
removed.
However, without the presence of weakening and contraction, we have to be more
careful about the rules we state, so as not to accidentally admit weakening
and contraction.
The lack of these structural rules also allows us to observe a new phenomenon:
the distinction (at the level of provability) between \emph{additive} and
\emph{multiplicative} formulations of existing connectives (in particular, the
conjunction connective).

I present MALL in \cref{fig:mall} in a sequent calculus style, as it was
presented by \citet{girard87linear}.

To encode what it means to use a hypothesis \emph{exactly once}, we first need
to decide what counts as a use.
The simplest case is that the identity sequent counts as a single use of its
sole hypothesis, and conversely does not count as a use of any other hypotheses.
For sequential proofs, created by the \TirName{Cut} rule, if we have a proof of
$A$ using $\Gamma$, and a proof of $B$ using $\Delta$ and $A$, then we have a
proof of $B$ transitively using $\Delta$ and $\Gamma$.
The exchange rule Exch says that use is invariant under permutation.

For the logical connectives, we have genuine choices as to what it means to use
them.
Two cases --- disjunction ($\oplus$) and (linear) implication ($\multimap$) ---
are somewhat intuitive from intuitionistic logic.
A canonical proof of a disjunction is a tag and a proof of one of the two
disjuncts.
This suggests that a proof of a disjunction only uses the same hypotheses as
the proof of the disjunct we actually choose, with the other disjunct being
irrelevant.
Correspondingly, when we use a disjunction hypothesis, we will only actually use
one of the cases, so each branch should use the same hypotheses.
For implication, use is sequential like with the \TirName{Cut} rule, and its
left rule is more or less the only choice that allows use of the surrounding
hypotheses.

For conjunction, there are two choices: Either the conjuncts \emph{together} use
all of the hypotheses, or each of the conjuncts \emph{individually} uses all of
the hypotheses.
The former choice gives us the tensor-product ($\otimes$), and the latter choice
gives us the with-product ($\with$).
These products are equivalent up to provability in logics with weakening and
contraction, but distinct in linear logic.

\begin{figure}
  \begin{align*}
    A, B, C &\Coloneqq X \mid I \mid A \otimes B \mid A \multimap B
              \mid 0 \mid A \oplus B \mid \top \mid A \with B \\
    \Gamma, \Delta, \Theta &\Coloneqq {\cdot} \mid \Gamma, A
  \end{align*}
  \begin{mathpar}
    \ebrule{%
      \infer0[Id]{A \vdash A}
    }

    \and

    \ebrule{%
      \hypo{\Gamma \vdash A}
      \hypo{\Delta, A \vdash B}
      \infer2[Cut]{\Gamma, \Delta \vdash B}
    }

    \and

    \ebrule{%
      \hypo{\Gamma, B, A, \Delta \vdash C}
      \infer1[Exch]{\Gamma, A, B, \Delta \vdash C}
    }

    \and

    \ebrule{%
      \hypo{\Gamma \vdash C}
      \infer1[$I$-L]{\Gamma, I \vdash C}
    }

    \and

    \ebrule{%
      \infer0[$I$-R]{{\cdot} \vdash I}
    }

    \and

    \ebrule{%
      \hypo{\Gamma, A, B \vdash C}
      \infer1[$\otimes$-L]{\Gamma, A \otimes B \vdash C}
    }

    \and

    \ebrule{%
      \hypo{\Gamma \vdash A}
      \hypo{\Delta \vdash B}
      \infer2[$\otimes$-R]{\Gamma, \Delta \vdash A \otimes B}
    }

    \and

    \ebrule{%
      \hypo{\Gamma \vdash A}
      \hypo{\Delta, B \vdash C}
      \infer2[$\multimap$-L]{\Gamma, \Delta, A \multimap B \vdash C}
    }

    \and

    \ebrule{%
      \hypo{\Gamma, A \vdash B}
      \infer1[$\multimap$-R]{\Gamma \vdash A \multimap B}
    }

    \and

    \ebrule{%
      \infer0[$0$-L]{\Gamma, 0 \vdash C}
    }

    \and

    \text{(no $0$-R)}

    \and

    \ebrule{%
      \hypo{\Gamma, A \vdash C}
      \hypo{\Gamma, B \vdash C}
      \infer2[$\oplus$-L]{\Gamma, A \oplus B \vdash C}
    }

    \and

    \ebrule{%
      \hypo{\Gamma \vdash A_i}
      \infer1[$\oplus$-R$_i$]{\Gamma \vdash A_0 \oplus A_1}
    }

    \and

    \text{(no $\top$-L)}

    \and

    \ebrule{%
      \infer0[$\top$-R]{\Gamma \vdash \top}
    }

    \and

    \ebrule{%
      \hypo{\Gamma, A_i \vdash C}
      \infer1[$\with$-L$_i$]{\Gamma, A_0 \with A_1 \vdash C}
    }

    \and

    \ebrule{%
      \hypo{\Gamma \vdash A}
      \hypo{\Gamma \vdash B}
      \infer2[$\with$-R]{\Gamma \vdash A \with B}
    }
  \end{mathpar}
  \caption{Multiplicative-additive fragment of linear logic}
  \label{fig:mall}
\end{figure}

Implication ($\multimap$) and the tensor-product ($\otimes$, $I$) comprise the
\emph{multiplicative} fragment, while disjunction ($\oplus$, $0$) and the
with-product ($\with$, $\top$) comprise the \emph{additive} fragment.
Categorically, the additive fragment corresponds to products and coproducts,
while the multiplicative fragment corresponds to multicategorical tensor
products and closure.

\subsection{The $\oc$-modality}\label{sec:bang-modality}

\begin{figure}
  \begin{align*}
    A, B, C &\Coloneqq \dots \mid \oc A
  \end{align*}
  \begin{mathpar}
    \ebrule{%
      \hypo{\oc\Gamma \vdash A}
      \infer1[Promotion]{\oc\Gamma \vdash \oc A}
    }

    \and

    \ebrule{%
      \hypo{\Gamma, A \vdash B}
      \infer1[Dereliction]{\Gamma, \oc A \vdash B}
    }

    \and

    \ebrule{%
      \hypo{\Gamma \vdash B}
      \infer1[Weakening]{\Gamma, \oc A \vdash B}
    }

    \and

    \ebrule{%
      \hypo{\Gamma, \oc A, \oc A \vdash B}
      \infer1[Contraction]{\Gamma, \oc A \vdash B}
    }
  \end{mathpar}
  \caption{The sequent calculus rules for the $\oc$-modality}
  \label{fig:bang-seq}
\end{figure}

\Cref{fig:bang-seq} shows the rules we can add to MALL to get the full sequent
calculus for intuitionistic linear logic (ILL).
In ILL, $\oc A$ is defined
to be a proposition whose occurrences as antecedents can be deleted
(\TirName{Weakening}) and duplicated (\TirName{Contraction}), from which we can
extract $A$ (\TirName{Dereliction}), and which we can form from a conclusion
$A$ only when all antecedents are of the form $\oc X$ for some proposition $X$
(\TirName{Promotion}).
In short, $\oc A$ can be seen as an intuitionistic version of $A$, supporting
all of the structural rules of LJ, and only being able to be formed when it
does not depend on anything linear.

The \TirName{Dereliction} rule is comparable to the \TirName{T} axiom we saw in
\cref{sec:modal}.
The \TirName{Promotion} rule is exactly like the \TirName{Promotion} rule we saw
in \cref{sec:modal}, and for the same reasons we want to avoid having such a
rule in a natural deduction system for intuitionistic linear logic.
Note that \TirName{Promotion} is not problematic in the sequent calculus, at
least insofar as it maintains \TirName{Cut}-elimination.

Additional to the problems with \TirName{Promotion} in a natural deduction
system, defining the $\oc$-modality as in \cref{fig:bang-seq} has the odd
feature that, unlike all of the other connectives of ILL, $\oc$ is not
characterised by a universal property.
This can be seen by the fact that taking the rules for $\oc$ and replacing each
occurrence of $\oc$ by a fresh connective $\oc'$ produces a logically distinct
connective.
One cannot produce any derivation of $\oc' A \vdash \oc A$ because
\TirName{Promotion} does not apply when there are antecedents not of the form
$\oc X$.
This lack of characterisation also holds of the rule and axioms we first gave to
the $\Box$-modality in \cref{fig:S4-axioms}, and essentially any presentation
that does not encorporate the modality into the judgemental structure of the
calculus.

A noteworthy solution to the substitution problem of \TirName{Promotion} is
given by \citet{BBdePH93}.
The solution they give is compatible with the form of sequents introduced above,
but works purely in terms of right-rules like standard natural deduction
calculi.
I give their rules for the $\oc$-modality in \cref{fig:BBdePH}.
The \TirName{Weakening} and \TirName{Contraction} rules are reminiscent of
pattern-matching elimination principles, allowing us to form an inhabitant of
the type of interest ($\oc A$) and then continue producing the originally
desired result ($B$) in an updated context.
However, somewhat unusually, we can choose which of the two ``pattern-matching''
principles to use, depending on whether we want to delete or duplicate the
value of type $\oc A$.
Another elimination-like rule is \TirName{Dereliction}, which straightforwardly
lets us derive $A$ from $\oc A$ similarly to the corresponding sequent calculus
rule from \cref{fig:bang-seq}, but on the right.

\begin{figure}
  \begin{mathpar}
    \ebrule{%
      \hypo{\Delta_1 \vdash \oc A_1}
      \hypo{\cdots}
      \hypo{\Delta_n \vdash \oc A_n}
      \hypo{\oc A_1, \ldots, \oc A_n \vdash B}
      \infer4[Promotion]{\Delta_1, \ldots, \Delta_n \vdash \oc B}
    }
    \and
    \ebrule{%
      \hypo{\Gamma \vdash \oc A}
      \infer1[Dereliction]{\Gamma \vdash A}
    }
    \and
    \ebrule{%
      \hypo{\Gamma \vdash \oc A}
      \hypo{\Delta \vdash B}
      \infer2[Weakening]{\Gamma, \Delta \vdash B}
    }
    \and
    \ebrule{%
      \hypo{\Gamma \vdash \oc A}
      \hypo{\Delta, \oc A, \oc A \vdash B}
      \infer2[Contraction]{\Gamma, \Delta \vdash B}
    }
  \end{mathpar}
  \caption{The \citet{BBdePH93} rules for the $\oc$-modality}
  \label{fig:BBdePH}
\end{figure}

The \citeauthor{BBdePH93} \TirName{Promotion} rule is more formally complex, but
can be understood as follows.
We notice that the problem with the \TirName{Promotion} rule of
\TirName{fig:bang-seq} when added to a natural deduction system is that it
restricts the types of assumptions in the concluding sequent.
In other words, the sequent calculus \TirName{Promotion} rule is acting like
both a right-rule and, problematically, a left-rule, to some degree.
Acting on the context in a natural deduction system risks making substitution
inadmissible, as happens in this case.
Instead, \citeauthor{BBdePH93} paraphrase this restriction on the context as the
construction of a \emph{new} context $\oc A_1, \ldots, \oc A_n$ for the primary
subderivation.
Heuristically, this new \TirName{Promotion} rule admits substitution because any
substitution will pass into the relevant premises deriving $\oc A_1$ to
$\oc A_n$.
More abstractly, we can notice that the first $n$ premises essentially form an
explicit substitution from $\Delta_1, \ldots, \Delta_n$ to
$\oc A_1, \ldots, \oc A_n$, so any further substitution applied
to a \TirName{Promotion}-headed term is precomposed onto this explicit
substitution~\citep{ACCL91}.

The \citeauthor{BBdePH93} \TirName{Promotion} rule is a clever solution to the
substitution problem of (intuitionistic) linear logic, but the resulting system
comes with its own problems.
If we were to write programs in a language directly implementing the
\citeauthor{BBdePH93} calculus, then the \TirName{Promotion} rule would be a
pain point because it makes us rebind all of the $oc$-typed variables we need to
new variables.
Additionally, we have to be explicit everywhere about weakening and contraction,
potentially making working with $\oc$-typed variables much more fiddly than the
corresponding variables would be in a non-substructural programming language.
These syntactic annoyances may be worked around in a realistic implementation by
some new elaboration procedures, but we would prefer not to rely on such
elaboration if there is an acceptable core calculus which more closely matches
the programs we want to write.
Therefore, I investigate such a system, Dual Intuitionistic Linear Logic, in the
following subsection.

\subsection{Dual Intuitionistic Linear Logic}\label{sec:dill}
Dual Intuitionistic Linear Logic (DILL) is a syntax for intuitionistic linear
logic introduced by \citet{Barber1996}.
Its key feature is splitting assumptions into \emph{linear} assumptions and
\emph{intuitionistic} assumptions --- sometimes called the \emph{dual context}
approach.
Intuitionistic assumptions behave like the variables of simply typed
$\lambda$-calculus.
In contrast, the linear assumptions behave as in the linear calculus we saw in
\cref{sec:ill}.
For example, an intuitionistic assumption of $A$ in an instance of the
$\otimes$-introduction rule is automatically copied to both premises.
This contrasts with an assumption of $\oc A$ in the purely linear calculus,
which must first
be contracted into two assumptions, with one going to each premise.
Compared to the modal system we saw in \cref{sec:modal}, linear assumptions
correspond to true assumptions, and intuitionistic assumptions correspond to
valid assumptions.

The new feature when dealing with linear logic, compared to modal logic, is that
linear and intuitionistic assumptions satisfy different structural rules.
In our De Bruijn index style, where every multi-premise rule does the maximal
contraction and weakening occurs in the leafwardmost possible position,
this means that all of the rules have to manage the substructurality of linear
assumptions.
To help manage linear assumptions, I introduce three pieces of notation.

\begin{definition}\label{def:DILL-empty}
  I write $\Gamma~\mathit{int}$ to state that context $\Gamma$ contains only
  intuitionistic assumptions, i.e., no linear assumptions.
\end{definition}

\begin{definition}\label{def:DILL-split}
  I write $\Gamma + \Delta$ to stand for a context combining $\Gamma$ and
  $\Delta$.
  Specifically, the operation is only defined when $\Gamma$ and $\Delta$ contain
  the same intuitionistic assumptions, and in that case the result contains the
  same intuitionistic assumptions.
  The linear assumptions of the result are the disjoint union of the linear
  assumptions of each of $\Gamma$ and $\Delta$.
\end{definition}

\begin{definition}\label{def:DILL-var}
  I write $\Gamma \ni A$, like in \cref{sec:modal}, to stand for the set of
  variables of type $A$ we can get from $\Gamma$.
  Specifically, if $\Gamma$ contains an assumption $x : A~\mathit{int}$, then
  $x$ gives rise to an inhabitant of $\Gamma \ni A$ if and only if
  $\Gamma~\mathit{int}$; and if $\Gamma$ contains an assumption
  $x : A~\mathit{lin}$, then $x$ gives rise to an inhabitant of $\Gamma \ni A$
  if and only if $x$ is the only linear assumption in $\Gamma$.
  These restrictions essentially encode the fact that we do not allow linear
  variables to be discarded via the use of another variable.
\end{definition}

\Citet{Barber1996} uses the notational convenience of dividing linear and
intuitionistic assumptions into two separate contexts, hence the name of the
\emph{dual context} approach.
I choose not to do this so as to draw out the connection with the approach of
\citet{judgmental}, as well as the approach I will take in later chapters.
I also think that it is more instructive to treat the context as a single
unified thing, so that we can clearly see what operations on the context have
to be preserved by environments (for use in renaming, substitution, and other
traversals).

I list the complete rules of DILL in \cref{fig:dill}.
Given \cref{def:DILL-empty,def:DILL-split,def:DILL-var}, the rules for
multiplicative and additive connectives look not too dissimilar to those in
\cref{fig:mall}.
However, their treatment of intuitionistic assumptions (contraction and
weakening automatically, as in non-substructural calculi) is new.
Meanwhile, the rules for the $\oc$-modality are the same as \TirName{$\Box$-I}
and \TirName{$\Box$-E} from \cref{sec:modal}, except for the implicit structural
rules applying to linear assumptions.
For example, where the \TirName{$\Box$-I} rule discards true assumptions, the
\TirName{$\oc$-I} rule cannot discard linear assumptions, but instead requires
that there are no linear assumptions in the context when the rule is applied, so
that no linear assumptions can be used in the subproof.

\begin{figure}
  \begin{align*}
    A, B, C &\Coloneqq X \mid I \mid A \otimes B \mid A \multimap B
              \mid 0 \mid A \oplus B \mid \top \mid A \with B \mid \oc A \\
    \Gamma, \Delta, \Theta &\Coloneqq {\cdot} \mid \Gamma, A~\mathit{lin}
                             \mid \Gamma, A~\mathit{int} \\
    \mathcal S &\Coloneqq \Gamma \vdash A~\mathit{lin}
  \end{align*}
  \begin{mathpar}
    \ebrule{%
      \hypo{\Gamma \ni A}
      \infer1[Var]{\Gamma \vdash A~\mathit{lin}}
    }
    \\
    \ebrule{%
      \hypo{\Gamma \vdash I~\mathit{lin}}
      \hypo{\Delta \vdash C~\mathit{lin}}
      \infer2[$I$-E]{\Gamma + \Delta \vdash C~\mathit{lin}}
    }
    \and
    \ebrule{%
      \hypo{\Gamma~\mathit{int}}
      \infer1[$I$-I]{\Gamma \vdash I~\mathit{lin}}
    }
    \\
    \ebrule{%
      \hypo{\Gamma \vdash A \otimes B~\mathit{lin}}
      \hypo{\Delta, A~\mathit{lin}, B~\mathit{lin} \vdash C~\mathit{lin}}
      \infer2[$I$-E]{\Gamma + \Delta \vdash C~\mathit{lin}}
    }
    \and
    \ebrule{%
      \hypo{\Gamma \vdash A~\mathit{lin}}
      \hypo{\Delta \vdash B~\mathit{lin}}
      \infer2[$\otimes$-I]{\Gamma + \Delta \vdash A \otimes B~\mathit{lin}}
    }
    \\
    \ebrule{%
      \hypo{\Gamma \vdash A \multimap B~\mathit{lin}}
      \hypo{\Delta \vdash A~\mathit{lin}}
      \infer2[$\multimap$-E]{\Gamma + \Delta \vdash B~\mathit{lin}}
    }
    \and
    \ebrule{%
      \hypo{\Gamma, A~\mathit{lin} \vdash B~\mathit{lin}}
      \infer1[$\multimap$-I]{\Gamma \vdash A \multimap B~\mathit{lin}}
    }
    \\
    \ebrule{%
      \hypo{\Gamma \vdash 0~\mathit{lin}}
      \infer1[$0$-E]{\Gamma + \Delta \vdash C~\mathit{lin}}
    }
    \and
    \text{(no \TirName{$0$-I})}
    \\
    \ebrule{%
      \hypo{\Gamma \vdash A \oplus B~\mathit{lin}}
      \hypo{\Delta, A~\mathit{lin} \vdash C~\mathit{lin}}
      \hypo{\Delta, B~\mathit{lin} \vdash C~\mathit{lin}}
      \infer3[$\oplus$-E]{\Gamma + \Delta \vdash C~\mathit{lin}}
    }
    \and
    \ebrule{%
      \hypo{\Gamma \vdash A_i~\mathit{lin}}
      \infer1[$\oplus$-I$_i$]{\Gamma \vdash A_0 \oplus A_1~\mathit{lin}}
    }
    \\
    \text{(no \TirName{$\top$-E})}
    \and
    \ebrule{%
      \infer0[$\top$-I]{\Gamma \vdash \top~\mathit{lin}}
    }
    \\
    \ebrule{%
      \hypo{\Gamma \vdash A_0 \with A_1~\mathit{lin}}
      \infer1[$\with$-E$_i$]{\Gamma \vdash A_i~\mathit{lin}}
    }
    \and
    \ebrule{%
      \hypo{\Gamma \vdash A~\mathit{lin}}
      \hypo{\Gamma \vdash B~\mathit{lin}}
      \infer2[$\with$-I]{\Gamma \vdash A \with B~\mathit{lin}}
    }
    \\
    \ebrule{%
      \hypo{\Gamma \vdash \oc A~\mathit{lin}}
      \hypo{\Delta, A~\mathit{int} \vdash \oc C~\mathit{lin}}
      \infer2[$\oc$-E]{\Gamma + \Delta \vdash C~\mathit{lin}}
    }
    \and
    \ebrule{%
      \hypo{\Gamma \vdash A~\mathit{lin}}
      \hypo{\Gamma~\mathit{int}}
      \infer2[$\oc$-I]{\Gamma \vdash \oc A~\mathit{lin}}
    }
  \end{mathpar}
  \caption{Dual Intuitionistic Linear Logic}
  \label{fig:dill}
\end{figure}

I will not give a direct substitution procedure for DILL like the one I gave for
the modal system of \cref{sec:modal}.
Suffice to say, where environments for the modal calculus had to be preserved by
binding of variables and pruning of all true variables, environments for DILL
have to be preserved by similar plus the operations of
\cref{def:DILL-empty,def:DILL-split}.
These definitions themselves are somewhat technical, and I do not believe that
delving into the further technicalities of how they interact with environments
will provide useful enough intuitions to justify including here.
However, I will revisit the idea of environments for linear calculi in a more
general semiring-annotated setting in \cref{sec:ren-sub-lr}.

Notice that the rules given in \cref{fig:dill} characterise $\oc$ up to
logical equivalence.
The main feature ensuring this property is that each logical rule contains
exactly one occurrence of a logical connective, meaning that each logical rule
is only defining that connective in that place.
For the full characterisation result, we also require local soundness and
completeness of $\oc$, as discussed by \citet{judgmental} in the modal setting.

As a concluding note on DILL, and as motivation for \cref{sec:semirings}, I
consider the extension of DILL to type dependency.
Such an extension has been studied by \citet{Vakar14}, and a similar extension
to linear/non-linear logic~\citep{Benton94} has been studied by \citet{KPB15}.
While these systems are well behaved on their own terms, both conspicuously omit
dependency on linear values.
This limitation means that one is not able to reason inside the language about
linear values --- for example, we cannot state equations between linear values.
The limitation arises from the fact that, in DILL and most other traditional
presentations of linear logic, linear variables go out of scope in any subterms
in which they are not used.
As noted by \citet{McBride16}, appearance in types should not be considered
\emph{use}.
According to DILL, unused linear variables are not in scope, and thus linear
variables are not in scope for types, and thus cannot appear in types.
\Citet{McBride16} develops a system, refined by \citet{Atkey18}, in which unused
linear variables remain in scope, and thus can appear in types.
It is this system, based on semiring annotations on variables, which is
most closely related to the system I present in \cref{sec:semirings}.

