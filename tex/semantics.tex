Having fixed a universe of syntaxes, in which we can build terms, the next thing
to do is to write recursive functions on terms.
With terms being given by an inductive \AgdaSymbol{data} type definition, they
already come with a recursion and an induction principle.
However, these principles do not handle variable-binding, which we have seen
with the fact that we had to write the \AgdaFunction{bind} helper
function for renaming and substitution in \cref{sec:lrsub}.

In this chapter, the central construct is a function \AgdaFunction{semantics}
which, for a $\V$-environment $\rho : \Gamma \env\V \Delta$,
maps a term $M : \Delta \vdash A$ to some semantic value in the type
$\C\,\Gamma\,A$.
This is a direct adaptation of the \AgdaFunction{semantics} function of
\cref{sec:gen-sem}, which has the same kind of action on intuitionistic terms,
given similar operations on $\V$ and $\C$ as what we had earlier.
The \AgdaFunction{semantics} function recurses on the term $M$, updating $\rho$
whenever new variables are bound.
In our usage-aware case, $\rho$ is also updated whenever we come across linear
combinations induced by premise combinators $I^{\sep}$, $\sep$, and
$\gr r \cdot {}$.

This chapter is structured as follows.
I start by giving a quick introduction to linear relations --- a generalisation
of linear maps --- with reference to their use in mechanised algebraic
reasoning, in \cref{sec:lin-rel}.
Using linear relations, I give a functorial \emph{map} operation to a single
layer of syntax in \cref{sec:functorial}.
I then adapt the \AgdaFunction{Kripke} function space to the usage-aware
setting in \cref{sec:kripke}.
Then I apply the \AgdaFunction{Kripke} function space, along with much of the
machinery I have introduced in previous chapters, to give the
\AgdaFunction{semantics} function in \cref{sec:traversal}.
Finally, I give the syntax-generic simultaneous renaming and substitution
operations in \cref{sec:kit-to-sem}.

% Our goal in this section is to define \AgdaFunction{semantics}, a
% recursor that turns a term into a \AgdaBound{$\C$}-value using a
% \AgdaBound{$\V$}-environment, in a type preserving way:\bob{Get rid of
%   ``body'' here}

% \ExecuteMetaData[\Semanticstex]{semantics-type}

% The \AgdaBound{$\V$} and \AgdaBound{$\C$} are \AgdaFunction{OpenFam}s,
% representing the interpretations of variables and terms
% respectively. In \cref{sec:traversal} we will see the data that must
% be provided to make a \AgdaFunction{semantics} for a given
% system. Before that, we must see how to deal with the two complicated
% features of our syntax: the usage annotations (\cref{sec:functorial})
% and variable binding (\cref{sec:kripke}). \todo{fwd ref to where these are used}

\section{Linear relations in Agda}\label{sec:lin-rel}

In \cref{sec:lrkits}, I defined \emph{usage-annotated environments}
(\cref{def:lr-env}).
One component of a usage-annotated environment is a linear map $\gr\Psi$ which,
when applied to the target usage vector, gives a vector compatible with the
source usage vector.

When it comes to mechanisation, I prefer to replace an assertion
``$\grP \leq \grQ\gr\Psi$'', involving a linear map $\gr\Psi$, by an assertion
``$\grP\gr\Psi\grQ$'', where $\gr\Psi$ is now a linear \emph{relation} said to
relate $\grP$ and $\grQ$.
I define linear relations as follows, where the reader may wish to check that
a linear map gives rise to a linear relation via the expression
$\grP \leq \grQ\gr\Psi$.

\begin{definition}\label{def:linear-relation}
  Given a posemiring $\Ann$ and modules $\mathscr M$ and $\mathscr N$ over
  $\Ann$, a \emph{linear relation} between $\mathscr M$ and $\mathscr N$ is
  a relation $\gr\Psi$ between the underlying sets of $\mathscr M$ and
  $\mathscr N$ such that the following properties hold of all
  $\grP, \grPprime, \grPl, \grPr \in \mathscr M$ and all
  $\grQ, \grQprime, \grQl, \grQr \in \mathscr N$.
  \begin{align*}
    \grPprime \leq \grP \land \grP\gr\Psi\grQ \land \grQ \leq \grQprime
    &\implies \grPprime\gr\Psi\grQprime
    \\
    \plr{\exists\grQ.~\grP\gr\Psi\grQ \land \grQ \leq \gr0}
    &\implies \grP \leq \gr0
    \\
    \plr{\exists\grQ.~\grP\gr\Psi\grQ \land \grQ \leq \grQl + \grQr}
    &\implies \plr{\exists\grPl,\grPr.~\grP \leq \grPl + \grPr
      \land \grPl\gr\Psi\grQl \land \grPr\gr\Psi\grQr}
    \\
    \plr{\exists\grQ.~\grP\gr\Psi\grQ \land \grQ \leq \gr r\grQprime}
    &\implies \plr{\exists\grPprime.~\grP \leq \gr r\grPprime
      \land \grPprime\gr\Psi\grQprime}
  \end{align*}
  I write $\mathscr M \rel \mathscr N$ as the type of linear relations between
  $\mathscr M$ and $\mathscr N$.
\end{definition}

Relations have several advantages over functions when doing mechanised algebra
in type theory.
For one, what are compound expressions in functional style --- for example
$x \leq f(y) + g(z)$ --- become collections of simple relationships in
relational style --- for example $\exists v,w.~vfy \land wgz \land
\mathrm{Add}\,x\,v\,w$.
The advantage of this is that we have immediate access to all of the expressions
and subexpressions, and the proofs of the relationships between them.
This means that there is no need for congruence or monotonicity lemmas, and
correspondingly no need to explicitly describe the syntactic context in which
some algebraic manipulation is being applied and we rely less on the unifier.
Another advantage is that one can design relations so that pattern-matching
suggestively decomposes complex relationships.
For example, given $F : \mathscr M \rel \mathscr M'$ and
$G : \mathscr N \rel \mathscr N'$, we can define a relation
$F \oplus G : \mathscr M \oplus \mathscr N \rel \mathscr M' \oplus \mathscr N'$
pointwise, so that a proof of $(x, x')(F \oplus G)(y, y')$ is a proof of $xFy$
together with a proof of $x'Gy'$.
Pattern-matching on such a proof immediately gives us these constituent parts,
whereas proofs of the corresponding statement involving functions would require
using a lemma to get the parts.
There is a dual advantage when producing one of these proofs, where we can
introduce the canonical constructor (for pairs, in this example) rather than
having to find the appropriate lemma.

Relations also have several disadvantages, though I have found that for my use
case, these are outweighed by the advantages.
For example, automated algebraic solvers are better developed for function-based
algebraic expressions, and sometimes the fact that functions satisfy unitality
and associativity up to decidable judgemental equality means that some proofs
can be avoided.
The handling of compound expressions can also be a disadvantage in that it
necessitates lots of new variable names and obscures goal and context displays.
Finally, in predicative systems such as Agda, relations typically live in a
larger universe than the corresponding functions.
In practice, this means quantifying over an extra level variable for each
relation involved in general lemmas.

There are more relations than there are functions, so statements involving
relations are more general than the corresponding statements involving
functions.
However, one part of the development requires functions rather than relations,
so I impose functionality on relations after the fact.
The appropriate notion of functional relation I use is slightly different to the
standard one, in that I take account of the order on the codomain, and thus ask
for the \emph{largest} solution rather than the \emph{unique} solution.

\begin{definition}\label{def:functional-linear-relation}
  A linear relation $\gr\Psi$ between $\mathscr M$ and $\mathscr N$ is
  \emph{(right-to-left) functional} if, for every $\grQ \in \mathscr N$, there
  exists universally a $\grP \in \mathscr M$ such that $\grP\gr\Psi\grQ$.
  Universality means that, for all $\grPprime$ such that $\grPprime\gr\Psi\grQ$,
  we have $\grPprime \leq \grP$ (i.e.\ $\grP$ is the largest solution).
\end{definition}

In Agda code, $\gr\Psi$ becomes \AgdaBound{$\Psi$} and the fact that $\gr\Psi$
relates $\grP$ and $\grQ$ (in this section written $\grP\gr\Psi\grQ$) is
rendered as
\PsiDot{rel}\AgdaSpace{}\AgdaBound{P}\AgdaSpace{}\AgdaBound{Q}.
That $\gr\Psi$ respects the orders on its arguments is given by
\PsiDot{rel-$\leq_m$}, and the various linearity properties are given by
\PsiDot{rel-0$_m$}, \PsiDot{rel-+$_m$}, and \PsiDot{rel-*$_m$}.

\section{A layer of syntax is functorial}\label{sec:functorial}

A basic property of the universe of syntaxes
is that every syntax supports a functorial action on subterms, realised by a
function \AgdaFunction{map-s}.
Its type says that to map a function \AgdaBound{f}
over a layer of syntax, there must be a linear map \AgdaBound{$\Psi$} relating the
domain and codomain usage contexts, and \AgdaBound{f} should be usable
wherever the domain and codomain usage contexts are similarly related by
\AgdaBound{$\Psi$}.

\ExecuteMetaData[\Maptex]{map-s-type}

This generality is needed because usage contexts change between
a term and its immediate subterms---they are decomposed according to the bunched connectives used in the rules.
\AgdaBound{X} and \AgdaBound{Y} are \AgdaFunction{ExtOpenFam}s, with
\AgdaBound{$\Theta$} being the context extension for a subterm (i.e., the
variables newly bound in that subterm).
Unlike usage annotations, types in the contexts \AgdaBound{$\gamma$} and \AgdaBound{$\delta$}, and the conclusion types implicit here, are preserved throughout.
This is the essence of the usage annotation based approach---we use traditional techniques for variable binding, with the usage annotations layered on top.

The heart of \AgdaFunction{map-s} is \AgdaFunction{map-p}, which recursively
works through the structure \AgdaBound{ps} of premises of the rule applied,
acting on each subterm it finds.
Here, particularly in the clauses for \AgdaInductiveConstructor{`$\sep$} and
\AgdaInductiveConstructor{`$\cdot$}, we see why it is not enough for the
function on subterms to apply at usage contexts \AgdaBound{P} and \AgdaBound{Q}
--- rather, it also needs to apply at any similarly related \AgdaBound{P$'$}
and \AgdaBound{Q$'$}.
In the case of \AgdaInductiveConstructor{`$\sep$}, we have that
$\grP \leq \grP_M + \grP_N$, with \AgdaBound{M} and \AgdaBound{N} being
collections of subterms in usage contexts $\grP_M$ and $\grP_N$, respectively.
Linearity of \AgdaBound{$\Psi$} yields $\grQ_M$ and $\grQ_N$ such that
$\grQ \leq \grQ_M + \grQ_N$ and we use \AgdaFunction{map-p} recursively at
$(\grP_M, \grQ_M)$ and $(\grP_N, \grQ_N)$ on \AgdaBound{M} and \AgdaBound{N}.
The cases for \AgdaInductiveConstructor{`$\cdot$} and
\AgdaInductiveConstructor{`$I^*$} are similar, each using a different aspect
of linearity.
In contrast, the cases for \AgdaInductiveConstructor{`$\dot1$} and
\AgdaInductiveConstructor{`$\dot\times$}, which are the only constructors used in fully structural
systems, do not involve any changes in the usage contexts.

The linearity of relation \AgdaBound{$\Psi$} is given by fields
\AgdaField{rel-0$_m$}, \AgdaField{rel-+$_m$}, and \AgdaField{rel-*$_m$} (with
the subscript-m being a mnemonic for \emph{module}, as opposed to scalar).
\todo{Refer to preliminaries}

\ExecuteMetaData[\Maptex]{map-p}

I have also extended \AgdaFunction{map-p} to handle the various
$\Box$-modalities described in \cref{sec:dup-lnl}.
The Agda code for this extension is not particularly readable, so I do not
include it in this document.
However, this extension is notable as the only part of the framework requiring
that the linear relation \AgdaBound{$\Psi$} be functional (i.e., total and
deterministic).

\section{The Kripke function space}\label{sec:kripke}

At this point I introduce a minor generalisation to
\AgdaFunction{OpenFam} and \AgdaFunction{ExtOpenFam} (as defined in
\cref{sec:terms}):
\AgdaBound{I}\AgdaSpace{}\AgdaFunction{---OpenFam} and
\AgdaBound{I}\AgdaSpace{}\AgdaFunction{---ExtOpenFam}.  We obtain the
definition of \AgdaBound{I}\AgdaSpace{}\AgdaFunction{---OpenFam} by
replacing the textual occurrence of \AgdaBound{Ty} by the parameter
\AgdaBound{I}, though there is still reference to the ambient \AgdaBound{Ty}
via \AgdaRecord{Ctx}.
The main value I am interested in \AgdaBound{I} taking, other than
\AgdaBound{Ty}, is \AgdaRecord{Ctx} --- for example, the type family of
$\V$-environments, for a given $\V$, is a
\AgdaRecord{Ctx}\AgdaSpace{}\AgdaFunction{---OpenFam}%
\AgdaSpace{}\AgdaSymbol{\_}.
However, I only use this generalisation much later.
\todo{When?}

\ExecuteMetaData[\Syntaxtex]{dashOpenFam}

The definition
\AgdaFunction{Kripke}\AgdaSpace{}\AgdaBound{$\V$}\AgdaSpace{}\AgdaBound{$\C$}%
\AgdaSpace{}\AgdaBound{$\Delta$} is a kind
of function space that describes a \AgdaBound{$\C$}-value parametrised by
\AgdaBound{$\Delta$}-many additional \AgdaBound{$\V$}-values (all correctly
typed and usage-annotated).
It is used to describe how to go under binders in a
Higher-Order Abstract Syntax style: To go under a binder we must
provide semantic interpretations for all the additional variables.

% When going under binders during a recursion, the context will be extended by some $\Theta$. This means that the current environment must be extended with $\Theta$s-worth of $\V$s

% we need the ability to say that

% Kripke V C is given the extension \Theta

% In \cref{sec:terms}, we defined \AgdaFunction{Scope} to let a
% judgement-indexed family admit context extensions. However, a key
% component of our generic semantic traversal is to make use of the open
% family \AgdaBound{$\V$} of \emph{values}, which are the sort of thing
% we store in an environment.  The definition \AgdaFunction{Kripke}
% gives an alternative to \AgdaFunction{Scope} which interprets the
% newly bound variables via a requirement of $\V$-values rather than
% extra assumptions for the $\C$-computation.

\ExecuteMetaData[\Semanticstex]{Kripke}

\AgdaFunction{Wrap}
is a device that turns any type family into an equivalent type family
that is judgementally injective in its indices, which helps with
Agda's type inference.
It turns the type family into a parametrised
record with a single field \AgdaField{get} whose type is the type in
the body of the $\lambda$-abstraction.
For understanding the meaning of
\AgdaFunction{Kripke}, \AgdaFunction{Wrap} can be ignored.

If $\Delta$ is of the form $\gr{s_1}B_1, \ldots, \gr{s_n}B_n$, then
\ExecuteMetaData[\Snippetstex]{KripkeVCDGA}\ is equivalent to
\ExecuteMetaData[\Snippetstex]{KripkeExpanded}\ by Currying.  That is
to say, the Kripke function is expecting a value for each newly bound
variable, at the multiplicity of its annotation, together with the
resources supporting each of those values. We use the ``magic wand''
function space here to enforce the invariant that the freshly bound
variables have usage annotations that are added to the existing
variables, not shared with them. The use of the
\AgdaFunction{$\Box^r$} modality ensures that we can still use it in
the presence of additional variables introduced by weakening.
\todo{Is wand defined?}

\AgdaFunction{Kripke} is functorial in the \AgdaBound{$\C$} argument,
as witnessed by the \AgdaFunction{mapK$\C$} function, which is essentially
post-composition:

\ExecuteMetaData[\Semanticstex]{mapKC}

% is exemplified by the following construct
% \AgdaFunction{reify}, where we weaken \AgdaBound{$\Gamma$} by a $\gr0$ed-out
% version of \AgdaBound{$\Delta$}.
% The \AgdaBound{$\Delta$} then gets filled in by the $\V$-values.

% \bob{Move this para}
% This means that \AgdaBound{A} in the definition of \AgdaFunction{Kripke} has
% type \AgdaBound{I}, rather than specifically \AgdaBound{Ty}.
% We use this generality later in \AgdaFunction{extend}, setting \AgdaBound{I}
% to \AgdaDatatype{Ctx}.

\section{Semantic traversal}\label{sec:traversal}

We can now state the data required to implement a traversal assigning
semantics to terms. For open families $\V$ and $\C$, interpreting
variables and terms respectively, we assume that $\V$ is renameable
(i.e., that $\sdtstile{}\V A \rightarrowtriangle \Box\plr{\sdtstile{}\V A}$ for
all $A$),
that $\V$ is embeddable in $\C$, and that we have an algebra for a
layer of syntax, where bound variables are handled using the Kripke
function space:

% The aim of this subsection is to give an alternative recursion principle for
% terms that incorporates some of the environment-handling seen in the
% implementations of renaming and substitution.
% The rest of this section assumes the following: a renameable open family
% \AgdaBound{$\V$} that embeds into the open family \AgdaBound{$\C$}, and an
% algebra for a layer of syntax at \AgdaBound{$\C$}.

\ExecuteMetaData[\Semanticstex]{Semantics}

%\ExecuteMetaData[\Semanticstex]{Comp}

We mutually define the action \AgdaFunction{semantics} and its lemma
\AgdaFunction{body}.
The purpose of \AgdaFunction{semantics} is to turn a term into a
\AgdaBound{$\C$}-value using a \AgdaBound{$\V$}-environment and the fields of
\AgdaRecord{Semantics}.
Meanwhile, \AgdaFunction{body} does a similar job, but also deals with
newly bound variables.
In particular, \AgdaFunction{body} takes a term in a context extended by
\AgdaBound{$\Theta$}, and produces a Kripke function from
\AgdaBound{$\V$}-values for \AgdaBound{$\Theta$} to \AgdaBound{$\C$}-values.

\ExecuteMetaData[\Semanticstex]{semantics-type}

To implement the new recursor \AgdaFunction{semantics}, we use the standard
recursor, which in one case gives us a variable \AgdaBound{v}, and in the other
gives us a structure of subterms \AgdaBound{M}, each of which is in an extended
context.
To deal with a variable \AgdaBound{v}, we look it
up in the environment \AgdaBound{$\rho$}, then use the
\AgdaField{$\sem{\text{var}}$} field to map the resulting
\AgdaBound{$\V$}-value to a \AgdaBound{$\C$}-value.
To deal with a structure of subterms \AgdaBound{M}, we use the functoriality of
the syntactic structure to consider each subterm separately.
On a subterm, we apply \AgdaFunction{body}, which amounts to a recursive call
to \AgdaFunction{semantics} with an extended environment.
Recall that \AgdaFunction{relocate} (\cref{thm:env-resize}) adjusts the
environment \AgdaBound{$\rho$} to work in the usage contexts of the subterms.

\ExecuteMetaData[\Semanticstex]{semantics}

For \AgdaFunction{body}, we are given a subterm \AgdaBound{M}, to
which we want to apply \AgdaFunction{semantics}.  To do so, we need an
extended version of the initial environment \AgdaBound{$\rho$}. We
express this as the generation of a Kripke function that produces the
extended environment given interpretations of the fresh variables. We
take \AgdaBound{$\rho$}, which is an environment covering
\AgdaBound{$\Delta$}, and \AgdaBound{$\sigma$}, which is an
environment covering \AgdaBound{$\Theta$}, and glue them together
using the inductive rules for generating environments, after having
renamed \AgdaBound{$\rho$} via \cref{thm:env-ren} to make it fit the
new context \AgdaBound{$\Gamma^+$} (intended to be
\ExecuteMetaData[\Snippetstex]{GT}):

\ExecuteMetaData[\Semanticstex]{extend}

% The best we can achieve without identity environments for \AgdaBound{$\V$} is
% a Kripke function returning an extended environment.
To define \AgdaFunction{body}, we use \AgdaFunction{mapK$\C$} to
post-compose the environment extension by the
\AgdaSymbol{$\lambda$}-function taking an extended environment and
acting with it on \AgdaBound{M}.

\ExecuteMetaData[\Semanticstex]{body}

% \todo{FIX} Under the assumption that \AgdaBound{$\V$} is renameable, we can decompose
% \cref{thm:lr-bind} as
% \AgdaFunction{reify}\AgdaSpace{}\AgdaOperator{\AgdaFunction{$\circ$}}%
% \AgdaSpace{}\AgdaFunction{extend}, with \AgdaFunction{extend} defined below.
% We can think of \AgdaFunction{extend} as our best effort to extend an
% environment by \AgdaBound{$\Theta$} without access to an identity environment
% at \AgdaBound{$\Theta$}.

\AgdaFunction{semantics} is the fundamental lemma of the framework.
With it proven, I move onto corollaries and specific applications.

\section{Reifying the Kripke function space}\label{sec:reify}

A result I will use throughout the rest of this thesis is \emph{reification}.
When we have an index-preserving mapping from usage-checked variables to
$\V$-environments, we can construct environments of the form
$\Gamma \env\V \Gamma$ (identity environments) for all $\Gamma$.
This lets us write the \AgdaFunction{reify} function, which  simplifies our
obligations in giving a \AgdaRecord{Semantics} by coercing
\AgdaFunction{Kripke} functions into just
\AgdaBound{$\C$}-values in an extended context.

\begin{lemma}[\AgdaFunction{reify}]\label{thm:reify}
  If $\V$ is an open family such that there is a function
  $v : {\sqni} \rightarrowtriangle \V$, we get a function of type
  $\mathrm{Kripke}\,\V\,\C \rightarrowtriangle \mathrm{Scope}\,\C$ for any $\C$.
\end{lemma}
\begin{proof}
  Let $b : \mathrm{Kripke}\,\V\,\C\,\Delta\,\Gamma\,A$.
  That is, $b$ is a Kripke function yielding $\C$-computations
  We want to apply $b$ so as to get a $\C\,\plr{\Gamma, \Delta}\,A$.
  Let $\grP\gamma = \Gamma$ and $\grQ\delta = \Delta$.
  The $\Box^r$ in the type of $b$ allows us to reverse-rename $\Gamma$ to
  $\Gamma, \gr0\delta$.
  Then we give the $\wand$-function an argument in context
  $\gr0\gamma, \Delta$, noting that
  $\plr{\Gamma, \gr0\delta} + \plr{\gr0\gamma, \Delta} = \plr{\Gamma, \Delta}$,
  as we wanted from the result.
  The argument needs type $\gr0\gamma, \Delta \env\V \Delta$.
  We produce this via \cref{thm:env-postren} from an environment
  $\rho : \gr0\gamma, \Delta \env\V \gr0\gamma, \Delta$ created using $v$
  and a renaming which is the complement to that used on $\Box^r$.
\end{proof}

All of the $\V$s used in examples in this paper support identity environments.
However, \citet[p.~27]{AACMM21} give some important examples that do not
support identity environments, and thus cannot use \AgdaFunction{reify}
(\cref{thm:reify}).
The feature that causes the lack of support for identity environments is that
a semantics can make use of the fact that only variables of particular kinds
are bound by the syntax.
In the examples of \citeauthor{AACMM21}, a bidirectionally typed language only
binds variables that synthesise their type, as opposed to those whose type is
checked.
The semantics of type-checking and elaboration rely on variables synthesising
their type, so \AgdaBound{$\V$} is chosen to cover only those variables.
Instead of using \AgdaFunction{reify}, we must observe that each syntactic form
only binds such synthesising variables.
Similar phenomena would appear in, say, a call-by-value language where
variables are values (not computations), or a polarised language where
variables always have a polarity matching their type.

\section{Renaming and substitution}\label{sec:kit-to-sem}

The final completely syntax-generic result I present is simultaneous
substitution.
I derive this as I did in the simply typed case in \cref{sec:gen-sem}:
I first show that a syntactic kit can be turned into a semantics, and then by
instantiating the notion of kit for, in turn, renaming and substitution, the
general semantic traversal gives the result we want.

The notion of \AgdaRecord{Kit} is essentially the same as in the simply typed
case, once we allow for changes to the basic definitions of variables, terms,
and environments (in particular, renamings).

\ExecuteMetaData[\Syntactictex]{Kit}

The first two fields of \AgdaRecord{Semantics} derive directly from fields of
\AgdaRecord{Kit}.
Meanwhile, to handle term constructors, we first \AgdaFunction{reify} to get a
collection of traversed subterms, and then use \AgdaInductiveConstructor{`con}
to assemble these subterms into a similarly shaped syntactic form as we started
with.
The \AgdaField{vr} field is used implicitly in \AgdaFunction{reify}, as it is
used to show that $\V$-identity environments exist.

\ExecuteMetaData[\Syntactictex]{kit-to-sem}

The action of a syntactic traversal on logical rules is basically fixed: we
preserve the logical rule and extend the environment with any newly bound
variables according to \AgdaField{vr}.
Meanwhile, the action on variables is relatively unconstrained: we look up the
variable in the environment to get a $\V$-value, then transform that $\V$-value
into a term using \AgdaField{tm}.

The idea of simultaneous renaming is that variables replace variables, whereas
with simultaneous substitution, terms replace variables.
This translates to environments for renaming containing $\sqni$-values
(variables), and environments for substitution containing $\vdash$-values
(terms).

%To implement renaming and substitution for terms, we now just implement
%syntactic kits for variables and terms, respectively.

\ExecuteMetaData[\Syntactictex]{Ren-Kit}

Notice that \AgdaFunction{ren\textasciicircum$\vdash$}, witnessing the fact
that terms are renameable, is a corollary of \AgdaFunction{Ren-Kit}.

\ExecuteMetaData[\Syntactictex]{Sub-Kit}
