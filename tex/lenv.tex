I settle on \cref{def:lr-env}, and prove various properties about it.

\begin{lemma}\label{thm:env-resize}
  Given an environment $\rho : \grP\gamma \env\V \grQ\delta$ and a $\grPprime$
  and a $\grQprime$ such that $\grPprime \leq \grQprime\plr{\rho.\gr\Psi}$,
  there is also an environment of type $\grPprime\gamma \env\V \grQprime\delta$
  with the same linear map and action on variables.
\end{lemma}
\begin{proof}
  The only part of the definition of an environment dependent on $\grP$ or
  $\grQ$ is the constraint $\grP \leq \grQ\gr\Psi$, which we are able to
  replace for $\grPprime$ and $\grQprime$.
\end{proof}

When constructing an environment, we can do so by cases on the shape of the
target context.
We can create an environment into the empty context when all usage annotations
on the source context are $\gr0$.
We can create an environment into a concatenated context when we can additively
split up the annotations of the source context and produce environments into
both halves from the split sources.
We can create an environment into a singleton context when there is a context
$\gr r$ times smaller than the source context in which we can produce a value
of the appropriate type.

\begin{lemma}\label{thm:construct-env}
  We can define all of the following equivalences for any values of the free
  variables, assuming that $\V$ respects subusaging (i.e.,
  $\grPprime \leq \grP \to
  \V\,\grP\gamma \rightarrowtriangle \V\,\grPprime\gamma$).
  \begin{itemize}
    \item $I^{\sep} \leftrightarrowtriangle \plr{{-} \env\V {\cdot}}$
    \item $\plr{{-} \env\V \Delta_l} \sep \plr{{-} \env\V \Delta_r}
      \leftrightarrowtriangle \plr{{-} \env\V \Delta_l, \Delta_r}$
    \item $\gr r \cdot \plr{\V\,(-)\,A}
      \leftrightarrowtriangle \plr{{-} \env\V \gr rA}$
  \end{itemize}
\end{lemma}
\begin{proof}
  There are 6 cases to check.
  Throughout, we write $\Gamma$ as $\grP\gamma$ and $\Delta$ as $\grQ\delta$
  when convenient.
  \begin{description}
    \item[$I^{\sep}(\rightarrowtriangle)$]
      Let $\gr\Psi$ be the unique linear map out of the zero space.
      By assumption and definition, $\grP \leq \gr0 = \grQ\gr\Psi$.
      There are no variables to act upon.
    \item[$I^{\sep}(\leftarrowtriangle)$]
      $\grQ\gr\Psi$ is an empty sum, so if $\grP \leq \grQ\gr\Psi$ then
      $\grP \leq \gr0$.
    \item[$\sep(\rightarrowtriangle)$]
      Let the given environments be $\rho_l : \grPl\gamma \env\V \grQl\delta$
      and $\rho_r : \grPr\gamma \env\V \grQr\delta$, with
      $\grP \leq \grPl + \grPr$.
      Define $\gr\Psi \coloneqq [\rho_l.\gr\Psi, \rho_r.\gr\Psi]$, using the
      coproduct structure of the concatenated vector space.
      We have $\grP \leq \grPl + \grPr \leq
      \grQl\plr{\rho_l.\gr\Psi} + \grQr\plr{\rho_r.\gr\Psi} =
      \begin{pmatrix} \grQl & \grQr \end{pmatrix}\gr\Psi$.
      To act on variables, we are given $\grPprime \leq
      \begin{pmatrix} \gr{\grQ'_l} & \gr{\grQ'_r} \end{pmatrix}\gr\Psi$ and
      $\gr{\grQ'_l}\delta_l, \gr{\grQ'_r}\delta_r \sqni A$.
      Without loss of generality, let us have $\gr{\grQ'_l}\delta_l \sqni A$
      and $\gr{\grQ'_r} \leq \gr0$.
      Thus, $\grPprime \leq
      \gr{\grQ'_l}\plr{\rho_l.\gr\Psi} + \gr{\grQ'_r}\plr{\rho_r.\gr\Psi} \leq
      \gr{\grQ'_l}\plr{\rho_l.\gr\Psi}$,
      and we can act on the variable using $\rho_l$.
    \item[$\sep(\leftarrowtriangle)$]
      Let the unnamed context be $\Gamma$, also written $\grP\gamma$.
      The linear map
      $\gr\Psi : \Ann^{\size{\Delta_l} + \size{\Delta_r}} \to \Ann^{\size\Gamma}$
      splits into
      $\gr\Psi_{\gr l} : \Ann^{\size{\Delta_l}} \to \Ann^{\size\Gamma}
      \coloneqq \alr{\id, 0}; \gr\Psi$ and
      $\gr\Psi_{\gr r} : \Ann^{\size{\Delta_r}} \to \Ann^{\size\Gamma}
      \coloneqq \alr{0, \id}; \gr\Psi$, using the product structure of
      the concatenated vector space.
      Let $\grPl \coloneqq \grQl\gr\Psi_{\gr l}$ and
      $\grPr \coloneqq \grQr\gr\Psi_{\gr r}$, by definition satisfying the
      required constraints.
      For the action on variables, let us consider the left environment (with
      the right environment following symmetrically).
      We are given $\gr{\grP'_l} \leq \gr{\grQ'_l}\gr\Psi_{\gr l}$ and
      $\gr{\grQ'_l}\delta_l \sqni A$.
      From these, we get
      $\gr{\grP'_l} \leq \gr{\grQ'_l}\gr\Psi_{\gr l} =
      \begin{pmatrix} \gr{\grQ'_l} & \gr0 \end{pmatrix}\gr\Psi$ and
      $\gr{\grQ'_l}\delta_l, \gr0\delta_r \sqni A$.
      We can therefore act using the original environment.
    \item[$\cdot(\rightarrowtriangle)$]
      Let $\grP$ and $\grPprime$ be such that $\grP \leq \gr r\grPprime$ and let
      $v : \V\,\grPprime\gamma\,A$.
      Let $\gr\Psi : \Ann \to \Ann^{\size\gamma}
      \coloneqq \gr r\gr' \mapsto \gr r\gr'\grPprime$.
      By definition and the previous assumption, we have
      $\grP \leq \gr r\gr\Psi$.
      When acting on a variable, we have $\grP\gr{''} \leq \gr r\gr'\gr\Psi$
      and $\gr r\gr'A \sqni A'$.
      The latter tells us that $A = A'$ and $\gr r\gr' \leq \gr1$.
      Thus, $\grP\gr{''} \leq \grPprime$.
      Therefore, by subusaging, we may produce a value of type
      $\V\,\grPprime\gamma\,A$, which we can take to be $v$.
    \item[$\cdot(\leftarrowtriangle)$]
      Let us have an environment of type $\grP\gamma \env\V \gr rA$.
      We want to use its action on variables to yield a value.
      To do this, we let $\grPprime \coloneqq \gr1\gr\Psi$, and use this
      equation, together with the fact that we have a variable of type
      $\gr1A \sqni A$, to get a value of type $\V\,\grPprime\gamma\,A$.
      Furthermore, we derive $\grP \leq \gr r\gr\Psi = \gr r\grPprime$, as
      required.
  \end{description}
\end{proof}

We could, as in \cref{def:lr-rec-env}, use these three clauses to define what an
environment is.
However, I find them difficult to work with, as it is often easier to do
linear algebraic proofs separately from the rest of an environment.
For example, for identity and composition of environments, as we are about to
see, our working definition (\cref{def:lr-env})
is easier to use because we can rely on the closure of linear maps under
identity and composition.
Concretely, an inductive construction of, for example, an identity environment
would involve constructing an environment of type
$\grP\gamma, \grQ\delta \env\V \grP\gamma, \grQ\delta$ by constructing
environments of types $\grP\gamma, \gr0\delta \env\V \grP\gamma$ and
$\gr0\gamma, \grQ\delta \env\V \grQ\delta$.
These are not identity environments, so we would have to strengthen the
induction hypothesis.
Similar strengthening of the induction hypothesis would be required for
composition.

One of the primary test cases for environments is simultaneous substitution,
which will look like the \TirName{sub} rule below.
Note that we have taken $\V \coloneqq {\vdash}$ --- i.e.\ that the values
yielded by the environment are terms, namely the terms to be substituted in for
the free variables of the derivation of $\Delta \vdash A$.

\begin{displaymath}
  \begin{prooftree}
    \hypo{\Gamma \env{\vdash} \Delta}
    \hypo{\Delta \vdash A}
    \infer2[sub]{\Gamma \vdash A}
  \end{prooftree}
\end{displaymath}

The admissibility of substitution will be by induction on the derivation of
$\Delta \vdash A$, so we will need to be able to adapt any environment we are
given to work with any possible context of new premises yielded by the rules of
\cref{fig:lr-bunched}.
In the simply typed case, the only change to the context we encountered was the
binding of new variables.
With usage annotations, we furthermore have linear decompositions of the
context, necessitating changes to the environment whenever usage annotations
change.

There are three kinds of linear decompositions we have to deal with: zero,
addition, and scaling; corresponding to bunched connectives $I^*$, $\sep$, and
$\gr r \cdot {}$, respectively.
In each of these three cases, we have a simple preservation lemma, transforming
an environment
of type $\Gamma \env\V \Delta$ and a decomposition of $\Delta$ into a
decomposition of $\Gamma$ and environments for all of the decomposed fragments
of $\Gamma$ and $\Delta$.

\begin{lemma}[environments preserve zero]\label{thm:lr-env-zero}
  Given an environment $\rho : \grP\gamma \env\V \grQ\delta$ such that
  $\grQ \leq \gr 0$, we also have that $\grP \leq \gr 0$.
\end{lemma}
\begin{proof}
  $\grP \leq \grQ\gr\Psi \leq \gr0\gr\Psi = \gr0$, by environment
  compatibility from $\rho$ and monotonicity and linearity of $\gr\Psi$.
\end{proof}

\begin{lemma}[environments preserve addition]\label{thm:lr-env-add}
  Given an environment $\rho : \grP\gamma \env\V \grQ\delta$ such that
  $\grQ \leq \grQl + \grQr$ for some $\grQl$ and $\grQr$, we also have $\grPl$
  and $\grPr$ such that $\grP \leq \grPl + \grPr$ and there are environments
  $\rho_l : \grPl\gamma \env\V \grQl\delta$ and
  $\rho_r : \grPr\gamma \env\V \grQr\delta$.
\end{lemma}
\begin{proof}
  Let $\grPl \coloneqq \grQl\gr\Psi$ and $\grPr \coloneqq \grQr\gr\Psi$.
  Then, $\grP \leq \grQ\gr\Psi \leq \plr{\grQl + \grQr}\gr\Psi =
  \grQl\gr\Psi + \grQr\gr\Psi = \grPl + \grPr$, satisfying the first condition.
  Because clearly $\grPl \leq \grQl\gr\Psi$ and $\grPr \leq \grQr\gr\Psi$,
  applying \cref{thm:env-resize} to $\rho$ gives us the required
  new environments $\rho_l$ and $\rho_r$.
\end{proof}

\begin{lemma}[environments preserve scaling]\label{thm:lr-env-scale}
  Given an environment $\rho : \grP\gamma \env\V \grQ\delta$ such that
  $\grQ \leq \gr r\grQprime$ for some $\grQprime$, we also have a $\grPprime$
  such that $\grP \leq \gr r\grPprime$ and there is an environment
  $\rho' : \grPprime\gamma \env\V \grQprime\delta$.
\end{lemma}
\begin{proof}
  Let $\grPprime \coloneqq \grQprime\gr\Psi$.
  Then, $\grP \leq \grQ\gr\Psi \leq \plr{\gr r\grQprime}\gr\Psi =
  \gr r\plr{\grQprime\gr\Psi} = \gr r\grPprime$, satisfying the first condition.
  Because clearly $\grPprime \leq \grQprime\gr\Psi$,
  applying \cref{thm:env-resize} to $\rho$ gives us the required
  new environment $\rho'$.
\end{proof}

The final change environments need to preserve is the binding of new free
variables.
In \cref{sec:syntactic-kits}, we had the operation \AgdaFunction{bindEnv} for
this purpose in the intuitionistic setting.
There, we relied on $\V$ supporting a map from $\ni$-variables and admitting
weakening.
In the usage-annotated setting, the former requirement is updated to having a
map from usage-checked $\sqni$-variables.
As for the latter requirement, it turns out that we only need $\V$ to admit
weakening by $\gr0$-annotated variables, which is much more reasonable than
general weakening.
\Cref{thm:lr-bind} adapts \AgdaFunction{bindEnv} for the usage-annotated
setting.

\begin{lemma}[\AgdaFunction{bindEnv}]\label{thm:lr-bind}
  Given functions
  ${\swarrow^k} : \forall \Gamma, \grR, \theta.~\grR \leq \gr0 \to
  \V\,\Gamma \rightarrowtriangle \V\,\plr{\Gamma, \grR\theta}$ and
  $\mathrm{vr} : {\sqni} \rightarrowtriangle \V$, we can turn an environment of
  type $\Gamma \env\V \Delta$ into an environment of type
  $\Gamma, \Theta \env\V \Delta, \Theta$ for any context $\Theta$.
\end{lemma}
\begin{proof}
  Let $\grP\gamma \coloneqq \Gamma$, $\grQ\delta \coloneqq \Delta$, and
  $\grR\theta \coloneqq \Theta$.
  Let the new linear map $\gr\Psi\gr' : \Ann^{\size\Delta + \size\Theta} \to
  \Ann^{\size\Gamma + \size\Theta}$ be $\gr\Psi \oplus \gr I$.
  That is, in block matrix notation,
  $\begin{pmatrix} \gr\Psi & \gr0 \\ \gr0 & \gr I \end{pmatrix}$.
  Checking that this linear map fits, we have
  $\begin{pmatrix}\grP & \grR\end{pmatrix}
  \leq \begin{pmatrix}\grQ\gr\Psi & \grR\gr I\end{pmatrix}
  = \begin{pmatrix}\grQ & \grR\end{pmatrix}\plr{\gr\Psi \oplus \gr I}$.
  For the action on variables, we are given vectors $\grPprime$,
  $\grR\gr'_\grP$, $\grQprime$, and $\grR\gr'_\grQ$ such that
  $\begin{pmatrix} \grPprime & \grR\gr'_\grP \end{pmatrix} \leq
  \begin{pmatrix} \grQprime & \grR\gr'_\grQ \end{pmatrix}
  \plr{\gr\Psi \oplus \gr I}$ and we have a variable of type
  $\grQprime\delta, \grR\gr'_\grQ\theta \sqni A$ for some type $A$.
  The constraint on the new vectors reduces to $\grPprime \leq \grQprime\gr\Psi$
  and $\grR\gr'_\grP \leq \grR\gr'_\grQ$.
  From the variable we either have a variable $x$ in $\delta$ with
  $\grQprime \leq \langle x \rvert$ and $\grR\gr'_\grQ \leq \gr0$, or a
  variable $y$ in $\theta$ with $\grQprime \leq \gr0$ and
  $\grR\gr'_\grQ \leq \langle y \rvert$.
  In the former case, the action of the original environment on $x$ gives us a
  $\V$-value in $\grPprime\gamma$, and the $\gr0$-weakening principle
  $\swarrow^k$, noting that $\grR\gr'_\grP \leq \grR\gr'_\grQ \leq \gr0$, gives
  us a $\V$-value in $\grPprime\gamma, \grR\gr'_\grP\theta$.
  In the latter case, we have that
  $\begin{pmatrix} \grPprime & \grR\gr'_\grP \end{pmatrix}
  \leq \begin{pmatrix} \grQprime\gr\Psi & \grR\gr'_\grQ \end{pmatrix}
  \leq \begin{pmatrix} \gr0\gr\Psi & \langle y \rvert \end{pmatrix}
  = \begin{pmatrix} \gr0 & \langle y \rvert \end{pmatrix}
  = \left\langle {\searrow}y \right\rvert$, so $y$ also serves as a
  usage-checked variable in $\grPprime\gamma, \grR\gr'_\grP\theta$.
  From this usage-checked variable, we get a $\V$-value in the same context
  using $\mathrm{vr}$.
\end{proof}

I put together the preceeding pieces to give a syntactic traversal operation over
$\name$ in the following section.
For the rest of this section, I observe some more constructions purely on
environments --- in particular, composition of environments given certain
assumptions about the families of values.

Following \citet{ACU15}, we expect (intuitionistic) ST$\lambda$C syntax to form
a relative monad over $\ni$ seen as a functor from the category of contexts
under renaming to the functor category $\blr{\mathrm{Ty}, \Set}$, where
$\mathrm{Ty}$ is the discrete category of ST$\lambda$C types.
Notice that, given $F,G : \blr{\mathrm{Ty}, \Set}$, a morphism from $F$ to $G$
is a function of type $F \rightarrowtriangle G$ (with naturality being trivial).
Therefore, we expect a relative monad, given as a Kleisli triple, to have a unit
$\eta_\Gamma : \Gamma \ni \plr{-} \rightarrowtriangle \Gamma \vdash \plr{-}$
given by the variable rule, and a Kleisli extension operator
${^*}_{\Gamma,\Delta} :
\plr{\Gamma \ni \plr{-} \rightarrowtriangle \Delta \vdash \plr{-}} \to
\plr{\Gamma \vdash \plr{-} \rightarrowtriangle \Delta \vdash \plr{-}}$
given by substitution.
Composition of substitutions falls out of this framework as Kleisli composition.
However, in the usage-aware case, substitution needs not just a mapping of
variables $f : \Gamma \sqni \plr{-} \rightarrowtriangle \Delta \vdash \plr{-}$,
but rather an environment $\rho : \Delta \env\vdash \Gamma$, as we have already
discussed.
It therefore makes sense for our replacement for the Kleisli extension operator
to similarly take an environment rather than a simple variable mapping.

\Cref{thm:env-comp} below amounts to deriving a modified notion of Kleisli
composition from a modified Kleisli extension.
Additionally, \cref{thm:env-id} is required to turn a monadic unit into an
identity environment.
Both lemmas are stated in terms of general $\U$/$\V$/$\W$-environments, with
some specific examples (e.g.\ for renaming and substitution) below them.

\begin{lemma}[Identity environment]\label{thm:env-id}
  Given a function
  \[
    \mathrm{vr}_{\Gamma'} :
    \Gamma' \sqni \plr{-} \rightarrowtriangle \V\,\Gamma'
  \]
  for any
  $\Gamma$ we have an environment $\mathrm{id} : \Gamma \env\V \Gamma$.
\end{lemma}
\begin{proof}
  Let $\Gamma = \grP\gamma$.
  Let $\gr\Psi$ be the identity map, which clearly satisfies
  $\grP \leq \grP\gr\Psi$.
  When acting on a variable, the inequality $\grPprime \leq \grQprime\gr\Psi$
  means that $\grPprime \leq \grQprime$.
  We are given a variable of type $\grQprime\gamma \sqni A$, which we can
  coerce to a variable of type $\grPprime\gamma \sqni A$, upon which we apply
  $\mathrm{vr}$ to get the required value of type $\V\,\grPprime\gamma\,A$.
\end{proof}

\begin{lemma}[Composition of environments]\label{thm:env-comp}
  Given a function
  \[
    \mathrm{lift}_{\Gamma', \Delta'} :
    \Gamma' \env\U \Delta' \to \V\,\Delta' \rightarrowtriangle \W\,\Gamma'
  \]
  we can compose environments $\rho : \Gamma \env\U \Delta$ and
  $\sigma : \Delta \env\V \Theta$ into an environment
  $\rho \gg \sigma : \Gamma \env\W \Theta$.
\end{lemma}
\begin{proof}
  Let $\Gamma = \grP\gamma$, $\Delta = \grQ\delta$, and $\Theta = \grR\theta$.
  Take $\gr\Psi$ to be the composition $\plr{\sigma.\gr\Psi}\plr{\rho.\gr\Psi}$,
  noting that
  $\grP \leq \grQ\plr{\rho.\gr\Psi} \leq
  (\grR\plr{\sigma.\gr\Psi})\plr{\rho.\gr\Psi} = \grR\gr\Psi$
  thanks to the inequalities yielded by $\sigma$ and $\rho$.
  When acting on a variable, we are given $\grPprime \leq \grRprime\gr\Psi$ and
  a variable $v : \grRprime\theta \sqni A$, and want a value of type
  $\W\,\grPprime\gamma\,A$.
  Let $\grQprime \coloneqq \grRprime\plr{\sigma.\gr\Psi}$, with inequality
  $\grQprime \leq \grRprime\plr{\sigma.\gr\Psi}$ giving us a value
  $\sigma(v) : \V\,\grQprime\delta\,A$.
  We wish to apply $\mathrm{lift}$ to $\sigma(v)$ with
  $\Gamma' \coloneqq \grPprime\gamma$ and $\Delta' \coloneqq \grQprime\delta$ to
  complete the construction of the $\W$-value.
  To do this, we need an environment of type
  $\grPprime\gamma \env\U \grQprime\delta$, which we can get from $\rho$ using
  \cref{thm:env-resize}, noting that
  $\grPprime \leq \grRprime\plr{\sigma.\gr\Psi}\plr{\rho.\gr\Psi} =
  \grQprime\plr{\rho.\gr\Psi}$.
\end{proof}

We can derive the following corollaries as instances of environment composition.

\begin{corollary}[Composition of renamings]\label{thm:ren-comp}
  Given renamings $\rho : \Gamma \env\sqni \Delta$ and
  $\sigma : \Delta \env\sqni \Theta$, we can form their composite
  $\rho; \sigma : \Gamma \env\sqni \Theta$.
\end{corollary}
\begin{proof}
  Take $\U = \V = \W = {\sqni}$ in \cref{thm:env-comp}.
  Then let $\mathrm{lift}\,\rho\,x \coloneqq \rho(x)$.
\end{proof}

\begin{corollary}[Post-composition with a renaming]\label{thm:ren-env-comp}
  Given an environment $\rho : \Gamma \env\U \Delta$ and a renaming
  $\sigma : \Delta \env\sqni \Theta$, we can form their composite
  $\rho; \sigma : \Gamma \env\U \Theta$.
\end{corollary}
\begin{proof}
  As in \cref{thm:ren-comp}.
\end{proof}

\begin{corollary}[Pointwise renaming of an environment]\label{thm:env-ren}
  If $\sdtstile{}\V$ respects renaming, then so does $\env\V$ (on the left).
\end{corollary}
\begin{proof}
  Suppose we have $\rho : \Gamma \env\sqni \Delta$ and
  $\sigma : \Delta \env\V \Theta$.
  We want to compose these via \cref{thm:env-comp} with $U = {\sqni}$ and
  $\V = \W$.
  The function $\mathrm{lift}$ is given exactly by the fact that $\V$ respects
  renaming.
\end{proof}

\begin{corollary}[Composition of substitutions]\label{thm:sub-comp}
  Given substitutions $\rho : \Gamma \env\vdash \Delta$ and
  $\sigma : \Delta \env\vdash \Theta$, we can form their composite
  $\rho; \sigma : \Gamma \env\vdash \Theta$.
\end{corollary}
\begin{proof}
  Take $\U = \V = \W = {\vdash}$ in \cref{thm:env-comp}.
  Then, $\mathrm{lift}$ is given by the action of a substitution on a term
  (see \AgdaFunction{sub} in the following section).
\end{proof}

\begin{corollary}[Composing semantics with substitution]
  If we have a semantics (in the sense of \cref{sec:gen-sem} and
  \cref{sec:traversal}) from $\U$ to $\W$, then from an environment
  $\rho : \Gamma \env\U \Delta$ and a substitution
  $\sigma : \Delta \env\vdash \Theta$, we can form the composite
  $\rho; \sigma : \Gamma \env\W \Theta$.
\end{corollary}

% Concatenation is difficult; save to after I've talked about renamings.

% Finally for this section, we give the conditions under which the
% context-forming operations (empty, concatenation, and singleton) have a
% functorial action with respect to $\V$-environments.
%
% \begin{lemma}
%   For any $\V$, there is an environment ${\cdot} \env\V {\cdot}$.
% \end{lemma}
% \begin{proof}
%   By \cref{thm:construct-env}, it suffices to show $I\,{\cdot}$, which is
%   trivially true.
% \end{proof}
