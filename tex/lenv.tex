\begin{remark}
  Given an environment $\rho : \grP\gamma \env\V \grQ\delta$ and a $\grPprime$
  and a $\grQprime$ such that $\grPprime = \grQprime\plr{\rho.\gr\Psi}$,
  there is also an environment of type $\grPprime\gamma \env\V \grQprime\delta$
  with the same linear map and action on variables.
\end{remark}
\begin{proof}
  The only part of the definition of an environment dependent on $\grP$ or
  $\grQ$ is the constraint $\grP = \grQ\gr\Psi$, which we are able to replace
  for $\grPprime$ and $\grQprime$.
\end{proof}

When constructing an environment, we can do so by cases on the shape of the
target context.
We can create an environment into the empty context when all usage annotations
on the source context are $\gr0$.
We can create an environment into a concatenated context when we can additively
split up the annotations of the source context and produce environments into
both halves from the split sources.
We can create an environment into a singleton context when there is a context
$\gr r$ times smaller than the source context in which we can produce a value
of the appropriate type.

\begin{lemma}\label{thm:construct-env}
  We can define all of the following equivalences for any values of the free
  variables.
  \begin{itemize}
    \item $\forallb{I \dotlr \plr{{-} \env\V {\cdot}}}$
    \item $\forallb{\plr{{-} \env\V \Gamma} \sep \plr{{-} \env\V \Delta}
      \dotlr \plr{{-} \env\V \Gamma, \Delta}}$
    \item
      $\forallb{\gr r \cdot \plr{\V\,(-)\,A} \dotlr \plr{{-} \env\V \gr rA}}$
  \end{itemize}
\end{lemma}
\begin{proof}
  There are 6 cases to check.
  Throughout, we write $\Gamma$ as $\grP\gamma$ and $\Delta$ as $\grQ\delta$
  when convenient.
  \begin{description}
    \item[$I(\to)$]
      Let $\gr\Psi$ be the unique linear map out of the zero space.
      By definition, $\gr0 = \grQ\gr\Psi$.
      There are no variables to act upon.
    \item[$I(\gets)$]
      $\grQ\gr\Psi$ is an empty sum, so if $\grP = \grQ\gr\Psi$ then
      $\grP = \gr0$.
    \item[$\sep(\to)$]
      Let the given environments be $\rho : \grRl\theta \env\V \Gamma$ and
      $\sigma : \grRr\theta \env\V \Delta$, with $\grR = \grRl + \grRr$.
      Define $\gr\Psi \coloneqq [\rho.\gr\Psi, \sigma.\gr\Psi]$, using the
      coproduct structure of the concatenated vector space.
      We have $\grR = \grRl + \grRr =
      \grP\plr{\rho.\gr\Psi} + \grQ\plr{\sigma.\gr\Psi} =
      \plr{\grP, \grQ}\gr\Psi$.
      To act on variables, we are given
      $\grRprime = \plr{\grPprime, \grQprime}\gr\Psi$ and
      $\grPprime\gamma, \grQprime\delta \sqni A$.
      Without loss of generality, let us have $\grPprime\gamma \sqni A$ and
      $\grQprime = \gr0$.
      Thus, $\grRprime = \grPprime\plr{\rho.\gr\Psi}$, and we can act on the
      variable using $\rho$.
    \item[$\sep(\gets)$]
      Let the unnamed context be $\Theta$, also written $\grR\theta$.
      The linear map
      $\gr\Psi : \Ann^{\size\Gamma + \size\Delta} \to \Ann^{\size\Theta}$ splits
      into
      $\gr\Psi_{\gr l} : \Ann^{\size\Gamma} \to \Ann^{\size\Theta}
      \coloneqq \langle \id, 0 \rangle; \gr\Psi$ and
      $\gr\Psi_{\gr r} : \Ann^{\size\Delta} \to \Ann^{\size\Theta}
      \coloneqq \langle 0, \id \rangle; \gr\Psi$, using the product structure of
      the concatenated vector space.
      Let $\grRl \coloneqq \grP\gr\Psi_{\gr l}$ and
      $\grRr \coloneqq \grQ\gr\Psi_{\gr r}$, by definition satisfying the
      required equations.
      For the action on variables, let us consider the left environment.
      We are given $\grRprime = \grPprime\gr\Psi_{\gr l}$ and
      $\grPprime\gamma \sqni A$.
      From these, we get
      $\grRprime = \grPprime\gr\Psi_{\gr l} = \plr{\grPprime, \gr0}\gr\Psi$ and
      $\grPprime\gamma, \gr0\delta \sqni A$.
      We can therefore act using the original environment.
    \item[$\cdot(\to)$]
      Let $\grP$ and $\grPprime$ be such that $\grP = \gr r\grPprime$ and let
      $v : \V\,\grPprime\gamma\,A$.
      Let $\gr\Psi : \Ann \to \Ann^{\size\gamma}
      \coloneqq \gr r\gr' \mapsto \gr r\gr'\grPprime$.
      By definition and the previous assumption, we have $\grP = \gr r\gr\Psi$.
      When acting on a variable, we have $\grP\gr{''} = \gr r\gr'\gr\Psi$
      and $\gr r\gr'A \sqni A'$.
      The latter tells us that $A = A'$ and $\gr r\gr' = \gr1$.
      Thus, $\grP\gr{''} = \grPprime$.
      We therefore need a value of type $\V\,\grPprime\gamma\,A$, which we can
      take to be $v$.
    \item[$\cdot(\gets)$]
      Let us have an environment of type $\grP\gamma \env\V \gr rA$.
      We want to use its action on variables to yield a value.
      To do this, we let $\grPprime \coloneqq \gr1\gr\Psi$, and use this
      equation, together with the fact that we have a variable of type
      $\gr1A \sqni A$, to get a value of type $\V\,\grPprime\gamma\,A$.
      Furthermore, we derive $\grP = \gr r\gr\Psi = \gr r\grPprime$, as
      required.
  \end{description}
\end{proof}

We could, indeed, use these three clauses to define what an environment is.
However, I find them difficult to work with, as it is often easier to do
linear algebraic proofs separately from the rest of an environment.
For identity and composition, as we are about to see, the original definition
is easier to use because we can rely on the identity and composition of linear
maps.
Concretely, an inductive proof of identity would, for example, involve
constructing an environment of type
$\grP\gamma, \grQ\delta \env\V \grP\gamma, \grQ\delta$ by constructing
environments of types $\grP\gamma, \gr0\delta \env\V \grP\gamma$ and
$\gr0\gamma, \grQ\delta \env\V \grQ\delta$.
These are not identity environments, so we would have to strengthen the
induction hypothesis.

One of the primary test cases for environments is simultaneous substitution,
which will look like the following rule.
The admissibility of substitution will be by induction on the derivation of
$\Delta \vdash A$, so we will need to be able to adapt any environment we are
given to work with any possible context of new premises.
In the simply typed case, the only change to the context we encountered was the
binding of new variables.
Now, with usage annotations, we furthermore have linear decompositions of the
context, necessitating changes to the environment whenever usage annotations
change.
I will deal first with linear decompositions.

\begin{displaymath}
  \begin{prooftree}
    \hypo{\Gamma \env{\vdash} \Delta}
    \hypo{\Delta \vdash A}
    \infer2[sub]{\Gamma \vdash A}
  \end{prooftree}
\end{displaymath}

There are three kinds of linear decompositions we have to deal with: zero,
addition, and scaling; corresponding to bunched connectives $I^*$, $\sep$, and
$\gr r \cdot {}$, respectively.
In each case, we have a simple preservation lemma, transforming an environment
of type $\Gamma \env\V \Delta$ and a decomposition of $\Delta$ into a
decomposition of $\Gamma$ and environments for all of the decomposed fragments
of $\Gamma$ and $\Delta$.

\begin{lemma}[environments preserve zero]\label{thm:lr-env-zero}
  Given an environment of type $\grP\gamma \env\V \grQ\delta$ such that
  $\grQ \leq \gr 0$, we also have that $\grP \leq \gr 0$.
\end{lemma}
\begin{proof}
  $\grP \leq \grQ\gr\Psi \leq \gr0\gr\Psi = \gr0$, by environment
  compatibility and monotonicity and linearity of $\gr\Psi$.
\end{proof}

\begin{lemma}[environments preserve addition]\label{thm:lr-env-add}
  Given an environment of type $\grP\gamma \env\V \grQ\delta$ such that
  $\grQ \leq \grQl + \grQr$ for some $\grQl$ and $\grQr$, we also have $\grPl$
  and $\grPr$ such that $\grP \leq \grPl + \grPr$ and there are environments
  of types $\grPl\gamma \env\V \grQl\delta$ and
  $\grPr\gamma \env\V \grQr\delta$.
\end{lemma}
\begin{proof}
  Let $\grPl \coloneqq \grQl\gr\Psi$ and $\grPr \coloneqq \grQr\gr\Psi$.
  Then, $\grP \leq \grQ\gr\Psi \leq \plr{\grQl + \grQr}\gr\Psi =
  \grQl\gr\Psi + \grQr\gr\Psi = \grPl + \grPr$, satisfying the first condition.
  Because clearly $\grPl \leq \grQl\gr\Psi$ and $\grPr \leq \grQr\gr\Psi$,
  \cref{thm:env-resize} on the original environment gives us the required
  pair of new environments.
\end{proof}

\begin{lemma}[environments preserve scaling]\label{thm:lr-env-scale}
  Given an environment of type $\grP\gamma \env\V \grQ\delta$ such that
  $\grQ \leq \gr r\grQprime$ for some $\grQprime$, we also have a $\grPprime$
  such that $\grP \leq \gr r\grPprime$ and there is an environment of type
  $\grPprime\gamma \env\V \grQprime\delta$.
\end{lemma}
\begin{proof}
  Let $\grPprime \coloneqq \grQprime\gr\Psi$.
  Then, $\grP \leq \grQ\gr\Psi \leq \plr{\gr r\grQprime}\gr\Psi =
  \gr r\plr{\grQprime\gr\Psi} = \gr r\grPprime$, satisfying the first condition.
  Because clearly $\grPprime \leq \grQprime\gr\Psi$,
  \cref{thm:env-resize} on the original environment gives us the required
  new environment.
\end{proof}

Finally, I will also take the opportunity to give the bind lemma, allowing
environments to incorporate newly bound variables.
In the intuitionistic case, the bind lemma had two requirements on $\V$: $\V$
admits weakening and we can map variables into $\V$-values.
With usage annotations, the former is unreasonable, but it turns out that we
only need weakening by variables whose usage annotation is less than or equal
to $\gr0$.
The latter stays as-is, with the note that ``variable'' now means a
usage-checked variable.

\begin{lemma}[bind]\label{thm:lr-bind}
  Given functions
  ${\swarrow^k} : \forall \Gamma, \grR, \theta.~\grR \leq \gr0 \to
  \forallb{\V\,\Gamma \dotto \V\,\plr{\Gamma, \grR\theta}}$ and
  $\mathrm{vr} : \forallb{{\sqni} \dotto \V}$, we can turn an environment of
  type $\Gamma \env\V \Delta$ into an environment of type
  $\Gamma, \Theta \env\V \Delta, \Theta$ for any context $\Theta$.
\end{lemma}
\begin{proof}
  Let $\grP\gamma \coloneqq \Gamma$, $\grQ\delta \coloneqq \Delta$, and
  $\grR\theta \coloneqq \Theta$.
  Let the new linear map $\gr\Psi\gr' : \Ann^{\size\Delta + \size\Theta} \to
  \Ann^{\size\Gamma + \size\Theta}$ be $\gr\Psi \oplus \gr I$.
  That is, in block matrix notation,
  $\begin{pmatrix} \gr\Psi & \gr0 \\ \gr0 & \gr I \end{pmatrix}$.
  Checking that this linear map fits, we have
  $\begin{pmatrix}\grP & \grR\end{pmatrix}
  \leq \begin{pmatrix}\grQ\gr\Psi & \grR\gr I\end{pmatrix}
  = \begin{pmatrix}\grQ & \grR\end{pmatrix}\plr{\gr\Psi \oplus \gr I}$.
  For the action on variables, we are given vectors $\grPprime$,
  $\grR\gr'_\grP$, $\grQprime$, and $\grR\gr'_\grQ$ such that
  $\begin{pmatrix} \grPprime & \grR\gr'_\grP \end{pmatrix} \leq
  \begin{pmatrix} \grQprime & \grR\gr'_\grQ \end{pmatrix}
  \plr{\gr\Psi \oplus \gr I}$ and we have a variable of type
  $\grQprime\delta, \grR\gr'_\grQ\theta \sqni A$ for some type $A$.
  The constraint on the new vectors reduces to $\grPprime \leq \grQprime\gr\Psi$
  and $\grR\gr'_\grP \leq \grR\gr'_\grQ$.
  From the variable we either have a variable $x$ in $\delta$ with
  $\grQprime \leq \langle x \rvert$ and $\grR\gr'_\grQ \leq \gr0$, or a
  variable $y$ in $\theta$ with $\grQprime \leq \gr0$ and
  $\grR\gr'_\grQ \leq \langle y \rvert$.
  In the former case, the action of the original environment on $x$ gives us a
  $\V$-value in $\grPprime\gamma$, and the $\gr0$-weakening principle
  $\swarrow^k$, noting that $\grR\gr'_\grP \leq \grR\gr'_\grQ \leq \gr0$, gives
  us a $\V$-value in $\grPprime\gamma, \grR\gr'_\grP\theta$.
  In the latter case, we have that
  $\begin{pmatrix} \grPprime & \grR\gr'_\grP \end{pmatrix}
  \leq \begin{pmatrix} \grQprime\gr\Psi & \grR\gr'_\grQ \end{pmatrix}
  \leq \begin{pmatrix} \gr0\gr\Psi & \langle y \rvert \end{pmatrix}
  = \begin{pmatrix} \gr0 & \langle y \rvert \end{pmatrix}
  = \left\langle {\searrow}y \right\rvert$, so $y$ also serves as a
  usage-checked variable in $\grPprime\gamma, \grR\gr'_\grP\theta$.
  From this usage-checked variable, we get a $\V$-value in the same context
  using $\mathrm{vr}$.
\end{proof}

The requirements for identity and composition of environments look a bit like
the unit and lift of a Kleisli triple.

\begin{lemma}[Identity environment]
  Given a function $\mathrm{vr} : \forallb{{\sqni} \dotto \V}$, for any
  $\Gamma$ we have an environment of type $\Gamma \env\V \Gamma$.
\end{lemma}
\begin{proof}
  Let $\gr\Psi$ be the identity map, which clearly satisfies
  $\grP = \grP\gr\Psi$.
  When acting on a variable, the equation $\grPprime = \grQprime\gr\Psi$ means
  that $\grPprime = \grQprime$, so we want, from a variable of type
  $\grPprime\gamma \sqni A$, a value of type $\V\,\grPprime\gamma\,A$, which
  we can get from $\mathrm{vr}$.
\end{proof}

\begin{lemma}\label{thm:env-comp-lemma}
  Given an environment $\rho : \Gamma \env\U \Delta$ for which we have, for any
  $\grPprime$ and $\grQprime$ such that
  $\grPprime = \grQprime\plr{\rho.\gr\Psi}$, we have a function
  $\mathrm{lift}_\rho :
  \forallb{\V\,\grQprime\delta \dotto \W\,\grPprime\gamma}$,
  we can map environments of type $\Delta \env\V \Theta$ into environments of
  type $\Gamma \env\W \Theta$.
\end{lemma}
\begin{proof}
  Let $\rho$ be as in the statement, and let $\sigma : \Delta \env\V \Theta$.
  For the environment we are constructing, let
  $\gr\Psi \coloneqq \sigma.\gr\Psi; \rho.\gr\Psi$, noting that
  $\grP = \grQ\plr{\rho.\gr\Psi} =
  \plr{\grR\plr{\sigma.\gr\Psi}}\plr{\rho.\gr\Psi}$.
  For the action on variables, we are given $\grPprime = \grRprime\gr\Psi$ with
  $\grRprime\theta \sqni A$.
  We can immediately apply the action of $\sigma$, giving us a value of type
  $\V\,\plr{\grRprime\plr{\sigma.\gr\Psi}}\,A$.
  We note that
  $\grPprime = \plr{\grRprime\plr{\sigma.\gr\Psi}}\plr{\rho.\gr\Psi}$, and
  apply $\mathrm{lift}_\rho$ to get the desired value.
\end{proof}

\begin{corollary}[Composition of environments]
  Given a function
  $\mathrm{lift} : \plr{\rho : \grP\gamma \env\U \grQ\delta} \to
  \forall \grPprime, \grQprime.~\grPprime = \grQprime\plr{\rho.\gr\Psi} \to
  \forallb{\V\,\grQprime\delta \dotto \W\,\grPprime\gamma}$, then we can
  compose environments of types $\Gamma \env\U \Delta$ and
  $\Delta \env\V \Theta$ into an environment of type $\Gamma \env\W \Theta$.
\end{corollary}

\begin{example}
  We can derive the following instances of environment composition.
  \begin{itemize}
    \item If $\U = \V = \W = {\sqni}$, then $\mathrm{lift}$ is given by the
      action of the renaming $\rho$ on variables.
      This allows us to derive composition of renamings.
    \item More generally, if $\V = {\sqni}$ and $\U = \W$, we can still use
      the action of the environment $\rho$.
      This means that renamings post-compose with any other sort of environment.
    \item If $\V = \W = {\vdash}$, then $\mathrm{lift}$ is given by a
      syntactic traversal.
      For example, if $\U = {\sqni}$, we need the action of renaming on terms
      to show that a renaming followed by a substitution composes to a
      substitution.
      If $\U = {\vdash}$, then the action of substitution on terms gives us that
      substitutions compose.
    \item More generally, if $\V = {\vdash}$ and we have a semantics from
      $\U$ to $\W$, then $\mathrm{lift}$ can be given by the semantic traversal
      of terms.
  \end{itemize}
\end{example}

% Concatenation is difficult; save to after I've talked about renamings.

% Finally for this section, we give the conditions under which the
% context-forming operations (empty, concatenation, and singleton) have a
% functorial action with respect to $\V$-environments.
%
% \begin{lemma}
%   For any $\V$, there is an environment ${\cdot} \env\V {\cdot}$.
% \end{lemma}
% \begin{proof}
%   By \autoref{thm:construct-env}, it suffices to show $I\,{\cdot}$, which is
%   trivially true.
% \end{proof}
