In this section, I introduce the syntax of the type theory \name{}, which makes
use of posemiring usage annotations to express the usage restrictions found in
DILL and other calculi.
I use this syntax to write some example programs, which will motivate the
denotational semantics explored in \cref{sec:wrel}.
For the rest of this thesis, particularly
\cref{sec:semirings,sec:ren-sub-lr,sec:wrel},
\name{} will serve as both a prototypical
usage-constrained syntax and a target of semantic analyses.

The calculus \name{} is similar in spirit to intuitionistic linear logic (ILL),
which we saw in \cref{sec:linearity}.
The types of \name{}, listed in \cref{fig:lr-types}, are almost identical
to those of ILL, differing only in the exponential modality $\oc$
(read ``bang'').
In particular, I include distinguished tensor- and with-product types
($\otimes$, $\with$) and their units ($I$, $\top$), function types
($\multimap$), additive sum types and their unit ($\oplus$, $0$), and the
graded modality $\oc_{\gr r}$.
The idea of $\oc_{\gr r}$ is to internalise an annotation of $\gr r$ on a
variable in the context.

\begin{figure}
  \begin{displaymath}
    A, B, C \Coloneqq I \mid A \otimes B \mid A \multimap B \mid \top
    \mid A \with B \mid 0 \mid A \oplus B \mid \oc_{\gr r} A
  \end{displaymath}
  \caption{The types of \name{}}
  \label{fig:lr-types}
\end{figure}

\begin{figure}
  \begin{displaymath}
    \begin{prooftree}
      \hypo{\gamma \ni x : A}
      \hypo{\grP \le \langle x \rvert}
      \infer2[Var]{\grP\gamma \vdash A}
    \end{prooftree}
  \end{displaymath}

  \begin{displaymath}
    \begin{matrix}
      \begin{prooftree}
        \hypo{\grP \le \gr0}
        \infer1[$I$-I]{\grP\gamma \vdash I}
      \end{prooftree}
      &&
      \begin{prooftree}
        \hypo{\grR \le \grP + \grQ}
        \hypo{\grP\gamma \vdash I}
        \hypo{\grQ\gamma \vdash C}
        \infer3[$I$-E]{\grR\gamma \vdash C}
      \end{prooftree}
      \\\\
      \begin{prooftree}
        \hypo{\grR \le \grP + \grQ}
        \hypo{%
          \begin{matrix}
            \grP\gamma \vdash A \\ \grQ\gamma \vdash B
          \end{matrix}%
        }
        \infer2[$\otimes$-I]{\grR\gamma \vdash A \otimes B}
      \end{prooftree}
      &&
      \begin{prooftree}
        \hypo{%
          \begin{matrix}
            \grR \le \grP + \grQ \\ \grP\gamma \vdash A \otimes B
          \end{matrix}%
        }
        \hypo{\grQ\gamma, \gr1A, \gr1B \vdash C}
        \infer2[$\otimes$-E]{\grR\gamma \vdash C}
      \end{prooftree}
      \\\\
      \begin{prooftree}
        \hypo{\grR\gamma, \gr1A \vdash B}
        \infer1[$\multimap$-I]{\grR\gamma \vdash A \multimap B}
      \end{prooftree}
      &&
      \begin{prooftree}
        \hypo{\grR \le \grP + \grQ}
        \hypo{\grP\gamma \vdash A \multimap B}
        \hypo{\grQ\gamma \vdash A}
        \infer3[$\multimap$-E]{\grR\gamma \vdash B}
      \end{prooftree}
      \\\\
      \begin{prooftree}
        \infer0[$\top$-I]{\grR\gamma \vdash \top}
      \end{prooftree}
      &&
      \textrm{(no $\top$-E)}
      \\\\
      \begin{prooftree}
        \hypo{\grR\gamma \vdash A}
        \hypo{\grR\gamma \vdash B}
        \infer2[$\with$-I]{\grR\gamma \vdash A \with B}
      \end{prooftree}
      &&
      \begin{prooftree}
        \hypo{\grR\gamma \vdash A_0 \with A_1}
        \infer1[$\with$-E$_i$, for $i \in \{0,1\}$]{\grR\gamma \vdash A_i}
      \end{prooftree}
      \\\\
      \textrm{(no $0$-I)}
      &&
      \begin{prooftree}
        \hypo{\grR \le \grP + \grQ}
        \hypo{\grP\gamma \vdash 0}
        \infer2[$0$-E]{\grR\gamma \vdash C}
      \end{prooftree}
      \\\\
      \begin{prooftree}
        \hypo{\grR\gamma \vdash A_i}
        \infer1[$\oplus$-I$_i$, for $i \in \{0,1\}$]%
        {\grR\gamma \vdash A_0 \oplus A_1}
      \end{prooftree}
      &&
      \begin{prooftree}
        \hypo{%
          \begin{matrix}
            \grR \le \grP + \grQ \\ \grP\gamma \vdash A \oplus B
          \end{matrix}%
        }
        \hypo{%
          \begin{matrix}
            \grQ\gamma, \gr1A \vdash C \\ \grQ\gamma, \gr1B \vdash C
          \end{matrix}%
        }
        \infer2[$\oplus$-E]{\grR\gamma \vdash C}
      \end{prooftree}
      \\\\
      \begin{prooftree}
        \hypo{\grR \le \gr r\grP}
        \hypo{\grP\gamma \vdash A}
        \infer2[$\oc$-I]{\grR\gamma \vdash \oc\gr rA}
      \end{prooftree}
      &&
      \begin{prooftree}
        \hypo{\grR \le \grP + \grQ}
        \hypo{\grP\gamma \vdash \oc\gr rA}
        \hypo{\grQ\gamma, \gr rA \vdash C}
        \infer3[$\oc$-E]{\grR\gamma \vdash C}
      \end{prooftree}
    \end{matrix}
  \end{displaymath}
  \caption{\name{}}
  \label{fig:lr}
\end{figure}

I will not cover any operational semantics or equational theory of \name{} in
this thesis.
I will discuss a denotational semantics in \cref{sec:wrel}.

The following features are of note.

\paragraph{Subusaging}
Several typing rules contain constraints of the form $\grP \leq \grQ$, for
certain usage vectors $\grP$ and $\grQ$.
We saw subusaging in the introduction to this chapter in the specific case of
$\Ann$ being formed from the poset $\{\gr0 > \gr\omega < \gr1\}$.
This allowed variables annotated $\gr\omega$ (``unrestricted'') to be both
weakened/discarded (because $\gr\omega \leq \gr0$) and derelicted/used
(because $\gr\omega \leq \gr1$).
Subsumption of usage annotations is essential to nearly all interesting choices
of $\Ann$.
However, in the toy example of exact usage counting using the set $\mathbb N$ of
annotations, we set the order to be just equality as a matter of simplicity.

For usage annotations $\gr r$ and $\gr s$, the inequality $\gr r \leq \gr s$
states that an assumption with annotation $\gr r$ can be used wherever an
assumption with annotation $\gr s$ is required.
A mnemonic is that $\gr r$ is less specific than $\gr s$.
The principle is reflected by the admissible subusaging rule, where the order
has been lifted from annotations to usage contexts.
The subusaging rule is a simple corollary of renaming, as given in
\cref{sec:ren-sub-lr}.

\[
  \begin{prooftree}
    \hypo{\grP \leq \grQ}
    \hypo{\grQ\gamma \vdash A}
    \infer2[Subuse]{\grP\gamma \vdash A}
  \end{prooftree}
\]

\paragraph{Tensor- and with-products}
Like intuitionistic linear logic (ILL), \name{} distinguishes tensor-products
($A \otimes B$) from with-products ($A \with B$).
Whereas in ILL, rules like $\otimes$-introduction involve splitting the
assumptions between the two subterms, in \name{}, this splitting is done by
choosing usage annotations for the premises which add up to the usage
annotations of the conclusion.
For example, we can derive $\vdash A \otimes B \multimap B \otimes A$ as
follows.
Notice that the assumption $A \otimes B$ is still present in all subderivations,
even after it has been ``used up''.
The only thing that stops us using the assumption again is that, for a general
choice of $\Ann$, we do not have $\gr0 \leq \gr1$ or $\gr1 \leq \gr1 + \gr1$.

\begin{small}
  \[
    \nabla \coloneqq
    \begin{prooftree}
      \infer0{\plr{\gr0\;\gr1\;\gr1} \leq
        \plr{\gr0\;\gr0\;\gr1} + \plr{\gr0\;\gr1\;\gr0}}
      \infer0{\plr{\gr0\;\gr0\;\gr1} \leq \plr{\gr0\;\gr0\;\gr1}}
      \infer1[Var]{\gr0\plr{A \otimes B}, \gr0A, \gr1B \vdash B}
      \infer0{\plr{\gr0\;\gr1\;\gr0} \leq \plr{\gr0\;\gr1\;\gr0}}
      \infer1[Var]{\gr0\plr{A \otimes B}, \gr1A, \gr0B \vdash A}
      \infer3[$\otimes$-I]%
      {\gr0\plr{A \otimes B}, \gr1A, \gr1B \vdash B \otimes A}
    \end{prooftree}
  \]

  \[
    \begin{prooftree}
      \infer0{\plr{\gr1} \leq \plr{\gr1} + \plr{\gr0}}
      \infer0{\plr{\gr1} \leq \plr{\gr1}}
      \infer1[Var]{\gr1\plr{A \otimes B} \vdash A \otimes B}
      \hypo{\nabla}
      \infer[no rule]1{\gr0\plr{A \otimes B}, \gr1A, \gr1B \vdash B \otimes A}
      \infer3[$\otimes$-E]{\gr1\plr{A \otimes B} \vdash B \otimes A}
      \infer1[$\multimap$-I]{\vdash A \otimes B \multimap B \otimes A}
    \end{prooftree}
  \]
\end{small}

\begin{example}
  Let $A \multimapboth B$ abbreviate
  $\plr{A \multimap B} \with \plr{B \multimap A}$.
  Then the following judgements hold for any partially ordered semiring.
  Derivations are left as an exercise to the reader.
  \begin{itemize}
    \item $\vdash A \oplus A \multimap A$
    \item $\vdash A \multimap A \with A$
    \item $\vdash A \oplus 0 \multimapboth A$
    \item $\vdash A \otimes 0 \multimapboth 0$
    \item $\vdash \oc\gr1A \multimapboth A$
    \item If $\gr r \leq \gr s$, then $\vdash \oc\gr rA \multimap \oc\gr sA$
  \end{itemize}
\end{example}

\begin{example}
  Let $\Ann \coloneqq (\mathbb N, =, 0, +, 1, \times)$, that is, specialise to
  the posemiring made of
  the usual semiring of natural numbers with ordering given by equality.
  Under this discipline, the usage constraints enforce a form of exact usage
  counting.
  The following judgements then hold.
  Derivations are left as an exercise to the reader.
  \begin{itemize}
    \item $\vdash \oc\gr2A \multimap A \otimes A$
    \item $\vdash \oc\gr5A \multimap \oc\gr2A \otimes \oc\gr3A$
  \end{itemize}
\end{example}

\subsection{Other posemirings}\label{sec:example-posemirings}

Now that we have seen the role of usage annotations in $\name$, I will give more
examples of posemirings for tracking interesting usage patterns.

\begin{example}\label{def:trivial-posemiring}
  The singleton set gives rise to a posemiring in a unique way.
  When the usage annotations of $\name$ are taken from this trivial posemiring,
  we recover a version of intuitionistic simply typed $\lambda$-calculus
  featuring redundant connectives $\otimes$ (equivalent to $\with$ in the pure
  setting) and $\oc{\gr*}$ (where $\oc{\gr*} A \simeq A$).
\end{example}

\begin{example}\label{def:monotonicity-posemiring}
  The \emph{monotonicity} posemiring is defined over the set of symbols
  $\{\gr{\wn\wn}, \gr{\uparrow\uparrow}, \gr{\downarrow\downarrow},
  \gr{\sim\sim}\}$.
  The idea is that each symbol represents the possible \emph{variance} of an
  input (free variable) with respect to some partial ordering on a semantic
  domain of elements.
  $\gr{\uparrow\uparrow}$ represents covariance (if that input goes up, the
  output goes up), $\gr{\downarrow\downarrow}$ represents contravariance
  (if that input goes \emph{down}, the output goes up), $\gr{\sim\sim}$ gives no
  guarantees (if that input remains constant, the output (trivially) goes up),
  and $\gr{\wn\wn}$ says that that input is irrelevant (whatever changes are
  made to that input, the output (trivially) goes up).

  I take $0 \coloneqq \gr{\wn\wn}$, $1 \coloneqq \gr{\uparrow\uparrow}$,
  and define the following operations:

  \makebox[\textwidth][s]{
    \begin{tabular}{c|cccc}
      $+$ & $\gr{\wn\wn}$ & $\gr{\uparrow\uparrow}$ & $\gr{\downarrow\downarrow}$ & $\gr{\sim\sim}$ \\ \hline
      $\gr{\wn\wn}$ & $\gr{\wn\wn}$ & $\gr{\uparrow\uparrow}$ & $\gr{\downarrow\downarrow}$  & $\gr{\sim\sim}$ \\
      $\gr{\uparrow\uparrow}$ & $\gr{\uparrow\uparrow}$ & $\gr{\uparrow\uparrow}$ & $\gr{\sim\sim}$ & $\gr{\sim\sim}$ \\
      $\gr{\downarrow\downarrow}$ & $\gr{\downarrow\downarrow}$ & $\gr{\sim\sim}$ & $\gr{\downarrow\downarrow}$ & $\gr{\sim\sim}$ \\
      $\gr{\sim\sim}$ & $\gr{\sim\sim}$  & $\gr{\sim\sim}$ & $\gr{\sim\sim}$ & $\gr{\sim\sim}$ \\
    \end{tabular}
    \begin{tabular}{c|cccc}
      $*$ & $\gr{\wn\wn}$ & $\gr{\uparrow\uparrow}$ & $\gr{\downarrow\downarrow}$ & $\gr{\sim\sim}$ \\ \hline
      $\gr{\wn\wn}$ & $\gr{\wn\wn}$ & $\gr{\wn\wn}$ & $\gr{\wn\wn}$  & $\gr{\wn\wn}$ \\
      $\gr{\uparrow\uparrow}$ & $\gr{\wn\wn}$ & $\gr{\uparrow\uparrow}$ & $\gr{\downarrow\downarrow}$ & $\gr{\sim\sim}$ \\
      $\gr{\downarrow\downarrow}$ & $\gr{\wn\wn}$ & $\gr{\downarrow\downarrow}$ & $\gr{\uparrow\uparrow}$ & $\gr{\sim\sim}$ \\
      $\gr{\sim\sim}$ & $\gr{\wn\wn}$  & $\gr{\sim\sim}$ & $\gr{\sim\sim}$ & $\gr{\sim\sim}$ \\
    \end{tabular}
    \begin{tikzpicture}[baseline]
      \node(omega) at (0,-1) {$\gr{\sim\sim}$};
      \node(0) [above left of=omega] {$\gr{\uparrow\uparrow}$};
      \node(1) [above right of=omega] {$\gr{\downarrow\downarrow}$};
      \node(qq) [above right of=0] {$\gr{\wn\wn}$};

      \draw (omega) -- (0);
      \draw (omega) -- (1);
      \draw (0) -- (qq);
      \draw (1) -- (qq);
    \end{tikzpicture}
  }

  Addition represents an intersection of guarantees.
  For example, if a variable is used covariantly in one subterm and
  contravariantly in another, we can only make the trivial guaratee represented
  by $\gr{\sim\sim}$.
  Multiplication is mainly interesting for multiplication by
  $\gr{\downarrow\downarrow}$, which flips the variance on any other annotation.
  As such, $\oc\gr{\downarrow\downarrow}A$ represents a contravariant $A$.
  The flipping (involutive) behaviour of $\gr{\downarrow\downarrow}$ lets us
  notice that $x$ is covariant in
  a term like $-(-x)$, where $-$ is a constant of type
  $\oc\gr{\downarrow\downarrow}\mathbb Z \multimap \mathbb Z$.

  A similar, but distinct, collection of modalities for monotonicity is given by
  \citet{Arntzenius19}.
\end{example}

\begin{example}\label{def:sensitivity-posemiring}
  The \emph{sensitivity} posemiring~\citep{reed10distance} is given by
  $(\mathbb R^+, \geq, \gr0, +, \gr1, \times)$, where $\mathbb R^+$ is the
  non-negative real numbers extended with $\gr\infty$ (distances), and the rest
  of the structure comes from the standard operations on real numbers (except
  that $\gr0 \times \gr\infty = \gr\infty \times \gr0 = \gr0$).
  Note that the order is reversed, making $\gr\infty$ coercible to any other
  annotation and anything coercible to $\gr0$.

  The intended semantics for these annotations is to interpret types as metric
  spaces and terms as Lipschitz-continuous maps.
  That is to say, each type $A$ comes with a notion of distance $d_A$, and a map
  $f : A \to B$ satisfies
  \[
    \exists\gr r \in \mathbb R^+.\forall x,y.~%
    d_B(f(x), f(y)) \leq \gr rd_A(x, y).
  \]
  I internalise this constant $\gr r$ so that derivations of
  $\gr rA \vdash B$ are interpreted as $\gr r$-Lipschitz-continuous functions.
  If $\gr r = \gr\infty$, then the Lipschitz condition is trivially satisfied,
  meaning that $\gr\infty$-annotated variables can be used without constraint.
  If $\gr r = \gr0$, then $f$ has to be constant, and the corresponding
  syntactic constraint is that a $\gr0$-annotated variable cannot be used.

  Addition forbids general contraction, which would otherwise allow arbitrary
  finite blow-up of the effect of any non-$\gr0$-annotated variable.
  However, the ordering, with $\gr0$ at the top, means that we have general
  weakening, so the resulting sensitivity calculus has an affine flavour.

  Such a system has been applied to sensitivity analysis, a component of
  differential privacy, by \citet{reed10distance}.
\end{example}
