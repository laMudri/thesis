In this section, I introduce the syntax of the type theory \name{}, which makes
use of usage annotations.
I use this syntax to write some example programs, which will motivate the
denotational semantics explored in \cref{sec:wrel}.
For the rest of this thesis, \name{} will serve as both a prototypical
usage-constrained syntax and a target of semantic analyses.

The calculus \name{} is similar in spirit to intuitionistic linear logic (ILL).
The types of \name{}, listed in \cref{fig:lr-types}, are almost identical
to those of ILL, differing only in the exponential modality $\oc$
(read ``bang'').
In particular, I include distinguished tensor- and with-product types
($\otimes$, $\with$) and their units ($I$, $\top$), function types
($\multimap$), additive sum types and their unit ($\oplus$, $0$), and the
graded modality $\oc_{\gr r}$.
The idea of $\oc_{\gr r}$ is to internalise an annotation of $\gr r$ on a
variable in the context.
In this position, an assumption of type $\oc_{\gr r} A$ acts like an assumption
of type $A$ that is to be used according to $\gr r$ rather than the standard
$\gr1$.

\begin{figure}
  \begin{displaymath}
    A, B, C \Coloneqq I \mid A \otimes B \mid A \multimap B \mid \top
    \mid A \with B \mid 0 \mid A \oplus B \mid \oc_{\gr r} A
  \end{displaymath}
  \caption{The types of \name{}}
  \label{fig:lr-types}
\end{figure}

\begin{figure}
  \begin{displaymath}
    \begin{prooftree}
      \hypo{\gamma \ni x : A}
      \hypo{\grP \le \langle x \rvert}
      \infer2[Var]{\grP\gamma \vdash A}
    \end{prooftree}
  \end{displaymath}

  \begin{displaymath}
    \begin{matrix}
      \begin{prooftree}
        \hypo{\grP \le \gr0}
        \infer1[$I$-I]{\grP\gamma \vdash I}
      \end{prooftree}
      &&
      \begin{prooftree}
        \hypo{\grR \le \grP + \grQ}
        \hypo{\grP\gamma \vdash I}
        \hypo{\grQ\gamma \vdash C}
        \infer3[$I$-E]{\grR\gamma \vdash C}
      \end{prooftree}
      \\\\
      \begin{prooftree}
        \hypo{\grR \le \grP + \grQ}
        \hypo{%
          \begin{matrix}
            \grP\gamma \vdash A \\ \grQ\gamma \vdash B
          \end{matrix}%
        }
        \infer2[$\otimes$-I]{\grR\gamma \vdash A \otimes B}
      \end{prooftree}
      &&
      \begin{prooftree}
        \hypo{%
          \begin{matrix}
            \grR \le \grP + \grQ \\ \grP\gamma \vdash A \otimes B
          \end{matrix}%
        }
        \hypo{\grQ\gamma, \gr1A, \gr1B \vdash C}
        \infer2[$\otimes$-E]{\grR\gamma \vdash C}
      \end{prooftree}
      \\\\
      \begin{prooftree}
        \hypo{\grR\gamma, \gr1A \vdash B}
        \infer1[$\multimap$-I]{\grR\gamma \vdash A \multimap B}
      \end{prooftree}
      &&
      \begin{prooftree}
        \hypo{\grR \le \grP + \grQ}
        \hypo{\grP\gamma \vdash A \multimap B}
        \hypo{\grQ\gamma \vdash A}
        \infer3[$\multimap$-E]{\grR\gamma \vdash B}
      \end{prooftree}
      \\\\
      \begin{prooftree}
        \infer0[$\top$-I]{\grR\gamma \vdash \top}
      \end{prooftree}
      &&
      \textrm{(no $\top$-E)}
      \\\\
      \begin{prooftree}
        \hypo{\grR\gamma \vdash A}
        \hypo{\grR\gamma \vdash B}
        \infer2[$\with$-I]{\grR\gamma \vdash A \with B}
      \end{prooftree}
      &&
      \begin{prooftree}
        \hypo{\grR\gamma \vdash A_0 \with A_1}
        \infer1[$\with$-E$_i$, for $i \in \{0,1\}$]{\grR\gamma \vdash A_i}
      \end{prooftree}
      \\\\
      \textrm{(no $0$-I)}
      &&
      \begin{prooftree}
        \hypo{\grR \le \grP + \grQ}
        \hypo{\grP\gamma \vdash 0}
        \infer2[$0$-E]{\grR\gamma \vdash C}
      \end{prooftree}
      \\\\
      \begin{prooftree}
        \hypo{\grR\gamma \vdash A_i}
        \infer1[$\oplus$-I$_i$, for $i \in \{0,1\}$]%
        {\grR\gamma \vdash A_0 \oplus A_1}
      \end{prooftree}
      &&
      \begin{prooftree}
        \hypo{%
          \begin{matrix}
            \grR \le \grP + \grQ \\ \grP\gamma \vdash A \oplus B
          \end{matrix}%
        }
        \hypo{%
          \begin{matrix}
            \grQ\gamma, \gr1A \vdash C \\ \grQ\gamma, \gr1B \vdash C
          \end{matrix}%
        }
        \infer2[$\oplus$-E]{\grR\gamma \vdash C}
      \end{prooftree}
      \\\\
      \begin{prooftree}
        \hypo{\grR \le \gr r\grP}
        \hypo{\grP\gamma \vdash A}
        \infer2[$\oc$-I]{\grR\gamma \vdash \oc\gr rA}
      \end{prooftree}
      &&
      \begin{prooftree}
        \hypo{\grR \le \grP + \grQ}
        \hypo{\grP\gamma \vdash \oc\gr rA}
        \hypo{\grQ\gamma, \gr rA \vdash C}
        \infer3[$\oc$-E]{\grR\gamma \vdash C}
      \end{prooftree}
    \end{matrix}
  \end{displaymath}
  \caption{\name{}}
  \label{fig:lr}
\end{figure}

I will not cover the operational semantics or equational theory of \name{} in
this thesis.
I will discuss a denotational semantics in \cref{sec:wrel}.

The following features are of note.

\paragraph{Subusaging}
Several typing rules contain constraints of the form $\grP \leq \grQ$, for
certain usage vectors $\grP$ and $\grQ$.
To a first approximation, $\leq$ can simply be read as $=$.
The point of using $\leq$ rather than $=$ is to allow for \emph{subusaging},
i.e., subsumption of usage annotations.
For usage annotations $\gr r$ and $\gr s$, the inequality $\gr r \leq \gr s$
states that an assumption with annotation $\gr r$ can be used wherever an
assumption with annotation $\gr s$ is required.
A mnemonic is that $\gr r$ is less specific than $\gr s$.
The principle is reflected by the admissible subusaging rule.

\[
  \begin{prooftree}
    \hypo{\grP \leq \grQ}
    \hypo{\grQ\gamma \vdash A}
    \infer2[Subuse]{\grP\gamma \vdash A}
  \end{prooftree}
\]

Subusaging is essential to nearly all interesting choices of $\Ann$.
For example, we can capture intuitionistic linear logic by choosing $\Ann$ to
be $\{\gr0 > \gr\omega < \gr1\}$.
This allows variables annotated $\gr\omega$ (``unrestricted'') to be both
weakened/discarded (because $\gr\omega \leq \gr0$) and derelicted/used
(because $\gr\omega \leq \gr1$).

\paragraph{Tensor- and with-products}
Like intuitionistic linear logic (ILL), \name{} distinguishes tensor-products
($A \otimes B$) from with-products ($A \with B$).
Whereas in ILL, rules like $\otimes$-introduction involve splitting the
assumptions between the two subterms, in \name{}, this splitting is done by
choosing usage annotations which add up to the usage annotations of the
conclusion.
For example, we can derive $\vdash A \otimes B \multimap B \otimes A$ as
follows.
Notice that the assumption $A \otimes B$ is still present in all subderivations,
even after it has been ``used up''.
The only thing that stops us using the assumption again is that, for a general
choice of $\Ann$, we do not have $\gr0 \leq \gr1$ or $\gr1 \leq \gr1 + \gr1$.

\begin{small}
  \[
    \nabla \coloneqq
    \begin{prooftree}
      \infer0{\plr{\gr0\;\gr1\;\gr1} \leq
        \plr{\gr0\;\gr0\;\gr1} + \plr{\gr0\;\gr1\;\gr0}}
      \infer0{\plr{\gr0\;\gr0\;\gr1} \leq \plr{\gr0\;\gr0\;\gr1}}
      \infer1[Var]{\gr0\plr{A \otimes B}, \gr0A, \gr1B \vdash B}
      \infer0{\plr{\gr0\;\gr1\;\gr0} \leq \plr{\gr0\;\gr1\;\gr0}}
      \infer1[Var]{\gr0\plr{A \otimes B}, \gr1A, \gr0B \vdash A}
      \infer3[$\otimes$-I]%
      {\gr0\plr{A \otimes B}, \gr1A, \gr1B \vdash B \otimes A}
    \end{prooftree}
  \]

  \[
    \begin{prooftree}
      \infer0{\plr{\gr1} \leq \plr{\gr1} + \plr{\gr0}}
      \infer0{\plr{\gr1} \leq \plr{\gr1}}
      \infer1[Var]{\gr1\plr{A \otimes B} \vdash A \otimes B}
      \hypo{\nabla}
      \infer[no rule]1{\gr0\plr{A \otimes B}, \gr1A, \gr1B \vdash B \otimes A}
      \infer3[$\otimes$-E]{\gr1\plr{A \otimes B} \vdash B \otimes A}
      \infer1[$\multimap$-I]{\vdash A \otimes B \multimap B \otimes A}
    \end{prooftree}
  \]
\end{small}

\begin{example}
  The following judgements hold for any partially ordered semiring.
  Derivations are left as an exercise to the reader.
  \begin{itemize}
    \item $\vdash A \oplus A \multimap A$
    \item $\vdash A \multimap A \with A$
    \item $\vdash A \oplus 0 \multimapboth A$
    \item $\vdash A \otimes 0 \multimapboth 0$
    \item $\vdash \oc\gr1A \multimapboth A$
    \item If $\gr r \leq \gr s$, then $\vdash \oc\gr rA \multimap \oc\gr sA$
  \end{itemize}
\end{example}

To get a feeling for \name{}, I will temporarily fix
$\Ann \coloneqq (\mathbb N, =, 0, +, 1, \times)$.
That is, the usual semiring of natural numbers with ordering given by equality.
Under this discipline, the usage constraints enforce a form of exact usage
counting.

\begin{example}
  The following judgements hold for the natural number semiring.
  Derivations are left as an exercise to the reader.
  \begin{itemize}
    \item $\vdash \oc\gr2A \multimap A \otimes A$
    \item $\vdash \oc\gr5A \multimap \oc\gr2A \otimes \oc\gr3A$
  \end{itemize}
\end{example}
