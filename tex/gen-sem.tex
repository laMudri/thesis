\def\SimpleSem{../agda/processed-latex/SimpleSem.tex}

The traversal \AgdaFunction{trav} from the last section is generic in the sense
that $\V$, the type of entries in an environment, can be instantiated to many
different things.
However, in practice we only use $\ni$ and $\vdash$, giving us renaming and
substitution, respectively.
This is because \AgdaFunction{trav} only targets terms, and does so by keeping
term constructors intact and replacing only the variables by things from the
environment.
This makes substitution the most general possible traversal.

If we want to capture a broader range of traversals, including not just
syntactic but also \emph{semantic} operations, we must be able to target things
other than terms, and act in an interesting way on term constructors.
\Citet{ACMM17} show how to do this generalisation of \AgdaFunction{trav} to a
function \AgdaFunction{sem} with the following type, where
\AgdaBound{$\C$} is the type family we are targeting.
In this section, I discuss the assumptions needed to implement such a function.

\ExecuteMetaData[\SimpleSem]{semType}

Following the implementation of \AgdaFunction{trav}, we see that
\AgdaBound{$\C$} will need to support a semantic counterpart of each syntactic
form (\AgdaInductiveConstructor{var}, \AgdaInductiveConstructor{app}, and
\AgdaInductiveConstructor{lam}).
With syntactic kits, we already asked for the field \AgdaField{tm} to interpret
\AgdaBound{$\V$}-values as terms.
We rename \AgdaField{tm} to \AgdaField{⟦var⟧} to reflect its role in the
semantic traversal \AgdaFunction{sem}.
Now, we will also ask for fields to replace the right-hand side applications of
the other term constructors.
For application, we can stick with the obvious thing: we should be able to
combine a semantic function and its semantic argument to get the semantic
result.

\ExecuteMetaData[\SimpleSem]{semVarApp}

However, we want to treat binding constructs specially, particularly because
there are semantics with no notion of binding.
We instead provide a function from values to computations that works in any
\emph{extension} of the current context.
Keeping \AgdaField{$\swarrow^k$} as before, we get the following semantic
replacement for \AgdaRecord{Kit}.

\ExecuteMetaData[\SimpleSem]{SemanticsExplicit}

With the aim of abstracting away from explicit contexts, bringing us closer to
natural deduction, we can use some new notation to rephrase these requirements.
We will work in \AgdaRecord{Ctx}\AgdaSpace{}\AgdaSymbol{$\to$}\AgdaSpace{}%
\AgdaPrimitive{Set} rather than \AgdaPrimitive{Set}.
One of the basic connectives in this setting is the \emph{pointwise} arrow
\AgdaFunction{\_$\dotto$\_}, which acts in
\AgdaRecord{Ctx}\AgdaSpace{}\AgdaSymbol{$\to$}\AgdaSpace{}\AgdaPrimitive{Set}
like the non-dependent arrow does in \AgdaPrimitive{Set}.
Another basic component is the \AgdaFunction{$\forall[\_]$} notation, which
embeds \AgdaRecord{Ctx}\AgdaSpace{}\AgdaSymbol{$\to$}\AgdaSpace{}%
\AgdaPrimitive{Set} into \AgdaPrimitive{Set} by using an implicit $\Pi$-type
to quantify over \emph{all} contexts.
Finally, at this stage, I introduce a modality \AgdaFunction{$\bigcirc$}
encapsulating the pattern of considering arbitrary \emph{extensions} of a
context.
To facilitate working in this point-free setting, I give infix versions of
the families \AgdaBound{$\V$} and \AgdaBound{$\C$} (respectively
\AgdaFunction{\_$\V\vDash$\_} and \AgdaFunction{\_$\C\vDash$\_}).
The principal use of these aliases is to fill the right argument with a type
(occurring explicitly), and leave the left argument as \AgdaFunction{\_}, i.e.,
a context given through the point-free machinery.

\ExecuteMetaData[\SimpleSem]{Circle}

\ExecuteMetaData[\SimpleSem]{SemanticsCircle}

To illustrate this definition, I will discuss a syntactic traversal ---
renaming --- and a semantic traversal --- a standard $\mathrm{Set}$ semantics.

For the renaming semantics, as with the renaming kit, we specify that
environments hold variables (\AgdaDatatype{\_$\ni$\_}) and show that variables
satisfy the required form of weakening (\AgdaFunction{$\swarrow^v$}).
Meanwhile, whereas all syntactic kits target terms
(\AgdaDatatype{\_$\vdash$\_}), with a semantic traversal we must specify the
target.
The fields \AgdaField{$\llbracket$var$\rrbracket$} and
\AgdaField{$\llbracket$app$\rrbracket$} follow straightforwardly, with variables
embedding into terms and a pair of terms of the right types giving an
application term in the same context, via the relevant constructors.
For the \AgdaField{$\llbracket$lam$\rrbracket$} case, we are given
\ExecuteMetaData[\SimpleSem]{bRenSemCircle}, and after producing a
\AgdaInductiveConstructor{lam}, are left needing a term in
\ExecuteMetaData[\SimpleSem]{resRenSemCircle}.
That the type of \AgdaBound{b} is wrapped in \AgdaFunction{$\bigcirc$} gives
us the ability to use \AgdaBound{b} in the extended context
\ExecuteMetaData[\SimpleKits]{GA}.
In particular, we point at the new variable to yield the desired term in the
same context.

\ExecuteMetaData[\SimpleSem]{RenSemCircle}

To produce a $\mathrm{Set}$ semantics, we shift from targeting terms to
targeting the interpretation of terms.
In particular, \ExecuteMetaData[\SimpleSem]{semGA}\ is the type of functions
from the interpretation of \AgdaBound{$\Gamma$} to the interpretation of
\AgdaBound{A}.
The interpretation of a type is defined as usual, by recursion on the structure
of the type.
The interpretation of a context is the interpretation for each of its types.
We still have environments storing variables, which delays the interpretation of
variables to the \AgdaField{$\llbracket$var$\rrbracket$} case and allows newly
bound variables to be referred to directly as variables, rather than fetching
them up-front from an environment of interpretations.
In the \AgdaField{$\llbracket$app$\rrbracket$} case, we have
\ExecuteMetaData[\SimpleSem]{mSetSemCircle}\ and
\ExecuteMetaData[\SimpleSem]{nSetSemCircle}, and combine them in the usual way
by distributing the interpretation of the context
\ExecuteMetaData[\SimpleSem]{gammaSetSemCircle}.
The \AgdaField{$\llbracket$lam$\rrbracket$} case involves the same placement of
the new variable into the environment as in \AgdaFunction{RenSem}.
Finally, we get the function \AgdaFunction{set} from terms to their
interpretations by passing in the identity $\ni$-environment \AgdaFunction{id}.

\ExecuteMetaData[\SimpleSem]{SetSemCircle}

The definition of \AgdaRecord{Semantics} above essentially enforces that the
term being traversed and the result of the traversal share the same binding
structure.
Concretely, \AgdaInductiveConstructor{lam} is the only case where we can bind
new variables, and at that point we must do exactly one binding.
This is fine for renaming and substitution, which preserve the binding
structure, and also for a standard denotational semantics, which is sufficiently
abstracted from binding.
However, if we want to do other syntactic translations --- for example,
converting from a syntax with $n$-ary functions to a syntax with only unary
Curried functions --- it would be useful to allow more choices when going under
a binder.
To this end, I replace the one-step binding modality \AgdaFunction{$\bigcirc$}
by the all-possible-renamings modality \AgdaFunction{$\Box$}.
\AgdaFunction{$\Box$}\AgdaSpace{}\AgdaBound{T}\AgdaSpace{}\AgdaBound{$\Gamma$}
states that \AgdaBound{T} holds not only at \AgdaBound{$\Gamma$}, but also at
any context \AgdaBound{$\Gamma^+$} containing \AgdaBound{$\Gamma$} (including
\ExecuteMetaData[\SimpleKits]{GD}\ for any \AgdaBound{$\Delta$}).

\ExecuteMetaData[\SimpleSem]{Box}

As well as the flexibility in binding structure, the \AgdaFunction{$\Box$}
modality allows us to use the somewhat more well behaved and standard
relation of renaming, rather than strict context extension.
The resulting definition of \AgdaRecord{Semantics} is as follows, and is
simply a version of the previous definition where \AgdaFunction{$\bigcirc$}
has been replaced by \AgdaFunction{$\Box$}.
It will become apparent when we implement the traversal \AgdaFunction{sem} why
the first field also changes to include a \AgdaFunction{$\Box$}.

\ExecuteMetaData[\SimpleSem]{Semantics}

Writing a \AgdaFunction{$\Box$}-based semantics is very similar to writing a
\AgdaFunction{$\bigcirc$}-based semantics, so I will only give one further
example.
I generalise the renaming example to derive a semantic traversal from any
syntactic traversal.
We need a slightly modified definition of \AgdaRecord{Kit} to provide
\emph{renaming} of \AgdaBound{$\V$}-values, rather than just extension.

\ExecuteMetaData[\SimpleSem]{Kit}

The interesting feature of the corresponding \AgdaRecord{Semantics} is that
we now pass \AgdaBound{b} the renaming \AgdaFunction{$\swarrow^v$}, projecting
the original context \AgdaBound{$\Gamma$} out of the
\AgdaInductiveConstructor{lam}-extended context
\ExecuteMetaData[\SimpleKits]{GA}.

\ExecuteMetaData[\SimpleSem]{kit}

With \AgdaRecord{Semantics} fixed, I also give the corresponding implementation
of \AgdaFunction{sem}.
Like \AgdaFunction{trav} from \cref{sec:syntactic-kits}, we need a
\AgdaFunction{bindEnv} function, but I have updated this to deal with renamings
and the \AgdaFunction{$\Box$}-operator, and also to fit better with concepts
important for generic \emph{syntax} (as in \cref{sec:gen-syn}).
\AgdaFunction{bindEnv} now takes the old environment \AgdaBound{$\rho$}%
\AgdaSpace{}\AgdaSymbol{:}\AgdaSpace{}\AgdaFunction{[}\AgdaSpace{}%
\AgdaBound{$\V$}\AgdaSpace{}\AgdaFunction{]}\AgdaSpace{}\AgdaBound{$\Gamma$}%
\AgdaSpace{}\AgdaFunction{$\Rightarrow^e$}\AgdaSpace{}\AgdaBound{$\Delta$}, a
renaming \AgdaBound{r}\AgdaSpace{}\AgdaSymbol{:}\AgdaSpace{}%
\AgdaFunction{Ren}\AgdaSpace{}\AgdaBound{$\Gamma^{+}$}\AgdaSpace{}%
\AgdaBound{$\Gamma$}, and an environment \AgdaBound{$\sigma$} in the renamed
context \AgdaBound{$\Gamma^{+}$} targeting the context extension
\AgdaBound{$\Delta$r}.
In the \AgdaInductiveConstructor{lam} case of \AgdaFunction{sem}, the
\AgdaField{$\llbracket$lam$\rrbracket$} field gives us exactly the renaming to
pass to \AgdaFunction{bindEnv}, while the extension environment
\AgdaBound{$\sigma$} is made as a singleton environment containing just the
\AgdaBound{A}-value \AgdaBound{v}.

\ExecuteMetaData[\SimpleSem]{sem}
\ExecuteMetaData[\SimpleSem]{bindEnv}
