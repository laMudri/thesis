In a seminal paper~\cite{Gentzen64}, Gerhard Gentzen introduces two syntactic
paradigms which remain among the most studied to this day.
These paradigms are \emph{natural deduction} and \emph{sequent calculus}, as
exemplified by natural deduction calculi NJ and NK, and sequent calculi LJ and
LK.
Restricting attention to just the intuitionistic systems NJ and LJ, these
actually differ in two orthogonal ways, which I shall prise apart in this
section.
The simpler distinction is that the logical rules in NJ are introduction and
elimination rules, whereas the logical rules in LJ are left and right rules.
But the more important distinction for this thesis is that where NJ has
assumptions, LJ has sequents explictly manipulated by structural rules.
I take the latter distinction to define natural deduction and sequent calculus,
and wherever I need to make the former distinction I shall speak of
\emph{intro-elim systems} and \emph{left-right systems}.

I will use the rest of this section as follows.
First, I introduce NJ and LJ, and some of their basic metatheory.
Then, I will give examples of systems intermediate between Gentzen's natural
deduction and sequent calculi: a left-right natural deduction calculus
$\mu\tilde\mu$, and an intro-elim sequent calculus BBdPH\@.
Both of these examples will reappear in later chapters.

\subsection{Intro-elim natural deduction: NJ}

\subsection{Left-right sequent calculus: LJ}

\subsection{Left-right natural deduction: $\mu\tilde\mu$}
The $\mu\tilde\mu$-calculus~\cite{CH00} (also known as
$\overline\lambda\mu\tilde\mu$ or system L, and closely related to Wadler's
dual calculus~\cite{Wadler03}) can be seen as an adaptation of natural deduction
to classical logic.
Though originally presented as a sequent calculus, the underlying natural
deduction calculus was later given by Herbelin~\cite[p.\ 12]{Herbelin-hab}, and
I will follow the latter.
While Gentzen gave a natural deduction calculus NK for classical logic, NK
relies on adding the \emph{axiom} of excluded middle.
As axioms are not systematic parts of the calculus, they can hinder or break
metatheoretic properties like normalisation.
In contrast, the $\mu\tilde\mu$-calculus allows us to \emph{derive} excluded
middle from entirely systematic components.

In NJ, a derivation of $A$ from assumptions $\Gamma$ tells us that if each
formula in $\Gamma$ is \emph{true}, then $A$ is also \emph{true}.
The $\mu\tilde\mu$-calculus generalises the picture by allowing us to have
both \emph{true} and \emph{false} assumptions, and allowing us to conclude that
some $A$ is \emph{true}, that some $A$ is \emph{false}, or that we have reached
a contradiction.
% A similar judgement of contradiction appears in Prawitz' classical natural
% deduction calculus~\cite{Prawitz65}.
Following Herbelin, we notate the judgement that $A$ is true as ${}\vdash A$,
that $A$ is false as $A \vdash{}$, and of contradiction as $\vdash$.
The only way to derive a contradiction is to find some $A$ such that
${}\vdash A$ and $A \vdash{}$.
Meanwhile, we can derive ${}\vdash A$ by assuming $A \vdash{}$ and deriving
$\vdash$, i.e., we can prove $A$ by assuming that $A$ is false and deriving a
contradiction.
Dually, we can derive $A \vdash{}$ by assuming ${}\vdash A$ and deriving
$\vdash$.
These three methods of derivation are encoded in the following rules.

\begin{mathpar}
  \inferrule*[right=Cut]
  {{}\vdash A \\ A \vdash{}}
  {\vdash}

  \and

  \inferrule*[right=$\mu$]
  {
    [A \vdash{}] \\\\ \vdots \\\\ \vdash
    %\inferrule*[fraction={~~~}]
    %{[A \vdash{}] \\\\ \vdots}
    %{\vdash}
  }
  {{}\vdash A}

  \and

  \inferrule*[right=$\tilde\mu$]
  {[{}\vdash A] \\\\ \vdots \\\\ \vdash}
  {A \vdash{}}
\end{mathpar}

The rules for logical connectives describe how to \emph{prove} and how to
\emph{refute} a formula whose principal connective is that connective.
These correspond strongly with the right and left rules, respectively, of LJ,
and for this reason, $\mu\tilde\mu$ is usually described elsewhere as a
sequent calculus.
For example, we could choose the following rules for disjunction.
To prove $A \vee B$, we can assume that both $A$ and $B$ are false, and derive
a contradiction.
To refute $A \vee B$, we can refute $A$ and $B$ separately.

\begin{mathpar}
  \inferrule*[right=$\vee$-r]
  {[A \vdash{}][B \vdash{}] \\\\ \vdots \\\\ \vdash}
  {{}\vdash A \vee B}

  \and

  \inferrule*[right=$\vee$-l]
  {A \vdash{} \\ B \vdash{}}
  {A \vee B \vdash{}}
\end{mathpar}

\subsection{Intro-elim sequent calculus: BBdPH}

Term assignment system introduced in~\cite{BBdPH93}.
