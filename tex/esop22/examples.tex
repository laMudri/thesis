In this section, we provide three example uses of semantic traversals: generic
renaming and substitution, a usage elaborator, and a denotational semantics.
The reader is also encouraged to see the far greater range of examples in the
work of \citet{AACMM21}, which should adapt to our usage-annotated setting.
Renaming and substitution are essential results, while the latter two examples
focus on usage annotations.

A result we will use throughout this section is \emph{reification}.
When we have an index-preserving mapping from usage-checked variables to
$\V$-environments, we can construct environments of the form
$\Gamma \env\V \Gamma$ (identity environments) for all $\Gamma$.
This lets us write the \AgdaFunction{reify} function, which  simplifies our
obligations in giving a \AgdaRecord{Semantics} by coercing
\AgdaFunction{Kripke} functions into just
\AgdaBound{$\C$}-values in an extended context.

%To show this, we instantiate the Kripke function with the renaming
%$\swarrow^r : \Gamma, \gr0\delta \env\sqni \Gamma$ to extend the scope, and
%pass it an argument environment
%$\plr{\id^\V; \searrow^r} : \gr0\gamma, \Delta \env\V \Delta$ to fill in the
%extended part.
%Post-composition of a renaming onto an environment comes from
%\cref{thm:env-postren}.

%\ExecuteMetaData[\Syntactictex]{reify}

\begin{lemma}[\AgdaFunction{reify}]\label{thm:reify}
  If $\V$ is an open family such that there is a function
  $v : \forallb{{\sqni} \dotto \V}$, we get a function of type
  $\forallb{\mathrm{Kripke}\,\V\,\C \dotto \mathrm{Scope}\,\C}$ for any $\C$.
\end{lemma}
\begin{proof}
  Let $b : \mathrm{Kripke}\,\V\,\C\,\Delta\,\Gamma\,A$.
  That is, $b$ is a Kripke function yielding $\C$-computations
  We want to apply $b$ so as to get a $\C\,\plr{\Gamma, \Delta}\,A$.
  Let $\grP\gamma = \Gamma$ and $\grQ\delta = \Delta$.
  The $\Box^r$ in the type of $b$ allows us to reverse-rename $\Gamma$ to
  $\Gamma, \gr0\delta$.
  Then we give the $\wand$-function an argument in context
  $\gr0\gamma, \Delta$, noting that
  $\plr{\Gamma, \gr0\delta} + \plr{\gr0\gamma, \Delta} = \plr{\Gamma, \Delta}$,
  as we wanted from the result.
  The argument needs type $\gr0\gamma, \Delta \env\V \Delta$.
  We produce this via \cref{thm:env-postren} from an environment
  $\rho : \gr0\gamma, \Delta \env\V \gr0\gamma, \Delta$ created using $v$
  and a renaming which is the complement to that used on $\Box^r$.
\end{proof}

All of the $\V$s used in examples in this paper support identity environments.
However, \citet[p.~27]{AACMM21} give some important examples that do not
support identity environments, and thus cannot use \AgdaFunction{reify}
(\cref{thm:reify}).
The feature that causes the lack of support for identity environments is that
a semantics can make use of the fact that only variables of particular kinds
are bound by the syntax.
In the examples of \citeauthor{AACMM21}, a bidirectionally typed language only
binds variables that synthesise their type, as opposed to those whose type is
checked.
The semantics of type-checking and elaboration rely on variables synthesising
their type, so \AgdaBound{$\V$} is chosen to cover only those variables.
Instead of using \AgdaFunction{reify}, we must observe that each syntactic form
only binds such synthesising variables.
Similar phenomena would appear in, say, a call-by-value language where
variables are values (not computations), or a polarised language where
variables always have a polarity matching their type.


\subsection{Renaming and substitution}\label{sec:kits}

In an unpublished note, \citet{McBride05} gives a parametrised traversal
yielding homomorphisms of syntax, the canonical examples of which are
simultaneous renaming and simultaneous substitution.
The parameters are collected in the record \AgdaRecord{Kit}.
We make a minor change to the original presentation, where instead of our
\AgdaField{ren\textasciicircum{}$\V$} field, \citeauthor{McBride05} has the
field \AgdaField{wk} allowing only context extensions.
As for the other two fields, \AgdaField{vr} allows us to map variables to
$\V$-values, so as to put newly bound variables in environments; and
\AgdaField{tm} allows us to extract terms from $\V$-values, as required when
we use the environment to handle a free variable.

\ExecuteMetaData[\Syntactictex]{Kit}

Where \citeauthor{McBride05} gave the traversal explicitly, we go via our
generic semantic traversal.
The first two fields of \AgdaRecord{Semantics} derive directly from fields of
\AgdaRecord{Kit}.
Meanwhile, to handle term constructors, we first \AgdaFunction{reify} to get a
collection of traversed subterms, and then use \AgdaInductiveConstructor{`con}
to assemble these subterms into a similarly shaped syntactic form as we started
with.
The \AgdaField{vr} field is used implicitly in \AgdaFunction{reify}, as it is
used to show that $\V$-identity environments exist.

\ExecuteMetaData[\Syntactictex]{kit-to-sem}

The action of a syntactic traversal on logical rules is basically fixed: we
preserve the logical rule and extend the environment with any newly bound
variables according to \AgdaField{vr}.
Meanwhile, the action on variables is relatively unconstrained: we look up the
variable in the environment to get a $\V$-value, then transform that $\V$-value
into a term using \AgdaField{tm}.

The idea of simultaneous renaming is that variables replace variables, whereas
with simultaneous substitution, terms replace variables.
This translates to environments for renaming containing $\sqni$-values
(variables), and environments for substitution containing $\vdash$-values
(terms).

%To implement renaming and substitution for terms, we now just implement
%syntactic kits for variables and terms, respectively.

\ExecuteMetaData[\Syntactictex]{Ren-Kit}

Notice that \AgdaFunction{ren\textasciicircum$\vdash$}, witnessing the fact
that terms are renameable, is a corollary of \AgdaFunction{Ren-Kit}.

\ExecuteMetaData[\Syntactictex]{Sub-Kit}

\subsection{A usage elaborator}\label{sec:usage-elaborator}

Using the constructs we have seen so far, producing example terms soon becomes
extremely tedious.
We can achieve some abbreviation by using pattern synonyms to wrap around
\AgdaInductiveConstructor{`con} expressions, but we still have to
produce essentially bespoke proofs whenever we use a usage-sensitive part of the
syntax.
The size of each of these proofs is roughly proportional to the number of free
variables, so the amount of proof we have to write grows roughly quadratically
with the size of terms.
An additional factor, which we can't see on paper, is that type checking time
for these proofs soon becomes prohibitive to interactive development.

Our aim in this subsection is to automate usage constraint proofs, making terms
both easier to write and more performant to check.
We invoke the automation by writing terms in a syntax where usage constraints
have been trivialised, and then use a semantic traversal over the simplified
syntax to try to produce a fully elaborated term in the original syntax.
We write the automation in a way that is generic in the syntax description, thus
avoiding repetition and facilitating the prototyping of new type systems.

The type of syntax descriptions depends on the type of usage annotations because
of variable binding.
For example, in the $\oc{\gr r}$-E rule of \cref{fig:lr-comb}, the right
premise binds a new variable with annotation $\gr r$, where $\gr r$ is drawn
from the ambient posemiring.
The scaling combinator also makes direct reference to the posemiring.
To produce a simplified syntax description, where usage constraints are
trivialised, we set the ambient posemiring to the 1-element $\mathbf0$
posemiring.
In contrast to syntax descriptions, even though types can contain usage
annotations, the type of types does not depend on the type of usage annotations.
This means that, in our simplified syntax, terms have types from the original
system even though variables have trivial usage annotations.
We define the $\mathbf0$ posemiring as follows, being careful to use the
0-field record type \AgdaRecord{$\top$} so that everything algebraic gets
solved by Agda's $\eta$-laws.
Indeed, in this very definition, all of the semiring operations and laws are
canonically inferred.

\ExecuteMetaData[\UsageChecktex]{0-poSemiring}

The elaboration process is monadic.
In particular, we use the \AgdaDatatype{List}/non-determinism monad to give
\emph{all} of the possible annotation choices on the free variables of a term.
We believe the commitment to multiple solutions is inherent when the syntax
contains \AgdaInductiveConstructor{`$\dot1$}.
For example, in the intermediate stages of elaborating
$\plr{\vdash \lambda x.~\plr{*,*}} : A \multimap \top \otimes \top$ with a
usage counting posemiring (assuming reasonable rules for $\top$ and $\otimes$),
it is unclear whether to use the variable $x$ in the left $*$ or the right $*$.
This uncertainty should be reflected in the final result.

The non-deterministic choices we make during elaboration are enumerated by
the fields of \AgdaRecord{NonDetInverses}.
These choices are driven by the typing rules and a candidate usage vector for
the conclusion.
For example, \AgdaField{+$^{-1}$}\AgdaSpace{}\AgdaBound{r} is needed when we
encounter a \AgdaInductiveConstructor{`$\sep$} in the syntax and the candidate
usage annotation we are considering is \AgdaBound{r}.
Then, \AgdaField{+$^{-1}$}\AgdaSpace{}\AgdaBound{r} is a list of pairs of
annotations \AgdaBound{p} and \AgdaBound{q} that \AgdaBound{r} can split into,
together with a proof of the splitting.
For \AgdaField{0\#$^{-1}$} and \AgdaField{1\#$^{-1}$}, inverses to constants,
we are given the candidate \AgdaBound{r} and typically return an empty list if
the constraint cannot be satisfied, or a singleton list containing a proof.
\AgdaField{*$^{-1}$} is used when we encounter scaling, in which case we know
both the scaling factor \AgdaBound{r} (from the syntax description) and the
candidate \AgdaBound{q}.
These inverse operations combine monadically (in fact, applicatively) to give
inverses to the vector operations of zero, addition, scaling, and basis.

\ExecuteMetaData[\UsageChecktex]{NonDetInverses}

We choose the \AgdaBound{$\V$} of our semantics to be (unannotated) variables.
For the \AgdaBound{$\C$}, we consider \emph{functions} from candidate usage
vectors \AgdaBound{R} to the list of elaborated derivations with usage
annotations given by \AgdaBound{R}.
The protocol this encodes is that the user will provide an unannotated term
together with a candidate usage context \AgdaBound{R}, and usage elaboration
returns a list of possible ways the term could be annotated such that the
conclusion has usage context \AgdaBound{R}.
The module name \AgdaModule{U} refers to the fact that we are taking the
ambient posemiring to be $\mathbf0$ in \AgdaFunction{OpenFam}.
The effect on \AgdaFunction{OpenFam} is that the usage annotations of any
contexts we consider are uninformative (hence the \AgdaSymbol{\_} on the left).

\ExecuteMetaData[\UsageChecktex]{C}

To traverse the unannotated terms, we produce a \AgdaRecord{Semantics} over the
unannotated system \AgdaFunction{uSystem}\AgdaSpace{}\AgdaBound{sys}.
To write it, we make use of idiom brackets
\AgdaSymbol{(|}\AgdaSpace{}\AgdaSymbol{$\ldots$}\AgdaSpace{}\AgdaSymbol{|)},
which have the effect of replacing top-level spines of applications by
(\AgdaDatatype{List}-)applicative applications.
Field by field, we already know that variables are renameable.
To interpret a variable, we consider all the possible proofs that such a
variable could be well annotated, and package them up as a variable term via
the applicative machinery.
Finally, for compound terms, we first reify the unannotated subterms, and then
combine the subterms via a \AgdaFunction{lemma}.

\ExecuteMetaData[\UsageChecktex]{elab-sem}

The \AgdaFunction{lemma} essentially goes through the shape of the premises,
combining the collections of subterms in the natural way.
For example, at each
\AgdaInductiveConstructor{\AgdaUnderscore{}$\dottimes$\AgdaUnderscore{}},
we take the Cartesian product of the possibilities of each half, and at each
\AgdaInductiveConstructor{\AgdaUnderscore{}$\sep$\AgdaUnderscore{}},
we non-deterministically split the usage annotations coming in, and then take
the Cartesian product.
When it comes to newly bound variables, the \emph{syntax description} tells us
their annotations, so there is no further non-determinism introduced here.

\ExecuteMetaData[\UsageChecktex]{lemma-type}

To actually use \AgdaFunction{elab-sem} on terms, we take the associated
\AgdaFunction{semantics} and pass it the identity environment (an identity
\emph{renaming} in this case, because $\V$ is a family of variables).
We use \AgdaFunction{elab-unique}, which further
checks statically that exactly one derivation is returned.
The candidate usage vector \AgdaBound{R} will be \AgdaFunction{[]} for closed
terms, and otherwise we have to supply the intended usage annotations.

We can now use the elaborator to automatically infer the usage
annotations for the example at the end of \cref{sec:terms}. This
allows us to write:
\ExecuteMetaData[\PaperExamplestex]{cojoin}
We have instantiated the usage elaborator so that:
\AgdaField{0\#$^{-1}$} is a singleton on $\gr0$ and $\gr\omega$, and
empty on $\gr1$; \AgdaField{1\#$^{-1}$} is a singleton on $\gr1$ and
$\gr\omega$, and empty on $\gr0$; \AgdaField{+$^{-1}$} gives $\gr0
\mapsto [(\gr0,\gr0)]$, $\gr1 \mapsto [(\gr0,\gr1),(\gr1,\gr0)]$, and
$\gr\omega \mapsto [(\gr\omega,\gr\omega)]$; and \AgdaField{*$^{-1}$}
gives $(\gr\omega, \gr0) \mapsto [\gr0]$, $(\gr\omega, \gr1) \mapsto
[]$, and $(\gr\omega, \gr\omega) \mapsto [\gr\omega]$ (omitting
$(\gr0, \_)$ and $(\gr1, \_)$ cases for brevity). Note that we do not
consider splitting $\gr\omega$ up as, say, $\gr1 + \gr\omega$, because
this splitting would introduce more non-determinism but not allow any
more terms to be typed. As such, the only non-determinism comes when
we have variables annotated $\gr1$ and need to do an additive split,
like when we apply the \AgdaInductiveConstructor{!E} rule below. At
this point, the variable can become either $\gr0$-annotated in the
left subterm and $\gr1$-annotated on the right, or vice-versa. We will
find that, because the left subterm wants to use that variable, the
former choice will be rejected. The function \AgdaFunction{var\#} is a
convenience for converting statically known natural numbers,
representing de Bruijn \emph{levels}, into variable terms.


\subsection{A denotational semantics}

To justify the name \emph{semantics}, we give an example traversal that is a
denotational semantics in the usual sense.
The semantics we take is a refinement of that of \citet{AbelBernardy2020},
which gives a way to extract parametricity theorems from substructurally typed
programs.
Example theorems are that all linear terms act as permutations on some fixed
set of resources, and all monotonically typed terms are really monotonic in the
way the typing suggests they are.

To abbreviate this section, we use a simplified syntax compared to \name{}.
We allow for an arbitrary family of base types \AgdaBound{BaseTy}, and a single
type former \mbox{\ExecuteMetaData[\WReltex]{rAToB}}, equivalent to
\mbox{\ExecuteMetaData[\PaperExamplestex]{BangrAToB}} from the earlier system.

\ExecuteMetaData[\WReltex]{Ty}

In the term syntax, $\lambda$-abstraction now binds a variable with annotation
\AgdaBound{r}, while application needs to scale its argument by \AgdaBound{r}
(both in accordance with the function type they are acting on).

\ExecuteMetaData[\WReltex]{AnnArr}

As a running example, we take the usage annotations to be the 4-element
variance posemiring (\cref{thm:variance}).
We establish the property that all terms are monotonic in their free variables.
This monotonicity can be covariant or contravariant (or neither or both)
depending on the annotation of each free variable.
This provides an additional example to those of \citeauthor{AbelBernardy2020}.

We will take semantics of this system into
\emph{world-indexed relations}~\cite{AbelBernardy2020,context-constrained}.
A world-indexed relation (\AgdaRecord{WRel}) over a poset of worlds
\AgdaBound{W} is a set over which
we have a \AgdaBound{W}-indexed binary relation satisfying a presheaf-like
property with respect to the order on \AgdaBound{W}.
The Agda code for world-indexed relations and constructions on them
can be found in \cite{generic-lr}. %\cref{sec:den-sem-extras}.

\begin{example}
  When \AgdaBound{W} is the 1-element set, a world-indexed relation is just a
  set equipped with a binary relation.
\end{example}

Morphisms (\AgdaRecord{WRelMor}) between world-indexed relations \AgdaBound{R}
and \AgdaBound{S} consist of a mapping between the underlying sets such that, at
each fixed world \AgdaBound{w}, the mapping preserves relatedness from
\AgdaBound{R} to \AgdaBound{S}.

When the poset of worlds forms a (relational) commutative monoid, such
world-indexed relations support a symmetric monoidal closed structure, with
objects denoted \AgdaFunction{I$^R$},
\AgdaFunction{\AgdaUnderscore{}$\otimes^R$\AgdaUnderscore{}}, and
\AgdaFunction{\AgdaUnderscore{}$\multimap^R$\AgdaUnderscore{}},.
These reuse the bunched connectives \AgdaRecord{$I^*$}, \AgdaRecord{$\sep$}, and
\AgdaRecord{$\wand$}, now over worlds rather than contexts.

The final piece of semantics we need is a \emph{bang} operator.
We allow the
semantic \emph{bang} to be an arbitrary annotation-indexed functor on
world-indexed relations.
This functor must respect all of the structure on the indices, making it a
graded comonad over multiplication, as well as being lax monoidal at any
particular index \AgdaBound{r}.
These laws are listed in the \texttt{Generic.Linear.Example.WRel} module in \cite{generic-lr}. %\cref{sec:den-sem-extras}.

\begin{example}
  With \AgdaBound{W} being the 1-element set and annotations coming from the
  variance semiring, we can define the following \emph{bang}.
  It is always the identity on the set component, while the relation component
  consists of flipping the relation for contravariance and taking conjunctions
  to achieve both covariance and contravariance.
  When we want neither covariance nor contravariance, we use the always true
  predicate on worlds \AgdaFunction{$\dot1$}.

  \ExecuteMetaData[\Monotonicitytex]{BangR}
\end{example}

%To associate semantics to syntax, we start as standard by associating
%world-indexed relations to types.
%We also extend the semantics of types to contexts, using \AgdaFunction{I$^R$},
%\AgdaFunction{$\otimes^R$}, and \AgdaField{!$^R$} to interpret the empty
%context, concatenation, and usage annotations on singletons, respectively.
%
%\ExecuteMetaData[\WReltex]{sem}
%
%The semantics of a term is then to be a morphism from the interpretation of the
%context to the interpretation of the term's type.
%
%\ExecuteMetaData[\WReltex]{sem-vdash}

The semantics of a type is given by \AgdaFunction{$\llbracket$\_$\rrbracket$},
which maps into world-indexed relations.
The function type is interpreted using \AgdaFunction{$\multimap^R$} and
\AgdaField{!$^R$}.
Contexts are interpreted by \AgdaFunction{$\llbracket$\_$\rrbracket^c$}, using
\AgdaFunction{$\otimes^R$} and \AgdaFunction{I$^R$}.
Terms are interpreted as morphisms by the open family
\AgdaFunction{$\llbracket$\_$\vdash$\_$\rrbracket$}.
Variables are interpreted by \AgdaFunction{lookup$^R$} (definition omitted).

\ExecuteMetaData[\WReltex]{lookupR-type}

Now we give a \AgdaRecord{Semantics}.
The choice of \AgdaBound{$\V$} as
\AgdaRecord{\AgdaUnderscore{}$\sqni$\AgdaUnderscore{}} is somewhat arbitrary,
given that a standard denotational semantics would not use intermediate
environments in the same sense as renaming and substitution, but it allows us to
reuse the standard facts that variables support renaming and identity
environments.
With this choice of \AgdaBound{$\V$} and \AgdaBound{$\C$}, we interpret
environment entries by \AgdaFunction{lookup$^R$}.
Meanwhile, for the logical rules, we ignore environments by using
\AgdaFunction{reify} to just deal with morphisms in an extended context.
As such, $\lambda$-abstractions are easy to interpret, while applications
require some massaging to remove the extension by an empty context, followed by
some plumbing to split the interpretation of the context according to the usage
constraints and feed the interpretation of the argument \AgdaBound{n$'$} into
the interpretation of the function \AgdaBound{m$'$}.

\ExecuteMetaData[\WReltex]{Wrel}

Then, the semantics of terms is given by the function
\AgdaFunction{semantics}\AgdaSpace{}\AgdaFunction{Wrel}\AgdaSpace{}%
\AgdaFunction{1$^r$}, where \AgdaFunction{1$^r$} is the identity renaming.
%In order to map open terms to interpretations, we take the action of the
%semantics and give the identity renaming as the starting environment.
%
%\ExecuteMetaData[\WReltex]{wrel}

\begin{example}\label{thm:minus}
  We can make a subtraction function from primitive addition and negation on
  integers.
  Subtraction is covariant in its first argument and contravariant in its
  second argument.
  We give the definition in pseudocode, though it is also amenable to the
  usage elaborator of \cref{sec:usage-elaborator}, suitably instantiated.

  \begin{align*}
    &{\sim\sim}p :
      {\uparrow\uparrow}\mathbb Z \multimap
      {\uparrow\uparrow}\mathbb Z \multimap \mathbb Z,
      {\sim\sim}n : {\downarrow\downarrow}\mathbb Z \multimap \mathbb Z
      \vdash \mathnormal{minus} :
      {\uparrow\uparrow}\mathbb Z \multimap
      {\downarrow\downarrow}\mathbb Z \multimap
      \mathbb Z
    \\
    &\mathnormal{minus} \coloneqq \lambda x.~\lambda y.~p\,x\,(n\,y)
  \end{align*}

  After feeding in Agda's addition and negation functions as the
  interpretations of the free variables (noting that they are both monotonic
  in the required way), we get the following free theorem.

  \ExecuteMetaData[\Monotonicitytex]{thm-type}

  %We observe that the set component of this term's semantics is just the
  %expected Agda function when the two free variables are given appropriate
  %interpretations.

  %\ExecuteMetaData[\Monotonicitytex]{minus-set}

  %Furthermore, the relational component of the semantics yields the free
  %theorem that the Agda subtraction so defined is monotonic in the expected way.
  %This relies on library proofs that addition and negation are suitably
  %monotonic.

  %\ExecuteMetaData[\Monotonicitytex]{thm}
\end{example}
