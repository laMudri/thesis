\subsection{Renaming and substitution}

In an unpublished note, \citet{McBride05} gives a parametrised traversal
yielding homomorphisms of syntax.
The parameters are collected in the record \AgdaRecord{Kit}.
We make a minor change to the original presentation, where instead of our
\AgdaField{ren\textasciicircum{}$\V$} field, \citeauthor{McBride05} has the
field \AgdaField{wk} allowing only context extensions.
As for the other two fields, \AgdaField{vr} allows us to map variables to
$\V$-values, so as to put newly bound variables in environments; and
\AgdaField{tm} allows us to extract terms from $\V$-values, as required when
we use the environment to evaluate a free variable.

\ExecuteMetaData[\Syntactictex]{Kit}

Where \citeauthor{McBride05} gave the traversal explicitly, we go via our
generic semantic traversal.
The first two fields of \AgdaRecord{Semantics} derive directly from fields of
\AgdaRecord{Kit}.
Meanwhile, to handle term constructors, we first \AgdaFunction{reify} to get a
collection of traversed subterms, and then use \AgdaInductiveConstructor{`con}
to assemble these subterms into a similarly shaped syntactic form as we started
with.
The \AgdaField{vr} field is used implicitly in \AgdaFunction{reify}, as it is
used to show that $\V$-identity environments exist.

\ExecuteMetaData[\Syntactictex]{kit-to-sem}

The action of a syntactic traversal on logical rules is basically fixed: we
preserve the logical rule and extend the environment with any newly bound
variables according to \AgdaField{vr}.
Meanwhile, the action on variables is relatively unconstrained: we look up the
variable in the environment to get a $\V$-value, then transform that $\V$-value
into a term using \AgdaField{tm}.

The idea of renaming is that variables replace variables, whereas with
substitution, terms replace variables.
This translates to environments for renaming containing $\sqni$-values
(variables), and environments for substitution containing $\vdash$-values
(terms).

%To implement renaming and substitution for terms, we now just implement
%syntactic kits for variables and terms, respectively.

\ExecuteMetaData[\Syntactictex]{Ren-Kit}

Notice that \AgdaFunction{ren\textasciicircum$\vdash$}, witnessing the fact
that terms are renameable, is a corollary of \AgdaFunction{Ren-Kit}.

\ExecuteMetaData[\Syntactictex]{Sub-Kit}

\subsection{A denotational semantics}

\subsection{A usage elaborator}\label{sec:usage-elaborator}

Using the constructs we have seen so far, producing example terms soon becomes
extremely tedious.
We achieved some abbreviation by using pattern synonyms, but we still have to
produce essentially bespoke proofs whenever we use a usage-sensitive part of the
syntax.
The size of each of these proofs is roughly proportional to the number of free
variables, so the amount of proof we have to write grows roughly quadratically
with the size of terms.
An additional factor, which we can't see on paper, is that type checking time
for these proofs soon becomes prohibitive to interactive development.

Our aim in this subsection is to automate usage constraint proofs, making terms
both easier to write and more performant to check.
We invoke the automation by writing terms in a syntax where usage constraints
have been trivialised, and then use a semantic traversal over the simplified
syntax to try to produce a fully elaborated term in the original syntax.
We write the automation in a way that is generic in the syntax description, thus
avoiding repetition and facilitating the prototyping of new type systems.

The type of syntax descriptions depends on the type of usage annotations because
of variable binding.
For example, in the $\oc_{\gr r}$-E rule of \cref{fig:lr-comb}, the right
premise binds a new variable with annotation $\gr r$, where $\gr r$ is drawn
from the ambient posemiring.
The scaling combinator also makes direct reference to the posemiring.
To produce a simplified syntax description, where usage constraints are
trivialised, we set the ambient posemiring to the 1-element $\mathbf0$
posemiring.
In contrast to syntax descriptions, even though types can contain usage
annotations, the type of types does not depend on the type of usage annotations.
This means that, in our simplified syntax, terms have types from the original
system even though variables have trivial usage annotations.
We define the $\mathbf0$ posemiring as follows, being careful to use the
0-field record type \AgdaRecord{$\top$} so that everything algebraic gets
solved by Agda's $\eta$-laws.
Indeed, in this very definition, all of the semiring operations and laws are
canonically inferred.

\ExecuteMetaData[\UsageChecktex]{0-poSemiring}

The elaboration process is monadic.
In particular, we use the \AgdaDatatype{List}/non-determinism monad to give
\emph{all} of the possible annotation choices on the free variables of a term.
We believe the commitment to multiple solutions is inherent when the syntax
contains \AgdaInductiveConstructor{`$\dot1$}.
For example, in the intermediate stages of elaborating
$\plr{\vdash \lambda x.~\plr{*,*}} : A \multimap \top \otimes \top$ with a
usage counting posemiring (assuming reasonable rules for $\top$ and $\otimes$),
it is unclear whether to use the variable $x$ in the left $*$ or the right $*$.
This uncertainty should be reflected in the final result.

The non-deterministic choices we make during elaboration are enumerated by
the fields of \AgdaRecord{NonDetInverses}.
These choices are driven by the typing rules and a candidate usage vector for
the conclusion.
For example, \AgdaField{+$^{-1}$}\AgdaSpace{}\AgdaBound{r} is needed when we
encounter a \AgdaInductiveConstructor{`$\sep$} in the syntax and the candidate
usage annotation we are considering is \AgdaBound{r}.
Then, \AgdaField{+$^{-1}$}\AgdaSpace{}\AgdaBound{r} is a list of pairs of
annotations \AgdaBound{p} and \AgdaBound{q} that \AgdaBound{r} can split into,
together with a proof of the splitting.
For \AgdaField{0\#$^{-1}$} and \AgdaField{1\#$^{-1}$}, inverses to constants,
we are given the candidate \AgdaBound{r} and typically return an empty list if
the constraint cannot be satisfied, or a singleton list containing a proof.
\AgdaField{*$^{-1}$} is used when we encounter scaling, in which case we know
both the scaling factor \AgdaBound{r} (from the syntax description) and the
candidate \AgdaBound{q}.
These inverse operations combine monadically (in fact, applicatively) to give
inverses to the vector operations of zero, addition, scaling, and basis.

\ExecuteMetaData[\UsageChecktex]{NonDetInverses}

We choose the \AgdaBound{$\V$} of our semantics to be (unannotated) variables.
For the \AgdaBound{$\C$}, we consider \emph{functions} from candidate usage
vectors \AgdaBound{R} to the list of elaborated derivations with usage
annotations given by \AgdaBound{R}.
The module name \AgdaModule{U} refers to the fact that we are taking the
ambient posemiring to be $\mathbf0$ in \AgdaFunction{OpenFam}.
The effect on \AgdaFunction{OpenFam} is that the usage annotations of any
contexts we consider are uninformative (hence the \AgdaSymbol{\_} on the left).

\ExecuteMetaData[\UsageChecktex]{C}

To traverse the unannotated terms, we produce a \AgdaRecord{Semantics} over the
unannotated system \AgdaFunction{uSystem}\AgdaSpace{}\AgdaBound{sys}.
We already know that variables are renameable.
To interpret a variable, we consider all the possible proofs that such a
variable could be well annotated, and package them up as a variable term.

\ExecuteMetaData[\UsageChecktex]{elab-sem}

\ExecuteMetaData[\UsageChecktex]{lemma-type}

To actually use \AgdaFunction{elab-sem} on terms, we take the associated
\AgdaFunction{semantics} and pass it the identity environment (an identity
\emph{renaming} in this case, because $\V$ is a family of variables).
The candidate usage vector \AgdaBound{R} will be empty for closed terms, and
otherwise we have to supply the intended usage annotations.
