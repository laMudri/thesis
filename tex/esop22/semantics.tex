Given a $\V$-environment $\Gamma \Rightarrow \Delta$, the function
\AgdaFunction{semantics} we define in this section assigns a
$\C$-value in context $\Gamma$ to every term in context $\Delta$,
where $\C$ is an \AgdaFunction{OpenFam} being the carrier of the
semantic interpretation of terms ($\V$ being the semantic
interpretation of variables). Before we can define
\AgdaFunction{semantics}, we need to treat recursion through rules'
premises (\cref{sec:functorial}) and extension of environments when
going under variable binders (\cref{sec:kripke}).


% Our goal in this section is to define \AgdaFunction{semantics}, a
% recursor that turns a term into a \AgdaBound{$\C$}-value using a
% \AgdaBound{$\V$}-environment, in a type preserving way:\bob{Get rid of
%   ``body'' here}

% \ExecuteMetaData[\Semanticstex]{semantics-type}

% The \AgdaBound{$\V$} and \AgdaBound{$\C$} are \AgdaFunction{OpenFam}s,
% representing the interpretations of variables and terms
% respectively. In \cref{sec:traversal} we will see the data that must
% be provided to make a \AgdaFunction{semantics} for a given
% system. Before that, we must see how to deal with the two complicated
% features of our syntax: the usage annotations (\cref{sec:functorial})
% and variable binding (\cref{sec:kripke}). \todo{fwd ref to where these are used}

\subsection{A layer of syntax is functorial}\label{sec:functorial}

A basic property of the universe of syntaxes we described in \cref{sec:syntax}
is that every syntax supports a functorial action on subterms, realised by the function \AgdaFunction{map-s}.
Its type says that to map a function \AgdaBound{f}
over a layer of syntax, there must be a linear map \AgdaBound{F} relating the
domain and codomain usage contexts, and \AgdaBound{f} should be usable
wherever the domain and codomain usage contexts are similarly related by
\AgdaBound{F}.

\ExecuteMetaData[\Maptex]{map-s-type}

This generality is needed because usage contexts change between
a term and its immediate subterms---they are decomposed according to the bunched connectives used in the rules.
\AgdaBound{X} and \AgdaBound{Y} are \AgdaFunction{ExtOpenFam}s, with
\AgdaBound{$\Theta$} being the context extension for a subterm (i.e., the
variables newly bound in that subterm).
Unlike usage annotations, types in the contexts \AgdaBound{$\gamma$} and \AgdaBound{$\delta$}, and the conclusion types implicit here, are preserved throughout.
This is the essence of the usage annotation based approach---we use traditional techniques for variable binding, with the usage annotations layered on top.

The heart of \AgdaFunction{map-s} is \AgdaFunction{map-p}, which recursively
works through the structure \AgdaBound{ps} of premises of the rule applied,
acting on each subterm it finds.
Here, particularly in the clauses for \AgdaInductiveConstructor{`$\sep$} and
\AgdaInductiveConstructor{`$\cdot$}, we see why it is not enough for the
function on subterms to apply at usage contexts \AgdaBound{P} and \AgdaBound{Q}
--- rather, it also needs to apply at any similarly related \AgdaBound{P$'$}
and \AgdaBound{Q$'$}.
In the case of \AgdaInductiveConstructor{`$\sep$}, we have that
$\grP \leq \grP_M + \grP_N$, with \AgdaBound{M} and \AgdaBound{N} being
collections of subterms in usage contexts $\grP_M$ and $\grP_N$, respectively.
Linearity of \AgdaBound{F} yields $\grQ_M$ and $\grQ_N$ such that
$\grQ \leq \grQ_M + \grQ_N$ and we use \AgdaFunction{map-p} recursively at
$(\grP_M, \grQ_M)$ and $(\grP_N, \grQ_N)$ on \AgdaBound{M} and \AgdaBound{N}.
The cases for \AgdaInductiveConstructor{`$\cdot$} and
\AgdaInductiveConstructor{`$I^*$} are similar, each using a different aspect
of linearity.
In contrast, the cases for \AgdaInductiveConstructor{`$\dot1$} and
\AgdaInductiveConstructor{`$\dot\times$}, which are the only constructors used in fully structural
systems, do not involve any changes in the usage contexts.

\ExecuteMetaData[\Maptex]{map-p}

\subsection{The Kripke function space}\label{sec:kripke}

At this point we introduce a minor generalisation to
\AgdaFunction{OpenFam} and \AgdaFunction{ExtOpenFam}:
\AgdaBound{I}\AgdaSpace{}\AgdaFunction{---OpenFam} and
\AgdaBound{I}\AgdaSpace{}\AgdaFunction{---ExtOpenFam}.  We obtain the
definition of \AgdaBound{I}\AgdaSpace{}\AgdaFunction{---OpenFam} by
replacing the textual occurrence of \AgdaBound{Ty} by the parameter
\AgdaBound{I}.

The definition \AgdaFunction{Kripke}\,$\V$\,$\C$\,$\Delta$ is a kind
of function space that describes a $\C$ value parameterised by
$\Delta$-many additional $\V$s (all correctly typed and usage
annotated). It is used to describe how to go under binders in a
Higher-Order Abstract Syntax style---to go under a binder we must
provide semantic interpretations for all the additional variables:

% When going under binders during a recursion, the context will be extended by some $\Theta$. This means that the current environemnt must be extended with $\Theta$s-worth of $\V$s

% we need the ability to say that

% Kripke V C is given the extension \Theta

% In \secref{sec:terms}, we defined \AgdaFunction{Scope} to let a
% judgement-indexed family admit context extensions. However, a key
% component of our generic semantic traversal is to make use of the open
% family \AgdaBound{$\V$} of \emph{values}, which are the sort of thing
% we store in an environment.  The definition \AgdaFunction{Kripke}
% gives an alternative to \AgdaFunction{Scope} which interprets the
% newly bound variables via a requirement of $\V$-values rather than
% extra assumptions for the $\C$-computation.

\ExecuteMetaData[\Semanticstex]{Kripke}

\AgdaFunction{Wrap}
is a device that turns any type family into an equivalent type family
that is judgementally injective in its indices, which helps with
Agda's type inference.
It turns the type family into a parametrised
record with a single field \AgdaField{get} whose type is the type in
the body of the $\lambda$-abstraction.
For understanding the meaning of
\AgdaFunction{Kripke}, \AgdaFunction{Wrap} can be ignored.

If $\Delta$ is of the form $\gr{s_1}B_1, \ldots, \gr{s_n}B_n$, then
\ExecuteMetaData[\Snippetstex]{KripkeVCDGA}\ is equivalent to
\ExecuteMetaData[\Snippetstex]{KripkeExpanded}\ by Currying.  That is
to say, the Kripke function is expecting a value for each newly bound
variable, at the multiplicity of its annotation, together with the
resources supporting each of those values. We use the ``magic wand''
function space here to enforce the invariant that the freshly bound
variables have usage annotations that are added to the existing
variables, not shared with them. The use of the
\AgdaFunction{$\Box^r$} modality ensures that we can still use it in
the presence of additional variables introduced by weakening.

\AgdaFunction{Kripke} is functorial in the $\C$ argument, as witnessed
by the \AgdaFunction{mapKC} function, which is essentially
post-composition:

\ExecuteMetaData[\Semanticstex]{mapKC}

% is exemplified by the following construct
% \AgdaFunction{reify}, where we weaken \AgdaBound{$\Gamma$} by a $\gr0$ed-out
% version of \AgdaBound{$\Delta$}.
% The \AgdaBound{$\Delta$} then gets filled in by the $\V$-values.

% \bob{Move this para}
% This means that \AgdaBound{A} in the definition of \AgdaFunction{Kripke} has
% type \AgdaBound{I}, rather than specifically \AgdaBound{Ty}.
% We use this generality later in \AgdaFunction{extend}, setting \AgdaBound{I}
% to \AgdaDatatype{Ctx}.

\subsection{Semantic traversal}\label{sec:traversal}

We can now state the data required to implement a traversal assigning
semantics to terms. For open families $\V$ and $\C$, interpreting
variables and terms respectively, we assume that $\V$ is renamable,
that $\V$ is embeddable in $\C$, and that we have an algebra for a
layer of syntax, where bound variables are handled using the Kripke
function space:

% The aim of this subsection is to give an alternative recursion principle for
% terms that incorporates some of the environment-handling seen in the
% implementations of renaming and substitution.
% The rest of this section assumes the following: a renameable open family
% \AgdaBound{$\V$} that embeds into the open family \AgdaBound{$\C$}, and an
% algebra for a layer of syntax at \AgdaBound{$\C$}.

\ExecuteMetaData[\Semanticstex]{Semantics}

%\ExecuteMetaData[\Semanticstex]{Comp}

We mutually define the action \AgdaFunction{semantics} and its lemma
\AgdaFunction{body}.
The purpose of \AgdaFunction{semantics} is to turn a term into a
\AgdaBound{$\C$}-value using a \AgdaBound{$\V$}-environment and the fields of
\AgdaRecord{Semantics}.
Meanwhile, \AgdaFunction{body} does a similar job, but also deals with
newly bound variables.
In particular, \AgdaFunction{body} takes a term in a context extended by
\AgdaBound{$\Theta$}, and produces a Kripke function from
\AgdaBound{$\V$}-values for \AgdaBound{$\Theta$} to \AgdaBound{$\C$}-values.

\ExecuteMetaData[\Semanticstex]{semantics-type}

To implement the new recursor \AgdaFunction{semantics}, we use the standard
recursor, which in one case gives us a variable \AgdaBound{v}, and in the other
gives us a structure of subterms \AgdaBound{M}, each of which is in an extended
context.
To deal with a variable \AgdaBound{v}, we look it
up in the environment \AgdaBound{$\rho$}, then use the
\AgdaField{$\sem{\text{var}}$} field to map the resulting
\AgdaBound{$\V$}-value to a \AgdaBound{$\C$}-value.
To deal with a structure of subtems \AgdaBound{M}, we use the functoriality of
the syntactic structure to consider each subterm separately.
On a subterm, we apply \AgdaFunction{body}, which amounts to a recursive call
to \AgdaFunction{semantics} with an extended environment.
Recall that \AgdaFunction{relocate} (\cref{thm:env-resize}) adjusts the
environment \AgdaBound{$\rho$} to work in the usage contexts of the subterms.

\ExecuteMetaData[\Semanticstex]{semantics}

For \AgdaFunction{body}, we are given a subterm \AgdaBound{M}, to
which we want to apply \AgdaFunction{semantics}.  To do so, we need an
extended version of the initial environment \AgdaBound{$\rho$}. We
express this as the generation of a Kripke function that produces the
extended enviroment given interpretations of the fresh variables. We
take \AgdaBound{$\rho$}, which is an environment covering
\AgdaBound{$\Delta$}, and \AgdaBound{$\sigma$}, which is an
environment covering \AgdaBound{$\Theta$}, and glue them together
using the inductive rules for generating environments, after having
renamed \AgdaBound{$\rho$} via \cref{thm:env-ren} to make it fit the
new context \AgdaBound{$\Gamma^+$} (intended to be
\ExecuteMetaData[\Snippetstex]{GT}):

\ExecuteMetaData[\Semanticstex]{extend}

% The best we can achieve without identity environments for \AgdaBound{$\V$} is
% a Kripke function returning an extended environment.
To define \AgdaFunction{body}, we use \AgdaFunction{mapK$\C$} to
post-compose the environment extension by the
\AgdaSymbol{$\lambda$}-function taking an extended environment and
acting with it on \AgdaBound{M}.

\ExecuteMetaData[\Semanticstex]{body}

% \todo{FIX} Under the assumption that \AgdaBound{$\V$} is renameable, we can decompose
% \cref{thm:lr-bind} as
% \AgdaFunction{reify}\AgdaSpace{}\AgdaOperator{\AgdaFunction{$\circ$}}%
% \AgdaSpace{}\AgdaFunction{extend}, with \AgdaFunction{extend} defined below.
% We can think of \AgdaFunction{extend} as our best effort to extend an
% environment by \AgdaBound{$\Theta$} without access to an identity environment
% at \AgdaBound{$\Theta$}.
