Our aim with the following traversals is to derive a recursion scheme for
terms which abstracts over some of the details of variable binding and
changes to usage annotations.
We follow the structure given by \citet{AACMM20} closely, with
the main difference having come already in the syntax and environments.

\subsection{Renameability}

\todo{Maybe this goes in the Environment section.}
We saw in \cref{sec:env} that our notion of environment specialises to
simultaneous renaming by setting $\V$ to be $\sqni$.
This is encoded below as
\AgdaFunction{\AgdaUnderscore{}$\Rightarrow^r$\AgdaUnderscore{}}.

\ExecuteMetaData[\Renamingtex]{Ren}

In a setting where new variables can be bound in the middle of a derivation,
it is important that the values we carry around while traversing a term can
handle the existence of variables that appear but they do not use.
We call any such notion of value \emph{renameable}.
The cofree renameable \AgdaFunction{OpenType} on an \AgdaFunction{OpenType}
\AgdaBound{T} is \AgdaFunction{$\Box^r$}\AgdaSpace{}\AgdaBound{T}, with
\AgdaBound{T} then being \AgdaFunction{Renameable} if it forms a
\AgdaFunction{$\Box^r$}-coalgebra.

\ExecuteMetaData[\Renamingtex]{Boxr}
\ExecuteMetaData[\Renamingtex]{Renameable}

A renameable notion of value gives rise to a renameable notion of environment,
essentially by renaming each contained value in an appropriate way.

\begin{lemma}[\AgdaFunction{ren\textasciicircum{}Env}]\label{thm:env-ren}
  If $\plr{-} \sdtstile{}\V A$ is renameable for all $A$, then so is
  $\plr{-} \env\V \Delta$ for all $\Delta$.
\end{lemma}
\begin{proof}
  For renameable $\V$, we can implement $\mathrm{lift}$ as in
  \cref{thm:env-comp} with $\U \coloneqq {\sqni}$ and
  $\V \coloneqq \W \coloneqq \V$.
  Renaming the environment then becomes an instance of environment composition.
\end{proof}

\subsection{The Kripke function space}

Earlier, when talking about syntax, we saw \AgdaFunction{Scope} as a simple way
to let an open family admit context extensions.
However, a key component of our generic semantic traversal is to make use of
the open family \AgdaBound{$\V$} of \emph{values}, which are the sort of thing
we store in an environment.
The definition \AgdaFunction{Kripke} gives an alternative to
\AgdaFunction{Scope} which interprets the newly bound variables via a
requirement of $\V$-values rather than extra assumptions for the
$\C$-computation.

\ExecuteMetaData[\Semanticstex]{Kripke}

Just as $\dotto$ is the internal hom for $\dottimes$, we introduce $\wand$ as
the internal hom for $\sep$, with \AgdaFunction{$\wand^c$} acting on open types
analogously to \AgdaFunction{$\sep^c$}.
It is given by $\plr{T \wand U}\,\grP \coloneqq
\Pi\grQ,\grR.~\plr{\grR \leq \grP + \grQ} \to T\,\grQ \to U\,\grR$.

For understanding the meaning of the type family, \AgdaFunction{Wrap} can be
ignored.
\AgdaFunction{Wrap} is a device that turns any type family into an equivalent
type family that is judgementally injective in its indices.
It turns the type family into a parametrised record with a single field
\AgdaField{get} whose type is the type in the body of the $\lambda$-abstraction.
If $\Delta$ is of the form $\gr{s_1}B_1, \ldots, \gr{s_n}B_n$, then
\ExecuteMetaData[\Snippetstex]{KripkeVCDGA}\ is equivalent to
\ExecuteMetaData[\Snippetstex]{KripkeExpanded}\ by Currying.
That is to say, the Kripke function is expecting a value for each newly bound
variable, at the multiplicity of its annotation, together with the resources
supporting each of those values.
The need for \AgdaFunction{$\Box^r$} is exemplified by the following construct
\AgdaFunction{reify}, where we weaken \AgdaBound{$\Gamma$} by a $\gr0$ed-out
version of \AgdaBound{$\Delta$}.
The \AgdaBound{$\Delta$} then gets filled in by the $\V$-values.

At this point we introduce a minor generalisation to \AgdaFunction{OpenFam}
and \AgdaFunction{ExtOpenFam}:
\AgdaBound{I}\AgdaSpace{}\AgdaFunction{---OpenFam} and
\AgdaBound{I}\AgdaSpace{}\AgdaFunction{---ExtOpenFam}.
We obtain the definition of \AgdaBound{I}\AgdaSpace{}\AgdaFunction{---OpenFam}
by replacing the textual occurrence of \AgdaBound{Ty} by the parameter
\AgdaBound{I}.
This means that \AgdaBound{A} in the definition of \AgdaFunction{Kripke} has
type \AgdaBound{I}, rather than specifically \AgdaBound{Ty}.
We use this generality later in \AgdaFunction{extend}, setting \AgdaBound{I}
to \AgdaDatatype{Ctx}.

When \AgdaBound{$\V$} supports identity environments (as per \cref{thm:env-id}),
we can coerce a \AgdaFunction{Kripke} function into just a
\AgdaBound{$\C$}-value in an extended context.
To show this, we instantiate the Kripke function with the renaming
$\swarrow^r : \Gamma, \gr0\delta \env\sqni \Gamma$ to extend the scope, and
pass it an argument environment
$\plr{\id^\V; \searrow^r} : \gr0\gamma, \Delta \env\V \Delta$ to fill in the
extended part.
Post-composition of a renaming onto an environment comes from
\cref{thm:env-postren}.

\ExecuteMetaData[\Syntactictex]{reify}

We tend to use \AgdaFunction{reify} whenever \AgdaBound{$\V$} supports
identity environments.
In this paper, all examples support identity environments.
However, \citet[p.~27]{AACMM20} give some important examples that do not
support identity environments, and thus cannot use \AgdaFunction{reify}.
The feature that causes the lack of support for identity environments is that
a semantics can make use of the fact that only variables of particular kinds
are bound by the syntax.
In the examples of \citeauthor{AACMM20}, a bidirectionally typed language only
binds variables that synthesise their type, as opposed to those whose type is
checked.
The semantics of type-checking and elaboration rely on variables synthesising
their type, so \AgdaBound{$\V$} is chosen to cover only those variables.
Instead of using \AgdaFunction{reify}, we must observe that each syntactic form
only binds such synthesising variables.
Similar phenomena would appear in, say, a call-by-value language where
variables are values (not computations), or a polarised language where
variables always have a polarity matching their type.

\subsection{Semantic traversal}

The aim of this subsection is to give an alternative recursion principle for
terms that incorporates some of the environment-handling seen in the
implementations of renaming and substitution.
The rest of this section assumes the following: a renameable open family
\AgdaBound{$\V$} that embeds into the open family \AgdaBound{$\C$}, and an
algebra for a layer of syntax at \AgdaBound{$\C$}.

\ExecuteMetaData[\Semanticstex]{Semantics}

%\ExecuteMetaData[\Semanticstex]{Comp}

Under the assumption that \AgdaBound{$\V$} is renameable, we can decompose
\cref{thm:lr-bind} as
\AgdaFunction{reify}\AgdaSpace{}\AgdaOperator{\AgdaFunction{$\circ$}}%
\AgdaSpace{}\AgdaFunction{extend}, with \AgdaFunction{extend} defined below.
We can think of \AgdaFunction{extend} as our best effort to extend an
environment by \AgdaBound{$\Theta$} without access to an identity environment
at \AgdaBound{$\Theta$}.
We instead take \AgdaBound{$\rho$}, which is an environment covering
\AgdaBound{$\Delta$}, and \AgdaBound{$\sigma$}, which is an environment
covering \AgdaBound{$\Theta$}, and glue them together using
\cref{thm:env-rules}, after having renamed \AgdaBound{$\rho$} via
\cref{thm:env-ren} to make it fit the new context \AgdaBound{$\Gamma^+$}
(intended to be \ExecuteMetaData[\Snippetstex]{GT}).

\ExecuteMetaData[\Semanticstex]{extend}

We mutually define the action \AgdaFunction{semantics} and its lemma
\AgdaFunction{body}.
The purpose of \AgdaFunction{semantics} is to turn a term into a
\AgdaBound{$\C$}-value using a \AgdaBound{$\V$}-environment and the fields of
\AgdaRecord{Semantics}.
Meanwhile, \AgdaFunction{body} does a similar job, but also deals with
newly bound variables.
In particular, \AgdaFunction{body} takes a term in a context extended by
\AgdaBound{$\Theta$}, and produces a Kripke function from
\AgdaBound{$\V$}-values for \AgdaBound{$\Theta$} to \AgdaBound{$\C$}-values.

\ExecuteMetaData[\Semanticstex]{semantics-type}

To implement the new recursor \AgdaFunction{semantics}, we use the standard
recursor, which in one case gives us a variable \AgdaBound{v}, and in the other
gives us a structure of subterms \AgdaBound{M}, each of which is in an extended
context.
To deal with a variable \AgdaBound{v}, we look it
up in the environment \AgdaBound{$\rho$}, then use the
\AgdaField{$\sem{\text{var}}$} field to map the resulting
\AgdaBound{$\V$}-value to a \AgdaBound{$\C$}-value.
To deal with a structure of subtems \AgdaBound{M}, we use the functoriality of
the syntactic structure to consider each subterm separately.
On a subterm, we apply \AgdaFunction{body}, which amounts to a recursive call
to \AgdaFunction{semantics} with an extended environment.
Recall that \AgdaFunction{relocate} (\cref{thm:env-resize}) adjusts the
environment \AgdaBound{$\rho$} to work in the usage contexts of the subterms.

\ExecuteMetaData[\Semanticstex]{semantics}

For \AgdaFunction{body}, we are given a subterm \AgdaFunction{M}, to which we
want to apply \AgdaFunction{semantics}.
To do so, we need an extended version of the initial environment
\AgdaBound{$\rho$}.
The best we can achieve without identity environments for \AgdaBound{$\V$} is
a Kripke function returning an extended environment.
We use \AgdaFunction{mapK$\C$} to post-compose this function by the
\AgdaSymbol{$\lambda$}-function taking an extended environment and acting with
it on \AgdaBound{M}.

\ExecuteMetaData[\Semanticstex]{body}
