We have now seen how to generically build data types of terms for a
given \AgdaRecord{System}, giving us a setting in which to construct
intrinsically well typed and well annotated terms. Before defining a
generic assignment of semantics for terms, we need to develop the
notion of \emph{environment}, a mapping of variables to their semantic
interpretations that is preserved by structural decompositions and
variable bindings.

Given a semantic notion of variable $\V$ : \AgdaFunction{OpenFam}, we
use the notation $\Gamma \sdtstile{}\V A \coloneqq \V\,\Gamma\,A$ for
the type of inhabitants of $\V$ in the context $\Gamma$ at type
$A$. In the non-substructural systems of Allais et al. \cite{AACMM21},
a $\V$-environment $\Gamma \env\V \Delta$ is nothing more than a
function $\forall A \to \Delta \sqni A \to \Gamma \sdtstile{}\V A$,
mapping variables to $\V$-things. In our usage annotated setting
though, we must correctly distribute resources tracked by the
annotations; making sure that we have enough resources in $\Gamma$ to
cover all the demands in $\Delta$. Following the work of \citet{WA20},
this accounting is expressed via the presence of a linear
transformation:


% the
% abstract notion of variable in different ways, several familiar
% definitions can be expressed as environments: if the abstract
% variables

% If we take the latter to be


% This section is a development of the work of \citet{WA20}.

% Let us write $\sdtstile{}\V$ as the infix version of $\V$.
% That is, $\Gamma \sdtstile{}\V A \coloneqq \V\,\Gamma\,A$.
% In this section, we introduce \emph{$\V$-environments}, which will form a key
% part of the generic traversal of \cref{sec:semantics}.
% The idea of a $\V$-environment is to extend the notion of $\V$-values from
% types to contexts.
% Specifically, where the judgement $\Gamma \sdtstile{}\V A$ says that we
% have a $\V$-value of type $A$ in context $\Gamma$, the judgement
% $\Gamma \env\V \Delta$ says that we have a $\V$-value for each entry in
% $\Delta$ (at the specified multiplicity), all in context $\Gamma$.
% For our purposes, it is essential that the ``all'' in the previous sentence is
% multiplicative --- we want to split the usage annotations of $\Gamma$ up into
% parts such that each part supports one $\V$-value.

% We have seen two kinds of values already: $\sqni$-values are variables and
% $\vdash$-values are terms.
% The corresponding environments are also standard concepts: $\sqni$-environments
% are \emph{simultaneous renamings} and $\vdash$-environments are
% \emph{simultaneous substitutions}.

% \subsection{Definition}

\begin{definition}[Environment]\label{def:lr-env}
  A \emph{$\V$-environment} between annotated contexts $\Gamma$ and
  $\Delta$ (decomposed as $\grP\gamma$ and $\grQ\delta$, respectively,
  when convenient) is a linear map
  $\gr\Psi : \Ann^{\size\Delta} \to \Ann^{\size\Gamma}$ (written
  postfix) such that $\grP \leq \grQ\gr\Psi$ and for each $A$,
  $\grPprime$, and $\grQprime$ such that
  $\grPprime \leq \grQprime\gr\Psi$, a function from
  $\grQprime\delta \sqni A$ to $\grPprime\gamma \sdtstile{}\V A$.

  \bob{Write out the Agda notation for environments}
\end{definition}

By instantiating $\V$, we obtain resource correct versions of familiar
notions: letting $\V$ be $\sqni$ yields resource correct renamings;
and letting $\V$ be $\vdash$ (i.e., terms) yields resource correct
substitutions.

\begin{example}
  We can form the identity renaming on a two-variable context.
  \[
    \id : \plr{\gr rA, \gr sB} \env\sqni \plr{\gr rA, \gr sB}
  \]
  We take $\gr\Psi$ to be the identity map, clearly satisfying
  \(
    \begin{pmatrix} \gr r & \gr s \end{pmatrix} \leq
    \begin{pmatrix} \gr r & \gr s \end{pmatrix}\gr\Psi
  \).
  When considering values, the fact that $\grPprime \leq \grQprime\gr\Psi$
  reduces to $\grPprime \leq \grQprime$.
  The two cases to consider are when $\grQprime\delta \sqni A$ and when
  $\grQprime\delta \sqni B$.
  In the first case, $\grPprime \leq \grQprime \leq
  \begin{pmatrix} \gr1 & \gr0 \end{pmatrix}$, so we have
  $\grPprime\plr{A, B} \sqni A$.
  The second case follows symmetrically.
\end{example}

\begin{example}
  Assume $\Ann$ is the natural numbers with ordering given by $=$ and usual
  addition and multiplication.\bob{and $\vdash$ is the terms from a suitable system}
  There is a $\vdash$-environment (substitution)
  \[
    \plr{\gr0x : A, \gr2y : B \multimap C, \gr3z : B} \env\vdash
    \plr{\gr1B, \gr2C}.
  \]
  The linear map $\gr\Psi$ is given by the matrix
  \(
    \begin{pmatrix}
      \gr0 & \gr0 & \gr1 \\
      \gr0 & \gr1 & \gr1
    \end{pmatrix}
  \),
  noticing that $\gr1$ times the first row plus $\gr2$ times the second gives
  the original $\grP$.
  For $\grQprime = \begin{pmatrix} \gr1 & \gr0 \end{pmatrix}$, we have
  $\grPprime = \begin{pmatrix} \gr0 & \gr0 & \gr1 \end{pmatrix}$, and thus
  $\grPprime\gamma \vdash z : B$.
  For $\grQprime = \begin{pmatrix} \gr0 & \gr1 \end{pmatrix}$, we have
  $\grPprime = \begin{pmatrix} \gr0 & \gr1 & \gr1 \end{pmatrix}$, and thus
  $\grPprime\gamma \vdash y\,z : C$.
\end{example}

As a mnemonic, one may use notation like the following to see what values are
needed in the environment.
\[
  \begin{pmatrix}
    \gr0A & \gr0\plr{B \multimap C} & \gr1B \\
    \gr0A & \gr1\plr{B \multimap C} & \gr1B
  \end{pmatrix}
  \begin{matrix}
    {} \vdash B \\
    {} \vdash C
  \end{matrix}
\]
This notation assumes that $\V$ supports subusaging, which is
always the case when we are using environments for traversals.\bob{Forward ref to where this is explicitly assumed?}

\begin{example}
  Assume $\Ann$ is the natural numbers with ordering given by $=$ and the usual
  addition and multiplication.
  There is a $\sqni$-environment (renaming)
  \[
    \plr{\gr6a : A, \gr0b : B, \gr1c : C, \gr0d : D} \env\sqni
    \plr{\gr1C, \gr2A, \gr4A}.
  \]
  Linear map $\gr\Psi$ is given by the matrix
  \(
    \begin{pmatrix}
      \gr0 & \gr0 & \gr1 & \gr0 \\
      \gr1 & \gr0 & \gr0 & \gr0 \\
      \gr1 & \gr0 & \gr0 & \gr0
    \end{pmatrix}
  \),
  which we can check satisfies the required inequality.
  The values are given by
  \begin{align*}
    \gr0a : A, \gr0b : B, \gr1c : C, \gr0 : D &\sqni c : C \\
    \gr1a : A, \gr0b : B, \gr0c : C, \gr0 : D &\sqni a : A \\
    \gr1a : A, \gr0b : B, \gr0c : C, \gr0 : D &\sqni a : A.
  \end{align*}

  We can read from the columns of the matrix what happened to each of the
  variables in $\Gamma$.
  The first column, corresponding to variable $\gr6a : A$, contains two $\gr1$s
  because it has been duplicated (via contraction).
  Meanwhile, the second and fourth columns are all $\gr0$ because variables
  $b$ and $d$ have been discarded (via weakening).
  The third column contains one $\gr1$ because $c$ is used once.
  This $\gr1$ appears above the $\gr1$s to its left because $c$ has been
  permuted (via exchange) past $a$.
  Each of the rows in the matrix is a basis vector because variables can only
  be formed in contexts with basis annotations or less.
\end{example}

\paragraph{Relocation} An environment
$\rho : \grP\gamma \env\V \grQ\delta$ does not determine $\grP$ and
$\grQ$, we can replace them with any $\grPprime$ and $\grQprime$ that
are related by the linear map $\rho.\gr\Psi$:

\begin{lemma}[\AgdaFunction{relocate}]\label{thm:env-resize}
  Given an environment $\rho : \grP\gamma \env\V \grQ\delta$ and a $\grPprime$
  and a $\grQprime$ such that $\grPprime \leq \grQprime\plr{\rho.\gr\Psi}$,
  there is also an environment of type $\grPprime\gamma \env\V \grQprime\delta$
  with the same linear map and action on variables.
\end{lemma}
% \begin{proof}
%   The only part of the definition of an environment dependent on $\grP$ or
%   $\grQ$ is the constraint $\grP \leq \grQ\gr\Psi$, which we are able to
%   replace for $\grPprime$ and $\grQprime$.
% \end{proof}
Relocation will be used when pushing environments into subterms in
\secref{XX}.

\paragraph{Inductive Construction}

When constructing an environment, we can do so by cases on the shape
of the target context by the following rules, which use the bunched
connectives from \secref{sec:bunched-rules}:
\begin{displaymath}
  \begin{prooftree}[comb,center=false]
    \hypo{I^*}
    \infer1{\alr{} : {} \env\V {\cdot}}
  \end{prooftree}
  \qquad
  \begin{prooftree}[comb,center=false]
    \hypo{\rho : {} \env\V \Delta_l}
    \hypo{\sep}
    \hypo{\sigma : {} \env\V \Delta_r}
    \infer3{\alr{\rho,\sigma} : {} \env\V \Delta_l, \Delta_r}
  \end{prooftree}
  \qquad
  \begin{prooftree}[comb,center=false]
    \hypo{\gr r\cdot\plr{M : {} \sdtstile{}\V A}}
    \infer1{\alr{M} : {} \env\V \gr rA}
  \end{prooftree}
\end{displaymath}
Left to right, we can create an environment into the empty context
when all usage annotations on the source context are $\gr0$; we can
create an environment into a concatenated context when we can
additively split up the annotations of the source context and produce
environments into both halves from the split sources; and we can
create an environment into a singleton context $\gr rA$ when there is
a context $\gr r$ times smaller than the source context in which we
can produce a $\V$-value of the appropriate type.

% \begin{lemma}\label{thm:env-rules}
%   \Cref{def:lr-env} is sound and complete for the following syntax.
% \end{lemma}

\begin{example}
  Assume $\Ann$ is the natural numbers with ordering given by $=$ and
  the usual addition and multiplication, and $\vdash$ is the type of
  terms for a \AgdaRecord{System} with function application.  There is
  an environment (substitution)
  \[
    \alr{\alr{z},\alr{y\,z}} :
    \plr{\gr0x : A, \gr2y : B \multimap C, \gr3z : B} \env\vdash
    \plr{\gr1B, \gr2C}.
  \]
  We rely on the observations that
  $\begin{pmatrix} \gr0 & \gr2 & \gr3 \end{pmatrix} =
  \begin{pmatrix} \gr0 & \gr0 & \gr1 \end{pmatrix}
  + \begin{pmatrix} \gr0 & \gr2 & \gr2 \end{pmatrix}$ and, on the right, that
  $\begin{pmatrix} \gr0 & \gr2 & \gr2 \end{pmatrix} =
  \gr2\begin{pmatrix} \gr0 & \gr1 & \gr1 \end{pmatrix}$.
  Then, we have $\gr0x : A, \gr0y : B \multimap C, \gr1z : B \vdash z : B$ and
  $\gr0x : A, \gr1y : B \multimap C, \gr1z : B \vdash y\,z : C$.
\end{example}

We could have used these rules to inductively define what environments are.
However, we found that this was difficult to work with. It is often easier to do
linear algebraic proofs separately from the rest of an environment.
For example, for identity and composition of environments (below), \defref{def:lr-env}
is easier to use because we can rely on the identity and composition of linear
maps.
Concretely, an inductive proof of identity would, for example, involve
constructing an environment of type
$\grP\gamma, \grQ\delta \env\V \grP\gamma, \grQ\delta$ by constructing
environments of types $\grP\gamma, \gr0\delta \env\V \grP\gamma$ and
$\gr0\gamma, \grQ\delta \env\V \grQ\delta$.
These are not identity environments, so we would have to strengthen the
induction hypothesis.\bob{FIXME: make this paragraph more neatly flow into the next ones}

% One of the primary test cases for environments is simultaneous substitution,
% which will look like the following rule.
% The admissibility of substitution will be by induction on the derivation of
% $\Delta \vdash A$, so we will need to be able to adapt any environment we are
% given to work with any possible context of new premises.
% In the simply typed case, the only change to the context we encountered was the
% binding of new variables.
% Now, with usage annotations, we furthermore have linear decompositions of the
% context, necessitating changes to the environment whenever usage annotations
% change.
% We deal first with linear decompositions.

% \begin{displaymath}
%   \begin{prooftree}
%     \hypo{\Gamma \env{\vdash} \Delta}
%     \hypo{\Delta \vdash A}
%     \infer2[sub]{\Gamma \vdash A}
%   \end{prooftree}
% \end{displaymath}

% There are three kinds of linear decompositions we have to deal with: zero,
% addition, and scaling; corresponding to bunched connectives $I^*$, $\sep$, and
% $\gr r \cdot {}$, respectively.
% In each case, we have a simple preservation lemma, transforming an environment
% of type $\Gamma \env\V \Delta$ and a decomposition of $\Delta$ into a
% decomposition of $\Gamma$ and environments for all of the decomposed fragments
% of $\Gamma$ and $\Delta$.

% \begin{lemma}[environments preserve zero]\label{thm:lr-env-zero}
%   Given an environment of type $\grP\gamma \env\V \grQ\delta$ such that
%   $\grQ \leq \gr 0$, we also have that $\grP \leq \gr 0$.
% \end{lemma}

% \begin{lemma}[environments preserve addition]\label{thm:lr-env-add}
%   Given an environment of type $\grP\gamma \env\V \grQ\delta$ such that
%   $\grQ \leq \grQl + \grQr$ for some $\grQl$ and $\grQr$, we also have $\grPl$
%   and $\grPr$ such that $\grP \leq \grPl + \grPr$ and there are environments
%   of types $\grPl\gamma \env\V \grQl\delta$ and
%   $\grPr\gamma \env\V \grQr\delta$.
% \end{lemma}

% \begin{lemma}[environments preserve scaling]\label{thm:lr-env-scale}
%   Given an environment of type $\grP\gamma \env\V \grQ\delta$ such that
%   $\grQ \leq \gr r\grQprime$ for some $\grQprime$, we also have a $\grPprime$
%   such that $\grP \leq \gr r\grPprime$ and there is an environment of type
%   $\grPprime\gamma \env\V \grQprime\delta$.
% \end{lemma}

% Finally, we also take the opportunity to give the extend lemma, allowing
% environments to incorporate newly bound variables.
% \todo{Motivate}
% In the intuitionistic case, the extend lemma had two requirements on $\V$: $\V$
% admits weakening and we can map variables into $\V$-values.
% With usage annotations, the former is unreasonable, but it turns out that we
% only need weakening by variables whose usage annotation is less than or equal
% to $\gr0$.
% The latter stays as-is, with the note that ``variable'' now means a
% usage-checked variable.

% \begin{lemma}[extend]\label{thm:lr-bind}
%   Given functions
%   ${\swarrow^k} : \forall \Gamma, \grR, \theta.~\grR \leq \gr0 \to
%   \forallb{\V\,\Gamma \dotto \V\,\plr{\Gamma, \grR\theta}}$ and
%   $\mathrm{vr} : \forallb{{\sqni} \dotto \V}$, we can turn an environment of
%   type $\Gamma \env\V \Delta$ into an environment of type
%   $\Gamma, \Theta \env\V \Delta, \Theta$ for any context $\Theta$.
% \end{lemma}

\paragraph{Identity and Composition} $\V$-environments map variables
to $\V$-things, so they do not {\it a priori} support identities and
composition. However, by assuming extra structure we can obtain
identities and composition for environments:\bob{Need to define the
  $\forallb{\dotto}$ notation and mention when these are real id/comp}

\begin{lemma}[Identity environment]\label{thm:env-id}
  Given a function $\mathrm{vr} : \forallb{{\sqni} \dotto \V}$, for any
  $\Gamma$ we have an environment of type $\Gamma \env\V \Gamma$.
\end{lemma}

\begin{lemma}[Composition of environments]\label{thm:env-comp}
  Given a function
  \begin{displaymath}
    \mathrm{lift} : \plr{\rho : \grP\gamma \env\U \grQ\delta} \to \forallb{\V\,\grQ\delta \dotto \W\,\grP\gamma}
  \end{displaymath}
  then we can compose environments $\rho : \Gamma \env\U \Delta$ and
  $\rho' : \Delta \env\V \Theta$ into an environment
  $\rho; \rho' : \Gamma \env\W \Theta$.
\end{lemma}

\begin{corollary}\label{thm:env-postren}
  Given an environment $\rho : \Gamma \env\U \Delta$ and a renaming
  $\sigma : \Delta \env\sqni \Theta$, we can get an environment of type
  $\Gamma \env\U \Theta$.
  When $\U = {\sqni}$, this gives composition of renamings.
\end{corollary}
% \begin{proof}
%   By \cref{thm:env-comp}, with $\mathrm{lift}$ being given exactly by the
%   action of $\rho$ on variables.
% \end{proof}

\bob{Forward reference to where we use identity and composition}

% \begin{example}
%   We can derive the following instances of environment composition.
%   \begin{itemize}
%     \item If $\V = \W = {\vdash}$, then $\mathrm{lift}$ is given by a
%       syntactic traversal.
%       For example, if $\U = {\sqni}$, we need the action of renaming on terms
%       to show that a renaming followed by a substitution composes to a
%       substitution.
%       If $\U = {\vdash}$, then the action of substitution on terms gives us that
%       substitutions compose.
%     \item More generally, if $\V = {\vdash}$ and we have a semantics from
%       $\U$ to $\W$, then $\mathrm{lift}$ can be given by the semantic traversal
%       of terms.
%       This shows that a substitution and a semantics can turn a
%       $\U$-environment into a $\W$-environment.
%   \end{itemize}
% \end{example}
