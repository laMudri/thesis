Let $\Ann$ be a posemiring. Our framework represents well typed and
$\Ann$-resourced terms \emph{intrinsically}. Intrinsic typing means
that we represent well typed and $\Ann$-resourced terms (and
\emph{only} those) as inhabitants of an inductive family
$\grR\gamma \vdash A$ indexed by resource contexts $\grR$, type
contexts $\gamma$, and types $A$. We represent typing contexts as
trees with types at the leaves, and resource contexts as functions
that assign elements of $\Ann$ to members of the context. Using trees
instead of lists for typing contexts has the advantage that extension
of a context by multiple variables does not lead to complex counting
argument to access the pre-existing variables.\todo{left and right
  constructors for paths}

\begin{figure}
  \begin{displaymath}
    \begin{prooftree}
      \hypo{x : \grR\gamma \sqni A}
      \infer1{\AgdaInductiveConstructor{var}~x : \grR\gamma \vdash A}
    \end{prooftree}
  \end{displaymath}
  \begin{displaymath}
    \begin{prooftree}
      \hypo{t : \grR\gamma, \gr1A \vdash B}
      \infer1{\AgdaInductiveConstructor{⊸I}~t : \grR\gamma \vdash A \multimap B}
    \end{prooftree}
    \qquad
    \begin{prooftree}
      \hypo{p : \grR \leq \grP + \grQ}
      \hypo{s : \grP\gamma &\vdash A \multimap B}
      \infer[no rule]1{t : \grQ\gamma &\vdash A \phantom{{} \multimap B}}
      \infer2{\AgdaInductiveConstructor{⊸E}~p~s~t : \grR\gamma \vdash B}
    \end{prooftree}
  \end{displaymath}
  \begin{displaymath}
    \begin{prooftree}
      \hypo{t : \grR\gamma \vdash A_i}
      \infer1{\AgdaInductiveConstructor{⊕I}_i~t : \grR\gamma \vdash A_0 \oplus A_1}
    \end{prooftree}
    \qquad
    \begin{prooftree}
      \hypo{p : \grR \leq \grP + \grQ}
      \infer[no rule]1{s : \grP\gamma \vdash A \oplus B}
      \hypo{t : \grQ\gamma, \gr1A &\vdash C}
      \infer[no rule]1{u : \grQ\gamma, \gr1B &\vdash C}
      \infer2{\AgdaInductiveConstructor{⊕E}~p~s~t~u : \grR\gamma \vdash C}
    \end{prooftree}
  \end{displaymath}
  \begin{displaymath}
    \begin{prooftree}
      \hypo{p : \grR \leq \gr r\grP}
      \hypo{t : \grP\gamma \vdash A}
      \infer2{\AgdaInductiveConstructor{!I}~p~t : \grR\gamma \vdash \oc_{\gr r} A}
    \end{prooftree}
    \qquad
    \begin{prooftree}
      \hypo{p : \grR \leq \grP + \grQ}
      \hypo{s : \grP\gamma \vdash \oc_{\gr r} A}
      \infer[no rule]1{t : \grQ\gamma, \gr rA \vdash C}
      \infer2{\AgdaInductiveConstructor{!E}~p~s~t : \grR\gamma \vdash C}
    \end{prooftree}
  \end{displaymath}
  \caption{A fragment of \name{}}\label{fig:lr}
\end{figure}

\figref{fig:lr} presents a prototypical example of a system that our
framework can represent.

As a prototypical example of the kind of system that our framework can
handle, we

To motivate the way our framework represents






Need to introduce:
\begin{enumerate}
\item Intrinsically typed syntax
\item Intrinsically typed semiring annotated syntax
\item Context modification
\end{enumerate}




To build our universe of semiring annotated syntaxes with binding, we
need to closely examine the structures underlying the kinds of typing
rules present in these systems. We take a fragment of Wood and Atkey's
\name{} calculus, enough to show the key details, presented in
Figure \ref{fig:lr}.


In common with other mechanisations of syntax using dependent types,
we use a

These rules are the constructors of a datatype
$(\vdash) : \mathrm{Ctx} \to \mathrm{Ty} \to \mathrm{Set}$.



In keeping with our Agda formalisation, we have used an
\emph{intrinsically typed} syntax. Each typing rule is presented as a
constructor whose arguments are the premises above the line. Typed
terms are represented as nothing more than their typing derivations,
constructed from the constructors. To account for the semiring
annotations, the rules are

The key features we wish to describe from this

The $\AgdaInductiveConstructor{var}$ rule constructs variables,



In systems with Linear types, a crucial distinction is made between
multiplicative and additive typing rules that does not arise in
non-substructural systems. The distinction is exemplified by the
Intuitionistic Linear Logic introduction rules for \emph{tensor}
products $A \otimes B$ and \emph{with} products $A \with B$:
\begin{mathpar}
  \inferrule* [right=$\otimes$-Intro]
  {\Gamma \vdash s : A \\ \Delta \vdash t : B}
  {\Gamma \vdash s \otimes t : A \otimes B}

  \inferrule* [right=$\with$-Intro]
  {\Gamma \vdash s : A \\ \Gamma \vdash t : B}
  {\Gamma \vdash s \with t : A \with B}
\end{mathpar}
In the \TirName{$\otimes$-Intro} rule, the two premises have disjoint
contexts $\Gamma$ and $\Delta$ that are combined in the conclusion,
indicating that the terms $s$ and $t$ refer to separate
resources. This form of rule is referred to as
\emph{multiplicative}. In the \TirName{$\with$-Intro} rule, the two
premises share the context $\Gamma$, indicating that the terms $s$ and
$t$ may refer to the same resources.

In systems that use semiring annotations to track resource usage, the
distinction between multiplicative and additive still remains but now
contexts in multiplicative rules are combined by adding their semiring
annotations together.
