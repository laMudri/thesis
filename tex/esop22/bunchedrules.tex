We now let $\Ann$ be an arbitrary posemiring. Our framework represents
well typed and $\Ann$-usaged terms \emph{intrinsically}. Intrinsic
typing means that we represent well typed and $\Ann$-usaged terms
(and \emph{only} those) as inhabitants of an inductive family
$\grR\gamma \vdash A$ indexed by usage context $\grR$, type
context $\gamma$, and type $A$. We represent typing contexts as
trees with types at the leaves, and usage contexts as functions
that assign elements of $\Ann$ to members of the context. Using trees
instead of lists for typing contexts has the advantage that extension
of a context by multiple variables does not lead to complex counting
arguments to access the pre-existing variables.\todo{left and right
  constructors for paths}

\begin{figure}
  \begin{displaymath}
    \begin{prooftree}
      \hypo{x : \grR\gamma \sqni A}
      \infer1{\AgdaInductiveConstructor{var}~x : \grR\gamma \vdash A}
    \end{prooftree}
  \end{displaymath}
  \begin{displaymath}
    \begin{prooftree}
      \hypo{t : \grR\gamma, \gr1A \vdash B}
      \infer1{\AgdaInductiveConstructor{⊸I}~t : \grR\gamma \vdash A \multimap B}
    \end{prooftree}
    \qquad
    \begin{prooftree}
      \hypo{p : \grR \leq \grP + \grQ}
      \hypo{s : \grP\gamma &\vdash A \multimap B}
      \infer[no rule]1{t : \grQ\gamma &\vdash A \phantom{{} \multimap B}}
      \infer2{\AgdaInductiveConstructor{⊸E}~p~s~t : \grR\gamma \vdash B}
    \end{prooftree}
  \end{displaymath}
  \begin{displaymath}
    \begin{prooftree}
      \hypo{t : \grR\gamma \vdash A_i}
      \infer1{\AgdaInductiveConstructor{⊕I}_i~t : \grR\gamma \vdash A_0 \oplus A_1}
    \end{prooftree}
    \qquad
    \begin{prooftree}
      \hypo{p : \grR \leq \grP + \grQ}
      \infer[no rule]1{s : \grP\gamma \vdash A \oplus B}
      \hypo{t : \grQ\gamma, \gr1A &\vdash C}
      \infer[no rule]1{u : \grQ\gamma, \gr1B &\vdash C}
      \infer2{\AgdaInductiveConstructor{⊕E}~p~s~t~u : \grR\gamma \vdash C}
    \end{prooftree}
  \end{displaymath}
  \begin{displaymath}
    \begin{prooftree}
      \hypo{p : \grR \leq \gr r\grP}
      \hypo{t : \grP\gamma \vdash A}
      \infer2{\AgdaInductiveConstructor{!I}~p~t : \grR\gamma \vdash \oc{\gr r} A}
    \end{prooftree}
    \qquad
    \begin{prooftree}
      \hypo{p : \grR \leq \grP + \grQ}
      \hypo{s : \grP\gamma \vdash \oc{\gr r} A}
      \infer[no rule]1{t : \grQ\gamma, \gr rA \vdash C}
      \infer2{\AgdaInductiveConstructor{!E}~p~s~t : \grR\gamma \vdash C}
    \end{prooftree}
  \end{displaymath}
  \caption{A prototypical posemiring-usaged system}\label{fig:lr}
\end{figure}

\figref{fig:lr} presents a prototypical example of a system that our
framework can represent, which is a subsystem of the \name{} system of
\cite{WA20}. Each rule is given as a constructor: the premises are
named $p$, $s$, $t$, etc., and the conclusion is a constructor applied
to those variables. Variables are represented intrinsically as members
of the type $\grR\gamma \sqni A$, which is a proof that the type $A$
appears in the typing context, $i : \gamma \ni A$, together with a
proof that $\grR \leq \langle i \rvert$.  Expanding the vector
notation, the latter condition says that the selected variable $i$
must have a usage annotation $\leq \gr1$ in $\grR$, while all other
variables must have a usage annotation $\leq \gr0$. The
\AgdaInductiveConstructor{var} rule imports variables into terms.

The remaining rules are the introduction and elimination rules for
three type constructors: \AgdaInductiveConstructor{⊸I} and
\AgdaInductiveConstructor{⊸E} for function types $A \multimap B$ where
the bound variable is annotated with $\gr 1$ for ``use once'';
\AgdaInductiveConstructor{⊕I} and \AgdaInductiveConstructor{⊕E} for
sum types $A \oplus B$; and \AgdaInductiveConstructor{!I} and
\AgdaInductiveConstructor{!E} for a $\Ann$-annotated exponential
modality $\oc_{\gr r} A$.

There are two key observations to make about this system, which will
guide the way we build our generic framework for $\Ann$-annotated
substructural systems:
\begin{enumerate}
\item Every rule repeats the typing context $\gamma$ throughout its
  premises and conclusion. The only time the typing context is
  modified is to add additional variables in the rules that bind fresh
  variables (\AgdaInductiveConstructor{⊸I},
  \AgdaInductiveConstructor{⊕E}, \AgdaInductiveConstructor{!E}).
\item Rules with multiple typing premises must describe how the
  resources of the conclusion (always denoted $\grR$) are distributed
  across the premises. In the \AgdaInductiveConstructor{⊸E} rule, the
  resources are separated into two parts $\grP$ and $\grQ$ for the
  premises. This is an example of a \emph{multiplicative} rule in the
  terminology of Linear Logic \cite{girard87linear}. In the
  \AgdaInductiveConstructor{⊕E} we see an example of an
  \emph{additive} rule, where the resource context $\grQ$ is shared
  between the premises $t$ and $u$\footnote{There is an unfortunate
    clash of terminology here: multiplicative rules \emph{add} their
    resources, while additive rules \emph{share} their
    resources.}. The \AgdaInductiveConstructor{!I} rule uses scaling
  by $\gr r$ of the resources of the premise.
\end{enumerate}

% \begin{figure}
%   \begin{align*}
%     \dot1\,\grR &\coloneqq 1 \\
%     (T \dottimes U)\,\grR &\coloneqq T\,\grR \times U\,\grR \\
%     % (T \dotto U)\,\grR &\coloneqq T\,\grR \to U\,\grR \\
%     I^*\,\grR &\coloneqq \grR \leq \gr0 \\
%     (T \sep U)\,\grR &\coloneqq \Sigma \grP,\grQ.~\plr{\grR \leq \grP + \grQ}
%                        \times T\,\grP \times U\,\grQ \\
%     (\gr r \cdot T)\,\grR &\coloneqq \Sigma \grP.~\plr{\grR \leq \gr r\grP}
%                        \times T\,\grP \\
%     % (T \wand U)\,\grP &\coloneqq \Pi \grQ,\grR.~\plr{\grR \leq \grP + \grQ}
%     %                    \to T\,\grQ \to U\,\grR
%   \end{align*}
%   \caption{The bunched connectives for combining premises.}
%   \label{fig:bunched}
% \end{figure}

These observations indicate a way to regularise and streamline the
presentation of this system. Instead of treating each premise and the
conclusion as having potentially unrelated typing and usage
constraints, we make use of combinators for combining premises that will
relate their usage and typing contexts to the conclusion by
construction.
This idea comes from the work of \citet{RPKV20}, including the $\dotto$ and
$\wand$ connectives we introduce later.
To handle binders, which introduce variables, we make
use of a combinator that adds a variable with a given $Ann$-annotation
to an ambient context, without having to explicitly mention the parts
of the context that have not changed. This technique is already
present in some paper presentations of type systems, and is used as a
simplifying technique by Allais et al.~\cite{AACMM21}.
To manage how usage annotations are distributed between premises, we use the
separating ($*$) and sharing ($\dot{\times}$) conjunction connectives from
Bunched Implications~\cite{oHP99}.
To handle the \AgdaInductiveConstructor{!I} rule, we will need a
\emph{scaling} modality, $\gr r \cdot {-}$. The semantics of the bunched
connectives we will use in this paper are:
  \begin{align*}
    \dot1\,\grR &\coloneqq 1 \\
    (T \dottimes U)\,\grR &\coloneqq T\,\grR \times U\,\grR \\
    (T \dotto U)\,\grR &\coloneqq T\,\grR \to U\,\grR \\
    I^*\,\grR &\coloneqq \grR \leq \gr0 \\
    (T \sep U)\,\grR &\coloneqq \Sigma \grP,\grQ.~\plr{\grR \leq \grP + \grQ}
                       \times T\,\grP \times U\,\grQ \\
    (\gr r \cdot T)\,\grR &\coloneqq \Sigma \grP.~\plr{\grR \leq \gr r\grP}
                       \times T\,\grP \\
    (T \wand U)\,\grP &\coloneqq \Pi \grQ,\grR.~\plr{\grR \leq \grP + \grQ}
                        \to T\,\grQ \to U\,\grR
  \end{align*}
where the function connectives $\dotto$ and $\wand$ are not used in typing rules, but are used in the rest of the framework.
An important point to note is that bunched combinators induce
\emph{linear} combinations of substructures, in the sense of the
linear algebra of posemirings described in the previous section.

\begin{figure}
  \begin{displaymath}
    \begin{prooftree}[comb]
      \hypo{x : {} \sqni A}
      \infer1{\AgdaInductiveConstructor{var}~x : {} \vdash A}
    \end{prooftree}
    \qquad
    \begin{prooftree}[comb]
      \hypo{t : \gr1A \vdash B}
      \infer1{\AgdaInductiveConstructor{⊸I}~t : {} \vdash A \multimap B}
    \end{prooftree}
    \qquad
    \begin{prooftree}[comb]
      \hypo{(t : {} \vdash A \multimap B)}
      \hypo{\sep}
      \hypo{(s : {} \vdash A)}
      \infer3{\AgdaInductiveConstructor{⊸E}~t~s : {} \vdash B}
    \end{prooftree}
  \end{displaymath}
  \begin{displaymath}
    \begin{prooftree}[comb]
      \hypo{t : {} \vdash A_i}
      \infer1{\AgdaInductiveConstructor{⊕I}_i~t : {} \vdash A_0 \oplus A_1}
    \end{prooftree}
    \qquad
    \begin{prooftree}[comb]
      \hypo{(s : {} \vdash A \oplus B)}
      \hypo{\sep}
      \hypo{((t : \gr1A \vdash C)}
      \hypo{\dottimes}
      \hypo{(u : \gr1B \vdash C))}
      \infer5{\AgdaInductiveConstructor{⊕E}~s~t~u : {} \vdash C}
    \end{prooftree}
  \end{displaymath}
  \begin{displaymath}
    \begin{prooftree}[comb]
      \hypo{t : \gr r \cdot \plr{\vdash A}}
      \infer1{\AgdaInductiveConstructor{!I}~t : {} \vdash \oc_{\gr r} A}
    \end{prooftree}
    \qquad
    \begin{prooftree}[comb]
      \hypo{(s : {} \vdash \oc_{\gr r} A)}
      \hypo{\sep}
      \hypo{(t : \gr rA \vdash C)}
      \infer3{\AgdaInductiveConstructor{!E}~s~t : {} \vdash C}
    \end{prooftree}
  \end{displaymath}
  \caption{The prototypical system of \figref{fig:lr} restated in terms of bunched combinators.}
  \label{fig:lr-comb}
\end{figure}

\figref{fig:lr-comb} shows our prototypical system restated with
implicit contexts and the bunched combinators. The inductive family is
now denoted $\vdash A$, only mentioning context extensions, as we do
in the rules \AgdaInductiveConstructor{⊸I},
\AgdaInductiveConstructor{⊕E} and \AgdaInductiveConstructor{!E}. Thus,
in the \AgdaInductiveConstructor{var} rule, the context is completely
supressed. The \AgdaInductiveConstructor{⊸I} just has to state that a
new variable with usage annotation $\gr1$ is added to the
context. The \AgdaInductiveConstructor{⊸E} rule uses the separating
conjunction ($*$) to combine the premises, indicating that the
resources required by the two premises are added together for the
conclusion. The \AgdaInductiveConstructor{⊕E} rule demonstrates the
sharing conjunction $\dot\times$: the scrutinee term $s$ and the
clause terms $t, u$ are combined by separating conjunction, because
their resources must be combined, but the clause terms are combined by
the sharing conjunction, because they must use the same resources.

Bunched combinators, along with supression of unchanged typing
contexts, leads to a more streamlined presentation of the system
without the clutter of explicit usage context
inequalities. However, the larger advantage for us is that systems are
constructed using these combinators \emph{automatically} admit
renaming, substitution, and other scope-, type-, and usage-safe
traversals. If we were to allow arbitrary modification of the context
in premises, this would not always be possible, since there would be
no guarantee that a substitution (for instance) could be ``pushed'' up
from a conclusion to the premises. As we will see in \secref{sec:env},
our generic notion of environment (e.g., a simultaneous substitution)
is essentially a linear transformation, and so automatically commutes
with the linear combinations of premises induced by the bunched
connectives. This is the key to our generic results for all of the
systems describable in our framework.
