\documentclass[sigplan,10pt,anonymous,review]{acmart}
\settopmatter{printfolios=true,printccs=false,printacmref=false}

\citestyle{acmauthoryear}
\bibliographystyle{ACM-Reference-Format}

\usepackage[conor]{agda}
\usepackage{catchfilebetweentags}
\usepackage{cleveref}
\usepackage{cmll}
\usepackage{ebproof}
\usepackage{mathrsfs}
\usepackage{mathtools}
\usepackage{newunicodechar}
\usepackage{stmaryrd}
\setlength{\marginparwidth}{2cm}
\usepackage{todonotes}
\usepackage{turnstile}

\definecolor{use}{HTML}{008000}
\newcommand\gr[1]{{\color{use}#1}}
\newcommand\grctx[1]{\gr{\mathcal{#1}}}
\newcommand\grP{\grctx P}
\newcommand\grQ{\grctx Q}
\newcommand\grR{\grctx R}
\newcommand\grPprime{\grP\gr'}
\newcommand\grQprime{\grQ\gr'}
\newcommand\grRprime{\grR\gr'}
\newcommand\name{\ensuremath{\lambda\grR}}
\newcommand\grctxsub[2]{\grctx{#1}_{\gr{#2}}}
\newcommand\grPe{\grctxsub P e}
\newcommand\grPf{\grctxsub P f}
\newcommand\grPl{\grctxsub P l}
\newcommand\grPr{\grctxsub P r}
\newcommand\grQe{\grctxsub Q e}
\newcommand\grQf{\grctxsub Q f}
\newcommand\grQl{\grctxsub Q l}
\newcommand\grQr{\grctxsub Q r}
\newcommand\grRl{\grctxsub R l}
\newcommand\grRr{\grctxsub R r}
\newcommand\ps{\mathit{ps}}
\newcommand\qs{\mathit{qs}}
\newcommand\rs{\mathit{rs}}
\newcommand\dotto{\mathrel{\dot\to}}
\newcommand\dotlr{\mathrel{\dot\leftrightarrow}}
\newcommand\dottimes{\mathbin{\dot\times}}
%\newcommand\dotplus{\mathbin{\dot+}}
\newcommand\wand{\mathrel{\mathord{-}\hspace{-0.75ex}*}}
\newcommand\sep{\mathbin{*}}
\newcommand\Boxzp{\Box^{0{+}}}
\newcommand\Boxzpt{\Box^{0{+}{*}}}
\providecommand\U{}
\renewcommand\U{\mathcal U}
\newcommand\V{\mathcal V}
\newcommand\W{\mathcal W}
\providecommand\C{}
\renewcommand\C{\mathcal C}
\newcommand\sqin{\mathrel{\mathrlap{\sqsubset}{\mathord{-}}}}
\newcommand\sqni{\mathrel{\mathrlap{\sqsupset}{\mathord{-}}}}
\newcommand\Ann{\mathscr R}

\renewcommand\land{~\wedge~}
\renewcommand\lor{~\vee~}
\newcommand\rel{\mathrel{\mathord{\to}\hspace{-2.25ex}+}}

\DeclareMathOperator\Set{Set}
\DeclareMathOperator\obj{Obj}
\DeclareMathOperator\List{List}
\let\hom\relax
\DeclareMathOperator\hom{Hom}
\DeclareMathOperator\id{id}
\DeclareMathOperator\sub{Sub}

\newcommand\env[1]{\stackrel{#1}\Longrightarrow}

\usepackage{mleftright}
\newcommand\lr[3]{\mleft#1{#2}\mright#3}
\newcommand\sem[1]{\lr\llbracket{#1}\rrbracket}
\newcommand\size[1]{\lr\lvert{#1}\rvert}
\newcommand\plr[1]{\lr({#1})}
\newcommand\blr[1]{\lr[{#1}]}
\newcommand\forallb[1]{\forall\blr{~#1~}}
\newcommand\alr[1]{\lr\langle{#1}\rangle}
\newcommand\bra[1]{\lr\langle{#1}\rvert}
\newcommand\ket[1]{\lr\lvert{#1}\rangle}

\newcommand\leO{\;{\leq}0}
\newcommand\leI{\;{\leq}1}

\DeclareMathOperator{\lin}{lin}
\DeclareMathOperator{\intu}{int}
\newcommand\lnl{\ensuremath{\mathrm{L/nL}}}
\newcommand\vdashL{\mathrel{\vdash_{\mathcal L}}}
\newcommand\vdashC{\mathrel{\vdash_{\mathcal C}}}

\ebproofnewstyle{comb}{separation=0.75em}
\ebproofset{right label template=\TirName{\inserttext}}

\newenvironment{eqns}{\begin{array}{r@{\hspace{0.3em}}c@{\hspace{0.3em}}l}}{\end{array}}

\newcommand\PsiDot[1]{%
  \AgdaBound{$\Psi$}\AgdaSpace{}\AgdaSymbol{.}\AgdaField{#1}}

\newcommand\qto{\mathbin{`\!\to}}

\newcommand\Lock{\text{\faLock}}

\newcommand\oiw{\mathrm{\gr{01\upomega}}}

\newcommand\colour{%
{\color{AgdaField}co}{\color{AgdaFunction}lo}{\color{AgdaDatatype}ur}%
}

\usepackage[T1]{fontenc}

\newunicodechar{λ}{\ensuremath{\mathnormal\lambda}}
\newunicodechar{ρ}{\ensuremath{\mathnormal\rho}}
\newunicodechar{→}{\ensuremath{\mathnormal\to}}
\newunicodechar{∀}{\ensuremath{\mathnormal\forall}}
\newunicodechar{∃}{\ensuremath{\mathnormal\exists}}
\newunicodechar{ι}{\ensuremath{\mathnormal\iota}}
\newunicodechar{·}{\ensuremath{\mathnormal\cdot}}
\newunicodechar{⊸}{\ensuremath{\mathnormal\multimap}}
\newunicodechar{⊕}{\ensuremath{\mathnormal\oplus}}
\newunicodechar{⊗}{\ensuremath{\mathnormal\otimes}}
\newunicodechar{⅋}{\ensuremath{\mathnormal\parr}}
\newunicodechar{─}{\text{---}}
\newunicodechar{│}{\ensuremath{\mid}}
\newunicodechar{ᶜ}{\ensuremath{\mathnormal{^c}}}
\newunicodechar{ᵉ}{\ensuremath{\mathnormal{^e}}}
\newunicodechar{ᵏ}{\ensuremath{\mathnormal{^k}}}
\newunicodechar{ₗ}{\ensuremath{\mathnormal{_l}}}
\newunicodechar{ₘ}{\ensuremath{\mathnormal{_m}}}
\newunicodechar{ₙ}{\ensuremath{\mathnormal{_n}}}
\newunicodechar{ᵣ}{\ensuremath{\mathnormal{_r}}}
\newunicodechar{ʳ}{\ensuremath{\mathnormal{^r}}}
\newunicodechar{ˢ}{\ensuremath{\mathnormal{^s}}}
\newunicodechar{ᵗ}{\ensuremath{\mathnormal{^t}}}
\newunicodechar{ᵛ}{\ensuremath{\mathnormal{^v}}}
\newunicodechar{ʷ}{\ensuremath{\mathnormal{^w}}}
\newunicodechar{ᴹ}{\ensuremath{\mathnormal{^M}}}
\newunicodechar{ᴿ}{\ensuremath{\mathnormal{^R}}}
\newunicodechar{↙}{\ensuremath{\mathnormal\swarrow}}
\newunicodechar{↘}{\ensuremath{\mathnormal\searrow}}
\newunicodechar{⊢}{\ensuremath{\mathnormal\vdash}}
\newunicodechar{⊨}{\ensuremath{\mathnormal\vDash}}
\newunicodechar{⟦}{\ensuremath{\mathnormal\llbracket}}
\newunicodechar{⟧}{\ensuremath{\mathnormal\rrbracket}}
\newunicodechar{✴}{\ensuremath{\mathnormal*}}
\newunicodechar{ℓ}{\ensuremath{\mathnormal\ell}}
\newunicodechar{α}{\ensuremath{\mathnormal\alpha}}
\newunicodechar{Γ}{\ensuremath{\mathnormal\Gamma}}
\newunicodechar{γ}{\ensuremath{\mathnormal\gamma}}
\newunicodechar{Δ}{\ensuremath{\mathnormal\Delta}}
\newunicodechar{δ}{\ensuremath{\mathnormal\delta}}
\newunicodechar{ε}{\ensuremath{\mathnormal\epsilon}}
\newunicodechar{Θ}{\ensuremath{\mathnormal\Theta}}
\newunicodechar{μ}{\ensuremath{\mathnormal\mu}}
\newunicodechar{Σ}{\ensuremath{\mathnormal\Sigma}}
\newunicodechar{σ}{\ensuremath{\mathnormal\sigma}}
\newunicodechar{ω}{\ensuremath{\mathnormal\omega}}
\newunicodechar{∈}{\ensuremath{\mathnormal\in}}
\newunicodechar{′}{\ensuremath{\mathnormal'}}
\newunicodechar{≡}{\ensuremath{\mathnormal\equiv}}
\newunicodechar{⊤}{\ensuremath{\mathnormal\top}}
\newunicodechar{⊥}{\ensuremath{\mathnormal\bot}}
\newunicodechar{▹}{\ensuremath{\mathnormal\triangleright}}
\newunicodechar{₁}{\ensuremath{\mathnormal{_1}}}
\newunicodechar{□}{\ensuremath{\mathnormal\Box}}
\newunicodechar{○}{\ensuremath{\mathnormal\bigcirc}}
\newunicodechar{𝓒}{\ensuremath{\C}}
\newunicodechar{𝓥}{\ensuremath{\V}}
\newunicodechar{∘}{\ensuremath{\mathnormal\circ}}
\newunicodechar{≤}{\ensuremath{\mathnormal\leq}}
\newunicodechar{◇}{\ensuremath{\mathnormal\diamond}}
\newunicodechar{ℕ}{\ensuremath{\mathbb N}}
\newunicodechar{ℤ}{\ensuremath{\mathbb Z}}
\newunicodechar{⁺}{\ensuremath{\mathnormal{^+}}}
\newunicodechar{⁻}{\ensuremath{\mathnormal{^-}}}
\newunicodechar{⊔}{\ensuremath{\mathnormal\sqcup}}
\newunicodechar{⇒}{\ensuremath{\mathnormal\Rightarrow}}
\newunicodechar{Ψ}{\ensuremath{\mathnormal\Psi}}
\newunicodechar{⋯}{\ensuremath{\mathnormal\cdots}}
\newunicodechar{∞}{\ensuremath{\mathnormal\infty}}
\newunicodechar{≈}{\ensuremath{\mathnormal\approx}}
\newunicodechar{⟨}{\ensuremath{\mathnormal\langle}}
\newunicodechar{⟩}{\ensuremath{\mathnormal\rangle}}
\newunicodechar{∣}{\ensuremath{\mathnormal\vert}}
\newunicodechar{⋂}{\ensuremath{\mathnormal\bigcap}}
\newunicodechar{∙}{\ensuremath{\mathnormal\bullet}}
\newunicodechar{∼}{\ensuremath{\mathnormal\sim}}

% Characters that are different from what appears in the source code
\newunicodechar{∋}{\ensuremath{\mathnormal\sqni}}
\newunicodechar{ℑ}{\ensuremath{\mathnormal{I^*}}}
\newunicodechar{⇛}{\ensuremath{\mathnormal{\Longrightarrow}}}
\newunicodechar{∩}{\ensuremath{\mathnormal\dottimes}}
\newunicodechar{∧}{\ensuremath{\mathnormal\dottimes}}
\newunicodechar{⒈}{\ensuremath{\mathnormal{\dot1}}}
\newunicodechar{⇴}{\ensuremath{\mathnormal\dotto}}
\newunicodechar{⇥}{\ensuremath{\mathnormal\wand}}


\def\genericlr{../../generic-lr/src/processed-latex}
\def\Syntaxtex{\genericlr/Generic/Linear/Syntax.tex}
\def\Interpretationtex{\genericlr/Generic/Linear/Syntax/Interpretation.tex}
\def\Maptex{\genericlr/Generic/Linear/Syntax/Interpretation/Map.tex}
\def\Termtex{\genericlr/Generic/Linear/Syntax/Term.tex}
\def\Semanticstex{\genericlr/Generic/Linear/Semantics.tex}
\def\Syntactictex{\genericlr/Generic/Linear/Semantics/Syntactic.tex}
\def\Renamingtex{\genericlr/Generic/Linear/Renaming.tex}
\def\PaperExamplestex{\genericlr/Generic/Linear/Example/PaperExamples.tex}
\def\UsageChecktex{\genericlr/Generic/Linear/Example/UsageCheck.tex}
\def\Snippetstex{../../agda/processed-latex/Snippets.tex}

\begin{document}

\title{A Framework for Substructural Type Systems}

\author{James Wood}
\orcid{0000-0002-8080-3350}
\affiliation{%
  \institution{University of Strathclyde}%
  \city{Glasgow}%
  \country{UK}%
}
\email{james.wood.100@strath.ac.uk}

\author{Robert Atkey}
\orcid{0000-0002-4414-5047}
\affiliation{%
  \institution{University of Strathclyde}%
  \city{Glasgow}%
  \country{UK}%
}
\email{robert.atkey@strath.ac.uk}

\keywords{substructural, linear, metatheory, type system, mechanisation}

\begin{abstract}
  Mechanisation of programming language research is of growing interest, and
  the act of mechanising type systems and their metatheory is generally becoming
  easier as new techniques are invented.
  However, state-of-the-art techniques mostly rely on \emph{structurality} of
  the type system --- that weakening, contraction, and exchange are admissible
  and variables can be used unrestrictedly once assumed.
  Linear logic, and many related subsequent systems, provide motivations for
  breaking some of these assumptions.

  We present a framework for mechanising the metatheory of certain
  substructural type systems, in a style resembling mechanised metatheory of
  structural type systems.
  The framework covers a wide range of simply typed syntaxes with semiring
  usage annotations, via a metasyntax of typing rules.
  The metasyntax for the premises of a typing rule is related to bunched logic,
  featuring both sharing and separating conjunction, roughly corresponding to
  the additive and multiplicative features of linear logic.
  The bunched flavour is carried over into the semantics, together with the use
  of basic linear algebra constructs.
  For example, \emph{environments} are presented equivalently as values
  accumulated via separating conjunction, and as functions from variables to
  values supported by linear maps.
  Producing a generic semantic traversal has us combine environments with a
  separating implication, producing a Kripke function space of the form
  $\Box(A \wand B)$.
  From the generic semantic traversal, we derive totally generic renaming and
  substitution operations, a specific denotational semantics, and a
  syntax-generic \emph{usage elaborator} which greatly facilitates writing
  concrete terms.
\end{abstract}

\maketitle

\section{Introduction}\label{sec:intro}
\chapter{Introduction}

This thesis advances the frontier of programming languages work that can be done
in a proof assistant.
I will elaborate on both of these aspects in the following paragraphs.

The main aim of programming language research is to find common patterns in
computer programs and represent these patterns via programming language
features.
As a basic example, consider functions in, say, a low level language like C, in
contrast to what these functions are compiled to in assembly language/machine
code.
In order to support function calls, the C runtime system manages a
\emph{call stack}.
A function call is compiled to a sequence of instructions which store the return
address and the values of the arguments on the stack, move the stack pointer,
and jump to the compiled code of the function.
A return from a function then places the return value into the appropriate
register, moves the stack pointer back, and jumps back to the return address.
Advances in programming language techniques are evaluated by the usefulness of
behaviours captured and goodness of the mathematical properties of the
abstractions produced.

Within programming language research, type theory is a methodology for producing
abstractions.
A type system lets us enforce in our language complex invariants which maintain
abstraction boundaries while also ideally being machine-checkable.
For example, we give functions function type to abstract away their
implementation via call stacks and jumps and so on.

In this thesis, the invariants of interest revolve around restricting the usage
of variables.
I introduce this topic thoroughly in \cref{sec:linear}, but in brief\ldots

The work of this thesis relies upon type theory in two distinct ways.
Firstly, the main objects of study are programming languages with interesting
type systems.
Secondly, type theory provides the basis of the proof assistant Agda I use to
implement the aforementioned programming languages and operations upon them.

As for the other topic of this thesis,
a \emph{proof assistant}, also known as an \emph{interactive theorem prover}, is
a piece of software that allows for the encoding of mathematical definitions,
theorems, constructions, and proofs, and furthermore check that such encoded
proofs are correct and that such encoded constructions are well formed.
To truly by interactive, i.e.\ to actually assist, a proof assistant will
usually have a user interface which can read partial proofs, display
information about what more proof needs to be given, and provide actions that
will help complete the proof.

Proof assistants have seen increasing use in programming language research in
recent years.
The most obvious reason why working in a proof assistant is seen as beneficial
is that it ensures correctness.
If the proof assistant accurately implements a suitable mathematical foundation,
then any theorem proved in the proof assistant is guaranteed to be a true
theorem of that foundation.
These guarantees of correctness are particularly important when working with
combinatorially complex mathematical objects, proofs about which often require
the consideration of a large number of cases.
Programming language syntaxes are often such complex objects, motivating the use
of proof assistants when studying programming languages.

A second reason to use proof assistants is for the assistance they provide when
exploring a mathematical theory.
When we make a new definition, we may want to test how it works in a special
case, or what constructions it allows us to perform.
In a proof assistant, the assistance tools give us immediate feedback as to what
moves are and aren't allowed.
For example, if we define a complex type system, a proof assistant will let us
interactively build typing derivations, making clear any side conditions and
types of subderivations as we go.
Also, as I do later in this thesis, a proof assistant allows us to build a very
general theory, and practically use that theory directly in more specific cases
without losing rigour.

Thirdly, analogously to how a strong static type system can give us more
confidence when refactoring a program, the constant checking of proofs in a
proof assistant gives us the confidence to change definitions and lemmas knowing
that we will be guided towards the parts of our theory that need to be
correspondingly changed.
This can help if we are developing a new programming language with a changing
specification.

Finally, many proof assistants --- including Agda~\citep{Agda}, which I use in
this thesis --- double as programming languages themselves.
This means that we can write progams and prove properties of them using the same
tool.
Also, many theorems proven in such a proof assistant have computational content.
For example, if we prove a normalisation theorem for a programming language,
this will typically yield a (verified) normalisation algorithm for it, which we
can really run on a computer.
As such, a development in a proof assistant can provide a reference
implementation of a programming language, or even --- as with Idris
2~\citep{Brady21}, Lean 4~\citep{deMU21}, and Cedille~\citep{GRS16} --- the
actual implementation of a programming language.

FIXME: the start of this chapter is directly from the ESOP22 paper.

In this paper, we treat the metatheory of a class of substructural type
systems related to linear logic~\cite{girard87linear}.
This class is variously known as
\emph{coeffectful}~\cite{PetricekOM14,Granule18},
\emph{quantitative}~\cite{BrunelGMZ14,Atkey18}, or
\emph{resource-aware}~\cite{GhicaS14},
or is given no particular name~\cite{reed10distance,abadi99core},
and generalises bounded linear logic to track variable usage with semiring
annotations.
In all of these systems, we have some ambient semiring $\Ann$, and in the
judgements of the type system, variables are annotated by elements of $\Ann$
describing \emph{how} that variable can be used.
The additive structure of $\Ann$ gives the ability to count, or otherwise
accumulate, usages of variables in multiple subterms.
The multiplicative structure gives rise to a form of modality, for example
allowing multiple or unlimited reuse, or movement between security levels, in
the type system.

The aspect of such systems we tackle here is their basic metatheory and
mechanisation thereof.

We build upon both the general structural framework of
\citet{AACMM21} and the substructural techniques of \citet{WA21}.
The way \citeauthor{AACMM21} consolidate and codify mechanisation techniques for
propositional natural deduction systems based on intrinsically typed syntax and
de Bruijn indices, we aim to replicate for linear-like systems based on
semiring usage annotations.
By picking a trivial semiring, our work can subsume that of
\citeauthor{AACMM21}, except for the many pieces of machinery we have not yet
ported to this new framework.

Our work complements that of \citet{Granule18} on the Granule programming
language.
Where Granule focuses on writing programs \emph{in} the language and running
them, we focus on metatheoretic reasoning \emph{about} type systems.
%As such, whereas Granule has a convenient syntax, performant type-checker,
%interpreter

Our work is similar in scope to that of \citet{LicataSR17}, though we work in
a natural deduction style rather than a sequent calculus style.
Where \citeauthor{LicataSR17} are much more agnostic in terms of
substructurality --- allowing for non-commutative and bunched logics ---
we are much more agnostic in terms of syntax.
The system of \citeauthor{LicataSR17} is essentially a single calculus,
supporting ``product'' ($\mathrm F$) types and ``function'' ($\mathrm U$)
types, parametrised on a \emph{mode theory} describing its structural rules.
For this system, they derive the strong result of cut elimination.
Meanwhile, we leave syntax design to the user, and consequently can only
guarantee substitution (which we can only get because of our commitment to
natural deduction).

\section{Outline of the thesis}

This thesis proceeds as follows.
The next two chapters, \cref{sec:simple,sec:linearity}, are introductory in
nature, and cover two largely independent strands of prior work.
In \cref{sec:simple}, I introduce existing methods of representing and reasoning
about type systems in proof assistants based on dependent type theory.
I start from well established representations of well scoped and well typed
terms, and develop these towards a recent approach to environment-based
semantics given by \citet{AACMM21}.
In \cref{sec:linearity}, I discuss the challenges faced when one extends a
treatment of a simple type system, such as that given in \cref{sec:simple}, to
modal and linear type systems.
We see that modal and linear type systems apparently violate some of the nice
properties of the simply typed $\lambda$-calculus we required in
\cref{sec:simple}.
I present a solution for intuitionistic S4 modal logic, but leave a solution for
linear logic to the following chapters.

In the following two chapters, \cref{sec:semirings,sec:ren-sub-lr}, I present a
calculus $\name$ parametrised by a partially ordered semiring of \emph{usage
annotations}.
In \cref{sec:semirings}, I define the calculus, give some possible extensions,
and show that it subsumes intuitionistic S4 modal logic and Intuitionistic
Linear Logic.
In \cref{sec:ren-sub-lr}, I show that $\name$ enjoys generalised versions of the
nice properties required in \cref{sec:simple}, and I proceed to give novel
definitions of simultaneous substitutions and their action on $\name$ terms.
These two chapters are adapted from the work of \citet{WA21}.

The remaining three main chapters,
\cref{sec:framework,sec:semantics,sec:example-semantics}, adapt the syntactic
and semantic framework of \citet{AACMM21}, as presented at the end of
\cref{sec:simple}, to semiring-annotated calculi.
\Cref{sec:framework,sec:semantics} generalise the work on $\name$ presented in
\cref{sec:semirings,sec:ren-sub-lr}, respectively.
\Cref{sec:framework} shows how to formally describe the syntax of an arbitrary
semiring-annotated calculus, following the constructions used in
\cref{sec:semirings}.
\Cref{sec:semantics} then provides the generic environment-based semantic
traversal on such syntaxes, providing renaming and substitution as per
\cref{sec:ren-sub-lr} for all syntaxes as special cases of the generic
traversal.
\Cref{sec:example-semantics} then gives further example uses of the generic
traversal.

Finally, I conclude with \cref{sec:conc}, which discusses the achievements of
this thesis and openings for future work.

\section{Naming and notation conventions}

I assume familiarity with the Curry-Howard correspondence throughout this
thesis.
I make no distinction between logics and type theories, and use terminology from
each interchangeably.
Each following bullet point lists a collection of synonyms.

\begin{itemize}
  \item assumption, hypothesis, variable
  \item derivation, proof, term
  \item proposition, formula, type
  \item connective, type former
\end{itemize}

I carry out mechanised constructions and proofs in the proof assistant and
programming language Agda~\citep{Agda}.
Agda is based on Martin-L\"{o}f's intensional dependent type theory, so I
similarly present non-mechanised constructions and proofs assuming a foundation
given by dependent type theory, in a style inspired by the HoTT
Book~\citep{hottbook}.

\section{Agda primer}

I use the proof assistant and programming language Agda throughout this thesis,
with Agda code being used particularly in \cref{sec:simple} and the later
chapters.
As such, it is important for the reader to be able to read basic Agda syntax in
order to benefit from the parts of the exposition that reside in code listings.
The syntax of Agda is broadly similar to that of Haskell, and relatively close
to that of Standard ML, OCaml, and Coq's Gallina sublanguage.
I will assume that the reader is able to read basic Haskell code, and spend most
time explaining differences thereof.

\subsection{Lexing and parsing}

Agda is extremely liberal in its set of allowed names.
There is just a single lexical class (unlike in Haskell, where, for example,
constructors start with a capital letter and definitions start with a lowercase
letter), and names can be any string of Unicode characters except whitespace and
special characters \verb|.;{}()@"|, apart from those strings reserved as
keywords or literals.
Therefore, we can introduce names like
\AgdaFunction{0x-+-$\uplambda\rightarrow$}
to stand for any kind of identifiable thing.
With such free-form names, ample spacing is required between identifiers.
For example, while \AgdaNumber{0}\AgdaSpace{}\AgdaDatatype{$\leq$}\AgdaSpace{}%
\AgdaNumber{1} is a possible expression containing three
identifiers, \AgdaInductiveConstructor{0$\leq$1} is a single
valid identifier.
Only the special characters may appear next to names without being separated by
whitespace.

A character with unique behaviour in Agda's syntax is the underscore
(\AgdaSymbol{\_}).
Within a name, an underscore signifies that the name will function as a mixfix
operator, allowing for an argument in the position of the underscore.
For example, the full name of the \AgdaDatatype{$\leq$} operator used in the
previous paragraph is \AgdaDatatype{\_$\leq$\_}, signifying that it can take an
argument to its left and its right.
We can also introduce closed operators, like \AgdaDatatype{[\_]}, which can take
an argument between the square brackets (e.g.\ \AgdaDatatype{[}\AgdaSpace{}%
\AgdaNumber{1}\AgdaSpace{}\AgdaDatatype{]}, with spaces still being important).
Mixfix operators can be partially applied by leaving underscores in the name in
the application.
For example, \AgdaDatatype{\_$\leq$}\AgdaSpace{}\AgdaNumber{1} could be the
predicate asserting that a number is less than or equal to 1.

On its own, an underscore has a completely different meaning, which can depend
on context.
In patterns, an underscore has the same meaning as it has in Haskell and ML ---
it holds the place of a pattern variable, but does not name that variable.
In expressions, an underscore stands for an unspecified subterm which will be
solved by unification.
The solving of unspecified terms is canonical and respects $\beta\eta$-equality,
unlike in Coq.

Spacing is important particularly important when dealing with underscores.
For example,
\AgdaDatatype{\_$\leq$}\AgdaSpace{}\AgdaSymbol{\_} (with a space after the
\AgdaDatatype{$\leq$} but not before) standing for the predicate asserting that
a number is less than or equal to some unspecified number.

Like Haskell, Agda's syntax is indentation-sensitive.
The distinctions conveyed by indentation are largely obvious or intuitive to
human readers (for example, allowing for line-continuation or delineating nested
modules), so I will not discuss them explicitly here.

\subsection{Functions, $\Pi$-types}\label{sec:pi-types}

Simple function types take the form
\AgdaArgument{A}\AgdaSpace{}\AgdaSymbol{$\to$}\AgdaSpace{}\AgdaArgument{B},
coinciding with Haskell's syntax.
Also as in Haskell and ML, the function arrow nests to the right.
However, Agda has a termination checker ensuring that all defineable functions
are total, so many Haskell functions do not have a corresponding Agda function.

The key feature distinguishing Agda from Haskell is the presence of arbitrary
dependent types, including dependent function types ($\Pi$-types).
The basic syntax for $\Pi$-types is
\AgdaSymbol{(}\AgdaBound{x}\AgdaSpace{}\AgdaSymbol{:}\AgdaSpace{}%
\AgdaArgument{A}\AgdaSymbol{)}\AgdaSpace{}\AgdaSymbol{$\to$}\AgdaSpace{}%
\AgdaArgument{B},
where variable \AgdaBound{x} can occur free in expression \AgdaArgument{B}.
However, there are several syntactic conveniences I use throughout the code
listings.
For one, iterated $\Pi$-types can be abbreviated so that
\AgdaSymbol{(}\AgdaBound{x}\AgdaSpace{}\AgdaSymbol{:}\AgdaSpace{}%
\AgdaArgument{A}\AgdaSymbol{)}\AgdaSpace{}\AgdaSymbol{$\to$}\AgdaSpace{}%
\AgdaSymbol{(}\AgdaBound{y}\AgdaSpace{}\AgdaSymbol{:}\AgdaSpace{}%
\AgdaArgument{B}\AgdaSymbol{)}\AgdaSpace{}\AgdaSymbol{$\to$}\AgdaSpace{}%
\AgdaArgument{C}
is written just
\AgdaSymbol{(}\AgdaBound{x}\AgdaSpace{}\AgdaSymbol{:}\AgdaSpace{}%
\AgdaArgument{A}\AgdaSymbol{)}\AgdaSpace{}%
\AgdaSymbol{(}\AgdaBound{y}\AgdaSpace{}\AgdaSymbol{:}\AgdaSpace{}%
\AgdaArgument{B}\AgdaSymbol{)}\AgdaSpace{}\AgdaSymbol{$\to$}\AgdaSpace{}%
\AgdaArgument{C},
omitting the first arrow.
For another, prefixing an arrow with the \AgdaSymbol{$\forall$} symbol allows us
to omit domain types.
For example,
\AgdaSymbol{$\forall$}\AgdaSpace{}\AgdaBound{x}\AgdaSpace{}\AgdaSymbol{$\to$}%
\AgdaSpace{}\AgdaArgument{B}
is equivalent to
\AgdaSymbol{(}\AgdaBound{x}\AgdaSpace{}\AgdaSymbol{:}\AgdaSpace{}%
\AgdaSymbol{\_}\AgdaSymbol{)}\AgdaSpace{}\AgdaSymbol{$\to$}%
\AgdaSpace{}\AgdaArgument{B}.
Notice that this is a very different type to
\AgdaBound{x}\AgdaSpace{}\AgdaSymbol{$\to$}\AgdaSpace{}\AgdaArgument{B},
which is a non-dependent function type equivalent to
\AgdaSymbol{(}\AgdaSymbol{\_}\AgdaSpace{}\AgdaSymbol{:}\AgdaSpace{}%
\AgdaBound{x}\AgdaSymbol{)}\AgdaSpace{}\AgdaSymbol{$\to$}\AgdaSpace{}%
\AgdaArgument{B}.
When writing
\AgdaSymbol{$\forall$}\AgdaSpace{}\AgdaBound{x}\AgdaSpace{}\AgdaSymbol{$\to$}%
\AgdaSpace{}\AgdaArgument{B},
we assume that the occurrence of \AgdaBound{x} in \AgdaArgument{B} tells us what
type \AgdaBound{x} should have (i.e.\ there is enough information to solve the
underscore in
\AgdaSymbol{(}\AgdaBound{x}\AgdaSpace{}\AgdaSymbol{:}\AgdaSpace{}%
\AgdaSymbol{\_}\AgdaSymbol{)}\AgdaSpace{}\AgdaSymbol{$\to$}%
\AgdaSpace{}\AgdaArgument{B}).

Just like in Haskell, functions in Agda can be introduced via
\AgdaSymbol{$\uplambda$}-abstractions and clausal definitions, and are applied
by juxtaposition.
Agda also includes \emph{extended} \AgdaSymbol{$\uplambda$}-abstractions,
introduced via equivalent syntaxes
\AgdaSymbol{$\uplambda$}\AgdaSpace{}\AgdaKeyword{where}\AgdaSpace{}%
\AgdaBound{x}\AgdaSpace{}\AgdaSymbol{$\to$}\AgdaSpace{}\AgdaArgument{M} and
\AgdaSymbol{$\uplambda$}\AgdaSpace{}\AgdaSymbol{\{}\AgdaSpace{}%
\AgdaBound{x}\AgdaSpace{}\AgdaSymbol{$\to$}\AgdaSpace{}\AgdaArgument{M}%
\AgdaSpace{}\AgdaSymbol{\}},
which allow for pattern-matching on the variable \AgdaBound{x} (or all of the
variables, if there are multiple variables).

Agda allows for arbitrary function arguments to be marked as \emph{implicit} by
replacing the round brackets in the type by curly braces.
For example, if we have
\AgdaBound{f}\AgdaSpace{}\AgdaSymbol{:}\AgdaSpace{}%
\AgdaSymbol{\{}\AgdaBound{x}\AgdaSpace{}\AgdaSymbol{:}\AgdaSpace{}%
\AgdaArgument{A}\AgdaSymbol{\}}\AgdaSpace{}\AgdaSymbol{$\to$}\AgdaSpace{}%
\AgdaArgument{B},
then the argument to \AgdaBound{f} is implicit.
Being implicit means that an occurrence of \AgdaBound{f} is treated as if it has
been applied to an underscore, giving the expression \AgdaBound{f} the type
$\AgdaArgument{B}[\AgdaSymbol{\_}/\AgdaBound{x}]$ (i.e.\ \AgdaArgument{B} with
\AgdaSymbol{\_} substituted in for \AgdaBound{x}; the substitution syntax is not
part of Agda syntax).
An implicit argument can also be given explicitly in two ways.
The first of a series of implicit arguments can be given by surrounding the
argument in curly braces, and any other implicit arguments in the series can be
given by including the name of the argument.
For example,
\AgdaBound{f}\AgdaSpace{}\AgdaSymbol{\{}\AgdaSymbol{\_}\AgdaSymbol{\}},
\AgdaBound{f}\AgdaSpace{}\AgdaSymbol{\{}\AgdaArgument{x}\AgdaSpace{}%
\AgdaSymbol{=}\AgdaSpace{}\AgdaSymbol{\_}\AgdaSymbol{\}}, and just
\AgdaBound{f} are all equivalent expressions, and the underscore can be filled
in in either of the first two expressions to provide an actual value for the
implicit argument.
Implicit arguments are usually left out of \AgdaSymbol{$\uplambda$}-abstractions
and clausal definitions, but can be bound to names and pattern-matched on using
the same syntax as in expressions.

There are a few other places in the syntax using single curly braces, all of
which have meanings related to implicit arguments.
I also make a small amount of use of double curly braces
(\AgdaSymbol{\{\{} and \AgdaSymbol{\}\}}), which denote arguments which are to
be solved by instance resolution.
Instance resolution is very similar to Haskell's typeclass resolution ---
finding non-canonical solutions based on the instances in scope.

Agda uses $\Pi$-types where in Haskell we would use polymorphism.
For example, we can define an identity function as below.
The definition relies on quantifying over terms of type \AgdaPrimitive{Set},
i.e.\ (small) types.
This definition also gives an example of defining a function with an implicit
argument (\AgdaBound{X}), which can typically be inferred from either the
kargument type or the return type, so can be omitted.

\ExecuteMetaData[\SnippetsTwotex]{id0}

An unfortunate feature of the definition \AgdaFunction{id$_0$} is that we cannot
apply it to the expression \AgdaPrimitive{Set}, because \AgdaPrimitive{Set}
contains only small types, and itself is a large type.
We can work around these size issues using \emph{universe level polymorphism},
as in the following definition.

\ExecuteMetaData[\SnippetsTwotex]{id}

Universe levels start at \AgdaPrimitive{0$\ell$}, with \AgdaPrimitive{Set} being
an alias for \AgdaPrimitive{Set}\AgdaSpace{}\AgdaPrimitive{0$\ell$} (and also
\AgdaPrimitive{Set$_0$}).
Larger levels can be produced with the successor operator \AgdaPrimitive{suc},
and we can take the least upper bound of two levels using the operator
\AgdaPrimitive{\_$\sqcup$\_}.

\subsection{Data types}

Agda's \AgdaKeyword{data}-declarations are similar in scope to Haskell's, with
the addition of indexing by terms of arbitrary type.
\AgdaKeyword{data}-declarations give us indexed inductive sum-of-product types.

All \AgdaKeyword{data}-declarations use GADT syntax.
The body of a declaration comprises a list of constructor names paired with
their types.
Where two constructors have the same type, they may be written on the same line
with their names separated by whitespace, as I do with the two constructors of
\AgdaDatatype{Bool} below.
\AgdaDatatype{Bool} has two constructors --- \AgdaInductiveConstructor{true} and
\AgdaInductiveConstructor{false} --- both of which have type
\AgdaDatatype{Bool}.
\AgdaDatatype{$\mathbb N$} also has two constructors, where
\AgdaInductiveConstructor{zero} has type \AgdaDatatype{$\mathbb N$} and
\AgdaInductiveConstructor{suc} is inductive, with type
\AgdaDatatype{$\mathbb N$}\AgdaSpace{}\AgdaKeyword{$\to$}\AgdaSpace{}%
\AgdaDatatype{$\mathbb N$}.

\noindent
\begin{minipage}[t]{0.5\textwidth}
  \ExecuteMetaData[\SnippetsThreetex]{Bool}
\end{minipage}
\begin{minipage}[t]{0.5\textwidth}
  \ExecuteMetaData[\SnippetsThreetex]{Nat}
\end{minipage}

\AgdaDatatype{Bool} and \AgdaDatatype{$\mathbb N$} are both types, and indeed
small types, as we can see by the fact that they are annotated to have type
\AgdaPrimitive{Set}.
We can also use \AgdaKeyword{data}-declarations to define type \emph{families}
in various ways.
The simplest is to add \emph{parameters}, as in the type family
\AgdaDatatype{List} below.
Parameters always appear to the left of the colon of the first line of the
\AgdaKeyword{data}-declaration, and are constant throughout the
\AgdaKeyword{data}-declaration.
Variables to the left of the colon can appear in the body of the
\AgdaKeyword{data}-declaration without further quantification.

\ExecuteMetaData[\SnippetsThreetex]{List}

Slightly more flexible than parameters are \emph{Protestant indices}%
\footnote{The terminology of Protestant/Catholic indices comes orally from Conor
McBride. I am not aware of a written reference.}.
Protestant indices also appear to the left of the colon, and also must appear
unmodified in the \emph{return type} of all of the constructors.
However, they may take different values in inductive appearances of the type
family in the argument types of constructors.
Protestant indices give a generalisation of polymorphic
recursion to indices of arbitrary type~\citep{Mycroft84,Henglein93}.

I give two examples of type families with Protestant indices.
The first, \AgdaDatatype{NestedList} is standard from the polymorphic recursion
literature.
It is worth noting at this point that Agda permits overloading of constructors,
which are disambiguated by the type family they are being used to construct.
This overloading allows \AgdaDatatype{List} and \AgdaDatatype{NestedList} to
have constructors with the same names without confusion.
The second example, \AgdaDatatype{ScopedTerm} is a data structure representing
well scoped untyped $\lambda$-calculus terms.
The Protestant index \AgdaBound{s} describes the number of variables in scope,
which increases by 1 when we introduce a $\lambda$-abstraction.
I will introduce \AgdaDatatype{Fin}, a type family with a specified natural
number of inhabitants, in the next set of examples.
As a syntactic note, in the type of the \AgdaInductiveConstructor{app}
constructor, I use the two variable names \AgdaBound{M} and \AgdaBound{N}
separated by whitespace to name two arguments with the same type.

\ExecuteMetaData[\SnippetsThreetex]{NestedList}
\ExecuteMetaData[\SnippetsThreetex]{ScopedTerm}

The most general way to make a type family is to introduce a
\emph{Catholic index}.
The types of Catholic indices are specified to the right of the colon, and can
be instantiated arbitrarily throughout the \AgdaKeyword{data}-declaration.
Catholic indices are not in scope for the body of the
\AgdaKeyword{data}-declaration, so the values filling them may need to be
quantified over in each constructor.
When this quantification is over a large type, like \AgdaPrimitive{Set}, the
type family being defined will itself need to be large, e.g.\ inhabiting
\AgdaPrimitive{Set$_1$}.
This is a major reason for not defining types like \AgdaDatatype{List} and
\AgdaDatatype{NestedList} using Catholic indices.

I give two examples of type families with Catholic indices.
The first is the \AgdaDatatype{Fin} family, as used in \AgdaDatatype{ScopedTerm}
above.
By inspection of the return types of the constructors, there is no way to
produce a canonical inhabitant of
\AgdaDatatype{Fin}\AgdaSpace{}\AgdaInductiveConstructor{zero}.
For
\AgdaDatatype{Fin}\AgdaSpace{}\AgdaSymbol(\AgdaInductiveConstructor{suc}%
\AgdaSpace{}\AgdaBound{n}\AgdaSymbol),
we can potentially use either of the constructors.
Either we use \AgdaInductiveConstructor{zero} to get a canonical inhabitant, or
if we can make a number with a smaller bound (i.e.\ an inhabitant of
\AgdaDatatype{Fin}\AgdaSpace{}\AgdaBound{n}), we can use
\AgdaInductiveConstructor{suc} to produce a larger number.

\ExecuteMetaData[\SnippetsThreetex]{Fin}

The second example of a type family with Catholic indices is more general in
nature.
Below I define \emph{propositional equality}, written
\AgdaDatatype{\_$\equiv$\_}.
It has two parameters and one Catholic index (though the standard library
version of propositional equality I use throughout this thesis has an extra
level parameter for the sake of universe level polymorphism).
The constructor \AgdaInductiveConstructor{refl} constructs an inhabitant of
\AgdaArgument{M}\AgdaSpace{}\AgdaDatatype{\_$\equiv$\_}\AgdaSpace{}%
\AgdaArgument{N} only when \AgdaArgument{N} is definitionally equal to
\AgdaArgument{M} (because terms are considered ``the same'' to the type checker
exactly when they are definitionally equal).
Notice that \AgdaInductiveConstructor{refl} does not quantify over \AgdaBound{x}
because \AgdaBound{x} is already in scope as a parameter.

\ExecuteMetaData[\SnippetsThreetex]{Eq}

It is through type families like \AgdaDatatype{\_$\equiv$\_} that we can state
and prove mathematical theorems in Agda.
In the following subsection, I show how to use such indexed type families.

\subsection{Clausal definitions}

Clausal definitions of functions in Agda look very similar to their equivalents
in Haskell.
However, definitions in Agda regularly make use of
\emph{dependent pattern matching}, which is our primary way of using indexed
data types.
Recursive definitions are also conservatively checked for termination.

I will explain the salient aspects of clausal function definitions via two
examples.
The first, unimaginatively named \AgdaFunction{lemma}, shows a simple case where
pattern matching modifies the context through unification of Catholic indices.
The second, named \AgdaFunction{elim-Fin-zero}, gives an example of proper
dependent pattern matching.

In the following definition \AgdaFunction{lemma}, we want to chase equations in
order to prove that \AgdaBound{x} is propositionally equal to \AgdaBound{z}.
We start with the following incomplete definition, where the expression
\AgdaHole{\{ \}0} marks an interation point, or \emph{hole}, in the program, to
which we can apply interactive commands to complete the program.

\begin{code}
\>[0]\AgdaFunction{lemma}\AgdaSpace{}%
\AgdaSymbol{:}\AgdaSpace{}%
\AgdaSymbol{∀}\AgdaSpace{}%
\AgdaSymbol{\{}\AgdaBound{A}\AgdaSpace{}%
\AgdaSymbol{:}\AgdaSpace{}%
\AgdaPrimitive{Set}\AgdaSymbol{\}}\AgdaSpace{}%
\AgdaSymbol{\{}\AgdaBound{x}\AgdaSpace{}%
\AgdaBound{y}\AgdaSpace{}%
\AgdaBound{z}\AgdaSpace{}%
\AgdaSymbol{:}\AgdaSpace{}%
\AgdaBound{A}\AgdaSymbol{\}}\AgdaSpace{}%
\AgdaSymbol{→}\AgdaSpace{}%
\AgdaBound{x}\AgdaSpace{}%
\AgdaOperator{\AgdaDatatype{≡}}\AgdaSpace{}%
\AgdaBound{y}\AgdaSpace{}%
\AgdaSymbol{→}\AgdaSpace{}%
\AgdaBound{z}\AgdaSpace{}%
\AgdaOperator{\AgdaDatatype{≡}}\AgdaSpace{}%
\AgdaBound{y}\AgdaSpace{}%
\AgdaSymbol{→}\AgdaSpace{}%
\AgdaBound{x}\AgdaSpace{}%
\AgdaOperator{\AgdaDatatype{≡}}\AgdaSpace{}%
\AgdaBound{z}\<%
\\
\>[0]\AgdaFunction{lemma}\AgdaSpace{}%
\AgdaBound{p}\AgdaSpace{}%
\AgdaBound{q}\AgdaSpace{}%
\AgdaSymbol{=}\AgdaSpace{}%
\AgdaHole{\{ \}0}\<%
\end{code}

As a first step, I choose to match on the variable
\AgdaBound{p}\AgdaSpace{}\AgdaSymbol:\AgdaSpace{}%
\AgdaBound{x}\AgdaSpace{}%
\AgdaOperator{\AgdaDatatype{≡}}\AgdaSpace{}%
\AgdaBound{y}.
The only applicable pattern is \AgdaInductiveConstructor{refl}.
Doing this match has the effect of unifying \AgdaBound{y} --- which is taking
the position of the Catholic index of \AgdaDatatype{\_$\equiv$\_} --- with
\AgdaBound{x} --- which is the value of the index specified in the type of
\AgdaInductiveConstructor{refl}.
Local variables act as unification variables, so the unification succeeds with
most general unifier $[x \coloneqq x, y \coloneqq x]$.
Therefore, the type of \AgdaBound{q} becomes
\AgdaBound{z}\AgdaSpace{}%
\AgdaOperator{\AgdaDatatype{≡}}\AgdaSpace{}%
\AgdaBound{x}.

\begin{code}
\>[0]\AgdaFunction{lemma}\AgdaSpace{}%
\AgdaSymbol{:}\AgdaSpace{}%
\AgdaSymbol{∀}\AgdaSpace{}%
\AgdaSymbol{\{}\AgdaBound{A}\AgdaSpace{}%
\AgdaSymbol{:}\AgdaSpace{}%
\AgdaPrimitive{Set}\AgdaSymbol{\}}\AgdaSpace{}%
\AgdaSymbol{\{}\AgdaBound{x}\AgdaSpace{}%
\AgdaBound{y}\AgdaSpace{}%
\AgdaBound{z}\AgdaSpace{}%
\AgdaSymbol{:}\AgdaSpace{}%
\AgdaBound{A}\AgdaSymbol{\}}\AgdaSpace{}%
\AgdaSymbol{→}\AgdaSpace{}%
\AgdaBound{x}\AgdaSpace{}%
\AgdaOperator{\AgdaDatatype{≡}}\AgdaSpace{}%
\AgdaBound{y}\AgdaSpace{}%
\AgdaSymbol{→}\AgdaSpace{}%
\AgdaBound{z}\AgdaSpace{}%
\AgdaOperator{\AgdaDatatype{≡}}\AgdaSpace{}%
\AgdaBound{y}\AgdaSpace{}%
\AgdaSymbol{→}\AgdaSpace{}%
\AgdaBound{x}\AgdaSpace{}%
\AgdaOperator{\AgdaDatatype{≡}}\AgdaSpace{}%
\AgdaBound{z}\<%
\\
\>[0]\AgdaFunction{lemma}\AgdaSpace{}%
\AgdaInductiveConstructor{refl}\AgdaSpace{}%
\AgdaBound{q}\AgdaSpace{}%
\AgdaSymbol{=}\AgdaSpace{}%
\AgdaHole{\{ \}0}\<%
\end{code}

The next step is to match on \AgdaBound{q}.
This similarly unifies \AgdaBound{z} and \AgdaBound{x}, making the conclusion
type
\AgdaBound{z}\AgdaSpace{}%
\AgdaOperator{\AgdaDatatype{≡}}\AgdaSpace{}%
\AgdaBound{z}.
Finally, this conclusion type is in the image of the
\AgdaInductiveConstructor{refl} constructor, so we may fill the hole with
\AgdaInductiveConstructor{refl}.

\ExecuteMetaData[\SnippetsThreetex]{lemma}

Full \emph{dependent} pattern matching, as described by \citet{MM04}, is when
the unification of indices described above takes account of constructors.
In particular, the constructors of a data type satisfy the ``no confusion''
property --- constructors are injective and mutually disjoint.
Where we encounter disjoint constructors during unification, we may dismiss the
corresponding case as impossible.
Consider the following example (\AgdaFunction{elim-Fin-zero}).
We start with an argument
\AgdaBound{i}\AgdaSpace{}\AgdaSymbol:\AgdaSpace{}\AgdaDatatype{Fin}\AgdaSpace{}%
\AgdaInductiveConstructor{zero}, and consider which constructors could possibly
construct such a value.
However, as noted earlier, both constructors of \AgdaDatatype{Fin} target
successor values of the index, from which \AgdaInductiveConstructor{zero} is
disjoint.
Therefore, both cases are impossible.
The notation when all cases are impossible is to place empty round brackets
\AgdaSymbol{()} in the place of the impossible argument, and to not provide a
righthand side to the clause.

\ExecuteMetaData[\SnippetsThreetex]{elim-Fin-zero}

As an example of the injectivity of constructors, the obvious example is to
internalise the proof of injectivity for a given constructor, as I do in
\AgdaFunction{suc-injective}.
We start with an argument
\AgdaBound{p}\AgdaSpace{}\AgdaSymbol:\AgdaSpace{}%
\AgdaInductiveConstructor{suc}\AgdaSpace{}\AgdaBound{m}%
\AgdaSpace{}\AgdaOperator{\AgdaDatatype{$\equiv$}}\AgdaSpace{}%
\AgdaInductiveConstructor{suc}\AgdaSpace{}\AgdaBound{n}
and match on it.
This time, we do have a possible pattern --- \AgdaInductiveConstructor{refl} ---
but working out how to change the context relies on unifying
\AgdaInductiveConstructor{suc}\AgdaSpace{}\AgdaBound{m} with
\AgdaInductiveConstructor{suc}\AgdaSpace{}\AgdaBound{n}.
We are justified in doing this, with most general unifier
$[m \coloneqq m, n \coloneqq m]$, because \AgdaInductiveConstructor{suc} is
injective (with respect to propositional equality).
If the checker for dependent pattern matching did not know that
\AgdaInductiveConstructor{suc} was injective --- for example, if it were instead
a defined function --- then the unification would fail.
This leads to the intuition that constructors and variables are well behaved
with respect to dependent pattern matching, while other expressions are not.

\ExecuteMetaData[\SnippetsThreetex]{suc-injective}

Ordinarily, each clause of a definition gives rise to a \emph{definitional}
equation between its lefthand side and righthand side.
In intensional type theory, as implemented by Agda, definitional and
propositional equality are contrasted to each other.
Definitional equality corresponds to a decidable fragment of the natural
equational theory of the type theory.
As such, definitional equality is an entirely metatheoretic notion, and we can
neither assume nor prove directly definitional equations within the language.
Definitional equality is sometimes also called \emph{judgemental equality},
because it forms a judgement which plays a part in the rules of the type theory.
As well as from the clauses of definitions, we also get definitional equations
from $\beta$-reductions of $\lambda$-abstractions and $\eta$-laws of functions
and records.
Because the type checker treats definitionally equal terms equivalently, we are
able to refactor up to definitional equality without changing any downstream
code.

On the other hand, propositional equality is a notion internal to the language,
as we have seen by defining propositional equality (\AgdaDatatype{\_$\equiv$\_})
and proving things about it (\AgdaFunction{lemma}).
Propositional equality is sometimes known as \emph{typal equality} or
\emph{mathematical equality}.
The latter name comes from the fact that propositional equality is the closest
notion to what mathematicians usually call \emph{equality}, because, for
example, it allows us to prove things like
\AgdaBound{m}\AgdaSpace{}\AgdaFunction{+}\AgdaSpace{}\AgdaBound{n}%
\AgdaSpace{}\AgdaDatatype{$\equiv$}\AgdaSpace{}%
\AgdaBound{n}\AgdaSpace{}\AgdaFunction{+}\AgdaSpace{}\AgdaBound{m} for all
natural numbers \AgdaBound{m} and \AgdaBound{n}.
Propositional equality satisfies Leibniz' law, meaning that an inhabitant of a
type \AgdaBound{A} can be coerced into an inhabitant of any type propositionally
equal to \AgdaBound{A}.
However, this cast requires marking in the code, so is less convenient to use
than definitional equality.

Definitional equality between two terms implies their propositional equality,
because exactly when two terms are definitionally equal, the type checker is
happy to accept \AgdaInductiveConstructor{refl} as a proof.
This relationship between the two is simple, but can still be deceptive.
For example, consider the notion of injectivity with respect to definitional
and propositional equality.
A function $f$ is injective (with respect to some notion of equality $\approx$)
when, for all $x$ and $y$, we have $f\,x \approx f\,y \to x \approx y$.
Because $\approx$ appears both covariantly and contravariantly in this
definition, we have implications in neither direction between definitional
injectivity and propositional injectivity.
Indeed, we can find examples of all four possibilities:
constructors are injective in both senses; type formers, like \AgdaDatatype{Fin}
and \AgdaDatatype{List}, are definitionally injective but not propositionally
injective;
\ExecuteMetaData[\SnippetsThreetex]{double}\ can be proven to be
propositionally injective, but is not definitionally injective because
\AgdaFunction{\_+\_} is not injective;
and nearly everything else is not injective in either sense.

Because the notions of definitional and propositional injectivity are
incomparable, so too are the corresponding unification procedures.
Propositional unification (using only the injectivity of constructors) is used
during dependent pattern matching, while solving of implicit arguments and
underscores in expressions is done by definitional unification.

\subsection{Records, $\Sigma$-types}

While Agda provides built-in basic $\Pi$-types, with special syntax described in
\cref{sec:pi-types}, it does not do the same for $\Sigma$-types.
Instead, the default way to get the functionality of $\Sigma$-types is to
declare record types, similarly to how we get sums via
\AgdaKeyword{data}-declarations.
However, the standard library does provide $\Sigma$-types, via record types,
using the following declaration.

\ExecuteMetaData[\SnippetsThreetex]{Sigma}

As does the standard library, I will begin to use universe level polymorphism in
these example definitions.
Here, \AgdaBound{a} and \AgdaBound{b} are the levels of the two projections.
The level of the record type must be at least the level of the type of each
field, and in this case, the smallest such level is
\AgdaBound{a}\AgdaSpace{}\AgdaOperator{\AgdaPrimitive{$\sqcup$}}\AgdaSpace{}%
\AgdaBound{b}.
As for the main points of interest in this \AgdaKeyword{record}-declaration, it
contains two fields.
The first, \AgdaField{proj$_1$}, simply has type \AgdaBound{A}.
The second, \AgdaField{proj$_2$}, then has a type dependent on the value of the
first field.
Additionally, we give this record type a named constructor
\AgdaInductiveConstructor{\_,\_}.
Any record type can also be constructed using the more verbose syntax
\begin{code}[inline]%
\>[0]\AgdaKeyword{record}\AgdaSpace{}%
\AgdaSymbol{\{}\AgdaSpace{}%
\AgdaField{proj$_1$}\AgdaSpace{}%
\AgdaSymbol{=}\AgdaSpace{}%
\AgdaHole{\{ \}1}\AgdaSpace{}%
\AgdaSymbol{;}\AgdaSpace{}%
\AgdaField{proj$_2$}\AgdaSpace{}%
\AgdaSymbol{=}\AgdaSpace{}%
\AgdaHole{\{ \}2}\AgdaSpace{}%
\AgdaSymbol{\}}\<%
\end{code}.

The standard library provides various notations for
\AgdaRecord{$\Upsigma$}-types, useful in various situations.
In this thesis, I use \ExecuteMetaData[\SnippetsFourtex]{Sigma-syntax}\ and
\ExecuteMetaData[\SnippetsFourtex]{exists}\ as equivalent notations for
\AgdaRecord{$\Upsigma$}\AgdaSpace{}\AgdaBound{A}\AgdaSpace{}\AgdaBound{B}.
Indeed, the $\eta$-contracted form can be used with \AgdaFunction{$\exists$}, as
in \AgdaFunction{$\exists$}\AgdaSpace{}\AgdaBound{B} (\AgdaSymbol{\textbackslash}
is an alternative notation for \AgdaSymbol{$\uplambda$}, as in Haskell).
\AgdaRecord{$\Upsigma$} also specialises to non-dependent products, as given by
the infix operator \AgdaFunction{\_$\times$\_}.
This is achieved by setting the parameter \AgdaBound{B} to be a constant type
family.
The resulting operator \AgdaFunction{\_$\times$\_}, as well as the non-dependent
function type, behave better than their dependent counterparts with respect to
unification because they allow us to remain in the first order fragment of
higher order unification.

There are two main ways of using the fields of a record.
The first is to put the projections into scope using
\AgdaKeyword{open}\AgdaSpace{}\AgdaRecord{$\Upsigma$}, and then to use the field
names to project out of arbitrary terms of \AgdaRecord{$\Upsigma$}-type.
This is what I will always do when using the \AgdaRecord{$\Upsigma$}-type
family.
Within this paradigm, there are two further notational choices.
Either, we can use the field names as functions, so that
\AgdaBound{z}\AgdaSpace{}\AgdaSymbol{=}\AgdaSpace{}\AgdaField{proj$_1$}%
\AgdaSpace{}\AgdaBound{z}\AgdaSpace{}\AgdaOperator{\AgdaInductiveConstructor,}%
\AgdaSpace{}\AgdaField{proj$_2$}\AgdaSpace{}\AgdaBound{z},
or we can use postfix projections via the space-dot notation, as in
\AgdaBound{z}\AgdaSpace{}\AgdaSymbol{=}\AgdaSpace{}\AgdaBound{z}%
\AgdaSpace{}\AgdaSymbol.\AgdaField{proj$_1$}\AgdaSpace{}%
\AgdaOperator{\AgdaInductiveConstructor,}\AgdaSpace{}\AgdaBound{z}%
\AgdaSpace{}\AgdaSymbol.\AgdaField{proj$_2$}.
I tend to prefer the latter, also using it occasionally in ordinary mathematical
notation (without the space).
Both notations can also be used on the lefthand side of a clausal definition as
\emph{copatterns}.
Copatterns let us think of records as being function-like, with the fields of a
record type being the possible arguments we can pass to such a function.

The second way of using the fields of a record requires a motivating example.
Consider the below definition of the type of semigroups at universe level
\AgdaBound{$\ell$}.
A semigroup has a carrier set, a binary operation on that set, and an
associativity law for that binary operation.

\ExecuteMetaData[\SnippetsFourtex]{Semigroup}

In order to use the fields of \AgdaRecord{Semigroup} in the intended way, we
do not open them into global scope.
Doing so would mean that, for example, \AgdaField{\_$\bullet$\_} would take
three arguments: the semigroup and its two intended arguments.
Instead, we get to the point where we have a semigroup \AgdaBound{G} in scope
and use
\AgdaKeyword{open}\AgdaSpace{}\AgdaRecord{Semigroup}\AgdaSpace{}\AgdaBound{G}
to put into scope the components \emph{of \AgdaBound{G}}.
Then, the name \AgdaField{Carrier} in scope will refer to the carrier set of
\AgdaBound{G}, the name \AgdaField{\_$\bullet$\_} will refer to the binary
operator (which really takes two arguments), et cetera.
Doing this gives the impression of working ``inside'' \AgdaBound{G}, which is
the way I typically work with algebraic sets with structure.

By $\eta$-equality, two inhabitants of a record type are definitionally equal
exactly when they agree definitionally on all fields.
This often makes record types much more convenient to work with than the
corresponding single-constructor data types, which do not enjoy any $\eta$-laws.
Notably, all inhabitants of the record type \AgdaRecord{$\top$} with no fields
are definitionally equal.

Along with \AgdaRecord{$\Upsigma$} and \AgdaRecord{$\top$}, there are two more
general-purpose record types I need to cover which take advantage of two special
features of \AgdaKeyword{record}-declarations (and also
\AgdaKeyword{data}-delcarations, but I use \AgdaKeyword{record}-declarations for
the convenience reason given in the previous paragraph).
The first feature is that the universe level of a record type has a lower bound
(the level of each field) but no upper bound.
Therefore, we can introduce the following declaration \AgdaRecord{Lift}, which
takes a type \AgdaBound{A} at level \AgdaBound{a} and produces an equivalent
type at a potentially higher level
\AgdaBound{a}\AgdaSpace{}\AgdaOperator{\AgdaPrimitive{$\sqcup$}}\AgdaSpace{}%
\AgdaBound{$\ell$}.
This type former is useful in situations which require a type at a specific
level, such as when constructing a type using a function.

\ExecuteMetaData[\SnippetsFourtex]{Lift}

The other interesting property we get from \AgdaKeyword{record}-declarations is
that the resulting type family is definitionally injective in its parameters.
Therefore, record types behave well in the form of unification that solves
implicit arguments.
We can use this property to take any type family \AgdaBound{F} and produce an
equivalent family \AgdaRecord{Wrap}\AgdaSpace{}\AgdaBound{F} which is
definitionally injective.

\ExecuteMetaData[\SnippetsFourtex]{Wrap}

As an example, if we have a variable
\AgdaBound{f}\AgdaSpace{}\AgdaSymbol{:}\AgdaSpace{}\AgdaRecord{Wrap}%
\AgdaSpace{}\AgdaFunction{F}\AgdaSpace{}\AgdaBound{y}
and pass it to a function with a type of the form
\AgdaSymbol{$\forall$}\AgdaSpace{}\AgdaSymbol\{\AgdaBound{x}\AgdaSymbol\}%
\AgdaSpace{}\AgdaSymbol{$\to$}\AgdaSpace{}\AgdaRecord{Wrap}\AgdaSpace{}%
\AgdaFunction{F}\AgdaSpace{}\AgdaBound{x}\AgdaSpace{}\AgdaSymbol{$\to$}%
\AgdaSpace{}\AgdaSymbol{\_},
Agda will successfully unify the type of \AgdaBound{f} with the expected type of
the argument, setting $[x \coloneqq y]$.
However, without the \AgdaRecord{Wrap}, we would need to unify
\AgdaFunction{F}\AgdaSpace{}\AgdaBound{y} with
\AgdaFunction{F}\AgdaSpace{}\AgdaBound{x}, which would fail if \AgdaFunction{F}
were not injective, because there may be multiple acceptable values of
\AgdaBound{x} up to definitional equality.

The version of \AgdaRecord{Wrap} found in Agda's standard library is
significantly more complicated to allow for type families with arbitrarily many
arguments in a convenient syntax, using the $n$-ary functions of
\citet{Allais19}.
The version in the standard library is the one I use in this thesis.
In fact, both versions of the \AgdaRecord{Wrap} type family are the first pieces
of novel work to be presented in this thesis.

\subsection{Colours}

I use the ``Conor colours'' option for Agda syntax highlighting.
This set of colours is inspired by Conor McBride's syntax highlighting for
Epigram 2.
The colour given to a name is determined by the type of declaration that name
is bound to.
The main colours are \AgdaDatatype{blue} for types and type families,
\AgdaField{red} for constructors of data types and fields of records,
\AgdaFunction{green} for definitions which may unfold/compute, and
\AgdaBound{purple} for local variables.

Separately, I use \gr{green} in many places for usage annotations in traditional
typeset mathematical notation.
This usage of green contrasts only with ordinary black text.


\section{Vectors over semirings}\label{sec:algebra}
The basic algebraic structure we concern ourselves with is \emph{partially
ordered semirings}, or \emph{posemirings} for short.
A posemiring is a (not necessarily commutative) semiring on a partially ordered
set, where both operations are monotonic.
As in many similar formalisms, posemiring elements represent usage restrictions,
with addition collecting restrictions from multiple uses, multiplication
handling usage under a modality, and the order giving subsumption of
restrictions, comparable to subtyping.

\begin{definition}
  A \emph{posemiring} is a tuple $(\Ann, \leq, 0, +, 1, *)$ such that
  $(\Ann, \leq)$ is a partially ordered set, $(\Ann, 0, +)$ is a commutative
  monoid, $(\Ann, 1, *)$ is a monoid, $+$ and $*$ are monotonic, and $*$
  distributes over $0$ and $+$ in the usual way on both sides.
\end{definition}

\begin{example}[Zero-one-many]\label{thm:linearity}
  The poset $\{0 > \omega < 1\}$ forms a posemiring under normal numeric
  addition (with $1 + 1 = 1 + \omega = \omega + \omega = \omega$) and
  multiplication (with $\omega * \omega = \omega$).
  This gives us a way to mark whether variables are unused ($0$), used linearly
  ($1$), or used unrestrictedly ($\omega$) in the current (sub)term.
  The ordering says that unrestricted-use variables can also remain unused or
  be used linearly.
\end{example}

\begin{example}[Variance]\label{thm:variance}
  The set
  $\{{\sim\sim}, {\uparrow\uparrow}, {\downarrow\downarrow}, {\wn\wn}\}$,
  with ${\sim\sim}$ at the bottom and ${\wn\wn}$ at the top of the order, forms
  a posemiring with addition being \emph{meet}, $0$ being \emph{top}
  (${\wn\wn}$), $1$ being ${\uparrow\uparrow}$, and multiplication being
  commutative and determined by
  ${\downarrow\downarrow} * {\downarrow\downarrow} = {\uparrow\uparrow}$ and
  ${\sim\sim} * {\downarrow\downarrow} = {\sim\sim} * {\sim\sim} = {\sim\sim}$.
  This gives us a way to track the variance with which variables are used, in
  the aim of all terms being monotonic in their free variables.
  ${\uparrow\uparrow}$ stands for covariance, ${\downarrow\downarrow}$ for
  contravariance, ${\sim\sim}$ for invariance, and ${\wn\wn}$ for a variable
  with no guarantees, in which we must be constant.
\end{example}

An element of a chosen posemiring $\Ann$ describes the usage restrictions on
a variable.
Therefore, a \emph{vector} of elements from $\Ann$ describes the usage
restrictions of a whole context's worth of variables.
From the posemiring operations of $\Ann$, we derive the standard vector
operations of zero, addition, and multiplication by a scalar.
We can also form the standard basis vectors at any given dimension.
From the order on $\Ann$, we get a pointwise order on vectors.

Vectors of a given length form a \emph{module} over the posemiring $\Ann$,
analogously to how vectors over a field form a vector space.

\begin{definition}
  A \emph{(left) module over a posemiring}, given a posemiring $\Ann$, is a
  partially ordered commutative monoid $(M, 0_M, +_M)$ with, for each
  $r \in \Ann$, a pomonoid morphism $r \cdot \plr{-} : M \to M$, such that the
  collection of these respects the posemiring structure on $r$.
  Specifically, for all instantiations of the variables:
  \begin{itemize}
      \item If $r \leq r'$ and $u \leq u'$, then $r \cdot u \leq r' \cdot u'$.
    \item $r \cdot 0_M = 0_M$ and
      $r \cdot \plr{u +_M v} = r \cdot u +_M r \cdot v$.
    \item $0 \cdot u = 0_M$ and
      $\plr{r + s} \cdot u = r \cdot u +_M s \cdot u$.
    \item $1 \cdot u = u$ and
      $\plr{r * s} \cdot u = r \cdot \plr{s \cdot u}$.
  \end{itemize}
\end{definition}

We care to define modules so as to define \emph{module morphisms}, also known
as \emph{linear maps}, which we use extensively when relating two contexts (as
we do, for example, in simultaneous substitution).
For the sake of mechanisation, we choose to define module morphisms
\emph{relationally} rather than \emph{functionally}, giving a somewhat
unfamiliar-looking definition that is equivalent to the usual functional
definition.
The main advantage of this relational approach is that proofs of relatedness
for typical linear maps compose and decompose via data constructors and
pattern matching.
% I'm not sure this is a real difference:
%An auxiliary advantage is that with relations rather than functions, we can
%much more often take advantage of judgemental injectivity, thus making
%unification-based solving of implicits more effective.
%For example, if \AgdaBound{R} is a free variable of relation type, then
%\AgdaInductiveConstructor{refl} serves as a proof of
%\ExecuteMetaData[\Snippetstex]{Rxy-R}{}, solving the underscores as
%\AgdaBound{x} and \AgdaBound{y}, respectively.

\begin{definition}
  A \emph{(relational) linear map} $\Psi$ between modules $M$ and $N$ over a
  posemiring $\Ann$ is a relation $\sim$ on the underlying sets of $M$ and $N$
  satisfying the following laws (with $\to$ standing for implication and
  quantifiers binding most loosely).
  \begin{itemize}
    \item $\forall u,u',v,v'.~u \leq u' \to v' \leq v \to u \sim v \to u' \sim v'$
    \item $\forall v.~\plr{\exists u.~u \leq 0 \land u \sim v} \to v \leq 0$
    \item $\forall u_0,u_1,v.~\plr{\exists u.~u \leq u_0 + u_1 \land u \sim v}
      \to {}$\\$\plr{\exists v_0,v_1.~u_0 \sim v_0
      \land u_1 \sim v_1 \land v \leq v_0 + v_1}$
    \item $\forall r,u',v.~\plr{\exists u.~u \leq ru' \land u \sim v} \to
      \plr{\exists v'.~u' \sim v' \land v \leq rv'}$
    \item
      $\forall u.~\exists v.~u \sim v \land \forall v'.~u \sim v' \to v' \leq v$
  \end{itemize}
\end{definition}

Intuitively, $Q \sim P$, where $P$ and $Q$ are row vectors, is equivalent to
$P \leq Q\Psi$, where $\Psi$ is the matrix representing the linear map and on
the right is a vector-matrix multiplication.
It is important that we think of \emph{row} vectors and
\emph{right}-multiplication by a matrix because, without commutativity of the
underlying posemiring, we can only expect $\plr{rQ}\Psi = r\plr{Q\Psi}$ and
not $\Psi\plr{rQ} = r\plr{\Psi Q}$.
In \cref{sec:env}, we use the matrix notation for convenience, while in the
Agda code we see \ExecuteMetaData[\Snippetstex]{Psi-rel-P-Q}.

%Operations like renaming and substitution are essentially translations from one
%context to another.
%When faced with two vector spaces arranged in this way, a natural thing to
%consider is the \emph{linear maps} from one space to the other.


\section{Generic syntax}\label{sec:syntax}
We take the insights of the previous section and use them to build a
generic framework for posemiring-annotated substructural systems in
Agda. We will first show \emph{descriptions} of systems, which are
comprised of rules that have premises combined using the bunched
combinators. We then show how to construct the Agda data type of
intrinsically well scoped, typed, and resourced terms for any given
system of our framework. We use the prototypical system from
\figref{fig:comb-lr} as a running example. \secref{sec:other-syntaxes}
presents further examples that our framework can express.

We now start to use Agda notation for record and data type
declarations, to emphasise that our framework has been implemented.

\subsection{Descriptions of Systems}

% We capture the form of rules exemplified previously\todo{Previously?} via
% \emph{descriptions} of rules.
% The key to making these descriptions work is that they only allow syntactic
% forms that preserve environments.
% These forms are: absence and multiplicity of subterms with the same usage
% annotations, absence and multiplicity of subterms with summed usage annotations,
% scaling of a subterm, and variable binding.\todo{Switching to Agda}

\paragraph{\AgdaDatatype{Premises}, \AgdaRecord{Rule}s, and \AgdaRecord{System}s.}

A type \AgdaRecord{System} is made up of multiple \AgdaRecord{Rule}s.
Each \AgdaRecord{Rule} comprises a tree of \AgdaDatatype{Premises} and
a type of conclusion. We assume that there is a
$\AgdaBound{Ty} : \AgdaPrimitiveType{Set}$ of types for the system in
scope.

The \AgdaDatatype{Premise} data type describes premises of rules,
using the bunched combinators from the previous section. A single
premise is introduced by the
\AgdaInductiveConstructor{$\langle$\_`$\vdash$\_$\rangle$}
constructor.  This allows binding of additional variables
\AgdaBound{$\Delta$} (with specified types and usage annotations) and
the specification of a conclusion type \AgdaBound{A} for this premise.
The remaining constructors are descriptions for the first-order
bunched connectives, and will be interpreted directly as such, below.

\ExecuteMetaData[\Syntaxtex]{Premises}

A \AgdaRecord{Rule} is a pair of some \AgdaDatatype{Premises} and a
conclusion. We use an infix arrow as a suggestive notation for rules.

\ExecuteMetaData[\Syntaxtex]{Rule}

Finally, a \AgdaRecord{System} consists of a set of rule labels (i.e.,
constructor names), and for each label a description of the
corresponding rule. We use $\rhd$ as infix notation for systems to
associate the label set with the rules.

\ExecuteMetaData[\Syntaxtex]{System}

\paragraph{\figref{fig:lr-comb} as a \AgdaRecord{System}.}

As an example, we transcribe the system defined in
\figref{fig:lr-comb} into a description.  We give the set of types of
this system as a data type \AgdaDatatype{Ty} (together with a base
type \AgdaInductiveConstructor{$\iota$}). We assume that there is a
posemiring \AgdaInductiveConstructor{Ann} in scope for the
annotations.There is one label for each instantiation of a logical
rule, but the labels contain no further information about subterms or
restrictions on the context. This will be provided when we associate
labels with \AgdaRecord{Rule}s in a \AgdaRecord{System}.

\noindent
\begin{minipage}[t]{0.5\textwidth}
  \ExecuteMetaData[\PaperExamplestex]{Ty}
  \ExecuteMetaData[\PaperExamplestex]{Side}
\end{minipage}
\begin{minipage}[t]{0.5\textwidth}
  \ExecuteMetaData[\PaperExamplestex]{qlR}
\end{minipage}

To build a system, we associate with each label a rule:

\ExecuteMetaData[\PaperExamplestex]{lR}

Compared to \figref{fig:lr-comb}, modulo the Agda notation, we can see
that the fundamental structure has been preserved: the rules match
one-to-one, and the bunched premises are the same. A major difference
is that we do not include a counterpart to the
\AgdaInductiveConstructor{var} rule in a
\AgdaRecord{System}. Variables are common to all the systems
representable in our framework.

\paragraph{Terms of a \AgdaRecord{System}.}

The next thing we want to do is to build terms in the described type system.
The following definitions are useful for talking about types indexed over
contexts, judgement forms, and judgement forms admitting newly bound variables,
respectively.

\ExecuteMetaData[\Syntaxtex]{OpenFam}

To specify the meaning of descriptions, we assume some \AgdaBound{X} : \AgdaFunction{ExtOpenFam},
% \ExecuteMetaData[\Interpretationtex]{X},
over which we form one layer of syntax, using the function
\AgdaFunction{$\llbracket$\_$\rrbracket$p} that interprets
\AgdaDatatype{Premises} defined below.  The first argument to
\AgdaBound{X} is the new variables bound by this layer of syntax, as
exemplified in the first clause of
\AgdaFunction{$\llbracket$\_$\rrbracket$p}.  The second argument is
the context containing the variables being carried over from the
previous layer.  Notice that this is not, in general, the same as the
context from the previous layer, because the usage annotations may
have been changed by connectives like
\AgdaInductiveConstructor{\_`$*$\_} and
\AgdaInductiveConstructor{\_`$\cdot$\_}.  The third argument is the
type of subterm required.

With the first clause of \AgdaFunction{$\llbracket$\_$\rrbracket$p} explained,
the rest are simply interpretations of premises into bunched combinators.
The superscript \AgdaFunction{$^c$} on the bunched connectives denotes that
they have been lifted from predicates on usage vectors to predicates on
contexts, with the type component of the context shared throughout.
Additive connectives \AgdaFunction{$\dot1$} and \AgdaFunction{$\dottimes$} are
already polymorphic (not relying on anything specific about usage vectors), so
do not need a \AgdaFunction{$^c$} variant.

\ExecuteMetaData[\Interpretationtex]{semp}

The interpretation of a \AgdaRecord{Rule} checks that the rule targets
the desired type and then interprets the rule's premises \AgdaBound{ps}.
Notice that the interpretation of the premises is independent of the conclusion
of the rule, which accounts for the difference in type between
\AgdaFunction{$\llbracket$\_$\rrbracket$p} and
\AgdaFunction{$\llbracket$\_$\rrbracket$r}.

\ExecuteMetaData[\Interpretationtex]{semr}

The interpretation of a \AgdaRecord{System} is to choose a rule label
\AgdaBound{l} from \AgdaBound{L} and interpret the corresponding rule
\AgdaBound{rs}\AgdaSpace{}\AgdaBound{l} in the same context and for the same
conclusion.

\ExecuteMetaData[\Interpretationtex]{sems}

The most obvious way to make such an \AgdaBound{X} is to use some existing
\AgdaFunction{OpenFam} on an extended context.
We defined \AgdaFunction{Scope} to do this: take the new variables
\AgdaBound{$\Delta$}, concatenate them onto the existing context
\AgdaBound{$\Gamma$}, and pass the extended context onto the judgement
\AgdaBound{T}.

\ExecuteMetaData[\Syntaxtex]{Scope}

%{\color{red}(Forward ref: for now, we could have inlined \texttt{Scope}.)}

We use \AgdaFunction{Scope} to deal with new variables in syntax.
Terms resemble the free monad over a layer-of-syntax functor, though
that picture is complicated by variable binding.  A term is either a
variable or a use of a logical rule together with terms for each of
the required subterms. The \AgdaFunction{Size} argument is where we
use sized types to record that subterms are smaller than the
surrounding term.

\ExecuteMetaData[\Termtex]{Term}

This definition uses \AgdaFunction{$\dotto$}, which, analogously to
\AgdaFunction{$\dottimes$}, is an index-preserving version of the function
space.
We take \AgdaFunction{$\dotto$} to handle $n$ many indices --- in this
case two (the context and the type).
Informally,
\AgdaFunction{$\forall[$}\AgdaSpace{}\AgdaBound{T}\AgdaSpace{}\AgdaFunction{]}
stands for
\AgdaSymbol{$\forall$}\AgdaSpace{}\AgdaSymbol{\{}%
\AgdaBound{x$_1$}\AgdaSpace{}\AgdaSymbol{$\ldots$}\AgdaSpace{}\AgdaBound{x$_n$}%
\AgdaSymbol{\}}\AgdaSpace{}\AgdaSymbol{$\to$}\AgdaSpace{}\AgdaBound{T}%
\AgdaSpace{}%
\AgdaBound{x$_1$}\AgdaSpace{}\AgdaSymbol{$\ldots$}\AgdaSpace{}\AgdaBound{x$_n$},
where \AgdaBound{T} is a type family with $n$ many indices.

Terms defined like this are still quite difficult to write, mainly because of
frequently changing usage contexts and the need for proofs that they all match
up.
We will see a way of automating these proofs in \cref{sec:usage-elaborator}.

%Here is an example term, using the \AgdaFunction{$\lambda$R} system.
%First, for ease of writing, we introduce pattern synonyms for each of the
%typing rules we use.

%\ExecuteMetaData[\PaperExamplestex]{patterns}

%Our example term is a function that flips a tagged union wrapped in an
%arbitrarily annotated \emph{bang}.
%Much of the effort in writing such a term goes into writing the various
%relatedness proofs between usage contexts --- observing, for example, that two
%usage contexts sum together to make a third, or that a usage context used for
%a variable is a basis vector.
%We give a method of automating these proofs in \cref{sec:usage-elaborator}.
%\todo{To be clear, we don't actually write this.}

%\ExecuteMetaData[\HeavyItex]{lR-term}

% A layer of syntax supports the following functorial action.

% \ExecuteMetaData[\Maptex]{map-s-type}

\subsection{Other syntaxes and syntactic forms}\label{sec:other-syntaxes}

\paragraph{The system $\mu\tilde\mu$.}
We can encode a usage-annotated version of System $L$/the
$\mu\tilde\mu$-calculus~\cite{CH00} --- a syntax for classical logic --- in
such a way that contexts capture the undistinguished parts of the sequent.
As such, the generic substitution lemma we get in \cref{sec:kits} is the form
of substitution required in standard $\mu\tilde\mu$-calculus metatheory.
Though the $\mu\tilde\mu$-calculus is originally described as a sequent
calculus~\cite{CH00}, we use the techniques of
\citet[p.~12]{herbelin-hab} and \citet{LC06} to present it as a natural
deduction system, thus giving a notion of \emph{variable} to the system.

Unlike the single judgement form of \name{} and standard simply typed
$\lambda$-calculi, the $\mu\tilde\mu$-calculus has three judgement forms:
terms, coterms, and commands.
Read logically, terms and coterms are seen to, respectively, prove and refute
propositions (types), while commands exhibit contradictions.
This means that the abstract \AgdaBound{Ty} in the generic framework is
instantiated to \AgdaDatatype{Conc} (for \emph{conclusion}) as below, with
\AgdaDatatype{Ty} not being exposed directly to the generic framework.
For now, we just consider multiplicative disjunction $\parr$ (\emph{par}) and
negation/duality, beside an uninterpreted base type.
These are enough to exhibit classical behaviour.

\noindent
\begin{minipage}[t]{0.5\textwidth}
  \ExecuteMetaData[\MuMuTildetex]{Ty}
\end{minipage}
\begin{minipage}[t]{0.5\textwidth}
  \ExecuteMetaData[\MuMuTildetex]{Conc}
\end{minipage}

With \AgdaBound{Ty} instantiated as \AgdaDatatype{Conc}, all terms are assigned
\AgdaDatatype{Conc} type, as are all the variables.
No variables are given \AgdaInductiveConstructor{com} type, similar to how in
the bidirectional typing syntax of \citet[p.~25]{AACMM21}, no variables are
given \AgdaInductiveConstructor{Check} type.
How to observe this invariant is covered in the latter paper, so we will not
repeat it here (having not yet seen how to write traverals on terms).

The syntax comprises a \emph{cut} between a term and a coterm of the same type,
the eponymous $\mu$ and $\tilde\mu$ constructs for proof by contradiction, and
then term and coterm (introduction and elimination) forms for negation and
\emph{par}.

\ExecuteMetaData[\MuMuTildetex]{MMT}

%With a collection of pattern synonyms and the machinery from
%\cref{sec:usage-elaborator}, we can write an example term: a function which
%flips the disjuncts of a \emph{par}.

%\ExecuteMetaData[\MuMuTildeTermtex]{patterns}
%\ExecuteMetaData[\MuMuTildeTermtex]{myComm}

\paragraph{Duplicability}
There is one more bunched combinator we have experimented with adding to the
framework:

\[
  \plr{\Box T}\,\grR \coloneqq \Sigma\grRprime.~\plr{\grRprime \leq \grR}
  \times \plr{\grRprime \leq \gr0}
  \times \plr{\grRprime \leq \grRprime + \grRprime}
  \times T\,\grRprime
\]

The idea of $\plr{\Box T}\,\grR$ is to assert that $\grR$, or some refinement
of it, can be both discarded and duplicated indefinitely, and in the
refinement we have a $T$.
We use this combinator to introduce subterms that are used an unknown number of
times, for example the continuations of the eliminator of an inductive type,
or other fixed points.
We can also use it in linear/non-linear style systems~\cite{Benton94} to make
sure linear variables are not available in the intuitionistic fragment.

Adding the $\Box$ combinator is the only thing we have found that requires our
linear maps be functional rather than merely relational.


\section{Environments}\label{sec:env}
This section is a development of the work of \citet{WA20}.

Let us write $\sdtstile{}\V$ as the infix version of $\V$.
That is, $\Gamma \sdtstile{}\V A \coloneqq \V\,\Gamma\,A$.
The idea of a $\V$-environment is to extend the notion of $\V$-values from
types to contexts.
Specifically, where the judgement $\Gamma \sdtstile{}\V A$ says that we
have a $\V$-value of type $A$ in context $\Gamma$, the judgement
$\Gamma \env\V \Delta$ says that we have a $\V$-value for each entry in
$\Delta$ (at the specified multiplicity), all in context $\Gamma$.
For our purposes, it is essential that the ``all'' in the previous sentence is
multiplicative --- we want to split the usage annotations of $\Gamma$ up into
parts such that each part supports one $\V$-value.

We have seen two kinds of values already: $\sqni$-values are variables and
$\vdash$-values are terms.
The corresponding environments are also standard concepts: $\sqni$-environments
are \emph{simultaneous renamings} and $\vdash$-environments are
\emph{simultaneous substitutions}.

\subsection{Definition}

\begin{definition}\label{def:lr-env}
  A \emph{$\V$-environment} between annotated contexts $\Gamma$ and $\Delta$
  (written $\grP\gamma$ and $\grQ\delta$, respectively, when convenient)
  is a linear map $\gr\Psi : \Ann^{\size\Delta} \to \Ann^{\size\Gamma}$ (written
  postfix) such that $\grP \leq \grQ\gr\Psi$ and for each $A$, $\grPprime$, and
  $\grQprime$ such that $\grPprime \leq \grQprime\gr\Psi$, a function from
  $\grQprime\delta \sqni A$ to $\grPprime\gamma \sdtstile{}\V A$.
\end{definition}

\begin{example}
  We can form the identity renaming on a two-variable context.
  \[
    \id : \plr{\gr rA, \gr sB} \env\sqni \plr{\gr rA, \gr sB}
  \]
  Linear map $\gr\Psi$ is the identity map, clearly satisfying
  \(
    \begin{pmatrix} \gr r & \gr s \end{pmatrix} \leq
    \begin{pmatrix} \gr r & \gr s \end{pmatrix}\gr\Psi
  \).
  When considering values, the fact that $\grPprime \leq \grQprime\gr\Psi$
  reduces to $\grPprime \leq \grQprime$.
  The two cases to consider are when $\grQprime\delta \sqni A$ and when
  $\grQprime\delta \sqni B$.
  In the first case, $\grPprime \leq \grQprime \leq
  \begin{pmatrix} \gr1 & \gr0 \end{pmatrix}$, so we have
  $\grPprime\plr{A, B} \sqni A$.
  The second case follows symmetrically.
\end{example}

\begin{example}
  Assume $\Ann$ is the natural numbers with ordering given by $=$ and the usual
  addition and multiplication.
  There is an environment (substitution) of type
  \[
    \plr{\gr0x : A, \gr2y : B \multimap C, \gr3z : B} \env\vdash
    \plr{\gr1B, \gr2C}.
  \]
  Linear map $\gr\Psi$ is given by the matrix
  \(
    \begin{pmatrix}
      \gr0 & \gr0 & \gr1 \\
      \gr0 & \gr1 & \gr1
    \end{pmatrix}
  \),
  noticing that $\gr1$ times the first row plus $\gr2$ times the second gives
  the original $\grP$.
  For $\grQprime = \begin{pmatrix} \gr1 & \gr0 \end{pmatrix}$, we have
  $\grPprime = \begin{pmatrix} \gr0 & \gr0 & \gr1 \end{pmatrix}$, and thus
  $\grPprime\gamma \vdash z : B$.
  For $\grQprime = \begin{pmatrix} \gr0 & \gr1 \end{pmatrix}$, we have
  $\grPprime = \begin{pmatrix} \gr0 & \gr1 & \gr1 \end{pmatrix}$, and thus
  $\grPprime\gamma \vdash y\,z : C$.
\end{example}

As a mnemonic, one may use notation like the following to see what values are
needed in the environment.
\[
  \begin{pmatrix}
    \gr0A & \gr0\plr{B \multimap C} & \gr1B \\
    \gr0A & \gr1\plr{B \multimap C} & \gr1B
  \end{pmatrix}
  \begin{matrix}
    {} \vdash B \\
    {} \vdash C
  \end{matrix}
\]
This notation assumes that the notion of value supports subusaging, which is
always the case when we are using environments for traversals.

\begin{example}
  Assume $\Ann$ is the natural numbers with ordering given by $=$ and the usual
  addition and multiplication.
  There is an environment (renaming) of type
  \[
    \plr{\gr6a : A, \gr0b : B, \gr1c : C, \gr0d : D} \env\sqni
    \plr{\gr1C, \gr2A, \gr4A}.
  \]
  Linear map $\gr\Psi$ is given by the matrix
  \(
    \begin{pmatrix}
      \gr0 & \gr0 & \gr1 & \gr0 \\
      \gr1 & \gr0 & \gr0 & \gr0 \\
      \gr1 & \gr0 & \gr0 & \gr0
    \end{pmatrix}
  \),
  which we can check satisfies the required inequality.
  The values are given by
  \begin{align*}
    \gr0a : A, \gr0b : B, \gr1c : C, \gr0 : D &\sqni c : C \\
    \gr1a : A, \gr0b : B, \gr0c : C, \gr0 : D &\sqni a : A \\
    \gr1a : A, \gr0b : B, \gr0c : C, \gr0 : D &\sqni a : A.
  \end{align*}

  We can read from the columns of the matrix what happened to each of the
  variables in $\Gamma$.
  The first column, corresponding to variable $\gr6a : A$, contains two $\gr1$s
  because it has been duplicated (via contraction).
  Meanwhile, the second and fourth columns are all $\gr0$ because variables
  $b$ and $d$ have been discarded (via weakening).
  The third column contains one $\gr1$ because $c$ is used once.
  This $\gr1$ appears above the $\gr1$s to its left because $c$ has been
  permuted (via exchange) past $a$.
  Each of the rows in the matrix is a basis vector because variables can only
  be formed in contexts with basis annotations or less.
\end{example}

\subsection{Properties}

\begin{lemma}
  \cref{def:lr-env} is sound and complete for the following syntax.
  \begin{align*}
    \begin{prooftree}[comb,center=false]
      \hypo{I^*}
      \infer1{\alr{} : {} \env\V {\cdot}}
    \end{prooftree}
    &&
    \begin{prooftree}[comb,center=false]
      \hypo{\rho : {} \env\V \Delta_l}
      \hypo{\sep}
      \hypo{\sigma : {} \env\V \Delta_r}
      \infer3{\alr{\rho,\sigma} : {} \env\V \Delta_l, \Delta_r}
    \end{prooftree}
    &&
    \begin{prooftree}[comb,center=false]
      \hypo{\gr r\cdot\plr{M : {} \sdtstile{}\V A}}
      \infer1{\alr{M} : {} \env\V \gr rA}
    \end{prooftree}
  \end{align*}
\end{lemma}
\begin{proof}
  This replaces \cref{thm:construct-env}.
\end{proof}

\begin{example}
  Assume $\Ann$ is the natural numbers with ordering given by $=$ and the usual
  addition and multiplication.
  There is an environment (substitution) of type
  \[
    \alr{\alr{z},\alr{y\,z}} :
    \plr{\gr0x : A, \gr2y : B \multimap C, \gr3z : B} \env\vdash
    \plr{\gr1B, \gr2C}.
  \]
  We rely on the observations that
  $\begin{pmatrix} \gr0 & \gr2 & \gr3 \end{pmatrix} =
  \begin{pmatrix} \gr0 & \gr0 & \gr1 \end{pmatrix}
  + \begin{pmatrix} \gr0 & \gr2 & \gr2 \end{pmatrix}$ and, on the right, that
  $\begin{pmatrix} \gr0 & \gr2 & \gr2 \end{pmatrix} =
  \gr2\begin{pmatrix} \gr0 & \gr1 & \gr1 \end{pmatrix}$.
  Then, we have $\gr0x : A, \gr0y : B \multimap C, \gr1z : B \vdash z : B$ and
  $\gr0x : A, \gr1y : B \multimap C, \gr1z : B \vdash y\,z : C$.
\end{example}

When constructing an environment, we can do so by cases on the shape of the
target context.
We can create an environment into the empty context when all usage annotations
on the source context are $\gr0$.
We can create an environment into a concatenated context when we can additively
split up the annotations of the source context and produce environments into
both halves from the split sources.
We can create an environment into a singleton context when there is a context
$\gr r$ times smaller than the source context in which we can produce a value
of the appropriate type.

\begin{lemma}\label{thm:construct-env}
  We can define all of the following equivalences for any values of the free
  variables, assuming that $\V$ respects subusaging (i.e.,
  $\grPprime \leq \grP \to
  \forallb{\V\,\grP\gamma \dotto \V\,\grPprime\gamma}$).
  \begin{itemize}
    \item $\forallb{I \dotlr \plr{{-} \env\V {\cdot}}}$
    \item $\forallb{\plr{{-} \env\V \Delta_l} \sep \plr{{-} \env\V \Delta_r}
      \dotlr \plr{{-} \env\V \Delta_l, \Delta_r}}$
    \item
      $\forallb{\gr r \cdot \plr{\V\,(-)\,A} \dotlr \plr{{-} \env\V \gr rA}}$
  \end{itemize}
\end{lemma}
\begin{proof}
  There are 6 cases to check.
  Throughout, we write $\Gamma$ as $\grP\gamma$ and $\Delta$ as $\grQ\delta$
  when convenient.
  \begin{description}
    \item[$I(\to)$]
      Let $\gr\Psi$ be the unique linear map out of the zero space.
      By assumption and definition, $\grP \leq \gr0 = \grQ\gr\Psi$.
      There are no variables to act upon.
    \item[$I(\gets)$]
      $\grQ\gr\Psi$ is an empty sum, so if $\grP \leq \grQ\gr\Psi$ then
      $\grP \leq \gr0$.
    \item[$\sep(\to)$]
      Let the given environments be $\rho_l : \grPl\gamma \env\V \grQl\delta$
      and $\rho_r : \grPr\gamma \env\V \grQr\delta$, with
      $\grP \leq \grPl + \grPr$.
      Define $\gr\Psi \coloneqq [\rho_l.\gr\Psi, \rho_r.\gr\Psi]$, using the
      coproduct structure of the concatenated vector space.
      We have $\grP \leq \grPl + \grPr \leq
      \grQl\plr{\rho_l.\gr\Psi} + \grQr\plr{\rho_r.\gr\Psi} =
      \begin{pmatrix} \grQl & \grQr \end{pmatrix}\gr\Psi$.
      To act on variables, we are given $\grPprime \leq
      \begin{pmatrix} \gr{\grQ'_l} & \gr{\grQ'_r} \end{pmatrix}\gr\Psi$ and
      $\gr{\grQ'_l}\delta_l, \gr{\grQ'_r}\delta_r \sqni A$.
      Without loss of generality, let us have $\gr{\grQ'_l}\delta_l \sqni A$
      and $\gr{\grQ'_r} \leq \gr0$.
      Thus, $\grPprime \leq
      \gr{\grQ'_l}\plr{\rho_l.\gr\Psi} + \gr{\grQ'_r}\plr{\rho_r.\gr\Psi} \leq
      \gr{\grQ'_l}\plr{\rho_l.\gr\Psi}$,
      and we can act on the variable using $\rho_l$.
    \item[$\sep(\gets)$]
      Let the unnamed context be $\Gamma$, also written $\grP\gamma$.
      The linear map
      $\gr\Psi : \Ann^{\size{\Delta_l} + \size{\Delta_r}} \to \Ann^{\size\Gamma}$
      splits into
      $\gr\Psi_{\gr l} : \Ann^{\size{\Delta_l}} \to \Ann^{\size\Gamma}
      \coloneqq \alr{\id, 0}; \gr\Psi$ and
      $\gr\Psi_{\gr r} : \Ann^{\size{\Delta_r}} \to \Ann^{\size\Gamma}
      \coloneqq \alr{0, \id}; \gr\Psi$, using the product structure of
      the concatenated vector space.
      Let $\grPl \coloneqq \grQl\gr\Psi_{\gr l}$ and
      $\grPr \coloneqq \grQr\gr\Psi_{\gr r}$, by definition satisfying the
      required constraints.
      For the action on variables, let us consider the left environment (with
      the right environment following symmetrically).
      We are given $\gr{\grP'_l} \leq \gr{\grQ'_l}\gr\Psi_{\gr l}$ and
      $\gr{\grQ'_l}\delta_l \sqni A$.
      From these, we get
      $\gr{\grP'_l} \leq \gr{\grQ'_l}\gr\Psi_{\gr l} =
      \begin{pmatrix} \gr{\grQ'_l} & \gr0 \end{pmatrix}\gr\Psi$ and
      $\gr{\grQ'_l}\delta_l, \gr0\delta_r \sqni A$.
      We can therefore act using the original environment.
    \item[$\cdot(\to)$]
      Let $\grP$ and $\grPprime$ be such that $\grP \leq \gr r\grPprime$ and let
      $v : \V\,\grPprime\gamma\,A$.
      Let $\gr\Psi : \Ann \to \Ann^{\size\gamma}
      \coloneqq \gr r\gr' \mapsto \gr r\gr'\grPprime$.
      By definition and the previous assumption, we have
      $\grP \leq \gr r\gr\Psi$.
      When acting on a variable, we have $\grP\gr{''} \leq \gr r\gr'\gr\Psi$
      and $\gr r\gr'A \sqni A'$.
      The latter tells us that $A = A'$ and $\gr r\gr' \leq \gr1$.
      Thus, $\grP\gr{''} \leq \grPprime$.
      Therefore, by subusaging, we may produce a value of type
      $\V\,\grPprime\gamma\,A$, which we can take to be $v$.
    \item[$\cdot(\gets)$]
      Let us have an environment of type $\grP\gamma \env\V \gr rA$.
      We want to use its action on variables to yield a value.
      To do this, we let $\grPprime \coloneqq \gr1\gr\Psi$, and use this
      equation, together with the fact that we have a variable of type
      $\gr1A \sqni A$, to get a value of type $\V\,\grPprime\gamma\,A$.
      Furthermore, we derive $\grP \leq \gr r\gr\Psi = \gr r\grPprime$, as
      required.
  \end{description}
\end{proof}

We could, indeed, use these three clauses to define what an environment is.
However, I find them difficult to work with, as it is often easier to do
linear algebraic proofs separately from the rest of an environment.
For identity and composition, as we are about to see, the original definition
is easier to use because we can rely on the identity and composition of linear
maps.
Concretely, an inductive proof of identity would, for example, involve
constructing an environment of type
$\grP\gamma, \grQ\delta \env\V \grP\gamma, \grQ\delta$ by constructing
environments of types $\grP\gamma, \gr0\delta \env\V \grP\gamma$ and
$\gr0\gamma, \grQ\delta \env\V \grQ\delta$.
These are not identity environments, so we would have to strengthen the
induction hypothesis.

\begin{lemma}\label{thm:env-resize}
  Given an environment $\rho : \grP\gamma \env\V \grQ\delta$ and a $\grPprime$
  and a $\grQprime$ such that $\grPprime \leq \grQprime\plr{\rho.\gr\Psi}$,
  there is also an environment of type $\grPprime\gamma \env\V \grQprime\delta$
  with the same linear map and action on variables.
\end{lemma}
\begin{proof}
  The only part of the definition of an environment dependent on $\grP$ or
  $\grQ$ is the constraint $\grP \leq \grQ\gr\Psi$, which we are able to
  replace for $\grPprime$ and $\grQprime$.
\end{proof}

One of the primary test cases for environments is simultaneous substitution,
which will look like the following rule.
The admissibility of substitution will be by induction on the derivation of
$\Delta \vdash A$, so we will need to be able to adapt any environment we are
given to work with any possible context of new premises.
In the simply typed case, the only change to the context we encountered was the
binding of new variables.
Now, with usage annotations, we furthermore have linear decompositions of the
context, necessitating changes to the environment whenever usage annotations
change.
I will deal first with linear decompositions.

\begin{displaymath}
  \begin{prooftree}
    \hypo{\Gamma \env{\vdash} \Delta}
    \hypo{\Delta \vdash A}
    \infer2[sub]{\Gamma \vdash A}
  \end{prooftree}
\end{displaymath}

There are three kinds of linear decompositions we have to deal with: zero,
addition, and scaling; corresponding to bunched connectives $I^*$, $\sep$, and
$\gr r \cdot {}$, respectively.
In each case, we have a simple preservation lemma, transforming an environment
of type $\Gamma \env\V \Delta$ and a decomposition of $\Delta$ into a
decomposition of $\Gamma$ and environments for all of the decomposed fragments
of $\Gamma$ and $\Delta$.

\begin{lemma}[environments preserve zero]\label{thm:lr-env-zero}
  Given an environment of type $\grP\gamma \env\V \grQ\delta$ such that
  $\grQ \leq \gr 0$, we also have that $\grP \leq \gr 0$.
\end{lemma}
\begin{proof}
  $\grP \leq \grQ\gr\Psi \leq \gr0\gr\Psi = \gr0$, by environment
  compatibility and monotonicity and linearity of $\gr\Psi$.
\end{proof}

\begin{lemma}[environments preserve addition]\label{thm:lr-env-add}
  Given an environment of type $\grP\gamma \env\V \grQ\delta$ such that
  $\grQ \leq \grQl + \grQr$ for some $\grQl$ and $\grQr$, we also have $\grPl$
  and $\grPr$ such that $\grP \leq \grPl + \grPr$ and there are environments
  of types $\grPl\gamma \env\V \grQl\delta$ and
  $\grPr\gamma \env\V \grQr\delta$.
\end{lemma}
\begin{proof}
  Let $\grPl \coloneqq \grQl\gr\Psi$ and $\grPr \coloneqq \grQr\gr\Psi$.
  Then, $\grP \leq \grQ\gr\Psi \leq \plr{\grQl + \grQr}\gr\Psi =
  \grQl\gr\Psi + \grQr\gr\Psi = \grPl + \grPr$, satisfying the first condition.
  Because clearly $\grPl \leq \grQl\gr\Psi$ and $\grPr \leq \grQr\gr\Psi$,
  \cref{thm:env-resize} on the original environment gives us the required
  pair of new environments.
\end{proof}

\begin{lemma}[environments preserve scaling]\label{thm:lr-env-scale}
  Given an environment of type $\grP\gamma \env\V \grQ\delta$ such that
  $\grQ \leq \gr r\grQprime$ for some $\grQprime$, we also have a $\grPprime$
  such that $\grP \leq \gr r\grPprime$ and there is an environment of type
  $\grPprime\gamma \env\V \grQprime\delta$.
\end{lemma}
\begin{proof}
  Let $\grPprime \coloneqq \grQprime\gr\Psi$.
  Then, $\grP \leq \grQ\gr\Psi \leq \plr{\gr r\grQprime}\gr\Psi =
  \gr r\plr{\grQprime\gr\Psi} = \gr r\grPprime$, satisfying the first condition.
  Because clearly $\grPprime \leq \grQprime\gr\Psi$,
  \cref{thm:env-resize} on the original environment gives us the required
  new environment.
\end{proof}

Finally, I will also take the opportunity to give the extend lemma, allowing
environments to incorporate newly bound variables.
\todo{Motivate}
In the intuitionistic case, the extend lemma had two requirements on $\V$: $\V$
admits weakening and we can map variables into $\V$-values.
With usage annotations, the former is unreasonable, but it turns out that we
only need weakening by variables whose usage annotation is less than or equal
to $\gr0$.
The latter stays as-is, with the note that ``variable'' now means a
usage-checked variable.

\begin{lemma}[extend]\label{thm:lr-bind}
  Given functions
  ${\swarrow^k} : \forall \Gamma, \grR, \theta.~\grR \leq \gr0 \to
  \forallb{\V\,\Gamma \dotto \V\,\plr{\Gamma, \grR\theta}}$ and
  $\mathrm{vr} : \forallb{{\sqni} \dotto \V}$, we can turn an environment of
  type $\Gamma \env\V \Delta$ into an environment of type
  $\Gamma, \Theta \env\V \Delta, \Theta$ for any context $\Theta$.
\end{lemma}
\begin{proof}
  Let $\grP\gamma \coloneqq \Gamma$, $\grQ\delta \coloneqq \Delta$, and
  $\grR\theta \coloneqq \Theta$.
  Let the new linear map $\gr\Psi\gr' : \Ann^{\size\Delta + \size\Theta} \to
  \Ann^{\size\Gamma + \size\Theta}$ be $\gr\Psi \oplus \gr I$.
  That is, in block matrix notation,
  $\begin{pmatrix} \gr\Psi & \gr0 \\ \gr0 & \gr I \end{pmatrix}$.
  Checking that this linear map fits, we have
  $\begin{pmatrix}\grP & \grR\end{pmatrix}
  \leq \begin{pmatrix}\grQ\gr\Psi & \grR\gr I\end{pmatrix}
  = \begin{pmatrix}\grQ & \grR\end{pmatrix}\plr{\gr\Psi \oplus \gr I}$.
  For the action on variables, we are given vectors $\grPprime$,
  $\grR\gr'_\grP$, $\grQprime$, and $\grR\gr'_\grQ$ such that
  $\begin{pmatrix} \grPprime & \grR\gr'_\grP \end{pmatrix} \leq
  \begin{pmatrix} \grQprime & \grR\gr'_\grQ \end{pmatrix}
  \plr{\gr\Psi \oplus \gr I}$ and we have a variable of type
  $\grQprime\delta, \grR\gr'_\grQ\theta \sqni A$ for some type $A$.
  The constraint on the new vectors reduces to $\grPprime \leq \grQprime\gr\Psi$
  and $\grR\gr'_\grP \leq \grR\gr'_\grQ$.
  From the variable we either have a variable $x$ in $\delta$ with
  $\grQprime \leq \langle x \rvert$ and $\grR\gr'_\grQ \leq \gr0$, or a
  variable $y$ in $\theta$ with $\grQprime \leq \gr0$ and
  $\grR\gr'_\grQ \leq \langle y \rvert$.
  In the former case, the action of the original environment on $x$ gives us a
  $\V$-value in $\grPprime\gamma$, and the $\gr0$-weakening principle
  $\swarrow^k$, noting that $\grR\gr'_\grP \leq \grR\gr'_\grQ \leq \gr0$, gives
  us a $\V$-value in $\grPprime\gamma, \grR\gr'_\grP\theta$.
  In the latter case, we have that
  $\begin{pmatrix} \grPprime & \grR\gr'_\grP \end{pmatrix}
  \leq \begin{pmatrix} \grQprime\gr\Psi & \grR\gr'_\grQ \end{pmatrix}
  \leq \begin{pmatrix} \gr0\gr\Psi & \langle y \rvert \end{pmatrix}
  = \begin{pmatrix} \gr0 & \langle y \rvert \end{pmatrix}
  = \left\langle {\searrow}y \right\rvert$, so $y$ also serves as a
  usage-checked variable in $\grPprime\gamma, \grR\gr'_\grP\theta$.
  From this usage-checked variable, we get a $\V$-value in the same context
  using $\mathrm{vr}$.
\end{proof}

\subsection{Identity and composition}

The requirements for identity and composition of environments look a bit like
the unit and lift of a Kleisli triple.
\todo{Change from equality to inequality}

\begin{lemma}[Identity environment]\label{thm:env-id}
  Given a function $\mathrm{vr} : \forallb{{\sqni} \dotto \V}$, for any
  $\Gamma$ we have an environment of type $\Gamma \env\V \Gamma$.
\end{lemma}
\begin{proof}
  Let $\gr\Psi$ be the identity map, which clearly satisfies
  $\grP = \grP\gr\Psi$.
  When acting on a variable, the equation $\grPprime = \grQprime\gr\Psi$ means
  that $\grPprime = \grQprime$, so we want, from a variable of type
  $\grPprime\gamma \sqni A$, a value of type $\V\,\grPprime\gamma\,A$, which
  we can get from $\mathrm{vr}$.
\end{proof}

\begin{lemma}\label{thm:env-comp-lemma}
  Given an environment $\rho : \Gamma \env\U \Delta$ for which we have, for any
  $\grPprime$ and $\grQprime$ such that
  $\grPprime = \grQprime\plr{\rho.\gr\Psi}$, we have a function
  $\mathrm{lift}_\rho :
  \forallb{\V\,\grQprime\delta \dotto \W\,\grPprime\gamma}$,
  we can map environments of type $\Delta \env\V \Theta$ into environments of
  type $\Gamma \env\W \Theta$.
\end{lemma}
\begin{proof}
  Let $\rho$ be as in the statement, and let $\sigma : \Delta \env\V \Theta$.
  For the environment we are constructing, let
  $\gr\Psi \coloneqq \sigma.\gr\Psi; \rho.\gr\Psi$, noting that
  $\grP = \grQ\plr{\rho.\gr\Psi} =
  \plr{\grR\plr{\sigma.\gr\Psi}}\plr{\rho.\gr\Psi}$.
  For the action on variables, we are given $\grPprime = \grRprime\gr\Psi$ with
  $\grRprime\theta \sqni A$.
  We can immediately apply the action of $\sigma$, giving us a value of type
  $\V\,\plr{\grRprime\plr{\sigma.\gr\Psi}}\,A$.
  We note that
  $\grPprime = \plr{\grRprime\plr{\sigma.\gr\Psi}}\plr{\rho.\gr\Psi}$, and
  apply $\mathrm{lift}_\rho$ to get the desired value.
\end{proof}

\begin{corollary}[Composition of environments]\label{thm:env-comp}
  Given a function
  $\mathrm{lift} : \plr{\rho : \grP\gamma \env\U \grQ\delta} \to
  \forall \grPprime, \grQprime.~\grPprime = \grQprime\plr{\rho.\gr\Psi} \to
  \forallb{\V\,\grQprime\delta \dotto \W\,\grPprime\gamma}$, then we can
  compose environments of types $\Gamma \env\U \Delta$ and
  $\Delta \env\V \Theta$ into an environment of type $\Gamma \env\W \Theta$.
\end{corollary}

\todo{What do these choices mean?}
\begin{example}
  We can derive the following instances of environment composition.
  \begin{itemize}
    \item If $\U = \V = \W = {\sqni}$, then $\mathrm{lift}$ is given by the
      action of the renaming $\rho$ on variables.
      This allows us to derive composition of renamings.
    \item More generally, if $\V = {\sqni}$ and $\U = \W$, we can still use
      the action of the environment $\rho$.
      This means that renamings post-compose with any other sort of environment.
    \item If $\V = \W = {\vdash}$, then $\mathrm{lift}$ is given by a
      syntactic traversal.
      For example, if $\U = {\sqni}$, we need the action of renaming on terms
      to show that a renaming followed by a substitution composes to a
      substitution.
      If $\U = {\vdash}$, then the action of substitution on terms gives us that
      substitutions compose.
    \item More generally, if $\V = {\vdash}$ and we have a semantics from
      $\U$ to $\W$, then $\mathrm{lift}$ can be given by the semantic traversal
      of terms.
  \end{itemize}
\end{example}


\section{Semantics}\label{sec:semantics}
Having fixed a universe of syntaxes, in which we can build terms, the next thing
to do is to write recursive functions on terms.
With terms being given by an inductive \AgdaSymbol{data} type definition, they
already come with a recursion and an induction principle.
However, these principles do not handle variable-binding, which we have seen
with the fact that we had to write the \AgdaFunction{bind} helper
function for renaming and substitution in \cref{sec:lrsub}.

In this chapter, the central construct is a function \AgdaFunction{semantics}
which, for a $\V$-environment $\rho : \Gamma \env\V \Delta$,
maps a term $M : \Delta \vdash A$ to some semantic value in the type
$\C\,\Gamma\,A$.
This is a direct adaptation of the \AgdaFunction{semantics} function of
\cref{sec:gen-sem}, which has the same kind of action on intuitionistic terms,
given similar operations on $\V$ and $\C$ as what we had earlier.
The \AgdaFunction{semantics} function recurses on the term $M$, updating $\rho$
whenever new variables are bound.
In our usage-aware case, $\rho$ is also updated whenever we come across linear
combinations induced by premise combinators $I^{\sep}$, $\sep$, and
$\gr r \cdot {}$.

This chapter is structured as follows.
I start by giving a quick introduction to linear relations --- a generalisation
of linear maps --- with reference to their use in mechanised algebraic
reasoning, in \cref{sec:lin-rel}.
Using linear relations, I give a functorial \emph{map} operation to a single
layer of syntax in \cref{sec:functorial}.
I then adapt the \AgdaFunction{Kripke} function space to the usage-aware
setting in \cref{sec:kripke}.
Then I apply the \AgdaFunction{Kripke} function space, along with much of the
machinery I have introduced in previous chapters, to give the
\AgdaFunction{semantics} function in \cref{sec:traversal}.
Finally, I give the syntax-generic simultaneous renaming and substitution
operations in \cref{sec:kit-to-sem}.

% Our goal in this section is to define \AgdaFunction{semantics}, a
% recursor that turns a term into a \AgdaBound{$\C$}-value using a
% \AgdaBound{$\V$}-environment, in a type preserving way:\bob{Get rid of
%   ``body'' here}

% \ExecuteMetaData[\Semanticstex]{semantics-type}

% The \AgdaBound{$\V$} and \AgdaBound{$\C$} are \AgdaFunction{OpenFam}s,
% representing the interpretations of variables and terms
% respectively. In \cref{sec:traversal} we will see the data that must
% be provided to make a \AgdaFunction{semantics} for a given
% system. Before that, we must see how to deal with the two complicated
% features of our syntax: the usage annotations (\cref{sec:functorial})
% and variable binding (\cref{sec:kripke}). \todo{fwd ref to where these are used}

\section{Linear relations in Agda}\label{sec:lin-rel}

In \cref{sec:lrkits}, I defined \emph{usage-annotated environments}
(\cref{def:lr-env}).
One component of a usage-annotated environment is a linear map $\gr\Psi$ which,
when applied to the target usage vector, gives a vector compatible with the
source usage vector.

When it comes to mechanisation, I prefer to replace an assertion
``$\grP \leq \grQ\gr\Psi$'', involving a linear map $\gr\Psi$, by an assertion
``$\grP\gr\Psi\grQ$'', where $\gr\Psi$ is now a linear \emph{relation} said to
relate $\grP$ and $\grQ$.
I define linear relations as follows, where the reader may wish to check that
a linear map gives rise to a linear relation via the expression
$\grP \leq \grQ\gr\Psi$.

\begin{definition}\label{def:linear-relation}
  Given a posemiring $\Ann$ and modules $\mathscr M$ and $\mathscr N$ over
  $\Ann$, a \emph{linear relation} between $\mathscr M$ and $\mathscr N$ is
  a relation $\gr\Psi$ between the underlying sets of $\mathscr M$ and
  $\mathscr N$ such that the following properties hold of all
  $\grP, \grPprime, \grPl, \grPr \in \mathscr M$ and all
  $\grQ, \grQprime, \grQl, \grQr \in \mathscr N$.
  \begin{align*}
    \grPprime \leq \grP \land \grP\gr\Psi\grQ \land \grQ \leq \grQprime
    &\implies \grPprime\gr\Psi\grQprime
    \\
    \plr{\exists\grQ.~\grP\gr\Psi\grQ \land \grQ \leq \gr0}
    &\implies \grP \leq \gr0
    \\
    \plr{\exists\grQ.~\grP\gr\Psi\grQ \land \grQ \leq \grQl + \grQr}
    &\implies \plr{\exists\grPl,\grPr.~\grP \leq \grPl + \grPr
      \land \grPl\gr\Psi\grQl \land \grPr\gr\Psi\grQr}
    \\
    \plr{\exists\grQ.~\grP\gr\Psi\grQ \land \grQ \leq \gr r\grQprime}
    &\implies \plr{\exists\grPprime.~\grP \leq \gr r\grPprime
      \land \grPprime\gr\Psi\grQprime}
  \end{align*}
  I write $\mathscr M \rel \mathscr N$ as the type of linear relations between
  $\mathscr M$ and $\mathscr N$.
\end{definition}

Relations have several advantages over functions when doing mechanised algebra
in type theory.
For one, what are compound expressions in functional style --- for example
$x \leq f(y) + g(z)$ --- become collections of simple relationships in
relational style --- for example $\exists v,w.~vfy \land wgz \land
\mathrm{Add}\,x\,v\,w$.
The advantage of this is that we have immediate access to all of the expressions
and subexpressions, and the proofs of the relationships between them.
This means that there is no need for congruence or monotonicity lemmas, and
correspondingly no need to explicitly describe the syntactic context in which
some algebraic manipulation is being applied and we rely less on the unifier.
Another advantage is that one can design relations so that pattern-matching
suggestively decomposes complex relationships.
For example, given $F : \mathscr M \rel \mathscr M'$ and
$G : \mathscr N \rel \mathscr N'$, we can define a relation
$F \oplus G : \mathscr M \oplus \mathscr N \rel \mathscr M' \oplus \mathscr N'$
pointwise, so that a proof of $(x, x')(F \oplus G)(y, y')$ is a proof of $xFy$
together with a proof of $x'Gy'$.
Pattern-matching on such a proof immediately gives us these constituent parts,
whereas proofs of the corresponding statement involving functions would require
using a lemma to get the parts.
There is a dual advantage when producing one of these proofs, where we can
introduce the canonical constructor (for pairs, in this example) rather than
having to find the appropriate lemma.

Relations also have several disadvantages, though I have found that for my use
case, these are outweighed by the advantages.
For example, automated algebraic solvers are better developed for function-based
algebraic expressions, and sometimes the fact that functions satisfy unitality
and associativity up to decidable judgemental equality means that some proofs
can be avoided.
The handling of compound expressions can also be a disadvantage in that it
necessitates lots of new variable names and obscures goal and context displays.
Finally, in predicative systems such as Agda, relations typically live in a
larger universe than the corresponding functions.
In practice, this means quantifying over an extra level variable for each
relation involved in general lemmas.

There are more relations than there are functions, so statements involving
relations are more general than the corresponding statements involving
functions.
However, one part of the development requires functions rather than relations,
so I impose functionality on relations after the fact.
The appropriate notion of functional relation I use is slightly different to the
standard one, in that I take account of the order on the codomain, and thus ask
for the \emph{largest} solution rather than the \emph{unique} solution.

\begin{definition}\label{def:functional-linear-relation}
  A linear relation $\gr\Psi$ between $\mathscr M$ and $\mathscr N$ is
  \emph{(right-to-left) functional} if, for every $\grQ \in \mathscr N$, there
  exists universally a $\grP \in \mathscr M$ such that $\grP\gr\Psi\grQ$.
  Universality means that, for all $\grPprime$ such that $\grPprime\gr\Psi\grQ$,
  we have $\grPprime \leq \grP$ (i.e.\ $\grP$ is the largest solution).
\end{definition}

In Agda code, $\gr\Psi$ becomes \AgdaBound{$\Psi$} and the fact that $\gr\Psi$
relates $\grP$ and $\grQ$ (in this section written $\grP\gr\Psi\grQ$) is
rendered as
\PsiDot{rel}\AgdaSpace{}\AgdaBound{P}\AgdaSpace{}\AgdaBound{Q}.
That $\gr\Psi$ respects the orders on its arguments is given by
\PsiDot{rel-$\leq_m$}, and the various linearity properties are given by
\PsiDot{rel-0$_m$}, \PsiDot{rel-+$_m$}, and \PsiDot{rel-*$_m$}.

\section{A layer of syntax is functorial}\label{sec:functorial}

A basic property of the universe of syntaxes
is that every syntax supports a functorial action on subterms, realised by a
function \AgdaFunction{map-s}.
Its type says that to map a function \AgdaBound{f}
over a layer of syntax, there must be a linear map \AgdaBound{$\Psi$} relating the
domain and codomain usage contexts, and \AgdaBound{f} should be usable
wherever the domain and codomain usage contexts are similarly related by
\AgdaBound{$\Psi$}.

\ExecuteMetaData[\Maptex]{map-s-type}

This generality is needed because usage contexts change between
a term and its immediate subterms---they are decomposed according to the bunched connectives used in the rules.
\AgdaBound{X} and \AgdaBound{Y} are \AgdaFunction{ExtOpenFam}s, with
\AgdaBound{$\Theta$} being the context extension for a subterm (i.e., the
variables newly bound in that subterm).
Unlike usage annotations, types in the contexts \AgdaBound{$\gamma$} and \AgdaBound{$\delta$}, and the conclusion types implicit here, are preserved throughout.
This is the essence of the usage annotation based approach---we use traditional techniques for variable binding, with the usage annotations layered on top.

The heart of \AgdaFunction{map-s} is \AgdaFunction{map-p}, which recursively
works through the structure \AgdaBound{ps} of premises of the rule applied,
acting on each subterm it finds.
Here, particularly in the clauses for \AgdaInductiveConstructor{`$\sep$} and
\AgdaInductiveConstructor{`$\cdot$}, we see why it is not enough for the
function on subterms to apply at usage contexts \AgdaBound{P} and \AgdaBound{Q}
--- rather, it also needs to apply at any similarly related \AgdaBound{P$'$}
and \AgdaBound{Q$'$}.
In the case of \AgdaInductiveConstructor{`$\sep$}, we have that
$\grP \leq \grP_M + \grP_N$, with \AgdaBound{M} and \AgdaBound{N} being
collections of subterms in usage contexts $\grP_M$ and $\grP_N$, respectively.
Linearity of \AgdaBound{$\Psi$} yields $\grQ_M$ and $\grQ_N$ such that
$\grQ \leq \grQ_M + \grQ_N$ and we use \AgdaFunction{map-p} recursively at
$(\grP_M, \grQ_M)$ and $(\grP_N, \grQ_N)$ on \AgdaBound{M} and \AgdaBound{N}.
The cases for \AgdaInductiveConstructor{`$\cdot$} and
\AgdaInductiveConstructor{`$I^*$} are similar, each using a different aspect
of linearity.
In contrast, the cases for \AgdaInductiveConstructor{`$\dot1$} and
\AgdaInductiveConstructor{`$\dot\times$}, which are the only constructors used in fully structural
systems, do not involve any changes in the usage contexts.

The linearity of relation \AgdaBound{$\Psi$} is given by fields
\AgdaField{rel-0$_m$}, \AgdaField{rel-+$_m$}, and \AgdaField{rel-*$_m$} (with
the subscript-m being a mnemonic for \emph{module}, as opposed to scalar).

\ExecuteMetaData[\Maptex]{map-p}

I have also extended \AgdaFunction{map-p} to handle the various
$\Box$-modalities described in \cref{sec:dup-lnl}.
The Agda code for this extension is not particularly readable, so I do not
include it in this document.
However, this extension is notable as the only part of the framework requiring
that the linear relation \AgdaBound{$\Psi$} be functional (i.e., total and
deterministic).

\section{The Kripke function space}\label{sec:kripke}

At this point I introduce a minor generalisation to
\AgdaFunction{OpenFam} and \AgdaFunction{ExtOpenFam} (as defined in
\cref{sec:terms-of-system}):
\AgdaBound{I}\AgdaSpace{}\AgdaFunction{---OpenFam} and
\AgdaBound{I}\AgdaSpace{}\AgdaFunction{---ExtOpenFam}.  We obtain the
definition of \AgdaBound{I}\AgdaSpace{}\AgdaFunction{---OpenFam} by
replacing the textual occurrence of \AgdaBound{Ty} by the parameter
\AgdaBound{I}, though there is still reference to the ambient \AgdaBound{Ty}
via \AgdaRecord{Ctx}.
The main value I am interested in \AgdaBound{I} taking, other than
\AgdaBound{Ty}, is \AgdaRecord{Ctx} --- for example, the type family of
$\V$-environments, for a given $\V$, is a
\AgdaRecord{Ctx}\AgdaSpace{}\AgdaFunction{---OpenFam}%
\AgdaSpace{}\AgdaSymbol{\_}.
I use this generality in the type of \AgdaFunction{extend} in the next section.

\ExecuteMetaData[\Syntaxtex]{dashOpenFam}

The definition
\AgdaFunction{Kripke}\AgdaSpace{}\AgdaBound{$\V$}\AgdaSpace{}\AgdaBound{$\C$}%
\AgdaSpace{}\AgdaBound{$\Delta$} is a kind
of function space that describes a \AgdaBound{$\C$}-value parametrised by
\AgdaBound{$\Delta$}-many additional \AgdaBound{$\V$}-values (all correctly
typed and usage-annotated).
It is used to describe how to go under binders in a
Higher-Order Abstract Syntax style: To go under a binder we must
provide semantic interpretations for all the additional variables.

% When going under binders during a recursion, the context will be extended by some $\Theta$. This means that the current environment must be extended with $\Theta$s-worth of $\V$s

% we need the ability to say that

% Kripke V C is given the extension \Theta

% In \cref{sec:terms}, we defined \AgdaFunction{Scope} to let a
% judgement-indexed family admit context extensions. However, a key
% component of our generic semantic traversal is to make use of the open
% family \AgdaBound{$\V$} of \emph{values}, which are the sort of thing
% we store in an environment.  The definition \AgdaFunction{Kripke}
% gives an alternative to \AgdaFunction{Scope} which interprets the
% newly bound variables via a requirement of $\V$-values rather than
% extra assumptions for the $\C$-computation.

\ExecuteMetaData[\Semanticstex]{Kripke}

\AgdaFunction{Wrap}
is a device that turns any type family into an equivalent type family
that is judgementally injective in its indices, which helps with
Agda's type inference.
It turns the type family into a parametrised
record with a single field \AgdaField{get} whose type is the type in
the body of the $\lambda$-abstraction.
For understanding the meaning of
\AgdaFunction{Kripke}, \AgdaFunction{Wrap} can be ignored.

If $\Delta$ is of the form $\gr{s_1}B_1, \ldots, \gr{s_n}B_n$, then
\ExecuteMetaData[\Snippetstex]{KripkeVCDGA}\ is equivalent to
\ExecuteMetaData[\Snippetstex]{KripkeExpanded}\ by Currying.  That is
to say, the Kripke function is expecting a value for each newly bound
variable, at the multiplicity of its annotation, together with the
resources supporting each of those values. We use the ``magic wand''
function space here to enforce the invariant that the freshly bound
variables have usage annotations that are added to the existing
variables, not shared with them. The use of the
\AgdaFunction{$\Box^r$} modality ensures that we can still use it in
the presence of additional variables introduced by weakening.

\AgdaFunction{Kripke} is functorial in the \AgdaBound{$\C$} argument,
as witnessed by the \AgdaFunction{mapK$\C$} function, which is essentially
post-composition:

\ExecuteMetaData[\Semanticstex]{mapKC}

% is exemplified by the following construct
% \AgdaFunction{reify}, where we weaken \AgdaBound{$\Gamma$} by a $\gr0$ed-out
% version of \AgdaBound{$\Delta$}.
% The \AgdaBound{$\Delta$} then gets filled in by the $\V$-values.

% \bob{Move this para}
% This means that \AgdaBound{A} in the definition of \AgdaFunction{Kripke} has
% type \AgdaBound{I}, rather than specifically \AgdaBound{Ty}.
% We use this generality later in \AgdaFunction{extend}, setting \AgdaBound{I}
% to \AgdaDatatype{Ctx}.

\section{Semantic traversal}\label{sec:traversal}

We can now state the data required to implement a traversal assigning
semantics to terms. For open families $\V$ and $\C$, interpreting
variables and terms respectively, we assume that $\V$ is renameable
(i.e., that $\sdtstile{}\V A \rightarrowtriangle \Box\plr{\sdtstile{}\V A}$ for
all $A$),
that $\V$ is embeddable in $\C$, and that we have an algebra for a
layer of syntax, where bound variables are handled using the Kripke
function space:

% The aim of this subsection is to give an alternative recursion principle for
% terms that incorporates some of the environment-handling seen in the
% implementations of renaming and substitution.
% The rest of this section assumes the following: a renameable open family
% \AgdaBound{$\V$} that embeds into the open family \AgdaBound{$\C$}, and an
% algebra for a layer of syntax at \AgdaBound{$\C$}.

\ExecuteMetaData[\Semanticstex]{Semantics}

%\ExecuteMetaData[\Semanticstex]{Comp}

We mutually define the action \AgdaFunction{semantics} and its lemma
\AgdaFunction{body}.
The purpose of \AgdaFunction{semantics} is to turn a term into a
\AgdaBound{$\C$}-value using a \AgdaBound{$\V$}-environment and the fields of
\AgdaRecord{Semantics}.
Meanwhile, \AgdaFunction{body} does a similar job, but also deals with
newly bound variables.
In particular, \AgdaFunction{body} takes a term in a context extended by
\AgdaBound{$\Theta$}, and produces a Kripke function from
\AgdaBound{$\V$}-values for \AgdaBound{$\Theta$} to \AgdaBound{$\C$}-values.

\ExecuteMetaData[\Semanticstex]{semantics-type}

To implement the new recursor \AgdaFunction{semantics}, we use the standard
recursor, which in one case gives us a variable \AgdaBound{v}, and in the other
gives us a structure of subterms \AgdaBound{M}, each of which is in an extended
context.
To deal with a variable \AgdaBound{v}, we look it
up in the environment \AgdaBound{$\rho$}, then use the
\AgdaField{$\sem{\text{var}}$} field to map the resulting
\AgdaBound{$\V$}-value to a \AgdaBound{$\C$}-value.
To deal with a structure of subterms \AgdaBound{M}, we use the functoriality of
the syntactic structure to consider each subterm separately.
On a subterm, we apply \AgdaFunction{body}, which amounts to a recursive call
to \AgdaFunction{semantics} with an extended environment.
Recall that \AgdaFunction{relocate} (\cref{thm:env-resize}) adjusts the
environment \AgdaBound{$\rho$} to work in the usage contexts of the subterms.

\ExecuteMetaData[\Semanticstex]{semantics}

For \AgdaFunction{body}, we are given a subterm \AgdaBound{M}, to
which we want to apply \AgdaFunction{semantics}.  To do so, we need an
extended version of the initial environment \AgdaBound{$\rho$}. We
express this as the generation of a Kripke function that produces the
extended environment given interpretations of the fresh variables. We
take \AgdaBound{$\rho$}, which is an environment covering
\AgdaBound{$\Delta$}, and \AgdaBound{$\sigma$}, which is an
environment covering \AgdaBound{$\Theta$}, and glue them together
using the inductive rules for generating environments, after having
renamed \AgdaBound{$\rho$} via \cref{thm:env-ren} to make it fit the
new context \AgdaBound{$\Gamma^+$} (intended to be
\ExecuteMetaData[\Snippetstex]{GT}):

\ExecuteMetaData[\Semanticstex]{extend}

% The best we can achieve without identity environments for \AgdaBound{$\V$} is
% a Kripke function returning an extended environment.
To define \AgdaFunction{body}, we use \AgdaFunction{mapK$\C$} to
post-compose the environment extension by the
\AgdaSymbol{$\lambda$}-function taking an extended environment and
acting with it on \AgdaBound{M}.

\ExecuteMetaData[\Semanticstex]{body}

% \todo{FIX} Under the assumption that \AgdaBound{$\V$} is renameable, we can decompose
% \cref{thm:lr-bind} as
% \AgdaFunction{reify}\AgdaSpace{}\AgdaOperator{\AgdaFunction{$\circ$}}%
% \AgdaSpace{}\AgdaFunction{extend}, with \AgdaFunction{extend} defined below.
% We can think of \AgdaFunction{extend} as our best effort to extend an
% environment by \AgdaBound{$\Theta$} without access to an identity environment
% at \AgdaBound{$\Theta$}.

\AgdaFunction{semantics} is the fundamental lemma of the framework.
With it proven, I move onto corollaries and specific applications.

\section{Reifying the Kripke function space}\label{sec:reify}

A result I will use throughout the rest of this thesis is \emph{reification}.
When we have an index-preserving mapping from usage-checked variables to
$\V$-environments, we can construct environments of the form
$\Gamma \env\V \Gamma$ (identity environments) for all $\Gamma$.
This lets us write the \AgdaFunction{reify} function, which  simplifies our
obligations in giving a \AgdaRecord{Semantics} by coercing
\AgdaFunction{Kripke} functions into just
\AgdaBound{$\C$}-values in an extended context.

\begin{lemma}[\AgdaFunction{reify}]\label{thm:reify}
  If $\V$ is an open family such that there is a function
  $v : {\sqni} \rightarrowtriangle \V$, we get a function of type
  $\mathrm{Kripke}\,\V\,\C \rightarrowtriangle \mathrm{Scope}\,\C$ for any $\C$.
\end{lemma}
\begin{proof}
  Let $b : \mathrm{Kripke}\,\V\,\C\,\Delta\,\Gamma\,A$.
  That is, $b$ is a Kripke function yielding $\C$-computations
  We want to apply $b$ so as to get a $\C\,\plr{\Gamma, \Delta}\,A$.
  Let $\grP\gamma = \Gamma$ and $\grQ\delta = \Delta$.
  The $\Box^r$ in the type of $b$ allows us to reverse-rename $\Gamma$ to
  $\Gamma, \gr0\delta$.
  Then we give the $\wand$-function an argument in context
  $\gr0\gamma, \Delta$, noting that
  $\plr{\Gamma, \gr0\delta} + \plr{\gr0\gamma, \Delta} = \plr{\Gamma, \Delta}$,
  as we wanted from the result.
  The argument needs type $\gr0\gamma, \Delta \env\V \Delta$.
  We produce this via \cref{thm:env-postren} from an environment
  $\rho : \gr0\gamma, \Delta \env\V \gr0\gamma, \Delta$ created using $v$
  and a renaming which is the complement to that used on $\Box^r$.
\end{proof}

All of the $\V$s used in examples in this paper support identity environments.
However, \citet[p.~27]{AACMM21} give some important examples that do not
support identity environments, and thus cannot use \AgdaFunction{reify}
(\cref{thm:reify}).
The feature that causes the lack of support for identity environments is that
a semantics can make use of the fact that only variables of particular kinds
are bound by the syntax.
In the examples of \citeauthor{AACMM21}, a bidirectionally typed language only
binds variables that synthesise their type, as opposed to those whose type is
checked.
The semantics of type-checking and elaboration rely on variables synthesising
their type, so \AgdaBound{$\V$} is chosen to cover only those variables.
Instead of using \AgdaFunction{reify}, we must observe that each syntactic form
only binds such synthesising variables.
Similar phenomena would appear in, say, a call-by-value language where
variables are values (not computations), or a polarised language where
variables always have a polarity matching their type.

\section{Renaming and substitution}\label{sec:kit-to-sem}

The final completely syntax-generic result I present is simultaneous
substitution.
I derive this as I did in the simply typed case in \cref{sec:gen-sem}:
I first show that a syntactic kit can be turned into a semantics, and then by
instantiating the notion of kit for, in turn, renaming and substitution, the
general semantic traversal gives the result we want.

The notion of \AgdaRecord{Kit} is essentially the same as in the simply typed
case, once we allow for changes to the basic definitions of variables, terms,
and environments (in particular, renamings).

\ExecuteMetaData[\Syntactictex]{Kit}

The first two fields of \AgdaRecord{Semantics} derive directly from fields of
\AgdaRecord{Kit}.
Meanwhile, to handle term constructors, we first \AgdaFunction{reify} to get a
collection of traversed subterms, and then use \AgdaInductiveConstructor{`con}
to assemble these subterms into a similarly shaped syntactic form as we started
with.
The \AgdaField{vr} field is used implicitly in \AgdaFunction{reify}, as it is
used to show that $\V$-identity environments exist.

\ExecuteMetaData[\Syntactictex]{kit-to-sem}

The action of a syntactic traversal on logical rules is basically fixed: we
preserve the logical rule and extend the environment with any newly bound
variables according to \AgdaField{vr}.
Meanwhile, the action on variables is relatively unconstrained: we look up the
variable in the environment to get a $\V$-value, then transform that $\V$-value
into a term using \AgdaField{tm}.

The idea of simultaneous renaming is that variables replace variables, whereas
with simultaneous substitution, terms replace variables.
This translates to environments for renaming containing $\sqni$-values
(variables), and environments for substitution containing $\vdash$-values
(terms).

%To implement renaming and substitution for terms, we now just implement
%syntactic kits for variables and terms, respectively.

\ExecuteMetaData[\Syntactictex]{Ren-Kit}

Notice that \AgdaFunction{ren\textasciicircum$\vdash$}, witnessing the fact
that terms are renameable, is a corollary of \AgdaFunction{Ren-Kit}.

\ExecuteMetaData[\Syntactictex]{Sub-Kit}


\section{Example traversals}\label{sec:examples}
TODO: introductory paragraph

\bob{Reification dumped here for now}
When \AgdaBound{$\V$} supports identity environments (as per \cref{thm:env-id}),
we can simplify our obligations in giving a \AgdaRecord{Semanitics} by coercing
\AgdaFunction{Kripke} functions into just
\AgdaBound{$\C$}-values in an extended context.

%To show this, we instantiate the Kripke function with the renaming
%$\swarrow^r : \Gamma, \gr0\delta \env\sqni \Gamma$ to extend the scope, and
%pass it an argument environment
%$\plr{\id^\V; \searrow^r} : \gr0\gamma, \Delta \env\V \Delta$ to fill in the
%extended part.
%Post-composition of a renaming onto an environment comes from
%\cref{thm:env-postren}.

%\ExecuteMetaData[\Syntactictex]{reify}

\begin{lemma}[\AgdaFunction{reify}]\label{thm:reify}
  If $\V$ is an open family such that there is a function
  $v : \forallb{{\sqni} \dotto \V}$, we get a function of type
  $\forallb{\mathrm{Kripke}\,\V\,\C \dotto \mathrm{Scope}\,\C}$ for any $\C$.
\end{lemma}
\begin{proof}
  Let $b : \mathrm{Kripke}\,\V\,\C\,\Delta\,\Gamma\,A$.
  That is, $b$ is a Kripke function yielding $\C$-computations
  We want to apply $b$ so as to get a $\C\,\plr{\Gamma, \Delta}\,A$.
  Let $\grP\gamma = \Gamma$ and $\grQ\delta = \Delta$.
  The $\Box^r$ in the type of $b$ allows us to reverse-rename $\Gamma$ to
  $\Gamma, \gr0\delta$.
  Then we give the $\wand$-function an argument in context
  $\gr0\gamma, \Delta$, noting that
  $\plr{\Gamma, \gr0\delta} + \plr{\gr0\gamma, \Delta} = \plr{\Gamma, \Delta}$,
  as we wanted from the result.
  The argument needs type $\gr0\gamma, \Delta \env\V \Delta$.
  We produce this through the composition
  $\gr0\gamma, \Delta \env\V \gr0\gamma, \Delta \env{\sqni} \Delta$,
  using the instance of composition mentioned in \cref{thm:env-postren}.
  The $\V$-environment follows from $v$ and \cref{thm:env-id}, while the
  renaming is the complement to that used on $\Box^r$.
\end{proof}

All of the $\V$s used in examples in this paper support identity environments.
However, \citet[p.~27]{AACMM21} give some important examples that do not
support identity environments, and thus cannot use \AgdaFunction{reify}
(\cref{thm:reify}).
The feature that causes the lack of support for identity environments is that
a semantics can make use of the fact that only variables of particular kinds
are bound by the syntax.
In the examples of \citeauthor{AACMM21}, a bidirectionally typed language only
binds variables that synthesise their type, as opposed to those whose type is
checked.
The semantics of type-checking and elaboration rely on variables synthesising
their type, so \AgdaBound{$\V$} is chosen to cover only those variables.
Instead of using \AgdaFunction{reify}, we must observe that each syntactic form
only binds such synthesising variables.
Similar phenomena would appear in, say, a call-by-value language where
variables are values (not computations), or a polarised language where
variables always have a polarity matching their type.


\subsection{Renaming and substitution}\label{sec:kits}

In an unpublished note, \citet{McBride05} gives a parametrised traversal
yielding homomorphisms of syntax, the canonical examples of which are
simultaneous renaming and simultaneous substitution.
The parameters are collected in the record \AgdaRecord{Kit}.
We make a minor change to the original presentation, where instead of our
\AgdaField{ren\textasciicircum{}$\V$} field, \citeauthor{McBride05} has the
field \AgdaField{wk} allowing only context extensions.
As for the other two fields, \AgdaField{vr} allows us to map variables to
$\V$-values, so as to put newly bound variables in environments; and
\AgdaField{tm} allows us to extract terms from $\V$-values, as required when
we use the environment to handle a free variable.

\ExecuteMetaData[\Syntactictex]{Kit}

Where \citeauthor{McBride05} gave the traversal explicitly, we go via our
generic semantic traversal.
The first two fields of \AgdaRecord{Semantics} derive directly from fields of
\AgdaRecord{Kit}.
Meanwhile, to handle term constructors, we first \AgdaFunction{reify} to get a
collection of traversed subterms, and then use \AgdaInductiveConstructor{`con}
to assemble these subterms into a similarly shaped syntactic form as we started
with.
The \AgdaField{vr} field is used implicitly in \AgdaFunction{reify}, as it is
used to show that $\V$-identity environments exist.

\ExecuteMetaData[\Syntactictex]{kit-to-sem}

The action of a syntactic traversal on logical rules is basically fixed: we
preserve the logical rule and extend the environment with any newly bound
variables according to \AgdaField{vr}.
Meanwhile, the action on variables is relatively unconstrained: we look up the
variable in the environment to get a $\V$-value, then transform that $\V$-value
into a term using \AgdaField{tm}.

The idea of simultaneous renaming is that variables replace variables, whereas
with simultaneous substitution, terms replace variables.
This translates to environments for renaming containing $\sqni$-values
(variables), and environments for substitution containing $\vdash$-values
(terms).

%To implement renaming and substitution for terms, we now just implement
%syntactic kits for variables and terms, respectively.

\ExecuteMetaData[\Syntactictex]{Ren-Kit}

Notice that \AgdaFunction{ren\textasciicircum$\vdash$}, witnessing the fact
that terms are renameable, is a corollary of \AgdaFunction{Ren-Kit}.

\ExecuteMetaData[\Syntactictex]{Sub-Kit}

\subsection{A usage elaborator}\label{sec:usage-elaborator}

Using the constructs we have seen so far, producing example terms soon becomes
extremely tedious.
We can achieve some abbreviation by using pattern synonyms to wrap around
\AgdaInductiveConstructor{`con} expressions, but we still have to
produce essentially bespoke proofs whenever we use a usage-sensitive part of the
syntax.
The size of each of these proofs is roughly proportional to the number of free
variables, so the amount of proof we have to write grows roughly quadratically
with the size of terms.
An additional factor, which we can't see on paper, is that type checking time
for these proofs soon becomes prohibitive to interactive development.

Our aim in this subsection is to automate usage constraint proofs, making terms
both easier to write and more performant to check.
We invoke the automation by writing terms in a syntax where usage constraints
have been trivialised, and then use a semantic traversal over the simplified
syntax to try to produce a fully elaborated term in the original syntax.
We write the automation in a way that is generic in the syntax description, thus
avoiding repetition and facilitating the prototyping of new type systems.

The type of syntax descriptions depends on the type of usage annotations because
of variable binding.
For example, in the $\oc{\gr r}$-E rule of \cref{fig:lr-comb}, the right
premise binds a new variable with annotation $\gr r$, where $\gr r$ is drawn
from the ambient posemiring.
The scaling combinator also makes direct reference to the posemiring.
To produce a simplified syntax description, where usage constraints are
trivialised, we set the ambient posemiring to the 1-element $\mathbf0$
posemiring.
In contrast to syntax descriptions, even though types can contain usage
annotations, the type of types does not depend on the type of usage annotations.
This means that, in our simplified syntax, terms have types from the original
system even though variables have trivial usage annotations.
We define the $\mathbf0$ posemiring as follows, being careful to use the
0-field record type \AgdaRecord{$\top$} so that everything algebraic gets
solved by Agda's $\eta$-laws.
Indeed, in this very definition, all of the semiring operations and laws are
canonically inferred.

\ExecuteMetaData[\UsageChecktex]{0-poSemiring}

The elaboration process is monadic.
In particular, we use the \AgdaDatatype{List}/non-determinism monad to give
\emph{all} of the possible annotation choices on the free variables of a term.
We believe the commitment to multiple solutions is inherent when the syntax
contains \AgdaInductiveConstructor{`$\dot1$}.
For example, in the intermediate stages of elaborating
$\plr{\vdash \lambda x.~\plr{*,*}} : A \multimap \top \otimes \top$ with a
usage counting posemiring (assuming reasonable rules for $\top$ and $\otimes$),
it is unclear whether to use the variable $x$ in the left $*$ or the right $*$.
This uncertainty should be reflected in the final result.

The non-deterministic choices we make during elaboration are enumerated by
the fields of \AgdaRecord{NonDetInverses}.
These choices are driven by the typing rules and a candidate usage vector for
the conclusion.
For example, \AgdaField{+$^{-1}$}\AgdaSpace{}\AgdaBound{r} is needed when we
encounter a \AgdaInductiveConstructor{`$\sep$} in the syntax and the candidate
usage annotation we are considering is \AgdaBound{r}.
Then, \AgdaField{+$^{-1}$}\AgdaSpace{}\AgdaBound{r} is a list of pairs of
annotations \AgdaBound{p} and \AgdaBound{q} that \AgdaBound{r} can split into,
together with a proof of the splitting.
For \AgdaField{0\#$^{-1}$} and \AgdaField{1\#$^{-1}$}, inverses to constants,
we are given the candidate \AgdaBound{r} and typically return an empty list if
the constraint cannot be satisfied, or a singleton list containing a proof.
\AgdaField{*$^{-1}$} is used when we encounter scaling, in which case we know
both the scaling factor \AgdaBound{r} (from the syntax description) and the
candidate \AgdaBound{q}.
These inverse operations combine monadically (in fact, applicatively) to give
inverses to the vector operations of zero, addition, scaling, and basis.

\ExecuteMetaData[\UsageChecktex]{NonDetInverses}

We choose the \AgdaBound{$\V$} of our semantics to be (unannotated) variables.
For the \AgdaBound{$\C$}, we consider \emph{functions} from candidate usage
vectors \AgdaBound{R} to the list of elaborated derivations with usage
annotations given by \AgdaBound{R}.
The protocol this encodes is that the user will provide an unannotated term
together with a candidate usage context \AgdaBound{R}, and usage elaboration
returns a list of possible ways the term could be annotated such that the
conclusion has usage context \AgdaBound{R}.
The module name \AgdaModule{U} refers to the fact that we are taking the
ambient posemiring to be $\mathbf0$ in \AgdaFunction{OpenFam}.
The effect on \AgdaFunction{OpenFam} is that the usage annotations of any
contexts we consider are uninformative (hence the \AgdaSymbol{\_} on the left).

\ExecuteMetaData[\UsageChecktex]{C}

To traverse the unannotated terms, we produce a \AgdaRecord{Semantics} over the
unannotated system \AgdaFunction{uSystem}\AgdaSpace{}\AgdaBound{sys}.
To write it, we make use of idiom brackets
\AgdaSymbol{(|}\AgdaSpace{}\AgdaSymbol{$\ldots$}\AgdaSpace{}\AgdaSymbol{|)},
which have the effect of replacing top-level spines of applications by
(\AgdaDatatype{List}-)applicative applications.
Field by field, we already know that variables are renameable.
To interpret a variable, we consider all the possible proofs that such a
variable could be well annotated, and package them up as a variable term via
the applicative machinery.
Finally, for compound terms, we first reify the unannotated subterms, and then
combine the subterms via a \AgdaFunction{lemma}.

\ExecuteMetaData[\UsageChecktex]{elab-sem}

The \AgdaFunction{lemma} essentially goes through the shape of the premises,
combining the collections of subterms in the natural way.
For example, at each
\AgdaInductiveConstructor{\AgdaUnderscore{}$\dottimes$\AgdaUnderscore{}},
we take the Cartesian product of the possibilities of each half, and at each
\AgdaInductiveConstructor{\AgdaUnderscore{}$\sep$\AgdaUnderscore{}},
we non-deterministically split the usage annotations coming in, and then take
the Cartesian product.
When it comes to newly bound variables, the \emph{syntax description} tells us
their annotations, so there is no further non-determinism introduced here.

\ExecuteMetaData[\UsageChecktex]{lemma-type}

To actually use \AgdaFunction{elab-sem} on terms, we take the associated
\AgdaFunction{semantics} and pass it the identity environment (an identity
\emph{renaming} in this case, because $\V$ is a family of variables).
We use \AgdaFunction{elab-unique}, which further
checks statically that exactly one derivation is returned.
The candidate usage vector \AgdaBound{R} will be \AgdaFunction{[]} for closed
terms, and otherwise we have to supply the intended usage annotations.

The example below is of a term needed to show that, in the
$\{\gr0, \gr1, \gr\omega\}$ (linearity) posemiring of \cref{thm:linearity},
$\oc\gr\omega$ forms a comonad.
We have instantiated the usage elaborator so that:
\AgdaField{0\#$^{-1}$} is a singleton on $\gr0$ and $\gr\omega$, and empty on
$\gr1$;
\AgdaField{1\#$^{-1}$} is a singleton on $\gr1$ and $\gr\omega$, and empty on
$\gr0$;
\AgdaField{+$^{-1}$} gives $\gr0 \mapsto [(\gr0,\gr0)]$,
$\gr1 \mapsto [(\gr0,\gr1),(\gr1,\gr0)]$, and
$\gr\omega \mapsto [(\gr\omega,\gr\omega)]$; and
\AgdaField{*$^{-1}$} gives $(\gr\omega, \gr0) \mapsto [\gr0]$,
$(\gr\omega, \gr1) \mapsto []$, and
$(\gr\omega, \gr\omega) \mapsto [\gr\omega]$
(omitting $(\gr0, \_)$ and $(\gr1, \_)$ cases for brevity).
Note that we do not consider splitting $\gr\omega$ up as, say,
$\gr1 + \gr\omega$, because this splitting would introduce more
non-determinism but not allow any more terms to be typed.
As such, the only non-determinism comes when we have variables annotated
$\gr1$ and need to do an additive split, like when we apply the
\AgdaInductiveConstructor{!E} rule below.
At this point, the variable can become either $\gr0$-annotated in the left
subterm and $\gr1$-annotated on the right, or vice-versa.
We will find that, because the left subterm wants to use that variable, the
former choice will be rejected.
The function \AgdaFunction{var\#} is a convenience for converting statically
known natural numbers, representing de Bruijn \emph{levels}, into variable
terms.

\ExecuteMetaData[\PaperExamplestex]{cojoin}

\subsection{A denotational semantics}

To justify the name \emph{semantics}, we give an example traversal that is a
denotational semantics in the usual sense.
The semantics we take is a refinement of that of \citet{AbelBernardy2020},
which gives a way to extract parametricity theorems from substructurally typed
programs.
Example theorems are that all linear terms act as permutations on some fixed
set of resources, and all monotonically typed terms are really monotonic in the
way the typing suggests they are.

To abbreviate this section, we use a simplified syntax compared to \name{}.
We allow for an arbitrary family of base types \AgdaBound{BaseTy}, and a single
type former \mbox{\ExecuteMetaData[\WReltex]{rAToB}}, equivalent to
\mbox{\ExecuteMetaData[\PaperExamplestex]{BangrAToB}} from the earlier system.

\ExecuteMetaData[\WReltex]{Ty}

In the term syntax, $\lambda$-abstraction now binds a variable with annotation
\AgdaBound{r}, while application needs to scale its argument by \AgdaBound{r}
(both in accordance with the function type they are acting on).

\ExecuteMetaData[\WReltex]{AnnArr}

As a running example, we take the usage annotations to be the 4-element
variance posemiring (\cref{thm:variance}).
We establish the property that all terms are monotonic in their free variables.
This monotonicity can be covariant or contravariant (or neither or both)
depending on the annotation of each free variable.
This provides an additional example to those of \citeauthor{AbelBernardy2020}.

We will take semantics of this system into
\emph{world-indexed relations}~\cite{AbelBernardy2020,context-constrained}.
A world-indexed relation over a poset of worlds \AgdaBound{W} is a set over
which
we have a \AgdaBound{W}-indexed binary relation satisfying a presheaf-like
property with respect to the order on \AgdaBound{W}.

\ExecuteMetaData[\WReltex]{WRel}

\begin{example}
  When \AgdaBound{W} is the 1-element set, a world-indexed relation is just a
  set equipped with a binary relation.
\end{example}

Morphisms between world-indexed relations \AgdaBound{R} and \AgdaBound{S}
consist of a mapping between the underlying sets such that that mapping
preserves relatedness from \AgdaBound{R} to \AgdaBound{S}.

\ExecuteMetaData[\WReltex]{WRelMor}

\todo{Define big intersection.}
When the poset of worlds forms a (relational) commutative monoid, such
world-indexed relations support a symmetric monoidal closed structure, with
objects denoted \AgdaFunction{I$^R$},
\AgdaFunction{\AgdaUnderscore{}$\otimes^R$\AgdaUnderscore{}}, and
\AgdaFunction{\AgdaUnderscore{}$\multimap^R$\AgdaUnderscore{}},.
These reuse the bunched connectives \AgdaRecord{$I^*$}, \AgdaRecord{$\sep$}, and
\AgdaRecord{$\wand$}, now over worlds rather than contexts.

%\ExecuteMetaData[\WReltex]{IR}
%\ExecuteMetaData[\WReltex]{tensorR}
%\ExecuteMetaData[\WReltex]{lollyR}

The final piece of sematics we need is a \emph{bang} operator.
We allow the
semantic \emph{bang} to be an arbitrary annotation-indexed functor on
world-indexed relations.
This functor must respect all of the structure on the indices, making it a
graded comonad over multiplication, as well as being lax monoidal at any
particular index \AgdaBound{r}.

\ExecuteMetaData[\WReltex]{Bang}

\begin{example}
  With \AgdaBound{W} being the 1-element set and annotations coming from the
  variance semiring, we can define the following \emph{bang}.
  It is always the identity on the set component, while the relation component
  consists of flipping the relation for contravariance and taking conjunctions
  to achieve both covariance and contravariance.
  When we want neither covariance nor contravariance, we use the always true
  predicate on worlds \AgdaFunction{$\dot1$}.

  \ExecuteMetaData[\Monotonicitytex]{BangR}
\end{example}

To associate semantics to syntax, we start as standard by associating
world-indexed relations to types.
We also extend the semantics of types to contexts, using \AgdaFunction{I$^R$},
\AgdaFunction{$\otimes^R$}, and \AgdaField{!$^R$} to interpret the empty
context, concatenation, and usage annotations on singletons, respectively.

\ExecuteMetaData[\WReltex]{sem}

The semantics of a term is then to be a morphism from the interpretation of the
context to the interpretation of the term's type.

\ExecuteMetaData[\WReltex]{sem-vdash}

Variables are given semantics by \AgdaFunction{lookup$^R$} (definition omitted).

\ExecuteMetaData[\WReltex]{lookupR-type}

Now, we give a \AgdaRecord{Semantics}.
The choice of \AgdaBound{$\V$} as
\AgdaRecord{\AgdaUnderscore{}$\sqni$\AgdaUnderscore{}} is somewhat arbitrary,
given that a standard denotational semantics would not use intermediate
environments in the same sense as renaming and substitution, but allows us to
reuse the standard facts that variables support renaming and identity
environments.
With this choice of \AgdaBound{$\V$} and \AgdaBound{$\C$}, we interpret
environment entries by \AgdaFunction{lookup$^R$}.
Meanwhile, for the logical rules, we ignore environments by using
\AgdaFunction{reify} to just deal with morphisms in an extended context.
As such, $\lambda$-abstractions are easy to interpret, while applications
require some massaging to remove the extension by an empty context, followed by
some plumbing to split the interpretation of the context according to the usage
constraints and feed the interpretation of the argument \AgdaBound{n$'$} into
the interpretation of the function \AgdaBound{m$'$}.

\ExecuteMetaData[\WReltex]{Wrel}

In order to map open terms to interpretations, we take the action of the
semantics and give the identity renaming as the starting environment.

\ExecuteMetaData[\WReltex]{wrel}

\begin{example}\label{thm:minus}
  We can make a subtraction function from primitive addition and negation on
  integers.
  Subtraction is covariant in its first argument and contravariant in its
  second argument.
  We give the definition in pseudocode, though it is also amenable to the
  usage elaborator of \cref{sec:usage-elaborator}, suitably instantiated.

  \begin{align*}
    &{\sim\sim}p :
      {\uparrow\uparrow}\mathbb Z \multimap
      {\uparrow\uparrow}\mathbb Z \multimap \mathbb Z,
      {\sim\sim}n : {\downarrow\downarrow}\mathbb Z \multimap \mathbb Z
      \vdash \mathnormal{minus} :
      {\uparrow\uparrow}\mathbb Z \multimap
      {\downarrow\downarrow}\mathbb Z \multimap
      \mathbb Z
    \\
    &\mathnormal{minus} \coloneqq \lambda x.~\lambda y.~p\,x\,(n\,y)
  \end{align*}

  After feeding in Agda's addition and negation functions as the
  interpretations of the free variables (noting that they are both monotonic
  in the required way), we get the following free theorem.

  \ExecuteMetaData[\Monotonicitytex]{thm-type}

  %We observe that the set component of this term's semantics is just the
  %expected Agda function when the two free variables are given appropriate
  %interpretations.

  %\ExecuteMetaData[\Monotonicitytex]{minus-set}

  %Furthermore, the relational component of the semantics yields the free
  %theorem that the Agda subtraction so defined is monotonic in the expected way.
  %This relies on library proofs that addition and negation are suitably
  %monotonic.

  %\ExecuteMetaData[\Monotonicitytex]{thm}
\end{example}


\section{Conclusions}\label{sec:conc}

\bibliography{quant-framework}

\end{document}
