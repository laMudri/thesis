\documentclass[sigplan,10pt,anonymous,review]{acmart}
\settopmatter{printfolios=true,printccs=false,printacmref=false}

\citestyle{acmauthoryear}
\bibliographystyle{ACM-Reference-Format}

\usepackage[conor]{agda}
\usepackage{catchfilebetweentags}
\usepackage{cleveref}
\usepackage{cmll}
\usepackage{ebproof}
\usepackage{mathrsfs}
\usepackage{mathtools}
\usepackage{newunicodechar}
\usepackage{stmaryrd}
\setlength{\marginparwidth}{2cm}
\usepackage{todonotes}
\usepackage{turnstile}

\definecolor{use}{HTML}{008000}
\newcommand\gr[1]{{\color{use}#1}}
\newcommand\grctx[1]{\gr{\mathcal{#1}}}
\newcommand\grP{\grctx P}
\newcommand\grQ{\grctx Q}
\newcommand\grR{\grctx R}
\newcommand\grPprime{\grP\gr'}
\newcommand\grQprime{\grQ\gr'}
\newcommand\name{\ensuremath{\lambda\grR}}
\newcommand\grctxsub[2]{\grctx{#1}_{\gr{#2}}}
\newcommand\grPe{\grctxsub P e}
\newcommand\grPf{\grctxsub P f}
\newcommand\grQe{\grctxsub Q e}
\newcommand\grQf{\grctxsub Q f}
\newcommand\sem[1]{\left\llbracket{#1}\right\rrbracket}
\newcommand\size[1]{\left\lvert{#1}\right\rvert}
\newcommand\ps{\mathit{ps}}
\newcommand\qs{\mathit{qs}}
\newcommand\rs{\mathit{rs}}
\newcommand\dotto{\mathrel{\dot\to}}
\newcommand\dottimes{\mathbin{\dot\times}}
\newcommand\wand{\mathrel{\mathord{-}\hspace{-0.75ex}*}}
\newcommand\sep{\mathbin{*}}
\newcommand\env[1]{(#1\mathrm{-Env})}
\newcommand\thinningN{\mathrm{Thinning}}
\newcommand\thinning[2]{\thinningN~#1~#2}
\newcommand\V{\mathcal V}
\newcommand\C{\mathcal C}
\newcommand\sqin{\mathrel{\mathrlap{\sqsubset}{\mathord{-}}}}
\newcommand\sqni{\mathrel{\mathrlap{\sqsupset}{\mathord{-}}}}
\newcommand\subres{=}
\newcommand\Ann{\mathscr R}

\renewcommand\land{~\wedge~}
\renewcommand\lor{~\vee~}

\DeclareMathOperator\obj{Obj}
\let\hom\relax
\DeclareMathOperator\hom{Hom}
\DeclareMathOperator\id{id}
\DeclareMathOperator\sub{Sub}

\usepackage[T1]{fontenc}

\newunicodechar{λ}{\ensuremath{\mathnormal\lambda}}
\newunicodechar{ρ}{\ensuremath{\mathnormal\rho}}
\newunicodechar{→}{\ensuremath{\mathnormal\to}}
\newunicodechar{∀}{\ensuremath{\mathnormal\forall}}
\newunicodechar{ι}{\ensuremath{\mathnormal\iota}}
\newunicodechar{·}{\ensuremath{\mathnormal\cdot}}
\newunicodechar{⊸}{\ensuremath{\mathnormal\multimap}}
\newunicodechar{⊕}{\ensuremath{\mathnormal\oplus}}
\newunicodechar{─}{\textrm{---}}
\newunicodechar{│}{\ensuremath{\mid}}
\newunicodechar{ᶜ}{\ensuremath{\mathnormal{^c}}}
\newunicodechar{ᵉ}{\ensuremath{\mathnormal{^e}}}
\newunicodechar{ᵏ}{\ensuremath{\mathnormal{^k}}}
\newunicodechar{ₗ}{\ensuremath{\mathnormal{_l}}}
\newunicodechar{ₘ}{\ensuremath{\mathnormal{_m}}}
\newunicodechar{ₙ}{\ensuremath{\mathnormal{_n}}}
\newunicodechar{ᵣ}{\ensuremath{\mathnormal{_r}}}
\newunicodechar{ʳ}{\ensuremath{\mathnormal{^r}}}
\newunicodechar{ˢ}{\ensuremath{\mathnormal{^s}}}
\newunicodechar{ᵗ}{\ensuremath{\mathnormal{^t}}}
\newunicodechar{ᵛ}{\ensuremath{\mathnormal{^v}}}
\newunicodechar{ᴹ}{\ensuremath{\mathnormal{^M}}}
\newunicodechar{↙}{\ensuremath{\mathnormal\swarrow}}
\newunicodechar{↘}{\ensuremath{\mathnormal\searrow}}
\newunicodechar{⊢}{\ensuremath{\mathnormal\vdash}}
\newunicodechar{⊨}{\ensuremath{\mathnormal\vDash}}
\newunicodechar{⟦}{\ensuremath{\mathnormal\llbracket}}
\newunicodechar{⟧}{\ensuremath{\mathnormal\rrbracket}}
\newunicodechar{✴}{\ensuremath{\mathnormal*}}
\newunicodechar{ℓ}{\ensuremath{\mathnormal\ell}}
\newunicodechar{Γ}{\ensuremath{\mathnormal\Gamma}}
\newunicodechar{γ}{\ensuremath{\mathnormal\gamma}}
\newunicodechar{Δ}{\ensuremath{\mathnormal\Delta}}
\newunicodechar{δ}{\ensuremath{\mathnormal\delta}}
\newunicodechar{Θ}{\ensuremath{\mathnormal\Theta}}
\newunicodechar{Σ}{\ensuremath{\mathnormal\Sigma}}
\newunicodechar{σ}{\ensuremath{\mathnormal\sigma}}
\newunicodechar{∈}{\ensuremath{\mathnormal\in}}
\newunicodechar{∋}{\ensuremath{\mathnormal\ni}}
\newunicodechar{′}{\ensuremath{\mathnormal'}}
\newunicodechar{≡}{\ensuremath{\mathnormal\equiv}}
\newunicodechar{⊤}{\ensuremath{\mathnormal\top}}
\newunicodechar{⊥}{\ensuremath{\mathnormal\bot}}
\newunicodechar{▹}{\ensuremath{\mathnormal\triangleright}}
\newunicodechar{₁}{\ensuremath{\mathnormal{_1}}}
\newunicodechar{□}{\ensuremath{\mathnormal\Box}}
\newunicodechar{○}{\ensuremath{\mathnormal\bigcirc}}
\newunicodechar{𝓒}{\ensuremath{\C}}
\newunicodechar{𝓥}{\ensuremath{\V}}
\newunicodechar{∘}{\ensuremath{\mathnormal\circ}}
\newunicodechar{≤}{\ensuremath{\mathnormal\leq}}
\newunicodechar{◇}{\ensuremath{\mathnormal\diamond}}
\newunicodechar{ℕ}{\ensuremath{\mathbb N}}
\newunicodechar{⁺}{\ensuremath{\mathnormal{^+}}}
\newunicodechar{⊔}{\ensuremath{\mathnormal\sqcup}}
\newunicodechar{⇒}{\ensuremath{\mathnormal\Rightarrow}}

% Characters that are different from what appears in the source code
\newunicodechar{ℑ}{\ensuremath{\mathnormal{I^*}}}
\newunicodechar{⇛}{\ensuremath{\mathnormal{\Longrightarrow}}}
\newunicodechar{∩}{\ensuremath{\mathnormal\dottimes}}
\newunicodechar{∧}{\ensuremath{\mathnormal\dottimes}}
\newunicodechar{⒈}{\ensuremath{\mathnormal{\dot1}}}
\newunicodechar{⇴}{\ensuremath{\mathnormal\dotto}}
\newunicodechar{⇥}{\ensuremath{\mathnormal\wand}}


\def\genericlr{../../generic-lr/src/processed-latex}
\def\Syntaxtex{\genericlr/Generic/Linear/Syntax.tex}
\def\Interpretationtex{\genericlr/Generic/Linear/Syntax/Interpretation.tex}
\def\Maptex{\genericlr/Generic/Linear/Syntax/Interpretation/Map.tex}
\def\Termtex{\genericlr/Generic/Linear/Syntax/Term.tex}
\def\Semanticstex{\genericlr/Generic/Linear/Semantics.tex}
\def\Syntactictex{\genericlr/Generic/Linear/Semantics/Syntactic.tex}
\def\Renamingtex{\genericlr/Generic/Linear/Renaming.tex}
\def\PaperExamplestex{\genericlr/Generic/Linear/Example/PaperExamples.tex}
\def\Snippetstex{../../agda/processed-latex/Snippets.tex}

\begin{document}

\title{A Framework for Substructural Type Systems}

\author{James Wood}
\orcid{0000-0002-8080-3350}
\affiliation{%
  \institution{University of Strathclyde}%
  \city{Glasgow}%
  \country{UK}%
}
\email{james.wood.100@strath.ac.uk}

\author{Robert Atkey}
\orcid{0000-0002-4414-5047}
\affiliation{%
  \institution{University of Strathclyde}%
  \city{Glasgow}%
  \country{UK}%
}
\email{robert.atkey@strath.ac.uk}

\keywords{substructural, linear, metatheory, type system, mechanisation}

\begin{abstract}
  Mechanisation of programming language research is of growing interest, and
  the act of mechanising type systems and their metatheory is generally becoming
  easier as new techniques are invented.
  However, state-of-the-art techniques mostly rely on \emph{structurality} of
  the type system --- that weakening, contraction, and exchange are admissible
  and variables can be used unrestrictedly once assumed.
  Linear logic, and many related subsequent systems, provide motivations for
  breaking some of these assumptions.

  We present a framework for mechanising the metatheory of certain
  substructural type systems, in a style resembling mechanised metatheory of
  structural type systems.
  The framework covers a wide range of simply typed syntaxes with semiring
  usage annotations, via a metasyntax of typing rules.
  The metasyntax for the premises of a typing rule is related to bunched logic,
  featuring both sharing and separating conjunction, roughly corresponding to
  the additive and multiplicative features of linear logic.
  The bunched flavour is carried over into the semantics, together with the use
  of basic linear algebra constructs.
  For example, \emph{environments} are presented equivalently as values
  accumulated via separating conjunction, and as functions from variables to
  values supported by linear maps.
  Producing a generic semantic traversal has us combine environments with a
  separating implication, producing a Kripke function space of the form
  $\Box(A \wand B)$.
  From the generic semantic traversal, we derive totally generic renaming and
  substitution operations, a specific denotational semantics, and a
  syntax-generic \emph{usage elaborator} which greatly facilitates writing
  concrete terms.
\end{abstract}

\maketitle

\section{Introduction}\label{sec:intro}
\chapter{Introduction}

This thesis advances the frontier of programming languages work that can be done
in a proof assistant.
I will elaborate on both of these aspects in the following paragraphs.

The main aim of programming language research is to find common patterns in
computer programs and represent these patterns via programming language
features.
As a basic example, consider functions in, say, a low level language like C, in
contrast to what these functions are compiled to in assembly language/machine
code.
In order to support function calls, the C runtime system manages a
\emph{call stack}.
A function call is compiled to a sequence of instructions which store the return
address and the values of the arguments on the stack, move the stack pointer,
and jump to the compiled code of the function.
A return from a function then places the return value into the appropriate
register, moves the stack pointer back, and jumps back to the return address.
Advances in programming language techniques are evaluated by the usefulness of
behaviours captured and goodness of the mathematical properties of the
abstractions produced.

Within programming language research, type theory is a methodology for enforcing
abstractions by classifying program behaviour.
The simplest and most common type systems classify programs by what values they
may return.
For example, the body of a C function with return type \texttt{char} is a
program that, if it returns a value, will return a character.
If that function has a parameter of type \texttt{int}, then we can compose it
with a function with return type \texttt{int} to build up a larger program.
We expect type systems to stop us from running programs with arguments of the
wrong types.
This holds of both static and dynamic type systems --- with a static type
system, the compiler will refuse to compile our code if we pass a \texttt{char}
value to a function expecting an \texttt{int}, while with a dynamic type system,
our program will do a check before running the function to make sure that we
have passed it an \texttt{int}.
The abstraction produced in such simple type systems is the idea that a function
will take arguments in accordance with its parameter types and produce a result
in accordance with its result type, and furthermore, as readers, we do not have
to inspect the function's implementation to see that these properties are true.

A more recent trend in type theory is to use types to describe other parts of a
program's behaviour than what range of values it may return.
For example, Java's \emph{checked exceptions} can be seen as part of the static
type system, in which programs are classified by which exceptions they may
throw.
More generally, \emph{effect systems} classify programs based on all of the
effects they may have --- such as reading input and writing to files, and also
internally defined effects, such as non-determinism and using an accumulator.
In a type system which tracks effects (an \emph{effect system}), programs by
default are pure (i.e.\ have no effects), with effects being opt-in.
Pure programs generally enjoy good properties, making them easy to reason about
and easy for an optimising compiler to optimise.

In this thesis, I consider the dual of effect systems ---
\emph{coeffect systems} --- as introduced by \citet{POM14}.
In a coeffect system, we are interested not in what extra behaviour a program
may exhibit (as with effects), but rather what extra abilities the context of a
program may provide.
Analogously to the case of effect systems, we typically restrict our
coeffect-free programs to be ``more pure'' than usual.
A standard example is to restrict to \emph{linear} programs, in which each
variable in the context is used exactly once.
The ability to duplicate and discard variables is then seen as a coeffect, which
can be tracked by a coeffect system.
Restricting to linear programs may seem like an arbitrary restriction at first,
but the expectation of linearity arises naturally in applications such as
file-handling, session-typed communication, and approaches to mutable memory.
I introduce linearity and its applications more fully in \cref{sec:linearity}.

%A type system lets us enforce in our language complex invariants which maintain
%abstraction boundaries while also ideally being machine-checkable.
%For example, we give functions function type to abstract away their
%implementation via call stacks and jumps and so on.

%In this thesis, the invariants of interest revolve around restricting the usage
%of variables.
%I introduce this topic thoroughly in \cref{sec:linearity}, but in brief\ldots

The work of this thesis relies upon type theory in two distinct ways.
Firstly, as I have introduced above, the main objects of study in this thesis
are programming languages with interesting type systems.
Secondly, type theory provides the basis of the proof assistant Agda I
use to implement the aforementioned programming languages and operations upon
them.
I will now introduce the idea of proof assistants.

A \emph{proof assistant}, also known as an \emph{interactive theorem prover}, is
a piece of software that allows for the encoding of mathematical definitions,
theorems, constructions, and proofs, and furthermore check that such encoded
proofs are correct and that such encoded constructions are well formed.
To truly by interactive, i.e.\ to actually assist, a proof assistant will
usually have a user interface which can read partial proofs, display
information about what more proof needs to be given, and provide actions that
will help complete the proof.

Proof assistants have seen increasing use in programming language research in
recent years.
The most obvious reason why working in a proof assistant is seen as beneficial
is that it ensures correctness.
If the proof assistant accurately implements a suitable mathematical foundation,
then any theorem proved in the proof assistant is guaranteed to be a true
theorem of that foundation.
These guarantees of correctness are particularly important when working with
combinatorially complex mathematical objects, proofs about which often require
the consideration of a large number of cases.
Programming language syntaxes are often such complex objects, motivating the use
of proof assistants when studying programming languages.

A second reason to use proof assistants is for the assistance they provide when
exploring a mathematical theory.
When we make a new definition, we may want to test how it works in a special
case, or what constructions it allows us to perform.
In a proof assistant, the assistance tools give us immediate feedback as to what
moves are and aren't allowed.
For example, if we define a complex type system, a proof assistant will let us
interactively build typing derivations, making clear any side conditions and
types of subderivations as we go.
Also, as I do later in this thesis, a proof assistant allows us to build a very
general theory, and practically use that theory directly in more specific cases
without losing rigour.

Thirdly, analogously to how a strong static type system can give us more
confidence when refactoring a program, the constant checking of proofs in a
proof assistant gives us the confidence to change definitions and lemmas knowing
that we will be guided towards the parts of our theory that need to be
correspondingly changed.
This can help if we are developing a new programming language with a changing
specification.

Finally, many proof assistants --- including Agda~\citep{Agda}, which I use in
this thesis --- double as programming languages themselves.
This means that we can write programs and prove properties of them using the same
tool.
Also, many theorems proven in such a proof assistant have computational content.
For example, if we prove a normalisation theorem for a programming language,
this will typically yield a (verified) normalisation algorithm for it, which we
can really run on a computer.
As such, a development in a proof assistant can provide a reference
implementation of a programming language, or even --- as with Idris
2~\citep{Brady21}, Lean 4~\citep{deMU21}, and Cedille~\citep{GRS16} --- the
actual implementation of a programming language.

\section{Outline of the thesis}

This thesis proceeds as follows.
The next two chapters, \cref{sec:simple,sec:linearity}, are introductory in
nature, and cover two largely independent strands of prior work.
In \cref{sec:simple}, I introduce existing methods of representing and reasoning
about type systems in proof assistants based on dependent type theory.
I start from well established representations of well scoped and well typed
terms, and develop these towards a recent approach to environment-based
semantics given by \citet{AACMM21}.
In \cref{sec:linearity}, I discuss the challenges faced when one extends a
treatment of a simple type system, such as that given in \cref{sec:simple}, to
modal and linear type systems.
We see that modal and linear type systems apparently violate some of the nice
properties of the simply typed $\lambda$-calculus we required in
\cref{sec:simple}.
I present a solution for intuitionistic S4 modal logic, but leave a solution for
linear logic to the following chapters.

In the following two chapters, \cref{sec:semirings,sec:ren-sub-lr}, I present a
calculus $\name$ parametrised by a partially ordered semiring of \emph{usage
annotations}.
In \cref{sec:semirings}, I define the calculus, give some possible extensions,
and show that it subsumes intuitionistic S4 modal logic and Intuitionistic
Linear Logic.
In \cref{sec:ren-sub-lr}, I show that $\name$ enjoys generalised versions of the
nice properties required in \cref{sec:simple}, and I proceed to give novel
definitions of simultaneous substitutions and their action on $\name$ terms.
These two chapters are adapted from the work of \citet{WA21}.

The remaining three main chapters,
\cref{sec:framework,sec:semantics,sec:example-semantics}, adapt the syntactic
and semantic framework of \citet{AACMM21}, as presented at the end of
\cref{sec:simple}, to semiring-annotated calculi.
\Cref{sec:framework,sec:semantics} generalise the work on $\name$ presented in
\cref{sec:semirings,sec:ren-sub-lr}, respectively.
\Cref{sec:framework} shows how to formally describe the syntax of an arbitrary
semiring-annotated calculus, following the constructions used in
\cref{sec:semirings}.
\Cref{sec:semantics} then provides the generic environment-based semantic
traversal on such syntaxes, providing renaming and substitution as per
\cref{sec:ren-sub-lr} for all syntaxes as special cases of the generic
traversal.
\Cref{sec:example-semantics} then gives further example uses of the generic
traversal.

Finally, I conclude with \cref{sec:conc}, which discusses the achievements of
this thesis and openings for future work.

\section{Naming and notation conventions}

I assume familiarity with the Curry-Howard correspondence~\citep{Howard80}
throughout this thesis.
I make no distinction between logics and type theories, and use terminology from
each interchangeably.
Each following bullet point lists a collection of synonyms.

\begin{itemize}
  \item assumption, hypothesis, variable
  \item proposition, formula, type
  \item connective, type former
  \item derivation, proof, term
  \item derivable (formula), inhabited (type)
\end{itemize}

I carry out mechanised constructions and proofs in the proof assistant and
programming language Agda~\citep{Agda}.
Agda is based on Martin-L\"{o}f's intensional dependent type theory, so I
similarly present non-mechanised constructions and proofs assuming a foundation
given by dependent type theory, in a style inspired by the HoTT
Book~\citep{hottbook}.
I give a fuller introduction to Agda in \cref{sec:agda-primer}.

This thesis is written in \colour{}, but should be readable without.
Agda code has syntax highlighting, and various pieces of notation related to
usage annotations are coloured \gr{green} for emphasis.


\section{Vectors over semirings}\label{sec:algebra}
The basic algebraic structure we concern ourselves with is \emph{partially
ordered semirings}, or \emph{posemirings} for short.
A posemiring is a (not necessarily commutative) semiring on a partially ordered
set, where both operations are monotonic.

\begin{definition}
  A \emph{posemiring} is a tuple $(\Ann, \leq, 0, +, 1, *)$ such that the
  following laws hold (universally quantified in all the variables).
  \begin{itemize}
    \item $x \leq x' \to y \leq y' \to x + y \leq x' + y'$
    \item $x \leq x' \to y \leq y' \to x * y \leq x' * y'$
    \item $0 + x = x
      \land x + 0 = x
      \land (x + y) + z = x + (y + z)
      \land x + y = y + x$
    \item $1 * x = x
      \land x * 1 = x
      \land (x * y) * z = x * (y * z)$
    \item $0 * x = 0
      \land (x + y) * z = x * z + y * z$
    \item $x * 0 = 0
      \land x * (y + z) = x * y + x * z$
  \end{itemize}
\end{definition}

An element of a chosen posemiring $\Ann$ describes the usage restrictions on
a variable.
Therefore, a \emph{vector} of elements from $\Ann$ describes the usage
restrictions of a whole context's worth of variables.
From the posemiring operations of $\Ann$, we derive the standard vector
operations of zero, addition, and multiplication by a scalar.
We can also form the standard basis vectors at any given dimension.

Vectors of a given length form a \emph{module} over the posemiring $\Ann$,
analogously to how vectors over a field form a vector space.

\begin{definition}
  A \emph{(left) module over a posemiring}, given a posemiring $\Ann$, is a
  partially ordered commutative monoid $(M, 0_M, +_M)$ with, for each
  $r \in \Ann$, a pomonoid morphism $r \cdot \plr{-} : M \to M$, such that the
  collection of these respects the posemiring structure on $r$.
  Specifically:
  \begin{itemize}
    \item $r \leq r' \to u \leq u' \to r \cdot u \leq r' \cdot u'$
    \item $r \cdot 0_M = 0_M
      \land r \cdot \plr{u +_M v} = r \cdot u +_M r \cdot v$
    \item $0 \cdot u = 0_M
      \land \plr{r + s} \cdot u = r \cdot u +_M s \cdot u$
    \item $1 \cdot u = u
      \land \plr{r * s} \cdot u = r \cdot \plr{s \cdot u}$
  \end{itemize}
\end{definition}

We care to define modules so as to define \emph{module morphisms}, also known
as \emph{linear maps}, which we use extensively when relating two contexts (as
we do, for example, in simultaneous substitution).
For the sake of mechanisation, we choose to define module morphisms
\emph{relationally} rather than \emph{functionally}, giving a somewhat
unfamiliar-looking definition that is equivalent to the usual functional
definition.
The main advantage of this relational approach is that proofs of relatedness
for typical linear maps compose and decompose via data constructors and
pattern matching.
% I'm not sure this is a real difference:
%An auxiliary advantage is that with relations rather than functions, we can
%much more often take advantage of judgemental injectivity, thus making
%unification-based solving of implicits more effective.
%For example, if \AgdaBound{R} is a free variable of relation type, then
%\AgdaInductiveConstructor{refl} serves as a proof of
%\ExecuteMetaData[\Snippetstex]{Rxy-R}{}, solving the underscores as
%\AgdaBound{x} and \AgdaBound{y}, respectively.

\begin{definition}
  A \emph{(relational) linear map} $\Psi$ between modules $M$ and $N$ over a
  posemiring $\Ann$ is a relation $\sim$ on the underlying sets of $M$ and $N$
  satisfying the following laws.
  \begin{itemize}
    \item $u \leq u' \to v' \leq v \to u \sim v \to u' \sim v'$
    \item $\forall v.~\plr{\exists u.~u \leq 0 \land u \sim v} \to v \leq 0$
    \item $\forall u_0,u_1,v.~\plr{\exists u.~u \leq u_0 + u_1 \land u \sim v}
      \to {}$\\$\plr{\exists v_0,v_1.~u_0 \sim v_0
      \land u_1 \sim v_1 \land v \leq v_0 + v_1}$
    \item $\forall r,u',v.~\plr{\exists u.~u \leq ru' \land u \sim v} \to
      \plr{\exists v'.~u' \sim v' \land v \leq rv'}$
    \item
      $\forall u.~\exists v.~u \sim v \land \forall v'.~u \sim v' \to v' \leq v$
  \end{itemize}
\end{definition}

Intuitively, $Q \sim P$, where $P$ and $Q$ are row vectors, is equivalent to
$P \leq Q\Psi$, where $\Psi$ is the matrix representing the linear map and on
the right is a vector-matrix multiplication.
It is important that we think of \emph{row} vectors and
\emph{right}-multiplication by a matrix because, without commutativity of the
underlying posemiring, we can only expect $\plr{rQ}\Psi = r\plr{Q\Psi}$ and
not $\Psi\plr{rQ} = r\plr{\Psi Q}$.
In \cref{sec:env}, we use the matrix notation for convenience, while in the
Agda code we see \ExecuteMetaData[\Snippetstex]{Psi-rel-P-Q}.

\begin{figure}
  \begin{align*}
    \dot1\,\grR &\coloneqq 1 \\
    (T \dottimes U)\,\grR &\coloneqq T\,\grR \times U\,\grR \\
    (T \dotto U)\,\grR &\coloneqq T\,\grR \to U\,\grR \\
    I^*\,\grR &\coloneqq \grR \leq \gr0 \\
    (T \sep U)\,\grR &\coloneqq \Sigma \grP,\grQ.~\plr{\grR \leq \grP + \grQ}
                       \times T\,\grP \times U\,\grQ \\
    (\gr r \cdot T)\,\grR &\coloneqq \Sigma \grP.~\plr{\grR \leq \gr r\grP}
                       \times T\,\grP \\
    (T \wand U)\,\grP &\coloneqq \Pi \grQ,\grR.~\plr{\grR \leq \grP + \grQ}
                       \to T\,\grQ \to U\,\grR
  \end{align*}
  \caption{The bunched connectives}
  \label{fig:bunched}
\end{figure}

%Operations like renaming and substitution are essentially translations from one
%context to another.
%When faced with two vector spaces arranged in this way, a natural thing to
%consider is the \emph{linear maps} from one space to the other.


\section{Generic syntax}\label{sec:syntax}
We take the insights of the previous section and use them to build a
generic framework for posemiring-annotated substructural systems in
Agda. We will first show \emph{descriptions} of systems, which are
comprised of rules that have premises combined using the bunched
combinators. We then show how to construct the Agda data type of
intrinsically well scoped, typed, and resourced terms for any given
system of our framework. We use the prototypical system from
\cref{fig:lr-comb} as a running example. \cref{sec:other-syntaxes}
presents further examples that our framework can express.

We now start to use Agda notation for record and data type
declarations, to emphasise that our framework has been implemented.

\subsection{Descriptions of Systems}

% We capture the form of rules exemplified previously\todo{Previously?} via
% \emph{descriptions} of rules.
% The key to making these descriptions work is that they only allow syntactic
% forms that preserve environments.
% These forms are: absence and multiplicity of subterms with the same usage
% annotations, absence and multiplicity of subterms with summed usage annotations,
% scaling of a subterm, and variable binding.\todo{Switching to Agda}

\paragraph{\AgdaDatatype{Premises}, \AgdaRecord{Rule}s, and \AgdaRecord{System}s.}

A type \AgdaRecord{System} is made up of multiple \AgdaRecord{Rule}s.
Each \AgdaRecord{Rule} comprises a tree of \AgdaDatatype{Premises} and
a type of conclusion. We assume that there is a
$\AgdaBound{Ty} : \AgdaPrimitiveType{Set}$ of types for the system in
scope.

The \AgdaDatatype{Premise} data type describes premises of rules,
using the bunched combinators from the previous section. A single
premise is introduced by the
\AgdaInductiveConstructor{$\langle$\_`$\vdash$\_$\rangle$}
constructor.  This allows binding of additional variables
\AgdaBound{$\Delta$} (with specified types and usage annotations) and
the specification of a conclusion type \AgdaBound{A} for this premise.
The remaining constructors are descriptions for the first-order
bunched connectives, and will be interpreted directly as such, below.

\ExecuteMetaData[\Syntaxtex]{Premises}

A \AgdaRecord{Rule} is a pair of some \AgdaDatatype{Premises} and a
conclusion. We use an infix arrow as a suggestive notation for rules.

\ExecuteMetaData[\Syntaxtex]{Rule}

Finally, a \AgdaRecord{System} consists of a set of rule labels (i.e.,
constructor names), and for each label a decsription of the
corresponding rule. We use $\rhd$ as infix notation for systems to
associate the label set with the rules.

\ExecuteMetaData[\Syntaxtex]{System}

\paragraph{\cref{fig:lr-comb} as a \AgdaRecord{System}.}

As an example, we transcribe the system defined in
\cref{fig:lr-comb} into a description.  We give the set of types of
this system as a data type \AgdaDatatype{Ty} (together with a base
type \AgdaInductiveConstructor{$\iota$}). We assume that there is a
posemiring \AgdaInductiveConstructor{Ann} in scope for the
annotations.There is one label for each instantiation of a logical
rule, but the labels contain no further information about subterms or
restrictions on the context. This will be provided when we associate
labels with \AgdaRecord{Rule}s in a \AgdaRecord{System}.

\noindent
\begin{minipage}[t]{0.5\textwidth}
  \ExecuteMetaData[\PaperExamplestex]{Ty}
  \ExecuteMetaData[\PaperExamplestex]{Side}
\end{minipage}
\begin{minipage}[t]{0.5\textwidth}
  \ExecuteMetaData[\PaperExamplestex]{qlR}
\end{minipage}

To build a system, we associate with each label a rule:

\ExecuteMetaData[\PaperExamplestex]{lR}

Compared to \cref{fig:lr-comb}, modulo the Agda notation, we can see
that the fundamental structure has been preserved: the rules match
one-to-one, and the bunched premises are the same. A major difference
is that we do not include a counterpart to the
\AgdaInductiveConstructor{var} rule in a
\AgdaRecord{System}. Variables are common to all the systems
representable in our framework.

\paragraph{Terms of a \AgdaRecord{System}.}

The next thing we want to do is to build terms in the described type system.
The following definitions are useful for talking about types indexed over
contexts, judgement forms, and judgement forms admitting newly bound variables,
respectively.

\ExecuteMetaData[\Syntaxtex]{OpenFam}

To specify the meaning of descriptions, we assume some \AgdaBound{X} : \AgdaFunction{ExtOpenFam},
% \ExecuteMetaData[\Interpretationtex]{X},
over which we form one layer of syntax, using the function
\AgdaFunction{$\llbracket$\_$\rrbracket$p} that interprets
\AgdaDatatype{Premises} defined below.  The first argument to
\AgdaBound{X} is the new variables bound by this layer of syntax, as
exemplified in the first clause of
\AgdaFunction{$\llbracket$\_$\rrbracket$p}.  The second argument is
the context containing the variables being carried over from the
previous layer.  Notice that this is not, in general, the same as the
context from the previous layer, because the usage annotations may
have been changed by connectives like
\AgdaInductiveConstructor{\_`$*$\_} and
\AgdaInductiveConstructor{\_`$\cdot$\_}.  The third argument is the
type of subterm required.

With the first clause of \AgdaFunction{$\llbracket$\_$\rrbracket$p} explained,
the rest are simply interpretations of premises into bunched combinators.

\ExecuteMetaData[\Interpretationtex]{semp}

The interpretation of a \AgdaRecord{Rule} checks that the rule targets
the desired type and then interprets the rule's premises \AgdaBound{ps}.
Notice that the interpretation of the premises is independent of the conclusion
of the rule, which accounts for the difference in type between
\AgdaFunction{$\llbracket$\_$\rrbracket$p} and
\AgdaFunction{$\llbracket$\_$\rrbracket$r}.

\ExecuteMetaData[\Interpretationtex]{semr}

The interpretation of a \AgdaRecord{System} is to choose a rule label
\AgdaBound{l} from \AgdaBound{L} and interpret the corresponding rule
\AgdaBound{rs}\AgdaSpace{}\AgdaBound{l} in the same context and for the same
conclusion.

\ExecuteMetaData[\Interpretationtex]{sems}

The most obvious way to make such an \AgdaBound{X} is to use some existing
\AgdaFunction{OpenFam} on an extended context.
We defined \AgdaFunction{Scope} to do this: take the new variables
\AgdaBound{$\Delta$}, concatenate them onto the existing context
\AgdaBound{$\Gamma$}, and pass the extended context onto the judgement
\AgdaBound{T}.

\ExecuteMetaData[\Syntaxtex]{Scope}

%{\color{red}(Forward ref: for now, we could have inlined \texttt{Scope}.)}

We use \AgdaFunction{Scope} to deal with new variables in syntax.
Terms resemble the free monad over a layer-of-syntax functor, though
that picture is complicated by variable binding.  A term is either a
variable or a use of a logical rule together with terms for each of
the required subterms. The \AgdaFunction{Size} argument is where we
use sized types to convince Agda that this type is strictly positive.

\ExecuteMetaData[\Termtex]{Term}

Terms defined like this are still quite difficult to write, mainly because of
frequently changing usage contexts and the need for proofs that they all match
up.
We will see how to automate these proofs in \cref{sec:usage-elaborator}.

%Here is an example term, using the \AgdaFunction{$\lambda$R} system.
%First, for ease of writing, we introduce pattern synonyms for each of the
%typing rules we use.

%\ExecuteMetaData[\PaperExamplestex]{patterns}

%Our example term is a function that flips a tagged union wrapped in an
%arbitrarily annotated \emph{bang}.
%Much of the effort in writing such a term goes into writing the various
%relatedness proofs between usage contexts --- observing, for example, that two
%usage contexts sum together to make a third, or that a usage context used for
%a variable is a basis vector.
%We give a method of automating these proofs in \cref{sec:usage-elaborator}.
%\todo{To be clear, we don't actually write this.}

%\ExecuteMetaData[\HeavyItex]{lR-term}

% A layer of syntax supports the following functorial action.

% \ExecuteMetaData[\Maptex]{map-s-type}

\subsection{Other syntaxes and syntactic forms}\label{sec:other-syntaxes}

\paragraph{The system $\mu\tilde\mu$.}
We can encode a usage-annotated version of System $L$/the
$\mu\tilde\mu$-calculus~\cite{CH00} --- a syntax for classical logic --- in
such a way that contexts capture the undistinguished parts of the sequent.
As such, the generic substitution lemma we get in \cref{sec:kits} is the form
of substitution required in standard $\mu\tilde\mu$-calculus metatheory.
Though the $\mu\tilde\mu$-calculus is originally described as a sequent
calculus~\cite{CH00}, we use the techniques of
\citet[p.~12]{herbelin-hab} and \citet{LC06} to present it as a natural
deduction system, thus giving a notion of \emph{variable} to the system.

Unlike the single judgement form of \name{} and standard simply typed
$\lambda$-calculi, the $\mu\tilde\mu$-calculus has three judgement forms:
terms, coterms, and commands.
Read logically, terms and coterms are seen to, respectively, prove and refute
propositions (types), while commands exhibit contradictions.
This means that the abstract \AgdaBound{Ty} in the generic framework is
instantiated to \AgdaDatatype{Conc} (for \emph{conclusion}) as below, with
\AgdaDatatype{Ty} not being exposed directly to the generic framework.
For now, we just consider multiplicative disjunction $\parr$ (\emph{par}) and
negation/duality, beside an uninterpreted base type.
These are enough to exhibit classical behaviour.

\noindent
\begin{minipage}[t]{0.5\textwidth}
  \ExecuteMetaData[\MuMuTildetex]{Ty}
\end{minipage}
\begin{minipage}[t]{0.5\textwidth}
  \ExecuteMetaData[\MuMuTildetex]{Conc}
\end{minipage}

With \AgdaBound{Ty} instantiated as \AgdaDatatype{Conc}, all terms are assigned
\AgdaDatatype{Conc} type, as are all the variables.
No variables are given \AgdaInductiveConstructor{com} type, similar to how in
the bidirectional typing syntax of \citet[p.~25]{AACMM21}, no variables are
given \AgdaInductiveConstructor{Check} type.
How to observe this invariant is covered in the latter paper, so we will not
repeat it here (having not yet seen how to write traverals on terms).

The syntax comprises a \emph{cut} between a term and a coterm of the same type,
the eponymous $\mu$ and $\tilde\mu$ constructs for proof by contradiction, and
then term and coterm (introduction and elimination) forms for negation and
\emph{par}.

\ExecuteMetaData[\MuMuTildetex]{MMT}

%With a collection of pattern synonyms and the machinery from
%\cref{sec:usage-elaborator}, we can write an example term: a function which
%flips the disjuncts of a \emph{par}.

%\ExecuteMetaData[\MuMuTildeTermtex]{patterns}
%\ExecuteMetaData[\MuMuTildeTermtex]{myComm}

\paragraph{Duplicability}
There is one more bunched combinator we have experimented with adding to the
framework:

\[
  \plr{\Box T}\,\grR \coloneqq \Sigma\grRprime.~\plr{\grRprime \leq \grR}
  \times \plr{\grRprime \leq \gr0}
  \times \plr{\grRprime \leq \grRprime + \grRprime}
  \times T\,\grRprime
\]

The idea of $\plr{\Box T}\,\grR$ is to assert that $\grR$, or some refinement
of it, can be both discarded and duplicated indefinitely, and in the
refinement we have a $T$.
We use this combinator to introduce subterms that are used an unknown number of
times, for example the continuations of the eliminator of an inductive type,
or other fixed points.
We can also use it in linear/non-linear style systems~\cite{Benton94} to make
sure linear variables are not available in the intuitionistic fragment.

Adding the $\Box$ combinator is the only thing we have found that requires our
linear maps be functional rather than merely relational.


\section{Environments}\label{sec:env}
\subsection{Definition}

\begin{definition}[Usage-annotated recursive environment]\label{def:lr-rec-env}
  A \emph{recursive $\V$-environment} between annotated contexts $\Gamma$ and
  $\Delta$ is defined by cases on the shape of $\Delta$ (where
  $\Gamma \env\V_R \Delta$ is the notation for the
  type of recursive environments for given $\V$, $\Gamma$, and $\Delta$):
  \begin{itemize}
    \item There is one environment $\alr{} : \grP\gamma \env\V {\cdot}$
      whenever $\grP \leq \gr0$.
    \item For $\rho_l : \grPl\gamma \env\V_R \Delta_l$ and
      $\rho_r : \grPr\gamma \env\V \Delta_r$, we have an environment
      $\alr{\rho_l, \rho_r} : \grP\gamma \env\V_R \Delta_l, \Delta_r$ whenever
      $\grP \leq \grPl + \grPr$.
    \item For any value $v : \V\,\grPprime\gamma\,A$, we have an environment
      $\alr{v} : \grP\gamma \env\V_R \gr rA$ whenever
      $\grP \leq \gr r\grPprime$.
  \end{itemize}
\end{definition}

\begin{definition}[Usage-annotated environment]\label{def:lr-env}
  A \emph{$\V$-environment} between annotated contexts $\Gamma$ and $\Delta$
  (written $\grP\gamma$ and $\grQ\delta$, respectively, when convenient)
  is a linear map $\gr\Psi : \Ann^{\size\Delta} \to \Ann^{\size\Gamma}$ (written
  postfix) such that $\grP \leq \grQ\gr\Psi$ and for each $A$, $\grPprime$, and
  $\grQprime$ such that $\grPprime \leq \grQprime\gr\Psi$, a function from
  $\grQprime\delta \sqni A$ to $\V\,\grPprime\gamma\,A$.
\end{definition}

\subsection{Properties}

\begin{lemma}\label{thm:env-resize}
  Given an environment $\rho : \grP\gamma \env\V \grQ\delta$ and a $\grPprime$
  and a $\grQprime$ such that $\grPprime \leq \grQprime\plr{\rho.\gr\Psi}$,
  there is also an environment of type $\grPprime\gamma \env\V \grQprime\delta$
  with the same linear map and action on variables.
\end{lemma}
\begin{proof}
  The only part of the definition of an environment dependent on $\grP$ or
  $\grQ$ is the constraint $\grP \leq \grQ\gr\Psi$, which we are able to
  replace for $\grPprime$ and $\grQprime$.
\end{proof}

When constructing an environment, we can do so by cases on the shape of the
target context.
We can create an environment into the empty context when all usage annotations
on the source context are $\gr0$.
We can create an environment into a concatenated context when we can additively
split up the annotations of the source context and produce environments into
both halves from the split sources.
We can create an environment into a singleton context when there is a context
$\gr r$ times smaller than the source context in which we can produce a value
of the appropriate type.

\begin{lemma}\label{thm:construct-env}
  We can define all of the following equivalences for any values of the free
  variables, assuming that $\V$ respects subusaging (i.e.,
  $\grPprime \leq \grP \to
  \forallb{\V\,\grP\gamma \dotto \V\,\grPprime\gamma}$).
  \begin{itemize}
    \item $\forallb{I \dotlr \plr{{-} \env\V {\cdot}}}$
    \item $\forallb{\plr{{-} \env\V \Delta_l} \sep \plr{{-} \env\V \Delta_r}
      \dotlr \plr{{-} \env\V \Delta_l, \Delta_r}}$
    \item
      $\forallb{\gr r \cdot \plr{\V\,(-)\,A} \dotlr \plr{{-} \env\V \gr rA}}$
  \end{itemize}
\end{lemma}
\begin{proof}
  There are 6 cases to check.
  Throughout, we write $\Gamma$ as $\grP\gamma$ and $\Delta$ as $\grQ\delta$
  when convenient.
  \begin{description}
    \item[$I(\to)$]
      Let $\gr\Psi$ be the unique linear map out of the zero space.
      By assumption and definition, $\grP \leq \gr0 = \grQ\gr\Psi$.
      There are no variables to act upon.
    \item[$I(\gets)$]
      $\grQ\gr\Psi$ is an empty sum, so if $\grP \leq \grQ\gr\Psi$ then
      $\grP \leq \gr0$.
    \item[$\sep(\to)$]
      Let the given environments be $\rho_l : \grPl\gamma \env\V \grQl\delta$
      and $\rho_r : \grPr\gamma \env\V \grQr\delta$, with
      $\grP \leq \grPl + \grPr$.
      Define $\gr\Psi \coloneqq [\rho_l.\gr\Psi, \rho_r.\gr\Psi]$, using the
      coproduct structure of the concatenated vector space.
      We have $\grP \leq \grPl + \grPr \leq
      \grQl\plr{\rho_l.\gr\Psi} + \grQr\plr{\rho_r.\gr\Psi} =
      \begin{pmatrix} \grQl & \grQr \end{pmatrix}\gr\Psi$.
      To act on variables, we are given $\grPprime \leq
      \begin{pmatrix} \gr{\grQ'_l} & \gr{\grQ'_r} \end{pmatrix}\gr\Psi$ and
      $\gr{\grQ'_l}\delta_l, \gr{\grQ'_r}\delta_r \sqni A$.
      Without loss of generality, let us have $\gr{\grQ'_l}\delta_l \sqni A$
      and $\gr{\grQ'_r} \leq \gr0$.
      Thus, $\grPprime \leq
      \gr{\grQ'_l}\plr{\rho_l.\gr\Psi} + \gr{\grQ'_r}\plr{\rho_r.\gr\Psi} \leq
      \gr{\grQ'_l}\plr{\rho_l.\gr\Psi}$,
      and we can act on the variable using $\rho_l$.
    \item[$\sep(\gets)$]
      Let the unnamed context be $\Gamma$, also written $\grP\gamma$.
      The linear map
      $\gr\Psi : \Ann^{\size{\Delta_l} + \size{\Delta_r}} \to \Ann^{\size\Gamma}$
      splits into
      $\gr\Psi_{\gr l} : \Ann^{\size{\Delta_l}} \to \Ann^{\size\Gamma}
      \coloneqq \alr{\id, 0}; \gr\Psi$ and
      $\gr\Psi_{\gr r} : \Ann^{\size{\Delta_r}} \to \Ann^{\size\Gamma}
      \coloneqq \alr{0, \id}; \gr\Psi$, using the product structure of
      the concatenated vector space.
      Let $\grPl \coloneqq \grQl\gr\Psi_{\gr l}$ and
      $\grPr \coloneqq \grQr\gr\Psi_{\gr r}$, by definition satisfying the
      required constraints.
      For the action on variables, let us consider the left environment (with
      the right environment following symmetrically).
      We are given $\gr{\grP'_l} \leq \gr{\grQ'_l}\gr\Psi_{\gr l}$ and
      $\gr{\grQ'_l}\delta_l \sqni A$.
      From these, we get
      $\gr{\grP'_l} \leq \gr{\grQ'_l}\gr\Psi_{\gr l} =
      \begin{pmatrix} \gr{\grQ'_l} & \gr0 \end{pmatrix}\gr\Psi$ and
      $\gr{\grQ'_l}\delta_l, \gr0\delta_r \sqni A$.
      We can therefore act using the original environment.
    \item[$\cdot(\to)$]
      Let $\grP$ and $\grPprime$ be such that $\grP \leq \gr r\grPprime$ and let
      $v : \V\,\grPprime\gamma\,A$.
      Let $\gr\Psi : \Ann \to \Ann^{\size\gamma}
      \coloneqq \gr r\gr' \mapsto \gr r\gr'\grPprime$.
      By definition and the previous assumption, we have
      $\grP \leq \gr r\gr\Psi$.
      When acting on a variable, we have $\grP\gr{''} \leq \gr r\gr'\gr\Psi$
      and $\gr r\gr'A \sqni A'$.
      The latter tells us that $A = A'$ and $\gr r\gr' \leq \gr1$.
      Thus, $\grP\gr{''} \leq \grPprime$.
      Therefore, by subusaging, we may produce a value of type
      $\V\,\grPprime\gamma\,A$, which we can take to be $v$.
    \item[$\cdot(\gets)$]
      Let us have an environment of type $\grP\gamma \env\V \gr rA$.
      We want to use its action on variables to yield a value.
      To do this, we let $\grPprime \coloneqq \gr1\gr\Psi$, and use this
      equation, together with the fact that we have a variable of type
      $\gr1A \sqni A$, to get a value of type $\V\,\grPprime\gamma\,A$.
      Furthermore, we derive $\grP \leq \gr r\gr\Psi = \gr r\grPprime$, as
      required.
  \end{description}
\end{proof}

We could, indeed, use these three clauses to define what an environment is.
However, I find them difficult to work with, as it is often easier to do
linear algebraic proofs separately from the rest of an environment.
For identity and composition, as we are about to see, the original definition
is easier to use because we can rely on the identity and composition of linear
maps.
Concretely, an inductive proof of identity would, for example, involve
constructing an environment of type
$\grP\gamma, \grQ\delta \env\V \grP\gamma, \grQ\delta$ by constructing
environments of types $\grP\gamma, \gr0\delta \env\V \grP\gamma$ and
$\gr0\gamma, \grQ\delta \env\V \grQ\delta$.
These are not identity environments, so we would have to strengthen the
induction hypothesis.

One of the primary test cases for environments is simultaneous substitution,
which will look like the following rule.
The admissibility of substitution will be by induction on the derivation of
$\Delta \vdash A$, so we will need to be able to adapt any environment we are
given to work with any possible context of new premises.
In the simply typed case, the only change to the context we encountered was the
binding of new variables.
Now, with usage annotations, we furthermore have linear decompositions of the
context, necessitating changes to the environment whenever usage annotations
change.
I will deal first with linear decompositions.

\begin{displaymath}
  \begin{prooftree}
    \hypo{\Gamma \env{\vdash} \Delta}
    \hypo{\Delta \vdash A}
    \infer2[sub]{\Gamma \vdash A}
  \end{prooftree}
\end{displaymath}

There are three kinds of linear decompositions we have to deal with: zero,
addition, and scaling; corresponding to bunched connectives $I^*$, $\sep$, and
$\gr r \cdot {}$, respectively.
In each case, we have a simple preservation lemma, transforming an environment
of type $\Gamma \env\V \Delta$ and a decomposition of $\Delta$ into a
decomposition of $\Gamma$ and environments for all of the decomposed fragments
of $\Gamma$ and $\Delta$.

\begin{lemma}[environments preserve zero]\label{thm:lr-env-zero}
  Given an environment of type $\grP\gamma \env\V \grQ\delta$ such that
  $\grQ \leq \gr 0$, we also have that $\grP \leq \gr 0$.
\end{lemma}
\begin{proof}
  $\grP \leq \grQ\gr\Psi \leq \gr0\gr\Psi = \gr0$, by environment
  compatibility and monotonicity and linearity of $\gr\Psi$.
\end{proof}

\begin{lemma}[environments preserve addition]\label{thm:lr-env-add}
  Given an environment of type $\grP\gamma \env\V \grQ\delta$ such that
  $\grQ \leq \grQl + \grQr$ for some $\grQl$ and $\grQr$, we also have $\grPl$
  and $\grPr$ such that $\grP \leq \grPl + \grPr$ and there are environments
  of types $\grPl\gamma \env\V \grQl\delta$ and
  $\grPr\gamma \env\V \grQr\delta$.
\end{lemma}
\begin{proof}
  Let $\grPl \coloneqq \grQl\gr\Psi$ and $\grPr \coloneqq \grQr\gr\Psi$.
  Then, $\grP \leq \grQ\gr\Psi \leq \plr{\grQl + \grQr}\gr\Psi =
  \grQl\gr\Psi + \grQr\gr\Psi = \grPl + \grPr$, satisfying the first condition.
  Because clearly $\grPl \leq \grQl\gr\Psi$ and $\grPr \leq \grQr\gr\Psi$,
  \cref{thm:env-resize} on the original environment gives us the required
  pair of new environments.
\end{proof}

\begin{lemma}[environments preserve scaling]\label{thm:lr-env-scale}
  Given an environment of type $\grP\gamma \env\V \grQ\delta$ such that
  $\grQ \leq \gr r\grQprime$ for some $\grQprime$, we also have a $\grPprime$
  such that $\grP \leq \gr r\grPprime$ and there is an environment of type
  $\grPprime\gamma \env\V \grQprime\delta$.
\end{lemma}
\begin{proof}
  Let $\grPprime \coloneqq \grQprime\gr\Psi$.
  Then, $\grP \leq \grQ\gr\Psi \leq \plr{\gr r\grQprime}\gr\Psi =
  \gr r\plr{\grQprime\gr\Psi} = \gr r\grPprime$, satisfying the first condition.
  Because clearly $\grPprime \leq \grQprime\gr\Psi$,
  \cref{thm:env-resize} on the original environment gives us the required
  new environment.
\end{proof}

Finally, I will also take the opportunity to give the bind lemma, allowing
environments to incorporate newly bound variables.
In the intuitionistic case, the bind lemma had two requirements on $\V$: $\V$
admits weakening and we can map variables into $\V$-values.
With usage annotations, the former is unreasonable, but it turns out that we
only need weakening by variables whose usage annotation is less than or equal
to $\gr0$.
The latter stays as-is, with the note that ``variable'' now means a
usage-checked variable.

\begin{lemma}[bind]\label{thm:lr-bind}
  Given functions
  ${\swarrow^k} : \forall \Gamma, \grR, \theta.~\grR \leq \gr0 \to
  \forallb{\V\,\Gamma \dotto \V\,\plr{\Gamma, \grR\theta}}$ and
  $\mathrm{vr} : \forallb{{\sqni} \dotto \V}$, we can turn an environment of
  type $\Gamma \env\V \Delta$ into an environment of type
  $\Gamma, \Theta \env\V \Delta, \Theta$ for any context $\Theta$.
\end{lemma}
\begin{proof}
  Let $\grP\gamma \coloneqq \Gamma$, $\grQ\delta \coloneqq \Delta$, and
  $\grR\theta \coloneqq \Theta$.
  Let the new linear map $\gr\Psi\gr' : \Ann^{\size\Delta + \size\Theta} \to
  \Ann^{\size\Gamma + \size\Theta}$ be $\gr\Psi \oplus \gr I$.
  That is, in block matrix notation,
  $\begin{pmatrix} \gr\Psi & \gr0 \\ \gr0 & \gr I \end{pmatrix}$.
  Checking that this linear map fits, we have
  $\begin{pmatrix}\grP & \grR\end{pmatrix}
  \leq \begin{pmatrix}\grQ\gr\Psi & \grR\gr I\end{pmatrix}
  = \begin{pmatrix}\grQ & \grR\end{pmatrix}\plr{\gr\Psi \oplus \gr I}$.
  For the action on variables, we are given vectors $\grPprime$,
  $\grR\gr'_\grP$, $\grQprime$, and $\grR\gr'_\grQ$ such that
  $\begin{pmatrix} \grPprime & \grR\gr'_\grP \end{pmatrix} \leq
  \begin{pmatrix} \grQprime & \grR\gr'_\grQ \end{pmatrix}
  \plr{\gr\Psi \oplus \gr I}$ and we have a variable of type
  $\grQprime\delta, \grR\gr'_\grQ\theta \sqni A$ for some type $A$.
  The constraint on the new vectors reduces to $\grPprime \leq \grQprime\gr\Psi$
  and $\grR\gr'_\grP \leq \grR\gr'_\grQ$.
  From the variable we either have a variable $x$ in $\delta$ with
  $\grQprime \leq \langle x \rvert$ and $\grR\gr'_\grQ \leq \gr0$, or a
  variable $y$ in $\theta$ with $\grQprime \leq \gr0$ and
  $\grR\gr'_\grQ \leq \langle y \rvert$.
  In the former case, the action of the original environment on $x$ gives us a
  $\V$-value in $\grPprime\gamma$, and the $\gr0$-weakening principle
  $\swarrow^k$, noting that $\grR\gr'_\grP \leq \grR\gr'_\grQ \leq \gr0$, gives
  us a $\V$-value in $\grPprime\gamma, \grR\gr'_\grP\theta$.
  In the latter case, we have that
  $\begin{pmatrix} \grPprime & \grR\gr'_\grP \end{pmatrix}
  \leq \begin{pmatrix} \grQprime\gr\Psi & \grR\gr'_\grQ \end{pmatrix}
  \leq \begin{pmatrix} \gr0\gr\Psi & \langle y \rvert \end{pmatrix}
  = \begin{pmatrix} \gr0 & \langle y \rvert \end{pmatrix}
  = \left\langle {\searrow}y \right\rvert$, so $y$ also serves as a
  usage-checked variable in $\grPprime\gamma, \grR\gr'_\grP\theta$.
  From this usage-checked variable, we get a $\V$-value in the same context
  using $\mathrm{vr}$.
\end{proof}

The requirements for identity and composition of environments look a bit like
the unit and lift of a Kleisli triple.
\todo{Change from equality to inequality}

\begin{lemma}[Identity environment]
  Given a function $\mathrm{vr} : \forallb{{\sqni} \dotto \V}$, for any
  $\Gamma$ we have an environment of type $\Gamma \env\V \Gamma$.
\end{lemma}
\begin{proof}
  Let $\gr\Psi$ be the identity map, which clearly satisfies
  $\grP = \grP\gr\Psi$.
  When acting on a variable, the equation $\grPprime = \grQprime\gr\Psi$ means
  that $\grPprime = \grQprime$, so we want, from a variable of type
  $\grPprime\gamma \sqni A$, a value of type $\V\,\grPprime\gamma\,A$, which
  we can get from $\mathrm{vr}$.
\end{proof}

\begin{lemma}\label{thm:env-comp-lemma}
  Given an environment $\rho : \Gamma \env\U \Delta$ for which we have, for any
  $\grPprime$ and $\grQprime$ such that
  $\grPprime = \grQprime\plr{\rho.\gr\Psi}$, we have a function
  $\mathrm{lift}_\rho :
  \forallb{\V\,\grQprime\delta \dotto \W\,\grPprime\gamma}$,
  we can map environments of type $\Delta \env\V \Theta$ into environments of
  type $\Gamma \env\W \Theta$.
\end{lemma}
\begin{proof}
  Let $\rho$ be as in the statement, and let $\sigma : \Delta \env\V \Theta$.
  For the environment we are constructing, let
  $\gr\Psi \coloneqq \sigma.\gr\Psi; \rho.\gr\Psi$, noting that
  $\grP = \grQ\plr{\rho.\gr\Psi} =
  \plr{\grR\plr{\sigma.\gr\Psi}}\plr{\rho.\gr\Psi}$.
  For the action on variables, we are given $\grPprime = \grRprime\gr\Psi$ with
  $\grRprime\theta \sqni A$.
  We can immediately apply the action of $\sigma$, giving us a value of type
  $\V\,\plr{\grRprime\plr{\sigma.\gr\Psi}}\,A$.
  We note that
  $\grPprime = \plr{\grRprime\plr{\sigma.\gr\Psi}}\plr{\rho.\gr\Psi}$, and
  apply $\mathrm{lift}_\rho$ to get the desired value.
\end{proof}

\begin{corollary}[Composition of environments]
  Given a function
  $\mathrm{lift} : \plr{\rho : \grP\gamma \env\U \grQ\delta} \to
  \forall \grPprime, \grQprime.~\grPprime = \grQprime\plr{\rho.\gr\Psi} \to
  \forallb{\V\,\grQprime\delta \dotto \W\,\grPprime\gamma}$, then we can
  compose environments of types $\Gamma \env\U \Delta$ and
  $\Delta \env\V \Theta$ into an environment of type $\Gamma \env\W \Theta$.
\end{corollary}

\todo{What do these choices mean?}
\begin{example}
  We can derive the following instances of environment composition.
  \begin{itemize}
    \item If $\U = \V = \W = {\sqni}$, then $\mathrm{lift}$ is given by the
      action of the renaming $\rho$ on variables.
      This allows us to derive composition of renamings.
    \item More generally, if $\V = {\sqni}$ and $\U = \W$, we can still use
      the action of the environment $\rho$.
      This means that renamings post-compose with any other sort of environment.
    \item If $\V = \W = {\vdash}$, then $\mathrm{lift}$ is given by a
      syntactic traversal.
      For example, if $\U = {\sqni}$, we need the action of renaming on terms
      to show that a renaming followed by a substitution composes to a
      substitution.
      If $\U = {\vdash}$, then the action of substitution on terms gives us that
      substitutions compose.
    \item More generally, if $\V = {\vdash}$ and we have a semantics from
      $\U$ to $\W$, then $\mathrm{lift}$ can be given by the semantic traversal
      of terms.
  \end{itemize}
\end{example}


\section{Semantics}\label{sec:semantics}
Given a $\V$-environment $\Gamma \Rightarrow \Delta$, the function
\AgdaFunction{semantics} we define in this section assigns a
$\C$-value in context $\Gamma$ to every term in context $\Delta$,
where $\C$ is an \AgdaFunction{OpenFam} being the carrier of the
semantic interpretation of terms ($\V$ being the semantic
interpretation of variables). Before we can define
\AgdaFunction{semantics}, we need to treat recursion through rules'
premises (\cref{sec:functorial}) and extension of environments when
going under variable binders (\cref{sec:kripke}).
\todo{Maybe split the chapter here. Syntax/semantics}

% Our goal in this section is to define \AgdaFunction{semantics}, a
% recursor that turns a term into a \AgdaBound{$\C$}-value using a
% \AgdaBound{$\V$}-environment, in a type preserving way:\bob{Get rid of
%   ``body'' here}

% \ExecuteMetaData[\Semanticstex]{semantics-type}

% The \AgdaBound{$\V$} and \AgdaBound{$\C$} are \AgdaFunction{OpenFam}s,
% representing the interpretations of variables and terms
% respectively. In \cref{sec:traversal} we will see the data that must
% be provided to make a \AgdaFunction{semantics} for a given
% system. Before that, we must see how to deal with the two complicated
% features of our syntax: the usage annotations (\cref{sec:functorial})
% and variable binding (\cref{sec:kripke}). \todo{fwd ref to where these are used}

\subsection{A layer of syntax is functorial}\label{sec:functorial}

A basic property of the universe of syntaxes we described in \cref{sec:syntax}
is that every syntax supports a functorial action on subterms, realised by the function \AgdaFunction{map-s}.
Its type says that to map a function \AgdaBound{f}
over a layer of syntax, there must be a linear map \AgdaBound{F} relating the
domain and codomain usage contexts, and \AgdaBound{f} should be usable
wherever the domain and codomain usage contexts are similarly related by
\AgdaBound{F}.

\ExecuteMetaData[\Maptex]{map-s-type}

This generality is needed because usage contexts change between
a term and its immediate subterms---they are decomposed according to the bunched connectives used in the rules.
\AgdaBound{X} and \AgdaBound{Y} are \AgdaFunction{ExtOpenFam}s, with
\AgdaBound{$\Theta$} being the context extension for a subterm (i.e., the
variables newly bound in that subterm).
Unlike usage annotations, types in the contexts \AgdaBound{$\gamma$} and \AgdaBound{$\delta$}, and the conclusion types implicit here, are preserved throughout.
This is the essence of the usage annotation based approach---we use traditional techniques for variable binding, with the usage annotations layered on top.

The heart of \AgdaFunction{map-s} is \AgdaFunction{map-p}, which recursively
works through the structure \AgdaBound{ps} of premises of the rule applied,
acting on each subterm it finds.
Here, particularly in the clauses for \AgdaInductiveConstructor{`$\sep$} and
\AgdaInductiveConstructor{`$\cdot$}, we see why it is not enough for the
function on subterms to apply at usage contexts \AgdaBound{P} and \AgdaBound{Q}
--- rather, it also needs to apply at any similarly related \AgdaBound{P$'$}
and \AgdaBound{Q$'$}.
In the case of \AgdaInductiveConstructor{`$\sep$}, we have that
$\grP \leq \grP_M + \grP_N$, with \AgdaBound{M} and \AgdaBound{N} being
collections of subterms in usage contexts $\grP_M$ and $\grP_N$, respectively.
Linearity of \AgdaBound{F} yields $\grQ_M$ and $\grQ_N$ such that
$\grQ \leq \grQ_M + \grQ_N$ and we use \AgdaFunction{map-p} recursively at
$(\grP_M, \grQ_M)$ and $(\grP_N, \grQ_N)$ on \AgdaBound{M} and \AgdaBound{N}.
The cases for \AgdaInductiveConstructor{`$\cdot$} and
\AgdaInductiveConstructor{`$I^*$} are similar, each using a different aspect
of linearity.
In contrast, the cases for \AgdaInductiveConstructor{`$\dot1$} and
\AgdaInductiveConstructor{`$\dot\times$}, which are the only constructors used in fully structural
systems, do not involve any changes in the usage contexts.

\ExecuteMetaData[\Maptex]{map-p}

\subsection{The Kripke function space}\label{sec:kripke}

At this point we introduce a minor generalisation to
\AgdaFunction{OpenFam} and \AgdaFunction{ExtOpenFam}:
\AgdaBound{I}\AgdaSpace{}\AgdaFunction{---OpenFam} and
\AgdaBound{I}\AgdaSpace{}\AgdaFunction{---ExtOpenFam}.  We obtain the
definition of \AgdaBound{I}\AgdaSpace{}\AgdaFunction{---OpenFam} by
replacing the textual occurrence of \AgdaBound{Ty} by the parameter
\AgdaBound{I}.

The definition \AgdaFunction{Kripke}\,$\V$\,$\C$\,$\Delta$ is a kind
of function space that describes a $\C$ value parametrised by
$\Delta$-many additional $\V$s (all correctly typed and usage
annotated). It is used to describe how to go under binders in a
Higher-Order Abstract Syntax style---to go under a binder we must
provide semantic interpretations for all the additional variables:

% When going under binders during a recursion, the context will be extended by some $\Theta$. This means that the current environment must be extended with $\Theta$s-worth of $\V$s

% we need the ability to say that

% Kripke V C is given the extension \Theta

% In \cref{sec:terms}, we defined \AgdaFunction{Scope} to let a
% judgement-indexed family admit context extensions. However, a key
% component of our generic semantic traversal is to make use of the open
% family \AgdaBound{$\V$} of \emph{values}, which are the sort of thing
% we store in an environment.  The definition \AgdaFunction{Kripke}
% gives an alternative to \AgdaFunction{Scope} which interprets the
% newly bound variables via a requirement of $\V$-values rather than
% extra assumptions for the $\C$-computation.

\ExecuteMetaData[\Semanticstex]{Kripke}

\AgdaFunction{Wrap}
is a device that turns any type family into an equivalent type family
that is judgementally injective in its indices, which helps with
Agda's type inference.
It turns the type family into a parametrised
record with a single field \AgdaField{get} whose type is the type in
the body of the $\lambda$-abstraction.
For understanding the meaning of
\AgdaFunction{Kripke}, \AgdaFunction{Wrap} can be ignored.

If $\Delta$ is of the form $\gr{s_1}B_1, \ldots, \gr{s_n}B_n$, then
\ExecuteMetaData[\Snippetstex]{KripkeVCDGA}\ is equivalent to
\ExecuteMetaData[\Snippetstex]{KripkeExpanded}\ by Currying.  That is
to say, the Kripke function is expecting a value for each newly bound
variable, at the multiplicity of its annotation, together with the
resources supporting each of those values. We use the ``magic wand''
function space here to enforce the invariant that the freshly bound
variables have usage annotations that are added to the existing
variables, not shared with them. The use of the
\AgdaFunction{$\Box^r$} modality ensures that we can still use it in
the presence of additional variables introduced by weakening.

\AgdaFunction{Kripke} is functorial in the \AgdaBound{$\C$} argument,
as witnessed by the \AgdaFunction{mapK$\C$} function, which is essentially
post-composition:

\ExecuteMetaData[\Semanticstex]{mapKC}

% is exemplified by the following construct
% \AgdaFunction{reify}, where we weaken \AgdaBound{$\Gamma$} by a $\gr0$ed-out
% version of \AgdaBound{$\Delta$}.
% The \AgdaBound{$\Delta$} then gets filled in by the $\V$-values.

% \bob{Move this para}
% This means that \AgdaBound{A} in the definition of \AgdaFunction{Kripke} has
% type \AgdaBound{I}, rather than specifically \AgdaBound{Ty}.
% We use this generality later in \AgdaFunction{extend}, setting \AgdaBound{I}
% to \AgdaDatatype{Ctx}.

\subsection{Semantic traversal}\label{sec:traversal}

We can now state the data required to implement a traversal assigning
semantics to terms. For open families $\V$ and $\C$, interpreting
variables and terms respectively, we assume that $\V$ is renameable,
that $\V$ is embeddable in $\C$, and that we have an algebra for a
layer of syntax, where bound variables are handled using the Kripke
function space:

% The aim of this subsection is to give an alternative recursion principle for
% terms that incorporates some of the environment-handling seen in the
% implementations of renaming and substitution.
% The rest of this section assumes the following: a renameable open family
% \AgdaBound{$\V$} that embeds into the open family \AgdaBound{$\C$}, and an
% algebra for a layer of syntax at \AgdaBound{$\C$}.

\ExecuteMetaData[\Semanticstex]{Semantics}

%\ExecuteMetaData[\Semanticstex]{Comp}

We mutually define the action \AgdaFunction{semantics} and its lemma
\AgdaFunction{body}.
The purpose of \AgdaFunction{semantics} is to turn a term into a
\AgdaBound{$\C$}-value using a \AgdaBound{$\V$}-environment and the fields of
\AgdaRecord{Semantics}.
Meanwhile, \AgdaFunction{body} does a similar job, but also deals with
newly bound variables.
In particular, \AgdaFunction{body} takes a term in a context extended by
\AgdaBound{$\Theta$}, and produces a Kripke function from
\AgdaBound{$\V$}-values for \AgdaBound{$\Theta$} to \AgdaBound{$\C$}-values.

\ExecuteMetaData[\Semanticstex]{semantics-type}

To implement the new recursor \AgdaFunction{semantics}, we use the standard
recursor, which in one case gives us a variable \AgdaBound{v}, and in the other
gives us a structure of subterms \AgdaBound{M}, each of which is in an extended
context.
To deal with a variable \AgdaBound{v}, we look it
up in the environment \AgdaBound{$\rho$}, then use the
\AgdaField{$\sem{\text{var}}$} field to map the resulting
\AgdaBound{$\V$}-value to a \AgdaBound{$\C$}-value.
To deal with a structure of subterms \AgdaBound{M}, we use the functoriality of
the syntactic structure to consider each subterm separately.
On a subterm, we apply \AgdaFunction{body}, which amounts to a recursive call
to \AgdaFunction{semantics} with an extended environment.
Recall that \AgdaFunction{relocate} (\cref{thm:env-resize}) adjusts the
environment \AgdaBound{$\rho$} to work in the usage contexts of the subterms.

\ExecuteMetaData[\Semanticstex]{semantics}

For \AgdaFunction{body}, we are given a subterm \AgdaBound{M}, to
which we want to apply \AgdaFunction{semantics}.  To do so, we need an
extended version of the initial environment \AgdaBound{$\rho$}. We
express this as the generation of a Kripke function that produces the
extended environment given interpretations of the fresh variables. We
take \AgdaBound{$\rho$}, which is an environment covering
\AgdaBound{$\Delta$}, and \AgdaBound{$\sigma$}, which is an
environment covering \AgdaBound{$\Theta$}, and glue them together
using the inductive rules for generating environments, after having
renamed \AgdaBound{$\rho$} via \cref{thm:env-ren} to make it fit the
new context \AgdaBound{$\Gamma^+$} (intended to be
\ExecuteMetaData[\Snippetstex]{GT}):

\ExecuteMetaData[\Semanticstex]{extend}

% The best we can achieve without identity environments for \AgdaBound{$\V$} is
% a Kripke function returning an extended environment.
To define \AgdaFunction{body}, we use \AgdaFunction{mapK$\C$} to
post-compose the environment extension by the
\AgdaSymbol{$\lambda$}-function taking an extended environment and
acting with it on \AgdaBound{M}.

\ExecuteMetaData[\Semanticstex]{body}

% \todo{FIX} Under the assumption that \AgdaBound{$\V$} is renameable, we can decompose
% \cref{thm:lr-bind} as
% \AgdaFunction{reify}\AgdaSpace{}\AgdaOperator{\AgdaFunction{$\circ$}}%
% \AgdaSpace{}\AgdaFunction{extend}, with \AgdaFunction{extend} defined below.
% We can think of \AgdaFunction{extend} as our best effort to extend an
% environment by \AgdaBound{$\Theta$} without access to an identity environment
% at \AgdaBound{$\Theta$}.


\section{Example traversals}\label{sec:examples}
\subsection{Renaming and substitution}\label{sec:kits}

In an unpublished note, \citet{McBride05} gives a parametrised traversal
yielding homomorphisms of syntax.
The parameters are collected in the record \AgdaRecord{Kit}.
We make a minor change to the original presentation, where instead of our
\AgdaField{ren\textasciicircum{}$\V$} field, \citeauthor{McBride05} has the
field \AgdaField{wk} allowing only context extensions.
As for the other two fields, \AgdaField{vr} allows us to map variables to
$\V$-values, so as to put newly bound variables in environments; and
\AgdaField{tm} allows us to extract terms from $\V$-values, as required when
we use the environment to evaluate a free variable.

\ExecuteMetaData[\Syntactictex]{Kit}

Where \citeauthor{McBride05} gave the traversal explicitly, we go via our
generic semantic traversal.
The first two fields of \AgdaRecord{Semantics} derive directly from fields of
\AgdaRecord{Kit}.
Meanwhile, to handle term constructors, we first \AgdaFunction{reify} to get a
collection of traversed subterms, and then use \AgdaInductiveConstructor{`con}
to assemble these subterms into a similarly shaped syntactic form as we started
with.
The \AgdaField{vr} field is used implicitly in \AgdaFunction{reify}, as it is
used to show that $\V$-identity environments exist.

\ExecuteMetaData[\Syntactictex]{kit-to-sem}

The action of a syntactic traversal on logical rules is basically fixed: we
preserve the logical rule and extend the environment with any newly bound
variables according to \AgdaField{vr}.
Meanwhile, the action on variables is relatively unconstrained: we look up the
variable in the environment to get a $\V$-value, then transform that $\V$-value
into a term using \AgdaField{tm}.

The idea of renaming is that variables replace variables, whereas with
substitution, terms replace variables.
This translates to environments for renaming containing $\sqni$-values
(variables), and environments for substitution containing $\vdash$-values
(terms).

%To implement renaming and substitution for terms, we now just implement
%syntactic kits for variables and terms, respectively.

\ExecuteMetaData[\Syntactictex]{Ren-Kit}

Notice that \AgdaFunction{ren\textasciicircum$\vdash$}, witnessing the fact
that terms are renameable, is a corollary of \AgdaFunction{Ren-Kit}.

\ExecuteMetaData[\Syntactictex]{Sub-Kit}

\subsection{A denotational semantics}

\todo{Introduction}
To abbreviate this section, we use a simplified syntax compared to \name{}.
We allow for an arbitrary family of base types \AgdaBound{BaseTy}, and a single
type former \mbox{\ExecuteMetaData[\WReltex]{rAToB}}, equivalent to
\mbox{\ExecuteMetaData[\PaperExamplestex]{BangrAToB}} from the earlier system.

\ExecuteMetaData[\WReltex]{Ty}

In the term syntax, $\lambda$-abstraction now binds a variable with annotation
\AgdaBound{r}, while application needs to scale its argument by \AgdaBound{r}
(both in accordance with the function type they are acting on).

\ExecuteMetaData[\WReltex]{AnnArr}

In this subsection, we take the usage annotations to be the 4-element variance
posemiring.
\todo{This works for any semiring.}
We establish the property that all terms are monotonic in their free variables.
This monotonicity can be covariant or contravariant (or neither or both)
depending on the annotation of each free variable.
This provides an additional example to those of \citet{AbelBernardy2020}.
\todo{Cite before here.}

We will take semantics of this system into
\emph{worldly relations}~\cite{AbelBernardy2020}.
A worldly relation over a poset of worlds \AgdaBound{W} is a set over which
we have a \AgdaBound{W}-indexed binary relation satisfying a presheaf-like
property with respect to the order on \AgdaBound{W}.

\ExecuteMetaData[\WReltex]{WRel}

\begin{example}
  When \AgdaBound{W} is the 1-element set, a worldly relation is just a set
  equipped with a binary relation.
\end{example}

Morphisms between worldly relations \AgdaBound{R} and \AgdaBound{S} consist of
a mapping between the underlying sets such that that mapping preserves
relatedness from \AgdaBound{R} to \AgdaBound{S}.

\ExecuteMetaData[\WReltex]{WRelMor}

\todo{Define big intersection.}
When the poset of worlds forms a (relational) commutative monoid, such worldly
relations support a symmetric monoidal closed structure.
We reuse the bunched connectives \AgdaRecord{$I^*$}, \AgdaRecord{$\sep$}, and
\AgdaRecord{$\wand$}, now over worlds rather than contexts.

\ExecuteMetaData[\WReltex]{IR}
\ExecuteMetaData[\WReltex]{tensorR}
\ExecuteMetaData[\WReltex]{lollyR}

The final piece of sematics we need is a \emph{bang} operator.
\todo{No instead}
Instead of requiring extra algebraic structure on the worlds, we allow the
semantic \emph{bang} to be an arbitrary annotation-indexed functor on worldly
relations.
This functor must respect all of the structure on the indices, making it a
graded comonad over multiplication, as well as being lax monoidal at any
particular index \AgdaBound{r}.

\ExecuteMetaData[\WReltex]{Bang}

\begin{example}
  With \AgdaBound{W} being the 1-element set and annotations coming from the
  variance semiring, we can define the following \emph{bang}.
  It is always the identity on the set component, while the relation component
  consists of flipping the relation for contravariance and taking conjunctions
  to achieve both covariance and contravariance.
  When we want neither covariance nor contravariance, we use the always true
  predicate on worlds \AgdaFunction{$\dot1$}.

  \ExecuteMetaData[\Monotonicitytex]{BangR}
\end{example}

To associate semantics to syntax, we start as standard by associating worldly
relations to types.
We also extend the semantics of types to contexts, using \AgdaFunction{I$^R$},
\AgdaFunction{$\otimes^R$}, and \AgdaField{!$^R$} to interpret the empty
context, concatenation, and usage annotations on singletons, respectively.

\ExecuteMetaData[\WReltex]{sem}

The semantics of a term is then to be a morphism from the interpretation of the
context to the interpretation of the term's type.

\ExecuteMetaData[\WReltex]{sem-vdash}

Variables are given semantics by \AgdaFunction{lookup$^R$} (definition omitted).

\ExecuteMetaData[\WReltex]{lookupR-type}

Now, we give a \AgdaRecord{Semantics}.
The choice of \AgdaBound{$\V$} as
\AgdaRecord{\AgdaUnderscore{}$\sqni$\AgdaUnderscore{}} is somewhat arbitrary,
given that a standard denotational semantics would not use intermediate
environments in the same sense as renaming and substitution, but allows us to
reuse the standard facts that variables support renaming and identity
environments.
With this choice of \AgdaBound{$\V$} and \AgdaBound{$\C$}, we interpret
environment entries by \AgdaFunction{lookup$^R$}.
Meanwhile, for the logical rules, we ignore environments by using
\AgdaFunction{reify} to just deal with morphisms in an extended context.
As such, $\lambda$-abstractions are easy to interpret, while applications
require some massaging to remove the extension by an empty context, followed by
some plumbing to split the interpretation of the context according to the usage
constraints and feed the interpretation of the argument \AgdaBound{n$'$} into
the interpretation of the function \AgdaBound{m$'$}.

\ExecuteMetaData[\WReltex]{Wrel}

In order to map open terms to interpretations, we take the action of the
semantics and give the identity renaming as the starting environment.

\ExecuteMetaData[\WReltex]{wrel}

\begin{example}
  We can make a subtraction function from primitive addition and negation on
  integers.
  Subtraction is covariant in its first argument and contravariant in its
  second argument.
  We give the definition in pseudocode, as we have not yet seen how to
  conveniently write terms (\cref{sec:usage-elaborator}).

  \begin{align*}
    &{\sim\sim}p :
      {\uparrow\uparrow}\mathbb Z \multimap
      {\uparrow\uparrow}\mathbb Z \multimap \mathbb Z,
      {\sim\sim}n : {\downarrow\downarrow}\mathbb Z \multimap \mathbb Z
      \vdash \mathnormal{minus} :
      {\uparrow\uparrow}\mathbb Z \multimap
      {\downarrow\downarrow}\mathbb Z \multimap
      \mathbb Z
    \\
    &\mathnormal{minus} \coloneqq \lambda x.~\lambda y.~p\,x\,(n\,y)
  \end{align*}

  We observe that the set component of this term's semantics is just the
  expected Agda function when the two free variables are given appropriate
  interpretations.

  \ExecuteMetaData[\Monotonicitytex]{minus-set}

  Furthermore, the relational component of the semantics yields the free
  theorem that the Agda subtraction so defined is monotonic in the expected way.
  This relies on library proofs that addition and negation are suitably
  monotonic.

  \ExecuteMetaData[\Monotonicitytex]{thm}
\end{example}

\subsection{A usage elaborator}\label{sec:usage-elaborator}

Using the constructs we have seen so far, producing example terms soon becomes
extremely tedious.
We achieved some abbreviation by using pattern synonyms, but we still have to
produce essentially bespoke proofs whenever we use a usage-sensitive part of the
syntax.
The size of each of these proofs is roughly proportional to the number of free
variables, so the amount of proof we have to write grows roughly quadratically
with the size of terms.
An additional factor, which we can't see on paper, is that type checking time
for these proofs soon becomes prohibitive to interactive development.

Our aim in this subsection is to automate usage constraint proofs, making terms
both easier to write and more performant to check.
We invoke the automation by writing terms in a syntax where usage constraints
have been trivialised, and then use a semantic traversal over the simplified
syntax to try to produce a fully elaborated term in the original syntax.
We write the automation in a way that is generic in the syntax description, thus
avoiding repetition and facilitating the prototyping of new type systems.

The type of syntax descriptions depends on the type of usage annotations because
of variable binding.
For example, in the $\oc_{\gr r}$-E rule of \cref{fig:lr-comb}, the right
premise binds a new variable with annotation $\gr r$, where $\gr r$ is drawn
from the ambient posemiring.
The scaling combinator also makes direct reference to the posemiring.
To produce a simplified syntax description, where usage constraints are
trivialised, we set the ambient posemiring to the 1-element $\mathbf0$
posemiring.
In contrast to syntax descriptions, even though types can contain usage
annotations, the type of types does not depend on the type of usage annotations.
This means that, in our simplified syntax, terms have types from the original
system even though variables have trivial usage annotations.
We define the $\mathbf0$ posemiring as follows, being careful to use the
0-field record type \AgdaRecord{$\top$} so that everything algebraic gets
solved by Agda's $\eta$-laws.
Indeed, in this very definition, all of the semiring operations and laws are
canonically inferred.

\ExecuteMetaData[\UsageChecktex]{0-poSemiring}

The elaboration process is monadic.
In particular, we use the \AgdaDatatype{List}/non-determinism monad to give
\emph{all} of the possible annotation choices on the free variables of a term.
We believe the commitment to multiple solutions is inherent when the syntax
contains \AgdaInductiveConstructor{`$\dot1$}.
For example, in the intermediate stages of elaborating
$\plr{\vdash \lambda x.~\plr{*,*}} : A \multimap \top \otimes \top$ with a
usage counting posemiring (assuming reasonable rules for $\top$ and $\otimes$),
it is unclear whether to use the variable $x$ in the left $*$ or the right $*$.
This uncertainty should be reflected in the final result.

The non-deterministic choices we make during elaboration are enumerated by
the fields of \AgdaRecord{NonDetInverses}.
These choices are driven by the typing rules and a candidate usage vector for
the conclusion.
For example, \AgdaField{+$^{-1}$}\AgdaSpace{}\AgdaBound{r} is needed when we
encounter a \AgdaInductiveConstructor{`$\sep$} in the syntax and the candidate
usage annotation we are considering is \AgdaBound{r}.
Then, \AgdaField{+$^{-1}$}\AgdaSpace{}\AgdaBound{r} is a list of pairs of
annotations \AgdaBound{p} and \AgdaBound{q} that \AgdaBound{r} can split into,
together with a proof of the splitting.
For \AgdaField{0\#$^{-1}$} and \AgdaField{1\#$^{-1}$}, inverses to constants,
we are given the candidate \AgdaBound{r} and typically return an empty list if
the constraint cannot be satisfied, or a singleton list containing a proof.
\AgdaField{*$^{-1}$} is used when we encounter scaling, in which case we know
both the scaling factor \AgdaBound{r} (from the syntax description) and the
candidate \AgdaBound{q}.
These inverse operations combine monadically (in fact, applicatively) to give
inverses to the vector operations of zero, addition, scaling, and basis.

\ExecuteMetaData[\UsageChecktex]{NonDetInverses}

We choose the \AgdaBound{$\V$} of our semantics to be (unannotated) variables.
For the \AgdaBound{$\C$}, we consider \emph{functions} from candidate usage
vectors \AgdaBound{R} to the list of elaborated derivations with usage
annotations given by \AgdaBound{R}.
The module name \AgdaModule{U} refers to the fact that we are taking the
ambient posemiring to be $\mathbf0$ in \AgdaFunction{OpenFam}.
The effect on \AgdaFunction{OpenFam} is that the usage annotations of any
contexts we consider are uninformative (hence the \AgdaSymbol{\_} on the left).

\ExecuteMetaData[\UsageChecktex]{C}

To traverse the unannotated terms, we produce a \AgdaRecord{Semantics} over the
unannotated system \AgdaFunction{uSystem}\AgdaSpace{}\AgdaBound{sys}.
We already know that variables are renameable.
To interpret a variable, we consider all the possible proofs that such a
variable could be well annotated, and package them up as a variable term.

\ExecuteMetaData[\UsageChecktex]{elab-sem}

\ExecuteMetaData[\UsageChecktex]{lemma-type}

To actually use \AgdaFunction{elab-sem} on terms, we take the associated
\AgdaFunction{semantics} and pass it the identity environment (an identity
\emph{renaming} in this case, because $\V$ is a family of variables).
The candidate usage vector \AgdaBound{R} will be empty for closed terms, and
otherwise we have to supply the intended usage annotations.


\section{Conclusions}\label{sec:conc}

\bibliography{quant-framework}

\end{document}
